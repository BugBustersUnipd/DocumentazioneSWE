\documentclass[a4paper,12pt]{article}

\usepackage[utf8]{inputenc}
\usepackage[T1]{fontenc}
\usepackage[italian]{babel}
\usepackage[margin=2.5cm]{geometry}
\usepackage{graphicx}
\usepackage{grffile}
\usepackage{booktabs}
\usepackage{setspace}
\usepackage{titlesec}
\usepackage{float}
\usepackage{ifthen}
\usepackage{xcolor}
\usepackage{tcolorbox}
\usepackage{enumitem}
\usepackage[colorlinks=true,linkcolor=black,urlcolor=primaryblue,citecolor=primaryblue]{hyperref}

\definecolor{primaryblue}{RGB}{0,102,204}
\definecolor{secondaryblue}{RGB}{51,153,255}
\definecolor{lightgray}{RGB}{245,245,245}
\definecolor{darkgray}{RGB}{100,100,100}

\titleformat{\section}
  {\Large\bfseries\color{primaryblue}}
  {\thesection}{1em}{}

\setlength{\parskip}{4pt}
\setlength{\parindent}{0pt}

\setlist[itemize]{leftmargin=*,itemsep=3pt}
\setlist[enumerate]{leftmargin=*,itemsep=3pt}

\graphicspath{{./}{../assets/images/}{./images/}}

\begin{document}

\begin{center}  \IfFileExists{../../../assets/Logo.jpg}{%
    \includegraphics[width=6cm,height=3cm,keepaspectratio]{../../../assets/Logo.jpg} \\[0.8cm]
  }{%
    \fbox{\parbox[c][2.5cm][c]{6cm}{\centering Logo non trovato\\(Logo.jpg)}}\\[0.5cm]
  }
  
  {\Large\bfseries\color{primaryblue} BugBusters}\\[0.3cm]
  {\small\color{darkgray} Email: \texttt{bugbusters.unipd@gmail.com}} \\[0.1cm]
  {\small\color{darkgray} Gruppo: 4} \\[0.5cm]

  {\large\bfseries Università degli Studi di Padova}\\[0.3cm]
  {\small Laurea in Informatica}\\[0.2cm]
  {\small Corso: Ingegneria del Software}\\[0.2cm]
  {\small Anno Accademico: 2025/2026}\\[0.8cm]

  {\Huge\bfseries\color{primaryblue} Verbale Interno}\\[0.3cm]
  {\Large\color{secondaryblue} 18 ottobre 2025}\\[0.8cm]
\end{center}

\begin{center}
\begin{tcolorbox}[colback=lightgray,colframe=primaryblue,width=0.85\textwidth,arc=3mm,boxrule=0.5pt]
\begin{tabular}{@{}ll@{}}
\textbf{Redattori}    & Alberto Autiero, Alberto Pignat \\
\textbf{Verificatore}    &  Marco Favero \\
\textbf{Uso}          & Interno \\
\textbf{Destinatari}  & BugBusters \\
\textbf{Versione} & 1.0.0\\

\end{tabular}
\end{tcolorbox}
\end{center}

\vspace{0.5cm}

\begin{center}
\begin{tcolorbox}[colback=secondaryblue!10,colframe=secondaryblue,width=0.9\textwidth,arc=3mm,boxrule=0.8pt,title={\bfseries Abstract}]
Verbale della seconda riunione avvenuta in via telematica per confrontarci sulla risposta avuta dall'azienda Zucchetti S.P.A. Inizio creazione documento Scelta capitolato e Candidatura.
\end{tcolorbox}
\end{center}

\newpage

\tableofcontents
\newpage


\section{Informazioni generali}

\begin{itemize}
    \item \textbf{Tipo riunione:} interna
    \item \textbf{Piattaforma:} Discord
    \item \textbf{Data:} 18/10/2025
    \item \textbf{Orario di inizio:} 15:00
    \item \textbf{Orario di fine:} 16:15
    \item \textbf{Presenti:}
    \begin{itemize}[leftmargin=1.5em, itemsep=3pt, label={\rule[0.5ex]{0.4em}{0.4em}}]
        \item Alberto Autiero
        \item Alberto Pignat
        \item Marco Piro
        \item Marco Favero
    \end{itemize}
    \item \textbf{Assenti:}
    \begin{itemize}[leftmargin=1.5em, itemsep=3pt, label={\rule[0.5ex]{0.4em}{0.4em}}]
        \item Linor Sadè
        \item Luca Slongo
        \item Leonardo Salviato
    \end{itemize}
\end{itemize}

\section{Ordine del giorno}

\begin{enumerate}
    \item Analisi della risposta ricevuta dall'azienda Zucchetti S.P.A.
    \item Avvio della stesura del documento di Scelta del capitolato
    \item Inizio redazione della Candidatura per il capitolato scelto
    \item Creazione del repository GitHub del progetto
\end{enumerate}

\section{Svolgimento}

\begin{itemize}
    \item La riunione è iniziata alle ore 15:00.
    \item Il team ha analizzato in dettaglio la risposta pervenuta da Zucchetti S.P.A. riguardo alle domande formulate nella riunione precedente.
    \item Il gruppo ha concordato di allargare le possibili opzioni di capitolato, includendo anche Miriade S.R.L. e M31 S.R.L. come potenziali aziende partner.
    \item È stata avviata la stesura del documento formale di Scelta del capitolato e della Candidatura, definendone la struttura e i contenuti principali.
    \item È stato creato il repository GitHub del progetto per la gestione dei documenti e per il codice.
    \item La riunione è terminata alle ore 16:15.
\end{itemize}

\section{Decisioni prese}

\begin{itemize}
    \item \textbf{Struttura documenti:} definita la struttura base dei documenti da produrre.
\end{itemize}

\section{Todo}

\begin{tcolorbox}[colback=secondaryblue!8,colframe=secondaryblue!60,arc=2mm,boxrule=0.5pt,left=10pt,right=10pt]
\begin{itemize}[topsep=5pt]
    \item Scrivere alcune domande da proporre alle aziende Miriade e M31
    \item Iniziare a scrivere il contenuto di scelta del capitolato, analisi dei capitolati e candidatura
    \item Prossima riunione: 20/10/2025 alle ore 14:00
\end{itemize}
\end{tcolorbox}

\vfill
\begin{center}
    {\small\color{darkgray} Documento redatto e approvato dal gruppo BugBusters.}
\end{center}

\end{document}
