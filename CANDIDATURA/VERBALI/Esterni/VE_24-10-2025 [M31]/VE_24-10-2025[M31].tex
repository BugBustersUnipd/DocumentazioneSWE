\documentclass[a4paper,12pt]{article}
\newcommand{\CurrentVersion}{2.0.0}

\usepackage[utf8]{inputenc}
\usepackage[T1]{fontenc}
\usepackage[italian]{babel}
\usepackage[margin=2.5cm]{geometry}
\usepackage{graphicx}
\usepackage{grffile}
\usepackage{booktabs}
\usepackage{setspace}
\usepackage{titlesec}
\usepackage{float}
\usepackage{ifthen}
\usepackage{xcolor}
\usepackage{tcolorbox}
\usepackage{enumitem}
\usepackage[colorlinks=true,linkcolor=black,urlcolor=primaryblue,citecolor=primaryblue]{hyperref}
\usepackage{tocloft}
\usepackage{tabularx}
\usepackage{eurosym}

\definecolor{primaryblue}{RGB}{0,102,204}
\definecolor{secondaryblue}{RGB}{51,153,255}
\definecolor{lightgray}{RGB}{245,245,245}
\definecolor{darkgray}{RGB}{100,100,100}

% Configurazione semplice per l'indice
\setcounter{tocdepth}{2}
\renewcommand{\cftsecleader}{\cftdotfill{\cftdotsep}}

\titleformat{\section}
  {\Large\bfseries\color{primaryblue}}
  {\thesection}{1em}{}

\setlength{\parskip}{4pt}
\setlength{\parindent}{0pt}

\setlist[itemize]{leftmargin=*,itemsep=3pt}
\setlist[enumerate]{leftmargin=*,itemsep=3pt}

\graphicspath{{./}{../assets/images/}{./images/}}

\begin{document}

\begin{center}  \IfFileExists{../../../../assets/Logo.jpg}{%
    \includegraphics[width=6cm,height=3cm,keepaspectratio]{../../../../assets/Logo.jpg} \\[0.8cm]
  }{%
    \fbox{\parbox[c][2.5cm][c]{6cm}{\centering Logo non trovato\\(../../../../assets/Logo.jpg)}}\\[0.5cm]
  }
  
  {\Large\bfseries\color{primaryblue} BugBusters}\\[0.3cm]
  {\small\color{darkgray} Email: \texttt{bugbusters.unipd@gmail.com}} \\[0.1cm]
  {\small\color{darkgray} Gruppo: 4} \\[0.5cm]

  {\large\bfseries Università degli Studi di Padova}\\[0.3cm]
  {\small Laurea in Informatica}\\[0.2cm]
  {\small Corso: Ingegneria del Software}\\[0.2cm]
  {\small Anno Accademico: 2025/2026}\\[0.8cm]

  {\Huge\bfseries\color{primaryblue} Verbale Esterno}\\[0.3cm]
  {\Large\color{secondaryblue} 24 ottobre 2025}\\[0.8cm]
\end{center}

\begin{center}
\begin{tcolorbox}[colback=lightgray,colframe=primaryblue,width=0.85\textwidth,arc=3mm,boxrule=0.5pt]
\begin{tabularx}{\linewidth}{@{}lX@{}}
\textbf{Redattori}   & Alberto Pignat\\
\textbf{Verificatore} & Marco Favero\\
\textbf{Uso}          & Esterno\\
\textbf{Destinatari}  & Prof. Tullio Vardanega, Prof. Riccardo Cardin, M31\\
\textbf{Versione} & \CurrentVersion \\

\end{tabularx}
\end{tcolorbox}
\end{center}


\vspace{0.5cm}

\begin{center}
\begin{tcolorbox}[colback=secondaryblue!10,colframe=secondaryblue,width=0.9\textwidth,arc=3mm,boxrule=0.8pt,title={\bfseries Abstract}]
Verbale della prima riunione avvenuta con l'azienda M31 in via telematica per chiedere delucidazioni riguardanti il progetto "Sistema di acquisizione dati da sensori" e sui sistemi di collaborazione tra il gruppo e l'azienda.
\end{tcolorbox}
\end{center}


\newpage

\tableofcontents
\newpage


\section{Informazioni generali}

\begin{itemize}
    \item \textbf{Tipo riunione:} Esterna
    \item \textbf{Piattaforma:} Microsoft Teams
    \item \textbf{Data:} 24/10/2025
    \item \textbf{Orario di inizio:} 11:00
    \item \textbf{Orario di fine:} 11:50
    \item \textbf{Presenti:}
    \begin{itemize}[leftmargin=1.5em, itemsep=3pt, label={\rule[0.5ex]{0.4em}{0.4em}}]
        \item Alberto Autiero
        \item Marco Favero
        \item Alberto Pignat
        \item Marco Piro
        \item Linor Sadè
        \item Leonardo Salviato
        \item Luca Slongo
    \end{itemize}
    \item \textbf{Assenti:}
    \item \textbf{Presenti Esterni:}
    \begin{itemize}[leftmargin=1.5em, itemsep=3pt, label={\rule[0.5ex]{0.4em}{0.4em}}]
        \item Moones Mobaraki
        \item Cristian Pirlog
    \end{itemize}

\end{itemize}

\section{Ordine del giorno}

\begin{enumerate}
    \item Chiarimenti generali sul capitolato\textsubscript{\textbf{G}}
    \item Chiarimenti sui requisiti\textsubscript{\textbf{G}} minimi per una soddisfacente realizzazione del progetto\textsubscript{\textbf{G}}
    \item Domande tecniche sulle tecnologie che sarà necessario utilizzare
    \item Chiarimenti su come verrà disposta la formazione tecnica
    \item Chiarimenti sul supporto a livello contenutistico che avremo durante lo sviluppo
\end{enumerate}

\section{Svolgimento}
    La riunione è iniziata come previsto alle ore 11:00. In primo luogo è stata fatta un introduzione di chiarimento riguardante il progetto\textsubscript{\textbf{G}}, e poi successivamente sono state riviste le domande inviate via mail ed è stato possibile farne delle altre.
    \begin{itemize}
    \item \textbf{Chiarimenti generali sul capitolato\textsubscript{\textbf{G}} e sull'organizzazione}\\
    \noindent
    \textit{Risposta:} \\
    \begin{itemize}
        \item Il progetto\textsubscript{\textbf{G}} idealmente è concentrato principalmente sull'area medicale, e dunque sulla gestione dei segnali ricevuti da diverse tipologie di sensori.
        \item Molta attenzione è stata posta sui test\textsubscript{\textbf{G}} e validazione\textsubscript{\textbf{G}} del codice e sullo sviluppo esaustivo e corretto del relativo testbook.
        \item Ogni due settimane un incontro sullo stato di avanzamento delle attività, e la settimana dopo il SAL, incontro per fornire supporto tecnico.
    \end{itemize}
    
    \vspace{1cm}

    \item \textbf{Chiarimenti sulla formattazione dei dati e dataset}\\
    \noindent
    Riguardo alla formattazione dei dati ricevuti dal gateway, che deve essere uniformata, c'è un interesse particolare da parte dell'azienda di un certo formato da adottare? Sono presenti dataset già sviluppati?\\ \\
    \textit{Risposta:} 
    Non è definito un formato specifico di formattazione dei dati che vengono ricevuti dal gateway, se necessario è possibile discuterne. Riguardo ai dataset, non sono presenti, però in caso possono venire forniti se richiesti con preavviso.
    \vspace{1cm}
    

    \item \textbf{Chiarimenti sul provisioning}\\
    \noindent
    Come è previsto che avvenga il provisioning del gateway nel cloud?\\ \\
    \textit{Risposta:} 
    L'importante è che il provisioning sia sicuro, ma è un argomento che verrà trattato più accuratamente durante lo sviluppo del progetto\textsubscript{\textbf{G}}, attraverso degli incontri appositi.
    \vspace{1cm}

    \item \textbf{Chiarimenti su quali sensori verranno usati/simulati} \\
    \noindent
    Quale tipologia di sensori dobbiamo simulare e, di conseguenza, quali profili BLE standard è opportuno utilizzare? Sono presenti dei profili custom?\\ \\
    \textit{Risposta:} 
    Alcuni esempi di sensori consigliati: \\ 
    -Heartrate \\
    -Temperatura \\
    -Pressione sanguigna \\
    -Saturazione ossigeno (pulsossimetro) \\
    -ECG \\ 
    -Glicemia \\
    Non sono definiti profili custom.
    \vspace{1cm}
    
     \item \textbf{Chiarimenti sulle tecnologie consigliate}\\
    \noindent
    La domanda riguarda le tecnologie consigliate da usare per i dispositivi simulati ( sensori + gateway) e le tecnologie consigliate per il lato client (cloud + API\textsubscript{\textbf{G}}).\\ \\
    \textit{Risposta:} 
    Le tecnologie proposte sono molte, ma per sviluppare il progetto\textsubscript{\textbf{G}} bene non serve utilizzarle tutte. Ad esempio il simulatore e i microservizi sono sviluppabili in Node e la dashboard\textsubscript{\textbf{G}} in Angular, in modo tale da utilizzare TypeScript. Per la parte cloud e API\textsubscript{\textbf{G}} volendo viene consigliato anche Go, già utilizzato anche dall'azienda. Nats invece è consigliato per la trasmissione dati.
    \vspace{1cm}

    \item \textbf{Chiaramenti sulle conoscenze pregresse}\\
    \noindent
    Il team domanda se verrà fornito un supporto o qualche forma di affiancamento nell'utilizzare le tecnologie proposte.\\ \\
    \textit{Risposta:} 
    Non è previsto un supporto attivo, come corsi o lezioni dirette, e non è ritenuto necessario poiché parte integrante del progetto\textsubscript{\textbf{G}} consiste proprio nello studio e nella comprensione di queste tecnologie. Tuttavia viene fornito del supporto mirato anche tramite incontri dedicati nel caso sorgano dubbi o delle domande. 
    \vspace{1cm}
    
    \item \textbf{Chiarimenti sulla UI}\\
    \noindent
    È possibile indicare alcuni scenari\textsubscript{\textbf{G}} d'uso rappresentativi delle UI, con ruoli coinvolti, passi utente e risultati attesi?\\ \\
    \textit{Risposta:} 
    Questi dettagli verranno definiti più avanti nel progetto\textsubscript{\textbf{G}}, poiché nei primi incontri sarà necessario discutere e concordare i requisiti\textsubscript{\textbf{G}} elencati nel capitolato\textsubscript{\textbf{G}}. In ogni caso ci si aspetta una dashboard\textsubscript{\textbf{G}} dove si possano visualizzare i dati trasmessi dai sensori attivi in real time, possibilmente con qualche filtro per il tenant corrente.
    \vspace{1cm}

    \item \textbf{Chiarimenti sui rischi e mitigazioni}\\
    \noindent
    Quali rischi principali (tecnici, tempi, compliance) e quali mitigazioni sono accettabili nell'MVP? \\ \\
    \textit{Risposta:} 
    L'analisi dei rischi verrà svolta dal team durante il progetto\textsubscript{\textbf{G}}, riguardo alle tempistiche l'azienda è pienamente consapevole delle possibili difficoltà tecniche a cui il team può andare incontro ed è disposta a discutere con il team per contrattare i requisiti\textsubscript{\textbf{G}}.
    \vspace{1cm}
    
    \item \textbf{Chiarimenti sul cloud}\\
    \noindent
    Qualche consiglio sulla modalità di persistenza delle informazioni di commissioning dei sensori (nel gateway) e del gateway (nel cloud)? \\ \\
    \textit{Risposta:} 
    È importante che ci sia persistenza dei sensori, nel caso vada offline il gateway, non vadano persi i sensori salvati.
    \vspace{1cm}
    
    \item \textbf{Chiarimenti sulla sicurezza}\\
    \noindent
    Chiarimenti sui permessi assegnati agli amministratori\textsubscript{\textbf{G}} e agli utenti generici. Successivamente su tre tecnologie di sicurezza citate: JWT, OAuth2, mTLS, e se vadano sviluppate in modo "diverso" rispetto al consueto.\\ \\
    \textit{Risposta:} 
    L'amministratore\textsubscript{\textbf{G}} ha accesso a tutto, mentre un cliente generico ha accesso solamente ai suoi gateway.
    Le tecnologie di sicurezza citate verranno utilizzate come di consueto, nel caso più avanti verranno ridiscusse.
    \vspace{1cm}
    
\end{itemize}

La riunione si è conclusa alle 11:50.


\section{Esito riunione}
    L'esito della riunione è stato complessivamente positivo, le domande riguardanti il capitolato\textsubscript{\textbf{G}} sono state risposte in maniera esaustiva. Si ringraziano l'azienda M31 e i rappresentanti Moones Mobaraki e Cristian Pirlog per la disponibilità dimostrata.
    

% SEZIONE FIRMA E DATA SOSTITUITA CON L'IMMAGINE
\vspace{1.0cm}
\noindent
\includegraphics[width=0.5\textwidth]{Data e firma.png}

\vfill
\begin{center}
    {\small\color{darkgray} Documento redatto e approvato dal gruppo BugBusters.}
\end{center}

\end{document}