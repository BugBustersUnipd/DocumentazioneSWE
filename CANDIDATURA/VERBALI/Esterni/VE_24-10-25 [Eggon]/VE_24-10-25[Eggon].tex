\documentclass[a4paper,12pt]{article}
\newcommand{\CurrentVersion}{2.0.0}

\usepackage[utf8]{inputenc}
\usepackage[T1]{fontenc}
\usepackage[italian]{babel}
\usepackage[margin=2.5cm]{geometry}
\usepackage{graphicx}
\usepackage{grffile}
\usepackage{booktabs}
\usepackage{setspace}
\usepackage{titlesec}
\usepackage{float}
\usepackage{ifthen}
\usepackage{xcolor}
\usepackage{tcolorbox}
\usepackage{enumitem}
\usepackage[colorlinks=true,linkcolor=black,urlcolor=primaryblue,citecolor=primaryblue]{hyperref}
\usepackage{tocloft}

\definecolor{primaryblue}{RGB}{0,102,204}
\definecolor{secondaryblue}{RGB}{51,153,255}
\definecolor{lightgray}{RGB}{245,245,245}
\definecolor{darkgray}{RGB}{100,100,100}

% Configurazione semplice e funzionale per l'indice
\setcounter{tocdepth}{2}
\renewcommand{\cftsecleader}{\cftdotfill{\cftdotsep}}

\titleformat{\section}
  {\Large\bfseries\color{primaryblue}}
  {\thesection}{1em}{}

\setlength{\parskip}{4pt}
\setlength{\parindent}{0pt}

\setlist[itemize]{leftmargin=*,itemsep=3pt}
\setlist[enumerate]{leftmargin=*,itemsep=3pt}

\graphicspath{{./}{../assets/images/}{./images/}}

\begin{document}

\begin{center}  \IfFileExists{../../../../assets/Logo.jpg}{%
    \includegraphics[width=6cm,height=3cm,keepaspectratio]{../../../../assets/Logo.jpg} \\[0.8cm]
  }{%
    \fbox{\parbox[c][2.5cm][c]{6cm}{\centering Logo non trovato\\(Logo.jpg)}}\\[0.5cm]
  }
  
  {\Large\bfseries\color{primaryblue} BugBusters}\\[0.3cm]
  {\small\color{darkgray} Email: \texttt{bugbusters.unipd@gmail.com}} \\[0.1cm]
  {\small\color{darkgray} Gruppo: 4} \\[0.5cm]

  {\large\bfseries Università degli Studi di Padova}\\[0.3cm]
  {\small Laurea in Informatica}\\[0.2cm]
  {\small Corso: Ingegneria del Software}\\[0.2cm]
  {\small Anno Accademico: 2025/2026}\\[0.8cm]

  {\Huge\bfseries\color{primaryblue} Verbale Esterno}\\[0.3cm]
  {\Large\color{secondaryblue} 24 ottobre 2025}\\[0.8cm]
\end{center}

\begin{center}
\begin{tcolorbox}[colback=lightgray,colframe=primaryblue,width=0.85\textwidth,arc=3mm,boxrule=0.5pt]
\begin{tabular}{@{}ll@{}}
\textbf{Redattori}    & Leonardo Salviato \\
\textbf{Verificatore}    & Marco Favero \\
\textbf{Uso}          & Esterno \\
\textbf{Destinatari}  & Prof. Tullio Vardanega, Prof. Riccardo Cardin, Eggon \\
\textbf{Versione} & \CurrentVersion \\

\end{tabular}
\end{tcolorbox}
\end{center}

\vspace{0.5cm}

\begin{center}
\begin{tcolorbox}[colback=secondaryblue!10,colframe=secondaryblue,width=0.9\textwidth,arc=3mm,boxrule=0.8pt,title={\bfseries Abstract}]
Verbale della prima riunione avvenuta con l'azienda Eggon in via telematica per chiedere delucidazioni riguardanti il progetto\textsubscript{\scalebox{0.6}{\textbf{G}}}
 "Nexum" e sui sistemi di collaborazione tra il gruppo e l'azienda.
\end{tcolorbox}
\end{center}

\newpage

% INDICE
\tableofcontents

\newpage


\section{Informazioni generali}

\begin{itemize}
    \item \textbf{Tipo riunione:} Esterna
    \item \textbf{Piattaforma:} Google Meet
    \item \textbf{Data:} 24/10/2025
    \item \textbf{Orario di inizio:} 10:00
    \item \textbf{Orario di fine:} 10:35
    \item \textbf{Presenti:}
    \begin{itemize}[leftmargin=1.5em, itemsep=3pt, label={\rule[0.5ex]{0.4em}{0.4em}}]
        \item Alberto Autiero
        \item Marco Favero
        \item Alberto Pignat
        \item Marco Piro
        \item Linor Sadè
        \item Leonardo Salviato
        \item Luca Slongo
    \end{itemize}
    \item \textbf{Assenti:} Nessuno
    \item \textbf{Presenti Esterni:}
    \begin{itemize}[leftmargin=1.5em, itemsep=3pt, label={\rule[0.5ex]{0.4em}{0.4em}}]
        \item Gianpaolo Ferrarin
        \item Luca Luzzolino
    \end{itemize}

\end{itemize}

\section{Ordine del giorno}

\begin{enumerate}
    \item Chiarimenti generali sul capitolato\textsubscript{\scalebox{0.6}{\textbf{G}}}

    \item Chiarimenti sui requisiti\textsubscript{\scalebox{0.6}{\textbf{G}}}
 minimi per una soddisfacente realizzazione del progetto\textsubscript{\scalebox{0.6}{\textbf{G}}}

    \item Domande tecniche sulle tecnologie che sarà necessario utilizzare ed eventuali difficoltà
    \item Chiarimenti su come verrà disposta la formazione tecnica

\end{enumerate}

\section{Svolgimento}
    La riunione è iniziata come previsto alle ore 10:00. I componenti del gruppo hanno espresso le loro perplessità riguardanti il progetto\textsubscript{\scalebox{0.6}{\textbf{G}}}
. I punti trattati sono i seguenti
    \begin{itemize}
    \item \textbf{Chiarimenti generali sul capitolato\textsubscript{\scalebox{0.6}{\textbf{G}}}
 e sull'organizzazione}\\
    \noindent
    \textit{Risposta:} \\
    \begin{itemize}
        \item L'azienda per progetti\textsubscript{\scalebox{0.6}{\textbf{G}}}
 si organizza in sprint\textsubscript{\scalebox{0.6}{\textbf{G}}}
 bisettimanali tenendo in considerazione gli impegni universitari
    \end{itemize}
    
    \vspace{1em}

    \item \textbf{Chiarimenti sui requisiti\textsubscript{\scalebox{0.6}{\textbf{G}}}
 minimi per una soddisfacente realizzazione del progetto\textsubscript{\scalebox{0.6}{\textbf{G}}}
}\\
    \noindent
    Data la poca esperienza con progetti\textsubscript{\scalebox{0.6}{\textbf{G}}}
 proposti da aziende al team interessavano sapere le aspettative dell'azienda riguardanti questo progetto\textsubscript{\scalebox{0.6}{\textbf{G}}}
. 
    \\ \\
    \textit{Risposta:} 
    
    I requisiti\textsubscript{\scalebox{0.6}{\textbf{G}}}
 minimi non sono stati determinati. Ovviamente l'intenzione è quella di portare a termine gli obiettivi spiegati nel capitolato\textsubscript{\scalebox{0.6}{\textbf{G}}}
. Oltre agli obiettivi riportati i rappresentanti di Eggon hanno espresso il loro piacere nella collaborazione con team di studenti UniPd e hanno spiegato come per loro questo progetto\textsubscript{\scalebox{0.6}{\textbf{G}}}
 sia anche un'occasione di confrontarsi con elementi esterni e poter migliorare tramite questo confronto.
    
    \vspace{1em}

    \item \textbf{Domande tecniche sulle tecnologie che sarà necessario utilizzare ed eventuali difficoltà}\\
    \noindent
    Approfondimento delle tecnologie sulle quali è importante specializzarsi per la riuscita del progetto\textsubscript{\scalebox{0.6}{\textbf{G}}}
.\\ \\
    \textit{Risposta:} 
    Le tecnologie necessarie sono molteplici, è stato consigliato di informarsi principalmente sulle tecnologie proposte nel capitolato\textsubscript{\scalebox{0.6}{\textbf{G}}}
, con particolare attenzione ad AWS\textsubscript{\scalebox{0.6}{\textbf{G}}}
 (Bedrock) e al suo ruolo per lo sviluppo delle AI\textsubscript{\scalebox{0.6}{\textbf{G}}}
 del progetto\textsubscript{\scalebox{0.6}{\textbf{G}}}
. Tra le difficoltà che potremmo incontrare è stata segnalata la comprensione della documentazione AWS\textsubscript{\scalebox{0.6}{\textbf{G}}}
 e della struttura back-end\textsubscript{\scalebox{0.6}{\textbf{G}}}
.
    \vspace{1em}

    \item \textbf{Chiarimenti su come verrà disposta la formazione tecnica}\\
    \noindent
    Dato l'utilizzo necessario di tecnologie nuove per i membri del team si voleva ottenere un chiarimento sul tipo di supporto che avremmo avuto a livello tecnico. \\ \\
    \textit{Risposta:} 
    La formazione a livello tecnico verrà svolta autonomamente per le prime fasi del progetto\textsubscript{\scalebox{0.6}{\textbf{G}}}
 per fare in modo che il team acquisisca dimestichezza con le tecnologie proposte (Ruby on rails, AWS\textsubscript{\scalebox{0.6}{\textbf{G}}}
, etc), poi verrà fornito il supporto necessario per la comprensione degli strumenti utilizzati all'interno di Nexum. 
    \vspace{1em}

\end{itemize}

La riunione si è conclusa alle 10:35.


\section{Esito riunione}
    L'esito della riunione è stato molto positivo, le domande riguardanti il capitolato\textsubscript{\scalebox{0.6}{\textbf{G}}}
 sono state risposte in maniera esaustiva. È stata molto apprezzata la disponibilità con cui l'azienda si è posta nei confronti del progetto\textsubscript{\scalebox{0.6}{\textbf{G}}}
 e alla volontà di future collaborazioni anche in seguito alla sua conclusione. Si ringraziano l'azienda Eggon e i rappresentanti Gianpaolo Ferrarin e Luca Luzzolino per la disponibilità dimostrata.
    
% SEZIONE FIRMA E DATA SOSTITUITA CON L'IMMAGINE
\vspace{1.0cm}
\noindent
\includegraphics[width=0.5\textwidth]{Data e firma.png}

\vfill
\begin{center}
    {\small\color{darkgray} Documento redatto e approvato dal gruppo BugBusters.}
\end{center}

\end{document}