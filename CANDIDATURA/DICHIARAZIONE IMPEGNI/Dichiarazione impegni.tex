\documentclass[a4paper,11pt]{article}
\newcommand{\CurrentVersion}{2.0.0}

\usepackage[utf8]{inputenc}
\usepackage[T1]{fontenc}
\usepackage[italian]{babel}
\usepackage[margin=2.5cm]{geometry}
\usepackage{graphicx}
\usepackage{grffile}
\usepackage{booktabs}
\usepackage{setspace}
\usepackage{titlesec}
\usepackage{float}
\usepackage{ifthen}
\usepackage[table]{xcolor}
\usepackage{tabularx}
\usepackage{tcolorbox}
\usepackage{enumitem}
\usepackage[titles]{tocloft}
\usepackage{fancyhdr}
\usepackage{lastpage}
\usepackage{tikz}
\usepackage{pgf-pie}
\usepackage[colorlinks=true,linkcolor=black,urlcolor=primaryblue,citecolor=primaryblue]{hyperref}
\usepackage{eurosym}

\definecolor{primaryblue}{RGB}{0,102,204}
\definecolor{secondaryblue}{RGB}{51,153,255}
\definecolor{lightgray}{RGB}{245,245,245}
\definecolor{darkgray}{RGB}{100,100,100}

% Configurazione per l'indice
\setcounter{tocdepth}{2} % Mostra sezioni e sottosezioni nell'indice
\renewcommand{\cftsecleader}{\cftdotfill{\cftsecdotsep}}
\setlength{\cftbeforesecskip}{6pt}
\setlength{\cftbeforesubsecskip}{4pt}
\renewcommand{\cftsecfont}{\normalfont}
\renewcommand{\cftsecpagefont}{\normalfont}
\renewcommand{\cftsubsecfont}{\normalfont}
\renewcommand{\cftsubsecpagefont}{\normalfont}

\pagestyle{fancy}
\fancyhf{}
\fancyhead[L]{BugBusters}
\fancyhead[R]{Dichiarazione impegni}
\fancyfoot[L]{\thepage\ di \pageref{LastPage}}
\renewcommand{\headrulewidth}{0pt}
\renewcommand{\footrulewidth}{0pt}

\setlength{\headheight}{14pt}

\setlength{\parskip}{4pt}
\setlength{\parindent}{0pt}

\setlist[enumerate,1]{label=\textbf{\arabic*}, leftmargin=*}
\setlist[enumerate,2]{label=\arabic{enumi}.\arabic*}

\graphicspath{{./}{../assets/images/}{./images/}}
\titleformat{\section}
  {\Large\bfseries\color{primaryblue}}
  {\thesection}{1em}{}

\begin{document}

\begin{center}  \IfFileExists{../../assets/Logo.jpg}{%
    \includegraphics[width=6cm,height=3cm,keepaspectratio]{../../assets/Logo.jpg} \\[0.8cm]
  }{%
    \fbox{\parbox[c][2.5cm][c]{6cm}{\centering Logo non trovato\\(Logo.jpg)}}\\[0.5cm]
  }
  
  {\Large\bfseries\color{primaryblue} BugBusters}\\[0.3cm]
  {\small\color{darkgray} Email: \texttt{bugbusters.unipd@gmail.com}} \\[0.1cm]
  {\small\color{darkgray} Gruppo: 4} \\[0.5cm]

  {\large\bfseries Università degli Studi di Padova}\\[0.3cm]
  {\small Laurea in Informatica}\\[0.2cm]
  {\small Corso: Ingegneria del Software}\\[0.2cm]
  {\small Anno Accademico: 2025/2026}\\[0.8cm]

  {\Huge\bfseries\color{primaryblue} Dichiarazione impegni}\\[0.3cm]
  {\Large Versione \CurrentVersion}\\[0.8cm]
\end{center}

\begin{center}
\begin{tcolorbox}[colback=lightgray,colframe=primaryblue,width=0.85\textwidth,arc=3mm,boxrule=0.5pt]
\begin{tabularx}{\linewidth}{@{}lX@{}}
\textbf{Redattori}  & Alberto Autiero, Marco Favero, Luca Slongo\\
\textbf{Verificatore} & Marco Piro \\
\textbf{Destinatari} & Prof. Tullio Vardanega, Prof. Riccardo Cardin, BugBusters \\
\textbf{Data ultima modifica}     & 09/11/2025 \\
\end{tabularx}
\end{tcolorbox}
\end{center}

\newpage

\section*{Registro delle Modifiche}

\setlength{\extrarowheight}{2pt} % padding extra verticale
\renewcommand{\arraystretch}{1.5} 

\arrayrulecolor{primaryblue}
{\footnotesize
\begin{tabularx}{\textwidth}{|>{\raggedright\arraybackslash}p{1.5cm}|>{\raggedright\arraybackslash}p{2cm}|X|>{\raggedright\arraybackslash}p{2cm}|>{\raggedright\arraybackslash}p{2cm}|>{\raggedright\arraybackslash}p{2cm}|}
\hline
\rowcolor{primaryblue!40}
\textbf{\color{white} Versione} & \textbf{\color{white} Data} & \textbf{\color{white} Descrizione} & \textbf{\color{white} Redatto} & \textbf{\color{white} Verificato} & \textbf{\color{white} Approvato} \\
\rowcolor{secondaryblue!10} \CurrentVersion & 07/12/2025 & Approvazione del documento & - & - & Marco Favero \\
\hline
\rowcolor{secondaryblue!10} 1.1.0 & 06/12/2025 & Verifica del documento & - & Linor Sadè & - \\
\hline
\rowcolor{secondaryblue!10} 1.0.3 &06/12/2025 & Aggiunti ai termini presenti nel Glossario la G & Alberto Autiero & - & - \\
\hline
\rowcolor{secondaryblue!10} 1.0.2 & 09/11/2025 & Completamento ripartizione ore & Luca Slongo & - & - \\
\hline
\rowcolor{secondaryblue!10} 1.0.1 & 07/11/2025 & Aggiunta ripartizione delle ore & Luca Slongo & - & - \\
\hline
\rowcolor{secondaryblue!10} 1.0.0 & 30/10/2025 & Approvazione del documento & - & - & Alberto Autiero \\
\hline
\rowcolor{secondaryblue!10} 0.1.0 & 30/10/2025 & Verifica documento & - & Marco Piro & - \\
\hline
\rowcolor{secondaryblue!10} 0.0.2 & 26/10/2025 & Rielaborazioni e aggiunta testo & Marco Favero & - & - \\
\hline
\rowcolor{secondaryblue!10} 0.0.1 & 23/10/2025 & Prima stesura della struttura del documento & Alberto Autiero & - & - \\
\hline
\end{tabularx}
}

\newpage

% Aggiungiamo l'indice con una formattazione migliore
\renewcommand{\contentsname}{Indice}
\tableofcontents

\newpage

\section{Introduzione}
Con il presente documento il gruppo BugBusters (Gruppo 4), costituito per lo svolgimento del corso di Ingegneria del Software, dichiara l'impegno orario che ciascun membro intende dedicare all'implementazione\textsubscript{\scalebox{0.6}{\textbf{G}}} del capitolato\textsubscript{\scalebox{0.6}{\textbf{G}}}
 C5 - Nexum, proposto da Eggon. Di seguito sono riportate la descrizione dei ruoli e delle responsabilità previste per lo svolgimento del progetto\textsubscript{\scalebox{0.6}{\textbf{G}}},le ore produttive assegnate a ciascun componente e un riepilogo dei costi sostenuti.

\newpage

\section{Ruoli}

\subsection{Responsabile\textsubscript{\scalebox{0.6}{\textbf{G}}}}
Il Responsabile assume la guida del gruppo, coordinando le attività e gestiendo le comunicazioni con i docenti e il proponente\textsubscript{\scalebox{0.6}{\textbf{G}}}. Si occupa della pianificazione temporale, del monitoraggio dell'avanzamento e della gestione dei rischi\textsubscript{\scalebox{0.6}{\textbf{G}}}. Il suo impegno è particolarmente significativo nelle fasi iniziali del progetto\textsubscript{\scalebox{0.6}{\textbf{G}}}, per poi diminuire progressivamente man mano che il team acquisisce autonomia operativa.

\subsection{Amministratore\textsubscript{\scalebox{0.6}{\textbf{G}}}}
L'Amministratore\textsubscript{\scalebox{0.6}{\textbf{G}}} è responsabile della configurazione e del mantenimento degli strumenti di supporto allo sviluppo. Gestisce l'infrastruttura IT, i repository code, i sistemi di continuous integration\textsubscript{\scalebox{0.6}{\textbf{G}}} e la documentazione di progetto\textsubscript{\scalebox{0.6}{\textbf{G}}}
. Garantisce che tutti gli ambienti di lavoro siano correttamente configurati e funzionanti, con un carico di lavoro distribuito omogeneamente durante tutto il ciclo di vita del progetto\textsubscript{\scalebox{0.6}{\textbf{G}}}
.

\subsection{Analista\textsubscript{\scalebox{0.6}{\textbf{G}}}}
L'Analista si dedica allo studio e alla formalizzazione dei requisiti\textsubscript{\scalebox{0.6}{\textbf{G}}}, analizzando il capitolato\textsubscript{\scalebox{0.6}{\textbf{G}}} d'appalto e interfacciandosi con il proponente\textsubscript{\scalebox{0.6}{\textbf{G}}} per chiarimenti. Elabora specifiche tecniche dettagliate e modelli dei casi d'uso\textsubscript{\scalebox{0.6}{\textbf{G}}}. La sua attività è concentrata principalmente nelle fasi iniziali del progetto\textsubscript{\scalebox{0.6}{\textbf{G}}}, con possibile supporto successivo per l'evoluzione dei requisiti\textsubscript{\scalebox{0.6}{\textbf{G}}}.

\subsection{Progettista\textsubscript{\scalebox{0.6}{\textbf{G}}}}
Il Progettista\textsubscript{\scalebox{0.6}{\textbf{G}}} trasforma i requisiti\textsubscript{\scalebox{0.6}{\textbf{G}}} in un'architettura software robusta e coerente. Definisce le scelte tecnologiche, i pattern architetturali e i diagrammi di sistema. Redige la specifica tecnica e supervisiona l'integrazione tra i vari componenti. Il suo contributo è essenziale in tutte le fasi progettuali, con particolare intensità durante la definizione dell'architettura.

\subsection{Programmatore\textsubscript{\scalebox{0.6}{\textbf{G}}}}
Il Programmatore\textsubscript{\scalebox{0.6}{\textbf{G}}} implementa le soluzioni progettate, sviluppando codice secondo le best practice\textsubscript{\scalebox{0.6}{\textbf{G}}}
 dell'ingegneria del software. Si occupa dell'integrazione dei componenti, della scrittura dei test unitari e della documentazione del codice. Il suo lavoro è distribuito uniformemente durante le fasi di sviluppo e manutenzione del software.

\subsection{Verificatore\textsubscript{\scalebox{0.6}{\textbf{G}}}}
Il Verificatore\textsubscript{\scalebox{0.6}{\textbf{G}}} garantisce la qualità\textsubscript{\scalebox{0.6}{\textbf{G}}} del prodotto attraverso attività di testing, code review e validazione\textsubscript{\scalebox{0.6}{\textbf{G}}}. Verifica\textsubscript{\scalebox{0.6}{\textbf{G}}} la conformità della documentazione agli standard, esegue test\textsubscript{\scalebox{0.6}{\textbf{G}}} di integrazione, e assicura che tutto il materiale prodotto soddisfi i criteri qualitativi definiti. La sua presenza è costante durante l'intero ciclo di vita del progetto
.

\vspace{1cm}


\section{Impegno}
Ogni membro del gruppo si impegna a dedicare al progetto\textsubscript{\scalebox{0.6}{\textbf{G}}} un totale di 92 ore produttive, ripartite equamente tra i ruoli di Responsabile, Amministratore\textsubscript{\scalebox{0.6}{\textbf{G}}}, Analista\textsubscript{\scalebox{0.6}{\textbf{G}}}, Progettista\textsubscript{\scalebox{0.6}{\textbf{G}}}, Programmatore\textsubscript{\scalebox{0.6}{\textbf{G}}} e Verificatore\textsubscript{\scalebox{0.6}{\textbf{G}}}. Seguiranno nei successivi paragrafi i dettagli.

\subsection{Individuale}

{\scriptsize
\begin{center}
\begin{tabular}{|l|c|c|c|c|c|c|c|}
\hline
 & \rotatebox{45}{Responsabile} & \rotatebox{45}{Amministratore} & \rotatebox{45}{Analista} & \rotatebox{45}{Progettista} & \rotatebox{45}{Programmatore} & \rotatebox{45}{Verificatore} & \rotatebox{45}{Totale} \\
\hline
Alberto Autiero & 8 & 7 & 8 & 21 & 24 & 24 & 92 \\
\hline
Marco Favero & 8 & 8 & 8 & 21 & 23 & 24 & 92 \\
\hline
Alberto Pignat & 8 & 7 & 8 & 21 & 24 & 24 & 92 \\
\hline
Marco Piro & 8 & 7 & 8 & 21 & 24 & 24 & 92 \\
\hline
Linor Sadè & 8 & 8 & 8 & 21 & 24 & 23 & 92 \\
\hline
Leonardo Salviato & 8 & 8 & 8 & 21 & 24 & 23 & 92 \\
\hline
Luca Slongo & 8 & 7 & 8 & 21 & 24 & 24 & 92 \\
\hline
\textbf{Totale} & \textbf{56} & \textbf{52} & \textbf{56} & \textbf{147} & \textbf{167} & \textbf{166} & \textbf{644} \\
\hline
\end{tabular}
\end{center}
}

\begin{center}
\textit{Tabella 1: Ore di ogni componente per ciascun ruolo}
\end{center}

Il gruppo ha preso la decisione di dare a ogni membro la possibilità di svolgere ogni ruolo all'interno del progetto\textsubscript{\scalebox{0.6}{\textbf{G}}} in egual misura, per dare la possibilità di imparare come agisce ogni ruolo e non creare differenze tra di noi.

La stima delle ore è stata fatta dando un peso al verificatore molto simile a quello del programmatore\textsubscript{\scalebox{0.6}{\textbf{G}}}, questo è stato fatto per garantire un'alta qualità\textsubscript{\scalebox{0.6}{\textbf{G}}} del codice e limitare al più possibile i bug.

Anche al ruolo di progettista\textsubscript{\scalebox{0.6}{\textbf{G}}} sono state assegnate molte ore, perchè creare una base solida su cui scrivere codice permette, con l'avanzamento del progetto\textsubscript{\scalebox{0.6}{\textbf{G}}} e di conseguenza della sua complessità, di renderlo comunque mantenibile.

Ai ruoli di Responsabile
, Amministratore\textsubscript{\scalebox{0.6}{\textbf{G}}}
 e Analista\textsubscript{\scalebox{0.6}{\textbf{G}}} sono state riservate meno ore, che in caso sia necessario verranno leggermente aumentate riducendo altri ruoli, non perchè siano visti come ruoli meno importanti, ma perchè questi tipi di attività richiedono meno tempo con più attenzione qualitativa.


Il gruppo si impegna a ruotare i ruoli ogni 2 settimane, in linea con gli sprint\textsubscript{\scalebox{0.6}{\textbf{G}}} dell'azienda Eggon.

\newpage

\subsection{Riassunto costi}

{\footnotesize
\begin{center}
\label{tab:costi-ruoli}
\begin{tabular}{|l|c|c|c|}
\hline
\textbf{Ruolo} & \textbf{Costo Orario} & \textbf{Ore} & \textbf{Costo} \\
\hline
Responsabile & 30\euro/h & 56h & 1.680\euro \\
Amministratore & 20\euro/h & 52h & 1.040\euro \\
Analista & 25\euro/h & 56h & 1.400\euro \\
Progettista & 25\euro/h & 147h & 3.675\euro \\
Programmatore & 15\euro/h & 167h & 2.505\euro \\
Verificatore & 15\euro/h & 166h & 2.490\euro \\
\hline
\textbf{Totale} & - & \textbf{644h} & \textbf{12.790\euro} \\
\hline
\end{tabular}
\end{center}
}

\begin{center}
\textit{Tabella 2: riassunto dei costi derivanti dalle ore assegnate a ciascun ruolo}
\end{center}

\vspace{1cm}

\begin{center}
\begin{tikzpicture}
\pie[
    text=legend,
    radius=2.8,
    color={
        primaryblue!80,
        secondaryblue!70,
        green!70,
        orange!70,
        purple!70,
        cyan!70
    },
    explode=0.1,
    before number=,
    after number=,
    sum=auto,
    rotate=90
]{
    8.70/Responsabile,
    8.07/Amministratore,
    8.70/Analista,
    22.83/Progettista,
    25.93/Programmatore,
    25.77/Verificatore
}
\end{tikzpicture}
\end{center}

\begin{center}
\textit{Grafico 1: Distribuzione percentuale delle ore per ruolo}
\end{center}

\newpage

\section{Costi}
Il costo previsto per la realizzazione del progetto\textsubscript{\scalebox{0.6}{\textbf{G}}} è, come anche osservabile dalla \hyperref[tab:costi-ruoli]{\textcolor{secondaryblue}{Tabella 2}}, di \textbf{12.790}\euro.

\section{Consegna}
La data ultima per la consegna del progetto\textsubscript{\scalebox{0.6}{\textbf{G}}} è fissata improrogabilmente al giorno \textbf{21/03/2026}.

\end{document}
