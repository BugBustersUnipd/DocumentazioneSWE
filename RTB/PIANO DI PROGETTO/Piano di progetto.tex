\documentclass[a4paper,12pt]{article}

\usepackage[utf8]{inputenc}
\usepackage[T1]{fontenc}
\usepackage[italian]{babel}
\usepackage[margin=2.5cm]{geometry}
\usepackage{graphicx}
\usepackage{grffile}
\usepackage{booktabs}
\usepackage{setspace}
\usepackage{titlesec}
\usepackage{float}
\usepackage{tabularray}
\usepackage{ifthen}
\usepackage{tcolorbox}
\usepackage{enumitem}
\usepackage{longtable}
\usepackage{array}
\usepackage{tabularx}
\usepackage{caption}
\usepackage{colortbl}
\usepackage{pgfplots}
\usepackage{makecell}
% Comando per la versione corrente
\newcommand{\CurrentVersion}{0.2.0}
\pgfplotsset{compat=1.16}

% Comando per il pedice G compatibile con hyperref
\newcommand{\pediceG}{\texorpdfstring{\textsubscript{\scalebox{0.6}{\textbf{G}}}~}{G}}


% Configurazione delle didascalie per tabelle
\captionsetup[table]{
  position=below,       % Didascalia sotto la tabella
  labelfont=it,         % "Tabella X:" in italico
  textfont=it,          % Testo della didascalia in italico
  justification=centering, % Testo centrato
  singlelinecheck=false % Forza la centratura anche per didascalie lunghe
}
\captionsetup[longtable]{
  position=below,
  labelfont=it,
  textfont=it,
  justification=centering,
  singlelinecheck=false
}

% hyperref dovrebbe essere caricato quasi per ultimo
\usepackage[colorlinks=true,linkcolor=black,urlcolor=primaryblue,citecolor=primaryblue]{hyperref}
\hypersetup{
    pdfborder={0 0 0}, % Rimuove il bordo sottolineato
    allbordercolors={1 1 1}, % Bordi bianchi (invisibili)
}

\definecolor{primaryblue}{RGB}{0,102,204}
\definecolor{secondaryblue}{RGB}{51,153,255}
\definecolor{lightgray}{RGB}{245,245,245}
\definecolor{darkgray}{RGB}{100,100,100}

% Configurazione del titolo delle sezioni
\titleformat{\section}
  {\Large\bfseries\color{primaryblue}}
  {\thesection}{1em}{}

\titleformat{\subsection}
  {\large\bfseries\color{primaryblue}}
  {\thesubsection}{1em}{}

\titleformat{\subsubsection}
  {\normalsize\bfseries\color{primaryblue}}
  {\thesubsubsection}{1em}{}

\setlength{\parskip}{4pt}
\setlength{\parindent}{0pt}

\setlist[itemize]{leftmargin=*,itemsep=3pt}
\setlist[enumerate]{leftmargin=*,itemsep=3pt}

\graphicspath{{./}{../assets/images/}{./images/}}

\usepackage{multirow}
\begin{document}

\begin{center}
  \IfFileExists{../../assets/Logo.jpg}{%
    \includegraphics[width=6cm,height=3cm,keepaspectratio]{../../assets/Logo.jpg} \\[0.8cm]
  }{%
    \fbox{\parbox[c][2.5cm][c]{6cm}{\centering Logo non trovato\\(../../assets/Logo.jpg)}}\\[0.5cm]
  }
  
  {\Large\bfseries\color{primaryblue} BugBusters}\\[0.5cm]

  {\Huge\bfseries\color{primaryblue} Piano di Progetto\pediceG}\\[0.3cm]
  {\Large\color{secondaryblue} Versione \CurrentVersion}\\[0.8cm]
\end{center}

\begin{center}
\begin{tcolorbox}[colback=lightgray,colframe=primaryblue,width=0.85\textwidth,arc=3mm,boxrule=0.5pt]
% Usa tabularx con una colonna fissa per l'etichetta e una colonna X per il contenuto
\begin{tabularx}{\linewidth}{@{}>{\raggedright\arraybackslash}p{3.5cm}>{\raggedright\arraybackslash}X@{}}
	{Stato} & In redazione \\
	{Verificatore\pediceG} &  Luca Slongo \\
	{Redattore} & Alberto Autiero \\
	{Destinatari} & BugBusters, Eggon, Prof. Tullio Vardanega, Prof. Riccardo Cardin \\
\end{tabularx}
\end{tcolorbox}
\end{center}

\vspace{0.5cm}

\begin{center}
\begin{tcolorbox}[colback=secondaryblue!10,colframe=secondaryblue,width=0.9\textwidth,arc=3mm,boxrule=0.8pt,title={\bfseries\color{primaryblue}Descrizione}]
Il Piano di Progetto\pediceG riporta il riepilogo delle attività effettuate e quelle ancora da svolgere dal team BugBusters.\end{tcolorbox}
\end{center}

\newpage

% Registro delle Modifiche (pagina 2)
\section*{\color{primaryblue}Registro delle Modifiche}

\arrayrulecolor{primaryblue}
{\footnotesize
\begin{tabularx}{\textwidth}{|>{\raggedright\arraybackslash}p{1.5cm}|>{\raggedright\arraybackslash}p{2cm}|X|>{\raggedright\arraybackslash}p{2cm}|>{\raggedright\arraybackslash}p{2cm}|>{\raggedright\arraybackslash}p{2cm}|}
\hline
\rowcolor{primaryblue!40}
\textbf{\color{white} Versione} & \textbf{\color{white} Data} & \textbf{\color{white} Descrizione} & \textbf{\color{white} Redatto} & \textbf{\color{white} Verificato} & \textbf{\color{white} Approvato} \\
1.0.0 & 29/01/2026 & Approvazione\pediceG del documento & - & - & Luca Slongo \\
\hline
0.2.0 & 29/01/2026 & Verifica\pediceG del documento & - & Marco Favero & - \\
\hline
0.1.8 & 29/01/2026 & Aggiunti ai termini presenti nel Glossario\pediceG la G & Luca Slongo & - & - \\
\hline
0.1.7 & 19/01/2026 & Redatta la pianificazione dello sprint\pediceG 6 & Leonardo Salviato & - & - \\
\hline 
0.1.6 & 18/01/2026 & Compilate le attività "Redazione Analisi dei Requisiti nella sezione Attività previste per l'RTB\pediceG & Linor Sadè & - & - \\
\hline 
0.1.5 & 14/01/2026 & Redatta la pianificazione dello sprint\pediceG 5 & Linor Sadè & - & - \\
\hline 
0.1.4 & 12/01/2026 & Compilate le attività "Redazione Norme di Progetto\pediceG" e "Realizzazione PoC\pediceG" nella sezione Attività previste per l'RTB\pediceG & Linor Sadè & - & - \\
\hline 
 0.1.3 & 10/01/2026 & Compilazione tabella Attività previste per la Requirements and Technology Baseline (RTB)\pediceG & Linor Sadè & & - \\
\hline
 0.1.2 & 08/01/2026 & Creazione grafici dei primi 4 Sprint\pediceG & Linor Sadè & & - \\
\hline
0.1.1 & 05/01/2026 & Redatta la pianificazione dello sprint\pediceG 4 e sistemazione di alcune sezioni & Marco Piro & & - \\
\hline
0.1.0 & 03/01/2026 & Verifica\pediceG iniziale del documento & - & Luca Slongo & - \\
\hline
0.0.5 & 22/12/2025 & Redatta la pianificazione dello sprint\pediceG 3 & Alberto Pignat & - & - \\
\hline
0.0.4 & 18/12/2025 & Redatta la pianificazione dello sprint\pediceG 2 & Alberto Pignat & - & - \\
\hline
0.0.3 & 11/12/2025 & Aggiunta la sezione Analisi e gestione dei rischi\pediceG e redatta la pianificazione dello sprint\pediceG 1 & Alberto Pignat & - & - \\
\hline
0.0.2 & 03/12/2025 & Aggiunta la sezione Introduzione  & Marco Favero & - & - \\
\hline
0.0.1 & 01/12/2025 & Prima stesura della struttura del documento & Marco Favero & - & - \\
\hline
\end{tabularx}
}

\newpage

% Indice cliccabile
\setcounter{tocdepth}{3} % Mostra fino al livello di sottosottosezione (3)
\tableofcontents
\listoftables
\listoffigures

\newpage
\section{Introduzione}

\subsection{Informazioni generali}
Il Piano di Progetto\pediceG è un documento che descrive le attività svolte e quelle da svolgere durante la realizzazione del progetto di Ingegneria del Software. Riveste un ruolo fondamentale per una corretta pianificazione, poiché confronta i tempi previsti con i tempi effettivamente impiegati per ogni attività e individua i rischi\pediceG potenziali legati alle attività programmate. Dal momento che una pianificazione dettagliata a lungo termine risulterebbe poco efficace, le attività vengono pianificate di volta in volta: per questo motivo il Piano di Progetto\pediceG è un documento dinamico, che verrà aggiornato con un approccio incrementale integrando nuove informazioni man mano che il progetto\pediceG procede.

\subsection{Glossario\pediceG}
La realizzazione di un sistema software richiede, ancor prima della scrittura del codice, un'attenta fase di confronto, analisi e progettazione. Per supportare il lavoro asincrono del team, tutte le informazioni prodotte durante queste attività vengono opportunamente documentate.

È tuttavia del tutto prevedibile che tali documenti possano contenere termini tecnici o espressioni non immediatamente comprensibili. Per questo motivo è stato redatto un Glossario\pediceG, che raccoglie e spiega in modo puntuale i termini utilizzati. Il Glossario\pediceG è soggetto ad aggiornamenti continui ed è consultabile nella sua versione corrente al seguente indirizzo: DA INSERIRE QUANDO VERRÀ CARICATO SUL SITO.

I termini che dispongono di una definizione nel Glossario\pediceG saranno contrassegnati nel modo seguente: parola\pediceG.

\subsection{Fonti}

\subsubsection{Riferimenti normativi}
\begin{itemize}
    \item Capitolato\pediceG d'appalto C5: NEXUM - Eggon \\
    \url{https://www.math.unipd.it/~tullio/IS-1/2025/Progetto/C5.pdf} \\
    Ultimo accesso: XXX
    \item Norme di Progetto\pediceG versione 1.0.0 \\
    \url{DA AGGIUNGERE APPENA CARICATO SUL SITO} \\
    Ultimo accesso: XXX
\end{itemize}

\subsubsection{Riferimenti informativi}
\begin{itemize}
    \item I processi di ciclo di vita del software \\
    \url{https://www.math.unipd.it/~tullio/IS-1/2025/Dispense/T02.pdf} \\
    Ultimo accesso: XXX
    \item Gestione di progetto \\
    \url{https://www.math.unipd.it/~tullio/IS-1/2025/Dispense/T04.pdf} \\
    Ultimo accesso: XXX
      \item Glossario\pediceG ver. 1.0.0 \\
    \url{DA AGGIUNGERE APPENA SI CARICA SUL SITO} \\
    Ultimo accesso: XXX
\end{itemize}

\newpage

\section{Analisi e gestione dei rischi\pediceG}

\subsection{Introduzione}
Un'analisi e una classificazione efficaci dei possibili rischi legati alle attività da svolgere rappresentano una parte fondamentale nella redazione di un piano di progetto\pediceG. Una buona analisi consente infatti di prevedere quali attività potrebbero richiedere più tempo del previsto e, di conseguenza, di valutare se il carico di lavoro inserito nel backlog\pediceG settimanale è appropriato o eccessivo.

Un'adeguata analisi e gestione dei rischi\pediceG si articola in quattro fasi:
\begin{itemize}
\item \textbf{Identificazione}: individuazione dei possibili rischi associati a un'attività, considerando non solo l'ambito progettuale ma anche quello personale;
\item \textbf{Analisi}: valutazione della probabilità\pediceG che ciascun rischio\pediceG si verifichi e dei potenziali impatti sul successo dello sprint\pediceG e del progetto\pediceG complessivo;
\item \textbf{Pianificazione}: definizione delle strategie per ridurre la probabilità\pediceG di occorrenza dei rischi o, qualora non sia possibile prevenirli, per mitigarne gli effetti negativi;
\item \textbf{Controllo}: monitoraggio continuo delle attività, volto a rilevare tempestivamente l'insorgere di un rischio\pediceG e ad applicare le procedure di mitigazione\pediceG prestabilite.
\end{itemize}

È ragionevole supporre che, a causa di un'esperienza limitata, le misure di mitigazione\pediceG adottate possano rivelarsi inizialmente inefficaci. Per questo motivo è fondamentale registrare e analizzare gli errori emersi durante la fase di controllo, al fine di perfezionare progressivamente le strategie utilizzate.

Nel seguito, il gruppo BugBusters illustra i rischi individuati, suddivisi in tre categorie:
\begin{itemize}
\item RT: Rischio Tecnologico\pediceG;
\item RI: Rischio Individuale\pediceG;
\item RG: Rischio Globale\pediceG.
\end{itemize}

\newpage
\subsection{Rischio Tecnologico\pediceG}

\subsubsection{RT1: Rischio Tecnologico\pediceG legato alla tecnologia utilizzata}
\begin{table}[H]
\centering
\begin{tabular}{|>{\raggedright\arraybackslash}p{4cm}|>{\raggedright\arraybackslash}p{10cm}|}
\hline
\textbf{Codice} & RT1 \\
\hline
\textbf{Nome} & Rischio Tecnologico\pediceG legato alla tecnologia utilizzata \\
\hline
\textbf{Descrizione} & Rischio\pediceG derivante da inesperienza o conoscenza limitata di una tecnologia, un linguaggio o uno strumento specifico necessario per lo sviluppo del progetto\pediceG. \\
\hline
\textbf{Mitigazione\pediceG} & Prevedere nella pianificazione del tempo l'attività di formazione personale. \\
\hline
\textbf{Frequenza probabile} & Alta \\
\hline
\textbf{Pericolosità\pediceG} & Elevata \\
\hline
\end{tabular}
\caption{Informazioni sul rischio RT1}
\label{tab:rt1}
\end{table}

\subsubsection{RT2: Rischio Tecnologico\pediceG legato a errori nel codice}
\begin{table}[H]
\centering
\begin{tabular}{|>{\raggedright\arraybackslash}p{4cm}|>{\raggedright\arraybackslash}p{10cm}|}
\hline
\textbf{Codice} & RT2 \\
\hline
\textbf{Nome} & Rischio Tecnologico\pediceG legato a errori nel codice \\
\hline
\textbf{Descrizione} & È improbabile che il codice scritto funzioni correttamente al primo tentativo. Inoltre, un codice apparentemente funzionante può rivelare successivamente bug durante test\pediceG più approfonditi, richiedendo ulteriore tempo per l'analisi e la correzione. \\
\hline
\textbf{Mitigazione\pediceG} & Il Programmatore\pediceG tenta inizialmente di risolvere autonomamente il problema. In caso di difficoltà, chiederà supporto a membri del gruppo più esperti. Per problemi particolarmente gravi, le attività meno urgenti verranno posticipate per dedicare risorse alla risoluzione. \\
\hline
\textbf{Frequenza probabile} & Alta \\
\hline
\textbf{Pericolosità\pediceG} & Media \\
\hline
\end{tabular}
\caption{Informazioni sul rischio RT2}
\label{tab:rt2}
\end{table}

\newpage
\subsection{Rischio Individuale\pediceG}

\subsubsection{RI1: Rischio Individuale\pediceG derivante dalle altre attività universitarie}
\begin{table}[H]
\centering
\begin{tabular}{|>{\raggedright\arraybackslash}p{4cm}|>{\raggedright\arraybackslash}p{10cm}|}
\hline
\textbf{Codice} & RI1 \\
\hline
\textbf{Nome} & Rischio Individuale\pediceG derivante dalle altre attività universitarie \\
\hline
\textbf{Descrizione} & I membri del gruppo frequentano altri corsi di insegnamento oltre a Ingegneria del Software, pertanto è probabile che si verifichino periodi di minore disponibilità dovuti a esami, scadenze di altri progetti o impegni correlati. \\
\hline
\textbf{Mitigazione\pediceG} & Il membro interessato deve comunicare tempestivamente la propria ridotta disponibilità. Gli altri componenti si redistribuiranno il carico di lavoro. Il membro interessato recupererà le attività non appena possibile. \\
\hline
\textbf{Frequenza probabile} & Media \\
\hline
\textbf{Pericolosità\pediceG} & Media \\
\hline
\end{tabular}
\caption{Informazioni sul rischio RI1}
\label{tab:ri1}
\end{table}

\subsubsection{RI2: Rischio Individuale\pediceG derivato da improvviso impegno o indisponibilità personale}
\begin{table}[H]
\centering
\begin{tabular}{|>{\raggedright\arraybackslash}p{4cm}|>{\raggedright\arraybackslash}p{10cm}|}
\hline
\textbf{Codice} & RI2 \\
\hline
\textbf{Nome} & Rischio Individuale\pediceG derivato da improvviso impegno o indisponibilità personale \\
\hline
\textbf{Descrizione} & Un componente del gruppo deve affrontare un impegno imprevisto (personale, familiare, di salute) che ne impedisce temporaneamente la partecipazione attiva al progetto\pediceG. \\
\hline
\textbf{Mitigazione\pediceG} & Il membro interessato deve comunicare tempestivamente l'indisponibilità al resto del gruppo. Le attività assegnate al membro verranno ridistribuite o posticipate. Al termine dell'impedimento, il membro si impegnerà a recuperare il lavoro eventualmente pendente. \\
\hline
\textbf{Frequenza probabile} & Media \\
\hline
\textbf{Pericolosità\pediceG} & Media \\
\hline
\end{tabular}
\caption{Informazioni sul rischio RI2}
\label{tab:ri2}
\end{table}

\newpage
\subsection{Rischio Globale\pediceG}

\subsubsection{RG1: Rischio Globale\pediceG derivato da forte disaccordo nel gruppo}
\begin{table}[H]
\centering
\begin{tabular}{|>{\raggedright\arraybackslash}p{4cm}|>{\raggedright\arraybackslash}p{10cm}|}
\hline
\textbf{Codice} & RG1 \\
\hline
\textbf{Nome} & Rischio Globale\pediceG derivato da forte disaccordo nel gruppo \\
\hline
\textbf{Descrizione} & Insorgenza di un contrasto significativo tra i membri del gruppo riguardante una decisione tecnica, metodologica o organizzativa. \\
\hline
\textbf{Mitigazione\pediceG} & Ogni parte illustra le ragioni della propria posizione in un tempo definito. Si procede quindi a una valutazione collettiva per trovare un accordo o, in ultima istanza, a una votazione tramite sondaggio. \\
\hline
\textbf{Frequenza probabile} & Bassa \\
\hline
\textbf{Pericolosità\pediceG} & Bassa \\
\hline
\end{tabular}
\caption{Informazioni sul rischio RG1}
\label{tab:rg1}
\end{table}

\subsubsection{RG2: Rischio Globale\pediceG derivato da malcomprensione del capitolato\pediceG}
\begin{table}[H]
\centering
\begin{tabular}{|>{\raggedright\arraybackslash}p{4cm}|>{\raggedright\arraybackslash}p{10cm}|}
\hline
\textbf{Codice} & RG2 \\
\hline
\textbf{Nome} & Rischio Globale\pediceG derivato da malcomprensione del capitolato\pediceG \\
\hline
\textbf{Descrizione} & Un prodotto\pediceG (documentale o software) sviluppato dal gruppo non rispetta le specifiche o i vincoli definiti nel capitolato\pediceG d'appalto. \\
\hline
\textbf{Mitigazione\pediceG} & Riunioni periodiche con la proponente\pediceG, revisione dei requisiti\pediceG e feedback continuo soprattutto ad inizio e fine sprint\pediceG. \\
\hline
\textbf{Frequenza probabile} & Media \\
\hline
\textbf{Pericolosità\pediceG} & Media \\
\hline
\end{tabular}
\caption{Informazioni sul rischio RG2}
\label{tab:rg2}
\end{table}

\subsubsection{RG3: Rischio Globale\pediceG derivato da sottostima di attività}
\begin{table}[H]
\centering
\begin{tabular}{|>{\raggedright\arraybackslash}p{4cm}|>{\raggedright\arraybackslash}p{10cm}|}
\hline
\textbf{Codice} & RG3 \\
\hline
\textbf{Nome} & Rischio Globale\pediceG derivato da sottostima di attività \\
\hline
\textbf{Descrizione} & Un'attività pianificata richiede, per il suo completamento, più tempo di quanto inizialmente preventivato. \\
\hline
\textbf{Mitigazione\pediceG} & I responsabili dell'attività segnalano il potenziale ritardo il prima possibile. Il gruppo valuterà l'opportunità di assegnare risorse aggiuntive all'attività o di posticipare attività secondarie per concentrarsi su quella critica. \\
\hline
\textbf{Frequenza probabile} & Alta \\
\hline
\textbf{Pericolosità\pediceG} & Elevata \\
\hline
\end{tabular}
\caption{Informazioni sul rischio RG3}
\label{tab:rg3}
\end{table}

\subsubsection{RG4: Rischio Globale\pediceG derivato da sovrastima di attività}
\begin{table}[H]
\centering
\begin{tabular}{|>{\raggedright\arraybackslash}p{4cm}|>{\raggedright\arraybackslash}p{10cm}|}
\hline
\textbf{Codice} & RG4 \\
\hline
\textbf{Nome} & Rischio Globale\pediceG derivato da sovrastima di attività \\
\hline
\textbf{Descrizione} & Un'attività pianificata viene completata in un tempo significativamente inferiore a quello preventivato. \\
\hline
\textbf{Mitigazione\pediceG} & I membri che hanno completato in anticipo l'attività ne danno comunicazione al gruppo. Le risorse liberate verranno impiegate per supportare altre attività in ritardo o per iniziare anticipatamente la successiva attività, se fattibile nel tempo residuo. \\
\hline
\textbf{Frequenza probabile} & Media \\
\hline
\textbf{Pericolosità\pediceG} & Bassa \\
\hline
\end{tabular}
\caption{Informazioni sul rischio RG4}
\label{tab:rg4}
\end{table}

\newpage

\section{Pianificazione nel lungo termine}

Come anticipato nella Dichiarazione degli Impegni, il gruppo prevede di terminare il progetto\pediceG entro e non oltre il giorno 21 Marzo 2026, con un budget di spesa fissato a 12790€.

Al momento della candidatura\pediceG è stato stimato il seguente prospetto dei costi:

\begin{table}[H]
\centering
\begin{tabular}{|l|c|c|c|}
\hline
\textbf{Ruolo} & \textbf{Costo Orario} & \textbf{Ore} & \textbf{Costo} \\
\hline
Responsabile\pediceG & 30€/h & 56h & 1.680€ \\
\hline
Amministratore\pediceG & 20€/h & 52h & 1.040€ \\
\hline
Analista\pediceG & 25€/h & 56h & 1.400€ \\
\hline
Progettista\pediceG & 25€/h & 147h & 3.675€ \\
\hline
Programmatore\pediceG & 15€/h & 167h & 2.505€ \\
\hline
Verificatore\pediceG & 15€/h & 166h & 2.490€ \\
\hline
\textbf{Totale} & & \textbf{644h} & \textbf{12.790€} \\
\hline
\end{tabular}
\caption{Riassunto dei costi previsti derivanti dalle ore assegnate a ciascun ruolo}
\label{tab:costi-previsti}
\end{table}

Si stima inoltre una candidatura\pediceG per la Requirements and Technology Baseline (RTB)\pediceG entro il 20 Gennaio 2026.

Seguiranno ora le attività previste per la Requirements and Technology Baseline (RTB)\pediceG e la Product Baseline (PB)\pediceG: tali sezioni saranno utili per calendarizzare correttamente quanto realizzare in ogni sprint\pediceG.

\subsection{Attività previste per la Requirements and Technology Baseline (RTB)\pediceG}
{
% 1. Pulizia template
\DefTblrTemplate{caption}{default}{}
\DefTblrTemplate{capcont}{default}{}
\DefTblrTemplate{contfoot}{default}{}
\DefTblrTemplate{conthead}{default}{}
\DefTblrTemplate{contfoot-text}{default}{}
\DefTblrTemplate{conthead-text}{default}{}

% 2. Ancora per l'indice
\phantomsection

\begin{longtblr}[
  entry = {Attività previste per la RTB},
  label = {tab:attivita-rtb},
]{
  colspec = {|Q[c,m,3.0cm]|Q[l,m,4.8cm]|Q[l,m,3.2cm]|Q[c,m,2.8cm]|},
  vlines = {primaryblue},
  row{1} = {font=\bfseries},
  % Linea orizzontale sopra l'intestazione
  hline{1} = {primaryblue},
  % Linea orizzontale sotto l'intestazione
  hline{2} = {primaryblue},
  % Linee orizzontali PARZIALI (solo colonne 2 e 3) all'interno di ogni attività
  hline{3,4,5} = {2-3}{primaryblue},     % Dentro Analisi dei Requisiti
  hline{7,8} = {2-3}{primaryblue},       % Dentro Piano di Progetto
  hline{10,11,12,13,14,15} = {2-3}{primaryblue}, % Dentro Piano di Qualifica
  hline{17,18,19,20} = {2-3}{primaryblue}, % Dentro Norme di Progetto
  hline{23,24,25} = {2-3}{primaryblue},  % Dentro Realizzazione PoC
  % Linee orizzontali COMPLETE tra le attività principali
  hline{6} = {primaryblue},   % Dopo Analisi dei Requisiti
  hline{9} = {primaryblue},   % Dopo Piano di Progetto
  hline{16} = {primaryblue},  % Dopo Piano di Qualifica
  hline{21} = {primaryblue},  % Dopo Norme di Progetto
  hline{22} = {primaryblue},  % Prima di Glossario (MANCAVA!)
  hline{23} = {primaryblue},  % Dopo Glossario (MANCAVA!)
  hline{26} = {primaryblue},  % Fine tabella
}

% --- INTESTAZIONE ---
Attività & Descrizione & Periodo di svolgimento & Stato \\

% --- CONTENUTO ---
\SetCell[r=4]{c} \textbf{Redazione Analisi dei Requisiti\pediceG}
  & Introduzione e stesura paragrafi esplicativi del documento
  & Sprint\pediceG 1
  & \SetCell[r=4]{c} Completato per RTB \\
  & Individuazione requisiti funzionali\pediceG e non funzionali\pediceG con supporto dell'azienda
  & Sprint\pediceG 2
  & \\
  & Scrittura casi d'uso\pediceG
  & Sprint\pediceG 3 e 4
  & \\
  & Inserimento diagrammi UML\pediceG
  & Sprint\pediceG 5
  & \\

\SetCell[r=3]{c} \textbf{Redazione Piano di Progetto\pediceG}
  & Introduzione del documento
  & Sprint\pediceG 2
  & \SetCell[r=3]{c} Completato per RTB \\
  & Analisi e gestione dei rischi\pediceG
  & Tutti gli sprint\pediceG
  & \\
  & Pianificazione nel lungo e breve periodo
  & Tutti gli sprint\pediceG
  & \\

\SetCell[r=7]{c} \textbf{Redazione Piano di Qualifica\pediceG}
  & Prima stesura del documento
  & Sprint\pediceG 2
  & \SetCell[r=7]{c} Completato per RTB \\
  & Aggiunta sezioni 4 e 5
  & Sprint\pediceG 4
  & \\
  & Aggiunta Test\pediceG di Sistema e di Accettazione
  & Sprint\pediceG 4
  & \\
  & Aggiornamento Test\pediceG, aggiunti test\pediceG di sistema prestazionali, eliminata metrica errori ortografici
  & Sprint\pediceG 5
  & \\
  & Aggiunto contenuto alla sezione 5
  & Sprint\pediceG 6
  & \\
  & Aggiornamento Test\pediceG, aggiunti test\pediceG di sistema casi limite e integrazione, cambiate alcune metriche di prodotto\pediceG, aggiunta matrice di tracciamento
  & Sprint\pediceG 6
  & \\
  & Aggiunti grafici metriche, aggiornamento Test\pediceG, rimossa matrice di tracciamento, aggiunta descrizione grafici metriche
  & Sprint\pediceG 6
  & \\

\SetCell[r=5]{c} \textbf{Redazione Norme di Progetto\pediceG}
  & Introduzione del documento
  & Sprint\pediceG 1
  & \SetCell[r=5]{c} Completato per RTB \\
  & Redazione dei processi primari
  & Sprint\pediceG 1, 2 e modifiche minime successive
  & \\
  & Redazione dei processi di supporto
  & Sprint\pediceG 2 e modifiche successive
  & \\
  & Redazione dei processi organizzativi
  & Fine sprint\pediceG 2, inizio sprint\pediceG 3
  & \\
  & Redazione sezione Standard di progetto
  & Sprint\pediceG 3
  & \\

\textbf{Redazione Glossario\pediceG}
  & Redazione e aggiornamento dei termini
  & Tutti gli sprint\pediceG
  & Completato per RTB \\

\SetCell[r=4]{c} \textbf{Realizzazione PoC\pediceG}
  & Prima implementazione\pediceG solo con Rails\pediceG
  & Sprint\pediceG 3
  & \SetCell[r=4]{c} {\small Completato} \\
  & Ridefinizione struttura backend\pediceG-frontend\pediceG aggiungendo Angular\pediceG
  & Sprint\pediceG 3
  & \\
  & Implementazione\pediceG e modifica funzionalità\pediceG backend\pediceG con relativa implementazione\pediceG in frontend\pediceG
  & Sprint\pediceG 4
  & \\
  & Refactoring del codice
  & Sprint\pediceG 5
  & \\

\end{longtblr}

% 3. Didascalia manuale sotto
\nopagebreak
\begin{center}
\itshape Tabella \thetable: Attività previste per la Requirements and Technology Baseline (RTB)\pediceG
\end{center}

% 4. Sistemazione contatori
\addtocounter{table}{-1}
\refstepcounter{table}
}

\section{Pianificazione a breve termine}

\subsection{Premessa}
Il gruppo BugBusters ha adottato una metodologia di sviluppo di tipo Agile\pediceG. Si è valutato che un ciclo di lavoro di due settimane rappresenti il periodo ottimale per generare incrementi di prodotto\pediceG tangibili e di valore. Pertanto, lo svolgimento del progetto\pediceG sarà strutturato in sprint\pediceG della durata di due settimane

All'apertura di ogni sprint\pediceG si definiscono le attività dello sprint\pediceG e si attua la rotazione dei ruoli. Sebbene pianificata all'inizio, la distribuzione dei compiti potrà essere modificata a seconda delle necessità operative ed organizzative, con il fine ultimo di permettere a ciascun membro di ricoprire ogni ruolo.

Il gruppo si impegna a mantenere un confronto costante con il proponente\pediceG Eggon attraverso incontri regolari, in presenza o per via telematica. Tali colloqui, previsti con cadenza bisettimanale ma flessibili in base alle necessità emergenti, garantiscono un allineamento continuo con le aspettative del committente\pediceG e consentono di validare tempestivamente il lavoro svolto.

\vfill
\begin{center}
    {\small\color{darkgray} Documento redatto e approvato dal gruppo BugBusters.}
\end{center}

\newpage
\subsection{Requirements and Technology Baseline (RTB)\pediceG}

\subsubsection{Sprint\pediceG 1}

\textbf{Inizio:} 10/11/2025 \\
\textbf{Fine prevista:} 23/11/2025 \\
\textbf{Fine reale:} 23/11/2025 \\
\textbf{Giorni di ritardo:} 0

\noindent\textbf{Informazioni generali e attività da svolgere}\\
Questo primo sprint\pediceG ha l'obiettivo principale di risolvere tutti i problemi sorti durante la candidatura\pediceG; successivamente, avverrà la redazione dei documenti necessari per un buon inizio dei lavori.\\

In particolare, le attività svolte sono:
\begin{itemize}
    \item Aggiornamento del sito web
    \item Miglioramento del processo di redazione e verifica\pediceG dei verbali
    \item Formalizzazione della tabella decisione/azione nei verbali
    \item Redazione del verbale interno\pediceG del 06/11/2025
    \item Redazione del verbale interno\pediceG del 12/11/2025
    \item Redazione del verbale interno\pediceG del 17/11/2025
    \item Redazione del verbale interno\pediceG del 19/11/2025
    \item Redazione del verbale esterno\pediceG del 12/11/2025
    \item Prima redazione del glossario\pediceG
    \item Prima redazione dell'Analisi dei requisiti\pediceG
    \item Prima redazione delle Norme di Progetto\pediceG
    \item Stabilire un incontro con l'azienda proponente\pediceG Eggon
\end{itemize}

\noindent\textbf{Rischi attesi}\\
I componenti del team BugBusters ritengono possibili i seguenti rischi\pediceG:
\begin{itemize}
    \item RG2: Rischio Globale\pediceG derivato da malcomprensione del capitolato\pediceG
    \item RI1: Rischio Individuale\pediceG derivante dalle altre attività universitarie
\end{itemize}

\noindent\textbf{Preventivo\pediceG}\\
Si prospetta l'utilizzo delle seguenti risorse:

\begin{table}[H]
\centering
\scriptsize
\begin{tabular}{|l|c|c|c|c|c|c|}
\hline
 & \rotatebox{45}{Responsabile\pediceG} & \rotatebox{45}{Amministratore\pediceG} & \rotatebox{45}{Analista\pediceG} & \rotatebox{45}{Progettista\pediceG} & \rotatebox{45}{Programmatore\pediceG} & \rotatebox{45}{Verificatore\pediceG} \\
\hline
Alberto Autiero & 8 & - & - & - & - & - \\
\hline
Marco Favero & - & 8 & - & - & - & - \\
\hline
Alberto Pignat & - & - & 8 & - & - & - \\
\hline
Marco Piro & - & - & - & - & - &  12 \\
\hline
Linor Sadè & - & - & 8 & - & - & - \\
\hline
Leonardo Salviato & - & - & 4 & - & - & - \\
\hline
Luca Slongo & - & - & 4 & - & - & - \\
\hline
\end{tabular}
\caption{Preventivo\pediceG per componenti nello Sprint\pediceG 1}
\label{tab:preventivo-sprint1}
\end{table}

\begin{figure}[H]
\centering
\begin{tikzpicture}[scale=1.3]
  % Responsabile: 15.38% (55.4°)
  \draw[fill=blue!60, draw=black, line width=1.5pt] 
    (0,0) -- (2,0) arc (0:55.4:2) -- cycle;
  \node at (27.7:1.2) {\small\bfseries 15.4\%};
  
  % Amministratore: 15.38% (55.4° cumulative)
  \draw[fill=red!60, draw=black, line width=1.5pt] 
    (0,0) -- (55.4:2) arc (55.4:110.8:2) -- cycle;
  \node[rotate=83.1] at (83.1:1.5) {\small\bfseries 15.4\%};
  
  % Analista: 46.15% (166.1° cumulative)
  \draw[fill=orange!60, draw=black, line width=1.5pt] 
    (0,0) -- (110.8:2) arc (110.8:276.9:2) -- cycle;
  \node at (193.85:1.2) {\small\bfseries 46.2\%};
  
  % Verificatore: 23.08% (83.1° cumulative to 360°)
  \draw[fill=green!60, draw=black, line width=1.5pt] 
    (0,0) -- (276.9:2) arc (276.9:360:2) -- cycle;
  \node at (318.45:1.2) {\small\bfseries 23.1\%};
  
  % Legenda
  \draw[fill=blue!60, draw=black] (3.5, 1.5) rectangle (3.8, 1.8);
  \node[anchor=west, font=\small] at (4, 1.65) {Responsabile\pediceG};
  
  \draw[fill=red!60, draw=black] (3.5, 1) rectangle (3.8, 1.3);
  \node[anchor=west, font=\small] at (4, 1.15) {Amministratore\pediceG};
  
  \draw[fill=orange!60, draw=black] (3.5, 0.5) rectangle (3.8, 0.8);
  \node[anchor=west, font=\small] at (4, 0.65) {Analista\pediceG};
  
  \draw[fill=green!60, draw=black] (3.5, 0) rectangle (3.8, 0.3);
  \node[anchor=west, font=\small] at (4, 0.15) {Verificatore\pediceG};
\end{tikzpicture}
\caption{Grafico 1: Sprint\pediceG 1 - Preventivo\pediceG}
\label{fig:grafico-sprint1}
\end{figure}

\noindent\textbf{Consuntivo\pediceG}

\begin{table}[H]
\centering
\scriptsize
\begin{tabular}{|l|c|c|c|c|c|c|}
\hline
 & \rotatebox{45}{Responsabile\pediceG} & \rotatebox{45}{Amministratore\pediceG} & \rotatebox{45}{Analista\pediceG} & \rotatebox{45}{Progettista\pediceG} & \rotatebox{45}{Programmatore\pediceG} & \rotatebox{45}{Verificatore\pediceG} \\
\hline
Alberto Autiero & 8 & - & - & - & - & - \\
\hline
Marco Favero & - & 8 & - & - & - & - \\
\hline
Alberto Pignat & - & - & 7 \textcolor{green!50!black}{(-1)} & - & - & - \\
\hline
Marco Piro & - & - & - & - & - &  10 \textcolor{green!50!black}{(-2)} \\
\hline
Linor Sadè & - & - & 8 & - & - & - \\
\hline
Leonardo Salviato & - & - & 4 & - & - & - \\
\hline
Luca Slongo & - & - & 4 & - & - & - \\
\hline
\end{tabular}
\caption{Consuntivo\pediceG per componenti nello Sprint\pediceG 1}
\label{tab:consuntivo-sprint1}
\end{table}

\begin{figure}[H]
\centering
\begin{tikzpicture}[scale=1.3]
  % Responsabile: 16.33% (58.8°)
  \draw[fill=blue!60, draw=black, line width=1.5pt] 
    (0,0) -- (2,0) arc (0:58.8:2) -- cycle;
  \node at (29.4:1.2) {\small\bfseries 16.3\%};
  
  % Amministratore: 16.33% (58.8° cumulative)
  \draw[fill=red!60, draw=black, line width=1.5pt] 
    (0,0) -- (58.8:2) arc (58.8:117.6:2) -- cycle;
  \node[rotate=88.2] at (88.2:1.5) {\small\bfseries 16.3\%};
  
  % Analista: 46.94% (169.0° cumulative)
  \draw[fill=orange!60, draw=black, line width=1.5pt] 
    (0,0) -- (117.6:2) arc (117.6:286.6:2) -- cycle;
  \node at (202.1:1.2) {\small\bfseries 46.9\%};
  
  % Verificatore: 20.41% (73.5° cumulative to 360°)
  \draw[fill=green!60, draw=black, line width=1.5pt] 
    (0,0) -- (286.6:2) arc (286.6:360:2) -- cycle;
  \node at (323.3:1.2) {\small\bfseries 20.4\%};
  
  % Legenda
  \draw[fill=blue!60, draw=black] (3.5, 1.5) rectangle (3.8, 1.8);
  \node[anchor=west, font=\small] at (4, 1.65) {Responsabile\pediceG};
  
  \draw[fill=red!60, draw=black] (3.5, 1) rectangle (3.8, 1.3);
  \node[anchor=west, font=\small] at (4, 1.15) {Amministratore\pediceG};
  
  \draw[fill=orange!60, draw=black] (3.5, 0.5) rectangle (3.8, 0.8);
  \node[anchor=west, font=\small] at (4, 0.65) {Analista\pediceG};
  
  \draw[fill=green!60, draw=black] (3.5, 0) rectangle (3.8, 0.3);
  \node[anchor=west, font=\small] at (4, 0.15) {Verificatore\pediceG};
\end{tikzpicture}
\caption{Grafico Consuntivo: Sprint\pediceG 1}
\label{fig:grafico-consuntivo-sprint1}
\end{figure}

\noindent\textbf{Aggiornamento delle risorse rimanenti}

\begin{table}[H]
\centering
\footnotesize
\begin{tabular}{|l|c|c|c|c|c|}
\hline
Ruolo & Costo & Ore & Costo & Ore rimanenti & Budget rimanente \\
\hline
Responsabile\pediceG & 30€/h & 8 & 240€ & 48 & 1440€ \\
\hline
Amministratore\pediceG & 20€/h & 8 & 160€ & 44 & 880€ \\
\hline
Analista\pediceG & 25€/h & 23 & 575€ & 33 & 825€ \\
\hline
Progettista\pediceG & 25€/h & - & - & 147 & 3675€ \\
\hline
Programmatore\pediceG & 15€/h & - & - & 167 & 2505€ \\
\hline
Verificatore\pediceG & 15€/h & 10 & 150€ & 156 & 2340€ \\
\hline
\textbf{Totale} & - & \textbf{49} & \textbf{1125€} & \textbf{595} & \textbf{11665€} \\
\hline
\end{tabular}
\caption{Variazione nelle risorse disponibili nello Sprint\pediceG 1}
\label{tab:risorse-sprint1}
\end{table}

\noindent\textbf{Rischi incontrati}\\
Durante questo primo sprint\pediceG si è concretizzato il rischio RT1: Rischio Tecnologico\pediceG legato alla tecnologia utilizzata, nella redazione dell'Analisi dei requisiti\pediceG, dovuta a conoscenza limitata delle tecnologie e soprattutto a'inesperienza.\\

\noindent\textbf{Retrospettiva\pediceG}\\
Il primo sprint\pediceG ha avuto come obiettivo principale la prima stesura di Analisi dei requisiti\pediceG, Norme di Progetto\pediceG e Glossario\pediceG. Inoltre, ci si è concentrati sull'aggiornamento del sito web con la suddivisione in CANDIDATURA\pediceG, RTB\pediceG e DIAPOSITIVE.

\newpage
\subsubsection{Sprint\pediceG 2}

\textbf{Inizio:} 24/11/2025 \\
\textbf{Fine prevista:} 07/12/2025 \\
\textbf{Fine reale:} 07/12/2025 \\
\textbf{Giorni di ritardo:} 0

\noindent\textbf{Informazioni generali e attività da svolgere}\\
Questo secondo sprint\pediceG ha l'obiettivo principale di continuare la stesura delle Norme di Progetto\pediceG e dell'Analisi dei requisiti\pediceG.\\

In particolare, le attività svolte sono:
\begin{itemize}
    \item Prima redazione del Piano di Progetto\pediceG
    \item Incontro con l'azienda proponente\pediceG Eggon per discutere i requisiti e i casi d'uso\pediceG
    \item Redazione del verbale interno\pediceG del 01/12/2025
    \item Redazione del verbale esterno\pediceG del 03/12/2025
    \item Studio delle tecnologie necessarie
    \item Continuazione della redazione del Glossario\pediceG
    \item Continuazione della redazione dell'Analisi dei requisiti\pediceG
    \item Continuazione della redazione delle Norme di Progetto\pediceG
\end{itemize}

\noindent\textbf{Rischi attesi}\\
I componenti del team BugBusters ritengono possibili i seguenti rischi\pediceG:
\begin{itemize}
    \item RT1: Rischio Tecnologico\pediceG legato alla tecnologia utilizzata
    \item RI1: Rischio Individuale\pediceG derivante dalle altre attività universitarie
\end{itemize}

\noindent\textbf{Preventivo\pediceG}\\
Si prospetta l'utilizzo delle seguenti risorse:

\begin{table}[H]
\centering
\scriptsize
\begin{tabular}{|l|c|c|c|c|c|c|}
\hline
 & \rotatebox{45}{Responsabile\pediceG} & \rotatebox{45}{Amministratore\pediceG} & \rotatebox{45}{Analista\pediceG} & \rotatebox{45}{Progettista\pediceG} & \rotatebox{45}{Programmatore\pediceG} & \rotatebox{45}{Verificatore\pediceG} \\
\hline
Alberto Autiero & - & - & 4 & - & - & - \\
\hline
Marco Favero & 4 & - & - & - & - & 3 \\
\hline
Alberto Pignat & - & 7 & - & - & - & - \\
\hline
Marco Piro & - & - & 8 & - & - & \\
\hline
Linor Sadè & - & - & - & - & - & 12 \\
\hline
Leonardo Salviato & - & - & 4 & - & - & 2 \\
\hline
Luca Slongo & - & - & 4 & -- & - & - \\
\hline
\end{tabular}
\caption{Preventivo\pediceG per componenti nello Sprint\pediceG 2}
\label{tab:preventivo-sprint2}
\end{table}

\begin{figure}[H]
\centering
\begin{tikzpicture}[scale=1.3]
  % Pie chart - Responsabile\pediceG: 8.33% (30.0°)
  \draw[fill=blue!60, draw=black, line width=1.5pt] 
    (0,0) -- (2,0) arc (0:30.0:2) -- cycle;
  \node at (15.0:1.2) {\small\bfseries 8.3\%};
  
  % Amministratore\pediceG: 14.58% (52.5° cumulative)
  \draw[fill=red!60, draw=black, line width=1.5pt] 
    (0,0) -- (30.0:2) arc (30.0:82.5:2) -- cycle;
  \node[rotate=56.25] at (56.25:1.5) {\small\bfseries 14.6\%};
  
  % Analista\pediceG: 41.67% (150.0° cumulative)
  \draw[fill=orange!60, draw=black, line width=1.5pt] 
    (0,0) -- (82.5:2) arc (82.5:232.5:2) -- cycle;
  \node at (157.5:1.2) {\small\bfseries 41.7\%};
  
  % Verificatore\pediceG: 35.42% (127.5° cumulative to 360°)
  \draw[fill=green!60, draw=black, line width=1.5pt] 
    (0,0) -- (232.5:2) arc (232.5:360:2) -- cycle;
  \node at (296.25:1.2) {\small\bfseries 35.4\%};
  
  % Legenda con spaziatura uniforme
  \draw[fill=blue!60, draw=black] (3.5, 2) rectangle (3.8, 2.3);
  \node[anchor=west, font=\small] at (4, 2.15) {Responsabile\pediceG};
  
  \draw[fill=red!60, draw=black] (3.5, 1.5) rectangle (3.8, 1.8);
  \node[anchor=west, font=\small] at (4, 1.65) {Amministratore\pediceG};
  
  \draw[fill=orange!60, draw=black] (3.5, 1) rectangle (3.8, 1.3);
  \node[anchor=west, font=\small] at (4, 1.15) {Analista\pediceG};
  
  \draw[fill=green!60, draw=black] (3.5, 0.5) rectangle (3.8, 0.8);
  \node[anchor=west, font=\small] at (4, 0.65) {Verificatore\pediceG};
\end{tikzpicture}
\caption{Grafico 2: Sprint\pediceG 2 - Preventivo\pediceG}
\label{fig:grafico-sprint2}
\end{figure}

\noindent\textbf{Consuntivo\pediceG}

\begin{table}[H]
\centering
\scriptsize
\begin{tabular}{|l|c|c|c|c|c|c|}
\hline
 & \rotatebox{45}{Responsabile\pediceG} & \rotatebox{45}{Amministratore\pediceG} & \rotatebox{45}{Analista\pediceG} & \rotatebox{45}{Progettista\pediceG} & \rotatebox{45}{Programmatore\pediceG} & \rotatebox{45}{Verificatore\pediceG} \\
\hline
Alberto Autiero & - & - & 4 & - & - & - \\
\hline
Marco Favero & 4 & - & - & - & - & 3 \\
\hline
Alberto Pignat & - & 7 & - & - & - & - \\
\hline
Marco Piro & - & - & 8 & - & - & \\
\hline
Linor Sadè & - & - & - & - & - & 12 \\
\hline
Leonardo Salviato & - & - & 3 \textcolor{green!50!black}{(-1)} & - & - & 2 \\
\hline
Luca Slongo & - & - & 4 & - & - & - \\
\hline
\end{tabular}
\caption{Consuntivo\pediceG per componenti nello Sprint\pediceG 2}
\label{tab:consuntivo-sprint2}
\end{table}

\begin{figure}[H]
\centering
\begin{tikzpicture}[scale=1.3]
  % Responsabile: 8.51% (30.6°)
  \draw[fill=blue!60, draw=black, line width=1.5pt] 
    (0,0) -- (2,0) arc (0:30.6:2) -- cycle;
  \node at (15.3:1.2) {\small\bfseries 8.5\%};
  
  % Amministratore: 14.89% (53.6° cumulative)
  \draw[fill=red!60, draw=black, line width=1.5pt] 
    (0,0) -- (30.6:2) arc (30.6:84.2:2) -- cycle;
  \node[rotate=57.4] at (57.4:1.5) {\small\bfseries 14.9\%};
  
  % Analista: 40.43% (145.6° cumulative)
  \draw[fill=orange!60, draw=black, line width=1.5pt] 
    (0,0) -- (84.2:2) arc (84.2:229.8:2) -- cycle;
  \node at (157.0:1.2) {\small\bfseries 40.4\%};
  
  % Verificatore: 36.17% (130.2° cumulative to 360°)
  \draw[fill=green!60, draw=black, line width=1.5pt] 
    (0,0) -- (229.8:2) arc (229.8:360:2) -- cycle;
  \node at (294.9:1.2) {\small\bfseries 36.2\%};
  
  % Legenda con spaziatura uniforme
  \draw[fill=blue!60, draw=black] (3.5, 2) rectangle (3.8, 2.3);
  \node[anchor=west, font=\small] at (4, 2.15) {Responsabile\pediceG};
  
  \draw[fill=red!60, draw=black] (3.5, 1.5) rectangle (3.8, 1.8);
  \node[anchor=west, font=\small] at (4, 1.65) {Amministratore\pediceG};
  
  \draw[fill=orange!60, draw=black] (3.5, 1) rectangle (3.8, 1.3);
  \node[anchor=west, font=\small] at (4, 1.15) {Analista\pediceG};
  
  \draw[fill=green!60, draw=black] (3.5, 0.5) rectangle (3.8, 0.8);
  \node[anchor=west, font=\small] at (4, 0.65) {Verificatore\pediceG};
\end{tikzpicture}
\caption{Grafico Consuntivo: Sprint\pediceG 2}
\label{fig:grafico-consuntivo-sprint2}
\end{figure}

\noindent\textbf{Aggiornamento delle risorse rimanenti}

\begin{table}[H]
\centering
\footnotesize
\begin{tabular}{|l|c|c|c|c|c|}
\hline
Ruolo & Costo & Ore & Costo & Ore rimanenti & Budget rimanente \\
\hline
Responsabile\pediceG & 30€/h & 4 & 120€ & 44 & 1320€ \\
\hline
Amministratore\pediceG & 20€/h & 7 & 140€ & 37 & 740€ \\
\hline
Analista\pediceG & 25€/h & 19 & 475€ & 14 & 350€ \\
\hline
Progettista\pediceG & 25€/h & - & - & 147 & 3675€ \\
\hline
Programmatore\pediceG & 15€/h & - & - & 167 & 2505€ \\
\hline
Verificatore\pediceG & 15€/h & 17 & 255€ & 139 & 2085€ \\
\hline
\textbf{Totale} & - & \textbf{47} & \textbf{990€} & \textbf{548} & \textbf{10675€} \\
\hline
\end{tabular}
\caption{Variazione nelle risorse disponibili nello Sprint\pediceG 2}
\label{tab:risorse-sprint2}
\end{table}

\noindent\textbf{Rischi incontrati}\\
Durante questo secondo sprint\pediceG si è concretizzato il rischio RI1: Rischio Individuale\pediceG derivante dalle altre attività universitarie, a causa dell'inizio del progetto\pediceG di Tecnologie WEB.\\

\noindent\textbf{Retrospettiva\pediceG}\\
Nel secondo sprint\pediceG abbiamo concentrato gli sforzi soprattutto sull'Analisi dei requisiti\pediceG, individuata come priorità immediata e fondamentale per le successive fasi di progetto\pediceG e sviluppo. L'obiettivo raggiunto è stato quello di portarla ad uno stadio soddisfacente.


\newpage
\subsubsection{Sprint\pediceG 3}

\textbf{Inizio:} 08/12/2025 \\
\textbf{Fine prevista:} 21/12/2025 \\
\textbf{Fine reale:} 21/12/2025 \\
\textbf{Giorni di ritardo:} 0

\noindent\textbf{Informazioni generali e attività da svolgere}\\
Questo terzo sprint\pediceG ha l'obiettivo principale, oltre a quello di continuare la stesura delle Norme di Progetto\pediceG e dell'Analisi dei requisiti\pediceG, anche quello di iniziare il Proof of Concept (POC)\pediceG.

In particolare, le attività svolte sono:
\begin{itemize}
    \item Continuazione della redazione del Piano di Progetto\pediceG
    \item Continuazione della redazione dell'Analisi dei requisiti\pediceG
    \item Continuazione della redazione delle Norme di Progetto\pediceG
    \item Riorganizzazione strutturale e contenutistica del Glossario\pediceG e sua continuazione
    \item Prima redazione del Piano di Qualifica\pediceG
    \item Redazione del verbale interno\pediceG del 11/12/2025
    \item Redazione del verbale esterno\pediceG del 17/12/2025
    \item Aggiornamento del sito web
    \item Inizio dello sviluppo del Proof of Concept (POC)\pediceG
\end{itemize}

\noindent\textbf{Rischi attesi}\\
I componenti del team BugBusters ritengono possibili i seguenti rischi\pediceG:
\begin{itemize}
    \item RT1: Rischio Tecnologico\pediceG legato alla tecnologia utilizzata
    \item RI1: Rischio Individuale\pediceG derivante dalle altre attività universitarie
\end{itemize}

\noindent\textbf{Preventivo\pediceG}\\
Si prospetta l'utilizzo delle seguenti risorse:

\begin{table}[H]
\centering
\scriptsize
\begin{tabular}{|l|c|c|c|c|c|c|}
\hline
 & \rotatebox{45}{Responsabile\pediceG} & \rotatebox{45}{Amministratore\pediceG} & \rotatebox{45}{Analista\pediceG} & \rotatebox{45}{Progettista\pediceG} & \rotatebox{45}{Programmatore\pediceG} & \rotatebox{45}{Verificatore\pediceG} \\
\hline
Alberto Autiero & - & - & - & - & - & 12 \\
\hline
Marco Favero & - & - & 4 & - & - & - \\
\hline
Alberto Pignat & 4 & - & - & - & - & - \\
\hline
Marco Piro & - & 4 & - & - & - & - \\
\hline
Linor Sadè & - & - & - & - & 6 & - \\
\hline
Leonardo Salviato & - & - & - & - & - & 10 \\
\hline
Luca Slongo & - & - & - & - & - & 12 \\
\hline
\end{tabular}
\caption{Preventivo\pediceG per componenti nello Sprint\pediceG 3}
\label{tab:preventivo-sprint3}
\end{table}

\begin{figure}[H]
\centering
\begin{tikzpicture}[scale=1.3]
  % Responsabile: 7.69% (27.7°)
  \draw[fill=blue!60, draw=black, line width=1.5pt] 
    (0,0) -- (2,0) arc (0:27.7:2) -- cycle;
  \node at (13.85:1.2) {\small\bfseries 7.7\%};
  
  % Amministratore: 7.69% (27.7° cumulative)
  \draw[fill=red!60, draw=black, line width=1.5pt] 
    (0,0) -- (27.7:2) arc (27.7:55.4:2) -- cycle;
  \node[rotate=41.55] at (41.55:1.5) {\small\bfseries 7.7\%};
  
  % Analista: 7.69% (27.7° cumulative)
  \draw[fill=orange!60, draw=black, line width=1.5pt] 
    (0,0) -- (55.4:2) arc (55.4:83.1:2) -- cycle;
  \node[rotate=69.25] at (69.25:1.5) {\small\bfseries 7.7\%};
  
  % Programmatore: 11.54% (41.5° cumulative)
  \draw[fill=yellow!60, draw=black, line width=1.5pt] 
    (0,0) -- (83.1:2) arc (83.1:124.6:2) -- cycle;
  \node[rotate=103.85] at (103.85:1.5) {\small\bfseries 11.5\%};
  
  % Verificatore: 65.38% (235.4° cumulative to 360°)
  \draw[fill=green!60, draw=black, line width=1.5pt] 
    (0,0) -- (124.6:2) arc (124.6:360:2) -- cycle;
  \node at (242.3:1.2) {\small\bfseries 65.4\%};
  
  % Legenda
  \draw[fill=blue!60, draw=black] (3.5, 2.5) rectangle (3.8, 2.8);
  \node[anchor=west, font=\small] at (4, 2.65) {Responsabile\pediceG};
  
  \draw[fill=red!60, draw=black] (3.5, 2.0) rectangle (3.8, 2.3);
  \node[anchor=west, font=\small] at (4, 2.15) {Amministratore\pediceG};
  
  \draw[fill=orange!60, draw=black] (3.5, 1.5) rectangle (3.8, 1.8);
  \node[anchor=west, font=\small] at (4, 1.65) {Analista\pediceG};

  \draw[fill=yellow!60, draw=black] (3.5, 1.0) rectangle (3.8, 1.3);
  \node[anchor=west, font=\small] at (4, 1.15) {Programmatore\pediceG};
  
  \draw[fill=green!60, draw=black] (3.5, 0.5) rectangle (3.8, 0.8);
  \node[anchor=west, font=\small] at (4, 0.65) {Verificatore\pediceG};
\end{tikzpicture}
\caption{Grafico 3: Sprint\pediceG 3 - Preventivo\pediceG}
\label{fig:grafico-sprint3}
\end{figure}

\noindent\textbf{Consuntivo\pediceG}

\begin{table}[H]
\centering
\scriptsize
\begin{tabular}{|l|c|c|c|c|c|c|}
\hline
 & \rotatebox{45}{Responsabile\pediceG} & \rotatebox{45}{Amministratore\pediceG} & \rotatebox{45}{Analista\pediceG} & \rotatebox{45}{Progettista\pediceG} & \rotatebox{45}{Programmatore\pediceG} & \rotatebox{45}{Verificatore\pediceG} \\
\hline
Alberto Autiero & - & - & - & - & - & 12 \\
\hline
Marco Favero & - & - & 5 \textcolor{red}{(+1)} & - & - & - \\
\hline
Alberto Pignat & 4 & - & - & - & - & - \\
\hline
Marco Piro & - & 4 & - & - & - & - \\
\hline
Linor Sadè & - & - & - & - & 6 & - \\
\hline
Leonardo Salviato & - & - & - & - & - & 10 \\
\hline
Luca Slongo & - & - & - & - & - & 11 \textcolor{green!50!black}{(-1)} \\
\hline
\end{tabular}
\caption{Consuntivo\pediceG per componenti nello Sprint\pediceG 3}
\label{tab:consuntivo-sprint3}
\end{table}

\begin{figure}[H]
\centering
\begin{tikzpicture}[scale=1.3]
  % Responsabile: 7.69% (27.7°)
  \draw[fill=blue!60, draw=black, line width=1.5pt] 
    (0,0) -- (2,0) arc (0:27.7:2) -- cycle;
  \node at (13.85:1.2) {\small\bfseries 7.7\%};
  
  % Amministratore: 7.69% (27.7° cumulative)
  \draw[fill=red!60, draw=black, line width=1.5pt] 
    (0,0) -- (27.7:2) arc (27.7:55.4:2) -- cycle;
  \node[rotate=41.55] at (41.55:1.5) {\small\bfseries 7.7\%};
  
  % Analista: 9.62% (34.6° cumulative)
  \draw[fill=orange!60, draw=black, line width=1.5pt] 
    (0,0) -- (55.4:2) arc (55.4:90.0:2) -- cycle;
  \node[rotate=72.7] at (72.7:1.5) {\small\bfseries 9.6\%};
  
  % Programmatore: 11.54% (41.5° cumulative)
  \draw[fill=yellow!60, draw=black, line width=1.5pt] 
    (0,0) -- (90.0:2) arc (90.0:131.5:2) -- cycle;
  \node[rotate=110.75] at (110.75:1.5) {\small\bfseries 11.5\%};
  
  % Verificatore: 63.46% (228.5° cumulative to 360°)
  \draw[fill=green!60, draw=black, line width=1.5pt] 
    (0,0) -- (131.5:2) arc (131.5:360:2) -- cycle;
  \node at (245.75:1.2) {\small\bfseries 63.5\%};
  
  % Legenda
  \draw[fill=blue!60, draw=black] (3.5, 2.5) rectangle (3.8, 2.8);
  \node[anchor=west, font=\small] at (4, 2.65) {Responsabile\pediceG};
  
  \draw[fill=red!60, draw=black] (3.5, 2.0) rectangle (3.8, 2.3);
  \node[anchor=west, font=\small] at (4, 2.15) {Amministratore\pediceG};
  
  \draw[fill=orange!60, draw=black] (3.5, 1.5) rectangle (3.8, 1.8);
  \node[anchor=west, font=\small] at (4, 1.65) {Analista\pediceG};

  \draw[fill=yellow!60, draw=black] (3.5, 1.0) rectangle (3.8, 1.3);
  \node[anchor=west, font=\small] at (4, 1.15) {Programmatore\pediceG};
  
  \draw[fill=green!60, draw=black] (3.5, 0.5) rectangle (3.8, 0.8);
  \node[anchor=west, font=\small] at (4, 0.65) {Verificatore\pediceG};
\end{tikzpicture}
\caption{Grafico Consuntivo: Sprint 3}
\label{fig:grafico-consuntivo-sprint3}
\end{figure}

\noindent\textbf{Aggiornamento delle risorse rimanenti}

\begin{table}[H]
\centering
\footnotesize
\begin{tabular}{|l|c|c|c|c|c|}
\hline
Ruolo & Costo & Ore & Costo & Ore rimanenti & Budget rimanente \\
\hline
Responsabile\pediceG & 30€/h & 4 & 120€ & 40 & 1200€ \\
\hline
Amministratore\pediceG & 20€/h & 4 & 80€ & 33 & 660€ \\
\hline
Analista\pediceG & 25€/h & 5 & 125€ & 9 & 225€ \\
\hline
Progettista\pediceG & 25€/h & - & - & 147 & 3675€ \\
\hline
Programmatore\pediceG & 15€/h & 6 & 90€ & 161 & 2415€ \\
\hline
Verificatore\pediceG & 15€/h & 33 & 495€ & 106 & 1590€ \\
\hline
\textbf{Totale} & - & \textbf{52} & \textbf{910€} & \textbf{496} & \textbf{9765€} \\
\hline
\end{tabular}
\caption{Variazione nelle risorse disponibili nello Sprint 3}
\label{tab:risorse-sprint3}
\end{table}

\noindent\textbf{Rischi incontrati}\\
Durante questo terzo sprint\pediceG si è concretizzato il rischio RI1: Rischio Individuale\pediceG derivante dalle altre attività universitarie, a causa della prosecuzione del progetto\pediceG di Tecnologie WEB e dell'inizio dello studio per gli esami della sessione invernale.\\

\noindent\textbf{Retrospettiva\pediceG}\\
Nel terzo sprint\pediceG abbiamo concentrato gli sforzi soprattutto sull'Analisi dei requisiti\pediceG, che è ormai in uno stato molto avanzato. Inoltre, abbiamo iniziato lo sviluppo del Proof of Concept (POC)\pediceG.

\newpage
\subsubsection{Sprint\pediceG 4}

\textbf{Inizio:} 22/12/2025 \\
\textbf{Fine prevista:} 04/01/2026\\
\textbf{Fine reale:} 04/01/2026\\
\textbf{Giorni di ritardo:} 0

\noindent\textbf{Informazioni generali e attività da svolgere}\\
L'obiettivo principale di questo quarto sprint\pediceG è la continuazione dell'Analisi dei Requisiti\pediceG e la continuazione delle Norme di Progetto\pediceG. Parallelamente, proseguono le attività sul Piano di Qualifica\pediceG e sul Piano di Progetto\pediceG, insieme all'avanzamento del Proof of Concept (POC)\pediceG.

In particolare, le attività svolte sono:
\begin{itemize}
    \item Continuazione della redazione del Piano di Progetto\pediceG
    \item Continuazione della redazione del Piano di Qualifica\pediceG
    \item Continuazione della redazione dell'Analisi dei requisiti\pediceG
    \item Continuazione della redazione delle Norme di Progetto\pediceG
    \item Continuazione della redazione del Glossario\pediceG
    \item Continuazione dello sviluppo del Proof of Concept (POC)\pediceG
    \item Prima redazione della guida POC\pediceG Nexum
\end{itemize}

\noindent\textbf{Rischi attesi}\\
I componenti del team BugBusters ritengono possibili i seguenti rischi\pediceG:
\begin{itemize}
    \item RT1: Rischio Tecnologico\pediceG legato alla tecnologia utilizzata
    \item RI1: Rischio Individuale\pediceG derivante dalle altre attività universitarie
\end{itemize}

\noindent\textbf{Preventivo\pediceG}\\
Si prospetta l'utilizzo delle seguenti risorse:

\begin{table}[H]
\centering
\scriptsize
\begin{tabular}{|l|c|c|c|c|c|c|}
\hline
 & \rotatebox{45}{Responsabile\pediceG} & \rotatebox{45}{Amministratore\pediceG} & \rotatebox{45}{Analista\pediceG} & \rotatebox{45}{Progettista\pediceG} & \rotatebox{45}{Programmatore\pediceG} & \rotatebox{45}{Verificatore\pediceG} \\
\hline
Alberto Autiero & - & - & 4 & - & - & - \\
\hline
Marco Favero & - & - & 4 & - & - & 3 \\
\hline
Alberto Pignat & - & - & - & - & 6 & - \\
\hline
Marco Piro & 8 & - & - & - & - & - \\
\hline
Linor Sadè & - & 8 & - & - & - & - \\
\hline
Leonardo Salviato & - & - & - & - & 5 & - \\
\hline
Luca Slongo & - & - & - & - & - & 8 \\
\hline
\end{tabular}
\caption{Preventivo\pediceG per componenti nello Sprint\pediceG 4}
\label{tab:preventivo-sprint4}
\end{table}

\begin{figure}[H]
\centering
\begin{tikzpicture}[scale=1.3]
  % Responsabile: 17.39% (62.6°)
  \draw[fill=blue!60, draw=black, line width=1.5pt] 
    (0,0) -- (2,0) arc (0:62.6:2) -- cycle;
  \node at (31.3:1.2) {\small\bfseries 17.4\%};
  
  % Amministratore: 17.39% (62.6° cumulative)
  \draw[fill=red!60, draw=black, line width=1.5pt] 
    (0,0) -- (62.6:2) arc (62.6:125.2:2) -- cycle;
  \node[rotate=93.9] at (93.9:1.5) {\small\bfseries 17.4\%};
  
  % Analista: 17.39% (62.6° cumulative)
  \draw[fill=orange!60, draw=black, line width=1.5pt] 
    (0,0) -- (125.2:2) arc (125.2:187.8:2) -- cycle;
  \node[rotate=156.5] at (156.5:1.5) {\small\bfseries 17.4\%};
  
  % Programmatore: 23.91% (86.1° cumulative)
  \draw[fill=yellow!60, draw=black, line width=1.5pt] 
    (0,0) -- (187.8:2) arc (187.8:273.9:2) -- cycle;
  \node at (230.85:1.2) {\small\bfseries 23.9\%};
  
  % Verificatore: 23.91% (86.1° cumulative to 360°)
  \draw[fill=green!60, draw=black, line width=1.5pt] 
    (0,0) -- (273.9:2) arc (273.9:360:2) -- cycle;
  \node at (316.95:1.2) {\small\bfseries 23.9\%};
  
  % Legenda
  \draw[fill=blue!60, draw=black] (3.5, 2.5) rectangle (3.8, 2.8);
  \node[anchor=west, font=\small] at (4, 2.65) {Responsabile\pediceG};
  
  \draw[fill=red!60, draw=black] (3.5, 2.0) rectangle (3.8, 2.3);
  \node[anchor=west, font=\small] at (4, 2.15) {Amministratore\pediceG};
  
  \draw[fill=orange!60, draw=black] (3.5, 1.5) rectangle (3.8, 1.8);
  \node[anchor=west, font=\small] at (4, 1.65) {Analista\pediceG};

  \draw[fill=yellow!60, draw=black] (3.5, 1.0) rectangle (3.8, 1.3);
  \node[anchor=west, font=\small] at (4, 1.15) {Programmatore\pediceG};
  
  \draw[fill=green!60, draw=black] (3.5, 0.5) rectangle (3.8, 0.8);
  \node[anchor=west, font=\small] at (4, 0.65) {Verificatore\pediceG};
\end{tikzpicture}
\caption{Grafico 4: Sprint\pediceG 4 - Preventivo\pediceG}
\label{fig:grafico-sprint4}
\end{figure}

\noindent\textbf{Consuntivo\pediceG}

\begin{table}[H]
\centering
\scriptsize
\begin{tabular}{|l|c|c|c|c|c|c|}
\hline
 & \rotatebox{45}{Responsabile\pediceG} & \rotatebox{45}{Amministratore\pediceG} & \rotatebox{45}{Analista\pediceG} & \rotatebox{45}{Progettista\pediceG} & \rotatebox{45}{Programmatore\pediceG} & \rotatebox{45}{Verificatore\pediceG} \\
\hline
Alberto Autiero & - & - & 4 & - & - & - \\
\hline
Marco Favero & - & - & 3 \textcolor{green!50!black}{(-1)} & - & - & 3 \\
\hline
Alberto Pignat & - & - & - & - & 6 & - \\
\hline
Marco Piro & 8 & - & - & - & - & - \\
\hline
Linor Sadè & - & 8 & - & - & - & - \\
\hline
Leonardo Salviato & - & - & - & - & 5 & - \\
\hline
Luca Slongo & - & - & - & - & - & 8 \\
\hline
\end{tabular}
\caption{Consuntivo\pediceG per componenti nello Sprint\pediceG 4}
\label{tab:consuntivo-sprint4}
\end{table}

\begin{figure}[H]
\centering
\begin{tikzpicture}[scale=1.3]
  % Responsabile: 17.78% (64.0°)
  \draw[fill=blue!60, draw=black, line width=1.5pt] 
    (0,0) -- (2,0) arc (0:64.0:2) -- cycle;
  \node at (32.0:1.2) {\small\bfseries 17.8\%};
  
  % Amministratore: 17.78% (64.0° cumulative)
  \draw[fill=red!60, draw=black, line width=1.5pt] 
    (0,0) -- (64.0:2) arc (64.0:128.0:2) -- cycle;
  \node[rotate=96.0] at (96.0:1.5) {\small\bfseries 17.8\%};
  
  % Analista: 15.56% (56.0° cumulative)
  \draw[fill=orange!60, draw=black, line width=1.5pt] 
    (0,0) -- (128.0:2) arc (128.0:184.0:2) -- cycle;
  \node[rotate=156.0] at (156.0:1.5) {\small\bfseries 15.6\%};
  
  % Programmatore: 24.44% (88.0° cumulative)
  \draw[fill=yellow!60, draw=black, line width=1.5pt] 
    (0,0) -- (184.0:2) arc (184.0:272.0:2) -- cycle;
  \node at (228.0:1.2) {\small\bfseries 24.4\%};
  
  % Verificatore: 24.44% (88.0° cumulative to 360°)
  \draw[fill=green!60, draw=black, line width=1.5pt] 
    (0,0) -- (272.0:2) arc (272.0:360:2) -- cycle;
  \node at (316.0:1.2) {\small\bfseries 24.4\%};
  
  % Legenda
  \draw[fill=blue!60, draw=black] (3.5, 2.5) rectangle (3.8, 2.8);
  \node[anchor=west, font=\small] at (4, 2.65) {Responsabile\pediceG};
  
  \draw[fill=red!60, draw=black] (3.5, 2.0) rectangle (3.8, 2.3);
  \node[anchor=west, font=\small] at (4, 2.15) {Amministratore\pediceG};
  
  \draw[fill=orange!60, draw=black] (3.5, 1.5) rectangle (3.8, 1.8);
  \node[anchor=west, font=\small] at (4, 1.65) {Analista\pediceG};

  \draw[fill=yellow!60, draw=black] (3.5, 1.0) rectangle (3.8, 1.3);
  \node[anchor=west, font=\small] at (4, 1.15) {Programmatore\pediceG};
  
  \draw[fill=green!60, draw=black] (3.5, 0.5) rectangle (3.8, 0.8);
  \node[anchor=west, font=\small] at (4, 0.65) {Verificatore\pediceG};
\end{tikzpicture}
\caption{Grafico Consuntivo: Sprint 4}
\label{fig:grafico-consuntivo-sprint4}
\end{figure}


\noindent\textbf{Aggiornamento delle risorse rimanenti}

\begin{table}[H]
\centering
\footnotesize
\begin{tabular}{|l|c|c|c|c|c|}
\hline
Ruolo & Costo & Ore & Costo & Ore rimanenti & Budget rimanente \\
\hline
Responsabile\pediceG & 30€/h & 8 & 240€ & 32 & 960€ \\
\hline
Amministratore\pediceG & 20€/h & 8 & 160€ & 25 & 500€ \\
\hline
Analista\pediceG & 25€/h & 7 & 175€ & 2 & 50€ \\
\hline
Progettista\pediceG & 25€/h & - & - & 147 & 3675€ \\
\hline
Programmatore\pediceG & 15€/h & 11 & 165€ & 150 & 2250€ \\
\hline
Verificatore\pediceG & 15€/h & 11 & 165€ & 95 & 1425€ \\
\hline
\textbf{Totale} & - & \textbf{45} & \textbf{905€} & \textbf{451} & \textbf{8860€} \\
\hline
\end{tabular}
\caption{Variazione nelle risorse disponibili nello Sprint 4}
\label{tab:risorse-sprint4}
\end{table}

\noindent\textbf{Rischi incontrati}\\
Durante questo sesto sprint\pediceG si è concretizzato il rischio RI1: Rischio Individuale\pediceG derivante dalle altre attività universitarie, a causa della prosecuzione del progetto\pediceG di Tecnologie WEB e del continuo studio per gli esami della sessione invernale.\\

\noindent\textbf{Retrospettiva\pediceG}\\
Nel quarto sprint\pediceG abbiamo concentrato gli sforzi soprattutto sull'Analisi dei requisiti\pediceG e sulle Norme di Progetto\pediceG, che sono ormai concluse. Inoltre, abbiamo dedicato del tempo anche alla continuazione dello sviluppo del Proof of Concept (POC)\pediceG, che è ormai a uno stato avanzato.

\newpage
\subsubsection{Sprint\pediceG 5}

\textbf{Inizio:} 05/01/2026 \\
\textbf{Fine prevista:} 18/01/2026 \\
\textbf{Fine reale:} 18/01/2026 \\
\textbf{Giorni di ritardo:} 0

\noindent\textbf{Informazioni generali e attività da svolgere}\\
L'obiettivo principale di questo quinto sprint\pediceG è la riorganizzazione dell'Analisi dei Requisiti\pediceG a seguito del colloquio con il Prof. Cardin e delle Norme di Progetto\pediceG. Parallelamente, proseguono le attività relative al Piano di Qualifica\pediceG e al Piano di Progetto\pediceG, nonché l'avanzamento del Proof of Concept (POC)\pediceG fino al completamento dello sviluppo.

In particolare, le attività svolte sono:
\begin{itemize}
    \item Continuazione della redazione del Piano di Progetto\pediceG
    \item Continuazione della redazione del Piano di Qualifica\pediceG
    \item Riorganizzazione dell'Analisi dei requisiti\pediceG a seguito del colloquio con il Prof. Cardin e successiva conclusione del documento
    \item Continuazione della redazione delle Norme di Progetto\pediceG
    \item Continuazione della redazione del Glossario\pediceG
    \item Conclusione dello sviluppo del Proof of Concept (POC)\pediceG
    \item Redazione del verbale interno\pediceG del 05/01/2026
    \item Redazione del verbale esterno\pediceG del 07/01/2026
    \item Redazione del verbale interno\pediceG del 12/01/2026
    \item Redazione del verbale interno\pediceG del 15/01/2026
\end{itemize}

\noindent\textbf{Rischi attesi}\\
I componenti del team BugBusters ritengono possibili i seguenti rischi\pediceG:
\begin{itemize}
    \item RT1: Rischio Tecnologico\pediceG legato alla tecnologia utilizzata
    \item RI1: Rischio Individuale\pediceG derivante dalle altre attività universitarie
\end{itemize}

\noindent\textbf{Preventivo\pediceG}\\
Si prospetta l'utilizzo delle seguenti risorse:

\begin{table}[H]
\centering
\scriptsize
\begin{tabular}{|l|c|c|c|c|c|c|}
\hline
 & \rotatebox{45}{Responsabile\pediceG} & \rotatebox{45}{Amministratore\pediceG} & \rotatebox{45}{Analista\pediceG} & \rotatebox{45}{Progettista\pediceG} & \rotatebox{45}{Programmatore\pediceG} & \rotatebox{45}{Verificatore\pediceG} \\
\hline
Alberto Autiero & - & - & - & - & 6 & - \\
\hline
Marco Favero & - & - & - & - & - & 6 \\
\hline
Alberto Pignat & - & - & - & - & - & 6 \\
\hline
Marco Piro & - & - & - & - & - & 12 \\
\hline
Linor Sadè & 4 & - & - & - & - & - \\
\hline
Leonardo Salviato & - & 4 & - & - & - & - \\
\hline
Luca Slongo & - & - & - & - & 6 & - \\
\hline
\end{tabular}
\caption{Preventivo\pediceG per componenti nello Sprint\pediceG 5}
\label{tab:preventivo-sprint5}
\end{table}

\begin{figure}[H]
\centering
\begin{tikzpicture}[scale=1.3]
  % Responsabile: 9.09% (32.7°)
  \draw[fill=blue!60, draw=black, line width=1.5pt] 
    (0,0) -- (2,0) arc (0:32.7:2) -- cycle;
  \node at (16.35:1.2) {\small\bfseries 9.1\%};
  
  % Amministratore: 9.09% (32.7° cumulative)
  \draw[fill=red!60, draw=black, line width=1.5pt] 
    (0,0) -- (32.7:2) arc (32.7:65.4:2) -- cycle;
  \node[rotate=49.05] at (49.05:1.5) {\small\bfseries 9.1\%};
  
  % Programmatore: 27.27% (98.2° cumulative)
  \draw[fill=yellow!60, draw=black, line width=1.5pt] 
    (0,0) -- (65.4:2) arc (65.4:163.6:2) -- cycle;
  \node at (114.5:1.2) {\small\bfseries 27.3\%};
  
  % Verificatore: 54.55% (196.4° cumulative to 360°)
  \draw[fill=green!60, draw=black, line width=1.5pt] 
    (0,0) -- (163.6:2) arc (163.6:360:2) -- cycle;
  \node at (261.8:1.2) {\small\bfseries 54.5\%};
  
  % Legenda
  \draw[fill=blue!60, draw=black] (3.5, 2) rectangle (3.8, 2.3);
  \node[anchor=west, font=\small] at (4, 2.15) {Responsabile\pediceG};
  
  \draw[fill=red!60, draw=black] (3.5, 1.5) rectangle (3.8, 1.8);
  \node[anchor=west, font=\small] at (4, 1.65) {Amministratore\pediceG};
  
  \draw[fill=yellow!60, draw=black] (3.5, 1) rectangle (3.8, 1.3);
  \node[anchor=west, font=\small] at (4, 1.15) {Programmatore\pediceG};
  
  \draw[fill=green!60, draw=black] (3.5, 0.5) rectangle (3.8, 0.8);
  \node[anchor=west, font=\small] at (4, 0.65) {Verificatore\pediceG};
\end{tikzpicture}
\caption{Grafico 5: Sprint\pediceG 5 - Preventivo\pediceG}
\label{fig:grafico-sprint5}
\end{figure}

\noindent\textbf{Consuntivo\pediceG}

\begin{table}[H]
\centering
\scriptsize
\begin{tabular}{|l|c|c|c|c|c|c|}
\hline
 & \rotatebox{45}{Responsabile\pediceG} & \rotatebox{45}{Amministratore\pediceG} & \rotatebox{45}{Analista\pediceG} & \rotatebox{45}{Progettista\pediceG} & \rotatebox{45}{Programmatore\pediceG} & \rotatebox{45}{Verificatore\pediceG} \\
\hline
Alberto Autiero & - & - & - & - & 6 & - \\
\hline
Marco Favero & - & - & - & - & - & 5 \textcolor{green!50!black}{(-1)} \\
\hline
Alberto Pignat & - & - & - & - & - & 6 \\
\hline
Marco Piro & - & - & - & - & - & 13 \textcolor{red}{(+1)} \\
\hline
Linor Sadè & 4 & - & - & - & - & - \\
\hline
Leonardo Salviato & - & 4 & 2 & - & - & - \\
\hline
Luca Slongo & - & - & - & - & 6 & - \\
\hline
\end{tabular}
\caption{Consuntivo\pediceG per componenti nello Sprint\pediceG 5}
\label{tab:consuntivo-sprint5}
\end{table}

\begin{figure}[H]
\centering
\begin{tikzpicture}[scale=1.3]
  % Responsabile: 8.70% = 4h su 46h totali -> 31.3°
  \draw[fill=blue!60, draw=black, line width=1.5pt] 
    (0,0) -- (2,0) arc (0:31.3:2) -- cycle;
  \node at (15.65:1.2) {\small\bfseries 8.7\%};
  
  % Amministratore: 8.70% = 4h -> 31.3° (cumulativo: 31.3° -> 62.6°)
  \draw[fill=red!60, draw=black, line width=1.5pt] 
    (0,0) -- (31.3:2) arc (31.3:62.6:2) -- cycle;
  \node[rotate=46.95] at (46.95:1.5) {\small\bfseries 8.7\%};
  
  % Analista: 4.35% = 2h -> 15.7° (cumulativo: 62.6° -> 78.3°)
  \draw[fill=orange!60, draw=black, line width=1.5pt] 
    (0,0) -- (62.6:2) arc (62.6:78.3:2) -- cycle;
  \node[rotate=70.45] at (70.45:1.5) {\small\bfseries 4.4\%};
  
  % Programmatore: 26.09% = 12h -> 93.9° (cumulativo: 78.3° -> 172.2°)
  \draw[fill=yellow!60, draw=black, line width=1.5pt] 
    (0,0) -- (78.3:2) arc (78.3:172.2:2) -- cycle;
  \node at (125.25:1.2) {\small\bfseries 26.1\%};
  
  % Verificatore: 52.17% = 24h -> 187.8° (cumulativo: 172.2° -> 360°)
  \draw[fill=green!60, draw=black, line width=1.5pt] 
    (0,0) -- (172.2:2) arc (172.2:360:2) -- cycle;
  \node at (266.1:1.2) {\small\bfseries 52.2\%};
  
  % Legenda
  \draw[fill=blue!60, draw=black] (3.5, 2.5) rectangle (3.8, 2.8);
  \node[anchor=west, font=\small] at (4, 2.65) {Responsabile\pediceG};
  
  \draw[fill=red!60, draw=black] (3.5, 2.0) rectangle (3.8, 2.3);
  \node[anchor=west, font=\small] at (4, 2.15) {Amministratore\pediceG};
  
  \draw[fill=orange!60, draw=black] (3.5, 1.5) rectangle (3.8, 1.8);
  \node[anchor=west, font=\small] at (4, 1.65) {Analista\pediceG};

  \draw[fill=yellow!60, draw=black] (3.5, 1.0) rectangle (3.8, 1.3);
  \node[anchor=west, font=\small] at (4, 1.15) {Programmatore\pediceG};
  
  \draw[fill=green!60, draw=black] (3.5, 0.5) rectangle (3.8, 0.8);
  \node[anchor=west, font=\small] at (4, 0.65) {Verificatore\pediceG};
\end{tikzpicture}
\caption{Grafico Consuntivo: Sprint 5}
\label{fig:grafico-consuntivo-sprint5}
\end{figure}

\noindent\textbf{Aggiornamento delle risorse rimanenti}

\begin{table}[H]
\centering
\footnotesize
\begin{tabular}{|l|c|c|c|c|c|}
\hline
Ruolo & Costo & Ore & Costo & Ore rimanenti & Budget rimanente \\
\hline
Responsabile\pediceG & 30€/h & 4 & 120€ & 28 & 840€ \\
\hline
Amministratore\pediceG & 20€/h & 4 & 80€ & 21 & 420€ \\
\hline
Analista\pediceG & 25€/h & 2 & 50€ & 0 & 0€ \\
\hline
Progettista\pediceG & 25€/h & - & - & 147 & 3675€ \\
\hline
Programmatore\pediceG & 15€/h & 12 & 180€ & 138 & 2070€ \\
\hline
Verificatore\pediceG & 15€/h & 24 & 360€ & 71 & 1065€ \\
\hline
\textbf{Totale} & - & \textbf{46} & \textbf{790€} & \textbf{405} & \textbf{8070€} \\
\hline
\end{tabular}
\caption{Variazione nelle risorse disponibili nello Sprint 5}
\label{tab:risorse-sprint5}
\end{table}

\noindent\textbf{Rischi incontrati}\\
Durante questo quinto sprint si sono concretizzati i seguenti rischi:

\begin{itemize}
    \item \textbf{RI1 - Rischio Individuale\pediceG derivante dalle altre attività universitarie}: la prosecuzione del progetto\pediceG di Tecnologie Web e il continuo impegno nello studio per gli esami della sessione invernale hanno leggermente rallentato la prosecuzione del progetto\pediceG.
    
    \item \textbf{RG3 - Rischio Globale\pediceG derivante dalla sottostima delle attività}: per quanto riguarda l'Analisi dei Requisiti\pediceG, il gruppo ha sottoposto il documento a controllo del Prof. Cardin quando questo si trovava già in uno stato molto avanzato. Tale decisione ha reso necessaria una riorganizzazione sostanziale del documento per correggere le criticità segnalate dal docente.
\end{itemize}

\noindent\textbf{Retrospettiva\pediceG}\\
Nel corso del quinto sprint\pediceG, il gruppo ha concentrato gli sforzi sulla riorganizzazione dell'Analisi dei Requisiti\pediceG in seguito al colloquio con il Prof. Cardin. Inoltre, è stato completato lo sviluppo del Proof of Concept (POC)\pediceG.

\newpage
\subsubsection{Sprint\pediceG 6}

\textbf{Inizio:} 19/01/2026 \\
\textbf{Fine prevista:} 01/02/2026 \\
\textbf{Fine reale:} 01/02/2026 \\
\textbf{Giorni di ritardo:} 0

\noindent\textbf{Informazioni generali e attività da svolgere}\\
Questo sprint\pediceG si pone come finalità il completamento delle attività propedeutiche alla candidatura\pediceG per la Requirements and Technology Baseline (RTB)\pediceG. Gli obiettivi specifici comprendono la finalizzazione della documentazione richiesta, ovvero Piano di Progetto\pediceG, Piano di Qualifica\pediceG, Norme di Progetto\pediceG e Glossario\pediceG. Tutte le attività sopra indicate costituiscono i prerequisiti necessari per la presentazione della candidatura RTB\pediceG.


In particolare, le attività svolte sono:
\begin{itemize}
    \item Conclusione della redazione del Piano di Progetto\pediceG
    \item Conclusione della redazione del Piano di Qualifica\pediceG
    \item Conclusione della redazione delle Norme di Progetto\pediceG
    \item Conclusione della redazione del Glossario\pediceG
    \item Conclusione redazione della guida POC\pediceG Nexum
    \item Stesura della lettera di candidatura\pediceG
\end{itemize}

\noindent\textbf{Rischi attesi}\\
I componenti del team BugBusters ritengono possibili i seguenti rischi\pediceG:
\begin{itemize}
    \item RT1: Rischio Tecnologico\pediceG legato alla tecnologia utilizzata
    \item RI1: Rischio Individuale\pediceG derivante dalle altre attività universitarie
\end{itemize}

\noindent\textbf{Preventivo\pediceG}\\
Si prospetta l'utilizzo delle seguenti risorse:

\begin{table}[H]
\centering
\scriptsize
\begin{tabular}{|l|c|c|c|c|c|c|}
\hline
 & \rotatebox{45}{Responsabile\pediceG} & \rotatebox{45}{Amministratore\pediceG} & \rotatebox{45}{Analista\pediceG} & \rotatebox{45}{Progettista\pediceG} & \rotatebox{45}{Programmatore\pediceG} & \rotatebox{45}{Verificatore\pediceG} \\
\hline
Alberto Autiero & - & - & - & - & 5 & - \\
\hline
Marco Favero & - & - & - & - & 8 & - \\
\hline
Alberto Pignat & - & - & - & - & 6 & 6 \\
\hline
Marco Piro & - & - & - & - & 6 & - \\
\hline
Linor Sadè & - & - & - & - & 6 & - \\
\hline
Leonardo Salviato & 8 & - & - & - & - & - \\
\hline
Luca Slongo & - & 4 & - & - & - & 4 \\
\hline
\end{tabular}
\caption{Preventivo\pediceG per componenti nello Sprint\pediceG 6}
\label{tab:preventivo-sprint6}
\end{table}

\begin{figure}[H]
\centering
\begin{tikzpicture}[scale=1.3]
  % Responsabile: 15.09% (54.3°)
  \draw[fill=blue!60, draw=black, line width=1.5pt] 
    (0,0) -- (2,0) arc (0:54.3:2) -- cycle;
  \node at (27.15:1.2) {\small\bfseries 15.1\%};
  
  % Amministratore: 7.55% (27.2° cumulative)
  \draw[fill=red!60, draw=black, line width=1.5pt] 
    (0,0) -- (54.3:2) arc (54.3:81.5:2) -- cycle;
  \node[rotate=67.9] at (67.9:1.5) {\small\bfseries 7.6\%};
  
  % Programmatore: 58.49% (210.6° cumulative)
  \draw[fill=yellow!60, draw=black, line width=1.5pt] 
    (0,0) -- (81.5:2) arc (81.5:292.1:2) -- cycle;
  \node at (186.8:1.2) {\small\bfseries 58.5\%};
  
  % Verificatore: 18.87% (67.9° cumulative to 360°)
  \draw[fill=green!60, draw=black, line width=1.5pt] 
    (0,0) -- (292.1:2) arc (292.1:360:2) -- cycle;
  \node at (326.05:1.2) {\small\bfseries 18.9\%};
  
  % Legenda
  \draw[fill=blue!60, draw=black] (3.5, 2) rectangle (3.8, 2.3);
  \node[anchor=west, font=\small] at (4, 2.15) {Responsabile\pediceG};
  
  \draw[fill=red!60, draw=black] (3.5, 1.5) rectangle (3.8, 1.8);
  \node[anchor=west, font=\small] at (4, 1.65) {Amministratore\pediceG};
  
  \draw[fill=yellow!60, draw=black] (3.5, 1) rectangle (3.8, 1.3);
  \node[anchor=west, font=\small] at (4, 1.15) {Programmatore\pediceG};
  
  \draw[fill=green!60, draw=black] (3.5, 0.5) rectangle (3.8, 0.8);
  \node[anchor=west, font=\small] at (4, 0.65) {Verificatore\pediceG};
\end{tikzpicture}
\caption{Grafico 6: Sprint\pediceG 6 - Preventivo\pediceG}
\label{fig:grafico-sprint6}
\end{figure}

\noindent\textbf{Consuntivo\pediceG}

\begin{table}[H]
\centering
\scriptsize
\begin{tabular}{|l|c|c|c|c|c|c|}
\hline
 & \rotatebox{45}{Responsabile\pediceG} & \rotatebox{45}{Amministratore\pediceG} & \rotatebox{45}{Analista\pediceG} & \rotatebox{45}{Progettista\pediceG} & \rotatebox{45}{Programmatore\pediceG} & \rotatebox{45}{Verificatore\pediceG} \\
\hline
Alberto Autiero & - & - & - & - & 5 & - \\
\hline
Marco Favero & - & - & - & - & 8 & - \\
\hline
Alberto Pignat & - & - & - & - & 6 & 6 \\
\hline
Marco Piro & - & - & - & - & 6 & - \\
\hline
Linor Sadè & - & - & - & - & 6 & - \\
\hline
Leonardo Salviato & 8 & - & - & - & - & - \\
\hline
Luca Slongo & - & 4 & - & - & - & 4 \\
\hline
\end{tabular}
\caption{Consuntivo\pediceG per componenti nello Sprint\pediceG 6}
\label{tab:consuntivo-sprint6}
\end{table}

\begin{figure}[H]
\centering
\begin{tikzpicture}[scale=1.3]
  % Responsabile: 15.09% (54.3°)
  \draw[fill=blue!60, draw=black, line width=1.5pt] 
    (0,0) -- (2,0) arc (0:54.3:2) -- cycle;
  \node at (27.15:1.2) {\small\bfseries 15.1\%};
  
  % Amministratore: 7.55% (27.2° cumulative)
  \draw[fill=red!60, draw=black, line width=1.5pt] 
    (0,0) -- (54.3:2) arc (54.3:81.5:2) -- cycle;
  \node[rotate=67.9] at (67.9:1.5) {\small\bfseries 7.6\%};
  
  % Programmatore: 58.49% (210.6° cumulative)
  \draw[fill=yellow!60, draw=black, line width=1.5pt] 
    (0,0) -- (81.5:2) arc (81.5:292.1:2) -- cycle;
  \node at (186.8:1.2) {\small\bfseries 58.5\%};
  
  % Verificatore: 18.87% (67.9° cumulative to 360°)
  \draw[fill=green!60, draw=black, line width=1.5pt] 
    (0,0) -- (292.1:2) arc (292.1:360:2) -- cycle;
  \node at (326.05:1.2) {\small\bfseries 18.9\%};
  
  % Legenda
  \draw[fill=blue!60, draw=black] (3.5, 2) rectangle (3.8, 2.3);
  \node[anchor=west, font=\small] at (4, 2.15) {Responsabile\pediceG};
  
  \draw[fill=red!60, draw=black] (3.5, 1.5) rectangle (3.8, 1.8);
  \node[anchor=west, font=\small] at (4, 1.65) {Amministratore\pediceG};
  
  \draw[fill=yellow!60, draw=black] (3.5, 1) rectangle (3.8, 1.3);
  \node[anchor=west, font=\small] at (4, 1.15) {Programmatore\pediceG};
  
  \draw[fill=green!60, draw=black] (3.5, 0.5) rectangle (3.8, 0.8);
  \node[anchor=west, font=\small] at (4, 0.65) {Verificatore\pediceG};
\end{tikzpicture}
\caption{Grafico Consuntivo: Sprint 6}
\label{fig:grafico-consuntivo-sprint6}
\end{figure}


\noindent\textbf{Aggiornamento delle risorse rimanenti}

\begin{table}[H]
\centering
\footnotesize
\begin{tabular}{|l|c|c|c|c|c|}
\hline
Ruolo & Costo & Ore & Costo & Ore rimanenti & Budget rimanente \\
\hline
Responsabile\pediceG & 30€/h & 8 & 240€ & 20 & 600€ \\
\hline
Amministratore\pediceG & 20€/h & 4 & 80€ & 17 & 340€ \\
\hline
Analista\pediceG & 25€/h & - & - & 0 & 0€ \\
\hline
Progettista\pediceG & 25€/h & - & - & 147 & 3675€ \\
\hline
Programmatore\pediceG & 15€/h & 31 & 465€ & 107 & 1605€ \\
\hline
Verificatore\pediceG & 15€/h & 10 & 150€ & 61 & 915€ \\
\hline
\textbf{Totale} & - & \textbf{53} & \textbf{935€} & \textbf{352} & \textbf{7135€} \\
\hline
\end{tabular}
\caption{Variazione nelle risorse disponibili nello Sprint 6}
\label{tab:risorse-sprint6}
\end{table}

\noindent\textbf{Rischi incontrati}\\
Durante questo sesto sprint\pediceG si è concretizzato il rischio RI1: Rischio Individuale\pediceG derivante dalle altre attività universitarie, a causa della prosecuzione del progetto\pediceG di Tecnologie WEB e del continuo studio per gli esami della sessione invernale.\\

\noindent\textbf{Retrospettiva\pediceG}\\
Durante il sesto sprint\pediceG sono stati completati tutti i documenti propedeutici alla candidatura\pediceG alla Requirements and Technology Baseline (RTB)\pediceG.

\vfill
\begin{center}
    {\small\color{darkgray} Documento redatto e approvato dal gruppo BugBusters.}
\end{center}

\end{document}