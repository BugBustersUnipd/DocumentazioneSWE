\documentclass[a4paper,11pt]{article}

\usepackage[utf8]{inputenc}
\usepackage[T1]{fontenc}
\usepackage[italian]{babel}
\usepackage[margin=2.5cm]{geometry}
\usepackage{graphicx}
\usepackage{booktabs}
\usepackage{setspace}
\usepackage{titlesec}
\usepackage{float}
\usepackage[table]{xcolor}
\usepackage{tabularx}
\usepackage{tcolorbox}
\usepackage{enumitem}
\usepackage[titles]{tocloft}
\usepackage[colorlinks=true,linkcolor=black,urlcolor=blue,citecolor=blue]{hyperref}
\usepackage{fancyhdr}
\usepackage{lastpage}
\usepackage{amsmath}

\pagestyle{fancy}
\fancyhf{}
\fancyhead[L]{BugBusters}
\fancyhead[R]{Norme di Progetto}
\fancyfoot[L]{\thepage\ di \pageref{LastPage}}
\renewcommand{\headrulewidth}{0pt}
\renewcommand{\footrulewidth}{0pt}

\setlength{\headheight}{14pt}

\setlength{\parskip}{4pt}
\setlength{\parindent}{0pt}

\titleformat{\section}{\large\bfseries}{\thesection}{1em}{}
\titleformat{\subsection}{\normalsize\bfseries}{\thesubsection}{1em}{}

\begin{document}

\begin{center}
  \thispagestyle{empty}
  \IfFileExists{../assets/Logo.jpg}{%
    \includegraphics[width=6cm,height=3cm,keepaspectratio]{../assets/Logo.jpg} \\[0.8cm]
  }{%
    \fbox{\parbox[c][2.5cm][c]{6cm}{\centering Logo non trovato\\(Logo.jpg)}}\\[0.5cm]
  }
  {\LARGE\bfseries BugBusters}\\[0.8cm]
  
  \rule{\textwidth}{0.5pt}\\[0.5cm]
  {\Large\bfseries Norme di Progetto}\\[0.3cm]
  {\large Versione 0.1.0}\\[0.5cm]
  \rule{\textwidth}{0.5pt}\\[0.8cm]
\end{center}

\begin{center}
\begin{tcolorbox}[colback=gray!10,width=0.8\textwidth,arc=3mm,boxrule=0.5pt]
\begin{tabular}{ll}
\textbf{Stato} & XXX \\
\textbf{Responsabile} & XXX \\
\textbf{Verificatore} & XXX \\
\textbf{Redattori} & Alberto Autiero \\
\textbf{Distribuzione} & BugBusters \\
 & Prof. Vardanega Tullio \\
 & Prof. Cardin Riccardo \\
\end{tabular}
\end{tcolorbox}
\end{center}

\vspace{1cm}

\begin{center}
\textbf{Descrizione}
\end{center}

\begin{center}
\begin{minipage}{0.9\textwidth}
\small
Questo documento contiene le Norme di Progetto seguite da il gruppo \textbf{BugBusters} per il progetto C9 proposto dall'azienda Vimar
\end{minipage}
\end{center}

\newpage

\section*{Registro delle Modifiche}

{\footnotesize
\begin{center}
\begin{tabular}{|l|l|l|l|l|}
\hline
\textbf{Vers.} & \textbf{Data} & \textbf{Descrizione} & \textbf{Autore} & \textbf{Ruolo} \\
\hline
0.1.0 & 23/10/2025 & Prima stesura della struttura del documento & Alberto Autiero & - \\
\hline
\end{tabular}
\end{center}
}

\vfill
\begin{center}
2 di \pageref{LastPage}
\end{center}

\newpage

\section*{Indice}

\noindent
\begin{minipage}[t]{0.8\textwidth}
\subsection*{1 Introduzione}
1.1 Scopo del documento \\
1.2 Scopo del prodotto \\
1.3 Glossario \\
1.4 Riferimenti \\
\quad 1.4.1 Riferimenti normativi \\
\quad 1.4.2 Riferimenti informativi \\
\end{minipage}
\begin{minipage}[t]{0.2\textwidth}
\vspace{1.65\baselineskip}
9 \\
9 \\
9 \\
10 \\
10 \\
10 \\
\end{minipage}

\newpage

\section{Introduzione}

\subsection{Scopo del documento}
Questo documento nasce per descrivere il Way of Working adottato da parte di BugBusters durante lo svolgimento del progetto didattico.

\subsection{Scopo del prodotto}
XXXXXXXXXXXXXXXXXXXXXX

L'obiettivo che si è posto questo gruppo è realizzare questo progetto entro il 31 Marzo 2025 con un budget a disposizione di 12930€

\subsection{Glossario}
Prima di iniziare a scrivere codice, è importante analizzare e progettare il software. Per lavorare meglio insieme e in modo organizzato, tutte le informazioni utili vengono documentate.

Per evitare confusioni o incomprensioni sui termini tecnici usati nei documenti, il gruppo utilizza un Glossario. In questo documento vengono raccolte e spiegate in modo chiaro tutte le parole importanti del progetto.

Il gruppo BugBusters si impegna a tenere sempre aggiornato il Glossario e a consultarlo regolarmente, per assicurarsi che tutti i documenti siano facili da capire.

\newpage

\subsection{Riferimenti}

\subsubsection{Riferimenti normativi}
\begin{itemize}
\item \textbf{Capitolato d'appalto C9: Sistema di gestione di un magazzino distribuito}\\
\url{https://www.math.unipd.it/~tullio/IS-1/2025/Progetto/C9.pdf}
\end{itemize}

\subsubsection{Riferimenti informativi}
\begin{itemize}
\item \textbf{Glossario:}\\
\url{https://github.com/BugBustersUnipd/DocumentazioneSWE/blob/main/GLOSSARIO/Glossario.pdf}
\end{itemize}

\end{document}