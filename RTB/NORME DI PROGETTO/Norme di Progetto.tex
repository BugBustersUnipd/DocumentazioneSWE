\documentclass[a4paper,11pt]{article}
   
\newcommand{\CurrentVersion}{0.0.8} % ultima versione, da cambiare ad ogni push significativo

\usepackage[utf8]{inputenc}
\usepackage[T1]{fontenc}
\usepackage[italian]{babel}
\usepackage[margin=2.5cm]{geometry}
\usepackage{graphicx}
\usepackage{grffile}
\usepackage{booktabs}
\usepackage{setspace}
\usepackage{titlesec}
\usepackage{float}
\usepackage{ifthen}
\usepackage[table]{xcolor}
\usepackage{tabularx}
\usepackage{tcolorbox}
\usepackage{enumitem}
\usepackage[titles]{tocloft}
\usepackage[colorlinks=true,linkcolor=black,urlcolor=blue,citecolor=blue]{hyperref}

\definecolor{primaryblue}{RGB}{0,102,204}
\definecolor{secondaryblue}{RGB}{51,153,255}
\definecolor{lightgray}{RGB}{245,245,245}
\definecolor{darkgray}{RGB}{100,100,100}

\titleformat{\section}
 {\Large\bfseries\color{primaryblue}}
 {\thesection}{1em}{}

\titleformat{\subsection}
 {\large\bfseries\color{primaryblue}} % Sottosezione: colore secondaryblue
 {\thesubsection}{1em}{}

\titleformat{\subsubsection}
 {\normalsize\bfseries\color{secondaryblue}} % Sotto-sottosezione: colore secondaryblue
 {\thesubsubsection}{1em}{}

% per il footer con il numero di pagina
\usepackage{fancyhdr}
\usepackage{lastpage} % per ottenere il numero dell'ultima pagina da mettere nel footer


\usepackage{ltablex} %per far andare a capo le tabelle
\keepXColumns

\renewcommand{\sectionmark}[1]{\markright{#1}}
\newcommand{\G}{\textsubscript{\scalebox{0.6}{\textbf{G}}}}



\setlength{\parskip}{4pt}
\setlength{\parindent}{0pt}

\setlist[itemize]{leftmargin=*,itemsep=3pt}
\setlist[enumerate]{leftmargin=*,itemsep=3pt}

\graphicspath{{./}{../assets/images/}{./images/}}

\begin{document}

%configurazione per il footer
\pagestyle{fancy}
\fancyhf{} % pulisce tutti i campi del header e footer

% Header: sinistra e destra
\fancyhead[L]{Gruppo 4 - BugBusters} % sinistra
\fancyhead[R]{Norme di Progetto}   % destra

\fancyfoot[L]{ \thepage\ di \pageref{LastPage}} %definisce il formato del footer
\fancyfoot[R]{ \nouppercase{\rightmark}} % nome della sezione


\renewcommand{\headrulewidth}{0pt} % rimuove la linea dell'header
\renewcommand{\footrulewidth}{0pt} % se vuoi anche togliere eventuale linea del footer

% abilitare numerazione e TOC fino al livello "paragraph" (subsubsubsection)
\setcounter{secnumdepth}{4}
\setcounter{tocdepth}{4}
% formattazione del nuovo livello per avere aspetto coerente
\titleformat{\paragraph}[block]{\normalsize\bfseries\color{secondaryblue}}{\theparagraph}{1em}{}
% alias comodo per usare "subsubsubsection"
\newcommand{\subsubsubsection}{\paragraph}

\begin{center}
 \thispagestyle{empty}
 \IfFileExists{../../assets/Logo.jpg}{%
  \includegraphics[width=6cm,height=3cm,keepaspectratio]{../../assets/Logo.jpg} \\[0.8cm]
 }{%
  \fbox{\parbox[c][2.5cm][c]{6cm}{\centering Logo non trovato\\(Logo.jpg)}}\\[0.5cm]
 }
 {\Large\bfseries BugBusters}\\[0.3cm]
 {\small\color{darkgray} Email: \texttt{bugbusters.unipd@gmail.com}} \\[0.1cm]
 {\small\color{darkgray} Gruppo: 4} \\[0.5cm]

 {\large\bfseries Università degli Studi di Padova}\\[0.3cm]
 {\small Laurea in Informatica}\\[0.2cm]
 {\small Corso: Ingegneria del Software}\\[0.2cm]
 {\small Anno Accademico: 2025/2026}\\[0.8cm]

 {\Huge\bfseries\color{primaryblue} Norme di Progetto}\\[0.8cm]
 {\Large\color{secondaryblue}Versione \CurrentVersion}\\[0.8cm]
\end{center}

\begin{center}
\begin{tcolorbox}[colback=lightgray,colframe=primaryblue,width=0.85\textwidth,arc=3mm,boxrule=0.5pt]
\begin{tabularx}{\linewidth}{@{}lX@{}}
\textbf{Destinatari} & BugBusters, Prof. Tullio Vardanega, Prof. Riccardo Cardin \\
\textbf{Data ultima modifica}     & 17/11/2025 \\
\\
\end{tabularx}
\end{tcolorbox}
\end{center}

\vspace{0.5cm}

\begin{center}
\begin{tcolorbox}[colback=secondaryblue!10,colframe=secondaryblue,width=0.9\textwidth,arc=3mm,boxrule=0.8pt,title={\bfseries Abstract}]
Documento contenente le norme di progetto adottate dal team BugBusters per lo sviluppo del progetto Nexum proposto dall'azienda Eggon. Il documento include metodologie di lavoro, standard di codifica, processi di sviluppo e gestione del progetto.
\end{tcolorbox}
\end{center}

\newpage

\section*{Registro delle modifiche}

\setlength{\extrarowheight}{2pt} % padding extra verticale
\renewcommand{\arraystretch}{1.5} 

\arrayrulecolor{primaryblue}
{\footnotesize
\begin{tabularx}{\textwidth}{|>{\raggedright\arraybackslash}p{1.5cm}|>{\raggedright\arraybackslash}p{2cm}|X|>{\raggedright\arraybackslash}p{2cm}|>{\raggedright\arraybackslash}p{2cm}|>{\raggedright\arraybackslash}p{2cm}|}
\hline
\rowcolor{primaryblue!40}
\textbf{\color{white} Versione} & \textbf{\color{white} Data} & \textbf{\color{white} Descrizione} & \textbf{\color{white} Redatto} & \textbf{\color{white} Verificato} & \textbf{\color{white} Approvato} \\
\hline
\rowcolor{secondaryblue!10} 0.0.8 & 04/12/2025 & Conclusione momentanea della sezione relativa ai processi di supporto & Alberto Pignat & - & - \\
\hline
\rowcolor{secondaryblue!10} 0.0.7 & 02/12/2025 & Avanzamento processi di supporto: Configurazioni e Qualifica & Alberto Pignat & - & - \\
\hline
\rowcolor{secondaryblue!10} 0.0.6 & 28/11/2025 & Iniziati processi di supporto: documentazione & Marco Favero & - & - \\
\hline
\rowcolor{secondaryblue!10} 0.0.5 & 27/11/2025 & Correzione strutturale e sistemazione lavoro svolto & Marco Favero & - & - \\
\hline
\rowcolor{secondaryblue!10} 0.0.4 & 20/11/2025 & Finita la parte di Documentazione prodotta in processi di fornitura & Marco Favero & - & - \\
\hline
\rowcolor{secondaryblue!10} 0.0.3 & 19/11/2025 & Avanzamento, parte di documentazione prodotta & Marco Favero & - & - \\
\hline
\rowcolor{secondaryblue!10} 0.0.2 & 16/11/2025 & Introduzione, processi di acquisizione e iniziati processi di fornitura & Marco Favero & - & - \\
\hline
\rowcolor{secondaryblue!10} 0.0.1 & 04/11/2025 & 
\begin{minipage}[t]{\linewidth}
Prima stesura della struttura del documento dopo l'aggiudicazione dell'appalto per il capitolato C5 Nexum dell'azienda Eggon
\end{minipage} 
& Alberto Autiero & - & - \\
\hline
\end{tabularx}
}


\newpage
\tableofcontents

\newpage

\section{Introduzione}

\subsection{Scopo del documento}
Questo documento definisce le norme di progetto adottate dal team BugBusters per lo sviluppo del progetto Nexum, 
proposto dall'azienda Eggon. Le norme di progetto includono metodologie di lavoro, standard di codifica, 
processi di sviluppo e gestione del progetto, al fine di garantire un approccio strutturato e coerente 
durante l'intero ciclo di vita del progetto.

Per strutturare il nostro way of working\G{}, faremo riferimento alle best practice\G{} suggerite dallo standard ISO/IEC 12207:1995
adattandole alle esigenze specifiche del nostro team e del progetto Nexum. In particolare identifica tre tipologie di processi:
\begin{itemize}
    \item Processi primari\G{}: processi direttamente coinvolti nella creazione del prodotto software
    \item Processi di supporto\G{}: processi che supportano i processi primari\G{}
    \item Processi organizzativi\G{}: processi che gestiscono e coordinano le attività del team
\end{itemize}
La combinazione di questi processi ci permetterà di gestire in modo efficace lo sviluppo del progetto Nexum. 
La stesura di questo documento mira a fornire una guida chiara e condivisa per tutti i membri del team, 
assicurando che le attività di sviluppo siano svolte in modo efficiente e conforme agli standard di qualità 
previsti. Il documento è redatto in maniera incrementale: verrà aggiornato e ampliato progressivamente durante 
lo sviluppo del progetto per riflettere decisioni, modifiche e miglioramenti adottati dal team.


\subsection{Scopo del prodotto}
Nexum è una piattaforma con un modulo dedicato alle comunicazioni interne, alla raccolta di feedback e alla timbratura digitale.
Include una messaggistica top-down con tracciamento delle letture, un builder per survey con logiche di ramificazione e dashboard 
in tempo reale; inoltre un sistema di timbratura via badge o dispositivi mobili con regole automatiche per il controllo e l'aggregazione delle ore.
Le anagrafiche centralizzate, con ruoli e permessi, permettono una gestione granulare degli accessi e l'integrazione dei moduli, 
garantendo un flusso informativo coerente, tracciabile e adattabile alle esigenze operative.
Il nostro progetto mira a estendere la piattaforma Nexum con un AI assistant generativo per la scrittura di comunicazioni e con un AI Co-Pilot\G{} per
i Cdl. In particolare quest'ultimo deve essere in grado di riconoscere i documenti caricati dagli utenti, estrarne le informazioni rilevanti, capire
la tipologia, destinatari e consegnarli in modo massivo. \\
Il nostro obiettivo é realizzare questo progetto entro il 21 marzo 2026 con un budget di 12.790 euro.


\subsection{Glossario}
Il glossario raccoglie e definisce i termini, gli acronimi e le abbreviazioni 
impiegati nel documento e nel progetto Nexum. L'obiettivo è fornire definizioni univoche per ridurre 
ambiguità, garantire coerenza terminologica tra i membri del team e facilitare l'onboarding di nuovi partecipanti. \\
Per i termini tecnici e specifici utilizzati in questo documento, si fa riferimento al glossario disponibile al 
seguente \href{https://bugbustersunipd.github.io/DocumentazioneSWE/RTB/GLOSSARIO/Glossario.pdf}{link}.
Per maggiore usabilitá e facilitá di consultazione, il glossario è accessibile anche aprendo dal nostro sito web i vari documenti con il viewer
pdf da noi sviluppato.

\subsubsection{Riferimenti normativi}
\begin{itemize}
\item \textbf{Capitolato\textsubscript{\scalebox{0.6}{\textbf{G}}}
 d'appalto C5: Nexum - Piattaforma di consulenza e documentazione previdenziale}\\
\url{https://www.math.unipd.it/~tullio/IS-1/2025/Progetto/C5.pdf}
\end{itemize}

\subsubsection{Riferimenti informativi}
\begin{itemize}
    \item \textbf{Glossario\G{}:}\\
\url{https://github.com/BugBustersUnipd/DocumentazioneSWE/blob/main//RTB/GLOSSARIO/Glossario.pdf}
\end{itemize}

\newpage

\section{Processi Primari}
L'obiettivo principale di BugBusters è la realizzazione di un prodotto software di alta qualità che soddisfi 
le esigenze del cliente e degli utenti finali. Per raggiungere questo obiettivo, è indispensabile basarsi 
su un modello di riferimento che definisca processi chiari da seguire. \\
L'adozione di un simile framework metodologico, che indirizza ad esempio le fasi di acquisizione e di costruzione 
del software, è ciò che permette di passare da un semplice programma funzionante a un prodotto di valore e di 
lunga durata.
Basandosi dunque sullo standard ISO/IEC 12207:1995, il team BugBusters ha deciso di adottare i seguenti processi 
primari:
\begin{itemize}
    \item Processi di acquisizione
    \item Processi di fornitura
    \item Processi di sviluppo
\end{itemize}

\subsection{Processi di acquisizione}
BOH NON SO CHE DIRE AAEHTENGEGAEFANCIPETGEA


\subsection{Processi di fornitura}
\subsubsection{Scopo}
Lo scopo del processo di fornitura è definire e regolamentare l'insieme delle attività preliminari 
necessarie ad avviare il progetto in modo controllato, condiviso e verificabile. In particolare, 
tale processo ha l'obiettivo di chiarire i requisiti richiesti dal proponente, identificare vincoli e risorse, 
pianificare le modalità operative, stabilire gli strumenti di lavoro e formalizzare la documentazione necessaria, 
così da garantire una corretta base metodologica e contrattuale per le successive fasi di sviluppo.

\subsubsection{Attivitá}
La fornitura prevede varie attitá, in particolare: 
\begin{itemize}
    \item \textbf{Analisi del capitolato proposto}: raccolta di informazioni, requisiti, tecnologie e vincoli presenti nel 
    capitolato d'appalto. In questa fase si pongono domande al proponente per chiarire eventuali dubbi o ambiguità.
    \item \textbf{Stima delle risorse}: definizione delle tempistiche, dei costi totali e delle risorse necessarie per la 
    realizzazione del progetto.
    \item \textbf{Analisi dei requisiti e contrattazione con il proponente}: identificazione e documentazione dei requisiti 
    funzionali e non funzionali del software, analisi delle aspettative del proponente e negoziazione di eventuali modifiche o aggiunte. Definizione del Minimum Viable Product (MVP).
    \item \textbf{Pianificazione del progetto}: suddivisione del progetto in fasi, definizione delle milestone (RTB e PB), 
    assegnazione dei compiti ai membri del team e pianificazione delle attività. Individuazione degli strumenti di lavoro e delle 
    metodologie da adottare; definizione del Proof of Concept (PoC).
    \item \textbf{Documentazione}: redazione di tutti i documenti necessari per formalizzare la fornitura, come il Piano di Progetto, 
    il Analisi dei requisiti, Norme di Progetto e altri documenti di supporto. La documentazione prodotta sarà utilizzata sia come 
    strumento di lavoro interno al team sia come riferimento e strumento di controllo da parte del proponente.
    \item \textbf{Comunicazione con il proponente}: stabilire canali di comunicazione efficaci per garantire un flusso informativo 
    continuo e trasparente. Prevedere incontri regolari per aggiornamenti sullo stato del progetto, discussione di eventuali problemi 
    e raccolta di feedback.
    \item \textbf{Revisione e approvazione}: sottoporre tutta la documentazione prodotta alla revisione e approvazione del proponente, 
    assicurando l’allineamento sugli obiettivi e le aspettative prima di procedere con le fasi successive del progetto.
    \item \textbf{Consegna e chiusura della fase di fornitura}: consegna di quanto prodotto durante la fase di fornitura al proponente, 
    garantendo che tutti i documenti siano completi e corretti. Completata la consegna, si procede alla chiusura formale della fase.
\end{itemize}

\subsubsection{Strumenti di Supporto}
Per supportare le attività di fornitura, il team BugBusters usofruisce di vari strumenti:
\begin{itemize}
    \item \textbf{GitHub}: Per la documentazione collaborativa, il versionamento dei documenti, la produzione asincrona,
     il sistema di ticketing e delle project board.
    \item \textbf{GitLab}: Per il codice già esistente fornito dal proponente e per l'implementazioe delle nuove feature.
    \item \textbf{Discord}: Per le riunioni di team.
    \item \textbf{WhatsApp}: Per comunicazioni rapide e aggiornamenti interni al gruppo.
    \item \textbf{Calendar Google}: Per la pianificazione delle riunioni e delle scadenze.
    \item \textbf{Telegram e Gmail}: Per la comunicazione con il proponente e per inviare documenti ufficiali. 
\end{itemize}

\subsubsection{Comunicazione e Organizzazione}
Le comunicazioni con il proponente avvengono inizialmente 
con cadenza settimanale il mercoledì alle 15:00 e, successivamente, 
a progetto avviato, con cadenza bisettimanale. 
I verbali relativi agli incontri con il proponente 
vengono condivisi e archiviati nel repository della 
documentazione per garantire trasparenza e reperibilità. 
Tutte le decisioni rilevanti emerse negli incontri con il 
proponente devono essere formalmente riportate e documentate 
su GitHub e nella documentazione ufficiale. Eventuali problemi 
critici segnalati dal proponente vengono comunicati immediatamente 
sul canale concordato, in modo da garantirne la tracciabilità e la 
tempestività nelle azioni correttive.

\subsubsection{Documentazione Prodotta}\label{Documentazione Prodotta}
Durante la fase di fornitura, il team BugBusters produce e mantiene aggiornata la seguente documentazione:


\subsubsubsection{Glossario}
Per facilitare la lettura e la comprensione dei documenti di progetto, viene redatto un glossario che raccoglie
e definisce i termini tecnici, gli acronimi e le abbreviazioni utilizzati. Questo strumento è fondamentale per
garantire una comunicazione chiara e univoca tra tutti i membri del team, con il proponente, docenti e con i lettori esterni.
Per una consultazione rapida durante la visualizzazione dei documenti, il glossario è accessibile anche tramite il 
viewer PDF sviluppato dal team.

\begin{center}
\begin{tabularx}{\textwidth}{|>{\centering\arraybackslash}p{3.2cm}|>{\centering\arraybackslash}X|}
\hline
\rowcolor{primaryblue!40}
\multicolumn{1}{|>{\centering\arraybackslash}p{3.2cm}|}{\color{white}\textbf{Campo}} & \multicolumn{1}{>{\centering\arraybackslash}X|}{\color{white}\textbf{Dettaglio}} \\
\hline
\textbf{Redattore} & Amministratore \\
\hline
\textbf{Destinatari} & BugBusters, Eggon, Prof. Vardanega, Prof. Cardin  \\
\hline
\textbf{Uso} & Interno ed Esterno \\
\hline
\end{tabularx}
\end{center}



\subsubsubsection{Dichiarazione degli impegni}
La Dichiarazione degli Impegni definisce i ruoli, il monte ore previso e il costo orario di ciascun membro del team BugBusters
per la realizzazione del progetto Nexum e la data di consegna prevista, impegnandosi a rispettare tali condizioni durante l'intero ciclo di vita del progetto.

\begin{center}
\begin{tabularx}{\textwidth}{|>{\centering\arraybackslash}p{3.2cm}|>{\centering\arraybackslash}X|}
\hline
\rowcolor{primaryblue!40}
\multicolumn{1}{|>{\centering\arraybackslash}p{3.2cm}|}{\color{white}\textbf{Campo}} & \multicolumn{1}{>{\centering\arraybackslash}X|}{\color{white}\textbf{Dettaglio}} \\
\hline
\textbf{Redattore} & Responsabile \\
\hline
\textbf{Destinatari} & BugBusters, Eggon, Prof. Vardanega, Prof. Cardin  \\
\hline
\textbf{Uso} & Esterno \\
\hline
\end{tabularx}
\end{center}

\subsubsubsection{Valutazione dei capitolati}
Con questo documento si intende valutare i vari capitolati d'appalto proposti per il progetto di Ingegneria del Software,
analizzandone punti di forza, debolezze e opportunità in modo da scegliere il capitolato più adatto alle competenze e 
agli interessi del team BugBusters.
Per ogni capitolato vengono esaminati vari aspetti:
\begin{itemize}
    \item Descrizione breve
    \item Caratteristiche funzionali
    \item Tecnologie proposte
    \item Chiarimenti e colloqui con l'azienda
    \item Interesse del team
    \item Punti di forza e debolezza
\end{itemize}

\begin{center}
\begin{tabularx}{\textwidth}{|>{\centering\arraybackslash}p{3.2cm}|>{\centering\arraybackslash}X|}
\hline
\rowcolor{primaryblue!40}
\multicolumn{1}{|>{\centering\arraybackslash}p{3.2cm}|}{\color{white}\textbf{Campo}} & \multicolumn{1}{>{\centering\arraybackslash}X|}{\color{white}\textbf{Dettaglio}} \\
\hline
\textbf{Redattore} & Responsabile \\
\hline
\textbf{Destinatari} & BugBusters, Prof. Vardanega, Prof. Cardin  \\
\hline
\textbf{Uso} & Esterno \\
\hline
\end{tabularx}
\end{center}

\subsubsubsection{Analisi dei Requisiti}
Il documento di Analisi dei Requisiti descrive in dettaglio le funzionalità, i vincoli e le proprietà di qualità che il sistema Nexum dovrà soddisfare. Questo capitolo fornisce:
\begin{itemize}
    \item la definizione del contesto, degli stakeholder e degli attori coinvolti;
    \item l'elenco dei requisiti funzionali e non funzionali, classificati e dotati di codifica univoca e criteri di accettazione;
    \item casi d'uso e scenari principali con flussi e attori associati;
    \item vincoli tecnici, normativi e di integrazione con sistemi esterni;
    \item la prioritizzazione dei requisiti e l'identificazione del Minimum Viable Product (MVP);
    \item la strategia di tracciabilità e gestione delle modifiche (issue tracker, versionamento e mappatura requisiti–test);
    \item i criteri e i metodi per la verifica e la validazione dei requisiti.
\end{itemize}

\begin{center}
\begin{tabularx}{\textwidth}{|>{\centering\arraybackslash}p{3.2cm}|>{\centering\arraybackslash}X|}
\hline
\rowcolor{primaryblue!40}
\multicolumn{1}{|>{\centering\arraybackslash}p{3.2cm}|}{\color{white}\textbf{Campo}} & \multicolumn{1}{>{\centering\arraybackslash}X|}{\color{white}\textbf{Dettaglio}} \\
\hline
\textbf{Redattore} & Analista \\
\hline
\textbf{Destinatari} & Eggon, Prof. Vardanega, Prof. Cardin  \\
\hline
\textbf{Uso} & Esterno \\
\hline
\end{tabularx}
\end{center}

\subsubsubsection{Norme di Progetto}
Il documento delle Norme di Progetto definisce le metodologie, gli standard e i processi che 
il team BugBusters adotterà durante lo sviluppo del progetto Nexum. 

\begin{center}
\begin{tabularx}{\textwidth}{|>{\centering\arraybackslash}p{3.2cm}|>{\centering\arraybackslash}X|}
\hline
\rowcolor{primaryblue!40}
\multicolumn{1}{|>{\centering\arraybackslash}p{3.2cm}|}{\color{white}\textbf{Campo}} & \multicolumn{1}{>{\centering\arraybackslash}X|}{\color{white}\textbf{Dettaglio}} \\
\hline
\textbf{Redattore} & Amministratore \\
\hline
\textbf{Destinatari} & BugBusters, Prof. Vardanega, Prof. Cardin  \\
\hline
\textbf{Uso} & Interno \\
\hline
\end{tabularx}
\end{center}

\subsubsubsection{Lettera di Presentazione}
???

\begin{center}
\begin{tabularx}{\textwidth}{|>{\centering\arraybackslash}p{3.2cm}|>{\centering\arraybackslash}X|}
\hline
\rowcolor{primaryblue!40}
\multicolumn{1}{|>{\centering\arraybackslash}p{3.2cm}|}{\color{white}\textbf{Campo}} & \multicolumn{1}{>{\centering\arraybackslash}X|}{\color{white}\textbf{Dettaglio}} \\
\hline
\textbf{Redattore} & Responsabile \\
\hline
\textbf{Destinatari} & BugBusters, Eggon, Prof. Vardanega, Prof. Cardin  \\
\hline
\textbf{Uso} & Esterno \\
\hline
\end{tabularx}
\end{center}

\subsubsubsection{Verbali}
I verbali sono documenti di lavoro indispensabili per tracciare le decisioni, le discussioni e le 
azioni concordate durante le riunioni del team BugBusters e con il proponente Eggon.
Si dividono in verbali interni, prodotti durante le riunioni del team, e verbali esterni, 
redatti dopo gli incontri con il proponente.

\begin{center}
\begin{tabularx}{\textwidth}{|>{\centering\arraybackslash}p{3.2cm}|>{\centering\arraybackslash}X|>{\centering\arraybackslash}X|}
\hline
\rowcolor{primaryblue!40}
\multicolumn{1}{|>{\centering\arraybackslash}p{3.2cm}|}{\color{white}\textbf{Campo}} & \multicolumn{1}{>{\centering\arraybackslash}X|}{\color{white}\textbf{Dettaglio interni}} & \multicolumn{1}{>{\centering\arraybackslash}X|}{\color{white}\textbf{Dettaglio esterni}} \\
\hline
\textbf{Redattore} & Responsabile & Responsabile \\
\hline
\textbf{Destinatari} & BugBusters, Prof. Vardanega, Prof. Cardin & BugBusters, Eggon, Prof. Vardanega, Prof. Cardin \\
\hline
\textbf{Uso} & Interno & Esterno \\
\hline
\end{tabularx}
\end{center}

\subsubsubsection{Piano di Progetto}
DA FARE
\subsubsubsection{Piano di Qualifica}
DA FARE

\subsection{Processi di sviluppo}
Il processo di sviluppo definisce le attivitá tecniche e operative necessarie per la realizzazione del prodotto software Nexum,
in conformitá ai requisiti stabiliti durante la fase di fornitura. Questo processo include la progettazione, 
l'implementazione, il testing e la documentazione del software, garantendo che ogni fase sia eseguita secondo
standard di qualitá elevati e best practice del settore.

\subsubsubsection{attività previste}
DA FARE


\section{Processi di Supporto}
\subsection{Documentazione}
\subsubsection{Scopo}
Lo scopo del processo di documentazione è definire le linee guida e le metodologie per la creazione, gestione e manutenzione 
della documentazione di progetto. 
Questo processo mira a garantire che tutta la documentazione prodotta sia chiara, coerente, accessibile e aggiornata, 
facilitando la comunicazione tra i membri del team, il proponente e gli stakeholder esterni. Una documentazione ben strutturata supporta
l'efficace gestione del progetto, la tracciabilità delle decisioni e la conformità agli standard di qualità previsti.

\subsubsection{Strumenti Utilizzati}
Per supportare le attività di documentazione, il team BugBusters utilizza i seguenti strumenti:
\begin{itemize}
    \item \textbf{GitHub}: Per la gestione collaborativa della documentazione, il versionamento dei documenti e la produzione asincrona. 
    Vengono utilizzate le funzionalitá di issue tracking e project board per tracciare le attivitá di documentazione. 
    Inoltre BugBusters fa ampio uso dei branch per garantire che le modifiche alla documentazione siano revisionate 
    prima di essere integrate nella versione principale.
    \item \textbf{LaTeX}: Per la stesura di documenti tecnici e formali, garantendo un formato professionale e coerente. É stata stabilita
    un identitá visiva comune per tutti i documenti, inclusi template, stili e convenzioni di formattazione.
    \item \textbf{Viewer PDF sviluppato dal team}: Per facilitare la consultazione della documentazione prodotta.
\end{itemize}

\subsubsection{Documenti Prodotti}
Di seguito l'elenco dei documenti obbligatori, coerenti con le indicazioni del capitolato C4. Ogni documento verrà mantenuto aggiornato e versionato sul repository del progetto.

\begin{enumerate}
    \item \textbf{Norme di Progetto}: Il presente documento: linee guida metodologiche, standard di lavoro e convenzioni adottate dal team.
    \item \textbf{Piano di Progetto}: Cronoprogramma dettagliato e allocazione delle risorse umane e materiali per le milestone e le consegne.
    \item \textbf{Piano di Qualifica}: Strategia e metriche di collaudo; obiettivo minimo di qualità: copertura dei test pari o superiore al 80\%.
    \item \textbf{Analisi dei Requisiti}: Specifica dei requisiti funzionali e non funzionali, casi d'uso principali e criteri di accettazione.
    \item \textbf{Glossario}: Elenco e definizioni dei termini tecnici e degli acronimi usati nel progetto.
    \item \textbf{Lettera di Presentazione}: Documento di presentazione formale rivolto a RTB, completo e firmato, da utilizzare per la consegna iniziale.
    \item \textbf{Verbali (interni/esterni)}: Registrazione degli incontri e delle decisioni, sia per le riunioni interne sia per i confronti con Sync Lab e il proponente.
\end{enumerate}

Tutti i documenti saranno sottoposti a revisione interna e tracciati tramite GitHub 
(issue e pull request) per garantirne la tracciabilità e la storicizzazione.

\subsubsection{Struttura di un documento}
Un documento generalemente, esclusi verbali e diario di bordo, ha questa struttura:
\begin{itemize}
    \item \textbf{Copertina}: Logo e informazioni del team, titolo del documento, redattori, verificatori, versione, 
    data di ultima modifica, destinatari.
    \item \textbf{Registro delle modifiche}: Tabella che traccia le versioni del documento, le modifiche apportate,
     le date e i responsabili.
    \item \textbf{Indice}: Elenco delle sezioni e sottosezioni con i numeri di pagina.
    \item \textbf{Introduzione}: Scopo del documento, scopo del prodotto, glossario e riferimenti.
    \item \textbf{Corpo del documento}: Contenuto principale, suddiviso in sezioni e sottosezioni.
    \item \textbf{Appendici}: Materiale supplementare, come diagrammi, tabelle o esempi.
\end{itemize}

\subsubsubsection{Struttura dei verbali}
I verbali seguono una struttura semplificata:
\begin{itemize}
    \item \textbf{Copertina}: Logo e informazioni del team, titolo del documento, redattore, verificatore, 
     versione, data, uso e destinatari.
    \item \textbf{Abstract}: Elenco degli argomenti da trattare durante la riunione.
    \item \textbf{Indice}: Elenco delle sezioni e sottosezioni con i numeri di pagina.
    \item \textbf{Informazioni Generali}: Data, ora, luogo, partecipanti e assenti.
    \item \textbf{Ordine del Giorno}: Pianificazione degli argomenti da discutere.
    \item \textbf{Svolgimento}: Dettagli delle discussioni ed eventuali argomenti fuori programma.
    \item \textbf{Tabella delle decisioni e delle azioni}: elenco strutturato delle decisioni prese, con descrizione e incaricato/responsabile.
    \item \textbf{Esito Riunione}: Sintesi dei risultati e delle conclusioni.
\end{itemize}
Da questo schema é possibile notare che i verbali non hanno il registro delle modifiche. Questo perché sono considerati documenti di lavoro
e non documenti formali che richiedono una tracciabilità delle versioni.

\subsubsubsection{Struttura dei Diari di Bordo}
I diari di bordo sono diapositive presentate durante la lezione settimanale di Ingegneria del Software dedicata alla
discussione e condivisione delle difficoltá e dubbi incontrati durante lo sviluppo del progetto. 
Per questo motivo i diari di bordo hanno pochi punti:
\begin{itemize}
    \item \textbf{Copertina}: Logo e informazioni del team, titolo del documento.
    \item \textbf{Difficoltà incontrate}: Lista delle difficoltá incontrate durante la settimana.
    \item \textbf{Dubbi}: Elenco di domande e dubbi da porre ai docenti.
\end{itemize}
Visto che i diari di bordo sono da presentare in pochi minuti non hanno bisogno di una struttura complessa, bensí di essere 
sintetici e diretti per permettere a docenti e colleghi di capire velocemente quali sono i problemi riscontrati.


\subsubsection{Denominazione dei documenti}
I documenti prodotti dal team BugBusters seguono una convenzione di denominazione standardizzata per garantire coerenza, 
facilità di identificazione e tracciabilità. \\

La struttura del nome dei \textbf{verbali} è la seguente:
\begin{center}
\texttt{<TIPO>\_<DATA>.pdf}
\end{center}
Dove <TIPO> può essere:
\begin{itemize}
    \item \textbf{VI}: Verbale Interno
    \item \textbf{VE}: Verbale Esterno
\end{itemize}
E <DATA> è la data della riunione nel formato \texttt{GG-MM-AAAA}. \\
Esempio: \texttt{VE\_15-01-2026.pdf} rappresenta il verbale esterno della riunione del 15 gennaio 2026. \\

La struttura del nome dei \textbf{diari di bordo} è la seguente:
\begin{center}
\texttt{DB-<DATA>.pdf}
\end{center}
Dove <DATA> è la data di presentazione del diario nel formato \texttt{GG-MM-AAAA}. \\
Esempio: \texttt{DB-22-01-2026.pdf} rappresenta il diario di bordo presentato il 22 gennaio 2026. \\

Per gli altri \textbf{documenti formali} vengono utilizzati nomi descrittivi chiari e concisi del documento stesso,
ad esempio Norme di Progetto.pdf, Piano di Progetto.pdf, Analisi dei Requisiti.pdf ecc. 


\subsubsection{Produzione}
La produzione della documentazione segue un processo strutturato per garantire qualità, coerenza e tracciabilità.
\begin{itemize}
    \item \textbf{Pianificazione}: A seconda del documento, viene pianificata la sua creazione in base alle milestone 
    del progetto e alle esigenze del team. Secondo il tipo di documento, viene assegnato un redattore e un verificatore 
    secondo i ruoli definiti nel way of working (vedi sezione \ref{Documentazione Prodotta}).
    \item \textbf{Creazione del branch di lavoro}: Per ogni documento viene creato un branch dedicato nel repository GitHub per permettere
     la collaborazione e il versionamento. Per verbali e diari di bordo non vengono creati branch per ogni singolo documenti bensì 
     sono stati creati dei branch "verbali" e "diari-di-bordo" che raccolgono rispettivamente tutti i verbali e tutti i diari di bordo.
    \item \textbf{Redazione}: Il redattore incaricato crea la bozza del documento seguendo le linee guida stabilite
     nelle norme di progetto, utilizzando gli strumenti appropriati (ad esempio LaTeX per documenti formali).
    \item \textbf{Revisione}: Una volta completata la bozza, il documento viene sottoposto al verificatore designato che controlla
     la correttezza, la coerenza e la conformità agli standard di qualità. Il processo é asincrono e 
     avviene tramite pull request su GitHub. Se ci sono modifiche vengono tracciate e discusse all'interno della pull request. 
     La verifica verrá tracciata all'interno del registro delle modifiche del documento.
    \item \textbf{Approvazione e integrazione}: Dopo la revisione, il documento viene approvato dal responsabile e integrato 
    nel branch principale del repository attraverso una pull request.
    Viene aggiornato il registro delle modifiche per riflettere la versione finale.
    \item \textbf{Modifiche}: Qualsiasi modifica successiva al documento segue lo stesso processo di redazione, revisione e approvazione,
     garantendo che tutte le versioni siano tracciate e documentate.
\end{itemize}

\subsection{Gestione della configurazione}\label{gestione configurazione}
La gestione delle configurazioni è l’insieme di attività e strumenti che servono a controllare, 
tracciare e organizzare tutte le modifiche fatte a un qualsiasi elemento del progetto\G{}.\\
Basandosi sullo standard ISO/IEC 12207:1995, la gestione della configurazione deve occuparsi di monitorare
 le modifiche e le versioni degli elementi del sistema, tenere traccia del loro stato e delle
 richieste di cambiamento, assicurare che gli elementi siano completi, coerenti e corretti,
  e controllare come vengono archiviati, spostati e consegnati. 
\subsubsection{Strumenti utilizzati}
Per ognuna delle attività previste, BugBusters utilizza:
\begin{itemize}
    \item \textbf{Git\G{}}: come sistema di controllo di versione distribuito per 
    tracciare le modifiche al codice sorgente e ai documenti.
    \item \textbf{GitHub\G{}}: servizio di hosting per 
    repository Git utilizzato per il versionamento di codice 
    e documentazione. Fornisce strumenti per la collaborazione (branching, pull request e code review), per il tracciamento delle attività (issues, project board) e per l'integrazione continua (Actions). Su GitHub vengono centralizzate le risorse di progetto e registrate tutte le modifiche, garantendo controllo degli accessi e tracciabilità.
\end{itemize}
\subsubsection{Controllo della configurazione}
Il controllo  della configurazione consiste nell'amministrare le richieste di modifica che vengono poi approvate o meno.\\
Per tracciare le modifiche da approvare BugBusters utilizza le \textbf{issue}\G{} la \textbf{board} e le \textbf{pull request} predisposte da GitHub nel modo seguente:
\begin{itemize}
    \item \textbf{Issue}: ogni modifica da apportare viene documentata mediante l'apertura di una issue\G{} relativa, quest'ultima viene assegnata a un componente che la prenderà in carico e procederà con le modifiche al relativo documento o codice.
\end{itemize}
Ogni issue\G{} è identificata da un codice univoco dato dalla fase del progetto\G{} a cui essa è relativa, e da un numero incrementale (es: RTB1,RTB2,RTB3 ecc...).\\
Una issue\G{} viene chiusa solo nel momento in cui è stata verificata.
Per velocizzare il processo di gestione delle issue\G{}, BugBusters utilizza una GitHub action che, se in qualsiasi commit viene scritto: \textit{chiudi <NOME ISSUE>}, la issue\G{} 
corrispondente verrà immediatamente chiusa, qualsiasi sia il branch in cui ci si trova.\\
\begin{itemize}
    \item \textbf{Board}: Serve per definire lo stato in cui si trova una issue\G{}, ovvero se: è ancora da iniziare, è in sviluppo o è terminata.
    \item \textbf{Pull Request}: serve per richiedere la verifica\G{} o approvazione di una modifica prima di fonderla con un ramo della repository\G{}.
    Il team bugbuster, oltre che per un corretto workflow di verifica, usa le pull request come unico modo di modifica in produzione, in quanto si vuole che il branch main sia modificato solo a seguito di una pull request con 2 review positive.
\end{itemize}
\subsubsection{Registrazione stato di configurazione}
Per poter tenere traccia dei cambiamenti di ogni documento e codice, è previsto il sistema di versionamento\G{} seguente:
\begin{center}
    \textbf{X.Y.Z}
\end{center}
Dove:
\begin{itemize}
    \item \textbf{X}: subisce un incremento solo quando il file viene approvato\G{};
    \item \textbf{Y}: subisce un incremento solo quando il file viene verificato\G{};
    \item \textbf{Z}: subisce un incremento quando viene fatto un qualsiasi tipo di modifica al file;
\end{itemize}
\subsection{Qualifica}
\subsubsection{Verifica}
Il processo della Verifica\G{} ha come obiettivo principale quello di verificare che quanto prodotto sia a regola d'arte, ovvero conforme con i requisiti richiesti.\\
L'obiettivo della verifica\G{} è poter rispondere alla domanda <<Did I build the system right?>> gli esiti di tale processo sono riportari nel (Inserire Parte Relativa Nel Piano Di Qualifica).
\subsubsubsection{Attività di verifica}
Il team BugBusters si è principalmente occupato della verifica relativa alla documentazione, controllando la correttezza grammaticale, sintattica e la correttezza del contenuto di ogni documento.\\
Relativamente alle verifiche sul codice, sarà un argomento che verrà trattato più in dettaglio al raggiungimento della \textit{Requirements and Technology Baseline}\G{}.\\
In ogni caso tutte le informazioni relative alla verifica, verranno riportate nel Piano di Qualifica (Inserire Link Futuro alla sezione del Piano relativa ai test).
Il processo di verifica\G{} verrà realizzata in due modi: tramite Analisi Statica e Analisi Dinamica.
\subsubsubsection{Analisi Statica}
L'analisi statica non richiede l'esecuzione dell'oggetto di cui si fa la verifica\G{}, ciò permette di applicarla già prima della fine della codifica.\\
Può essere eseguita mediante \textbf{metodi formali}(ad esempio prove matematiche) o mediante \textbf{metodi di lettura}.
Tra i \textbf{metodi di lettura} ne troviamo due significativi:
\begin{itemize}
    \item \textbf{Walkthrough}: si esegue un esame privo di assunzioni o presupposti , non si sa esattamente dove è più probabile che vi siano difetti, dunque si guarda ovunque. Si percorre dunque il codice simulandone possibili esecuzioni e si studia ogni parte di documento come farebbe un compilatore. È un metodo di verifica\G{} costoso e non automatizzabile;
    \item \textbf{Inspection}: si esegue un esame focalizzato su presupposti, si sa già dove cercare, i verificatori dunque rilevano la presenza di difetti eseguendo una lettura mirata dell'oggetto di verifica\G{}. Non è esaustiva come il Walkthrough, ma permette di creare una lista di controllo apposita per ogni oggetto di verifica\G{}, così da individuare potenziali problemi;
\end{itemize}

\subsubsubsection{Analisi Dinamica}
L'analisi dinamica è chiamata così perché per essere eseguita necessita l'esecuzione dell'oggetto da verificare.\\
Serve per verificare se sono presenti comportamenti inattesi, e dunque rimuovere o modificare le parti che ne causano il fault.\\
Per raggiungere tale obiettivo, si usufruisce di Test\G{}, che devono essere ripetibili e automatizzabili.\\
Ripetibili poichè se si presenta un failure nel codice, e vengono corretti i fault, posso eseguire lo stesso test\G{} nelle stesse condizioni per verificare l'effettiva correzione del failure.\\
Automatizzabili mediante  driver (che serve per pilotare il test), stub (che simula i moduli necessari al test\G{} ma che non ne sono oggetto) e logger (che registra ciò che avviene durante l'esecuzione).\\
Le principali tipologie di Test\G{} sono:
\begin{itemize}
    \item \textbf{Test di Unità}\G{};
    \item \textbf{Test di Regressione};
    \item \textbf{Test di Integrazione}\G{};
    \item \textbf{Test di Sistema}\G{};
\end{itemize}
I test eseguiti da BugBusters (reperibili nel INSERIRE LINK PIANO DI QUALIFICA DOVE SONO PRESENTI I TEST) sono descritti nella seguente maniera:
DOBBIAMO ANCORA FARLI, SUCCESSIVAMENTE QUA INSERIRE NOMENCLATURA TEST.
\subsubsubsection{Test di Unità\G{}}
I \textbf{test di unità}\G{} hanno lo scopo di verificare le singole unità di un software, definite durante la progettazione di dettaglio. Per unità si intende la più piccola quantità di Software che sia utilmente sottoponibile a verifica\G{} come oggetto singolo.\\
I test di unità\G{} possono essere funzionali (black-box), cioè basati sulle specifiche dell’unità.
In questo caso, vengono verificati gli input e output dati dal sistema, ma non viene verificata la logica che fornisce tali risultati.\\
Tuttavia, i test\G{} funzionali da soli non sono sufficienti per verificare la correttezza della logica interna del modulo. Per questo motivo vengono affiancati dai test\G{} strutturali (white-box), che analizzano la logica interna dell’unità cercando di percorrere tutti i cammini di esecuzione al suo interno, raggiungendo una copertura massima.\\
L’esecuzione di questi test\G{} può essere facilitata dall’uso del debugger.\\
Il testing di unità si considera completo quando tutte le unità del sistema sono state verificate.
\subsubsubsection{Test di Regressione}
I \textbf{Test di regressione} sono necessari affinchè delle modifiche effettuate per aggiunta, correzione o rimozione non pregiudichino le funzionalità già verificate.\\
Per far ciò il test di regressione\G{} comprende tutti  i test\G{} necessari ad accertare che la modifica di una parte P $\in$ S, non causi errori in P o in alcuna altra parte di S esterna a P.
\subsubsubsection{Test di Integrazione\G{}}
I \textbf{Test di integrazione}\G{} verificano il corretto funzionamento di più unità software combinate tra loro, controllando che le loro interazioni avvengano come previsto.\\
L'obiettivo è assicurarsi che le unità già testate singolarmente comunichino e collaborino correttamente tra loro.\\
Ci sono diverse strategie per implementare tali test\G{}:
\begin{itemize}
    \item \textbf{Bottom-up}: si parte dalle componenti più basse(con meno dipendenze) e si integrano progressivamente verso l'alto, usando driver per simulare moduli superiori mancanti;
    \item \textbf{Top-down}: si dalle componenti di alto livello(con più dipendenze) e si scende verso quelli inferiori, usando stub per simulare moduli non ancora sviluppati;
\end{itemize}
\subsubsubsection{Test di Sistema\G{}}
I \textbf{Test di sistema}\G{} verificano il comportamento dell'intero software nel suo complesso, controllando che il sistema soddisfi i requisiti funzionali e non funzionali specificati. 
\subsubsection{Validazione}
La \textbf{validazione} è un processo per confermare in modo definitivo che le caratteristiche del software siano conformi ai bisogni dell'utente finale e all'uso previsto, risponde alla domanda: <<Did i build the right system?>>
\subsubsubsection{Processo di Validazione}
BugBusters ha raccolto tutte le richieste di \textbf{Eggon} e le ha raccolte nell'Analisi dei Requisiti\G{} (LINK DA INSERIRE QUANDO ANALISI DEI REQUISITI è PRONTA), in modo tale da poter tracciare correttamente i requisiti e da poter eseguire i test di accettazione\G{} per controllare che quanto sviluppato corrisponda alle richieste di \textbf{Eggon}.


\section{Processi Organizzativi}

\subsection{Descrizione e Scopo}
Secondo lo standard ISO/IEC 12207:1997, il processo di gestione include tutte le 
attività necessarie a portare a termine un progetto software garantendo il 
conseguimento degli obiettivi prefissati nel rispetto dei requisiti, dei tempi e
 dei costi. \\
Per raggiungere questi risultati il processo si basa su:
\begin{itemize}
    \item una pianificazione accurata delle attività, delle milestone e delle risorse;
    \item il monitoraggio continuo dell'avanzamento mediante indicatori e report periodici;
    \item la gestione dei rischi e delle modifiche attraverso procedure di controllo e di escalation;
    \item l'adozione tempestiva di azioni correttive ogni volta che gli scostamenti rispetto al piano lo richiedono;
    \item la definizione chiara di ruoli, responsabilità e canali di comunicazione fra gli stakeholder.
\end{itemize}
In sintesi, il processo di gestione assicura che il progetto sia 
eseguito in modo controllato e tracciabile, consentendo decisioni informate 
e interventi rapidi per mantenere il progetto allineato ai vincoli di qualità, 
tempi e costi.

\subsection{Attività}

\subsection{Gestione degli Sprint}
Ogni due settimane viene pianificato uno sprint in cui vengono assegnate le attività da svolgere
ai vari membri del team. Viene fatto un meeting di pianificazione in cui si scelgono le attività da svolgere
e si assegnano ai vari membri. Al termine dello sprint viene fatta una review in cui si mostrano i risultati ottenuti
e si discutono le difficoltà incontrate. Viene anche fatto un meeting di retrospettiva in cui si discute su 
cosa è andato bene e cosa può essere migliorato per il prossimo sprint.
Durante lo sprint vengono definiti i vari ruoli che i membri del team devono svolgere.

\subsection{Ruoli}
\subsubsection{Responsabile}
Il responsabile ha il compito di coordinare il team, pianificare le attività,
gestire le risorse e comunicare con il proponente e i docenti.
In particolare dovrà assegnare i compiti, redigere i documenti:
\begin{itemize}
    \item Verbali Esterni ed Interni, con assegnazione dei compiti previsti nelle riunioni.
    \item Diario di Bordo
    \item Avanzamento piano di progetto
\end{itemize}
\subsubsection{Amministratore}
È la figura incaricata di sviluppare, mantenere e migliorare gli strumenti, le 
risorse e i processi che garantiscono il regolare avanzamento del progetto. Si 
occupa di supervisionare la configurazione degli ambienti di lavoro e di 
monitorare le scadenze di carattere amministrativo, offrendo supporto al team 
nelle varie attività.
Tra le sue responsabilità rientrano:
\begin{itemize}
    \item predisporre e gestire gli ambienti di lavoro;
    \item controllare e aggiornare gli strumenti utilizzati dal gruppo per collaborare e comunicare;
    \item occuparsi del versionamento dei documenti;
    \item redigere e mantenere aggiornato il documento Norme di Progetto.
\end{itemize}
\subsubsection{Analista}
L'analista ha il compito di raccogliere, analizzare e documentare i requisiti del 
progetto, assicurandosi che siano chiari, completi e coerenti con le esigenze del 
proponente. Inoltre, collabora con il team per tradurre i requisiti in specifiche tecniche 
utilizzabili durante le fasi di progettazione e sviluppo.
In particolare dovrà occuparsi di:
\begin{itemize}
    \item redigere il documento di Analisi dei Requisiti;
    \item redigere il documento di Piano di Qualifica;
    \item gestire le richieste di chiarimento e modifica dei requisiti con il proponente;
    \item collaborare con il team per garantire che i requisiti siano compresi e implementati correttamente.
\end{itemize}

\subsubsection{Progettista}
Il progettista ha il compito di definire l'architettura e la struttura del sistema, 
assicurandosi che soddisfi i requisiti funzionali e non funzionali stabiliti. Collabora con il team per tradurre i requisiti in soluzioni tecniche 
efficienti e scalabili.
In particolare dovrà occuparsi di:
\begin{itemize}
    \item Determinare le tecnologie e gli strumenti più adatti per lo sviluppo del progetto;
    \item definire l'architettura del sistema e le sue componenti principali;
    \item Supervisionare lo sviluppo tecnico e garantire la coerenza con le specifiche progettuali;
\end{itemize}

\subsubsubsection{Programmatore}
Il programmatore ha il compito di implementare le funzionalità del sistema
secondo le specifiche tecniche fornite dal progettista. Collabora con il team per
garantire che il codice sia di alta qualità, efficiente e manutenibile.
In particolare dovrà occuparsi di:
\begin{itemize}
    \item scrivere il codice sorgente seguendo le linee guida e gli standard di
    programmazione stabiliti, traducendo le specifiche tecniche definite dal progettista in soluzioni funzionanti;
    \item Occuparsi del testing del codice per garantire la correttezza e 
    l'affidabilità delle funzionalità implementate;
    \item collaborare con il team per risolvere problemi tecnici e implementare
    nuove funzionalità.
\end{itemize}

\subsubsection{Verificatore}
Il verificatore ha il compito di assicurare che il prodotto sviluppato soddisfi i requisiti
stabiliti e che sia di alta qualità. Collabora con il team per identificare e risolvere
problemi, garantendo che il sistema sia affidabile, efficiente e conforme agli standard.
In particolare dovrà occuparsi di:
\begin{itemize}
    \item pianificare e condurre attività di verifica, come test funzionali, test di integrazione e test di sistema;
    \item revisionare la documentazione prodotta per garantire la correttezza e la completezza;
    \item documentare i risultati delle attività di verifica e segnalare eventuali difetti
    o non conformità riscontrate;
    \item identificare possibili miglioramenti o punti critici nel processo di sviluppo e proporre soluzioni per
    aumentarne l'efficacia.
\end{itemize}

\subsection{Tracciamento Ore}
Per garantire una corretta gestione del tempo e delle risorse, ogni 
membro del team BugBusters è tenuto a registrare le ore lavorate su 
ciascuna attività. Il tracciamento avviene tramite un foglio di 
calcolo condiviso su Google Sheets, dove ogni membro del team 
inserisce le ore dedicate alle varie attività e alle relative 
issue di progetto. Questo approccio centralizzato consente di 
monitorare l'avanzamento effettivo rispetto alla pianificazione, 
identificare tempestivamente discrepanze tra stima e consuntivo e 
facilitare decisioni di riallocazione delle risorse, se necessario.

Il foglio di calcolo contiene inoltre la tabella di stima iniziale, 
permettendo una comparazione immediata tra preventivo e consuntivo 
e garantendo visibilità sull'andamento complessivo del progetto. 
Ogni membro del team è responsabile dell'accuratezza e della propria registrazione.


\subsection{Tracciamento Azioni}
Le azioni sono tracciate nei verbali delle riunioni interne ed esterne.
Dopo ogni riunione il responsabile redige il verbale in cui vengono riportate tutte le decisioni 
prese e le azioni da svolgere.
Ogni azione viene assegnata a un membro del team tramite issue tracking su GitHub, dunque nei verbali è presente
il collegamento con il ticket aperto sulla piattaforma.
Nel caso ci fossero problemi o dubbi riguardo l'azione assegnata, il membro del team può aprire una discussione
all'interno della issue per chiedere chiarimenti o segnalare difficoltà.
Come riportato dalla sezione \ref{gestione configurazione}, il team utilizza automazioni per rendere
più fluido il processo di tracciamento delle azioni, garantendo maggior tracciamento possibile.

\subsection{Comunicazione}
\subsubsection{Comunicazione Interna}
Le comunicazioni interne al team BugBusters avvengono principalmente tramite:
\begin{itemize}
    \item \textbf{WhatsApp}: utilizzato per comunicazioni rapide, discussioni tecniche e coordinamento quotidiano tra i membri del team.
    \item \textbf{Discussioni GitHub}: utilizzate per comunicazioni formali relative a issue, pull request e documentazione, 
    garantendo tracciabilità e storicizzazione delle decisioni. Utili in quanto permettono di cominicare solo con i membri interessati alla discussione.
\end{itemize}
In caso di problemi tecnici o necessità di confronto più approfondito, il team può organizzare riunioni virtuali tramite piattaforme Discord.
Per ulteriori dettagli sulle riunioni, si veda la sezione \ref{riunioni}.

\subsubsection{Comunicazione Esterna}
Le comunicazioni esterne con il proponente e i docenti avvengono principalmente tramite:
\begin{itemize}
    \item \textbf{Email}: utilizzata per comunicazioni formali, invio di documenti e aggiornamenti sullo stato del progetto, 
    usando la mail bugbusters.unipd@gmail.com.
    \item \textbf{Telegram}: utilizzato per comunicazioni rapide e coordinamento con il proponente.
\end{itemize}
Analogmanete per quanto riguarda le comunicazioni interne, in caso 
di necessità di confronto più approfondito, il team può organizzare riunioni virtuali tramite piattaforma Google Meet.
Per ulteriori dettagli sulle riunioni, si veda la sezione \ref{riunioni}.

\subsection{Riunioni}\label{riunioni}
\subsubsection{Riunioni Interne}
\subsubsection{Riunioni Esterne}
\section{Standard di progetto}
\subsection{Standard ISO/IEC 9126}
Lo standard ISO/IEC 9126 fornisce delle linee guida e normative, in modo da poter fornire un modello univoco e riconosciuto globalmente che misura la qualità del software.\\
Questo modello è definito da sei caratteristiche generali:
\begin{itemize}
    \item \textbf{Funzionalità};
    \item \textbf{Affidabilità};
    \item \textbf{Efficienza};
    \item \textbf{Usabilità};
    \item \textbf{Manutentibilità};
    \item \textbf{Portabilità};
\end{itemize}
\subsubsection{Funzionalità}
La funzionalità verifica se sono soddisfatti i requisiti funzionali definiti con la proponente.\\ Nello specifico misura: 
\begin{itemize}
    \item \textbf{Appropriatezza}: definisce se il prodotto software offre le funzioni necessarie per svolgere i compiti fissati dal proponente.
    \item \textbf{Accuratezza}: è la capacità del prodotto software di produrre i precisi risultati richiesti.
    \item \textbf{Interoperabilità}: è la capacità del prodotto software di poter interagire e operare con altri sistemi. 
    \item \textbf{Conformità}: è la capacità del prodotto software di essere conforme a standard, convenzioni e regolamentazioni che sono di rilievo per il settore nel quale deve operare.
    \item \textbf{Sicurezza}: è la capacità del prodotto software di tutelare le informazioni, assicurando che ciascun utente possa accedere esclusivamente ai dati per i quali possiede le necessarie autorizzazioni.
\end{itemize}
\subsubsection{Affidabilità}
Consiste nella capacità con cui il software svolge i suoi compiti mantenendo un certo livello di prestazioni in qualsiasi momento, anche in periodi d'uso intensi.\\ Viene misurato nello specifico:
\begin{itemize}
    \item \textbf{Maturità}: è la capacità di un prodotto software di evitare che si verifichino errori, malfunzionamenti o che siano prodotti risultati errati.
    \item \textbf{Tolleranza agli errori}: è la capacità di mantenere livelli costanti di prestazioni anche in caso di malfunzionamenti o usi scorretti del prodotto.
    \item \textbf{Recuperabilità}: indica la capacità di un prodotto software di ripristinare un adeguato livello di funzionamento e recuperare le informazioni necessarie dopo un guasto o un malfunzionamento.
    \item \textbf{Aderenza}: indica la capacità di aderire a standard o convenzioni inerenti all'affidabilità.
\end{itemize}
\subsubsection{Efficienza}
L'efficienza misura la capacità di fornire appropriate prestazioni in confronto alla quantità di risorse utilizzate.\\ Viene misurato nello specifico:
\begin{itemize}
    \item \textbf{Comportamento rispetto al tempo}: è la capacità di fornire adeguati tempi di risposta ed elaborazione, sotto determinate condizioni.
    \item \textbf{Utilizzo delle risorse}: è la capacità di utilizzare quantità e tipo di risorse in maniera adeguata.
    \item \textbf{Conformità}: è la capacità del prodotto ad aderire a standard relativi all'efficienza.
\end{itemize}
\subsubsection{Usabilità}
L'usabilitá è la capacità del prodotto software di essere compreso e utilzzato dall'utente finale.\\ Viene misurato nello specifico:
\begin{itemize}
    \item \textbf{Comprensibilità}: misura la facilità di comprensione dei concetti del prodotto.
    \item \textbf{Apprendibilità}: misura la facilità nell'apprendere l'uso delle funzionalità del prodotto.
    \item \textbf{Operabilità}: definisce se l'utilizzo del prodotto da parte degli utenti risulta semplice o no.
    \item \textbf{Attrattiva}: è la capacità del software di risultare piacevole all'utente quando viene utilizzato.
    \item \textbf{Conformità}: è la capacità del prodotto ad aderire a standard relativi all'usabilitá.
\end{itemize}
\subsubsection{Manutentibilità}
La manutentibilità è la capacità del software di essere modificato, includendo correzioni, adattamenti o miglioramenti.\\ Viene misurato nello specifico:
\begin{itemize}
    \item \textbf{Analizzabilità}: misura la capacità con la quale è possibile analizzare il codice per trovare un errore in esso presente.
    \item \textbf{Modificabilità}: misura la capacità del prodotto software di permettere l'implementazione di modifiche specifiche.
    \item \textbf{Stabilità}: misura la capacità del prodotto software di evitare effetti inaspettati derivanti da modifiche errate.
    \item \textbf{Testabilità}: definisce la capacità del prodotto software di essere facilmente testato per validare le modifiche apportate.
\end{itemize}
\subsubsection{Portabilità}
La portabilità definisce la capacità del software di essere traspostato da un ambiente di lavoro ad un altro.\\ Viene misurato nello specifico:
\begin{itemize}
    \item \textbf{Adattabilità}: è la capacità del software di essere adattato per differenti ambienti operativi senza dover apportare modifiche.
    \item \textbf{Installabilità}: è la capacità del software di essere installato in uno specifico ambiente.
    \item \textbf{Conformità}: è la capacità del prodotto software di aderire a standard relativi alla portabilità.
    \item \textbf{Sostituibilità}: è la capacità del software di essere utilizzato  al posto di un altro software per svolgere gli stessi compiti di quest'ultimo nello stesso ambiente.
\end{itemize}
\end{document}



