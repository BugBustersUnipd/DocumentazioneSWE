\documentclass[a4paper,11pt]{article}
   
\newcommand{\CurrentVersion}{1.0.0} % ultima versione, da cambiare ad ogni push significativo

\usepackage[utf8]{inputenc}
\usepackage[T1]{fontenc}
\usepackage[italian]{babel}
\usepackage[margin=2.5cm]{geometry}
\usepackage{graphicx}
\usepackage{grffile}
\usepackage{booktabs}
\usepackage{setspace}
\usepackage{titlesec}
\usepackage{float}
\usepackage{ifthen}
\usepackage[table]{xcolor}
\usepackage{tabularx}
\usepackage{tcolorbox}
\usepackage{enumitem}
\usepackage[titles]{tocloft}
\usepackage[colorlinks=true,linkcolor=black,urlcolor=blue,citecolor=blue]{hyperref}

\definecolor{primaryblue}{RGB}{0,102,204}
\definecolor{secondaryblue}{RGB}{51,153,255}
\definecolor{lightgray}{RGB}{245,245,245}
\definecolor{darkgray}{RGB}{100,100,100}

\titleformat{\section}
 {\Large\bfseries\color{primaryblue}}
 {\thesection}{1em}{}

\titleformat{\subsection}
 {\large\bfseries\color{primaryblue}} % Sottosezione: colore secondaryblue
 {\thesubsection}{1em}{}

\titleformat{\subsubsection}
 {\normalsize\bfseries\color{secondaryblue}} % Sotto-sottosezione: colore secondaryblue
 {\thesubsubsection}{1em}{}

% per il footer con il numero di pagina
\usepackage{fancyhdr}
\usepackage{lastpage} % per ottenere il numero dell'ultima pagina da mettere nel footer


\usepackage{ltablex} %per far andare a capo le tabelle
\keepXColumns

\renewcommand{\sectionmark}[1]{\markright{#1}}
\newcommand{\G}{\textsubscript{\scalebox{0.6}{\textbf{G}}}}



\setlength{\parskip}{4pt}
\setlength{\parindent}{0pt}

\setlist[itemize]{leftmargin=*,itemsep=3pt}
\setlist[enumerate]{leftmargin=*,itemsep=3pt}

\graphicspath{{./}{../assets/images/}{./images/}}

\begin{document}

%configurazione per il footer
\pagestyle{fancy}
\fancyhf{} % pulisce tutti i campi del header e footer

% Header: sinistra e destra
\fancyhead[L]{Gruppo 4 - BugBusters} % sinistra
\fancyhead[R]{Norme di Progetto}   % destra

\fancyfoot[L]{ \thepage\ di \pageref{LastPage}} %definisce il formato del footer
\fancyfoot[R]{ \nouppercase{\rightmark}} % nome della sezione


\renewcommand{\headrulewidth}{0pt} % rimuove la linea dell'header
\renewcommand{\footrulewidth}{0pt} % se vuoi anche togliere eventuale linea del footer


\begin{center}
 \thispagestyle{empty}
 \IfFileExists{../../assets/Logo.jpg}{%
  \includegraphics[width=6cm,height=3cm,keepaspectratio]{../../assets/Logo.jpg} \\[0.8cm]
 }{%
  \fbox{\parbox[c][2.5cm][c]{6cm}{\centering Logo non trovato\\(Logo.jpg)}}\\[0.5cm]
 }
 {\Large\bfseries BugBusters}\\[0.3cm]
 {\small\color{darkgray} Email: \texttt{bugbusters.unipd@gmail.com}} \\[0.1cm]
 {\small\color{darkgray} Gruppo: 4} \\[0.5cm]

 {\large\bfseries Università degli Studi di Padova}\\[0.3cm]
 {\small Laurea in Informatica}\\[0.2cm]
 {\small Corso: Ingegneria del Software}\\[0.2cm]
 {\small Anno Accademico: 2025/2026}\\[0.8cm]

 {\Huge\bfseries\color{primaryblue} Norme di Progetto}\\[0.8cm]
 {\Large\color{secondaryblue}Versione \CurrentVersion}\\[0.8cm]
\end{center}

\begin{center}
\begin{tcolorbox}[colback=lightgray,colframe=primaryblue,width=0.85\textwidth,arc=3mm,boxrule=0.5pt]
\begin{tabularx}{\linewidth}{@{}lX@{}}
\textbf{Destinatari} & BugBusters, Prof. Tullio Vardanega, Prof. Riccardo Cardin \\
\textbf{Data ultima modifica}     & 13/11/2025 \\
\\
\end{tabularx}
\end{tcolorbox}
\end{center}

\vspace{0.5cm}

\begin{center}
\begin{tcolorbox}[colback=secondaryblue!10,colframe=secondaryblue,width=0.9\textwidth,arc=3mm,boxrule=0.8pt,title={\bfseries Abstract}]
Documento contenente le norme di progetto adottate dal team BugBusters per lo sviluppo del progetto Nexum proposto dall'azienda Eggon. Il documento include metodologie di lavoro, standard di codifica, processi di sviluppo e gestione del progetto.
\end{tcolorbox}
\end{center}

\newpage

\section*{Registro delle modifiche}

\setlength{\extrarowheight}{2pt} % padding extra verticale
\renewcommand{\arraystretch}{1.5} 

\arrayrulecolor{primaryblue}
{\footnotesize
\begin{tabularx}{\textwidth}{|>{\raggedright\arraybackslash}p{1.5cm}|>{\raggedright\arraybackslash}p{2cm}|X|>{\raggedright\arraybackslash}p{2cm}|>{\raggedright\arraybackslash}p{2cm}|>{\raggedright\arraybackslash}p{2cm}|}
\hline
\rowcolor{primaryblue!40}
\textbf{\color{white} Versione} & \textbf{\color{white} Data} & \textbf{\color{white} Descrizione} & \textbf{\color{white} Redatto} & \textbf{\color{white} Verificato} & \textbf{\color{white} Approvato} \\
\hline
\rowcolor{secondaryblue!10} \CurrentVersion & 13/11/2025 & Versione finale del documento & - & Marco Piro & Alberto Autiero \\
\hline
\rowcolor{secondaryblue!10} 0.0.1 & 04/11/2025 & 
\begin{minipage}[t]{\linewidth}
Prima stesura della struttura del documento dopo l'aggiudicazione dell'appalto per il capitolato C5 Nexum dell'azienda Eggon
\end{minipage} 
& Alberto Autiero & - & - \\
\hline
\end{tabularx}
}


\newpage
\tableofcontents

\newpage

\section{Introduzione}

\subsection{Scopo del documento}
Questo documento definisce le norme di progetto adottate dal team BugBusters per lo sviluppo del progetto Nexum, 
proposto dall'azienda Eggon. Le norme di progetto includono metodologie di lavoro, standard di codifica, 
processi di sviluppo e gestione del progetto, al fine di garantire un approccio strutturato e coerente 
durante l'intero ciclo di vita del progetto.

Per strutturare il nostro way of working\G, faremo riferimento alle best practice\G suggerite dallo standard ISO/IEC 12207:1995
adattandole alle esigenze specifiche del nostro team e del progetto Nexum. In particolare identifica tre tipologie di processi:
\begin{itemize}
    \item Processi primari\G: processi direttamente coinvolti nella creazione del prodotto software
    \item Processi di supporto\G: processi che supportano i processi primari\G
    \item Processi organizzativi\G: processi che gestiscono e coordinano le attività del team
\end{itemize}
La combinazione di questi processi ci permetterà di gestire in modo efficace lo sviluppo del progetto Nexum. 
La stesura di questo documento mira a fornire una guida chiara e condivisa per tutti i membri del team, 
assicurando che le attività di sviluppo siano svolte in modo efficiente e conforme agli standard di qualità 
previsti. Il documento è redatto in maniera incrementale: verrà aggiornato e ampliato progressivamente durante 
lo sviluppo del progetto per riflettere decisioni, modifiche e miglioramenti adottati dal team.


\subsection{Scopo del prodotto}
Nexum è una piattaforma con un modulo dedicato alle comunicazioni interne, alla raccolta di feedback e alla timbratura digitale.
Include una messaggistica top-down con tracciamento delle letture, un builder per survey con logiche di ramificazione e dashboard 
in tempo reale; inoltre un sistema di timbratura via badge o dispositivi mobili con regole automatiche per il controllo e l'aggregazione delle ore.
Le anagrafiche centralizzate, con ruoli e permessi, permettono una gestione granulare degli accessi e l'integrazione dei moduli, 
garantendo un flusso informativo coerente, tracciabile e adattabile alle esigenze operative.
Il nostro progetto mira a estendere la piattaforma Nexum con un AI assistant generativo per la scrittura di comunicazioni e con un AI Co-Pilot\G per
i Cdl. In particolare quest'ultimo deve essere in grado di riconoscere i documenti caricati dagli utenti, estrarne le informazioni rilevanti, capire
la tipologia, destinatari e consegnarli in modo massivo. \\
Il nostro obiettivo é realizzare questo progetto entro il 21 marzo 2026 con un budget di 12.790 euro.


\subsection{Glossario}
Il glossario raccoglie e definisce i termini, gli acronimi e le abbreviazioni 
impiegati nel documento e nel progetto Nexum. L'obiettivo è fornire definizioni univoche per ridurre 
ambiguità, garantire coerenza terminologica tra i membri del team e facilitare l'onboarding di nuovi partecipanti. \\
Per i termini tecnici e specifici utilizzati in questo documento, si fa riferimento al glossario disponibile al 
seguente \href{https://bugbustersunipd.github.io/DocumentazioneSWE/RTB/GLOSSARIO/Glossario.pdf}{link}.
Per maggiore usabilitá e facilitá di consultazione, il glossario è accessibile anche aprendo dal nostro sito web i vari documenti con il viewer
pdf da noi sviluppato.

\subsubsection{Riferimenti normativi}
\begin{itemize}
\item \textbf{Capitolato\textsubscript{\scalebox{0.6}{\textbf{G}}}
 d'appalto C5: Nexum - Piattaforma di consulenza e documentazione previdenziale}\\
\url{https://www.math.unipd.it/~tullio/IS-1/2025/Progetto/C5.pdf}
\end{itemize}

\subsubsection{Riferimenti informativi}
\begin{itemize}
\item \textbf{Glossario\textsubscript{\scalebox{0.6}{\textbf{G}}}
:}\\
\url{https://github.com/BugBustersUnipd/DocumentazioneSWE/blob/main//RTB/GLOSSARIO/Glossario.pdf}
\end{itemize}

\end{document}