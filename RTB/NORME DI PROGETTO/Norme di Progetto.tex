\documentclass[a4paper,11pt]{article}
   
\newcommand{\CurrentVersion}{1.0.0} % ultima versione, da cambiare ad ogni push significativo

\usepackage[utf8]{inputenc}
\usepackage[T1]{fontenc}
\usepackage[italian]{babel}
\usepackage[margin=2.5cm]{geometry}
\usepackage{graphicx}
\usepackage{grffile}
\usepackage{booktabs}
\usepackage{setspace}
\usepackage{titlesec}
\usepackage{float}
\usepackage{ifthen}
\usepackage[table]{xcolor}
\usepackage{tabularx}
\usepackage{tcolorbox}
\usepackage{enumitem}
\usepackage[titles]{tocloft}
\usepackage[colorlinks=true,linkcolor=black,urlcolor=blue,citecolor=blue]{hyperref}

\definecolor{primaryblue}{RGB}{0,102,204}
\definecolor{secondaryblue}{RGB}{51,153,255}
\definecolor{lightgray}{RGB}{245,245,245}
\definecolor{darkgray}{RGB}{100,100,100}

\titleformat{\section}
 {\Large\bfseries\color{primaryblue}}
 {\thesection}{1em}{}

\titleformat{\subsection}
 {\large\bfseries\color{primaryblue}} % Sottosezione: colore secondaryblue
 {\thesubsection}{1em}{}

\titleformat{\subsubsection}
 {\normalsize\bfseries\color{secondaryblue}} % Sotto-sottosezione: colore secondaryblue
 {\thesubsubsection}{1em}{}

% per il footer con il numero di pagina
\usepackage{fancyhdr}
\usepackage{lastpage} % per ottenere il numero dell'ultima pagina da mettere nel footer


\usepackage{ltablex} %per far andare a capo le tabelle
\keepXColumns

\renewcommand{\sectionmark}[1]{\markright{#1}}
\newcommand{\G}{\textsubscript{\scalebox{0.6}{\textbf{G}}}}



\setlength{\parskip}{4pt}
\setlength{\parindent}{0pt}

\setlist[itemize]{leftmargin=*,itemsep=3pt}
\setlist[enumerate]{leftmargin=*,itemsep=3pt}

\graphicspath{{./}{../assets/images/}{./images/}}

\begin{document}

%configurazione per il footer
\pagestyle{fancy}
\fancyhf{} % pulisce tutti i campi del header e footer

% Header: sinistra e destra
\fancyhead[L]{Gruppo 4 - BugBusters} % sinistra
\fancyhead[R]{Norme di Progetto}   % destra

\fancyfoot[L]{ \thepage\ di \pageref{LastPage}} %definisce il formato del footer
\fancyfoot[R]{ \nouppercase{\rightmark}} % nome della sezione


\renewcommand{\headrulewidth}{0pt} % rimuove la linea dell'header
\renewcommand{\footrulewidth}{0pt} % se vuoi anche togliere eventuale linea del footer

% abilitare numerazione e TOC fino al livello "paragraph" (subsubsubsection)
\setcounter{secnumdepth}{4}
\setcounter{tocdepth}{4}
% formattazione del nuovo livello per avere aspetto coerente
\titleformat{\paragraph}[block]{\normalsize\bfseries\color{secondaryblue}}{\theparagraph}{1em}{}
% alias comodo per usare "subsubsubsection"
\newcommand{\subsubsubsection}{\paragraph}

\begin{center}
 \thispagestyle{empty}
 \IfFileExists{../../assets/Logo.jpg}{%
  \includegraphics[width=6cm,height=3cm,keepaspectratio]{../../assets/Logo.jpg} \\[0.8cm]
 }{%
  \fbox{\parbox[c][2.5cm][c]{6cm}{\centering Logo non trovato\\(Logo.jpg)}}\\[0.5cm]
 }
 {\Large\bfseries BugBusters}\\[0.3cm]
 {\small\color{darkgray} Email: \texttt{bugbusters.unipd@gmail.com}} \\[0.1cm]
 {\small\color{darkgray} Gruppo: 4} \\[0.5cm]

 {\large\bfseries Università degli Studi di Padova}\\[0.3cm]
 {\small Laurea in Informatica}\\[0.2cm]
 {\small Corso: Ingegneria del Software}\\[0.2cm]
 {\small Anno Accademico: 2025/2026}\\[0.8cm]

 {\Huge\bfseries\color{primaryblue} Norme di Progetto}\\[0.8cm]
 {\Large\color{secondaryblue}Versione \CurrentVersion}\\[0.8cm]
\end{center}

\begin{center}
\begin{tcolorbox}[colback=lightgray,colframe=primaryblue,width=0.85\textwidth,arc=3mm,boxrule=0.5pt]
\begin{tabularx}{\linewidth}{@{}lX@{}}
\textbf{Destinatari} & BugBusters, Prof. Tullio Vardanega, Prof. Riccardo Cardin \\
\textbf{Data ultima modifica}     & 17/11/2025 \\
\\
\end{tabularx}
\end{tcolorbox}
\end{center}

\vspace{0.5cm}

\begin{center}
\begin{tcolorbox}[colback=secondaryblue!10,colframe=secondaryblue,width=0.9\textwidth,arc=3mm,boxrule=0.8pt,title={\bfseries Abstract}]
Documento contenente le norme di progetto adottate dal team BugBusters per lo sviluppo del progetto Nexum proposto dall'azienda Eggon. Il documento include metodologie di lavoro, standard di codifica, processi di sviluppo e gestione del progetto.
\end{tcolorbox}
\end{center}

\newpage

\section*{Registro delle modifiche}

\setlength{\extrarowheight}{2pt} % padding extra verticale
\renewcommand{\arraystretch}{1.5} 

\arrayrulecolor{primaryblue}
{\footnotesize
\begin{tabularx}{\textwidth}{|>{\raggedright\arraybackslash}p{1.5cm}|>{\raggedright\arraybackslash}p{2cm}|X|>{\raggedright\arraybackslash}p{2cm}|>{\raggedright\arraybackslash}p{2cm}|>{\raggedright\arraybackslash}p{2cm}|}
\hline
\rowcolor{primaryblue!40}
\textbf{\color{white} Versione} & \textbf{\color{white} Data} & \textbf{\color{white} Descrizione} & \textbf{\color{white} Redatto} & \textbf{\color{white} Verificato} & \textbf{\color{white} Approvato} \\
\hline
\rowcolor{secondaryblue!10} \CurrentVersion & 13/11/2025 & Versione finale del documento & - & Marco Piro & Alberto Autiero \\
\hline
\rowcolor{secondaryblue!10} 0.0.1 & 04/11/2025 & 
\begin{minipage}[t]{\linewidth}
Prima stesura della struttura del documento dopo l'aggiudicazione dell'appalto per il capitolato C5 Nexum dell'azienda Eggon
\end{minipage} 
& Alberto Autiero & - & - \\
\hline
\end{tabularx}
}


\newpage
\tableofcontents

\newpage

\section{Introduzione}

\subsection{Scopo del documento}
Questo documento definisce le norme di progetto adottate dal team BugBusters per lo sviluppo del progetto Nexum, 
proposto dall'azienda Eggon. Le norme di progetto includono metodologie di lavoro, standard di codifica, 
processi di sviluppo e gestione del progetto, al fine di garantire un approccio strutturato e coerente 
durante l'intero ciclo di vita del progetto.

Per strutturare il nostro way of working\G, faremo riferimento alle best practice\G suggerite dallo standard ISO/IEC 12207:1995
adattandole alle esigenze specifiche del nostro team e del progetto Nexum. In particolare identifica tre tipologie di processi:
\begin{itemize}
    \item Processi primari\G: processi direttamente coinvolti nella creazione del prodotto software
    \item Processi di supporto\G: processi che supportano i processi primari\G
    \item Processi organizzativi\G: processi che gestiscono e coordinano le attività del team
\end{itemize}
La combinazione di questi processi ci permetterà di gestire in modo efficace lo sviluppo del progetto Nexum. 
La stesura di questo documento mira a fornire una guida chiara e condivisa per tutti i membri del team, 
assicurando che le attività di sviluppo siano svolte in modo efficiente e conforme agli standard di qualità 
previsti. Il documento è redatto in maniera incrementale: verrà aggiornato e ampliato progressivamente durante 
lo sviluppo del progetto per riflettere decisioni, modifiche e miglioramenti adottati dal team.


\subsection{Scopo del prodotto}
Nexum è una piattaforma con un modulo dedicato alle comunicazioni interne, alla raccolta di feedback e alla timbratura digitale.
Include una messaggistica top-down con tracciamento delle letture, un builder per survey con logiche di ramificazione e dashboard 
in tempo reale; inoltre un sistema di timbratura via badge o dispositivi mobili con regole automatiche per il controllo e l'aggregazione delle ore.
Le anagrafiche centralizzate, con ruoli e permessi, permettono una gestione granulare degli accessi e l'integrazione dei moduli, 
garantendo un flusso informativo coerente, tracciabile e adattabile alle esigenze operative.
Il nostro progetto mira a estendere la piattaforma Nexum con un AI assistant generativo per la scrittura di comunicazioni e con un AI Co-Pilot\G per
i Cdl. In particolare quest'ultimo deve essere in grado di riconoscere i documenti caricati dagli utenti, estrarne le informazioni rilevanti, capire
la tipologia, destinatari e consegnarli in modo massivo. \\
Il nostro obiettivo é realizzare questo progetto entro il 21 marzo 2026 con un budget di 12.790 euro.


\subsection{Glossario}
Il glossario raccoglie e definisce i termini, gli acronimi e le abbreviazioni 
impiegati nel documento e nel progetto Nexum. L'obiettivo è fornire definizioni univoche per ridurre 
ambiguità, garantire coerenza terminologica tra i membri del team e facilitare l'onboarding di nuovi partecipanti. \\
Per i termini tecnici e specifici utilizzati in questo documento, si fa riferimento al glossario disponibile al 
seguente \href{https://bugbustersunipd.github.io/DocumentazioneSWE/RTB/GLOSSARIO/Glossario.pdf}{link}.
Per maggiore usabilitá e facilitá di consultazione, il glossario è accessibile anche aprendo dal nostro sito web i vari documenti con il viewer
pdf da noi sviluppato.

\subsubsection{Riferimenti normativi}
\begin{itemize}
\item \textbf{Capitolato\textsubscript{\scalebox{0.6}{\textbf{G}}}
 d'appalto C5: Nexum - Piattaforma di consulenza e documentazione previdenziale}\\
\url{https://www.math.unipd.it/~tullio/IS-1/2025/Progetto/C5.pdf}
\end{itemize}

\subsubsection{Riferimenti informativi}
\begin{itemize}
    \item \textbf{Glossario\G:}\\
\url{https://github.com/BugBustersUnipd/DocumentazioneSWE/blob/main//RTB/GLOSSARIO/Glossario.pdf}
\end{itemize}

\newpage

\section{Processi Primari}
L'obiettivo principale di BugBusters è la realizzazione di un prodotto software di alta qualità che soddisfi 
le esigenze del cliente e degli utenti finali. Per raggiungere questo obiettivo, è indispensabile basarsi 
su un modello di riferimento che definisca processi chiari da seguire. \\
L'adozione di un simile framework metodologico, che indirizza ad esempio le fasi di acquisizione e di costruzione 
del software, è ciò che permette di passare da un semplice programma funzionante a un prodotto di valore e di 
lunga durata.
Basandosi dunque sullo standard ISO/IEC 12207:1995, il team BugBusters ha deciso di adottare i seguenti processi 
primari:
\begin{itemize}
    \item Processi di acquisizione
    \item Processi di fornitura
    \item Processi di sviluppo
\end{itemize}

\subsection{Processi di acquisizione}
BOH NON SO CHE DIRE AAEHTENGEGAEFANCIPETGEA


\subsection{Processi di fornitura}
\subsubsection{Scopo}
Lo scopo del processo di fornitura è definire e regolamentare l'insieme delle attività preliminari 
necessarie ad avviare il progetto in modo controllato, condiviso e verificabile. In particolare, 
tale processo ha l'obiettivo di chiarire i requisiti richiesti dal proponente, identificare vincoli e risorse, 
pianificare le modalità operative, stabilire gli strumenti di lavoro e formalizzare la documentazione necessaria, 
così da garantire una corretta base metodologica e contrattuale per le successive fasi di sviluppo.

\subsubsection{Attivitá}
La fornitura prevede varie attitá, in particolare: 
\begin{itemize}
    \item \textbf{Analisi del capitolato proposto}: raccolta di informazioni, requisiti, tecnologie e vincoli presenti nel 
    capitolato d'appalto. In questa fase si pongono domande al proponente per chiarire eventuali dubbi o ambiguità.
    \item \textbf{Stima delle risorse}: definizione delle tempistiche, dei costi totali e delle risorse necessarie per la 
    realizzazione del progetto.
    \item \textbf{Analisi dei requisiti e contrattazione con il proponente}: identificazione e documentazione dei requisiti 
    funzionali e non funzionali del software, analisi delle aspettative del proponente e negoziazione di eventuali modifiche o aggiunte. Definizione del Minimum Viable Product (MVP).
    \item \textbf{Pianificazione del progetto}: suddivisione del progetto in fasi, definizione delle milestone (RTB e PB), 
    assegnazione dei compiti ai membri del team e pianificazione delle attività. Individuazione degli strumenti di lavoro e delle 
    metodologie da adottare; definizione del Proof of Concept (PoC).
    \item \textbf{Documentazione}: redazione di tutti i documenti necessari per formalizzare la fornitura, come il Piano di Progetto, 
    il Analisi dei requisiti, Norme di Progetto e altri documenti di supporto. La documentazione prodotta sarà utilizzata sia come 
    strumento di lavoro interno al team sia come riferimento e strumento di controllo da parte del proponente.
    \item \textbf{Comunicazione con il proponente}: stabilire canali di comunicazione efficaci per garantire un flusso informativo 
    continuo e trasparente. Prevedere incontri regolari per aggiornamenti sullo stato del progetto, discussione di eventuali problemi 
    e raccolta di feedback.
    \item \textbf{Revisione e approvazione}: sottoporre tutta la documentazione prodotta alla revisione e approvazione del proponente, 
    assicurando l’allineamento sugli obiettivi e le aspettative prima di procedere con le fasi successive del progetto.
    \item \textbf{Consegna e chiusura della fase di fornitura}: consegna di quanto prodotto durante la fase di fornitura al proponente, 
    garantendo che tutti i documenti siano completi e corretti. Completata la consegna, si procede alla chiusura formale della fase.
\end{itemize}

\subsubsection{Strumenti di Supporto}
Per supportare le attività di fornitura, il team BugBusters usofruisce di vari strumenti:
\begin{itemize}
    \item \textbf{GitHub}: Per la documentazione collaborativa, il versionamento dei documenti, la produzione asincrona,
     il sistema di ticketing e delle project board.
    \item \textbf{GitLab}: Per il codice già esistente fornito dal proponente e per l'implementazioe delle nuove feature.
    \item \textbf{Discord}: Per le riunioni di team.
    \item \textbf{WhatsApp}: Per comunicazioni rapide e aggiornamenti interni al gruppo.
    \item \textbf{Calendar Google}: Per la pianificazione delle riunioni e delle scadenze.
    \item \textbf{Telegram e Gmail}: Per la comunicazione con il proponente e per inviare documenti ufficiali. 
\end{itemize}

\subsubsection{Comunicazione e Organizzazione}
Avvalendosi degli strumenti sopra elencati, il team BugBusters adotta una politica di comunicazione 
chiara e un'organizzazione strutturata per garantire l'efficacia delle attività di fornitura. 
Le riunioni interne si tengono settimanalmente il giovedi alle 14:30 per fare il punto sullo 
stato di avanzamento, assegnare le attività e pianificare la settimana successiva; i verbali 
e le attività risultanti vengono registrati e tracciati su GitHub (issues e project board). 
Gli aggiornamenti con il proponente avvengono inizialmente con cadenza settimanale il mercoledì alle 15:00 e successivamente, a progetto avviato, bisettimanale.
I relativi verbali vengono poi condivisi nel repository della documentazione per garantire trasparenza e reperibilità.
Per le comunicazioni operative asincrone il team utilizza Discord, WhatsApp e Telegram, mentre 
tutte le decisioni rilevanti devono essere formalmente riportate e documentate su GitHub e 
nella documentazione ufficiale. Per ogni ambito di lavoro è nominato un referente responsabile 
che assicura il monitoraggio delle attività e la responsabilità decisionale; eventuali problemi 
critici vengono segnalati immediatamente sul canale concordato e aperti come issue su 
GitHub, in modo da garantirne la tracciabilità e la tempestività nelle azioni correttive.

\subsubsection{Documentazione Prodotta}
Durante la fase di fornitura, il team BugBusters produce e mantiene aggiornata la seguente documentazione:


\subsubsubsection{Glossario}
Per facilitare la lettura e la comprensione dei documenti di progetto, viene redatto un glossario che raccoglie
e definisce i termini tecnici, gli acronimi e le abbreviazioni utilizzati. Questo strumento è fondamentale per
garantire una comunicazione chiara e univoca tra tutti i membri del team, con il proponente, docenti e con i lettori esterni.
Per una consultazione rapida durante la visualizzazione dei documenti, il glossario è accessibile anche tramite il 
viewer PDF sviluppato dal team.

\begin{center}
\begin{tabularx}{\textwidth}{|>{\centering\arraybackslash}p{3.2cm}|>{\centering\arraybackslash}X|}
\hline
\rowcolor{primaryblue!40}
\multicolumn{1}{|>{\centering\arraybackslash}p{3.2cm}|}{\color{white}\textbf{Campo}} & \multicolumn{1}{>{\centering\arraybackslash}X|}{\color{white}\textbf{Dettaglio}} \\
\hline
\textbf{Redattore} & Amministratore \\
\hline
\textbf{Destinatari} & BugBusters, Eggon, Prof. Vardanega, Prof. Cardin  \\
\hline
\textbf{Uso} & Interno ed Esterno \\
\hline
\end{tabularx}
\end{center}

\subsubsubsection{Lettera di Candidatura}
La Lettera di Candidatura formalizza l'interesse e le motivazioni del team BugBusters a partecipare al progetto Nexum 
proposto dall'azienda Eggon.

\begin{center}
\begin{tabularx}{\textwidth}{|>{\centering\arraybackslash}p{3.2cm}|>{\centering\arraybackslash}X|}
\hline
\rowcolor{primaryblue!40}
\multicolumn{1}{|>{\centering\arraybackslash}p{3.2cm}|}{\color{white}\textbf{Campo}} & \multicolumn{1}{>{\centering\arraybackslash}X|}{\color{white}\textbf{Dettaglio}} \\
\hline
\textbf{Redattore} & Responsabile \\
\hline
\textbf{Destinatari} & BugBusters, Eggon, Prof. Vardanega, Prof. Cardin  \\
\hline
\textbf{Uso} & Esterno \\
\hline
\end{tabularx}
\end{center}


\subsubsubsection{Dichiarazione degli impegni}
La Dichiarazione degli Impegni definisce i ruoli, il monte ore previso e il costo orario di ciascun membro del team BugBusters
per la realizzazione del progetto Nexum e la data di consegna prevista, impegnandosi a rispettare tali condizioni durante l'intero ciclo di vita del progetto.

\begin{center}
\begin{tabularx}{\textwidth}{|>{\centering\arraybackslash}p{3.2cm}|>{\centering\arraybackslash}X|}
\hline
\rowcolor{primaryblue!40}
\multicolumn{1}{|>{\centering\arraybackslash}p{3.2cm}|}{\color{white}\textbf{Campo}} & \multicolumn{1}{>{\centering\arraybackslash}X|}{\color{white}\textbf{Dettaglio}} \\
\hline
\textbf{Redattore} & Responsabile \\
\hline
\textbf{Destinatari} & BugBusters, Eggon, Prof. Vardanega, Prof. Cardin  \\
\hline
\textbf{Uso} & Esterno \\
\hline
\end{tabularx}
\end{center}

\subsubsubsection{Valutazione dei capitolati}
Con questo documento si intende valutare i vari capitolati d'appalto proposti per il progetto di Ingegneria del Software,
analizzandone punti di forza, debolezze e opportunità in modo da scegliere il capitolato più adatto alle competenze e 
agli interessi del team BugBusters.
Per ogni capitolato vengono esaminati vari aspetti:
\begin{itemize}
    \item Descrizione breve
    \item Caratteristiche funzionali
    \item Tecnologie proposte
    \item Chiarimenti e colloqui con l'azienda
    \item Interesse del team
    \item Punti di forza e debolezza
\end{itemize}

\begin{center}
\begin{tabularx}{\textwidth}{|>{\centering\arraybackslash}p{3.2cm}|>{\centering\arraybackslash}X|}
\hline
\rowcolor{primaryblue!40}
\multicolumn{1}{|>{\centering\arraybackslash}p{3.2cm}|}{\color{white}\textbf{Campo}} & \multicolumn{1}{>{\centering\arraybackslash}X|}{\color{white}\textbf{Dettaglio}} \\
\hline
\textbf{Redattore} & Responsabile \\
\hline
\textbf{Destinatari} & BugBusters, Prof. Vardanega, Prof. Cardin  \\
\hline
\textbf{Uso} & Esterno \\
\hline
\end{tabularx}
\end{center}

\subsubsubsection{Analisi dei Requisiti}
Il documento di Analisi dei Requisiti descrive in dettaglio le funzionalità, i vincoli e le proprietà di qualità che il sistema Nexum dovrà soddisfare. Questo capitolo fornisce:
\begin{itemize}
    \item la definizione del contesto, degli stakeholder e degli attori coinvolti;
    \item l'elenco dei requisiti funzionali e non funzionali, classificati e dotati di codifica univoca e criteri di accettazione;
    \item casi d'uso e scenari principali con flussi e attori associati;
    \item vincoli tecnici, normativi e di integrazione con sistemi esterni;
    \item la prioritizzazione dei requisiti e l'identificazione del Minimum Viable Product (MVP);
    \item la strategia di tracciabilità e gestione delle modifiche (issue tracker, versionamento e mappatura requisiti–test);
    \item i criteri e i metodi per la verifica e la validazione dei requisiti.
\end{itemize}

\begin{center}
\begin{tabularx}{\textwidth}{|>{\centering\arraybackslash}p{3.2cm}|>{\centering\arraybackslash}X|}
\hline
\rowcolor{primaryblue!40}
\multicolumn{1}{|>{\centering\arraybackslash}p{3.2cm}|}{\color{white}\textbf{Campo}} & \multicolumn{1}{>{\centering\arraybackslash}X|}{\color{white}\textbf{Dettaglio}} \\
\hline
\textbf{Redattore} & Analista \\
\hline
\textbf{Destinatari} & Eggon, Prof. Vardanega, Prof. Cardin  \\
\hline
\textbf{Uso} & Esterno \\
\hline
\end{tabularx}
\end{center}

\subsubsubsection{Norme di Progetto}
Il documento delle Norme di Progetto definisce le metodologie, gli standard e i processi che 
il team BugBusters adotterà durante lo sviluppo del progetto Nexum. 

\begin{center}
\begin{tabularx}{\textwidth}{|>{\centering\arraybackslash}p{3.2cm}|>{\centering\arraybackslash}X|}
\hline
\rowcolor{primaryblue!40}
\multicolumn{1}{|>{\centering\arraybackslash}p{3.2cm}|}{\color{white}\textbf{Campo}} & \multicolumn{1}{>{\centering\arraybackslash}X|}{\color{white}\textbf{Dettaglio}} \\
\hline
\textbf{Redattore} & Amministratore \\
\hline
\textbf{Destinatari} & BugBusters, Prof. Vardanega, Prof. Cardin  \\
\hline
\textbf{Uso} & Interno \\
\hline
\end{tabularx}
\end{center}

\subsubsubsection{Verbali}
I verbali sono documenti di lavoro indispensabili per tracciare le decisioni, le discussioni e le 
azioni concordate durante le riunioni del team BugBusters e con il proponente Eggon.
Si dividono in verbali interni, prodotti durante le riunioni del team, e verbali esterni, 
redatti dopo gli incontri con il proponente.

\begin{center}
\begin{tabularx}{\textwidth}{|>{\centering\arraybackslash}p{3.2cm}|>{\centering\arraybackslash}X|>{\centering\arraybackslash}X|}
\hline
\rowcolor{primaryblue!40}
\multicolumn{1}{|>{\centering\arraybackslash}p{3.2cm}|}{\color{white}\textbf{Campo}} & \multicolumn{1}{>{\centering\arraybackslash}X|}{\color{white}\textbf{Dettaglio interni}} & \multicolumn{1}{>{\centering\arraybackslash}X|}{\color{white}\textbf{Dettaglio esterni}} \\
\hline
\textbf{Redattore} & Amministratore & Amministratore \\
\hline
\textbf{Destinatari} & BugBusters, Prof. Vardanega, Prof. Cardin & BugBusters, Eggon, Prof. Vardanega, Prof. Cardin \\
\hline
\textbf{Uso} & Interno & Esterno \\
\hline
\end{tabularx}
\end{center}

\subsubsubsection{Piano di Progetto}
DA FARE
\subsubsubsection{Piano di Qualifica}
DA FARE

\subsection{Processi di sviluppo}

\end{document}