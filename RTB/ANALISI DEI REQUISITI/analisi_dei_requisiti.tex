\documentclass[a4paper,11pt]{article}

\usepackage[utf8]{inputenc}
\usepackage[T1]{fontenc}
\usepackage[italian]{babel}
\usepackage[margin=2.5cm]{geometry}
\usepackage{graphicx}
\usepackage{booktabs}
\usepackage{setspace}
\usepackage{titlesec}
\usepackage{float}
\usepackage[table]{xcolor}
\usepackage{tabularx}
\usepackage{tcolorbox}
\usepackage{enumitem}
\usepackage[titles]{tocloft}
\usepackage[colorlinks=true,linkcolor=black,urlcolor=blue,citecolor=blue]{hyperref}
\usepackage{fancyhdr}
\usepackage{lastpage}
\usepackage{amsmath}

\pagestyle{fancy}
\fancyhf{}
\fancyhead[L]{BugBusters}
\fancyhead[R]{Analisi dei Requisiti}
\fancyfoot[L]{\thepage\ di \pageref{LastPage}}
\renewcommand{\headrulewidth}{0pt}
\renewcommand{\footrulewidth}{0pt}

\setlength{\headheight}{14pt}

\setlength{\parskip}{4pt}
\setlength{\parindent}{0pt}

\titleformat{\section}{\large\bfseries}{\thesection}{1em}{}
\titleformat{\subsection}{\normalsize\bfseries}{\thesubsection}{1em}{}

\begin{document}

\begin{center}
  \thispagestyle{empty}
  \IfFileExists{../../assets/Logo.jpg}{%
    \includegraphics[width=6cm,height=3cm,keepaspectratio]{../../assets/Logo.jpg} \\[0.8cm]
  }{%
    \fbox{\parbox[c][2.5cm][c]{6cm}{\centering Logo non trovato\\(Logo.jpg)}}\\[0.5cm]
  }
  {\LARGE\bfseries BugBusters}\\[0.8cm]
  
  \rule{\textwidth}{0.5pt}\\[0.5cm]
  {\Large\bfseries Analisi dei Requisiti}\\[0.3cm]
  {\large Versione 0.0.1}\\[0.5cm]
  \rule{\textwidth}{0.5pt}\\[0.8cm]
\end{center}

\begin{center}
\begin{tcolorbox}[colback=gray!10,width=0.8\textwidth,arc=3mm,boxrule=0.5pt]
\begin{tabular}{ll}
\textbf{Stato} & In redazione \\
\textbf{Redattori} & ----- \\
\textbf{Destinatari} & BugBusters \\
 & Prof. Vardanega Tullio \\
 & Prof. Cardin Riccardo \\
 & Eggon \\
\end{tabular}
\end{tcolorbox}
\end{center}

\vspace{1cm}

\begin{center}
\textbf{Descrizione}
\end{center}

\begin{center}
\begin{minipage}{0.9\textwidth}
\small
Questo documento contiene le Norme di Progetto seguite dal team \textbf{BugBusters} per il progetto\textsubscript{\scalebox{0.6}{\textbf{G}}} C5 proposto dall'azienda Eggon
\end{minipage}
\end{center}



\newpage

\section*{Registro delle Modifiche}

{\footnotesize
\begin{center}
\begin{tabularx}{\textwidth}{|l|l|X|l|l|}
\hline
\textbf{Versione} & \textbf{Data} & \textbf{Descrizione} & \textbf{Autore} & \textbf{Ruolo} \\
\hline
0.0.12 & 12/12/2025 & 
\begin{minipage}[t]{\linewidth}
Aggiunta sotto-casi d'uso alla sezione 2\end{minipage} 
& Leonardo Salviato & - \\
\hline
0.0.11 & 11/12/2025 & 
\begin{minipage}[t]{\linewidth}
Aggiunto modulo 3\end{minipage} 
& Marco Piro & - \\
\hline
0.0.10 & 10/12/2025 & 
\begin{minipage}[t]{\linewidth}
Rinominazione casi d'uso, aggiunta sotto-casi d'uso alle sezioni 0 e 1.
\end{minipage} 
& Leonardo Salviato & - \\
\hline
0.0.9 & 06/12/2025 & 
\begin{minipage}[t]{\linewidth}
Aggiunti a ogni sezione scenari secondari, relativi casi d'uso (eccezioni e varianti) e trigger.
\end{minipage} 
& Leonardo Salviato & - \\
\hline
0.0.8 & 05/12/2025 & 
\begin{minipage}[t]{\linewidth}
Riscrittura casi d'uso con aggiornato grado di precisione (sezione 2).
\end{minipage} 
& Leonardo Salviato & - \\
\hline
0.0.7 & 04/12/2025 & 
\begin{minipage}[t]{\linewidth}
Riscrittura casi d'uso con aggiornato grado di precisione (sezioni 0 e 1).
\end{minipage} 
& Leonardo Salviato & - \\
\hline
0.0.6 & 03/12/2025 & 
\begin{minipage}[t]{\linewidth}
Aggiunti casi d'uso sezione 2, aggiunta varianti/exceptions sezione 2, scritta possibile struttura requisiti.
\end{minipage} 
& Leonardo Salviato & - \\
\hline
0.0.5 & 02/12/2025 & 
\begin{minipage}[t]{\linewidth}
Aggiunto schema attori
\end{minipage} 
& Marco Piro & - \\
\hline
0.0.4 & 30/11/2025 & 
\begin{minipage}[t]{\linewidth}
    Sistemazione Attori.
\end{minipage} 
& Marco Piro & - \\
\hline
0.0.3 & 29/11/2025 & 
\begin{minipage}[t]{\linewidth}
Correzione casi d'uso e aggiunta schemi.
\end{minipage} 
& Leonardo Salviato & - \\
\hline
0.0.2 & 25/11/2025 & 
\begin{minipage}[t]{\linewidth}
Riscrittura della prima stesura e modifica casi d'uso.
\end{minipage} 
& Leonardo Salviato & - \\
\hline
0.0.1 & 16/11/2025 & 
\begin{minipage}[t]{\linewidth}
Prima stesura della struttura del documento.
\end{minipage} 
& Leonardo Salviato e Marco Piro & - \\
\hline
\end{tabularx}
\end{center}
}

\vfill
\begin{center}
2 di \pageref{LastPage}
\end{center}

\newpage

\section*{Indice}

\noindent
\begin{minipage}[t]{0.8\textwidth}
\subsection*{1 Introduzione}
1.1 Scopo del documento \\
1.2 Prospettiva del prodotto \\
1.3 Funzioni del prodotto \\
1.4 Caratterisitiche dell'utente \\
1.5 Definizioni, acronimi e abbreviazioni \\
1.6 Riferimenti \\
\quad 1.6.1 Riferimenti normativi \\
\quad 1.6.2 Riferimenti informativi \\
\subsection*{2 Casi d'uso}
2.1 Introduzione \\
2.2 Attori \\
2.3 Lista casi d'uso \\
2.4 Sezione 0 – Applicazione standalone\\
\quad 2.4.1 UC-0A: Registrazione nuovo utente \\
\quad 2.4.2 UC-0B: Login / Autenticazione utente \\
\quad 2.4.3 UC-0C: Apertura dashboard \\
\quad 2.4.4 UC-0D: Apertura gestione profilo utente \\
\quad 2.4.5 UC-0E: Apertura gestione ruoli (Admin / Editor)\\
\quad 2.4.6 UC-0F: Logout \\
2.5 Sezione 1 – Modulo “AI Assistant Generativo\\
\quad 2.5.1 UC-1A: Apertura modulo di generazione contenuti AI\\
\quad 2.5.2 UC-1B: Apertura storico prompt\\
\quad 2.5.3 UC-1C: Generazione contenuto tramite AI\\
\quad 2.5.4 UC-1D: Apertura modifica contenuto generato\\

2.6 Sezione 2 – Modulo "AI Co-Pilot per i CdL"\\
\quad 2.6.1 UC-2A: Apertura modulo di upload e gestione documentale\\
\quad 2.6.2 UC-2B: Apertura lista documenti\\
\quad 2.6.3 UC-2C: Apertura documento\\
\quad 2.6.4 UC-2D: Apertura informazioni destinatario\\
\quad 2.6.5 UC-2E: Apertura storico documenti\\
\quad 2.6.6 UC-2F: Gestione messaggio\\
\quad 2.6.7 UC-2G: Tabella template salvati\\
\quad 2.6.8 UC-2H: Invio documento e messaggio\\


\end{minipage}
\begin{minipage}[t]{0.2\textwidth}
\vspace{1.65\baselineskip}
9 \\
9 \\
9 \\
10 \\
10 \\
10 \\
\end{minipage}

\newpage

\section{Introduzione}

\subsection{Scopo del documento}
Questo documento di Analisi dei Requisiti\textsubscript{\scalebox{0.6}{\textbf{G}}}, adottato da parte di BugBusters durante lo svolgimento del progetto\textsubscript{\scalebox{0.6}{\textbf{G}}} didattico, ha lo scopo di definire in maniera precisa e dettagliata i requisiti funzionali\textsubscript{\scalebox{0.6}{\textbf{G}}} e non funzionali del Sistema software da sviluppare.

A seguito delle nuove decisioni progettuali rispetto alle proposte del capitolato, il Sistema non sarà inizialmente integrato nella piattaforma NEXUM, ma verrà realizzato come \textbf{applicazione standalone}, autonoma e indipendente. Tale applicazione implementerà i moduli “AI Assistant Generativo” e "AI Co-Pilot per i CdL" in un ambiente isolato, così da consentire una fase di sviluppo, test e validazione più controllata. Solo in una fase successiva si valuterà l’\textbf{integrazione con la piattaforma NEXUM}, garantendo continuità architetturale e coerenza con i moduli già presenti.

Il documento include una descrizione approfondita dei Casi d’Uso, che costituiscono la principale fonte dei requisiti finali. Per agevolare la comprensione, verranno utilizzati anche i \textbf{Diagrammi dei Casi d’Uso}, che visualizzano le interazioni tra utenti e Sistema.

Questo documento rappresenta il riferimento fondamentale per la progettazione, l’implementazione e il collaudo dell’applicazione standalone, assicurando che essa soddisfi pienamente le esigenze del Committente e gli obiettivi formativi del progetto.

I requisiti identificati sono classificati nelle seguenti categorie:
\begin{itemize}
    \item \textbf{Obbligatori}: necessari e imprescindibili per garantire il corretto funzionamento dell’applicazione standalone;
    \item \textbf{Desiderabili}: non strettamente necessari, ma capaci di migliorare l’esperienza utente o l’efficienza del Sistema;
    \item \textbf{Opzionali}: funzionalità aggiuntive utili per estensioni future, in particolare in vista della possibile integrazione con NEXUM.
\end{itemize}

Il documento è rivolto ai seguenti destinatari:
\begin{itemize}
    \item Il \textbf{Committente}, che potrà verificare che i requisiti siano stati compresi e documentati correttamente;
    \item Il \textbf{Team di Progettisti e Programmatori}, che utilizzerà questa analisi come base per la realizzazione del Sistema;
    \item Il \textbf{Team di Verificatori}, che impiegherà il presente documento per definire i casi di Test e validare il comportamento del prodotto.
\end{itemize}

\subsection{Prospettiva del prodotto}

Il prodotto che BugBusters si propone di sviluppare è una versione standalone dei moduli “AI Assistant Generativo” e "AI Co-Pilot per i CdL", inizialmente svincolata dalla piattaforma NEXUM. Tale applicazione costituirà un prototipo funzionale in grado di operare autonomamente e di implementare le principali funzionalità richieste dal Committente, senza dipendere dagli altri moduli della piattaforma.

L’app standalone permetterà di testare e consolidare le funzionalità richieste, offrendo un ambiente controllato che faciliti la sperimentazione e lo sviluppo incrementale. Questa fase costituirà la base per un’eventuale integrazione futura con la piattaforma NEXUM, la quale fornirà un ecosistema HR completo e dotato di servizi quali la messaggistica top-down, la timbratura digitale, la gestione delle anagrafiche e dei ruoli, e la collaborazione con gli studi dei Consulenti del Lavoro (CdL).

L’integrazione futura con NEXUM sarà concepita in modo modulare, consentendo alla nuova applicazione di inserirsi nell’architettura esistente come componente riutilizzabile e scalabile. L’integrazione includerà l’adattamento delle API, l’allineamento della gestione utenti e la centralizzazione dei dati all’interno dell’infrastruttura NEXUM.

\subsection{Funzioni del prodotto}

Dal punto di vista dell’utilizzatore finale, l’applicazione standalone dovrà fornire le seguenti funzionalità:

\begin{itemize}
    \item \textbf{Generazione di contenuti tramite AI (Modulo AI Assistant Generativo)}: 
    generazione di titolo, testo e immagine di copertina a partire da un prompt, 
    con possibilità di selezionare tono, stile e configurazioni avanzate del modello AI.

    \item \textbf{Salvataggio locale}: 
    gestione interna di prompt, contenuti generati, immagini e valutazioni, 
    tramite archivio locale dedicato all’app standalone, indipendente dalla piattaforma NEXUM.

    \item \textbf{Sistema di rating}: 
    valutazione della qualità dei contenuti generati dall’AI, utile per analisi interne e miglioramento continuo.

    \item \textbf{Gestione dei prompt}: 
    storico dei prompt utilizzati, con possibilità di riutilizzo, duplicazione e ricerca interna.

    \item \textbf{Dashboard standalone}: 
    visualizzazione e gestione di storico, filtri, ricerca e analisi delle interazioni con l’AI generativa.

    \item \textbf{Gestione delle immagini}: 
    possibilità di caricare immagini dall’utente o di generarle tramite AI, con salvataggio locale.

    \item \textbf{Gestione utenti}: 
    registrazione, autenticazione, gestione del profilo e configurazione dei parametri AI 
    (per utenti privilegiati come amministratori o editor avanzati).

    \item \textbf{Upload e gestione documentale (Modulo AI Co-Pilot per i CdL)}: 
    possibilità di caricare documenti (PDF, ZIP, etc), salvarli localmente e gestirne lo stato di elaborazione.

    \item \textbf{Riconoscimento automatico della tipologia di documento}: 
    classificazione tramite AI (cedolini, CU, comunicazioni, lettere, moduli da firmare, ecc.) 
    sfruttando modelli OCR e classificatori addestrati.

    \item \textbf{Estrazione dei destinatari}: 
    riconoscimento automatico di informazioni contenute nei documenti 
    (nome, cognome, codice fiscale, matricola, reparto) tramite tecniche AI di entity extraction.

    \item \textbf{Split dei documenti massivi}: 
    suddivisione automatica dei documenti multi-destinatario (es. cedolini massivi) 
    in documenti singoli, ognuno associato al proprio destinatario riconosciuto.

    \item \textbf{Revisione manuale (Human-in-the-Loop)}: 
    interfaccia dedicata per verificare, correggere o confermare i risultati ottenuti dall’AI 
    in ogni fase (classificazione, destinatari, split).

    \item \textbf{Creazione di messaggi e liste di distribuzione}: 
    generazione automatica di bozze di messaggi e liste di destinatari derivanti dai documenti processati.

    \item \textbf{Tracciamento locale}: 
    storico delle operazioni effettuate (upload, riconoscimento, revisioni, esportazioni), 
    utile per audit interni e analisi del flusso documentale.

\end{itemize}

Queste funzionalità permetteranno all’app standalone di essere completamente operativa e autonoma nei due moduli (AI Assistant Generativo e AI Co-Pilot per i CdL). 
In una fase successiva, tali componenti saranno progettati per essere integrati nella piattaforma NEXUM, 
consentendo così un’evoluzione verso un ecosistema HR completo, scalabile e basato su automazioni intelligenti.

\subsection{Caratterisitiche dell'utente}

Gli utilizzatori finali dell'applicazione standalone non appartengono a un’unica categoria specifica: 
l’obiettivo del progetto è quello di progettare moduli intelligenti e interoperabili 
che possano essere integrati all’interno dell’ecosistema NEXUM o utilizzati autonomamente durante la fase standalone.

In generale, è possibile affermare che gli utenti finali sono coloro che necessitano di uno strumento scalabile, 
intelligente e semplice da utilizzare per generare contenuti tramite AI e per gestire flussi documentali complessi con il supporto del modulo Co-Pilot.
Rientrano in questa categoria:

\begin{itemize}
    \item \textbf{Responsabili e amministratori HR}, che necessitano di strumenti avanzati per la creazione di comunicazioni interne, 
    la gestione dei contenuti generativi e l’analisi delle produzioni.

    \item \textbf{Consulenti del Lavoro (CdL) e personale amministrativo}, che richiedono un sistema in grado di caricare, riconoscere, suddividere e preparare documenti per la distribuzione ai destinatari.

    \item \textbf{Dipendenti e collaboratori} (in fase integrata), che potranno interagire con la piattaforma NEXUM per consultare documenti e comunicazioni, 
    pur non essendo utenti della versione standalone.

    \item \textbf{Manager aziendali}, interessati a monitorare la consistenza delle comunicazioni e l’efficienza dei processi documentali, 
    sia nella versione standalone che nella futura integrazione.
\end{itemize}

In sintesi, il prodotto è rivolto a organizzazioni di varie dimensioni — in particolare aziende medio-grandi e studi professionali — 
che necessitano di strumenti intelligenti per la creazione di contenuti, 
la gestione automatizzata dei documenti e la collaborazione con gli studi dei Consulenti del Lavoro.
L’app standalone funge da primo passo verso una piattaforma HR completa, modulare e potenziata dall’AI.


\subsection{Definizioni, acronimi e abbreviazioni}
Per tutte le definizioni, acronimi e abbreviazioni utilizzati in questo documento, si faccia
riferimento al \textbf{Glossario}, fornito come documento separato, che contiene tutte le spiegazioni
necessarie per garantire una comprensione uniforme dei termini tecnici e dei concetti
rilevanti per il progetto.

\newpage

\subsection{Riferimenti}

\subsubsection{Riferimenti normativi}
\begin{itemize}
\item \textbf{Capitolato\textsubscript{\scalebox{0.6}{\textbf{G}}}
 d'appalto C5: Nexum - Piattaforma di consulenza e documentazione previdenziale}\\
\url{https://www.math.unipd.it/~tullio/IS-1/2025/Progetto/C5.pdf}
\end{itemize}

\subsubsection{Riferimenti informativi}
\begin{itemize}
\item \textbf{Glossario\textsubscript{\scalebox{0.6}{\textbf{G}}}
:}\\
\url{https://github.com/BugBustersUnipd/DocumentazioneSWE/blob/main//RTB/GLOSSARIO/Glossario.pdf}
\end{itemize}



\section{Casi d'uso}
\subsection{Introduzione}
I casi d’uso si compongono di un grafico UML e una descrizione testuale che permetta di
comprendere al meglio cosa il prodotto deve fornire. La descrizione testuale, in particolar
modo, dovrà contenere le informazioni sotto presenti, salvo i casi in cui lo
specifico campo non risulti rilevante (ad esempio, un Caso d’Uso\textsubscript{\scalebox{0.6}{\textbf{G}}} che non prevede la
possibilità di errori non avrà Scenari secondari):

\begin{itemize}
    \item \textbf{Attori}: Sono coloro che interagiscono attivamente con il Sistema e
    svolgono l’azione indicata dal Caso d’Uso
    \item \textbf{Precondizioni}: Lista di elementi che sono necessari affinché l’Attore possa
    compiere l’azione indicata dal caso d’uso
    \item \textbf{Postcondizioni}: Lista di elementi che descrivono come il Sistema risulta
    essere internamente cambiato dopo che l’Attore ha effettuato
    l’azione prevista dal Caso d’Uso
    \item \textbf{Scenario principale}: Descrizione ragionevole delle operazioni che l’attore deve
    fare per compiere l’azione descritta dal Caso d’Uso
    \item \textbf{Scenario secondario}: Descrizione ragionevole degli eventi che possono accadere
    qualora una delle operazioni descritte nello Scenario
    principale non vada a buon fine
    \item \textbf{Trigger}: Evento che innesca l’inizio del Caso d’Uso
    \item \textbf{Inclusioni}: Casi d’Uso ulteriori che l’Attore deve compiere per realizzare
    il Caso d’Uso attualmente descritto
    \item \textbf{Estensioni}: Casi d’Uso ulteriori che possono realizzarsi durante
    l’esecuzione delle operazioni del Caso d’Uso principale
    
\end{itemize}
Motivazioni che portano l’Attore a svolgere l’azione descritta
dal Caso d’Uso. Non sempre disponibile in quanto il Caso
d’Uso potrebbe essere incluso da un altro caso d’uso «principale».

\subsection{Attori}
Di seguito sono esposti gli attori utilizzati:
\begin{figure}[H]
    \centering
    \includegraphics[width=0.8\textwidth]{Diagrammi casi d'uso/diagramma_attori.jpg}
    \caption{Diagramma degli attori principali}
\end{figure}
\begin{itemize}
    \item \textbf{Utente}: Rappresenta un utente che vuole accedere al Sistema.
    \item \textbf{HR Manager}: È la figura responsabile della comunicazione interna. Utilizza il modulo AI Assistant per generare, revisionare e pubblicare messaggi o avvisi rivolti ai dipendenti, definendone tono e stile.
    \item Redattore: 
    \item \textbf{Data Analyst}: Figura incaricata di monitorare le prestazioni. Accede alle dashboard di analisi per consultare le statistiche di utilizzo, i rating di qualità dei contenuti generati e i KPI del riconoscimento documentale.
    \item \textbf{Amministratore}: Gestisce la configurazione tecnica dell'applicazione standalone. Si occupa della creazione degli utenti, della gestione dei ruoli e della configurazione dei parametri globali dell'AI (es. prompt di sistema o soglie di confidenza).
    \item \textbf{Operatore Studio CdL}: È l'utente principale del modulo AI Co-Pilot. Si occupa di caricare i flussi documentali (es. cedolini massivi), supervisionare il riconoscimento automatico (validazione Human-in-the-Loop) e gestire le liste di distribuzione.
    \item \textbf{Sistema NEXUM (AI Doc Classifier)}: Il modulo intelligente incaricato di analizzare visivamente il documento, applicare l'OCR e classificarne la tipologia (es. "Cedolino", "CUD").
    \item \textbf{Sistema NEXUM (Entity Resolver)}: Il componente che analizza il testo estratto per identificare univocamente i destinatari (es. Nome, Cognome, CF) confrontandoli con l'anagrafica.
    \item \textbf{Sistema NEXUM (Splitter)}: L'agente automatico che scansiona i documenti massivi (es. PDF multipagina) e li suddivide in singoli file, uno per ciascun destinatario individuato.
    \item \textbf{Sistema NEXUM (Dispatcher \& Tracking)}: Il modulo responsabile della creazione dei pacchetti di invio e della generazione delle ricevute di consegna (simulata in ambiente standalone).
    \item \textbf{Destinatario finale}: Rappresenta il dipendente a cui sono indirizzati i documenti o i messaggi. Nell'applicazione standalone, la sua interazione (ricezione e lettura) è simulata per verificare il corretto funzionamento del dispaccio.
    \item \textbf{Auditor interno}: Utente con permessi di sola lettura focalizzato sul controllo. Verifica lo storico delle operazioni (audit trail) per garantire la tracciabilità e la sicurezza dei flussi documentali.
    \item \textbf{Sistema}: L'applicazione standalone nel suo complesso, che gestisce autenticazione, database e interfaccia.
    \item \textbf{Admin Cliente, Admin Eggon}: Figure di alto livello responsabili, rispettivamente, della gestione dell'organizzazione cliente e della supervisione tecnica del progetto per conto di Eggon.
\end{itemize}


\subsection{Lista casi d'uso}

\text{L'elenco dei casi d'uso sará diviso in tre parti:}
\begin{itemize}
    \item 0 - Casi d'uso per la gestione utenti e autenticazione
    \item 1 - Casi d'uso per il modulo "AI Assistant Generativo"
    \item 2 - Casi d'uso per il modulo "AI Doc Classifier"
\end{itemize}


\subsection{Sezione 0 – Applicazione standalone}

\subsubsection{UC-0A – Registrazione nuovo utente}

\begin{figure}[H]
    \centering
    \includegraphics[width=1\textwidth]{Diagrammi casi d'uso/UC0A.jpg}
    \caption{Didascalia dell'immagine}
\end{figure}


\textbf{Attori}
\begin{itemize}
    \item Utente non autenticato (nuovo utente).
    \item Sistema di autenticazione dell’applicazione standalone.
\end{itemize}

\textbf{Pre-condizioni}
\begin{itemize}
    \item L’utente non ha una sessione attiva.
    \item L’utente non è ancora registrato nel Sistema (l’e-mail inserita non risulta già presente).
\end{itemize}

\textbf{Post-condizioni}
\begin{itemize}
    \item Esiste un nuovo account utente registrato nel Sistema.
    \item L’utente può effettuare il login utilizzando le credenziali appena create.
\end{itemize}

\textbf{Scenario principale}
\begin{enumerate}
    \item L’utente accede alla schermata di registrazione dell’applicazione standalone.
    \item L’utente inserisce i dati richiesti (ad esempio: nome, cognome, e-mail, password).
    \item Il Sistema verifica la correttezza formale dei dati inseriti (es. formato e-mail, forza della password).
    \item Il Sistema controlla che l’indirizzo e-mail non sia già associato a un account esistente.
    \item In caso di esito positivo, il Sistema crea un nuovo account utente e lo memorizza nel proprio archivio.
    \item Il Sistema conferma l’avvenuta registrazione e può opzionalmente eseguire il login automatico del nuovo utente.
\end{enumerate}

\textbf{Scenario secondario}
\begin{enumerate}
    \item Nell'inserimento dei dati, uno o più campi non rispettano i requisiti di validità (es. e-mail non valida, password debole).
    \item Il Sistema mostra un messaggio di errore specifico per il campo non valido.
    \item L’utente corregge i dati e ripete l’inserimento.
\end{enumerate}

\textbf{Trigger}
\begin{itemize}
    \item Avviare l'applicativo standalone e selezionare l'opzione di registrazione.
\end{itemize}

\textbf{Relazioni con altri casi d'uso (\textit{include} / \textit{extend})}
\begin{itemize}
    \item \textit{include}:
    \begin{itemize}
        \item UC-0A.1 - Inserimento email
        \item UC-0A.2 - Inserimento password
        \item UC-0A.3 - Inserimento username
        \item UC-0A.4 - Inserimento nome 
        \item UC-0A.5 - Inserimento cognome
        \item UC-0A.6 - Inserimento matricola
    \end{itemize}
    \item \textit{extend}: 
    \begin{itemize}
        \item UC-0B – Login / Autenticazione utente (in caso di login automatico al termine della registrazione).
        \item UC-0A.7 – email non valida.
        \item UC-0A.8 – password non valida.
        \item UC-0A.9 – email già registrata.
        \item UC-0A.10 – username già registrato.
        \item UC-0A.11 – matricola già registrata.
        \item UC-0A.12 – matricola non valida.
    \end{itemize}
\end{itemize}

\vspace{0.5cm}

\subsubsection{UC-0A.1 – Inserimento email}

\textbf{Attori}
\begin{itemize}
    \item utente
\end{itemize}

\textbf{Pre-condizioni}
\begin{itemize}
    \item Casella di testo vuota
\end{itemize}

\textbf{Post-condizioni}
\begin{itemize}
    \item Casella di testo contenente l'email inserita
\end{itemize}

\textbf{Scenario principale}
\begin{enumerate}
    \item L'utente durante la registrazione inserisce la propria email nell'apposita casella di testo
\end{enumerate}

\textbf{Trigger}
\begin{itemize}
    \item L'utente clicca sulla casella di testo per l'inserimento dell'email
\end{itemize}

\vspace{0.5cm}

\subsubsection{UC-0A.2 – Inserimento password}

\textbf{Attori}
\begin{itemize}
    \item utente
\end{itemize}

\textbf{Pre-condizioni}
\begin{itemize}
    \item Casella di testo vuota
\end{itemize}

\textbf{Post-condizioni}
\begin{itemize}
    \item Casella di testo contenente la password inserita
\end{itemize}

\textbf{Scenario principale}
\begin{enumerate}
    \item L'utente durante la registrazione inserisce la propria password nell'apposita casella di testo
\end{enumerate}

\textbf{Trigger}
\begin{itemize}
    \item L'utente clicca sulla casella di testo per l'inserimento della password
\end{itemize}

\vspace{0.5cm}

\subsubsection{UC-0A.3 – Inserimento username}

\textbf{Attori}
\begin{itemize}
    \item utente
\end{itemize}

\textbf{Pre-condizioni}
\begin{itemize}
    \item Casella di testo vuota
\end{itemize}

\textbf{Post-condizioni}
\begin{itemize}
    \item Casella di testo contenente l'username inserito
\end{itemize}

\textbf{Scenario principale}
\begin{enumerate}
    \item L'utente durante la registrazione inserisce il proprio username nell'apposita casella di testo
\end{enumerate}

\textbf{Trigger}
\begin{itemize}
    \item L'utente clicca sulla casella di testo per l'inserimento dell'username
\end{itemize}

\vspace{0.5cm}

\subsubsection{UC-0A.4 – Inserimento nome}

\textbf{Attori}
\begin{itemize}
    \item utente
\end{itemize}

\textbf{Pre-condizioni}
\begin{itemize}
    \item Casella di testo vuota
\end{itemize}

\textbf{Post-condizioni}
\begin{itemize}
    \item Casella di testo contenente il nome inserito
\end{itemize}

\textbf{Scenario principale}
\begin{enumerate}
    \item L'utente durante la registrazione inserisce il proprio nome nell'apposita casella di testo
\end{enumerate}

\textbf{Trigger}
\begin{itemize}
    \item L'utente clicca sulla casella di testo per l'inserimento del nome
\end{itemize}

\vspace{0.5cm}

\subsubsection{UC-0A.5 – Inserimento cognome}

\textbf{Attori}
\begin{itemize}
    \item utente
\end{itemize}

\textbf{Pre-condizioni}
\begin{itemize}
    \item Casella di testo vuota
\end{itemize}

\textbf{Post-condizioni}
\begin{itemize}
    \item Casella di testo contenente il cognome inserito
\end{itemize}

\textbf{Scenario principale}
\begin{enumerate}
    \item L'utente durante la registrazione inserisce il proprio cognome nell'apposita casella di testo
\end{enumerate}

\textbf{Trigger}
\begin{itemize}
    \item L'utente clicca sulla casella di testo per l'inserimento del cognome
\end{itemize}

\vspace{0.5cm}

\subsubsection{UC-0A.6 – Inserimento matricola}

\textbf{Attori}
\begin{itemize}
    \item utente
\end{itemize}

\textbf{Pre-condizioni}
\begin{itemize}
    \item Casella di testo vuota
\end{itemize}

\textbf{Post-condizioni}
\begin{itemize}
    \item Casella di testo contenente la matricola inserita
\end{itemize}

\textbf{Scenario principale}
\begin{enumerate}
    \item L'utente durante la registrazione inserisce la propria matricola nell'apposita casella di testo
\end{enumerate}

\textbf{Trigger}
\begin{itemize}
    \item L'utente clicca sulla casella di testo per l'inserimento della matricola
\end{itemize}

\vspace{0.5cm}

\subsubsection{UC-0A.7 – Email non valida}

\textbf{Attori}
\begin{itemize}
    \item utente
\end{itemize}

\textbf{Pre-condizioni}
\begin{itemize}
    \item L'utente ha inserito dei caratteri nel campo email
\end{itemize}

\textbf{Post-condizioni}
\begin{itemize}
    \item Viene visualizzato un messaggio di errore relativo al formato non valido dell'email
\end{itemize}

\textbf{Scenario principale}
\begin{enumerate}
    \item L'utente inserisce un indirizzo email che non rispetta il formato standard (es. manca la chiocciola o il dominio)
    \item Il sistema rileva che il formato non è corretto
\end{enumerate}

\textbf{Trigger}
\begin{itemize}
    \item L'utente conferma l'inserimento o passa al campo successivo
\end{itemize}

\vspace{0.5cm}

\subsubsection{UC-0A.8 – Password non valida}

\textbf{Attori}
\begin{itemize}
    \item utente
\end{itemize}

\textbf{Pre-condizioni}
\begin{itemize}
    \item L'utente ha inserito dei caratteri nel campo password
\end{itemize}

\textbf{Post-condizioni}
\begin{itemize}
    \item Viene visualizzato un messaggio di errore relativo ai requisiti di sicurezza della password
\end{itemize}

\textbf{Scenario principale}
\begin{enumerate}
    \item L'utente inserisce una password che non soddisfa i criteri minimi di sicurezza (es. lunghezza minima, caratteri speciali)
    \item Il sistema rileva che la password è troppo debole
\end{enumerate}

\textbf{Trigger}
\begin{itemize}
    \item L'utente conferma l'inserimento o passa al campo successivo
\end{itemize}

\vspace{0.5cm}

\subsubsection{UC-0A.9 – Email già registrata}

\textbf{Attori}
\begin{itemize}
    \item utente
\end{itemize}

\textbf{Pre-condizioni}
\begin{itemize}
    \item L'indirizzo email inserito è già presente all'interno del sistema
\end{itemize}

\textbf{Post-condizioni}
\begin{itemize}
    \item Viene visualizzato un messaggio di errore che notifica l'esistenza dell'account
\end{itemize}

\textbf{Scenario principale}
\begin{enumerate}
    \item L'utente inserisce un'email formalmente valida ma già associata ad un altro utente registrato
    \item Il sistema verifica la presenza dell'email nel database e blocca l'operazione
\end{enumerate}

\textbf{Trigger}
\begin{itemize}
    \item L'utente tenta di completare la registrazione o conferma il campo email
\end{itemize}

\vspace{0.5cm}

\subsubsection{UC-0A.10 – Username già registrato}

\textbf{Attori}
\begin{itemize}
    \item utente
\end{itemize}

\textbf{Pre-condizioni}
\begin{itemize}
    \item Lo username inserito è già presente all'interno del sistema
\end{itemize}

\textbf{Post-condizioni}
\begin{itemize}
    \item Viene visualizzato un messaggio di errore che indica che lo username non è disponibile
\end{itemize}

\textbf{Scenario principale}
\begin{enumerate}
    \item L'utente inserisce uno username già utilizzato da un altro utente
    \item Il sistema verifica l'univocità dello username e ne segnala l'indisponibilità
\end{enumerate}

\textbf{Trigger}
\begin{itemize}
    \item L'utente tenta di completare la registrazione o conferma il campo username
\end{itemize}

\vspace{0.5cm}

\subsubsection{UC-0A.11 – Matricola già registrata}

\textbf{Attori}
\begin{itemize}
    \item utente
\end{itemize}

\textbf{Pre-condizioni}
\begin{itemize}
    \item La matricola inserita è già associata ad un account esistente
\end{itemize}

\textbf{Post-condizioni}
\begin{itemize}
    \item Viene visualizzato un messaggio di errore che impedisce la registrazione multipla con la stessa matricola
\end{itemize}

\textbf{Scenario principale}
\begin{enumerate}
    \item L'utente inserisce un numero di matricola già presente nel sistema
    \item Il sistema rileva la duplicazione e impedisce il proseguimento
\end{enumerate}

\textbf{Trigger}
\begin{itemize}
    \item L'utente tenta di completare la registrazione
\end{itemize}

\vspace{0.5cm}

\subsubsection{UC-0A.12 – Matricola non valida}

\textbf{Attori}
\begin{itemize}
    \item utente
\end{itemize}

\textbf{Pre-condizioni}
\begin{itemize}
    \item L'utente ha inserito dei dati nel campo matricola
\end{itemize}

\textbf{Post-condizioni}
\begin{itemize}
    \item Viene visualizzato un messaggio di errore sul formato della matricola
\end{itemize}

\textbf{Scenario principale}
\begin{enumerate}
    \item L'utente inserisce una matricola che non rispetta il formato atteso (es. contiene lettere dove non previste o ha una lunghezza errata)
    \item Il sistema invalida il dato inserito
\end{enumerate}

\textbf{Trigger}
\begin{itemize}
    \item L'utente conferma l'inserimento o passa al campo successivo
\end{itemize}

\vspace{0.5cm}

\subsubsection{UC-0B – Login / Autenticazione utente}

\begin{figure}[H]
    \centering
    \includegraphics[width=0.7\textwidth]{Diagrammi casi d'uso/UC0B.jpg}
    \caption{Didascalia dell'immagine}
\end{figure}

\textbf{Attori}
\begin{itemize}
    \item Utente registrato.
    \item Sistema di autenticazione dell’applicazione standalone.
\end{itemize}

\textbf{Pre-condizioni}
\begin{itemize}
    \item L’utente è già registrato nel Sistema.
    \item Non esiste una sessione attiva associata all’utente sul dispositivo corrente.
\end{itemize}

\textbf{Post-condizioni}
\begin{itemize}
    \item L’utente risulta autenticato nel Sistema.
    \item È attiva una sessione associata all’utente, che consente l’accesso alle funzionalità riservate (es. generazione contenuti, upload documenti).
\end{itemize}

\textbf{Scenario principale}
\begin{enumerate}
    \item L’utente accede alla schermata di login.
    \item L’utente inserisce le proprie credenziali (e-mail e password).
    \item Il Sistema verifica la correttezza delle credenziali.
    \item In caso di credenziali valide, il Sistema crea una nuova sessione autenticata per l’utente.
    \item Il Sistema reindirizza l’utente alla dashboard principale dell’applicazione standalone.
\end{enumerate}

\textbf{Scenario secondario}
\begin{enumerate}
    \item L'utente inserisce un’e-mail non registrata nel Sistema.
    \item L'utente inserisce una password errata.
    \item L'utente inserisce credenziali non valide.
\end{enumerate}

\textbf{Trigger}
\begin{itemize}
    \item L'utente avvia l'applicazione standalone e seleziona l'opzione di login.
\end{itemize}



\textbf{Relazioni con altri casi d'uso (\textit{include} / \textit{extend})}
\begin{itemize}
    \item \textit{include}:
    \begin{itemize}
        \item UC-0A.1 - Inserimento email
        \item UC-0A.2 - Inserimento password
    \end{itemize}
    \item \textit{extend}: 
    \begin{itemize}
        \item UC-0A.7 – email non valida.
        \item UC-0A.8 – password non valida.
        \item UC-0B.1 – email non registrata.
        \item UC-0B.2 – password errata.
    \end{itemize}
\end{itemize}

\vspace{0.5cm}

\subsubsection{UC-0B.1 – Email non registrata}

\textbf{Attori}
\begin{itemize}
    \item utente
\end{itemize}

\textbf{Pre-condizioni}
\begin{itemize}
    \item L'email inserita nel modulo di accesso non è presente nel database del sistema
\end{itemize}

\textbf{Post-condizioni}
\begin{itemize}
    \item Viene visualizzato un messaggio di errore e l'accesso viene negato
\end{itemize}

\textbf{Scenario principale}
\begin{enumerate}
    \item L'utente tenta di effettuare il login inserendo un indirizzo email non associato ad alcun account
    \item Il sistema verifica l'esistenza dell'email e non trova corrispondenze
\end{enumerate}

\textbf{Trigger}
\begin{itemize}
    \item L'utente clicca sul pulsante di conferma per effettuare il login
\end{itemize}

\vspace{0.5cm}

\subsubsection{UC-0B.2 – Password errata}

\textbf{Attori}
\begin{itemize}
    \item utente
\end{itemize}

\textbf{Pre-condizioni}
\begin{itemize}
    \item L'email inserita è corretta, ma la password non corrisponde a quella salvata nel sistema
\end{itemize}

\textbf{Post-condizioni}
\begin{itemize}
    \item Viene visualizzato un messaggio di errore relativo alle credenziali non valide
\end{itemize}

\textbf{Scenario principale}
\begin{enumerate}
    \item L'utente inserisce la propria email (corretta) e una password errata
    \item Il sistema verifica la corrispondenza delle credenziali e rileva l'errore
\end{enumerate}

\textbf{Trigger}
\begin{itemize}
    \item L'utente clicca sul pulsante di conferma per effettuare il login
\end{itemize}

\vspace{0.5cm}


\subsubsection{UC-0C – Apertura dashboard}

\begin{figure}[H]
    \centering
    \includegraphics[width=1\textwidth]{Diagrammi casi d'uso/UC0C.jpg}
    \caption{Didascalia dell'immagine}
\end{figure}

\textbf{Attori}
\begin{itemize}
    \item Utente autenticato.
\end{itemize}

\textbf{Pre-condizioni}
\begin{itemize}
    \item L’utente ha effettuato il login ed è autenticato.
    \item Esiste un profilo associato all’utente nel Sistema (dati anagrafici e preferenze).
\end{itemize}


\textbf{Post-condizioni}
\begin{itemize}
    \item L'utente autenticato puó scegliere tra diverse azioni.
\end{itemize}

\textbf{Scenario principale}
\begin{enumerate}
    \item L’utente dopo aver effettuato il login si ritrova in una schermata con la scelta di diversi moduli.
\end{enumerate}

\textbf{Scenario secondario}
\begin{enumerate}
    \item 
\end{enumerate}

\textbf{Trigger}
\begin{itemize}
    \item Effettuare il login con successo nell'applicazione standalone.
\end{itemize}



\textbf{Relazioni con altri casi d'uso (\textit{include} / \textit{extend})}
\begin{itemize}
    \item \textit{include}: 
    \begin{itemize}
        \item 
    \end{itemize}
    \item \textit{extend}: 
    \begin{itemize}
        \item UC-1A - Apertura modulo di generazione contenuti AI.
        \item UC-2A - Apertura modulo di upload e gestione documentale.
        \item UC-0D - Apertura modulo gestione profilo utente.
        \item UC-0E - Apertura modulo gestione ruoli.
    \end{itemize}
\end{itemize}


\vspace{0.5cm}

\subsubsection{UC-0D – Apertura gestione profilo utente}

\begin{figure}[H]
    \centering
    \includegraphics[width=1\textwidth]{Diagrammi casi d'uso/UC0C.jpg}
    \caption{Didascalia dell'immagine}
\end{figure}

\textbf{Attori}
\begin{itemize}
    \item Utente autenticato.
    \item Sistema di gestione profilo dell’applicazione standalone.
\end{itemize}


\textbf{Pre-condizioni}
\begin{itemize}
    \item L’utente ha effettuato il login ed è autenticato.
    \item Esiste un profilo associato all’utente nel Sistema (dati anagrafici e preferenze).
    \item L'utente speciale é entrato nel modulo di gestione profilo utente dalla dashboard principale.

\end{itemize}

\textbf{Post-condizioni}
\begin{itemize}
    \item Le informazioni del profilo utente risultano aggiornate nel Sistema.
    \item Le nuove preferenze (ad esempio tono/stile predefinito) verranno utilizzate nelle interazioni successive con i moduli AI.
\end{itemize}

\textbf{Scenario principale}
\begin{enumerate}
    \item L’utente accede alla sezione “Profilo” dalla dashboard dell’applicazione.
    \item Il Sistema mostra i dati correnti del profilo (es. nome, cognome, e-mail, ruolo, preferenze AI come tono/stile predefinito).
    \item L’utente puó uno o più campi del profilo (es. nome visualizzato, preferenze di tono, lingua).
    \item L’utente conferma le modifiche.
    \item Il Sistema valida i dati inseriti (ad esempio formato dell’e-mail, campi obbligatori).
    \item Il Sistema salva le modifiche nel proprio archivio.
    \item Il Sistema conferma l’avvenuto aggiornamento del profilo.
\end{enumerate}

\textbf{Scenario secondario}
\begin{enumerate}
    \item 
\end{enumerate}

\textbf{Trigger}
\begin{itemize}
    \item Selezionare l'opzione di gestione profilo utente dalla dashboard principale.
\end{itemize}


\textbf{Relazioni con altri casi d'uso (\textit{include} / \textit{extend})}
\begin{itemize}
    \item \textit{include}: 
    \begin{itemize}
        \item UC-0D.1 - Visualizzazione email
        \item UC-0D.2 - Visualizzazione password
        \item UC-0D.3 - Visualizzazione username
        \item UC-0D.4 - Visualizzazione nome 
        \item UC-0D.5 - Visualizzazione cognome
        \item UC-0D.6 - Visualizzazione matricola
    \end{itemize}
    \item \textit{extend}: 
    \begin{itemize}
        \item UC-0A.1 - Inserimento email
        \item UC-0A.2 - Inserimento password
        \item UC-0A.3 - Inserimento username
        \item UC-0A.4 - Inserimento nome 
        \item UC-0A.5 - Inserimento cognome
        \item UC-0A.6 - Inserimento matricola
        \item UC-0D.7 – Salva profilo utente. %in questo caso d'uso ci saranno tutti gli errori su inserimenti sbaliati (uguali a quelli della registrazione)
        \item UC-0D.8 – Uscita senza salvare profilo utente.
    \end{itemize}
\end{itemize}

\vspace{0.5cm}

\subsubsection{UC-0D.1 – Visualizzazione email}

\textbf{Attori}
\begin{itemize}
    \item utente
\end{itemize}

\textbf{Pre-condizioni}
\begin{itemize}
    \item L'utente ha effettuato l'accesso e si trova nella pagina del profilo
\end{itemize}

\textbf{Post-condizioni}
\begin{itemize}
    \item Casella di testo precompilata con l'email attuale dell'utente
\end{itemize}

\textbf{Scenario principale}
\begin{enumerate}
    \item Il sistema recupera l'email associata all'account e la mostra nell'apposita casella
\end{enumerate}

\textbf{Trigger}
\begin{itemize}
    \item L'utente accede alla sezione "Modifica Profilo" o "I miei dati"
\end{itemize}

\vspace{0.5cm}

\subsubsection{UC-0D.2 – Visualizzazione password}

\textbf{Attori}
\begin{itemize}
    \item utente
\end{itemize}

\textbf{Pre-condizioni}
\begin{itemize}
    \item L'utente ha effettuato l'accesso e si trova nella pagina del profilo
\end{itemize}

\textbf{Post-condizioni}
\begin{itemize}
    \item Casella di testo contenente la password attuale (tipicamente oscurata)
\end{itemize}

\textbf{Scenario principale}
\begin{enumerate}
    \item Il sistema predispone il campo password permettendo all'utente di visualizzarne lo stato o modificarla
\end{enumerate}

\textbf{Trigger}
\begin{itemize}
    \item L'utente accede alla sezione "Modifica Profilo"
\end{itemize}

\vspace{0.5cm}

\subsubsection{UC-0D.3 – Visualizzazione username}

\textbf{Attori}
\begin{itemize}
    \item utente
\end{itemize}

\textbf{Pre-condizioni}
\begin{itemize}
    \item L'utente ha effettuato l'accesso e si trova nella pagina del profilo
\end{itemize}

\textbf{Post-condizioni}
\begin{itemize}
    \item Casella di testo precompilata con lo username attuale dell'utente
\end{itemize}

\textbf{Scenario principale}
\begin{enumerate}
    \item Il sistema recupera lo username associato all'account e lo mostra nell'apposita casella
\end{enumerate}

\textbf{Trigger}
\begin{itemize}
    \item L'utente accede alla sezione "Modifica Profilo"
\end{itemize}

\vspace{0.5cm}

\subsubsection{UC-0D.4 – Visualizzazione nome}

\textbf{Attori}
\begin{itemize}
    \item utente
\end{itemize}

\textbf{Pre-condizioni}
\begin{itemize}
    \item L'utente ha effettuato l'accesso e si trova nella pagina del profilo
\end{itemize}

\textbf{Post-condizioni}
\begin{itemize}
    \item Casella di testo precompilata con il nome attuale dell'utente
\end{itemize}

\textbf{Scenario principale}
\begin{enumerate}
    \item Il sistema recupera il nome dell'utente e lo mostra nell'apposita casella
\end{enumerate}

\textbf{Trigger}
\begin{itemize}
    \item L'utente accede alla sezione "Modifica Profilo"
\end{itemize}

\vspace{0.5cm}

\subsubsection{UC-0D.5 – Visualizzazione cognome}

\textbf{Attori}
\begin{itemize}
    \item utente
\end{itemize}

\textbf{Pre-condizioni}
\begin{itemize}
    \item L'utente ha effettuato l'accesso e si trova nella pagina del profilo
\end{itemize}

\textbf{Post-condizioni}
\begin{itemize}
    \item Casella di testo precompilata con il cognome attuale dell'utente
\end{itemize}

\textbf{Scenario principale}
\begin{enumerate}
    \item Il sistema recupera il cognome dell'utente e lo mostra nell'apposita casella
\end{enumerate}

\textbf{Trigger}
\begin{itemize}
    \item L'utente accede alla sezione "Modifica Profilo"
\end{itemize}

\vspace{0.5cm}

\subsubsection{UC-0D.6 – Visualizzazione matricola}

\textbf{Attori}
\begin{itemize}
    \item utente
\end{itemize}

\textbf{Pre-condizioni}
\begin{itemize}
    \item L'utente ha effettuato l'accesso e si trova nella pagina del profilo
\end{itemize}

\textbf{Post-condizioni}
\begin{itemize}
    \item Casella di testo precompilata con la matricola attuale dell'utente
\end{itemize}

\textbf{Scenario principale}
\begin{enumerate}
    \item Il sistema recupera la matricola associata all'account e la mostra nell'apposita casella
\end{enumerate}

\textbf{Trigger}
\begin{itemize}
    \item L'utente accede alla sezione "Modifica Profilo"
\end{itemize}

\vspace{0.5cm}

\subsubsection{UC-0D.7 – Salva profilo utente}

\textbf{Attori}
\begin{itemize}
    \item utente
\end{itemize}

\textbf{Pre-condizioni}
\begin{itemize}
    \item L'utente ha modificato uno o più campi del proprio profilo
\end{itemize}

\textbf{Post-condizioni}
\begin{itemize}
    \item I dati aggiornati vengono salvati nel database oppure vengono mostrati i messaggi di errore pertinenti
\end{itemize}

\textbf{Scenario principale}
\begin{enumerate}
    \item L'utente preme il pulsante di salvataggio delle modifiche
    \item Il sistema verifica la validità dei nuovi dati inseriti (controllo formato, unicità email/username/matricola analogamente alla fase di registrazione)
    \item Se non sono presenti errori, il sistema sovrascrive i dati precedenti con quelli nuovi; in caso contrario, segnala l'errore specifico al campo interessato
\end{enumerate}

\textbf{Trigger}
\begin{itemize}
    \item L'utente clicca sul pulsante "Salva" o "Aggiorna"
\end{itemize}

\vspace{0.5cm}

\subsubsection{UC-0D.8 – Uscita senza salvare profilo utente}

\textbf{Attori}
\begin{itemize}
    \item utente
\end{itemize}

\textbf{Pre-condizioni}
\begin{itemize}
    \item L'utente si trova nella pagina di modifica profilo
\end{itemize}

\textbf{Post-condizioni}
\begin{itemize}
    \item L'utente viene reindirizzato alla pagina precedente o alla home senza che alcuna modifica venga applicata al database
\end{itemize}

\textbf{Scenario principale}
\begin{enumerate}
    \item L'utente decide di annullare l'operazione di modifica
    \item Il sistema scarta le modifiche pendenti nei campi di testo e chiude la schermata
\end{enumerate}

\textbf{Trigger}
\begin{itemize}
    \item L'utente clicca sul pulsante "Annulla" o "Indietro"
\end{itemize}

\vspace{0.5cm}

\subsubsection{UC-0E – Apertura gestione ruoli (Admin / Editor)}

\begin{figure}[H]
    \centering
    \includegraphics[width=0.7\textwidth]{Diagrammi casi d'uso/UC0B.jpg}
    \caption{Didascalia dell'immagine}
\end{figure}

\textbf{Attori}
\begin{itemize}
    \item Utenti speciali (Admin o altri ruoli definiti).
    \item Sistema di gestione ruoli e permessi.
\end{itemize}


\textbf{Pre-condizioni}
\begin{itemize}
    \item L’utente amministratore ha effettuato il login ed è autenticato come Admin.
    \item Esistono uno o più account utente registrati nel Sistema.
    \item L'utente speciale é entrato nel modulo di gestione ruoli dalla dashboard principale.
\end{itemize}

\textbf{Post-condizioni}
\begin{itemize}
    \item I ruoli e i permessi degli utenti risultano aggiornati nel Sistema.
    \item Le funzionalità accessibili a ciascun utente dipendono dal nuovo ruolo assegnato (es. solo Admin può modificare i parametri AI globali).
\end{itemize}

\textbf{Scenario principale}
\begin{enumerate}
    \item L’Amministratore accede alla sezione di amministrazione utenti.
    \item Il Sistema mostra l’elenco degli utenti registrati, con i rispettivi ruoli correnti.
    \item L’Amministratore seleziona un utente da modificare.
    \item L’Amministratore assegna o modifica il ruolo dell’utente (es. da Editor a Admin, oppure rimozione privilegi).
    \item L’Amministratore conferma le modifiche.
    \item Il Sistema aggiorna i ruoli e i permessi associati all’utente.
    \item Il Sistema registra l’operazione per finalità di audit interno.
\end{enumerate}

\textbf{Scenario secondario}
\begin{enumerate}
    \item Un utente non autorizzato tenta di accedere al modulo di gestione ruoli.
\end{enumerate}

\textbf{Trigger}
\begin{itemize}
    \item Selezionare l'opzione di gestione ruoli dalla dashboard principale (solo per utenti con permessi Admin).
\end{itemize}


\textbf{Relazioni con altri casi d'uso (\textit{include} / \textit{extend})}
\begin{itemize}
    \item \textit{include}: 
    \begin{itemize}
        \item UC-0E.1 - Visualizzazione nome utenti registrati.
        \item UC-0E.2 - Visualizzazione cognome utenti registrati.
        \item UC-0E.3 - Visualizzazione ruolo utenti registrati.
    \end{itemize}
    \item \textit{extend}: 
    \begin{itemize}
        \item UC-0E.4 - Modifica ruolo utente registrato.
        \item UC-0E.5 - Salva modifica ruolo utente registrato.
        \item UC-0E.6 - Annulla modifica ruolo utente registrato
        \item UC-0E.7 - Utente non autorizzato.
    \end{itemize}
    
\end{itemize}

\vspace{0.5cm}

\subsubsection{UC-0E.1 – Visualizzazione nome utenti registrati}

\textbf{Attori}
\begin{itemize}
    \item Amministratore
\end{itemize}

\textbf{Pre-condizioni}
\begin{itemize}
    \item L'amministratore ha effettuato l'accesso e si trova nella sezione di gestione utenti
\end{itemize}

\textbf{Post-condizioni}
\begin{itemize}
    \item Viene visualizzata la lista dei nomi degli utenti registrati nel sistema
\end{itemize}

\textbf{Scenario principale}
\begin{enumerate}
    \item Il sistema recupera dal database la lista degli utenti e ne mostra i nomi in formato tabellare o a lista
\end{enumerate}

\textbf{Trigger}
\begin{itemize}
    \item L'amministratore accede alla pagina "Gestione Utenti"
\end{itemize}

\vspace{0.5cm}

\subsubsection{UC-0E.2 – Visualizzazione cognome utenti registrati}

\textbf{Attori}
\begin{itemize}
    \item Amministratore
\end{itemize}

\textbf{Pre-condizioni}
\begin{itemize}
    \item L'amministratore ha effettuato l'accesso e si trova nella sezione di gestione utenti
\end{itemize}

\textbf{Post-condizioni}
\begin{itemize}
    \item Viene visualizzata la lista dei cognomi degli utenti registrati nel sistema
\end{itemize}

\textbf{Scenario principale}
\begin{enumerate}
    \item Il sistema recupera dal database la lista degli utenti e ne mostra i cognomi in corrispondenza dei rispettivi nomi
\end{enumerate}

\textbf{Trigger}
\begin{itemize}
    \item L'amministratore accede alla pagina "Gestione Utenti"
\end{itemize}

\vspace{0.5cm}

\subsubsection{UC-0E.3 – Visualizzazione ruolo utenti registrati}

\textbf{Attori}
\begin{itemize}
    \item Amministratore
\end{itemize}

\textbf{Pre-condizioni}
\begin{itemize}
    \item L'amministratore ha effettuato l'accesso e si trova nella sezione di gestione utenti
\end{itemize}

\textbf{Post-condizioni}
\begin{itemize}
    \item Viene visualizzato il ruolo attuale (es. Utente, Admin, Moderatore) associato a ciascun utente
\end{itemize}

\textbf{Scenario principale}
\begin{enumerate}
    \item Il sistema identifica i permessi di ogni utente e mostra l'etichetta del ruolo corrispondente
\end{enumerate}

\textbf{Trigger}
\begin{itemize}
    \item L'amministratore accede alla pagina "Gestione Utenti"
\end{itemize}

\vspace{0.5cm}

\subsubsection{UC-0E.4 – Modifica ruolo utente registrato}

\textbf{Attori}
\begin{itemize}
    \item Amministratore
\end{itemize}

\textbf{Pre-condizioni}
\begin{itemize}
    \item L'amministratore ha selezionato un utente specifico dalla lista
\end{itemize}

\textbf{Post-condizioni}
\begin{itemize}
    \item L'interfaccia permette la selezione di un nuovo ruolo per l'utente scelto
\end{itemize}

\textbf{Scenario principale}
\begin{enumerate}
    \item L'amministratore interagisce con il campo relativo al ruolo (es. tramite menu a tendina) e seleziona una nuova tipologia di permessi
\end{enumerate}

\textbf{Trigger}
\begin{itemize}
    \item L'amministratore clicca sul comando di modifica o sul menu del ruolo utente
\end{itemize}

\vspace{0.5cm}

\subsubsection{UC-0E.5 – Salva modifica ruolo utente registrato}

\textbf{Attori}
\begin{itemize}
    \item Amministratore
\end{itemize}

\textbf{Pre-condizioni}
\begin{itemize}
    \item L'amministratore ha modificato il ruolo di un utente ma non ha ancora confermato
\end{itemize}

\textbf{Post-condizioni}
\begin{itemize}
    \item Il nuovo ruolo viene aggiornato nel database e l'utente acquisisce i nuovi permessi
\end{itemize}

\textbf{Scenario principale}
\begin{enumerate}
    \item L'amministratore conferma l'operazione di cambio ruolo
    \item Il sistema registra la modifica e aggiorna lo stato dell'utente
\end{enumerate}

\textbf{Trigger}
\begin{itemize}
    \item L'amministratore clicca sul pulsante "Salva" o "Conferma"
\end{itemize}

\vspace{0.5cm}

\subsubsection{UC-0E.6 – Annulla modifica ruolo utente registrato}

\textbf{Attori}
\begin{itemize}
    \item Amministratore
\end{itemize}

\textbf{Pre-condizioni}
\begin{itemize}
    \item L'amministratore sta modificando il ruolo di un utente
\end{itemize}

\textbf{Post-condizioni}
\begin{itemize}
    \item Il ruolo dell'utente rimane invariato e l'interfaccia torna allo stato precedente
\end{itemize}

\textbf{Scenario principale}
\begin{enumerate}
    \item L'amministratore decide di non applicare le modifiche al ruolo
    \item Il sistema ripristina il valore originale visualizzato
\end{enumerate}

\textbf{Trigger}
\begin{itemize}
    \item L'amministratore clicca sul pulsante "Annulla"
\end{itemize}

\vspace{0.5cm}

\subsubsection{UC-0E.7 – Utente non autorizzato}

\textbf{Attori}
\begin{itemize}
    \item Utente generico (non amministratore)
\end{itemize}

\textbf{Pre-condizioni}
\begin{itemize}
    \item Un utente senza privilegi di amministrazione tenta di accedere a funzionalità o pagine riservate alla gestione ruoli
\end{itemize}

\textbf{Post-condizioni}
\begin{itemize}
    \item L'accesso viene negato e viene mostrato un messaggio di errore o effettuato un reindirizzamento
\end{itemize}

\textbf{Scenario principale}
\begin{enumerate}
    \item L'utente tenta di navigare verso l'URL di gestione utenti o di inviare una richiesta di modifica ruolo
    \item Il sistema verifica i permessi, rileva l'assenza del ruolo di amministratore e blocca l'operazione
\end{enumerate}

\textbf{Trigger}
\begin{itemize}
    \item L'utente tenta l'accesso diretto alla risorsa protetta
\end{itemize}

\vspace{0.5cm}

\subsubsection{UC-0F – Logout}

\begin{figure}[H]
    \centering
    \includegraphics[width=1\textwidth]{Diagrammi casi d'uso/UC0E.jpg}
    \caption{Didascalia dell'immagine}
\end{figure}

\textbf{Attori}
\begin{itemize}
    \item Utente autenticato.
    \item Sistema di gestione sessione dell’applicazione standalone.
\end{itemize}

\textbf{Pre-condizioni}
\begin{itemize}
    \item L’utente ha una sessione attiva nel Sistema.
\end{itemize}

\textbf{Post-condizioni}
\begin{itemize}
    \item Non esiste più una sessione attiva associata all’utente sul dispositivo corrente.
    \item Per accedere nuovamente alle funzionalità riservate è necessario eseguire un nuovo login.
\end{itemize}

\textbf{Scenario principale}
\begin{enumerate}
    \item L’utente seleziona l’opzione di logout (ad esempio dal menu della dashboard).
    \item Il Sistema invalida la sessione corrente associata all’utente (es. rimozione token di sessione).
    \item Il Sistema reindirizza l’utente alla schermata di login o alla schermata iniziale pubblica.
\end{enumerate}

\textbf{Trigger}
\begin{itemize}
    \item Selezionare l'opzione di logout dall'interfaccia utente.
\end{itemize}

\textbf{Relazioni con altri casi d'uso (\textit{include} / \textit{extend})}
\begin{itemize}
    \item \textit{include}: 
    \begin{itemize}
        \item Nessuno.
    \end{itemize}
    \item \textit{extend}:
    \begin{itemize}
        \item Nessuno.
    \end{itemize}
\end{itemize}

\vspace{0.5cm}

\subsection{Sezione 1 – Modulo AI Assistant Generativo}

\subsubsection{UC-1A – Apertura modulo di generazione contenuti AI}


\begin{figure}[H]
    \centering
    \includegraphics[width=1\textwidth]{Diagrammi casi d'uso/UC1A.jpg}
    \caption{Diagramma del caso d'uso UC-1A – Apertura modulo di generazione contenuti AI}
\end{figure}

\textbf{Attori}
\begin{itemize}
    \item Utente autorizzato (Editor o Admin).
\end{itemize}

\textbf{Pre-condizioni}
\begin{itemize}
    \item L’utente é entrato nel modulo AI Assistant Generativo dalla dashboard principale.
    \item L’utente dispone dei permessi necessari per utilizzare il modulo AI Assistant.
\end{itemize}

\textbf{Post-condizioni}
\begin{itemize}
    \item L'utente visualizza l’interfaccia di generazione contenuti AI, pronta per l’inserimento del prompt e la selezione delle opzioni.
\end{itemize}

\textbf{Scenario principale}
\begin{enumerate}
    \item L’utente accede alla sezione “AI Assistant Generativo”.
    \item Il Sistema mostra il campo per l’inserimento del prompt e le azione che possono essere eseguite dall'utente.
\end{enumerate}

\textbf{Scenario secondario}
\begin{enumerate}
    \item 
\end{enumerate}

\textbf{Trigger}
\begin{itemize}
    \item Selezionare l'opzione di modulo AI Assistant Generativo dalla dashboard principale.
\end{itemize}

\textbf{Relazioni con altri casi d'uso (\textit{include} / \textit{extend})}
\begin{itemize}
    \item \textit{include}: 
    \begin{itemize}
        \item Nessuna.
    \end{itemize}
    \item \textit{extend}: 
    \begin{itemize}
        \item UC-1A.1 - Inserimento prompt
        \item UC-1A.2 - Selezione tono
        \item UC-1A.3 - Selezione stile
        \item UC-1C - Generazione contenuto tramite AI
        \item UC-1B - Apertura storico prompt
    \end{itemize}
\end{itemize}

\vspace{0.5cm}

\subsubsection{UC-1A.1 – Inserimento prompt}

\textbf{Attori}
\begin{itemize}
    \item Utente autorizzato (Editor o Admin)
\end{itemize}

\textbf{Pre-condizioni}
\begin{itemize}
    \item L’utente si trova nell'interfaccia di generazione contenuti AI (UC-1A)
    \item La casella di testo del prompt è vuota o modificabile
\end{itemize}

\textbf{Post-condizioni}
\begin{itemize}
    \item La casella di testo contiene la descrizione (prompt) inserita dall'utente
\end{itemize}

\textbf{Scenario principale}
\begin{enumerate}
    \item L'utente inserisce o incolla il testo descrittivo per la generazione del contenuto nell'apposita area di testo
\end{enumerate}

\textbf{Trigger}
\begin{itemize}
    \item L'utente clicca sulla casella di testo dedicata al prompt
\end{itemize}

\vspace{0.5cm}

\subsubsection{UC-1A.2 – Selezione tono}

\textbf{Attori}
\begin{itemize}
    \item Utente autorizzato (Editor o Admin)
\end{itemize}

\textbf{Pre-condizioni}
\begin{itemize}
    \item L’utente si trova nell'interfaccia di generazione contenuti AI
\end{itemize}

\textbf{Post-condizioni}
\begin{itemize}
    \item Il parametro "Tono" risulta impostato sulla scelta effettuata (es. Professionale, Informale, Spiritoso)
\end{itemize}

\textbf{Scenario principale}
\begin{enumerate}
    \item L'utente interagisce con il selettore del tono (es. menu a tendina)
    \item L'utente sceglie l'opzione desiderata tra quelle disponibili
\end{enumerate}

\textbf{Trigger}
\begin{itemize}
    \item L'utente clicca sul menu di selezione del tono
\end{itemize}

\vspace{0.5cm}

\subsubsection{UC-1A.3 – Selezione stile}

\textbf{Attori}
\begin{itemize}
    \item Utente autorizzato (Editor o Admin)
\end{itemize}

\textbf{Pre-condizioni}
\begin{itemize}
    \item L’utente si trova nell'interfaccia di generazione contenuti AI
\end{itemize}

\textbf{Post-condizioni}
\begin{itemize}
    \item Il parametro "Stile" risulta impostato sulla scelta effettuata (es. Articolo di blog, Post social, Email)
\end{itemize}

\textbf{Scenario principale}
\begin{enumerate}
    \item L'utente interagisce con il selettore dello stile
    \item L'utente sceglie la tipologia di formato desiderata per il testo in output
\end{enumerate}

\textbf{Trigger}
\begin{itemize}
    \item L'utente clicca sul menu di selezione dello stile
\end{itemize}

\vspace{0.5cm}

\subsubsection{UC-1B – Apertura storico prompt}

\begin{figure}[H]
    \centering
    \includegraphics[width=0.7\textwidth]{Diagrammi casi d'uso/UC0B.jpg}
    \caption{Didascalia dell'immagine}
\end{figure}

\textbf{Attori}
\begin{itemize}
    \item Utente autorizzato (Editor o Admin).
    \item Sistema di persistenza locale.
\end{itemize}

\textbf{Pre-condizioni}
\begin{itemize}
    \item Ci sono stati precedenti utilizzi del modulo AI Assistant Generativo.
    \item L’utente é entrato nel modulo AI Assistant Generativo dalla dashboard principale.
\end{itemize}

\textbf{Post-condizioni}
\begin{itemize}
    \item L'utente visualizza lo storico dei prompt utilizzati e puó interagire con essi (es. riutilizzo, duplicazione, ricerca).
\end{itemize}

\textbf{Scenario principale}
\begin{enumerate}
    \item L’utente accede alla sezione “Storico Prompt” all’interno del modulo AI Assistant Generativo.
    \item Il Sistema mostra l’elenco dei prompt precedentemente utilizzati, con i relativi dettagli (tono, stile, risultato generato, data/ora, valutazione).
\end{enumerate}

\textbf{Scenario secondario}
\begin{enumerate}
    \item Non ci sono prompt salvati nello storico.
\end{enumerate}

\textbf{Trigger}
\begin{itemize}
    \item Selezionare l'opzione di visualizzazione storico prompt all’interno del modulo AI Assistant Generativo.
\end{itemize}

\textbf{Relazioni con altri casi d'uso (\textit{include} / \textit{extend})}
\begin{itemize}
    \item \textit{include}: 
    \begin{itemize}
        \item Nessuna
    \end{itemize}
    \item \textit{extend}: 
    \begin{itemize}
        \item UC-1B.11 – visualizzazione informazioni elemento
        \item UC-1B.7 – Ricerca
        \item UC-1B.8 – Riutilizza (rigenera con lo stesso prompt)
        \item UC-1B.9 – Duplica (Ti riporta al modulo di generazione con prompt, tono e stile precompilati)
        \item UC-1B.10 – Nessun prompt salvato.
    \end{itemize}
\end{itemize}

\vspace{0.5cm}

\subsubsection{UC-1B.7 – Ricerca}

\textbf{Attori}
\begin{itemize}
    \item Utente autorizzato
\end{itemize}

\textbf{Pre-condizioni}
\begin{itemize}
    \item Lo storico è popolato da diversi elementi
\end{itemize}

\textbf{Post-condizioni}
\begin{itemize}
    \item La lista visualizzata viene filtrata mostrando solo gli elementi che corrispondono ai criteri di ricerca
\end{itemize}

\textbf{Scenario principale}
\begin{enumerate}
    \item L'utente inserisce una parola chiave (es. parte del prompt o data) nella barra di ricerca
    \item Il sistema filtra gli elementi dello storico in tempo reale o alla conferma
\end{enumerate}

\textbf{Trigger}
\begin{itemize}
    \item Inserimento testo nel campo di ricerca o click sul pulsante "Cerca"
\end{itemize}

\textbf{Relazioni con altri casi d'uso (\textit{include} / \textit{extend})}
\begin{itemize}
    \item \textit{extend}: 
    \begin{itemize}
        \item UC-1B.12 – Ricerca per prompt
        \item UC-1B.13 – Ricerca per tono
        \item UC-1B.14 – Ricerca per stile
        \item UC-1B.15 – Ricerca per risultato
        \item UC-1B.16 – Ricerca per timestamp
        \item UC-1B.17 – Ricerca per valutazione
    \end{itemize}
\end{itemize}

\vspace{0.5cm}

\subsubsection{UC-1B.8 – Riutilizza}

\textbf{Attori}
\begin{itemize}
    \item Utente autorizzato
    \item AI Assistant System
\end{itemize}

\textbf{Pre-condizioni}
\begin{itemize}
    \item L'utente ha individuato un prompt passato che vuole rieseguire immediatamente
\end{itemize}

\textbf{Post-condizioni}
\begin{itemize}
    \item Il sistema avvia una nuova generazione utilizzando esattamente gli stessi parametri (prompt, tono, stile) dell'elemento selezionato
\end{itemize}

\textbf{Scenario principale}
\begin{enumerate}
    \item L'utente clicca sul comando "Riutilizza" associato ad un elemento dello storico
    \item Il sistema invia direttamente i dati al motore AI per generare un nuovo output senza passare per la schermata di modifica
\end{enumerate}

\textbf{Trigger}
\begin{itemize}
    \item Click sul pulsante "Riutilizza" o "Rigenera"
\end{itemize}

\vspace{0.5cm}

\subsubsection{UC-1B.9 – Duplica}

\textbf{Attori}
\begin{itemize}
    \item Utente autorizzato
\end{itemize}

\textbf{Pre-condizioni}
\begin{itemize}
    \item L'utente vuole usare un vecchio prompt come base per una nuova generazione
\end{itemize}

\textbf{Post-condizioni}
\begin{itemize}
    \item L'utente viene riportato al modulo di generazione (UC-1A) con i campi prompt, tono e stile già compilati con i dati storici, pronti per essere modificati
\end{itemize}

\textbf{Scenario principale}
\begin{enumerate}
    \item L'utente clicca sul comando "Duplica" associato ad un elemento dello storico
    \item Il sistema reindirizza l'utente alla pagina di creazione, pre-popolando i campi con i valori recuperati dallo storico
    \item L'utente può ora modificare il prompt o i parametri prima di generare
\end{enumerate}

\textbf{Trigger}
\begin{itemize}
    \item Click sul pulsante "Duplica" o "Usa come bozza"
\end{itemize}

\vspace{0.5cm}

\subsubsection{UC-1B.10 – Nessun prompt salvato}

\textbf{Attori}
\begin{itemize}
    \item Utente autorizzato
\end{itemize}

\textbf{Pre-condizioni}
\begin{itemize}
    \item L'utente accede alla sezione dello storico ma non ha mai effettuato generazioni precedenti o lo storico è stato cancellato
\end{itemize}

\textbf{Post-condizioni}
\begin{itemize}
    \item Viene visualizzato un messaggio informativo (placeholder) che indica l'assenza di dati
\end{itemize}

\textbf{Scenario principale}
\begin{enumerate}
    \item Il sistema interroga il database per recuperare lo storico
    \item Il sistema non trova record associati all'utente e mostra un messaggio del tipo "Nessun prompt presente nello storico"
\end{enumerate}

\textbf{Trigger}
\begin{itemize}
    \item L'apertura della sezione "Storico Prompt"
\end{itemize}

\vspace{0.5cm}

\subsubsection{UC-1B.11 – Visualizzazione informazioni elemento}

\textbf{Attori}
\begin{itemize}
    \item Utente autorizzato
\end{itemize}

\textbf{Pre-condizioni}
\begin{itemize}
    \item La lista dello storico contiene almeno un elemento
\end{itemize}

\textbf{Post-condizioni}
\begin{itemize}
    \item L'utente visualizza i dettagli completi di una specifica generazione (prompt completo, parametri usati, output ottenuto, data)
\end{itemize}

\textbf{Scenario principale}
\begin{enumerate}
    \item L'utente seleziona un elemento dalla lista dello storico
    \item Il sistema espande l'elemento o apre una modale mostrando tutti i metadati associati a quella generazione
\end{enumerate}

\textbf{Trigger}
\begin{itemize}
    \item Click su una riga o su un'icona di dettaglio dell'elemento in elenco
\end{itemize}

\textbf{Relazioni con altri casi d'uso (\textit{include} / \textit{extend})}
\begin{itemize}
    \item \textit{include}: 
    \begin{itemize}
        \item UC-1B.1 – Visualizzazione prompt
        \item UC-1B.2 – Visualizzazione tono
        \item UC-1B.3 – Visualizzazione stile
        \item UC-1B.4 – Visualizzazione risultato
        \item UC-1B.5 – Visualizzazione timestamp
        \item UC-1B.6 – Visualizzazione Valutazione
    \end{itemize}
\end{itemize}

\vspace{0.5cm}

\subsubsection{UC-1B.1 – Visualizzazione prompt}

\textbf{Attori}
\begin{itemize}
    \item Utente autorizzato
\end{itemize}

\textbf{Pre-condizioni}
\begin{itemize}
    \item L'utente ha aperto la visualizzazione di dettaglio di un elemento dello storico (UC-1B.11)
\end{itemize}

\textbf{Post-condizioni}
\begin{itemize}
    \item Viene visualizzato il testo completo del prompt inserito originariamente dall'utente
\end{itemize}

\textbf{Scenario principale}
\begin{enumerate}
    \item Il sistema recupera dal database il testo della richiesta (prompt) associato alla generazione selezionata e lo mostra a video
\end{enumerate}

\textbf{Trigger}
\begin{itemize}
    \item Apertura del dettaglio elemento (incluso in UC-1B.11)
\end{itemize}

\vspace{0.5cm}

\subsubsection{UC-1B.2 – Visualizzazione tono}

\textbf{Attori}
\begin{itemize}
    \item Utente autorizzato
\end{itemize}

\textbf{Pre-condizioni}
\begin{itemize}
    \item L'utente ha aperto la visualizzazione di dettaglio di un elemento dello storico
\end{itemize}

\textbf{Post-condizioni}
\begin{itemize}
    \item Viene visualizzato il parametro "Tono" utilizzato per la generazione
\end{itemize}

\textbf{Scenario principale}
\begin{enumerate}
    \item Il sistema recupera e visualizza l'etichetta del tono (es. Professionale, Amichevole) salvato con la generazione
\end{enumerate}

\textbf{Trigger}
\begin{itemize}
    \item Apertura del dettaglio elemento (incluso in UC-1B.11)
\end{itemize}

\vspace{0.5cm}

\subsubsection{UC-1B.3 – Visualizzazione stile}

\textbf{Attori}
\begin{itemize}
    \item Utente autorizzato
\end{itemize}

\textbf{Pre-condizioni}
\begin{itemize}
    \item L'utente ha aperto la visualizzazione di dettaglio di un elemento dello storico
\end{itemize}

\textbf{Post-condizioni}
\begin{itemize}
    \item Viene visualizzato il parametro "Stile" utilizzato per la generazione
\end{itemize}

\textbf{Scenario principale}
\begin{enumerate}
    \item Il sistema recupera e visualizza la tipologia di contenuto o stile (es. Articolo, Post Social) salvato con la generazione
\end{enumerate}

\textbf{Trigger}
\begin{itemize}
    \item Apertura del dettaglio elemento (incluso in UC-1B.11)
\end{itemize}

\vspace{0.5cm}

\subsubsection{UC-1B.4 – Visualizzazione risultato}

\textbf{Attori}
\begin{itemize}
    \item Utente autorizzato
\end{itemize}

\textbf{Pre-condizioni}
\begin{itemize}
    \item L'utente ha aperto la visualizzazione di dettaglio di un elemento dello storico
\end{itemize}

\textbf{Post-condizioni}
\begin{itemize}
    \item Viene visualizzato il testo generato dall'intelligenza artificiale
\end{itemize}

\textbf{Scenario principale}
\begin{enumerate}
    \item Il sistema mostra il contenuto testuale prodotto dall'AI in risposta al prompt, mantenendo la formattazione originale se presente
\end{enumerate}

\textbf{Trigger}
\begin{itemize}
    \item Apertura del dettaglio elemento (incluso in UC-1B.11)
\end{itemize}

\vspace{0.5cm}

\subsubsection{UC-1B.5 – Visualizzazione timestamp}

\textbf{Attori}
\begin{itemize}
    \item Utente autorizzato
\end{itemize}

\textbf{Pre-condizioni}
\begin{itemize}
    \item L'utente ha aperto la visualizzazione di dettaglio di un elemento dello storico
\end{itemize}

\textbf{Post-condizioni}
\begin{itemize}
    \item Viene visualizzata la data e l'ora in cui è stata effettuata la generazione
\end{itemize}

\textbf{Scenario principale}
\begin{enumerate}
    \item Il sistema mostra il timestamp (data e ora) di creazione del contenuto
\end{enumerate}

\textbf{Trigger}
\begin{itemize}
    \item Apertura del dettaglio elemento (incluso in UC-1B.11)
\end{itemize}

\vspace{0.5cm}

\subsubsection{UC-1B.6 – Visualizzazione Valutazione}

\textbf{Attori}
\begin{itemize}
    \item Utente autorizzato
\end{itemize}

\textbf{Pre-condizioni}
\begin{itemize}
    \item L'utente ha aperto la visualizzazione di dettaglio di un elemento dello storico
\end{itemize}

\textbf{Post-condizioni}
\begin{itemize}
    \item Viene visualizzato il feedback o il voto assegnato dall'utente (se presente)
\end{itemize}

\textbf{Scenario principale}
\begin{enumerate}
    \item Il sistema verifica se esiste una valutazione associata al contenuto
    \item Se presente, mostra l'indicatore grafico (es. stelle, thumbs up/down); in caso contrario mostra un campo vuoto o l'opzione per aggiungere una valutazione
\end{enumerate}

\textbf{Trigger}
\begin{itemize}
    \item Apertura del dettaglio elemento (incluso in UC-1B.11)
\end{itemize}

\vspace{0.5cm}

\subsubsection{UC-1C – Generazione contenuto tramite AI}

\begin{figure}[H]
    \centering
    \includegraphics[width=1\textwidth]{Diagrammi casi d'uso/UC1C.jpg}
    \caption{Diagramma del caso d'uso UC-1C – Generazione contenuto tramite AI}
\end{figure}

\textbf{Attori}
\begin{itemize}
    \item Utente autorizzato.
    \item Sistema AI (in caso di nuove generazioni parziali).
\end{itemize}

\textbf{Pre-condizioni}
\begin{itemize}
    \item Un contenuto è stato generato tramite UC-1A.
\end{itemize}

\textbf{Post-condizioni}
\begin{itemize}
    \item L'utente puó eseguire le varie azioni mostrate a schermo
\end{itemize}


\textbf{Scenario principale}
\begin{enumerate}
    \item Il contenuto viene generato e mostrato in anteprima all’utente.
\end{enumerate}

\textbf{Scenario secondario}
\begin{enumerate}
    \item 
\end{enumerate}

\textbf{Trigger}
\begin{itemize}
    \item Completamento della generazione del contenuto AI.
\end{itemize}

\textbf{Relazioni con altri casi d'uso (\textit{include} / \textit{extend})}
\begin{itemize}
    \item \textit{include}: 
    \begin{itemize}
        \item UC-1C.1 – Visualizzazione anteprima contenuto generato.
        \item UC-1C.3 – Salva post generato nello storico.
        \item UC-1A.1 - Inserimento prompt
        \item UC-1A.2 - Selezione tono
        \item UC-1A.3 - Selezione stile
    \end{itemize}
    \item \textit{extend}: 
    \begin{itemize}
        \item UC-1C.2 – Rigenera contenuto tramite AI 
        \item UC-1C.4 – Valuta contenuto generato
        \item UC-1C.5 – Scarta contenuto generato
        \item UC-1C.6 – Pubblica contenuto generato
        \item UC-1D – Apertura modifica contenuto generato
    \end{itemize}
\end{itemize}

\vspace{0.5cm}

\subsubsection{UC-1C.1 – Visualizzazione anteprima contenuto generato}

\textbf{Attori}
\begin{itemize}
    \item Utente autorizzato
\end{itemize}

\textbf{Pre-condizioni}
\begin{itemize}
    \item Il sistema AI ha completato l'elaborazione della richiesta
\end{itemize}

\textbf{Post-condizioni}
\begin{itemize}
    \item Il testo generato è visibile a schermo e formattato correttamente
\end{itemize}

\textbf{Scenario principale}
\begin{enumerate}
    \item Il sistema riceve l'output dall'AI e lo renderizza nell'area di visualizzazione principale, permettendo all'utente di leggerlo
\end{enumerate}

\textbf{Trigger}
\begin{itemize}
    \item Conclusione del processo di generazione (UC-1C)
\end{itemize}

\vspace{0.5cm}

\subsubsection{UC-1C.2 – Rigenera contenuto tramite AI}

\textbf{Attori}
\begin{itemize}
    \item Utente autorizzato
    \item Sistema AI
\end{itemize}

\textbf{Pre-condizioni}
\begin{itemize}
    \item È presente un contenuto generato visibile, ma l'utente non è soddisfatto del risultato
\end{itemize}

\textbf{Post-condizioni}
\begin{itemize}
    \item Il contenuto precedente viene sostituito da una nuova versione generata con gli stessi parametri (o parametri lievemente aggiustati)
\end{itemize}

\textbf{Scenario principale}
\begin{enumerate}
    \item L'utente richiede al sistema di provare nuovamente a generare il contenuto
    \item Il sistema invia una nuova richiesta all'AI
    \item Il nuovo risultato sovrascrive quello visualizzato in anteprima
\end{enumerate}

\textbf{Trigger}
\begin{itemize}
    \item L'utente clicca sul pulsante "Rigenera" o sull'icona di refresh
\end{itemize}

\vspace{0.5cm}

\subsubsection{UC-1C.3 – Salva post generato nello storico}

\textbf{Attori}
\begin{itemize}
    \item Sistema
\end{itemize}

\textbf{Pre-condizioni}
\begin{itemize}
    \item Un contenuto è stato generato con successo
\end{itemize}

\textbf{Post-condizioni}
\begin{itemize}
    \item I dati della generazione (prompt, parametri, output) sono persistiti nel database e accessibili tramite lo storico (UC-1B)
\end{itemize}

\textbf{Scenario principale}
\begin{enumerate}
    \item Il sistema memorizza automaticamente il risultato della generazione e i metadati associati nel database dello storico
\end{enumerate}

\textbf{Trigger}
\begin{itemize}
    \item Completamento della generazione (o click esplicito su "Salva bozza" se previsto manualmente)
\end{itemize}

\vspace{0.5cm}

\subsubsection{UC-1C.4 – Valuta contenuto generato}

\textbf{Attori}
\begin{itemize}
    \item Utente autorizzato
\end{itemize}

\textbf{Pre-condizioni}
\begin{itemize}
    \item Il contenuto generato è visualizzato a schermo
\end{itemize}

\textbf{Post-condizioni}
\begin{itemize}
    \item Il feedback dell'utente viene registrato e associato al contenuto nello storico
\end{itemize}

\textbf{Scenario principale}
\begin{enumerate}
    \item L'utente esprime un giudizio sulla qualità del contenuto (es. pollice in su/giù o stelle)
    \item Il sistema salva la valutazione
\end{enumerate}

\textbf{Trigger}
\begin{itemize}
    \item L'utente clicca sulle icone di valutazione (feedback)
\end{itemize}

\vspace{0.5cm}

\subsubsection{UC-1C.5 – Scarta contenuto generato}

\textbf{Attori}
\begin{itemize}
    \item Utente autorizzato
\end{itemize}

\textbf{Pre-condizioni}
\begin{itemize}
    \item Il contenuto generato è visualizzato a schermo
\end{itemize}

\textbf{Post-condizioni}
\begin{itemize}
    \item Il contenuto attuale viene rimosso dalla vista e l'utente ritorna alla fase di inserimento prompt (UC-1A)
\end{itemize}

\textbf{Scenario principale}
\begin{enumerate}
    \item L'utente decide che il contenuto non è utile e vuole ricominciare da zero
    \item Il sistema pulisce l'area di anteprima e riporta il focus sui campi di input
\end{enumerate}

\textbf{Trigger}
\begin{itemize}
    \item L'utente clicca sul pulsante "Indietro", "Annulla" o "Scarta"
\end{itemize}

\vspace{0.5cm}

\subsubsection{UC-1C.6 – Pubblica contenuto generato}

\textbf{Attori}
\begin{itemize}
    \item Utente autorizzato
\end{itemize}

\textbf{Pre-condizioni}
\begin{itemize}
    \item Il contenuto è stato generato e l'utente lo reputa idoneo
\end{itemize}

\textbf{Post-condizioni}
\begin{itemize}
    \item Il contenuto viene reso pubblico sulla piattaforma o inviato al CMS di destinazione
\end{itemize}

\textbf{Scenario principale}
\begin{enumerate}
    \item L'utente conferma la volontà di utilizzare il contenuto generato
    \item Il sistema cambia lo stato del contenuto in "Pubblicato" o lo trasferisce alla sezione blog/news del sito
\end{enumerate}

\textbf{Trigger}
\begin{itemize}
    \item L'utente clicca sul pulsante "Pubblica" o "Usa questo contenuto"
\end{itemize}

\vspace{0.5cm}

\subsubsection{UC-1D – Apertura modifica contenuto generato}

\begin{figure}[H]
    \centering
    \includegraphics[width=1\textwidth]{Diagrammi casi d'uso/UC1D.png}
    \caption{Diagramma del caso d'uso UC-1D – Apertura modifica contenuto generato}
\end{figure}

\textbf{Attori}
\begin{itemize}
    \item Utente autorizzato (Editor o Admin).
    \item Sistema di persistenza locale.
\end{itemize}

\textbf{Pre-condizioni}
\begin{itemize}
    \item L'utente ha generato un contenuto tramite UC-1A e lo sta visualizzando in anteprima tramite UC-1C.
\end{itemize}

\textbf{Post-condizioni}
\begin{itemize}
    \item L'utente si trova nella pagina di modifica e puó eseguire le opzioni mostrate.
\end{itemize}


\textbf{Scenario principale}
\begin{enumerate}
    \item L'utente é nella pagina del contenuto generato.
    \item L'utente seleziona l’opzione di modifica del contenuto generato.
\end{enumerate}

\textbf{Scenario secondario}
\begin{enumerate}
    \item 
\end{enumerate}

\textbf{Trigger}
\begin{itemize}
    \item Selezionare l'opzione di modifica del contenuto generato dalla pagina di visualizzazione del contenuto.
\end{itemize}


\textbf{Relazioni con altri casi d'uso (\textit{include} / \textit{extend})}
\begin{itemize}
    \item \textit{include}: 
    \begin{itemize}
        \item UC-1C.1 – Visualizzazione anteprima contenuto generato
    \end{itemize}
    \item \textit{extend}: 
    \begin{itemize}
        \item UC-1D.1 – Modifica immagine
        \item UC-1D.2 – Modifica titolo
        \item UC-1D.3 – Modifica testo
        \item UC-1D.4 – Salva modifiche
        \item UC-1D.5 – Annulla modifiche
        \item UC-1D.6 - File immagine non valido
    \end{itemize}
\end{itemize}

\vspace{0.5cm}

\subsubsection{UC-1D.1 – Modifica immagine}

\textbf{Attori}
\begin{itemize}
    \item Utente autorizzato
\end{itemize}

\textbf{Pre-condizioni}
\begin{itemize}
    \item L'utente si trova nella pagina di modifica del contenuto (UC-1D)
\end{itemize}

\textbf{Post-condizioni}
\begin{itemize}
    \item L'immagine associata al contenuto viene aggiornata con quella caricata dall'utente
\end{itemize}

\textbf{Scenario principale}
\begin{enumerate}
    \item L'utente clicca sull'opzione per cambiare l'immagine (upload o selezione da libreria)
    \item L'utente seleziona un file valido dal proprio dispositivo
    \item Il sistema sostituisce l'immagine generata dall'AI con quella fornita dall'utente
\end{enumerate}

\textbf{Trigger}
\begin{itemize}
    \item L'utente interagisce con l'area dedicata all'immagine o clicca su "Carica immagine"
\end{itemize}

\vspace{0.5cm}

\subsubsection{UC-1D.2 – Modifica titolo}

\textbf{Attori}
\begin{itemize}
    \item Utente autorizzato
\end{itemize}

\textbf{Pre-condizioni}
\begin{itemize}
    \item L'utente si trova nella pagina di modifica del contenuto
\end{itemize}

\textbf{Post-condizioni}
\begin{itemize}
    \item Il titolo del contenuto risulta modificato secondo l'input dell'utente
\end{itemize}

\textbf{Scenario principale}
\begin{enumerate}
    \item L'utente modifica il testo presente nella casella di input del titolo
\end{enumerate}

\textbf{Trigger}
\begin{itemize}
    \item L'utente clicca o fa focus sulla casella di testo del titolo
\end{itemize}

\vspace{0.5cm}

\subsubsection{UC-1D.3 – Modifica testo}

\textbf{Attori}
\begin{itemize}
    \item Utente autorizzato
\end{itemize}

\textbf{Pre-condizioni}
\begin{itemize}
    \item L'utente si trova nella pagina di modifica del contenuto
\end{itemize}

\textbf{Post-condizioni}
\begin{itemize}
    \item Il corpo del testo risulta modificato secondo l'input dell'utente
\end{itemize}

\textbf{Scenario principale}
\begin{enumerate}
    \item L'utente agisce sull'editor di testo (corpo del contenuto) aggiungendo, rimuovendo o formattando il testo generato
\end{enumerate}

\textbf{Trigger}
\begin{itemize}
    \item L'utente clicca all'interno dell'area di testo principale
\end{itemize}

\vspace{0.5cm}

\subsubsection{UC-1D.4 – Salva modifiche}

\textbf{Attori}
\begin{itemize}
    \item Utente autorizzato
\end{itemize}

\textbf{Pre-condizioni}
\begin{itemize}
    \item L'utente ha apportato delle modifiche ai campi del contenuto
\end{itemize}

\textbf{Post-condizioni}
\begin{itemize}
    \item Le modifiche vengono persistite nel database e il contenuto aggiornato è pronto per l'uso
\end{itemize}

\textbf{Scenario principale}
\begin{enumerate}
    \item L'utente conferma le operazioni di modifica premendo il pulsante di salvataggio
    \item Il sistema valida i dati e sovrascrive la versione precedente del contenuto
\end{enumerate}

\textbf{Trigger}
\begin{itemize}
    \item L'utente clicca sul pulsante "Salva modifiche"
\end{itemize}

\vspace{0.5cm}

\subsubsection{UC-1D.5 – Annulla modifiche}

\textbf{Attori}
\begin{itemize}
    \item Utente autorizzato
\end{itemize}

\textbf{Pre-condizioni}
\begin{itemize}
    \item L'utente si trova nella pagina di modifica
\end{itemize}

\textbf{Post-condizioni}
\begin{itemize}
    \item Le modifiche non salvate vengono scartate e l'utente ritorna alla visualizzazione precedente (anteprima originale)
\end{itemize}

\textbf{Scenario principale}
\begin{enumerate}
    \item L'utente decide di non applicare le modifiche correnti
    \item Il sistema ripristina lo stato del contenuto a quello precedente l'apertura dell'editor
\end{enumerate}

\textbf{Trigger}
\begin{itemize}
    \item L'utente clicca sul pulsante "Annulla"
\end{itemize}

\vspace{0.5cm}

\subsubsection{UC-1D.6 – File immagine non valido}

\textbf{Attori}
\begin{itemize}
    \item Utente autorizzato
\end{itemize}

\textbf{Pre-condizioni}
\begin{itemize}
    \item L'utente tenta di caricare un'immagine tramite UC-1D.1
\end{itemize}

\textbf{Post-condizioni}
\begin{itemize}
    \item Viene mostrato un messaggio di errore e l'immagine precedente rimane invariata
\end{itemize}

\textbf{Scenario principale}
\begin{enumerate}
    \item L'utente seleziona un file che non rispetta i requisiti del sistema (formato non supportato, dimensione eccessiva, file corrotto)
    \item Il sistema rileva l'anomalia durante il caricamento e blocca l'operazione
\end{enumerate}

\textbf{Trigger}
\begin{itemize}
    \item Conferma della selezione del file da parte dell'utente
\end{itemize}

\vspace{0.5cm}


\subsection{Sezione 2 – Modulo AI Co-Pilot per i Consulenti del Lavoro (CdL)}

\subsubsection{UC-2A – Apertura modulo di upload e gestione documentale.}

\begin{figure}[H]
    \centering
    \includegraphics[width=0.7\textwidth]{Diagrammi casi d'uso/UC0B.jpg}
    \caption{Didascalia dell'immagine}
\end{figure}

\textbf{Attori}
\begin{itemize}
    \item Utente autorizzato (Operatore di studio CdL).
\end{itemize}


\textbf{Pre-condizioni}
\begin{itemize}
    \item L’utente é entrato nel modulo AI Co-Pilot per i CdL dalla dashboard principale.
    \item L’utente dispone dei permessi necessari per utilizzare il modulo AI Co-Pilot per i CdL.
\end{itemize}

\textbf{Post-condizioni}
\begin{itemize}
    \item L'utente visualizza l’interfaccia di upload documenti, pronta per il caricamento e la gestione dei file.
\end{itemize}

\textbf{Scenario principale}
\begin{enumerate}
    \item L'utente dalla dashboard principale accede al Modulo AI Co-Pilot per i Consulenti del Lavoro.
    \item Il Sistema mostra l’interfaccia di upload documenti, con le azioni
\end{enumerate}

\textbf{Scenario secondario}
\begin{enumerate}
    \item 
\end{enumerate}

\textbf{Trigger}
\begin{itemize}
    \item Selezionare l'opzione di modulo AI Co-Pilot per i CdL dalla dashboard principale.
\end{itemize}


\textbf{Relazioni con altri casi d'uso (\textit{include} / \textit{extend})}
\begin{itemize}
    \item \textit{include}: 
    \begin{itemize}
        \item Nessuna.
    \end{itemize}
    \item \textit{extend}: 
    \begin{itemize}
        \item UC-2A.1 - Caricamento file %Errore nel caricamento file (format non valido, dimensione eccessiva)
        \item UC-2A.2 - Inserimento categoria
        \item UC-2A.3 - Inserimento mese/anno di competenza
        \item UC-2A.4 - Inserimento azienda
        \item UC-2A.5 - Inserimento reparto
        \item UC-2A.6 - Avvio upload
    \end{itemize}
\end{itemize}

\vspace{0.5cm}

\subsubsection{UC-2A.1 – Caricamento file}

\textbf{Attori}
\begin{itemize}
    \item Utente autorizzato (Operatore di studio CdL)
\end{itemize}

\textbf{Pre-condizioni}
\begin{itemize}
    \item L'utente si trova nell'interfaccia di upload
\end{itemize}

\textbf{Post-condizioni}
\begin{itemize}
    \item Il file è stato selezionato e pre-caricato nel browser, pronto per l'associazione dei metadati (oppure viene mostrato un errore)
\end{itemize}

\textbf{Scenario principale}
\begin{enumerate}
    \item L'utente clicca sull'area di upload o trascina un file dal proprio computer
    \item Il sistema verifica che il file rispetti i requisiti di formato (es. PDF, Excel) e dimensione massima
    \item Se la verifica è positiva, il file viene mostrato in lista d'attesa
\end{enumerate}

\textbf{Eccezioni}
\begin{itemize}
    \item \textbf{Formato non valido:} L'utente tenta di caricare un'estensione non supportata. Il sistema blocca il caricamento e mostra un messaggio di errore.
    \item \textbf{Dimensione eccessiva:} Il file supera il limite di megabyte consentiti. Il sistema rifiuta il file e notifica l'utente.
\end{itemize}

\textbf{Trigger}
\begin{itemize}
    \item Selezione del file tramite "Sfoglia" o "Drag \& Drop"
\end{itemize}

\vspace{0.5cm}

\subsubsection{UC-2A.2 – Inserimento categoria}

\textbf{Attori}
\begin{itemize}
    \item Utente autorizzato (Operatore di studio CdL)
\end{itemize}

\textbf{Pre-condizioni}
\begin{itemize}
    \item Un file è stato selezionato per l'upload
\end{itemize}

\textbf{Post-condizioni}
\begin{itemize}
    \item Al file viene associata una categoria documentale (es. Cedolino, F24, Contratto)
\end{itemize}

\textbf{Scenario principale}
\begin{enumerate}
    \item L'utente interagisce con il menu a tendina "Categoria"
    \item L'utente seleziona la tipologia corretta per il documento in lavorazione
\end{enumerate}

\textbf{Trigger}
\begin{itemize}
    \item L'utente clicca sul selettore della categoria
\end{itemize}

\vspace{0.5cm}

\subsubsection{UC-2A.3 – Inserimento mese/anno di competenza}

\textbf{Attori}
\begin{itemize}
    \item Utente autorizzato (Operatore di studio CdL)
\end{itemize}

\textbf{Pre-condizioni}
\begin{itemize}
    \item Un file è stato selezionato per l'upload
\end{itemize}

\textbf{Post-condizioni}
\begin{itemize}
    \item Al file viene associato il periodo temporale di riferimento
\end{itemize}

\textbf{Scenario principale}
\begin{enumerate}
    \item L'utente inserisce o seleziona tramite date-picker il mese e l'anno a cui il documento fa riferimento
\end{enumerate}

\textbf{Trigger}
\begin{itemize}
    \item L'utente clicca sui campi Mese e Anno
\end{itemize}

\vspace{0.5cm}

\subsubsection{UC-2A.4 – Inserimento azienda}

\textbf{Attori}
\begin{itemize}
    \item Utente autorizzato (Operatore di studio CdL)
\end{itemize}

\textbf{Pre-condizioni}
\begin{itemize}
    \item Un file è stato selezionato per l'upload
\end{itemize}

\textbf{Post-condizioni}
\begin{itemize}
    \item Al file viene associata l'azienda cliente proprietaria del documento
\end{itemize}

\textbf{Scenario principale}
\begin{enumerate}
    \item L'utente inizia a digitare il nome dell'azienda o la seleziona da un elenco predefinito
    \item Il sistema collega il documento all'anagrafica aziendale selezionata
\end{enumerate}

\textbf{Trigger}
\begin{itemize}
    \item L'utente interagisce con il campo di ricerca/selezione azienda
\end{itemize}

\vspace{0.5cm}

\subsubsection{UC-2A.5 – Inserimento reparto}

\textbf{Attori}
\begin{itemize}
    \item Utente autorizzato (Operatore di studio CdL)
\end{itemize}

\textbf{Pre-condizioni}
\begin{itemize}
    \item Un file è stato selezionato e l'azienda è stata indicata
\end{itemize}

\textbf{Post-condizioni}
\begin{itemize}
    \item Al file viene associato uno specifico reparto o dipartimento (opzionale)
\end{itemize}

\textbf{Scenario principale}
\begin{enumerate}
    \item L'utente specifica il reparto di riferimento (es. Amministrazione, Produzione) se pertinente per il documento
\end{enumerate}

\textbf{Trigger}
\begin{itemize}
    \item L'utente clicca sul campo di inserimento reparto
\end{itemize}

\vspace{0.5cm}

\subsubsection{UC-2A.6 – Avvio upload}

\textbf{Attori}
\begin{itemize}
    \item Utente autorizzato (Operatore di studio CdL)
    \item Sistema di archiviazione (Cloud/Server)
\end{itemize}

\textbf{Pre-condizioni}
\begin{itemize}
    \item Il file è stato selezionato e tutti i metadati obbligatori (Categoria, Azienda, Periodo) sono stati compilati
\end{itemize}

\textbf{Post-condizioni}
\begin{itemize}
    \item Il file viene trasferito fisicamente sul server e indicizzato nel database documentale
\end{itemize}

\textbf{Scenario principale}
\begin{enumerate}
    \item L'utente conferma l'operazione premendo il pulsante di avvio upload
    \item Il sistema visualizza una barra di progresso durante il trasferimento
    \item Al termine, il sistema conferma l'avvenuto salvataggio e pulisce il modulo per un nuovo inserimento
\end{enumerate}

\textbf{Trigger}
\begin{itemize}
    \item L'utente clicca sul pulsante "Carica" o "Salva Documento"
\end{itemize}

\vspace{0.5cm}

\subsubsection{UC-2B – Apertura lista documenti.}

\begin{figure}[H]
    \centering
    \includegraphics[width=0.7\textwidth]{Diagrammi casi d'uso/UC0B.jpg}
    \caption{Didascalia dell'immagine}
\end{figure}

\textbf{Attori}
\begin{itemize}
    \item Utente autorizzato (Operatore di studio CdL).
    \item Sistema AI Doc Classifier.
\end{itemize}


\textbf{Pre-condizioni}
\begin{itemize}
    \item É stato caricato almeno un documento tramite UC-2A.6.
    \item L’utente é entrato nel modulo AI Co-Pilot per i CdL dalla dashboard principale.
    \item L’utente dispone dei permessi necessari per utilizzare il modulo AI Co-Pilot
\end{itemize}


\textbf{Post-condizioni}
\begin{itemize}
    \item L'utente puó visualizzare la lista dei documenti caricati e le relative informazioni.
\end{itemize}

\textbf{Scenario principale}
\begin{enumerate}
    \item L'utente dopo aver caricato uno o più documenti accede alla sezione "Lista documenti".
    \item Il Sistema mostra l’elenco dei documenti caricati, con le relative informazioni (tipologia, competenza, azienda, causale, lingua, numero pagine, nome originale).
\end{enumerate}

\textbf{Scenario secondario}
\begin{enumerate}
    \item Il sistema non riconosce alcun documento caricato.
\end{enumerate}

\textbf{Trigger}
\begin{itemize}
    \item Selezionare l'opzione di visualizzazione lista documenti all’interno del modulo AI Co-Pilot per i CdL.
\end{itemize}

\textbf{Relazioni con altri casi d'uso (\textit{include} / \textit{extend})}
\begin{itemize}
    \item \textit{include}: 
    \begin{itemize}
        \item UC-2B.1 – Visualizzazione codice documento
        \item UC-2B.2 - Visualizzazione tipologia documento
        \item UC-2B.3 - Visualizzazione competenza documento
        \item UC-2B.4 – Visualizzazione azienda documento
        \item UC-2B.5 - Visualizzazione causale
        \item UC-2B.6 - Visualizzazione lingua
        \item UC-2B.7 - Visualizzazione numero pagine documento
        \item UC-2B.8 - Visualizzazione nome documento originale
        \item UC-2B.9 - Visualizzazione data redazione documento
    \end{itemize}
    \item \textit{extend}: 
    \begin{itemize}
        \item UC-2B.10 - Filtraggio documenti
        \item UC-2B.11 - Nessun documento riconosciuto.
        \item UC-2C – Apertura documento
        \item UC-2D - Apertura informazioni destinatario
    \end{itemize}
\end{itemize}


\vspace{0.5cm}

\subsubsection{UC-2B.1 – Visualizzazione codice documento}

\textbf{Attori}
\begin{itemize}
    \item Utente autorizzato
\end{itemize}

\textbf{Pre-condizioni}
\begin{itemize}
    \item L'utente visualizza la lista dei documenti (incluso in UC-2B)
\end{itemize}

\textbf{Post-condizioni}
\begin{itemize}
    \item Viene visualizzato l'identificativo univoco (ID) assegnato dal sistema al documento
\end{itemize}

\textbf{Scenario principale}
\begin{enumerate}
    \item Il sistema recupera il codice identificativo dal database e lo mostra nella riga corrispondente al documento
\end{enumerate}

\textbf{Trigger}
\begin{itemize}
    \item Caricamento della lista documenti
\end{itemize}

\vspace{0.5cm}

\subsubsection{UC-2B.2 – Visualizzazione tipologia documento}

\textbf{Attori}
\begin{itemize}
    \item Utente autorizzato
\end{itemize}

\textbf{Pre-condizioni}
\begin{itemize}
    \item L'utente visualizza la lista dei documenti
\end{itemize}

\textbf{Post-condizioni}
\begin{itemize}
    \item Viene visualizzata la categoria del documento (es. Cedolino, LUL, F24)
\end{itemize}

\textbf{Scenario principale}
\begin{enumerate}
    \item Il sistema mostra l'etichetta della tipologia documentale associata al file
\end{enumerate}

\textbf{Trigger}
\begin{itemize}
    \item Caricamento della lista documenti
\end{itemize}

\vspace{0.5cm}

\subsubsection{UC-2B.3 – Visualizzazione competenza documento}

\textbf{Attori}
\begin{itemize}
    \item Utente autorizzato
\end{itemize}

\textbf{Pre-condizioni}
\begin{itemize}
    \item L'utente visualizza la lista dei documenti
\end{itemize}

\textbf{Post-condizioni}
\begin{itemize}
    \item Viene visualizzato il periodo di competenza (Mese/Anno)
\end{itemize}

\textbf{Scenario principale}
\begin{enumerate}
    \item Il sistema mostra il mese e l'anno di riferimento fiscale/contabile del documento
\end{enumerate}

\textbf{Trigger}
\begin{itemize}
    \item Caricamento della lista documenti
\end{itemize}

\vspace{0.5cm}

\subsubsection{UC-2B.4 – Visualizzazione azienda documento}

\textbf{Attori}
\begin{itemize}
    \item Utente autorizzato
\end{itemize}

\textbf{Pre-condizioni}
\begin{itemize}
    \item L'utente visualizza la lista dei documenti
\end{itemize}

\textbf{Post-condizioni}
\begin{itemize}
    \item Viene visualizzata la ragione sociale dell'azienda a cui il documento appartiene
\end{itemize}

\textbf{Scenario principale}
\begin{enumerate}
    \item Il sistema recupera il nome dell'azienda collegata e lo mostra in elenco
\end{enumerate}

\textbf{Trigger}
\begin{itemize}
    \item Caricamento della lista documenti
\end{itemize}

\vspace{0.5cm}

\subsubsection{UC-2B.5 – Visualizzazione causale}

\textbf{Attori}
\begin{itemize}
    \item Utente autorizzato
\end{itemize}

\textbf{Pre-condizioni}
\begin{itemize}
    \item L'utente visualizza la lista dei documenti
\end{itemize}

\textbf{Post-condizioni}
\begin{itemize}
    \item Viene visualizzata la descrizione o causale specifica del documento
\end{itemize}

\textbf{Scenario principale}
\begin{enumerate}
    \item Il sistema mostra eventuali note o causali inserite in fase di upload o rilevate automaticamente
\end{enumerate}

\textbf{Trigger}
\begin{itemize}
    \item Caricamento della lista documenti
\end{itemize}

\vspace{0.5cm}

\subsubsection{UC-2B.6 – Visualizzazione lingua}

\textbf{Attori}
\begin{itemize}
    \item Utente autorizzato
\end{itemize}

\textbf{Pre-condizioni}
\begin{itemize}
    \item L'utente visualizza la lista dei documenti
\end{itemize}

\textbf{Post-condizioni}
\begin{itemize}
    \item Viene indicata la lingua principale del contenuto del documento
\end{itemize}

\textbf{Scenario principale}
\begin{enumerate}
    \item Il sistema mostra un indicatore (es. bandiera o codice ISO) della lingua rilevata
\end{enumerate}

\textbf{Trigger}
\begin{itemize}
    \item Caricamento della lista documenti
\end{itemize}

\vspace{0.5cm}

\subsubsection{UC-2B.7 – Visualizzazione numero pagine documento}

\textbf{Attori}
\begin{itemize}
    \item Utente autorizzato
\end{itemize}

\textbf{Pre-condizioni}
\begin{itemize}
    \item L'utente visualizza la lista dei documenti
\end{itemize}

\textbf{Post-condizioni}
\begin{itemize}
    \item Viene mostrato il conteggio totale delle pagine che compongono il file
\end{itemize}

\textbf{Scenario principale}
\begin{enumerate}
    \item Il sistema calcola e visualizza il numero di pagine del documento PDF o immagine
\end{enumerate}

\textbf{Trigger}
\begin{itemize}
    \item Caricamento della lista documenti
\end{itemize}

\vspace{0.5cm}

\subsubsection{UC-2B.8 – Visualizzazione nome documento originale}

\textbf{Attori}
\begin{itemize}
    \item Utente autorizzato
\end{itemize}

\textbf{Pre-condizioni}
\begin{itemize}
    \item L'utente visualizza la lista dei documenti
\end{itemize}

\textbf{Post-condizioni}
\begin{itemize}
    \item Viene visualizzato il nome del file originale caricato dall'utente
\end{itemize}

\textbf{Scenario principale}
\begin{enumerate}
    \item Il sistema mostra il filename originale per facilitare il riconoscimento
\end{enumerate}

\textbf{Trigger}
\begin{itemize}
    \item Caricamento della lista documenti
\end{itemize}

\vspace{0.5cm}

\subsubsection{UC-2B.9 – Visualizzazione data redazione documento}

\textbf{Attori}
\begin{itemize}
    \item Utente autorizzato
\end{itemize}

\textbf{Pre-condizioni}
\begin{itemize}
    \item L'utente visualizza la lista dei documenti
\end{itemize}

\textbf{Post-condizioni}
\begin{itemize}
    \item Viene visualizzata la data in cui il documento è stato creato o caricato a sistema
\end{itemize}

\textbf{Scenario principale}
\begin{enumerate}
    \item Il sistema recupera il timestamp di creazione o upload e lo mostra formattato
\end{enumerate}

\textbf{Trigger}
\begin{itemize}
    \item Caricamento della lista documenti
\end{itemize}

\vspace{0.5cm}

\subsubsection{UC-2B.10 – Filtraggio documenti}

\textbf{Attori}
\begin{itemize}
    \item Utente autorizzato
\end{itemize}

\textbf{Pre-condizioni}
\begin{itemize}
    \item La lista dei documenti è visibile e contiene più elementi
\end{itemize}

\textbf{Post-condizioni}
\begin{itemize}
    \item La lista viene aggiornata mostrando solo i documenti che soddisfano i criteri selezionati
\end{itemize}

\textbf{Scenario principale}
\begin{enumerate}
    \item L'utente imposta uno o più filtri (es. per Azienda, per Periodo di competenza, per Tipologia)
    \item Il sistema elabora la richiesta e ridisegna la tabella mostrando solo i risultati pertinenti
\end{enumerate}

\textbf{Trigger}
\begin{itemize}
    \item L'utente clicca sul pulsante "Filtra" o modifica i campi di filtro
\end{itemize}

\vspace{0.5cm}

\subsubsection{UC-2B.11 – Nessun documento riconosciuto}

\textbf{Attori}
\begin{itemize}
    \item Utente autorizzato
\end{itemize}

\textbf{Pre-condizioni}
\begin{itemize}
    \item L'utente accede alla lista o effettua una ricerca/filtro
\end{itemize}

\textbf{Post-condizioni}
\begin{itemize}
    \item Viene mostrato un messaggio che informa l'utente dell'assenza di documenti
\end{itemize}

\textbf{Scenario principale}
\begin{enumerate}
    \item Il sistema interroga il database in base ai criteri correnti (o visualizzazione generale)
    \item Non trovando corrispondenze (o se l'archivio è vuoto), il sistema mostra un placeholder grafico o testuale "Nessun documento trovato"
\end{enumerate}

\textbf{Trigger}
\begin{itemize}
    \item Accesso alla lista vuota o applicazione di filtri troppo restrittivi
\end{itemize}

\vspace{0.5cm}

\subsubsection{UC-2C – Apertura documento}

\begin{figure}[H]
    \centering
    \includegraphics[width=0.7\textwidth]{Diagrammi casi d'uso/UC0B.jpg}
    \caption{Didascalia dell'immagine}
\end{figure}

\textbf{Attori}
\begin{itemize}
    \item Utente autorizzato (Operatore di studio CdL).
\end{itemize}


\textbf{Pre-condizioni}
\begin{itemize}
    \item Ci si trova in una delle pagine precedenti (UC-2B o UC-2D o UC-2E).
    \item Si é selezionato un documento da una delle lista.
    \item Il documento é disponibile per la visualizzazione.
\end{itemize}

\textbf{Post-condizioni}
\begin{itemize}
    \item L'utente ha apportato le modifiche desiderate al documento e puó procedere con le azioni successive (salvataggio, invio, ecc.).
    \item L'utente é giá soddisfatto delle informazioni del documento e puó tornare alla pagina precedente.
\end{itemize}



\textbf{Scenario principale}
\begin{enumerate}
    \item L'utente ha caricato vari documenti e vuole ricontrollare le informazioni prima di inviarli.
    \item L'utente seleziona un documento dalla lista per visualizzarne i dettagli.
\end{enumerate}

\textbf{Scenario secondario}
\begin{enumerate}
    \item 
\end{enumerate}

\textbf{Trigger}
\begin{itemize}
    \item Selezionare un documento dalla lista per visualizzarne i dettagli.
\end{itemize}


\textbf{Relazioni con altri casi d'uso (\textit{include} / \textit{extend})}
\begin{itemize}
    \item \textit{include}: 
    \begin{itemize}
        \item UC-2B.1 – Visualizzazione codice documento
        \item UC-2B.2 - Visualizzazione tipologia documento
        \item UC-2D.1 – Visualizzazione destinatario
        \item UC-2B.8 - Visualizzazione nome documento originale
        \item UC-2C.1 - Visualizzazione anteprima documento
        \item UC-2E.2 - Visualizzazione percentuale confidenza
    \end{itemize}
    \item \textit{extend}: 
    \begin{itemize}
        \item UC-2C.2 – Modifica destinatario
        \item UC-2C.3 – Modifica tipologia documento
        \item UC-2C.4 – Rivaluta percentuale confidenza
        \item UC-2C.5 – Salva modifiche documento
        \item UC-2B – Apertura lista documenti.
        \item UC-2D - Apertura informazioni destinatario
        \item UC-2E - Apertura storico documenti
    \end{itemize}
\end{itemize}

\vspace{0.5cm}

\subsubsection{UC-2C.1 – Visualizzazione anteprima documento}

\textbf{Attori}
\begin{itemize}
    \item Utente autorizzato (Operatore di studio CdL).
\end{itemize}

\textbf{Pre-condizioni}
\begin{itemize}
    \item L'utente ha effettuato l'accesso al dettaglio del documento (UC-2C).
    \item Il file fisico o digitale del documento è accessibile nel sistema di storage.
\end{itemize}

\textbf{Post-condizioni}
\begin{itemize}
    \item L'anteprima del documento viene mostrata correttamente a video permettendo il confronto con i dati estratti.
\end{itemize}

\textbf{Scenario principale}
\begin{enumerate}
    \item Il sistema recupera il file sorgente (PDF o immagine).
    \item Il sistema renderizza l'anteprima del documento nell'area dedicata della pagina di dettaglio.
\end{enumerate}

\textbf{Trigger}
\begin{itemize}
    \item Apertura della pagina di dettaglio del documento (automatico all'avvio di UC-2C).
\end{itemize}

\vspace{0.5cm}

\subsubsection{UC-2C.2 – Modifica destinatario}

\textbf{Attori}
\begin{itemize}
    \item Utente autorizzato (Operatore di studio CdL).
\end{itemize}

\textbf{Pre-condizioni}
\begin{itemize}
    \item Si è all'interno della visualizzazione di dettaglio (UC-2C).
    \item Il campo destinatario è modificabile.
\end{itemize}

\textbf{Post-condizioni}
\begin{itemize}
    \item Il documento è associato provvisoriamente a un nuovo destinatario (in attesa di salvataggio).
\end{itemize}

\textbf{Scenario principale}
\begin{enumerate}
    \item L'utente nota che il destinatario assegnato automaticamente o precedentemente è errato o mancante.
    \item L'utente attiva il campo di modifica del destinatario.
    \item L'utente ricerca e seleziona il destinatario corretto dall'anagrafica.
    \item Il sistema aggiorna il campo visualizzato con il nuovo valore.
\end{enumerate}

\textbf{Trigger}
\begin{itemize}
    \item Click sull'icona di modifica o sul campo del destinatario.
\end{itemize}

\vspace{0.5cm}

\subsubsection{UC-2C.3 – Modifica tipologia documento}

\textbf{Attori}
\begin{itemize}
    \item Utente autorizzato (Operatore di studio CdL).
\end{itemize}

\textbf{Pre-condizioni}
\begin{itemize}
    \item Si è all'interno della visualizzazione di dettaglio (UC-2C).
\end{itemize}

\textbf{Post-condizioni}
\begin{itemize}
    \item La tipologia del documento è aggiornata.
    \item I campi dati specifici per la nuova tipologia vengono predisposti.
\end{itemize}

\textbf{Scenario principale}
\begin{enumerate}
    \item L'utente rileva che la classificazione automatica del documento è errata (es. classificato come "Fattura" invece di "Contratto").
    \item L'utente seleziona la nuova tipologia da un menu a tendina.
    \item Il sistema aggiorna la tipologia.
\end{enumerate}

\textbf{Trigger}
\begin{itemize}
    \item Selezione di una nuova categoria dal menu "Tipologia documento".
\end{itemize}

\vspace{0.5cm}

\subsubsection{UC-2C.4 – Rivaluta percentuale confidenza}

\textbf{Attori}
\begin{itemize}
    \item Utente autorizzato (Operatore di studio CdL).
    \item Sistema (Attore secondario).
\end{itemize}

\textbf{Pre-condizioni}
\begin{itemize}
    \item L'utente ha apportato modifiche sostanziali al documento (es. UC-2C.3 Modifica tipologia).
\end{itemize}

\textbf{Post-condizioni}
\begin{itemize}
    \item La percentuale di confidenza viene ricalcolata o impostata al 100\% (validazione manuale).
\end{itemize}

\textbf{Scenario principale}
\begin{enumerate}
    \item L'utente ha modificato la tipologia del documento (UC-2C.3).
    \item Il sistema ri-analizza i campi in base alla nuova struttura dati prevista.
    \item Il sistema aggiorna la percentuale di confidenza per riflettere la nuova coerenza dei dati o la validazione manuale dell'utente.
\end{enumerate}

\textbf{Trigger}
\begin{itemize}
    \item Completamento di UC-2C.3 o azione esplicita "Rivaluta dati".
\end{itemize}
\vspace{0.5cm}

\subsubsection{UC-2C.5 – Salva modifiche documento}

\textbf{Attori}
\begin{itemize}
    \item Utente autorizzato (Operatore di studio CdL).
\end{itemize}

\textbf{Pre-condizioni}
\begin{itemize}
    \item Sono state apportate modifiche ai metadati del documento (destinatario, tipologia, note, ecc.).
    \item I dati inseriti rispettano i vincoli di validazione.
\end{itemize}

\textbf{Post-condizioni}
\begin{itemize}
    \item Le modifiche sono persistite nel database.
    \item Il documento viene marcato come "Validato" o "Processato".
    \item L'utente viene riportato alla lista documenti o rimane sulla pagina con notifica di successo.
\end{itemize}

\textbf{Scenario principale}
\begin{enumerate}
    \item L'utente clicca sul pulsante di salvataggio/conferma.
    \item Il sistema valida la coerenza dei dati inseriti.
    \item Il sistema salva le modifiche nel database.
    \item Il sistema notifica all'utente l'avvenuto salvataggio.
\end{enumerate}

\textbf{Scenario secondario}
\begin{enumerate}
    \item Il sistema rileva dati mancanti o incongruenti (Errore validazione).
    \item Il salvataggio viene bloccato e viene mostrato un messaggio di errore all'utente.
\end{enumerate}

\textbf{Trigger}
\begin{itemize}
    \item Click sul pulsante "Salva" o "Conferma modifiche".
\end{itemize}

\subsubsection{UC-2D - Apertura informazioni destinatario }

\begin{figure}[H]
    \centering
    \includegraphics[width=0.7\textwidth]{Diagrammi casi d'uso/UC0B.jpg}
    \caption{Didascalia dell'immagine}
\end{figure}

\textbf{Attori}
\begin{itemize}
    \item Utente autorizzato (Operatore di studio CdL).
\end{itemize}


\textbf{Pre-condizioni}
\begin{itemize}
    \item Si é nella pagina lista documenti tramite UC-2B.
\end{itemize}

\textbf{Post-condizioni}
\begin{itemize}
    \item L'utente visualizza le informazioni dettagliate del destinatario associate al documento selezionato.
    \item L'utente puó nuovamente accedere alla pagina del documento tramite UC-2C.
\end{itemize}

\textbf{Scenario principale}
\begin{enumerate}
    \item L'utente ha finito di visualizzare le informazioni del documento e seleziona l’opzione per visualizzare i dettagli del destinatario.
\end{enumerate}

\textbf{Scenario secondario}
\begin{enumerate}
    \item 
\end{enumerate}

\textbf{Trigger}
\begin{itemize}
    \item Selezionare l'opzione di visualizzazione dettagli destinatario dalla pagina lista documenti.
\end{itemize}


\textbf{Relazioni con altri casi d'uso (\textit{include} / \textit{extend})}
\begin{itemize}
    \item \textit{include}: 
    \begin{itemize}
        \item UC-2B.1 – Visualizzazione codice documento
        \item UC-2D.1 – Visualizzazione destinatario
        \item UC-2D.2 - Visualizzazione codice fiscale
        \item UC-2D.3 - Visualizzazione matricola
        \item UC-2D.4 - Visualizzazione reparto
    \end{itemize}
    \item \textit{extend}: 
    \begin{itemize}
        \item UC-2B – Visualizzazione pagina lista documenti.
        \item UC-2C – Apertura documento
        \item UC-2E - Apertura storico documenti
    \end{itemize}
\end{itemize}

\vspace{0.5cm}

\subsubsection{UC-2D.1 – Visualizzazione destinatario}

\textbf{Attori}
\begin{itemize}
    \item Utente autorizzato (Operatore di studio CdL).
\end{itemize}

\textbf{Pre-condizioni}
\begin{itemize}
    \item L'utente ha acceduto alla pagina di informazioni del destinatario (UC-2D).
    \item Il destinatario è correttamente associato al documento nel sistema.
\end{itemize}

\textbf{Post-condizioni}
\begin{itemize}
    \item Il nome e cognome (o ragione sociale) del destinatario sono visibili a schermo.
\end{itemize}

\textbf{Scenario principale}
\begin{enumerate}
    \item Il sistema recupera l'identificativo del destinatario associato al documento selezionato.
    \item Il sistema interroga l'anagrafica per ottenere i dati anagrafici principali.
    \item Il sistema mostra il nome del destinatario nell'intestazione della scheda.
\end{enumerate}

\textbf{Trigger}
\begin{itemize}
    \item Caricamento della pagina UC-2D (Automatico).
\end{itemize}

\vspace{0.5cm}

\subsubsection{UC-2D.2 – Visualizzazione codice fiscale}

\textbf{Attori}
\begin{itemize}
    \item Utente autorizzato (Operatore di studio CdL).
\end{itemize}

\textbf{Pre-condizioni}
\begin{itemize}
    \item L'utente ha acceduto alla pagina di informazioni del destinatario (UC-2D).
\end{itemize}

\textbf{Post-condizioni}
\begin{itemize}
    \item Il codice fiscale del destinatario è visibile per la verifica dell'identità.
\end{itemize}

\textbf{Scenario principale}
\begin{enumerate}
    \item Il sistema recupera il codice fiscale dal record anagrafico del destinatario.
    \item Il sistema formatta e visualizza il codice fiscale nel campo dedicato.
\end{enumerate}

\textbf{Trigger}
\begin{itemize}
    \item Caricamento della pagina UC-2D (Automatico).
\end{itemize}

\vspace{0.5cm}

\subsubsection{UC-2D.3 – Visualizzazione matricola}

\textbf{Attori}
\begin{itemize}
    \item Utente autorizzato (Operatore di studio CdL).
\end{itemize}

\textbf{Pre-condizioni}
\begin{itemize}
    \item L'utente ha acceduto alla pagina di informazioni del destinatario (UC-2D).
    \item Il destinatario è un dipendente censito con numero di matricola.
\end{itemize}

\textbf{Post-condizioni}
\begin{itemize}
    \item Il numero di matricola è visualizzato correttamente.
\end{itemize}

\textbf{Scenario principale}
\begin{enumerate}
    \item Il sistema verifica se al destinatario è associato un numero di matricola aziendale.
    \item Il sistema mostra il numero di matricola per permettere la correlazione con i sistemi paghe.
\end{enumerate}

\textbf{Scenario secondario}
\begin{enumerate}
    \item Il destinatario non possiede una matricola (es. collaboratore esterno).
    \item Il campo viene mostrato vuoto o con dicitura "N/D".
\end{enumerate}

\textbf{Trigger}
\begin{itemize}
    \item Caricamento della pagina UC-2D (Automatico).
\end{itemize}

\vspace{0.5cm}

\subsubsection{UC-2D.4 – Visualizzazione reparto}

\textbf{Attori}
\begin{itemize}
    \item Utente autorizzato (Operatore di studio CdL).
\end{itemize}

\textbf{Pre-condizioni}
\begin{itemize}
    \item L'utente ha acceduto alla pagina di informazioni del destinatario (UC-2D).
\end{itemize}

\textbf{Post-condizioni}
\begin{itemize}
    \item L'utente visualizza il reparto o l'unità operativa di appartenenza del destinatario.
\end{itemize}

\textbf{Scenario principale}
\begin{enumerate}
    \item Il sistema recupera le informazioni sull'organigramma o l'assegnazione del destinatario.
    \item Il sistema visualizza il nome del reparto di afferenza.
\end{enumerate}

\textbf{Trigger}
\begin{itemize}
    \item Caricamento della pagina UC-2D (Automatico).
\end{itemize}

\subsubsection{UC-2E – Apertura storico documenti}

\begin{figure}[H]
    \centering
    \includegraphics[width=0.7\textwidth]{Diagrammi casi d'uso/UC0B.jpg}
    \caption{Didascalia dell'immagine}
\end{figure}

\textbf{Attori}
\begin{itemize}
    \item Utente autorizzato (Operatore di studio CdL).
\end{itemize}

\textbf{Pre-condizioni}
\begin{itemize}
    \item L'utente si trova nella pagina destinatario tramite UC-2D.
    \item L'utente ha selezionato l’opzione per inviare il documento al destinatario.
\end{itemize}

\textbf{Post-condizioni}
\begin{itemize}
    \item L'utente va alla pagina di gestione dell'invio e del rispettivo messaggio.
\end{itemize}

\textbf{Scenario principale}
\begin{enumerate}
    \item L'utente dopo aver fatto le opportune verifiche sul documento e sul destinatario seleziona l’opzione per inviare il documento.
\end{enumerate}

\textbf{Scenario secondario}
\begin{enumerate}
    \item Non sono mai stati caricati documenti, quindi la visualizzazione di dati non è possibile.
\end{enumerate}

\textbf{Trigger}
\begin{itemize}
    \item Selezionare l'opzione di invio documento dalla pagina destinatario.
\end{itemize}

\textbf{Relazioni con altri casi d'uso (\textit{include} / \textit{extend})}
\begin{itemize}
    \item \textit{include}: 
    \begin{itemize}
        \item nessuna
    \end{itemize}
    \item \textit{extend}: 
    \begin{itemize}
        \item UC-2E.3 - Filtraggio documenti
        \item UC-2E.4 - visualizzazione informazioni documento
        \item UC-2C – Apertura documento
        \item UC-2D - Apertura informazioni destinatario
        \item UC-2E.1 - Selezione documento
        \item UC-2F - Gestione messaggio
        \item UC-2E.2 - Documenti assenti.
    \end{itemize}
\end{itemize}

\vspace{0.5cm}

\subsubsection{UC-2E.1 – Selezione documento}

\textbf{Attori}
\begin{itemize}
    \item Utente autorizzato (Operatore di studio CdL).
\end{itemize}

\textbf{Pre-condizioni}
\begin{itemize}
    \item L'utente visualizza la lista dello storico documenti (UC-2E).
    \item È presente almeno un documento in lista.
\end{itemize}

\textbf{Post-condizioni}
\begin{itemize}
    \item Il sistema apre il dettaglio del documento selezionato (reindirizzando a UC-2C).
\end{itemize}

\textbf{Scenario principale}
\begin{enumerate}
    \item L'utente individua il documento di interesse scorrendo la lista.
    \item L'utente clicca sulla riga corrispondente o sull'icona di dettaglio.
    \item Il sistema naviga alla pagina di dettaglio del documento.
\end{enumerate}

\textbf{Trigger}
\begin{itemize}
    \item Click su un elemento della lista documenti.
\end{itemize}

\vspace{0.5cm}

\subsubsection{UC-2E.2 – Documenti assenti}

\textbf{Attori}
\begin{itemize}
    \item Sistema.
\end{itemize}

\textbf{Pre-condizioni}
\begin{itemize}
    \item L'utente ha richiesto l'accesso allo storico documenti (UC-2E).
    \item Non esistono documenti associati al destinatario o ai filtri correnti.
\end{itemize}

\textbf{Post-condizioni}
\begin{itemize}
    \item Viene mostrato un messaggio informativo che indica l'assenza di risultati.
    \item La lista appare vuota.
\end{itemize}

\textbf{Scenario principale}
\begin{enumerate}
    \item Il sistema interroga il database per recuperare i documenti.
    \item La ricerca non produce risultati (0 record).
    \item Il sistema visualizza un placeholder o un messaggio "Nessun documento trovato" per informare l'utente.
\end{enumerate}

\textbf{Trigger}
\begin{itemize}
    \item Caricamento della pagina (automatico) o applicazione di un filtro troppo restrittivo.
\end{itemize}

\vspace{0.5cm}

\subsubsection{UC-2E.3 – Filtraggio documenti}

\textbf{Attori}
\begin{itemize}
    \item Utente autorizzato (Operatore di studio CdL).
\end{itemize}

\textbf{Pre-condizioni}
\begin{itemize}
    \item La lista dei documenti è visualizzata e popolata.
\end{itemize}

\textbf{Post-condizioni}
\begin{itemize}
    \item La lista mostra solo i documenti che soddisfano i criteri di ricerca inseriti.
\end{itemize}

\textbf{Scenario principale}
\begin{enumerate}
    \item L'utente desidera restringere la ricerca (es. per data, per tipologia o per stato invio).
    \item L'utente utilizza la barra di ricerca o i filtri laterali.
    \item Il sistema aggiorna la vista in tempo reale o dopo la conferma, nascondendo i documenti non pertinenti.
\end{enumerate}

\textbf{Scenario secondario}
\begin{enumerate}
    \item L'utente rimuove i filtri.
    \item Il sistema ripristina la visualizzazione completa dello storico.
\end{enumerate}

\textbf{Trigger}
\begin{itemize}
    \item Inserimento testo nella barra di ricerca o selezione di un filtro.
\end{itemize}

\vspace{0.5cm}

\subsubsection{UC-2E.4 – Visualizzazione informazioni documento}

\textbf{Attori}
\begin{itemize}
    \item Utente autorizzato (Operatore di studio CdL).
\end{itemize}

\textbf{Pre-condizioni}
\begin{itemize}
    \item L'utente si trova nella pagina dello storico documenti (UC-2E).
    \item I documenti sono caricati in lista.
\end{itemize}

\textbf{Post-condizioni}
\begin{itemize}
    \item Per ogni riga della lista vengono mostrati i metadati essenziali del documento.
\end{itemize}

\textbf{Scenario principale}
\begin{enumerate}
    \item Il sistema recupera i metadati principali per ogni documento in elenco.
    \item Il sistema renderizza le colonne della tabella mostrando: Codice documento, Data caricamento, Stato elaborazione e Percentuale di confidenza.
\end{enumerate}

\textbf{Trigger}
\begin{itemize}
    \item Caricamento della pagina UC-2E (Automatico).
\end{itemize}

\textbf{Relazioni con altri casi d'uso (\textit{include} / \textit{extend})}
\begin{itemize}
    \item \textit{include}: 
    \begin{itemize}
        \item UC-2B.1 – Visualizzazione codice documento
        \item UC-2E.5 - Visualizzazione stato
        \item UC-2E.6 - Visualizzazione percentuale confidenza
        \item UC-2E.7 - Visualizzazione lista distribuzione
    \end{itemize}
    \item \textit{extend}: 
    \begin{itemize}
        \item Nessuna
    \end{itemize}
\end{itemize}

\vspace{0.5cm}

\subsubsection{UC-2E.5 – Visualizzazione stato}

\textbf{Attori}
\begin{itemize}
    \item Utente autorizzato (Operatore di studio CdL).
\end{itemize}

\textbf{Pre-condizioni}
\begin{itemize}
    \item L'utente sta visualizzando la lista o il dettaglio dei documenti (tramite UC-2E o UC-2E.4).
\end{itemize}

\textbf{Post-condizioni}
\begin{itemize}
    \item L'utente conosce lo stato attuale del ciclo di vita del documento (es. "Da validare", "Pronto per l'invio", "Inviato", "Errore").
\end{itemize}

\textbf{Scenario principale}
\begin{enumerate}
    \item Il sistema interroga il workflow del documento per determinarne lo stato corrente.
    \item Il sistema visualizza un'etichetta testuale o un'icona colorata rappresentativa dello stato (es. Verde per inviato, Giallo per in attesa).
\end{enumerate}

\textbf{Trigger}
\begin{itemize}
    \item Caricamento della pagina (Automatico).
\end{itemize}

\vspace{0.5cm}

\subsubsection{UC-2E.6 – Visualizzazione percentuale confidenza}

\textbf{Attori}
\begin{itemize}
    \item Utente autorizzato (Operatore di studio CdL).
\end{itemize}

\textbf{Pre-condizioni}
\begin{itemize}
    \item Il documento è stato elaborato dal motore di analisi (OCR/AI).
    \item L'utente sta visualizzando le informazioni del documento.
\end{itemize}

\textbf{Post-condizioni}
\begin{itemize}
    \item Viene mostrato un indicatore numerico o visivo dell'affidabilità dei dati estratti.
\end{itemize}

\textbf{Scenario principale}
\begin{enumerate}
    \item Il sistema recupera il punteggio di confidenza calcolato durante l'importazione.
    \item Il sistema formatta il dato come percentuale.
    \item Il sistema visualizza il dato, evidenziandolo (es. in rosso) se la confidenza è sotto una soglia di sicurezza, suggerendo un controllo manuale.
\end{enumerate}

\textbf{Trigger}
\begin{itemize}
    \item Caricamento della pagina (Automatico).
\end{itemize}

\vspace{0.5cm}

\subsubsection{UC-2E.7 – Visualizzazione lista distribuzione}

\textbf{Attori}
\begin{itemize}
    \item Utente autorizzato (Operatore di studio CdL).
\end{itemize}

\textbf{Pre-condizioni}
\begin{itemize}
    \item Al documento sono stati associati uno o più destinatari.
    \item L'utente sta visualizzando i dettagli del documento nello storico.
\end{itemize}

\textbf{Post-condizioni}
\begin{itemize}
    \item L'utente visualizza chi riceverà o ha ricevuto il documento.
\end{itemize}

\textbf{Scenario principale}
\begin{enumerate}
    \item Il sistema recupera i collegamenti tra il documento e le anagrafiche destinatari.
    \item Il sistema mostra i nomi dei destinatari o, in caso di lista lunga, un riepilogo (es. "Mario Rossi + 2 altri") espandibile al passaggio del mouse o al click.
\end{enumerate}

\textbf{Trigger}
\begin{itemize}
    \item Caricamento della pagina (Automatico).
\end{itemize}

\subsubsection{UC-2F – Gestione messaggio}

\begin{figure}[H]
    \centering
    \includegraphics[width=0.7\textwidth]{Diagrammi casi d'uso/UC0B.jpg}
    \caption{Didascalia dell'immagine}
\end{figure}

\textbf{Attori}
\begin{itemize}
    \item Utente autorizzato (Operatore di studio CdL).
    \item Sistema di invio documenti.
\end{itemize}

\textbf{Pre-condizioni}
\begin{itemize}
    \item Ci si trova nella pagina gestione messaggio tramite UC-2E.
    \item Almeno un documento è stato selezionato per l'invio.
\end{itemize}

\textbf{Post-condizioni}
\begin{itemize}
    \item Il messaggio é pronto per essere inviato assieme ai documenti selezionati.
\end{itemize}

\textbf{Scenario principale}
\begin{enumerate}
    \item Un utente ha selezionato uno o piú documenti da inviare a uno o piú destinatari e sta prepara in messaggio da inviare assieme ai documenti.
\end{enumerate}

\textbf{Scenario secondario}
\begin{enumerate}
    \item 
\end{enumerate}

\textbf{Trigger}
\begin{itemize}
    \item Selezionare l'opzione di gestione messaggio dalla pagina storico documenti.
\end{itemize}


\textbf{Relazioni con altri casi d'uso (\textit{include} / \textit{extend})}
\begin{itemize}
    \item \textit{include}: 
    \begin{itemize}
        \item UC-2B.1 – Visualizzazione codice documento
        \item UC-2B.4 – Visualizzazione azienda documento
        \item UC-2B.9 - Visualizzazione data redazione documento
        \item UC-2F.6 - Visualizzazione oggetto messaggio
        \item UC-2F.7 - Visualizzazione testo messaggio
    \end{itemize}
    \item \textit{extend}: 
    \begin{itemize}
        \item UC-2F.1 - Salva template messaggio
        \item UC-2G - Tabella template salvati
        \item UC-1A.2 - Selezione tono
        \item UC-1A.3 - Selezione stile
        \item UC-2F.2 - Modifica oggetto messaggio
        \item UC-2F.3 - Modifica testo messaggio
        \item UC-2F.4 - Conferma modifica messaggio
        \item UC-2F.5 - Genera messaggio
        \item UC-2E – Apertura storico documenti
        \item UC-2H – Invio documento e messaggio
    \end{itemize}
\end{itemize}

\vspace{0.5cm}

\subsubsection{UC-2F.1 – Salva template messaggio}

\textbf{Attori}
\begin{itemize}
    \item Utente autorizzato (Operatore di studio CdL).
\end{itemize}

\textbf{Pre-condizioni}
\begin{itemize}
    \item L'utente ha modificato o generato un messaggio che ritiene utile per usi futuri.
    \item Il campo testo del messaggio non è vuoto.
\end{itemize}

\textbf{Post-condizioni}
\begin{itemize}
    \item Il contenuto del messaggio viene salvato nel database dei template personali o globali.
\end{itemize}

\textbf{Scenario principale}
\begin{enumerate}
    \item L'utente, soddisfatto del messaggio attuale, clicca sul pulsante "Salva come template".
    \item Il sistema richiede di assegnare un nome al nuovo template.
    \item L'utente inserisce il nome e conferma.
    \item Il sistema salva il template rendendolo disponibile per futuri utilizzi (vedi UC-2G).
\end{enumerate}

\textbf{Trigger}
\begin{itemize}
    \item Click sul pulsante "Salva come template".
\end{itemize}

\vspace{0.5cm}

\subsubsection{UC-2F.2 – Modifica oggetto messaggio}

\textbf{Attori}
\begin{itemize}
    \item Utente autorizzato (Operatore di studio CdL).
\end{itemize}

\textbf{Pre-condizioni}
\begin{itemize}
    \item L'utente si trova nella pagina di gestione messaggio (UC-2F).
\end{itemize}

\textbf{Post-condizioni}
\begin{itemize}
    \item L'oggetto della mail/messaggio è aggiornato con il testo inserito dall'utente.
\end{itemize}

\textbf{Scenario principale}
\begin{enumerate}
    \item L'utente seleziona il campo di input relativo all'oggetto del messaggio.
    \item L'utente digita o modifica il testo esistente (es. aggiungendo riferimenti specifici).
\end{enumerate}

\textbf{Trigger}
\begin{itemize}
    \item Click o focus sul campo di input "Oggetto".
\end{itemize}

\vspace{0.5cm}

\subsubsection{UC-2F.3 – Modifica testo messaggio}

\textbf{Attori}
\begin{itemize}
    \item Utente autorizzato (Operatore di studio CdL).
\end{itemize}

\textbf{Pre-condizioni}
\begin{itemize}
    \item L'utente si trova nella pagina di gestione messaggio (UC-2F).
\end{itemize}

\textbf{Post-condizioni}
\begin{itemize}
    \item Il corpo del messaggio è aggiornato.
\end{itemize}

\textbf{Scenario principale}
\begin{enumerate}
    \item L'utente seleziona l'area di testo contenente il corpo del messaggio.
    \item L'utente apporta modifiche manuali, aggiunge note o corregge il testo generato/preimpostato.
\end{enumerate}

\textbf{Trigger}
\begin{itemize}
    \item Click o focus sull'area di testo del messaggio ("Body").
\end{itemize}

\vspace{0.5cm}

\subsubsection{UC-2F.4 – Conferma modifica messaggio}

\textbf{Attori}
\begin{itemize}
    \item Utente autorizzato (Operatore di studio CdL).
\end{itemize}

\textbf{Pre-condizioni}
\begin{itemize}
    \item Sono state apportate modifiche ai campi oggetto o testo (UC-2F.2 o UC-2F.3).
\end{itemize}

\textbf{Post-condizioni}
\begin{itemize}
    \item Le modifiche vengono validate temporaneamente in preparazione all'invio.
\end{itemize}

\textbf{Scenario principale}
\begin{enumerate}
    \item L'utente termina la digitazione.
    \item L'utente clicca fuori dall'area di testo o preme un pulsante di conferma parziale.
    \item Il sistema mantiene in memoria la versione aggiornata del messaggio pronta per l'invio (UC-2H).
\end{enumerate}

\textbf{Trigger}
\begin{itemize}
    \item Evento di \textit{blur} (perdita del focus) dai campi di testo o click su "Conferma/Applica".
\end{itemize}

\vspace{0.5cm}

\subsubsection{UC-2F.5 – Genera messaggio}

\textbf{Attori}
\begin{itemize}
    \item Sistema (AI/Generatore di template).
    \item Utente autorizzato (Operatore di studio CdL).
\end{itemize}

\textbf{Pre-condizioni}
\begin{itemize}
    \item Sono stati selezionati i parametri di generazione (es. tono, stile) o è stato richiesto un suggerimento automatico.
\end{itemize}

\textbf{Post-condizioni}
\begin{itemize}
    \item I campi Oggetto e Testo vengono popolati automaticamente dal sistema.
\end{itemize}

\textbf{Scenario principale}
\begin{enumerate}
    \item L'utente clicca sul pulsante "Genera messaggio" (eventualmente dopo aver selezionato tono e stile in UC-1A).
    \item Il sistema elabora i dati del documento e del destinatario.
    \item Il sistema crea una proposta di testo coerente con il contesto.
    \item Il sistema sovrascrive o riempie i campi di visualizzazione.
\end{enumerate}

\textbf{Trigger}
\begin{itemize}
    \item Click sul pulsante "Genera messaggio" o "Crea bozza con IA".
\end{itemize}

\vspace{0.5cm}

\subsubsection{UC-2F.6 – Visualizzazione oggetto messaggio}

\textbf{Attori}
\begin{itemize}
    \item Utente autorizzato (Operatore di studio CdL).
\end{itemize}

\textbf{Pre-condizioni}
\begin{itemize}
    \item Accesso alla pagina di gestione messaggio (UC-2F).
\end{itemize}

\textbf{Post-condizioni}
\begin{itemize}
    \item L'oggetto corrente (vuoto, di default o generato) è visibile.
\end{itemize}

\textbf{Scenario principale}
\begin{enumerate}
    \item Il sistema recupera l'eventuale oggetto predefinito basato sulla tipologia di documento.
    \item Il sistema mostra il contenuto nel campo dedicato.
\end{enumerate}

\textbf{Trigger}
\begin{itemize}
    \item Caricamento della pagina UC-2F (Automatico).
\end{itemize}

\vspace{0.5cm}

\subsubsection{UC-2F.7 – Visualizzazione testo messaggio}

\textbf{Attori}
\begin{itemize}
    \item Utente autorizzato (Operatore di studio CdL).
\end{itemize}

\textbf{Pre-condizioni}
\begin{itemize}
    \item Accesso alla pagina di gestione messaggio (UC-2F).
\end{itemize}

\textbf{Post-condizioni}
\begin{itemize}
    \item Il corpo del messaggio è visibile e leggibile.
\end{itemize}

\textbf{Scenario principale}
\begin{enumerate}
    \item Il sistema recupera il testo (vuoto o pre-generato).
    \item Il sistema renderizza l'area di testo permettendo la lettura del contenuto.
\end{enumerate}

\textbf{Trigger}
\begin{itemize}
    \item Caricamento della pagina UC-2F (Automatico).
\end{itemize}

\subsubsection{UC-2G - Tabella template salvati }

\begin{figure}[H]
    \centering
    \includegraphics[width=0.7\textwidth]{Diagrammi casi d'uso/UC0B.jpg}
    \caption{Didascalia dell'immagine}
\end{figure}

\textbf{Attori}
\begin{itemize}
    \item Utente autorizzato (Operatore di studio CdL).
    \item Sistema di gestione template messaggi.
\end{itemize}

\textbf{Pre-condizioni}
\begin{itemize}
    \item L'utente si trova nella pagina di gestione messaggio tramite UC-2F.
    \item Esistono uno o più template di messaggi salvati nel sistema.
\end{itemize}

\textbf{Post-condizioni}
\begin{itemize}
    \item L'utente si ritrova nella pagina di gestione messaggio con il template selezionato caricato nei campi oggetto e testo del messaggio.
\end{itemize}


\textbf{Scenario principale}
\begin{enumerate}
    \item L'utente ha bisogno di utilizzare un template di messaggio salvato per preparare il messaggio da inviare assieme ai documenti selezionati.
\end{enumerate}

\textbf{Scenario secondario}
\begin{enumerate}
    \item 
\end{enumerate}

\textbf{Trigger}
\begin{itemize}
    \item Selezionare l'opzione di visualizzazione tabella template salvati dalla pagina gestione messaggio.
\end{itemize}


\textbf{Relazioni con altri casi d'uso (\textit{include} / \textit{extend})}
\begin{itemize}
    \item \textit{include}: 
    \begin{itemize}
        \item UC-1B.2 – Visualizzazione tono
        \item UC-1B.3 – Visualizzazione stile 
        \item UC-2F.6 – Visualizzazione oggetto messaggio
        \item UC-2F.7 – Visualizzazione testo messaggio
    \end{itemize}
    \item \textit{extend}: 
    \begin{itemize}
        \item UC-2G.1 – Elimina template
        \item UC-2G.2 – Nessun template salvato
        \item UC-2G.3 – Carica template
    \end{itemize}
\end{itemize}

\vspace{0.5cm}

\subsubsection{UC-2G.1 – Elimina template}

\textbf{Attori}
\begin{itemize}
    \item Utente autorizzato (Operatore di studio CdL).
\end{itemize}

\textbf{Pre-condizioni}
\begin{itemize}
    \item L'utente visualizza la lista dei template salvati (UC-2G).
    \item Il template da eliminare è presente in lista.
\end{itemize}

\textbf{Post-condizioni}
\begin{itemize}
    \item Il template viene rimosso definitivamente dal sistema.
    \item La lista viene aggiornata e non mostra più il template eliminato.
\end{itemize}

\textbf{Scenario principale}
\begin{enumerate}
    \item L'utente identifica un template obsoleto o errato nella tabella.
    \item L'utente clicca sull'icona di eliminazione (es. cestino) in corrispondenza della riga.
    \item Il sistema richiede conferma dell'operazione.
    \item L'utente conferma e il sistema cancella il record dal database.
\end{enumerate}

\textbf{Trigger}
\begin{itemize}
    \item Click sul pulsante "Elimina" associato a un template.
\end{itemize}

\vspace{0.5cm}

\subsubsection{UC-2G.2 – Nessun template salvato}

\textbf{Attori}
\begin{itemize}
    \item Sistema.
\end{itemize}

\textbf{Pre-condizioni}
\begin{itemize}
    \item L'utente ha richiesto l'apertura della tabella template (UC-2G).
    \item Non sono presenti template nel database per l'utente o per lo studio.
\end{itemize}

\textbf{Post-condizioni}
\begin{itemize}
    \item Viene visualizzato un messaggio informativo ("Nessun template trovato").
    \item La tabella appare vuota o nascosta.
\end{itemize}

\textbf{Scenario principale}
\begin{enumerate}
    \item Il sistema interroga il database per recuperare la lista dei template disponibili.
    \item La query restituisce un risultato vuoto.
    \item Il sistema notifica all'utente che non ci sono template da caricare.
\end{enumerate}

\textbf{Trigger}
\begin{itemize}
    \item Caricamento della pagina UC-2G (Automatico in assenza di dati).
\end{itemize}

\vspace{0.5cm}

\subsubsection{UC-2G.3 – Carica template}

\textbf{Attori}
\begin{itemize}
    \item Utente autorizzato (Operatore di studio CdL).
    \item Sistema.
\end{itemize}

\textbf{Pre-condizioni}
\begin{itemize}
    \item L'utente ha selezionato un template valido dalla lista (UC-2G).
\end{itemize}

\textbf{Post-condizioni}
\begin{itemize}
    \item I dati del template (Oggetto e Corpo del messaggio) vengono trasferiti nei campi di input della pagina di gestione messaggio (UC-2F).
    \item Il modale o la pagina di selezione template si chiude.
\end{itemize}

\textbf{Scenario principale}
\begin{enumerate}
    \item L'utente clicca sul pulsante "Carica" o sulla riga del template desiderato.
    \item Il sistema recupera il contenuto del template (testo e oggetto).
    \item Il sistema sovrascrive o compila i campi corrispondenti nell'interfaccia di composizione del messaggio.
    \item L'utente può ora procedere a modificare o inviare il messaggio (ritorno a UC-2F).
\end{enumerate}

\textbf{Trigger}
\begin{itemize}
    \item Selezione di un template dalla lista e conferma di caricamento.
\end{itemize}

\subsubsection{UC-2H – Invio documento e messaggio}

\begin{figure}[H]
    \centering
    \includegraphics[width=0.7\textwidth]{Diagrammi casi d'uso/UC0B.jpg}
    \caption{Didascalia dell'immagine}
\end{figure}

\textbf{Attori}
\begin{itemize}
    \item Utente autorizzato (Operatore di studio CdL).
    \item Sistema di invio documenti e messaggi.
\end{itemize}

\textbf{Pre-condizioni}
\begin{itemize}
    \item L'utente si trova nella pagina di gestione messaggio tramite UC-2F.
    \item Il messaggio é stato completato con oggetto, testo e documenti allegati.
\end{itemize}

\textbf{Post-condizioni}
\begin{itemize}
    \item Il messaggio e i documenti sono stati inviati con successo ai destinatari previsti.
\end{itemize}

\textbf{Scenario principale}
\begin{enumerate}
    \item L'utente ha completato la preparazione del messaggio e dei documenti da inviare.
    \item L'utente seleziona l’opzione per inviare il messaggio e i documenti ai destinatari.
\end{enumerate}

\textbf{Scenario secondario}
\begin{enumerate}
    \item 
\end{enumerate}

\textbf{Trigger}
\begin{itemize}
    \item Selezionare l'opzione di invio documento e messaggio dalla pagina gestione messaggio.
\end{itemize}

\textbf{Relazioni con altri casi d'uso (\textit{include} / \textit{extend})}
\begin{itemize}
    \item \textit{include}: 
    \begin{itemize}
        \item UC-2F.6 – Visualizzazione oggetto messaggio
        \item UC-2F.7 – Visualizzazione testo messaggio
        \item UC-2H.4 – Visualizzazione lista destinatari
        \item UC-2H.5 – Visualizzazione lista documenti
    \end{itemize}
    \item \textit{extend}: 
    \begin{itemize}
        \item UC-2H.1 – Allega file
        \item UC-2H.2 – Pianifica invio
        \item UC-2H.3 – Conferma invio
        \item UC-2F – Gestione messaggio %Torna a gestione messaggio
    \end{itemize}
\end{itemize}

\subsubsection{UC-2H.1 – Allega file}

\textbf{Attori}
\begin{itemize}
    \item Utente autorizzato (Operatore di studio CdL).
\end{itemize}

\textbf{Pre-condizioni}
\begin{itemize}
    \item L'utente si trova nella pagina di riepilogo invio (UC-2H).
    \item Si desidera aggiungere un documento aggiuntivo (es. informativa generica) non presente nello storico personale del destinatario.
\end{itemize}

\textbf{Post-condizioni}
\begin{itemize}
    \item Il nuovo file viene aggiunto alla lista degli allegati pronti per l'invio.
\end{itemize}

\textbf{Scenario principale}
\begin{enumerate}
    \item L'utente clicca sul pulsante "Aggiungi allegato".
    \item Il sistema apre la finestra di selezione file del sistema operativo.
    \item L'utente seleziona il file desiderato.
    \item Il sistema carica il file e lo mostra nella lista allegati (UC-2H.5).
\end{enumerate}

\textbf{Trigger}
\begin{itemize}
    \item Click sul pulsante/icona "Allega file".
\end{itemize}

\vspace{0.5cm}

\subsubsection{UC-2H.2 – Pianifica invio}

\textbf{Attori}
\begin{itemize}
    \item Utente autorizzato (Operatore di studio CdL).
    \item Sistema (Scheduler).
\end{itemize}

\textbf{Pre-condizioni}
\begin{itemize}
    \item Il messaggio e i documenti sono pronti.
    \item L'utente non desidera l'invio immediato.
\end{itemize}

\textbf{Post-condizioni}
\begin{itemize}
    \item L'invio viene schedulato per la data e ora selezionate.
    \item Lo stato dei documenti passa a "Pianificato".
\end{itemize}

\textbf{Scenario principale}
\begin{enumerate}
    \item L'utente seleziona l'opzione "Pianifica invio".
    \item Il sistema mostra un selettore di data e ora.
    \item L'utente imposta il momento desiderato per l'invio e conferma.
    \item Il sistema prende in carico la richiesta e la accoda per l'elaborazione futura.
\end{enumerate}

\textbf{Trigger}
\begin{itemize}
    \item Selezione dell'opzione "Pianifica" o click sull'icona orologio.
\end{itemize}

\vspace{0.5cm}

\subsubsection{UC-2H.3 – Conferma invio}

\textbf{Attori}
\begin{itemize}
    \item Utente autorizzato (Operatore di studio CdL).
\end{itemize}

\textbf{Pre-condizioni}
\begin{itemize}
    \item Tutti i dati (destinatari, allegati, messaggio) sono stati verificati.
\end{itemize}

\textbf{Post-condizioni}
\begin{itemize}
    \item Il sistema avvia il processo di spedizione delle email/messaggi.
    \item L'utente riceve un feedback di successo.
    \item L'utente viene reindirizzato alla pagina iniziale o allo storico.
\end{itemize}

\textbf{Scenario principale}
\begin{enumerate}
    \item L'utente clicca sul pulsante finale "Invia ora".
    \item Il sistema chiede una conferma definitiva (opzionale).
    \item Il sistema processa l'invio e notifica l'esito positivo ("Messaggio inviato con successo").
\end{enumerate}

\textbf{Scenario secondario}
\begin{enumerate}
    \item Il sistema rileva un errore durante l'invio (es. errore SMTP).
    \item Viene mostrato un messaggio di errore e l'invio viene annullato permettendo di riprovare.
\end{enumerate}

\textbf{Trigger}
\begin{itemize}
    \item Click sul pulsante "Invia" o conferma del modale di riepilogo.
\end{itemize}

\vspace{0.5cm}

\subsubsection{UC-2H.4 – Visualizzazione lista destinatari}

\textbf{Attori}
\begin{itemize}
    \item Utente autorizzato (Operatore di studio CdL).
\end{itemize}

\textbf{Pre-condizioni}
\begin{itemize}
    \item Si è nella fase di riepilogo invio (UC-2H).
\end{itemize}

\textbf{Post-condizioni}
\begin{itemize}
    \item L'elenco completo dei destinatari (principali, copia conoscenza, copia nascosta) è visibile.
\end{itemize}

\textbf{Scenario principale}
\begin{enumerate}
    \item Il sistema recupera i dati di contatto del destinatario selezionato in UC-2D.
    \item Il sistema visualizza gli indirizzi email o i recapiti che riceveranno la comunicazione.
\end{enumerate}

\textbf{Trigger}
\begin{itemize}
    \item Caricamento della pagina UC-2H (Automatico).
\end{itemize}

\vspace{0.5cm}

\subsubsection{UC-2H.5 – Visualizzazione lista documenti}

\textbf{Attori}
\begin{itemize}
    \item Utente autorizzato (Operatore di studio CdL).
\end{itemize}

\textbf{Pre-condizioni}
\begin{itemize}
    \item Si è nella fase di riepilogo invio (UC-2H).
    \item Sono stati selezionati documenti dallo storico (UC-2E) o caricati manualmente (UC-2H.1).
\end{itemize}

\textbf{Post-condizioni}
\begin{itemize}
    \item L'utente vede l'elenco dei file che verranno allegati al messaggio.
\end{itemize}

\textbf{Scenario principale}
\begin{enumerate}
    \item Il sistema aggrega i documenti selezionati.
    \item Il sistema mostra una lista contenente i nomi dei file, la dimensione e l'estensione (es. PDF).
    \item L'utente può verificare visivamente la correttezza degli allegati prima dell'invio.
\end{enumerate}

\textbf{Trigger}
\begin{itemize}
    \item Caricamento della pagina UC-2H (Automatico).
\end{itemize}

\section{Sezione 3 - Modulo Analytics e Monitoraggio Trasversale}
\subsection{UC-3A - Visualizzazione dashboard di analisi e Monitoraggio AI}

\begin{figure}[H]
    \centering
   
    \caption{Diagramma del caso d'uso UC-3A}
    \label{fig:uc3a}
\end{figure}

\textbf{Attori}
\begin{itemize}
    \item Auditor/Data Analyst (principale).
    \item Amministratore.
    \item Sistema di Analytics.
\end{itemize}

\textbf{Pre-condizioni}
\begin{itemize}
    \item L'utente ha effettuato il login ed è autenticato come Auditor, Data Analyst o Amministratore.
    \item Il sistema ha accumulato dati storici.
    \item L'utente accede alla sezione ``Analytics e Monitoraggio'' dalla dashboard principale.
\end{itemize}

\textbf{Post-condizioni}
\begin{itemize}
    \item L'utente visualizza le metriche aggiornate in base ai filtri applicati (o di default).
\end{itemize}

\textbf{Scenario principale}
\begin{enumerate}
    \item L'utente accede al modulo centralizzato di Analytics.
    \item Il Sistema calcola e mostra una panoramica generale (Dashboard) applicando i filtri di default (es. Periodo: Ultimo Mese, Utenti: Tutti).
    \item Il Sistema visualizza i grafici e le tabelle relative alle metriche (confidenza, rating, volumi) per il contesto di default.
\end{enumerate}

\textbf{Scenario secondario}
\begin{enumerate}
    \item Il sistema non dispone di dati sufficienti nel periodo di default.
    \item Il Sistema mostra un messaggio ``Nessun dato disponibile per il periodo selezionato''.
\end{enumerate}

\textbf{Trigger}
\begin{itemize}
    \item Selezionare la voce ``Analytics / Monitoraggio AI'' dal menu principale.
\end{itemize}

\textbf{Relazioni con altri casi d'uso (\textit{include} / \textit{extend})}
\begin{itemize}
    \item \textit{include}: 
    \begin{itemize}
        \item UC-3A.1 - Visualizzazione volume prompt generati.
        \item UC-3A.2 - Visualizzazione statistiche toni e stili utilizzati.
        \item UC-3A.3 - Visualizzazione rating medio e feedback.
        \item UC-3A.4 - Visualizzazione confidenza media riconoscimento.
        \item UC-3A.5 - Visualizzazione percentuale interventi manuali.
    \end{itemize}
    \item \textit{extend}: 
    \begin{itemize}
        \item UC-3A.6 - Filtraggio per periodo temporale (L'utente modifica le date).
        \item UC-3A.7 - Filtraggio per utente/operatore (L'utente seleziona uno specifico operatore).
        \item UC-3A.8 - Esportazione Report (PDF/CSV).
        \item UC-3A.9 - Assenza di dati storici.
    \end{itemize}
\end{itemize} 

\vspace{0.5cm}

\section{Requisiti}

Questa sezione classifica i requisiti del sistema in quattro categorie principali: funzionali, di qualità, di vincolo e prestazionali.

\subsection{Requisiti Funzionali}
Descrivono i servizi specifici e le funzioni che il sistema deve fornire. Rappresentano il "cosa" il sistema deve fare in risposta a determinati input o situazioni.
\vspace{0.5cm}

\textbf{Caratteristiche}
\begin{enumerate}
    \item Descrivono le interazioni tra l'utente (o altri sistemi) e il software.
    \item Sono diretti: se il requisito non è soddisfatto, il sistema non funziona come previsto.
    \item Derivano direttamente dai casi d'uso e dalle user stories.
\end{enumerate}

\vspace{0.5cm}
\begin{table}[h!]
    \centering
    \renewcommand{\arraystretch}{1.5} % Aumenta la spaziatura delle righe
    \begin{tabular}{|l|p{8cm}|l|l|}
        \hline
        \textbf{Codice} & \textbf{Descrizione} & \textbf{Fonti} & \textbf{Priorità} \\
        \hline
        UC-0A & Descrizione 1 & link & Facoltativo \\
        \hline
        UC-0A.1 & Descrizione 2 & link & Facoltativo \\
        \hline
        UC-0A.2 & Descrizione 3 & link & Facoltativo \\
        \hline
    \end{tabular}
    \caption{Tabella dei Requisiti Funzionali}
    \label{tab:req_funzionali}
\end{table}

\subsection{Requisiti di Qualità}
Definiscono gli attributi qualitativi del software che influenzano l'esperienza d'uso e la manutenibilità del progetto. Spesso indicati come "attributi di qualità" (es. usabilità, affidabilità, manutenibilità).
\vspace{0.5cm}

\textbf{Caratteristiche}
\begin{enumerate}
    \item Determinano il livello di soddisfazione dell'utente e la facilità di evoluzione del software.
    \item Devono essere misurabili tramite metriche specifiche o feedback utente.
\end{enumerate}

\vspace{0.5cm}
\begin{table}[h!]
    \centering
    \renewcommand{\arraystretch}{1.5}
    \begin{tabular}{|l|p{8cm}|l|l|}
        \hline
        \textbf{Codice} & \textbf{Descrizione} & \textbf{Fonti} & \textbf{Priorità} \\
        \hline
        boh & Descrizione 1 & link & Desiderabile \\
        \hline
        boh & Descrizione 2 & link & Obbligatorio \\
        \hline
        boh & Descrizione 3 & link & Obbligatorio \\
        \hline
    \end{tabular}
    \caption{Tabella dei Requisiti di Qualità}
    \label{tab:req_qualita}
\end{table}

\subsection{Requisiti di Vincolo}
Rappresentano le limitazioni e le restrizioni entro cui il sistema deve essere sviluppato o operare. Questi vincoli restringono lo spazio delle soluzioni possibili.
\vspace{0.5cm}

\textbf{Caratteristiche}
\begin{enumerate}
    \item Possono essere di natura tecnologica (hardware, linguaggio), normativa (GDPR) o di business (budget).
    \item Sono mandatori e non negoziabili.
\end{enumerate}

\vspace{0.5cm}
\begin{table}[h!]
    \centering
    \renewcommand{\arraystretch}{1.5}
    \begin{tabular}{|l|p{8cm}|l|l|}
        \hline
        \textbf{Codice} & \textbf{Descrizione} & \textbf{Fonti} & \textbf{Priorità} \\
        \hline
        boh & Descrizione 1 & link & Desiderabile \\
        \hline
        boh & Descrizione 2 & link & Obbligatorio \\
        \hline
        boh & Descrizione 3 & link & Obbligatorio \\
        \hline
    \end{tabular}
    \caption{Tabella dei Requisiti di Vincolo}
    \label{tab:req_vincolo}
\end{table}

\subsection{Requisiti Prestazionali}
Specificano i parametri numerici relativi all'efficienza del sistema. Sebbene siano tecnicamente un sottoinsieme della qualità, vengono trattati separatamente per la loro criticità e misurabilità quantitativa.
\vspace{0.5cm}

\textbf{Caratteristiche}
\begin{enumerate}
    \item Definiscono limiti su tempi di risposta, throughput e utilizzo delle risorse.
    \item Sono sempre espressi con valori numerici e soglie precise.
\end{enumerate}

\vspace{0.5cm}
\begin{table}[h!]
    \centering
    \renewcommand{\arraystretch}{1.5}
    \begin{tabular}{|l|p{8cm}|l|l|}
        \hline
        \textbf{Codice} & \textbf{Descrizione} & \textbf{Fonti} & \textbf{Priorità} \\
        \hline
        boh & Descrizione 1 & link & Desiderabile \\
        \hline
        boh & Descrizione 2 & link & Obbligatorio \\
        \hline
        boh & Descrizione 3 & link & Obbligatorio \\
        \hline
    \end{tabular}
    \caption{Tabella dei Requisiti Prestazionali}
    \label{tab:req_prestazionali}
\end{table}

\end{document}