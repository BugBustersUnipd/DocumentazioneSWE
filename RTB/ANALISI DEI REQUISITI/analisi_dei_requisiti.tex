\documentclass[a4paper,11pt]{article}

\usepackage[utf8]{inputenc}
\usepackage[T1]{fontenc}
\usepackage[italian]{babel}
\usepackage[margin=2.5cm]{geometry}
\usepackage{graphicx}
\usepackage{booktabs}
\usepackage{setspace}
\usepackage{titlesec}
\usepackage{float}
\usepackage[table]{xcolor}
\usepackage{tabularx}
\usepackage{tcolorbox}
\usepackage{enumitem}
\usepackage[titles]{tocloft}
\usepackage[colorlinks=true,linkcolor=black,urlcolor=blue,citecolor=blue]{hyperref}
\usepackage{fancyhdr}
\usepackage{lastpage}
\usepackage{amsmath}

\pagestyle{fancy}
\fancyhf{}
\fancyhead[L]{BugBusters}
\fancyhead[R]{Analisi dei Requisiti}
\fancyfoot[L]{\thepage\ di \pageref{LastPage}}
\renewcommand{\headrulewidth}{0pt}
\renewcommand{\footrulewidth}{0pt}

\setlength{\headheight}{14pt}

\setlength{\parskip}{4pt}
\setlength{\parindent}{0pt}

\titleformat{\section}{\large\bfseries}{\thesection}{1em}{}
\titleformat{\subsection}{\normalsize\bfseries}{\thesubsection}{1em}{}

\begin{document}

\begin{center}
  \thispagestyle{empty}
  \IfFileExists{../../assets/Logo.jpg}{%
    \includegraphics[width=6cm,height=3cm,keepaspectratio]{../../assets/Logo.jpg} \\[0.8cm]
  }{%
    \fbox{\parbox[c][2.5cm][c]{6cm}{\centering Logo non trovato\\(Logo.jpg)}}\\[0.5cm]
  }
  {\LARGE\bfseries BugBusters}\\[0.8cm]
  
  \rule{\textwidth}{0.5pt}\\[0.5cm]
  {\Large\bfseries Analisi dei Requisiti}\\[0.3cm]
  {\large Versione 0.0.1}\\[0.5cm]
  \rule{\textwidth}{0.5pt}\\[0.8cm]
\end{center}

\begin{center}
\begin{tcolorbox}[colback=gray!10,width=0.8\textwidth,arc=3mm,boxrule=0.5pt]
\begin{tabular}{ll}
\textbf{Stato} & In redazione \\
\textbf{Redattori} & ----- \\
\textbf{Destinatari} & BugBusters \\
 & Prof. Vardanega Tullio \\
 & Prof. Cardin Riccardo \\
 & Eggon \\
\end{tabular}
\end{tcolorbox}
\end{center}

\vspace{1cm}

\begin{center}
\textbf{Descrizione}
\end{center}

\begin{center}
\begin{minipage}{0.9\textwidth}
\small
Questo documento contiene le Norme di Progetto seguite dal team \textbf{BugBusters} per il progetto\textsubscript{\scalebox{0.6}{\textbf{G}}} C5 proposto dall'azienda Eggon
\end{minipage}
\end{center}

\newpage

\section*{Registro delle Modifiche}

{\footnotesize
\begin{center}
\begin{tabularx}{\textwidth}{|l|l|X|l|l|}
\hline
\textbf{Versione} & \textbf{Data} & \textbf{Descrizione} & \textbf{Autore} & \textbf{Ruolo} \\
\hline
0.0.1 & ------- & 
\begin{minipage}[t]{\linewidth}
Prima stesura della struttura del documento.
\end{minipage} 
& ------ & - \\
\hline
\end{tabularx}
\end{center}
}

\vfill
\begin{center}
2 di \pageref{LastPage}
\end{center}

\newpage

\section*{Indice}

\noindent
\begin{minipage}[t]{0.8\textwidth}
\subsection*{1 Introduzione}
1.1 Scopo del documento \\
1.2 Prospettiva del prodotto \\
1.3 Funzioni del prodotto \\
1.4 Caratterisitiche dell'utente \\
1.5 Definizioni, acronimi e abbreviazioni \\
1.6 Riferimenti \\
\quad 1.6.1 Riferimenti normativi \\
\quad 1.6.2 Riferimenti informativi \\
\subsection*{2 Casi d'uso}
2.1 Introduzione \\
2.2 Attori \\
2.3 Lista casi d'uso \\
\quad 2.3.1 UC1: Creazione di un nuovo contenuto tramite prompt \\

\end{minipage}
\begin{minipage}[t]{0.2\textwidth}
\vspace{1.65\baselineskip}
9 \\
9 \\
9 \\
10 \\
10 \\
10 \\
\end{minipage}

\newpage

\section{Introduzione}

\subsection{Scopo del documento}
Questo documento di Analisi dei Requisiti\textsubscript{\scalebox{0.6}{\textbf{G}}}
 adottato da parte di BugBusters durante lo svolgimento del progetto\textsubscript{\scalebox{0.6}{\textbf{G}}}
 didattico, ha lo scopo di definire in maniera precisa e
 dettagliata i requisiti funzionali\textsubscript{\scalebox{0.6}{\textbf{G}}} e non funzionali del Sistema software da sviluppare.

 Per tale scopo, il documento include una descrizione approfondita dei Casi d’Uso, che sono
fonte dei requisiti infine elencati. Verranno utilizzati, per agevolare la comprensione, i
\textbf{Diagrammi dei Casi d’Uso } per visualizzare le interazioni tra attori e Sistema.

Il presente documento rappresenterà il riferimento di base per la progettazione,
l’implementazione e il collaudo del prodotto finale, garantendo che il Sistema realizzato
soddisfi pienamente le esigenze e le aspettative del committente.

I requisiti identificati sono classificati in:
\begin{itemize}
    \item Obbligatori: irrinunciabili e imprescindibili per il corretto funzionamento del Sistema;
    \item Desiderabili: non strettamente necessari ma in grado di apportare un valore aggiunto riconoscibile;
    \item Opzionali: relativamente utili oppure contrattabili per essere implementati in futuro.
\end{itemize}

Il documento di Analisi dei Requisiti\textsubscript{\scalebox{0.6}{\textbf{G}}} viene redatto dagli Analisti del team di progetto ed è
rivolto ai seguenti principali destinatari:
\begin{itemize}
    \item Il Committente, che potrà verificare che i requisiti siano stati correttamente interpretati
    e documentati secondo le sue specifiche;
    \item Il Team di Progettisti e di Programmatori, che utilizzerà questo documento come guida
    fondamentale per la realizzazione del Sistema software;
    \item Il Team di Verificatori, che baserà sull’Analisi dei Requisiti\textsubscript{\scalebox{0.6}{\textbf{G}}} la definizione dei casi di Test
    e la verifica del corretto funzionamento del prodotto.
\end{itemize}
Inoltre, il documento potrà essere consultato da altri soggetti coinvolti nel processo di
sviluppo, come Amministratori e Responsabili di Progetto, al fine di acquisire una
comprensione esauriente delle specifiche del Sistema.


\subsection{Prospettiva del prodotto}

Il prodotto che BugBusters si promette di sviluppare è una significativa evoluzione della piattaforma NEXUM: 
un ecosistema HR intelligente, interoperabile e scalabile per la gestione della comunicazione interna, delle presenze e dei processi amministrativi legati alle risorse umane.
Nello specifico, si intende realizzare una piattaforma che non solo supporti le funzionalità già presenti 
(messaggistica top-down, timbratura digitale base, gestione anagrafiche/ruoli/permessi), 
ma che fornisca una visione unificata e in tempo reale delle attività HR, delle scadenze e dei documenti rilevanti per le organizzazioni e per gli studi dei Consulenti del Lavoro (CdL) connessi al sistema.

Per ottenere ciò, il prodotto dovrà essere reattivo alle molteplici operazioni che avvengono quotidianamente nell’ambito HR 
(timbrature, invio di documenti, elaborazione di cedolini, gestione ferie/permessi, segnalazioni di anomalie). 
L’architettura dovrà quindi supportare scalabilità orizzontale e verticale dei componenti (microservizi) per mantenere bassa latenza e 
alta disponibilità anche in condizioni di carico elevato. 
Fondamentale è inoltre garantire la sicurezza e l’isolamento dei flussi dati tra organizzazioni e studi esterni.

Elemento distintivo del prodotto sarà l’integrazione di moduli AI e di Data Analytics: 
un AI Assistant generativo per la creazione e l’adattamento di contenuti aziendali (titoli, testi e immagini coerenti con il tono aziendale), 
moduli di anomaly detection per individuare comportamenti anomali nelle presenze e negli straordinari, 
e capacità di riconoscimento e smistamento automatico dei documenti provenienti dagli studi dei Consulenti del Lavoro. 
Queste funzionalità abiliteranno automazioni proattive (es. notifica scadenze, suggerimenti per la gestione ferie, dispaccio automatico di documenti) e analizzeranno i dati per supportare decisioni HR data-driven.

Infine, il progetto assumerà un approccio modulare e sperimentale: 
i componenti sviluppati dovranno essere integrabili nella piattaforma NEXUM 
(backend, frontend, database, architettura) come moduli riutilizzabili e documentati, favorendo sia l’adozione industriale sia l’attività di ricerca e 
formazione tramite prototipi realizzati dagli studenti UNIPD.




\subsection{Funzioni del prodotto}
Dal punto di vista dell’utilizzatore finale, il prodotto dovrà fornire le seguenti funzionalità:
\begin{itemize}
    \item Disponibilità delle nuove funzionalità sulla Dashboard amministrativa e, ove opportuno, tramite PWA per gli utenti finali.
    \item AI Assistant Generativo per la creazione rapida di comunicazioni interne: generazione di titolo, contenuto testuale e immagine di copertina a partire da un prompt; selezione del tono/stile (formale, informale, neutro); salvataggio dei prompt e dei risultati; sistema di rating per valutare la qualità dei contenuti generati.
    \item Estensione del modulo di timbratura per la raccolta e gestione di ferie, permessi, malattia e straordinari, con storicizzazione e reportistica.
    \item Modulo di anomaly detection per individuare automaticamente incongruenze nelle presenze e negli straordinari e invio di alert ai responsabili.
    \item Repository documentale per gli studi dei Consulenti del Lavoro: upload sicuro dei documenti, riconoscimento automatico della tipologia di documento (cedolini, comunicazioni, documenti da firmare, ecc.) e individuazione dei destinatari direttamente dal contenuto.
    \item Split e dispaccio automatico dei documenti: capacità di suddividere upload massivi in base ai destinatari e creare liste di distribuzione per invii massivi.
    \item Sistema di interoperabilità e API per lo scambio sicuro di documenti, scadenze e metadati con gli studi dei CdL, potenziato da moduli AI per il riconoscimento, l’estrazione metadati e il dispaccio intelligente.
    \item Funzionalità di configurazione amministrativa: possibilità di definire ruoli, permessi, template di messaggi, soglie e policy per le rilevazioni di anomalie.
    \item Strumenti di esportazione: esportare report di presenza, inventari di scadenze e archive documentali in formati standard (CSV, PDF).
    \item Monitoraggio in tempo reale: visualizzazione del numero di richieste e dello stato operativo dei servizi della piattaforma (metriche di utilizzo, code, latenza).
    \item Tracciamento e audit: log delle azioni amministrative e dei dispacci documentali per esigenze di compliance.
\end{itemize}

Queste funzionalità mirano a rendere NEXUM una piattaforma modulare, scalabile e orientata all’automazione intelligente, capace di migliorare l’efficienza dei processi HR e la collaborazione con gli studi dei Consulenti del Lavoro.


\subsection{Caratterisitiche dell'utente}
Gli utilizzatori finali del prodotto non appartengono a un’unica categoria specifica: l’obiettivo del progetto è infatti quello di progettare moduli intelligenti e interoperabili che possano essere integrati all’interno dell’ecosistema NEXUM, rendendolo utilizzabile in modo efficace da una vasta gamma di figure professionali.
In generale, è possibile affermare che gli utenti finali sono tutti coloro che necessitano di uno strumento scalabile, intelligente e semplice da utilizzare per la gestione delle attività HR, della comunicazione interna e dello scambio documentale con gli studi dei Consulenti del Lavoro. Rientrano in questa categoria:
\begin{itemize}
    \item Responsabili e amministratori HR, che richiedono strumenti avanzati per la gestione di presenze, ferie, permessi, anomalie e comunicazioni aziendali.
    \item Consulenti del Lavoro (CdL) e personale amministrativo degli studi professionali, che necessitano di un sistema interoperabile per caricare, riconoscere e distribuire documenti in modo rapido e automatico.
    \item Dipendenti e collaboratori, che interagiscono con la piattaforma per consultare comunicazioni, gestire le proprie richieste e ricevere documenti in formato digitale.
    \item Manager aziendali, interessati a disporre di una panoramica affidabile e in tempo reale dei flussi HR e delle scadenze operative.
\end{itemize}
In sintesi, il prodotto è rivolto a organizzazioni di varie dimensioni — in particolare aziende medio-grandi e studi professionali — che necessitano di una piattaforma HR completa, modulare, automatizzata e potenziata dall’AI, capace di semplificare e migliorare la gestione dei processi legati alle risorse umane e alla collaborazione con i Consulenti del Lavoro.

\subsection{Definizioni, acronimi e abbreviazioni}
Per tutte le definizioni, acronimi e abbreviazioni utilizzati in questo documento, si faccia
riferimento al \textbf{Glossario}, fornito come documento separato, che contiene tutte le spiegazioni
necessarie per garantire una comprensione uniforme dei termini tecnici e dei concetti
rilevanti per il progetto.

\newpage

\subsection{Riferimenti}

\subsubsection{Riferimenti normativi}
\begin{itemize}
\item \textbf{Capitolato\textsubscript{\scalebox{0.6}{\textbf{G}}}
 d'appalto C5: Nexum - Piattaforma di consulenza e documentazione previdenziale}\\
\url{https://www.math.unipd.it/~tullio/IS-1/2025/Progetto/C5.pdf}
\end{itemize}

\subsubsection{Riferimenti informativi}
\begin{itemize}
\item \textbf{Glossario\textsubscript{\scalebox{0.6}{\textbf{G}}}
:}\\
\url{https://github.com/BugBustersUnipd/DocumentazioneSWE/blob/main//RTB/GLOSSARIO/Glossario.pdf}
\end{itemize}



\section{Casi d'uso}
\subsection{Introduzione}
I casi d’uso si compongono di un grafico UML e una descrizione testuale che permetta di
comprendere al meglio cosa il prodotto deve fornire. La descrizione testuale, in particolar
modo, dovrà contenere le informazioni sotto presenti, salvo i casi in cui lo
specifico campo non risulti rilevante (ad esempio, un Caso d’Uso\textsubscript{\scalebox{0.6}{\textbf{G}}} che non prevede la
possibilità di errori non avrà Scenari secondari):

\begin{itemize}
    \item Attori: Sono coloro che interagiscono attivamente con il Sistema e
    svolgono l’azione indicata dal Caso d’Uso
    \item Precondizioni: Lista di elementi che sono necessari affinché l’Attore possa
    compiere l’azione indicata dal caso d’uso
    \item Postcondizioni: Lista di elementi che descrivono come il Sistema risulta
    essere internamente cambiato dopo che l’Attore ha effettuato
    l’azione prevista dal Caso d’Uso
    \item Scenario principale: Descrizione ragionevole delle operazioni che l’attore deve
    fare per compiere l’azione descritta dal Caso d’Uso
    \item Scenario secondario: Descrizione ragionevole degli eventi che possono accadere
    qualora una delle operazioni descritte nello Scenario
    principale non vada a buon fine
    \item Inclusioni: Casi d’Uso ulteriori che l’Attore deve compiere per realizzare
    il Caso d’Uso attualmente descritto
    \item Estensioni: Casi d’Uso ulteriori che possono realizzarsi durante
    l’esecuzione delle operazioni del Caso d’Uso principale
    \item Motivazioni che portano l’Attore a svolgere l’azione descritta
    dal Caso d’Uso. Non sempre disponibile in quanto il Caso
    d’Uso potrebbe essere incluso da un altro caso d’uso
    «principale»
\end{itemize}

\subsection{Attori}
Di seguito sono esposti gli attori utilizzati:
\begin{itemize}
    \item DA SPIEGARE OGNUNO DI QUESTI ATTORI
    \item DA AGGIUNGERE DIAGRAMMA DEGLI ATTORI
    \item HR Manager
    \item Redattore
    \item Data Analyst (Eggon)
    \item Amministratore
    \item Operatore Studio CdL (Admin/Editor), Sistema NEXUM
    \item Sistema NEXUM (AI Doc Classifier), Operatore CdL
    \item Sistema NEXUM (Entity Resolver), Operatore CdL
    \item Sistema NEXUM (Splitter), Operatore CdL
    \item Operatore CdL, Sistema
    \item Sistema NEXUM (Dispatcher), Operatore CdL
    \item Sistema NEXUM (Dispatcher \& Tracking), Operatore CdL, Destinatario finale
    \item Operatore CdL, Auditor interno, Sistema
    \item Admin Cliente, Admin Eggon, Sistema
    \item Data Analyst, Admin Eggon, Sistema
\end{itemize}


\subsection{Lista casi d'uso}

\subsubsection{UC1: Creazione di un nuovo contenuto tramite prompt}

\begin{figure}[h]
    \centering
    \includegraphics[width=\textwidth]{Diagrammi casi d'uso/diagramma_caso_1.jpeg}
    \caption{Figura n: UC1: Creazione di un nuovo contenuto tramite prompt}
    \label{fig:your_image_label}
\end{figure}

\textbf{Attore}: Amministratore / HR Manager \\
\textbf{Pre-condizioni}: L’utente è autenticato nella dashboard NEXUM e ha i permessi di redazione.
\textbf{Post-condizioni}: Il contenuto è salvato nel database come bozza o pubblicato. \\
\textbf{Scenario principale}:
\begin{itemize}
    \item L’utente apre il modulo “AI Assistant Generativo”.
    \item Inserisce un prompt (es. “Comunicato di benvenuto ai nuovi dipendenti”).
    \item Seleziona il tono (formale/informale/neutro) e lo stile (istituzionale, empatico, tecnico).
    \item Il sistema invia la richiesta al motore AI.
    \item L’AI genera titolo, testo e immagine di copertina coerenti con il tono e il prompt.
    \item Il risultato viene visualizzato in anteprima.
    \item L’utente può accettare, rigenerare o modificare manualmente.
\end{itemize}

\end{document}