\documentclass[a4paper,11pt]{article}
\newcommand{\CurrentVersion}{0.2.0} % ultima versione, da cambiare ad ogni push significativo


\usepackage[utf8]{inputenc}
\usepackage[T1]{fontenc}
\usepackage[italian]{babel}
\usepackage[margin=2.5cm]{geometry}
\usepackage{graphicx}
\usepackage{booktabs}
\usepackage{setspace}
\usepackage{titlesec}
\usepackage{float}
\usepackage[table]{xcolor}
\usepackage{tabularx}
\usepackage{tcolorbox}
\usepackage{enumitem}
\usepackage[titles]{tocloft}
\usepackage[colorlinks=true,linkcolor=black,urlcolor=blue,citecolor=blue]{hyperref}
\usepackage{fancyhdr}
\usepackage{lastpage}
\usepackage{amsmath}
\usepackage{longtable}
\usepackage{tabularx}
\usepackage{array}
\usepackage{ragged2e}



\definecolor{primaryblue}{RGB}{0,102,204}
\definecolor{secondaryblue}{RGB}{51,153,255}
\definecolor{lightgray}{RGB}{245,245,245}
\definecolor{darkgray}{RGB}{100,100,100}

% TODO è il caso di inserire queste regole in un file e includerlo in tutti i documenti
\titleformat{\section}
 {\Large\bfseries\color{primaryblue}}
 {\thesection}{1em}{}

\titleformat{\subsection}
 {\large\bfseries\color{primaryblue}} % Sottosezione: colore secondaryblue
 {\thesubsection}{1em}{}

\titleformat{\subsubsection}
 {\normalsize\bfseries\color{secondaryblue}} % Sotto-sottosezione: colore secondaryblue
 {\thesubsubsection}{1em}{}

\setlength{\LTpre}{0pt}
\setlength{\LTpost}{0pt}
\raggedbottom

\newcolumntype{L}[1]{>{\RaggedRight\arraybackslash}p{#1}}


\pagestyle{fancy}
\fancyhf{}
\fancyhead[L]{BugBusters}
\fancyhead[R]{Analisi dei Requisiti}
\fancyfoot[L]{\thepage\ di \pageref{LastPage}}
\renewcommand{\headrulewidth}{0pt}
\renewcommand{\footrulewidth}{0pt}

\setlength{\headheight}{14pt}

\setlength{\parskip}{4pt}
\setlength{\parindent}{0pt}


\begin{document}

\begin{center}
  \thispagestyle{empty}
  \IfFileExists{../../assets/Logo.jpg}{%
    \includegraphics[width=6cm,height=3cm,keepaspectratio]{../../assets/Logo.jpg} \\[0.8cm]
  }{%
    \fbox{\parbox[c][2.5cm][c]{6cm}{\centering Logo non trovato\\(Logo.jpg)}}\\[0.5cm]
  }
  {\LARGE\bfseries BugBusters}\\[0.8cm]
  
  \rule{\textwidth}{0.5pt}\\[0.5cm]
  {\Large\bfseries Analisi dei Requisiti}\\[0.3cm]
  {\large Versione \CurrentVersion}\\[0.5cm]
  \rule{\textwidth}{0.5pt}\\[0.8cm]
\end{center}

\begin{center}
\begin{tcolorbox}[colback=gray!10,width=0.8\textwidth,arc=3mm,boxrule=0.5pt]
\begin{tabular}{ll}
\textbf{Stato} & In redazione \\
\textbf{Redattori} & ----- \\
\textbf{Destinatari} & BugBusters \\
 & Prof. Vardanega Tullio \\
 & Prof. Cardin Riccardo \\
 & Eggon \\
\end{tabular}
\end{tcolorbox}
\end{center}

\vspace{1cm}

\begin{center}
\textbf{Descrizione}
\end{center}

\begin{center}
\begin{minipage}{0.9\textwidth}
\small
Questo documento contiene le Norme di Progetto seguite dal team \textbf{BugBusters} per il progetto\textsubscript{\scalebox{0.6}{\textbf{G}}} C5 proposto dall'azienda Eggon
\end{minipage}
\end{center}



\newpage

\section*{Registro delle Modifiche}

\setlength{\extrarowheight}{2pt}
\renewcommand{\arraystretch}{1.5}
\arrayrulecolor{primaryblue}

{\footnotesize
\begin{longtable}{|>{\raggedright\arraybackslash}p{1.5cm}|>{\raggedright\arraybackslash}p{2cm}|>{\raggedright\arraybackslash}p{3cm}|>{\raggedright\arraybackslash}p{2cm}|>{\raggedright\arraybackslash}p{2cm}|>{\raggedright\arraybackslash}p{2cm}|}
\hline
\rowcolor{primaryblue!40}
\textbf{\color{white} Versione} & \textbf{\color{white} Data} & \textbf{\color{white} Descrizione} & \textbf{\color{white} Redatto} & \textbf{\color{white} Verificato} & \textbf{\color{white} Approvato} \\
\hline
\endfirsthead

\hline
\rowcolor{primaryblue!40}
\textbf{\color{white} Versione} & \textbf{\color{white} Data} & \textbf{\color{white} Descrizione} & \textbf{\color{white} Redatto} & \textbf{\color{white} Verificato} & \textbf{\color{white} Approvato} \\
\hline
\endhead

\rowcolor{secondaryblue!10} 0.2.0 & 04/01/2026 & Verifica & - & Luca Slongo & - \\
\hline
\rowcolor{secondaryblue!10} 0.1.11 & 04/01/2026 & Inserimento diagrammi casi d'uso sezioni 2 e 3 & Leonardo Salviato & - & - \\
\hline
\rowcolor{secondaryblue!10} 0.1.10 & 02/01/2026 & Inserimento diagrammi casi d'uso sezioni 0 e 1 & Leonardo Salviato & - & - \\
\hline
\rowcolor{secondaryblue!10} 0.1.9 & 29/12/2025 & Correzioni a casi d'uso e a paragrafi nel documento & Leonardo Salviato & - & - \\
\hline
\rowcolor{secondaryblue!10} 0.1.8 & 27/12/2025 & Aggiunti requisiti prestazionali, di qualità e di vincolo & Marco Favero & - & - \\
\hline
\rowcolor{secondaryblue!10} 0.1.7 & 23/12/2025 & Correzioni e aggiunte minime nel contenuto e nel registro delle modifiche & Linor Sadè & - & - \\
\hline
\rowcolor{secondaryblue!10} 0.1.6 & 21/12/2025 & Correzione attori, precondizioni, postcondizioni & Leonardo Salviato & - & - \\
\hline
\rowcolor{secondaryblue!10} 0.1.5 & 19/12/2025 & Scritta sezione 3, rimozione trigger, finita tabella requisisti funzionali & Leonardo Salviato & - & - \\
\hline
\rowcolor{secondaryblue!10} 0.1.4 & 16/12/2025 & Correzione errori, aggiunta index e sistemata visualizzazione tabelle nei casi d'uso, sistemazione indicizzazione & Leonardo Salviato & - & - \\
\hline
\rowcolor{secondaryblue!10} 0.1.3 & 13/12/2025 & Aggiunta sotto-casi d'uso sezione due, riscritta struttura requisiti e relative tabelle & Leonardo Salviato & - & - \\
\hline
\rowcolor{secondaryblue!10} 0.1.2 & 11/12/2025 & Aggiunto modulo 3 & Marco Piro & - & - \\
\hline
\rowcolor{secondaryblue!10} 0.1.1 & 10/12/2025 & Rinominazione casi d'uso, aggiunta sotto-casi d'uso alle sezioni 0 e 1. & Leonardo Salviato & - & - \\
\hline
\rowcolor{secondaryblue!10} 0.1.0 & 07/12/2025 & Verifica dei primi casi d'uso & - & Linor Sadè & - \\
\hline
\rowcolor{secondaryblue!10} 0.0.9 & 06/12/2025 & Aggiunti a ogni sezione scenari secondari, relativi casi d'uso (eccezioni e varianti) e trigger. & Leonardo Salviato & - & - \\
\hline
\rowcolor{secondaryblue!10} 0.0.8 & 05/12/2025 & Riscrittura casi d'uso con aggiornato grado di precisione (sezione 2). & Leonardo Salviato & - & - \\
\hline
\rowcolor{secondaryblue!10} 0.0.7 & 04/12/2025 & Riscrittura casi d'uso con aggiornato grado di precisione (sezioni 0 e 1). & Leonardo Salviato & - & - \\
\hline
\rowcolor{secondaryblue!10} 0.0.6 & 03/12/2025 & Aggiunti casi d'uso sezione 2, aggiunta varianti/exceptions sezione 2, scritta possibile struttura requisiti. & Leonardo Salviato & - & - \\
\hline
\rowcolor{secondaryblue!10} 0.0.5 & 02/12/2025 & Aggiunto schema attori & Marco Piro & - & - \\
\hline
\rowcolor{secondaryblue!10} 0.0.4 & 30/11/2025 & Sistemazione Attori. & Marco Piro & - & - \\
\hline
\rowcolor{secondaryblue!10} 0.0.3 & 29/11/2025 & Correzione casi d'uso e aggiunta schemi. & Leonardo Salviato & - & - \\
\hline
\rowcolor{secondaryblue!10} 0.0.2 & 25/11/2025 & Riscrittura della prima stesura e modifica casi d'uso. & Leonardo Salviato & - & - \\
\hline
\rowcolor{secondaryblue!10} 0.0.1 & 16/11/2025 & Prima stesura della struttura del documento. & Leonardo Salviato e Marco Piro & - & - \\
\hline
\end{longtable}
}

\vfill
\begin{center}
2 di \pageref{LastPage}
\end{center}

\newpage

\tableofcontents

\newpage
\listoftables
\listoffigures


\newpage

\section{Introduzione}

\subsection{Scopo del documento}
Il documento di Analisi dei Requisiti\textsubscript{\scalebox{0.6}{\textbf{G}}} ha lo scopo di definire in maniera precisa e dettagliata i requisiti funzionali\textsubscript{\scalebox{0.6}{\textbf{G}}} e non funzionali del sistema software da sviluppare.
A seguito delle nuove decisioni progettuali rispetto alle proposte del capitolato, il sistema non sarà inizialmente integrato nella piattaforma NEXUM, ma verrà realizzato come 
\textbf{applicazione standalone}, cioè indipendente dal sistema già esistente. 
Tale applicazione implementerà i moduli "AI Assistant Generativo" e "AI Co-Pilot per i CdL" in un ambiente isolato, così da consentire una fase di sviluppo, test e validazione più controllata. 
Eggon avrà la possibilità di valutare il prototipo standalone ed eventualmente procedere con l'integrazione nella piattaforma NEXUM.
Il documento include una descrizione approfondita dei casi d'uso, che costituiscono la principale fonte dei requisiti finali. Per agevolare la comprensione, verranno utilizzati anche i \textbf{diagrammi dei casi d'uso}, che visualizzano le interazioni tra utenti e sistema.
Questo documento rappresenta il riferimento fondamentale per la progettazione, l'implementazione e il collaudo dell'applicazione standalone, 
assicurando che essa soddisfi pienamente le esigenze del committente e gli obiettivi formativi del progetto.

I requisiti identificati sono classificati nelle seguenti categorie:
\begin{itemize}
    \item \textbf{Obbligatori}: necessari e imprescindibili per garantire il corretto funzionamento dell'applicazione standalone;
    \item \textbf{Desiderabili}: non strettamente necessari, ma capaci di migliorare l'esperienza utente o l'efficienza del sistema;
    \item \textbf{Opzionali}: funzionalità aggiuntive utili per estensioni future, in particolare in vista della possibile integrazione con NEXUM.
\end{itemize}

Il documento è rivolto ai seguenti destinatari:
\begin{itemize}
    \item Il \textbf{committente}, che potrà verificare che i requisiti siano stati compresi e documentati correttamente;
    \item La \textbf{proponente} (Eggon), che potrà utilizzare questo documento per monitorare l'aderenza del progetto alle specifiche concordate.
    \item Il \textbf{team di progettisti e programmatori}, che utilizzerà questa analisi come base per la realizzazione del sistema;
    \item Il \textbf{team di verificatori}, che impiegherà il presente documento per definire i casi di test e validare il comportamento del prodotto.
\end{itemize}

\subsection{Prospettiva del prodotto}

Il prodotto che BugBusters intende sviluppare è una versione standalone dei moduli "AI Assistant Generativo" e "AI Co-Pilot per i CdL", svincolata dalla piattaforma NEXUM. Tale applicazione costituirà un prototipo funzionale in grado di operare autonomamente e di implementare le principali funzionalità richieste dal committente, senza dipendere dalla piattaforma esistente.

Tale scelta deriva dalla necessità di rispettare le politiche di privacy e sicurezza dei dati della proponente, le quali risultano incompatibili con la richiesta del committente di rendere il prodotto accessibile al pubblico.

L'app standalone permetterà di testare e consolidare le funzionalità richieste, offrendo un ambiente controllato che faciliti la sperimentazione e lo sviluppo incrementale. Questa fase costituirà la base per un'eventuale integrazione futura con la piattaforma NEXUM, la quale fornirà un ecosistema HR completo e dotato di servizi quali la messaggistica top-down, la timbratura digitale, la gestione delle anagrafiche e dei ruoli, e la collaborazione con gli studi dei Consulenti del Lavoro (CdL).

Per questo motivo, l'architettura software prevede l'utilizzo di mock o simulazioni delle componenti NEXUM. Tale approccio garantisce alla proponente una futura e agevole integrazione del sistema, evitando al contempo che il team di sviluppo debba reimplementare funzionalità già esistenti ma non strettamente necessarie nella fase attuale.

I moduli saranno quindi concepiti in modo modulare per permettere all'applicazione, in un'eventuale integrazione futura con NEXUM, di inserirsi nell'architettura esistente come componente riutilizzabile e scalabile. L'integrazione includerà l'adattamento delle API, l'allineamento della gestione utenti e la centralizzazione dei dati all'interno dell'infrastruttura NEXUM.

\subsection{Funzioni del prodotto}

Qui di seguito le funzionalità del prodotto descritte in breve e divise per i due moduli che andranno sviluppati:

\textbf{Modulo AI Assistant Generativo}
\begin{itemize}
    \item \textbf{Generazione di contenuti tramite AI}: 
    generazione di titolo, testo e immagine di copertina a partire da un prompt, 
    con possibilità di selezionare tono e stile.
    \item \textbf{Salvataggio locale}:
    gestione interna di prompt, contenuti generati, immagini e valutazioni, 
    tramite archivio locale dedicato all'app standalone, indipendente dalla piattaforma NEXUM.
    
    \item \textbf{Sistema di rating}: 
    valutazione della qualità dei contenuti generati dall'AI, utile per analisi interne e miglioramento continuo.

    \item \textbf{Gestione dei prompt}: 
    storico dei prompt utilizzati con possibilità di riutilizzo, duplicazione e ricerca interna.

    \item \textbf{Dashboard standalone}: 
    visualizzazione e gestione di storico, filtri, ricerca e analisi delle interazioni con l'AI generativa.

    \item \textbf{Gestione delle immagini}: 
    possibilità di caricare immagini dall'utente o di generarle tramite AI, con salvataggio locale.

\end{itemize}

\textbf{Modulo AI Co-Pilot per i CdL}
\begin{itemize}
    \item \textbf{Upload e gestione documentale}: 
    possibilità di caricare documenti (PDF, ZIP, ecc.), salvarli localmente e gestirne lo stato di elaborazione.
    
    \item \textbf{Riconoscimento automatico della tipologia di documento}: 
    classificazione tramite AI (cedolini, CU, comunicazioni, lettere, moduli da firmare, ecc.) 
    sfruttando modelli OCR e classificatori addestrati.
    
    \item \textbf{Estrazione dei destinatari}: 
    riconoscimento automatico di informazioni contenute nei documenti 
    (nome, cognome, codice fiscale, matricola, reparto) tramite tecniche AI di entity extraction.
    
    \item \textbf{Split dei documenti massivi}: 
    suddivisione automatica dei documenti multi-destinatario (es. cedolini massivi) 
    in documenti singoli, ognuno associato al proprio destinatario riconosciuto.
    
    \item \textbf{Revisione manuale (Human-in-the-Loop)}: 
    interfaccia dedicata per verificare, correggere o confermare i risultati ottenuti dall'AI 
    in ogni fase (classificazione, destinatari, split).
    
    \item \textbf{Creazione di messaggi e liste di distribuzione}: 
    generazione automatica di bozze di messaggi e liste di destinatari derivanti dai documenti processati.
    
    \item \textbf{Tracciamento locale}: 
    storico delle operazioni effettuate (upload, riconoscimento, revisioni, esportazioni), 
    utile per audit interni e analisi del flusso documentale.
\end{itemize}  
\textbf{Altre funzionalità dell'applicazione:}
\begin{itemize}
    \item \textbf{Gestione utenti}: 
    registrazione, autenticazione, gestione del profilo e configurazione dei parametri AI 
    (per utenti privilegiati come amministratori o editor avanzati).
    \item \textbf{Analisi e reportistica}:
    dashboard di monitoraggio delle performance dei moduli AI per consentire valutazioni semplici su efficacia ed efficienza dei moduli. L'appicazione consentirà inoltre
    la possibilità di rilasciare miglioramenti in seguito ad uun intervento umano. 
\end{itemize}

Queste funzionalità permetteranno all'applicazione standalone di essere completamente operativa e autonoma nei due moduli (AI Assistant Generativo e AI Co-Pilot per i CdL). 
In una fase successiva, tali componenti saranno progettati per essere integrati nella piattaforma NEXUM, 
consentendo così un'evoluzione verso un ecosistema HR completo, scalabile e basato su automazioni intelligenti.

\subsection{Caratterisitiche dell'utente}


Gli utilizzatori finali dell'applicazione standalone non appartengono a un'unica categoria specifica: 
l'obiettivo è quello di progettare moduli intelligenti e interoperabili per essere in futuro integrati nella piattaforma NEXUM. Tale ecosistema è concepito per rispondere alle esigenze di un ampio spettro di organizzazioni e professionisti nel settore delle risorse umane e della consulenza del lavoro.

In generale, è possibile affermare che gli utenti finali sono coloro che necessitano di uno strumento scalabile, 
intelligente e semplice da utilizzare per generare contenuti tramite AI e per gestire flussi documentali complessi con il supporto del modulo Co-Pilot.
Rientrano in questa categoria:

\begin{itemize}
    \item \textbf{Responsabili e amministratori HR}, che necessitano di strumenti avanzati per la creazione di comunicazioni interne, 
    la gestione dei contenuti generativi e l'analisi delle produzioni.

    \item \textbf{Consulenti del Lavoro (CdL) e personale amministrativo}, che richiedono un sistema in grado di caricare, riconoscere, suddividere e preparare documenti per la distribuzione ai destinatari.

    \item \textbf{Dipendenti e collaboratori} (in fase integrata), che potranno interagire con la piattaforma NEXUM per consultare documenti e comunicazioni, 
    pur non essendo utenti della versione standalone.

    \item \textbf{Manager aziendali}, interessati a monitorare la consistenza delle comunicazioni e l'efficienza dei processi documentali, 
    sia nella versione standalone che nella futura integrazione.
\end{itemize}

In sintesi, il prodotto è rivolto a organizzazioni di varie dimensioni — in particolare aziende medio-grandi e studi professionali — 
che necessitano di strumenti intelligenti per la creazione di contenuti, 
la gestione automatizzata dei documenti e la collaborazione con gli studi dei Consulenti del Lavoro.
L'app standalone funge da primo passo verso una piattaforma HR completa, modulare e potenziata dall'AI.


\subsection{Definizioni, acronimi e abbreviazioni (Glossario)}
Per tutte le definizioni, acronimi e abbreviazioni utilizzati in questo documento, si faccia
riferimento al \textbf{Glossario}, fornito come documento separato, che contiene tutte le spiegazioni
necessarie per garantire una comprensione uniforme dei termini tecnici e dei concetti
rilevanti per il progetto.

\newpage

\subsection{Riferimenti}

\subsubsection{Riferimenti normativi}
\begin{itemize}
\item \textbf{Capitolato\textsubscript{\scalebox{0.6}{\textbf{G}}}
 d'appalto C5: Nexum - Piattaforma di consulenza e documentazione previdenziale}\\
\url{https://www.math.unipd.it/~tullio/IS-1/2025/Progetto/C5.pdf}
\end{itemize}

\subsubsection{Riferimenti informativi}
\begin{itemize}
\item \textbf{Glossario\textsubscript{\scalebox{0.6}{\textbf{G}}}
:}\\
\url{https://github.com/BugBustersUnipd/DocumentazioneSWE/blob/main//RTB/GLOSSARIO/Glossario.pdf}
\end{itemize}



\section{Casi d'uso}
\subsection{Introduzione}
I casi d'uso si compongono di un grafico UML e una descrizione testuale che permette di
comprendere al meglio cosa il prodotto deve fornire. La descrizione testuale, in particolar
modo, dovrà contenere le informazioni sotto presenti, salvo i casi in cui lo
specifico campo non risulti rilevante (ad esempio, un caso d'uso\textsubscript{\scalebox{0.6}{\textbf{G}}} che non prevede la
possibilità di errori non avrà scenari alternativi):

\begin{itemize}
    \item \textbf{Attori}: Sono coloro che interagiscono attivamente con il sistema e
    svolgono l'azione indicata dal caso d'uso
    \item \textbf{Attori secondari}: Sono coloro che interagiscono passivamente con il sistema
    \item \textbf{Precondizioni}: Lista di elementi che sono necessari affinchè l'attore possa
    compiere l'azione indicata dal caso d'uso
    \item \textbf{Postcondizioni}: Lista di elementi che descrivono come il sistema risulta
    essere internamente cambiato dopo che l'attore ha effettuato
    l'azione prevista dal caso d'uso
    \item \textbf{Scenario principale}: Descrizione ragionevole delle operazioni che l'attore deve
    fare per compiere l'azione descritta dal caso d'uso
    \item \textbf{Scenario secondario} (scenari alternativi): Descrizione ragionevole degli eventi che possono accadere
    qualora una delle operazioni descritte nello scenario
    principale non vada a buon fine
    \item \textbf{Inclusioni}: Casi d'uso ulteriori che l'attore deve compiere per realizzare
    il caso d'uso attualmente descritto
    \item \textbf{Estensioni}: Casi d'uso ulteriori che possono realizzarsi durante
    l'esecuzione delle operazioni del caso d'uso principale
    
\end{itemize}
Motivazioni che portano l'attore a svolgere l'azione descritta
dal caso d'uso. Non sempre disponibile in quanto il caso
d'uso potrebbe essere incluso da un altro caso d'uso principale.

\subsection{Attori}
Nella \hyperref[fig:1]{figura~1} sono riportati gli attori considerati in questa analisi dei requisiti.

Si fa presente che alcuni attori hanno senso nel contesto dell'applicazione che implementa l'autenticazione e la gestione dei ruoli. Tuttavia tale funzionalità, come verrà descritto più avanti, è requisito opzionale, pertanto gli attori devono essere considerati semplice utente se letti in un contesto in cui tale funzionalità non è implementata. 
\begin{figure}[H]
    \centering
    \includegraphics[width=0.8\textwidth]{Diagrammi casi d'uso/diagramma_attori.jpg}
    \caption{Diagramma degli attori principali}
    \label{fig:1}
\end{figure}
\begin{itemize}
    \item \textbf{Utente}: Rappresenta un utente che vuole accedere al sistema.
    \item \textbf{HR Manager}: È la figura responsabile della comunicazione interna. Utilizza il modulo AI Assistant per generare, revisionare e pubblicare messaggi o avvisi rivolti ai dipendenti, definendone tono e stile.
    \item \textbf{Redattore}: Utente con la capacitá di generare post
    \item \textbf{Data Analyst}: Figura incaricata di monitorare le prestazioni. Accede alle dashboard di analisi per consultare le statistiche di utilizzo, i rating di qualità dei contenuti generati e i KPI del riconoscimento documentale.
    \item \textbf{Auditor interno}: Utente con permessi di sola lettura focalizzato sul controllo. Verifica lo storico delle operazioni (audit trail) per garantire la tracciabilità e la sicurezza dei flussi documentali.
    \item \textbf{Amministratore}: Gestisce la configurazione tecnica dell'applicazione standalone. Si occupa della creazione degli utenti, della gestione dei ruoli e della configurazione dei parametri globali dell'AI (es. prompt di sistema o soglie di confidenza).
    \item \textbf{Operatore Studio CdL}: È l'utente principale del modulo AI Co-Pilot. Si occupa di caricare i flussi documentali (es. cedolini massivi), supervisionare il riconoscimento automatico (validazione Human-in-the-Loop) e gestire le liste di distribuzione.
    \item \textbf{AI Post Generator}: Modello AI esterno utilizzato per la creazione di contenuti all'interno del modulo AI-assistant
    \item \textbf{AI Analyst}: Modello AI esterno utilizzato per l'analisi dei documenti all'interno del modulo AI-Copilot
\end{itemize}


\subsection{Lista casi d'uso}

\text{L'elenco dei casi d'uso sarà diviso in tre parti:}
\begin{itemize}
    \item 0 - Casi d'uso per la gestione utenti e autenticazione
    \item 1 - Casi d'uso per il modulo "AI Assistant Generativo"
    \item 2 - Casi d'uso per il modulo "AI Co-Pilot"
\end{itemize}


\subsection{Sezione 0 – Applicazione standalone}

\subsubsection{UC-0A – Registrazione nuovo utente}\label{sec:uc-0a} 

\begin{figure}[H]
    \centering
    \includegraphics[width=1\textwidth]{Diagrammi casi d'uso/UC0A.jpg}
    \caption{Diagramma del caso d'uso UC-0A – Registrazione nuovo utente}
\end{figure}


\textbf{Attori}
\begin{itemize}
    \item Utente
\end{itemize}

\textbf{Pre-condizioni}
\begin{itemize}
    \item L'utente non ha una sessione attiva.
    \item L'utente non è ancora registrato nel sistema (l'e-mail inserita non risulta già presente).
\end{itemize}

\textbf{Post-condizioni}
\begin{itemize}
    \item Esiste un nuovo account utente registrato nel sistema.
    \item L'utente può effettuare il login utilizzando le credenziali appena create.
\end{itemize}

\textbf{Scenario principale}
\begin{enumerate}
    \item L'utente accede alla schermata di registrazione dell'applicazione standalone.
    \item L'utente inserisce i dati richiesti (ad esempio: nome, cognome, e-mail, password).
    \item Il sistema verifica la correttezza formale dei dati inseriti (es. formato e-mail, forza della password).
    \item Il sistema controlla che l'indirizzo e-mail non sia già associato a un account esistente.
    \item In caso di esito positivo, il sistema crea un nuovo account utente e lo memorizza nel proprio archivio.
    \item Il sistema conferma l'avvenuta registrazione e può opzionalmente eseguire il login automatico del nuovo utente.
\end{enumerate}

\textbf{Scenario secondario}
\begin{enumerate}
    \item Nell'inserimento dei dati, uno o più campi non rispettano i requisiti di validità (es. e-mail non valida, password debole).
    \item Il sistema mostra un messaggio di errore specifico per il campo non valido.
    \item L'utente corregge i dati e ripete l'inserimento.
\end{enumerate}

\textbf{Relazioni con altri casi d'uso (\textit{include} / \textit{extend})}
\begin{itemize}
    \item \textit{include}:
    \begin{itemize}
        \item UC-0A.1 - Inserimento email
        \item UC-0A.2 - Inserimento password
        \item UC-0A.3 - Inserimento username
        \item UC-0A.4 - Inserimento nome 
        \item UC-0A.5 - Inserimento cognome
        \item UC-0A.6 - Inserimento matricola
    \end{itemize}
    \item \textit{extend}: 
    \begin{itemize}
        \item UC-0B – Login / Autenticazione utente (in caso di login automatico al termine della registrazione).
        \item UC-0A.7 – email non valida.
        \item UC-0A.8 – password non valida.
        \item UC-0A.9 – email già registrata.
        \item UC-0A.10 – username già registrato.
        \item UC-0A.11 – matricola già registrata.
        \item UC-0A.12 – matricola non valida.
    \end{itemize}
\end{itemize}

\vspace{0.5cm}

\subsubsection{UC-0A.1 – Inserimento email}\label{sec:uc-0a-1}

\textbf{Attori}
\begin{itemize}
    \item utente
\end{itemize}

\textbf{Pre-condizioni}
\begin{itemize}
    \item Casella di testo vuota
\end{itemize}

\textbf{Post-condizioni}
\begin{itemize}
    \item Casella di testo contenente l'email inserita
\end{itemize}

\textbf{Scenario principale}
\begin{enumerate}
    \item L'utente inserisce la propria email nell'apposita casella di testo
\end{enumerate}

\vspace{0.5cm}

\subsubsection{UC-0A.2 – Inserimento password}\label{sec:uc-0a-2}

\textbf{Attori}
\begin{itemize}
    \item utente
\end{itemize}

\textbf{Pre-condizioni}
\begin{itemize}
    \item Casella di testo vuota
\end{itemize}

\textbf{Post-condizioni}
\begin{itemize}
    \item Casella di testo contenente la password inserita
\end{itemize}

\textbf{Scenario principale}
\begin{enumerate}
    \item L'utente inserisce la propria password nell'apposita casella di testo
\end{enumerate}


\vspace{0.5cm}

\subsubsection{UC-0A.3 – Inserimento username}\label{sec:uc-0a-3}

\textbf{Attori}
\begin{itemize}
    \item utente
\end{itemize}

\textbf{Pre-condizioni}
\begin{itemize}
    \item Casella di testo vuota
\end{itemize}

\textbf{Post-condizioni}
\begin{itemize}
    \item Casella di testo contenente l'username inserito
\end{itemize}

\textbf{Scenario principale}
\begin{enumerate}
    \item L'utente inserisce il proprio username nell'apposita casella di testo
\end{enumerate}

\vspace{0.5cm}

\subsubsection{UC-0A.4 – Inserimento nome\label{sec:uc-0a-4}}

\textbf{Attori}
\begin{itemize}
    \item utente
\end{itemize}

\textbf{Pre-condizioni}
\begin{itemize}
    \item Casella di testo vuota
\end{itemize}

\textbf{Post-condizioni}
\begin{itemize}
    \item Casella di testo contenente il nome inserito
\end{itemize}

\textbf{Scenario principale}
\begin{enumerate}
    \item L'utente inserisce il proprio nome nell'apposita casella di testo
\end{enumerate}

\vspace{0.5cm}

\subsubsection{UC-0A.5 – Inserimento cognome}\label{sec:uc-0a-5}

\textbf{Attori}
\begin{itemize}
    \item utente
\end{itemize}

\textbf{Pre-condizioni}
\begin{itemize}
    \item Casella di testo vuota
\end{itemize}

\textbf{Post-condizioni}
\begin{itemize}
    \item Casella di testo contenente il cognome inserito
\end{itemize}

\textbf{Scenario principale}
\begin{enumerate}
    \item L'utente inserisce il proprio cognome nell'apposita casella di testo
\end{enumerate}


\vspace{0.5cm}

\subsubsection{UC-0A.6 – Inserimento matricola}\label{sec:uc-0a-6}

\textbf{Attori}
\begin{itemize}
    \item utente
\end{itemize}

\textbf{Pre-condizioni}
\begin{itemize}
    \item Casella di testo vuota
\end{itemize}

\textbf{Post-condizioni}
\begin{itemize}
    \item Casella di testo contenente la matricola inserita
\end{itemize}

\textbf{Scenario principale}
\begin{enumerate}
    \item L'utente inserisce la propria matricola nell'apposita casella di testo
\end{enumerate}

\vspace{0.5cm}

\subsubsection{UC-0A.7 – Email non valida}\label{sec:uc-0a-7}

\textbf{Attori}
\begin{itemize}
    \item utente
\end{itemize}

\textbf{Pre-condizioni}
\begin{itemize}
    \item L'utente ha inserito dei caratteri nel campo email
\end{itemize}

\textbf{Post-condizioni}
\begin{itemize}
    \item Viene visualizzato un messaggio di errore relativo al formato non valido dell'email
\end{itemize}

\textbf{Scenario principale}
\begin{enumerate}
    \item L'utente inserisce un indirizzo email che non rispetta il formato standard (es. manca la chiocciola o il dominio)
    \item Il sistema rileva che il formato non è corretto
\end{enumerate}

\vspace{0.5cm}

\subsubsection{UC-0A.8 – Password non valida}\label{sec:uc-0a-8}

\textbf{Attori}
\begin{itemize}
    \item utente
\end{itemize}

\textbf{Pre-condizioni}
\begin{itemize}
    \item L'utente ha inserito dei caratteri nel campo password
\end{itemize}

\textbf{Post-condizioni}
\begin{itemize}
    \item Viene visualizzato un messaggio di errore relativo ai requisiti di sicurezza della password
\end{itemize}

\textbf{Scenario principale}
\begin{enumerate}
    \item L'utente inserisce una password che non soddisfa i criteri minimi di sicurezza (es. lunghezza minima, caratteri speciali)
    \item Il sistema rileva che la password è troppo debole
\end{enumerate}

\vspace{0.5cm}

\subsubsection{UC-0A.9 – Email già registrata}\label{sec:uc-0a-9}

\textbf{Attori}
\begin{itemize}
    \item utente
\end{itemize}

\textbf{Pre-condizioni}
\begin{itemize}
    \item L'indirizzo email inserito è già presente all'interno del sistema
\end{itemize}

\textbf{Post-condizioni}
\begin{itemize}
    \item Viene visualizzato un messaggio di errore che notifica l'esistenza dell'account
\end{itemize}

\textbf{Scenario principale}
\begin{enumerate}
    \item L'utente inserisce un'email formalmente valida ma già associata ad un altro utente registrato
    \item Il sistema verifica la presenza dell'email nel database e blocca l'operazione
\end{enumerate}

\vspace{0.5cm}

\subsubsection{UC-0A.10 – Username già registrato}\label{sec:uc-0a-10}

\textbf{Attori}
\begin{itemize}
    \item utente
\end{itemize}

\textbf{Pre-condizioni}
\begin{itemize}
    \item Lo username inserito è già presente all'interno del sistema
\end{itemize}

\textbf{Post-condizioni}
\begin{itemize}
    \item Viene visualizzato un messaggio di errore che indica che lo username non è disponibile
\end{itemize}

\textbf{Scenario principale}
\begin{enumerate}
    \item L'utente inserisce uno username già utilizzato da un altro utente
    \item Il sistema verifica l'univocità dello username e ne segnala l'indisponibilità
\end{enumerate}

\vspace{0.5cm}

\subsubsection{UC-0A.11 – Matricola già registrata}\label{sec:uc-0a-11}

\textbf{Attori}
\begin{itemize}
    \item utente
\end{itemize}

\textbf{Pre-condizioni}
\begin{itemize}
    \item La matricola inserita è già associata ad un account esistente
\end{itemize}

\textbf{Post-condizioni}
\begin{itemize}
    \item Viene visualizzato un messaggio di errore che impedisce la registrazione multipla con la stessa matricola
\end{itemize}

\textbf{Scenario principale}
\begin{enumerate}
    \item L'utente inserisce un numero di matricola già presente nel sistema
    \item Il sistema rileva la duplicazione e impedisce il proseguimento
\end{enumerate}

\vspace{0.5cm}

\subsubsection{UC-0A.12 – Matricola non valida}\label{sec:uc-0a-12}

\textbf{Attori}
\begin{itemize}
    \item utente
\end{itemize}

\textbf{Pre-condizioni}
\begin{itemize}
    \item L'utente ha inserito dei dati nel campo matricola
\end{itemize}

\textbf{Post-condizioni}
\begin{itemize}
    \item Viene visualizzato un messaggio di errore sul formato della matricola
\end{itemize}

\textbf{Scenario principale}
\begin{enumerate}
    \item L'utente inserisce una matricola che non rispetta il formato atteso (es. contiene lettere dove non previste o ha una lunghezza errata)
    \item Il sistema invalida il dato inserito
\end{enumerate}

\vspace{0.5cm}


\subsubsection{UC-0B – Login / Autenticazione utente}\label{sec:uc-0b}

\begin{figure}[H]
    \centering
    \includegraphics[width=0.7\textwidth]{Diagrammi casi d'uso/UC0B.jpg}
    \caption{Diagramma del caso d'uso UC-0B – Login / Autenticazione utente}
\end{figure}

\textbf{Attori}
\begin{itemize}
    \item Utente 
\end{itemize}

\textbf{Pre-condizioni}
\begin{itemize}
    \item L'utente è già registrato nel sistema.
    \item Non esiste una sessione attiva associata all'utente sul dispositivo corrente.
\end{itemize}

\textbf{Post-condizioni}
\begin{itemize}
    \item L'utente risulta autenticato nel sistema.
    \item È attiva una sessione associata all'utente, che consente l'accesso alle funzionalità riservate (es. generazione contenuti, upload documenti).
\end{itemize}

\textbf{Scenario principale}
\begin{enumerate}
    \item L'utente accede alla schermata di login.
    \item L'utente inserisce le proprie credenziali (e-mail e password).
    \item Il sistema verifica la correttezza delle credenziali.
    \item In caso di credenziali valide, il sistema crea una nuova sessione autenticata per l'utente.
    \item Il sistema reindirizza l'utente alla dashboard principale dell'applicazione standalone.
\end{enumerate}

\textbf{Scenario secondario}
\begin{enumerate}
    \item L'utente inserisce un'e-mail non registrata nel sistema.
    \item L'utente inserisce una password errata.
    \item L'utente inserisce credenziali non valide.
\end{enumerate}

\textbf{Relazioni con altri casi d'uso (\textit{include} / \textit{extend})}
\begin{itemize}
    \item \textit{include}:
    \begin{itemize}
        \item UC-0A.1 - Inserimento email
        \item UC-0A.2 - Inserimento password
    \end{itemize}
    \item \textit{extend}: 
    \begin{itemize}
        \item UC-0A.7 – email non valida.
        \item UC-0A.8 – password non valida.
        \item UC-0B.1 – email non registrata.
        \item UC-0B.2 – password errata.
    \end{itemize}
\end{itemize}

\vspace{0.5cm}

\subsubsection{UC-0B.1 – Email non registrata}\label{sec:uc-0b-1}

\textbf{Attori}
\begin{itemize}
    \item utente
\end{itemize}

\textbf{Pre-condizioni}
\begin{itemize}
    \item L'email inserita nel modulo di accesso non è presente nel database del sistema
\end{itemize}

\textbf{Post-condizioni}
\begin{itemize}
    \item Viene visualizzato un messaggio di errore e l'accesso viene negato
\end{itemize}

\textbf{Scenario principale}
\begin{enumerate}
    \item L'utente tenta di effettuare il login inserendo un indirizzo email non associato ad alcun account
    \item Il sistema verifica l'esistenza dell'email e non trova corrispondenze
\end{enumerate}

\vspace{0.5cm}

\subsubsection{UC-0B.2 – Password errata}\label{sec:uc-0b-2}

\textbf{Attori}
\begin{itemize}
    \item utente
\end{itemize}

\textbf{Pre-condizioni}
\begin{itemize}
    \item L'email inserita è corretta, ma la password non corrisponde a quella salvata nel sistema
\end{itemize}

\textbf{Post-condizioni}
\begin{itemize}
    \item Viene visualizzato un messaggio di errore relativo alle credenziali non valide
\end{itemize}

\textbf{Scenario principale}
\begin{enumerate}
    \item L'utente inserisce la propria email (corretta) e una password errata
    \item Il sistema verifica la corrispondenza delle credenziali e rileva l'errore
\end{enumerate}

\vspace{0.5cm}


\subsubsection{UC-0C – Apertura dashboard}\label{sec:uc-0c}

\begin{figure}[H]
    \centering
    \includegraphics[width=0.7\textwidth]{Diagrammi casi d'uso/UC0C.jpg}
    \caption{Diagramma del caso d'uso UC-0C – Apertura dashboard}
\end{figure}

\textbf{Attori}
\begin{itemize}
    \item Utente 
\end{itemize}

\textbf{Pre-condizioni}
\begin{itemize}
    \item L'utente ha effettuato il login ed è autenticato.
    \item Esiste un profilo associato all'utente nel sistema
\end{itemize}


\textbf{Post-condizioni}
\begin{itemize}
    \item L'utente autenticato si può eseguire diverse azioni dal modulo dashboard
\end{itemize}

\textbf{Scenario principale}
\begin{enumerate}
    \item L'utente dopo aver effettuato il login si ritrova in una schermata con la scelta di diversi moduli.
\end{enumerate}

\textbf{Scenario secondario}
\begin{enumerate}
    \item 
\end{enumerate}

\textbf{Relazioni con altri casi d'uso (\textit{include} / \textit{extend})}
\begin{itemize}
    \item \textit{include}: 
    \begin{itemize}
        \item Nessuna
    \end{itemize}
    \item \textit{extend}: 
    \begin{itemize}
        \item Nessuna
    \end{itemize}
\end{itemize}


\vspace{0.5cm}

\subsubsection{UC-0D – Gestione profilo utente}\label{sec:uc-0d}

\begin{figure}[H]
    \centering
    \includegraphics[width=1\textwidth]{Diagrammi casi d'uso/UC0D.jpg}
    \caption{Diagramma del caso d'uso UC-0D – Gestione profilo utente}
\end{figure}

\textbf{Attori}
\begin{itemize}
    \item Utente 
\end{itemize}


\textbf{Pre-condizioni}
\begin{itemize}
    \item L'utente ha effettuato il login ed è autenticato.
    \item Esiste un profilo associato all'utente nel sistema
    \item L'utente è entrato nel modulo di gestione profilo utente dalla dashboard principale

\end{itemize}

\textbf{Post-condizioni}
\begin{itemize}
    \item Sono ottenibili le informazioni dal profilo utente e si possono modificare
\end{itemize}

\textbf{Scenario principale}
\begin{enumerate}
    \item L'utente accede alla sezione “Profilo” dalla dashboard dell'applicazione.
    \item Il sistema mostra i dati correnti del profilo
    \item L'utente può modificare uno o più campi del profilo 
    \item L'utente conferma le modifiche.
    \item Il sistema valida i dati inseriti (ad esempio formato dell'e-mail, campi obbligatori).
    \item Il sistema salva le modifiche nel proprio archivio.
    \item Il sistema conferma l'avvenuto aggiornamento del profilo.
\end{enumerate}

\textbf{Scenario secondario}
\begin{enumerate}
    \item 
\end{enumerate}


\textbf{Relazioni con altri casi d'uso (\textit{include} / \textit{extend})}
\begin{itemize}
    \item \textit{include}: 
    \begin{itemize}
        \item UC-0D.1 - Visualizzazione email
        \item UC-0D.2 - Visualizzazione password
        \item UC-0D.3 - Visualizzazione username
        \item UC-0D.4 - Visualizzazione nome 
        \item UC-0D.5 - Visualizzazione cognome
        \item UC-0D.6 - Visualizzazione matricola
    \end{itemize}
    \item \textit{extend}: 
    \begin{itemize}
        \item UC-0D.7 - Modifica informazioni profilo utente
    \end{itemize}
\end{itemize}

\vspace{0.5cm}

\subsubsection{UC-0D.1 – Visualizzazione email}\label{sec:uc-0d-1}

\textbf{Attori}
\begin{itemize}
    \item utente
\end{itemize}

\textbf{Pre-condizioni}
\begin{itemize}
    \item L'utente ha effettuato l'accesso e si trova nella pagina del profilo
\end{itemize}

\textbf{Post-condizioni}
\begin{itemize}
    \item Casella di testo precompilata con l'email attuale è mostrata all'utente
\end{itemize}

\textbf{Scenario principale}
\begin{enumerate}
    \item Il sistema recupera l'email associata all'account e la mostra nell'apposita casella
\end{enumerate}

\vspace{0.5cm}

\subsubsection{UC-0D.2 – Visualizzazione password}\label{sec:uc-0d-2}

\textbf{Attori}
\begin{itemize}
    \item utente
\end{itemize}

\textbf{Pre-condizioni}
\begin{itemize}
    \item L'utente ha effettuato l'accesso e si trova nella pagina del profilo
\end{itemize}

\textbf{Post-condizioni}
\begin{itemize}
    \item Casella di testo contenente la password attuale (tipicamente oscurata) è mostrata all'utente
\end{itemize}

\textbf{Scenario principale}
\begin{enumerate}
    \item Il sistema predispone il campo password permettendo all'utente di visualizzarne lo stato o modificarla
\end{enumerate}

\vspace{0.5cm}

\subsubsection{UC-0D.3 – Visualizzazione username}\label{sec:uc-0d-3}

\textbf{Attori}
\begin{itemize}
    \item utente
\end{itemize}

\textbf{Pre-condizioni}
\begin{itemize}
    \item L'utente ha effettuato l'accesso e si trova nella pagina del profilo
\end{itemize}

\textbf{Post-condizioni}
\begin{itemize}
    \item Casella di testo precompilata con lo username è mostrata all'utente
\end{itemize}

\textbf{Scenario principale}
\begin{enumerate}
    \item Il sistema recupera lo username associato all'account e lo mostra nell'apposita casella
\end{enumerate}

\vspace{0.5cm}

\subsubsection{UC-0D.4 – Visualizzazione nome}\label{sec:uc-0d-4}

\textbf{Attori}
\begin{itemize}
    \item utente
\end{itemize}

\textbf{Pre-condizioni}
\begin{itemize}
    \item L'utente ha effettuato l'accesso e si trova nella pagina del profilo
\end{itemize}

\textbf{Post-condizioni}
\begin{itemize}
    \item Casella di testo precompilata con il nome attuale è mostrata all'utente
\end{itemize}

\textbf{Scenario principale}
\begin{enumerate}
    \item Il sistema recupera il nome dell'utente e lo mostra nell'apposita casella
\end{enumerate}

\vspace{0.5cm}

\subsubsection{UC-0D.5 – Visualizzazione cognome}\label{sec:uc-0d-5}

\textbf{Attori}
\begin{itemize}
    \item utente
\end{itemize}

\textbf{Pre-condizioni}
\begin{itemize}
    \item L'utente ha effettuato l'accesso e si trova nella pagina del profilo
\end{itemize}

\textbf{Post-condizioni}
\begin{itemize}
    \item Casella di testo precompilata con il cognome attuale è mostrata all'utente
\end{itemize}

\textbf{Scenario principale}
\begin{enumerate}
    \item Il sistema recupera il cognome dell'utente e lo mostra nell'apposita casella
\end{enumerate}

\vspace{0.5cm}

\subsubsection{UC-0D.6 – Visualizzazione matricola}\label{sec:uc-0d-6}

\textbf{Attori}
\begin{itemize}
    \item utente
\end{itemize}

\textbf{Pre-condizioni}
\begin{itemize}
    \item L'utente ha effettuato l'accesso e si trova nella pagina del profilo
\end{itemize}

\textbf{Post-condizioni}
\begin{itemize}
    \item Casella di testo precompilata con la matricola attuale è mostrata all'utente
\end{itemize}

\textbf{Scenario principale}
\begin{enumerate}
    \item Il sistema recupera la matricola associata all'account e la mostra nell'apposita casella
\end{enumerate}

\vspace{0.5cm}

\subsubsection{UC-0D.7 – Modifica informazioni profilo utente}\label{sec:uc-0d-9}

\textbf{Attori}
\begin{itemize}
    \item utente
\end{itemize}

\textbf{Pre-condizioni}
\begin{itemize}
    \item L'utente si trova nella pagina di modifica profilo
\end{itemize}

\textbf{Post-condizioni}
\begin{itemize}
    \item L'utente ha cambiato i dati del proprio profilo
\end{itemize}

\textbf{Scenario principale}
\begin{enumerate}
    \item L'utente cambia i dati del proprio profilo
\end{enumerate}

\textbf{Relazioni con altri casi d'uso (\textit{include} / \textit{extend})}
\begin{itemize}
    \item \textit{include}: 
    \begin{itemize}
        \item Nessuna
    \end{itemize}
    \item \textit{extend}: 
    \begin{itemize}
        \item UC-0A.1 - Inserimento email
        \item UC-0A.2 - Inserimento password
        \item UC-0A.3 - Inserimento username
        \item UC-0A.4 - Inserimento nome 
        \item UC-0A.5 - Inserimento cognome
        \item UC-0A.6 - Inserimento matricola
        \item UC-0D.9 – Salva profilo utente.
        \item UC-0D.8 – Uscita senza salvare profilo utente.
    \end{itemize}
    
\end{itemize}

\vspace{0.5cm}

\subsubsection{UC-0D.8 – Uscita senza salvare profilo utente}\label{sec:uc-0d-8}

\textbf{Attori}
\begin{itemize}
    \item utente
\end{itemize}

\textbf{Pre-condizioni}
\begin{itemize}
    \item L'utente si trova nella pagina di modifica profilo
\end{itemize}

\textbf{Post-condizioni}
\begin{itemize}
    \item L'utente viene reindirizzato alla pagina precedente o alla home senza che alcuna modifica venga applicata al database
\end{itemize}

\textbf{Scenario principale}
\begin{enumerate}
    \item L'utente decide di annullare l'operazione di modifica
    \item Il sistema scarta le modifiche pendenti nei campi di testo e chiude la schermata
\end{enumerate}

\vspace{0.5cm}

\subsubsection{UC-0D.9 – Salva profilo utente}\label{sec:uc-0d-7}

\textbf{Attori}
\begin{itemize}
    \item utente
\end{itemize}

\textbf{Pre-condizioni}
\begin{itemize}
    \item L'utente ha modificato uno o più campi del proprio profilo
\end{itemize}

\textbf{Post-condizioni}
\begin{itemize}
    \item I dati aggiornati vengono salvati nel database oppure vengono mostrati i messaggi di errore pertinenti
\end{itemize}

\textbf{Scenario principale}
\begin{enumerate}
    \item L'utente preme il pulsante di salvataggio delle modifiche
    \item Il sistema verifica la validità dei nuovi dati inseriti (controllo formato, unicità email/username/matricola analogamente alla fase di registrazione)
    \item Se non sono presenti errori, il sistema sovrascrive i dati precedenti con quelli nuovi; in caso contrario, segnala l'errore specifico al campo interessato
\end{enumerate}

\vspace{0.5cm}


\subsubsection{UC-0E – Gestione ruoli}\label{sec:uc-0e}

\begin{figure}[H]
    \centering
    \includegraphics[width=0.7\textwidth]{Diagrammi casi d'uso/UC0B.jpg}
    \caption{Diagramma del caso d'uso UC-0B – Login / Autenticazione utente}
\end{figure}

\textbf{Attori}
\begin{itemize}
    \item Amministratore
\end{itemize}


\textbf{Pre-condizioni}
\begin{itemize}
    \item L'utente amministratore ha effettuato il login ed è autenticato come Admin.
    \item Esistono uno o più account utente registrati nel sistema.
    \item L'utente amministratore è entrato nel modulo di gestione ruoli dalla dashboard principale.
\end{itemize}

\textbf{Post-condizioni}
\begin{itemize}
    \item L'amministratore vede gli utenti registrati con i rispettivi ruoli e ha la possibilitá di modificarli
\end{itemize}

\textbf{Scenario principale}
\begin{enumerate}
    \item L'Amministratore accede alla sezione di amministrazione utenti.
    \item Il sistema mostra l'elenco degli utenti registrati, con i rispettivi ruoli correnti.
    \item L'Amministratore seleziona un utente da modificare.
    \item L'Amministratore assegna o modifica il ruolo dell'utente (es. da Editor a Admin, oppure rimozione privilegi).
    \item L'Amministratore conferma le modifiche.
    \item Il sistema aggiorna i ruoli e i permessi associati all'utente.
    \item Il sistema registra l'operazione per finalità di audit interno.
\end{enumerate}

\textbf{Scenario secondario}
\begin{enumerate}
    \item Un utente non autorizzato tenta di accedere al modulo di gestione ruoli.
\end{enumerate}


\textbf{Relazioni con altri casi d'uso (\textit{include} / \textit{extend})}
\begin{itemize}
    \item \textit{include}: 
    \begin{itemize}
        \item UC-0E.1 - Visualizzazione lista utenti registrati.

    \end{itemize}
    \item \textit{extend}: 
    \begin{itemize}
        \item UC-0E.6 - Modifica ruolo utente registrato.
        \item UC-0E.9 - Salva modifica ruolo utente registrato.
        \item UC-0E.8 - Annulla modifica ruolo utente registrato
        \item UC-0E.7 - Utente non autorizzato.
    \end{itemize}
    
\end{itemize}

\vspace{0.5cm}

\subsubsection{UC-0E.1 – Visualizzazione lista utenti registrati}\label{sec:uc-0e-1}

\textbf{Attori}
\begin{itemize}
    \item Amministratore
\end{itemize}

\textbf{Pre-condizioni}
\begin{itemize}
    \item L'amministratore ha effettuato l'accesso e si trova nella sezione di gestione utenti
\end{itemize}

\textbf{Post-condizioni}
\begin{itemize}
    \item Viene visualizzata la lista degli utenti registrati nel sistema
\end{itemize}

\textbf{Scenario principale}
\begin{enumerate}
    \item Il sistema recupera dal database la lista degli utenti e ne mostra i dati in formato tabellare o a lista
\end{enumerate}


\textbf{Relazioni con altri casi d'uso (\textit{include} / \textit{extend})}
\begin{itemize}
    \item \textit{include}: 
    \begin{itemize}
        \item UC-0E.2 - Visualizzazione elemento lista utenti registrati.
    \end{itemize}
    \item \textit{extend}: 
    \begin{itemize}
        \item nessuna
    \end{itemize}
    
\end{itemize}

\vspace{0.5cm}

\subsubsection{UC-0E.2 – Visualizzazione elemento lista utenti registrati}\label{sec:uc-0e-2}

\textbf{Attori}
\begin{itemize}
    \item Amministratore
\end{itemize}

\textbf{Pre-condizioni}
\begin{itemize}
    \item L'amministratore ha effettuato l'accesso e si trova nella sezione di gestione utenti
    \item Esiste almeno un utente registrato nel sistema
\end{itemize}

\textbf{Post-condizioni}
\begin{itemize}
    \item Viene visualizzato un elemento dalla lista degli utenti registrati nel sistema
\end{itemize}

\textbf{Scenario principale}
\begin{enumerate}
    \item L'applicazione mostra un singolo utente con i relativi dettagli (nome, cognome, ruolo, ecc.) all'interno della lista
\end{enumerate}

\textbf{Relazioni con altri casi d'uso (\textit{include} / \textit{extend})}
\begin{itemize}
    \item \textit{include}: 
    \begin{itemize}
        \item UC-0E.4 - Visualizzazione nome utente registrato.
        \item UC-0E.5 - Visualizzazione cognome utente registrato.
        \item UC-0E.3 - Visualizzazione ruolo utente registrato.
    \end{itemize}
    \item \textit{extend}: 
    \begin{itemize}
        \item Nessuna
    \end{itemize}
    
\end{itemize}

\vspace{0.5cm}

\subsubsection{UC-0E.3 – Visualizzazione ruolo utente registrato}\label{sec:uc-0e-3}

\textbf{Attori}
\begin{itemize}
    \item Amministratore
\end{itemize}

\textbf{Pre-condizioni}
\begin{itemize}
    \item L'amministratore ha effettuato l'accesso e si trova nella sezione di gestione utenti
\end{itemize}

\textbf{Post-condizioni}
\begin{itemize}
    \item Viene visualizzato il ruolo attuale (es. Utente, Admin, Moderatore) associato a ciascun utente
\end{itemize}

\textbf{Scenario principale}
\begin{enumerate}
    \item Il sistema identifica i permessi di ogni utente e mostra l'etichetta del ruolo corrispondente
\end{enumerate}

\vspace{0.5cm}

\subsubsection{UC-0E.4 – Visualizzazione nome utente registrato}\label{sec:uc-0e-4}

\textbf{Attori}
\begin{itemize}
    \item Amministratore
\end{itemize}

\textbf{Pre-condizioni}
\begin{itemize}
    \item L'amministratore ha effettuato l'accesso e si trova nella sezione di gestione utenti
\end{itemize}

\textbf{Post-condizioni}
\begin{itemize}
    \item Viene visualizzatil nome di un utente 
\end{itemize}

\textbf{Scenario principale}
\begin{enumerate}
    \item Il sistema recupera dal database la lista degli utenti e ne mostra i nomi in formato tabellare o a lista
\end{enumerate}

\vspace{0.5cm}

\subsubsection{UC-0E.5 – Visualizzazione cognome utente registrato}\label{sec:uc-0e-5}

\textbf{Attori}
\begin{itemize}
    \item Amministratore
\end{itemize}

\textbf{Pre-condizioni}
\begin{itemize}
    \item L'amministratore ha effettuato l'accesso e si trova nella sezione di gestione utenti
\end{itemize}

\textbf{Post-condizioni}
\begin{itemize}
    \item Viene visualizzato il cognome di un utente
\end{itemize}

\textbf{Scenario principale}
\begin{enumerate}
    \item Il sistema recupera dal database la lista degli utenti e ne mostra i cognomi in corrispondenza dei rispettivi nomi
\end{enumerate}

\vspace{0.5cm}

\subsubsection{UC-0E.6 – Modifica ruolo utente registrato}\label{sec:uc-0e-6}

\textbf{Attori}
\begin{itemize}
    \item Amministratore
\end{itemize}

\textbf{Pre-condizioni}
\begin{itemize}
    \item L'amministratore ha selezionato un utente specifico dalla lista
\end{itemize}

\textbf{Post-condizioni}
\begin{itemize}
    \item L'interfaccia permette la selezione di un nuovo ruolo per l'utente scelto
\end{itemize}

\textbf{Scenario principale}
\begin{enumerate}
    \item L'amministratore interagisce con il campo relativo al ruolo (es. tramite menu a tendina) e seleziona una nuova tipologia di permessi
\end{enumerate}

\textbf{Relazioni con altri casi d'uso (\textit{include} / \textit{extend})}
\begin{itemize}
    \item \textit{include}: 
    \begin{itemize}
        \item Nessuna

    \end{itemize}
    \item \textit{extend}: 
    \begin{itemize}
        \item UC-0E.8 - Annulla modifica ruolo utente registrato
        \item UC-0E.9 - Salva modifica ruolo utente registrato.
    \end{itemize}
    
\end{itemize}

\vspace{0.5cm}

\subsubsection{UC-0E.7 – Utente non autorizzato}\label{sec:uc-0e-7}

\textbf{Attori}
\begin{itemize}
    \item Utente generico (non amministratore)
\end{itemize}

\textbf{Pre-condizioni}
\begin{itemize}
    \item Un utente senza privilegi di amministrazione tenta di accedere a funzionalità o pagine riservate alla gestione ruoli
\end{itemize}

\textbf{Post-condizioni}
\begin{itemize}
    \item L'accesso viene negato e viene mostrato un messaggio di errore o effettuato un reindirizzamento
\end{itemize}

\textbf{Scenario principale}
\begin{enumerate}
    \item L'utente tenta di navigare verso l'URL di gestione utenti o di inviare una richiesta di modifica ruolo
    \item Il sistema verifica i permessi, rileva l'assenza del ruolo di amministratore e blocca l'operazione
\end{enumerate}


\vspace{0.5cm}

\subsubsection{UC-0E.8 – Annulla modifica ruolo utente registrato}\label{sec:uc-0e-8}

\textbf{Attori}
\begin{itemize}
    \item Amministratore
\end{itemize}

\textbf{Pre-condizioni}
\begin{itemize}
    \item L'amministratore sta modificando il ruolo di un utente
\end{itemize}

\textbf{Post-condizioni}
\begin{itemize}
    \item Il ruolo dell'utente rimane invariato e l'interfaccia torna allo stato precedente
\end{itemize}

\textbf{Scenario principale}
\begin{enumerate}
    \item L'amministratore decide di non applicare le modifiche al ruolo
    \item Il sistema ripristina il valore originale visualizzato
\end{enumerate}

\vspace{0.5cm}

\subsubsection{UC-0E.9 – Salva modifica ruolo utente registrato}\label{sec:uc-0e-9}

\textbf{Attori}
\begin{itemize}
    \item Amministratore
\end{itemize}

\textbf{Pre-condizioni}
\begin{itemize}
    \item L'amministratore ha modificato il ruolo di un utente ma non ha ancora confermato
\end{itemize}

\textbf{Post-condizioni}
\begin{itemize}
    \item Il nuovo ruolo viene aggiornato nel database e l'utente acquisisce i nuovi permessi
\end{itemize}

\textbf{Scenario principale}
\begin{enumerate}
    \item L'amministratore conferma l'operazione di cambio ruolo
    \item Il sistema registra la modifica e aggiorna lo stato dell'utente
\end{enumerate}

\vspace{0.5cm}


\subsubsection{UC-0F – Logout}\label{sec:uc-0f}

\begin{figure}[H]
    \centering
    \includegraphics[width=0.7\textwidth]{Diagrammi casi d'uso/UC0F.jpg}
    \caption{Diagramma del caso d'uso UC-0F – Logout}
\end{figure}

\textbf{Attori}
\begin{itemize}
    \item Utente
\end{itemize}

\textbf{Pre-condizioni}
\begin{itemize}
    \item L'utente ha una sessione attiva nel sistema.
\end{itemize}

\textbf{Post-condizioni}
\begin{itemize}
    \item Non esiste più una sessione attiva associata all'utente sul dispositivo corrente.
    \item Per accedere nuovamente alle funzionalità riservate è necessario eseguire un nuovo login.
\end{itemize}

\textbf{Scenario principale}
\begin{enumerate}
    \item L'utente seleziona l'opzione di logout (ad esempio dal menu della dashboard).
    \item Il sistema invalida la sessione corrente associata all'utente (es. rimozione token di sessione).
    \item Il sistema reindirizza l'utente alla schermata di login o alla schermata iniziale pubblica.
\end{enumerate}

\textbf{Relazioni con altri casi d'uso (\textit{include} / \textit{extend})}
\begin{itemize}
    \item \textit{include}: 
    \begin{itemize}
        \item Nessuno.
    \end{itemize}
    \item \textit{extend}:
    \begin{itemize}
        \item Nessuno.
    \end{itemize}
\end{itemize}

\vspace{0.5cm}

\subsection{Sezione 1 – Modulo AI Assistant Generativo}


\subsubsection{UC-1A – Generazione contenuti AI}\label{sec:uc-1a}


\begin{figure}[H]
    \centering
    \includegraphics[width=1\textwidth]{Diagrammi casi d'uso/UC1A.jpg}
    \caption{Diagramma del caso d'uso UC-1A – Generazione contenuti AI}
\end{figure}

\textbf{Attori}
\begin{itemize}
    \item HR manager/Redattore
\end{itemize}

\textbf{Attori secondari}
\begin{itemize}
    \item AI Post Generator
\end{itemize}

\textbf{Pre-condizioni}
\begin{itemize}
    \item L'utente è entrato nel modulo AI Assistant Generativo dalla dashboard principale.
    \item L'utente dispone dei permessi necessari per utilizzare il modulo AI Assistant.
\end{itemize}

\textbf{Post-condizioni}
\begin{itemize}
    \item L'utente ha generato un contenuto testuale basato sul prompt e sui parametri selezionati
    \item Il contenuto generato è visualizzato nell'interfaccia utente
    \item Il contenuto generato viene salvato nel sistema
\end{itemize}

\textbf{Scenario principale}
\begin{enumerate}
    \item L'utente accede alla sezione "AI Assistant Generativo".
    \item Il sistema mostra il campo per l'inserimento del prompt e le azioni che possono essere eseguite dall'utente.
\end{enumerate}

\textbf{Scenario secondario}
\begin{enumerate}
    \item 
\end{enumerate}

\textbf{Relazioni con altri casi d'uso (\textit{include} / \textit{extend})}
\begin{itemize}
    \item \textit{include}: 
    \begin{itemize}
        \item UC-1C
    \end{itemize}
    \item \textit{extend}: 
    \begin{itemize}
        \item UC-1A.1 - Inserimento prompt
        \item UC-1A.2 - Selezione tono
        \item UC-1A.3 - Selezione stile
    \end{itemize}
\end{itemize}

\vspace{0.5cm}

\subsubsection{UC-1A.1 – Inserimento prompt}\label{sec:uc-1a-1}

\textbf{Attori}
\begin{itemize}
    \item HR manager/Redattore
\end{itemize}

\textbf{Pre-condizioni}
\begin{itemize}
    \item L'utente si trova nell'interfaccia di generazione contenuti AI (UC-1A)
    \item La casella di testo del prompt è vuota o modificabile
\end{itemize}

\textbf{Post-condizioni}
\begin{itemize}
    \item La casella di testo contiene la descrizione (prompt) inserita dall'utente
\end{itemize}

\textbf{Scenario principale}
\begin{enumerate}
    \item L'utente inserisce o incolla il testo descrittivo per la generazione del contenuto nell'apposita area di testo
\end{enumerate}

\vspace{0.5cm}

\subsubsection{UC-1A.2 – Selezione tono}\label{sec:uc-1a-2}

\textbf{Attori}
\begin{itemize}
    \item HR manager/Redattore
\end{itemize}

\textbf{Pre-condizioni}
\begin{itemize}
    \item L'utente si trova nell'interfaccia di generazione contenuti AI
\end{itemize}

\textbf{Post-condizioni}
\begin{itemize}
    \item Il parametro "Tono" risulta impostato sulla scelta effettuata (es. Professionale, Informale, Spiritoso)
\end{itemize}

\textbf{Scenario principale}
\begin{enumerate}
    \item L'utente interagisce con il selettore del tono (es. menu a tendina)
    \item L'utente sceglie l'opzione desiderata tra quelle disponibili
\end{enumerate}

\vspace{0.5cm}

\subsubsection{UC-1A.3 – Selezione stile}\label{sec:uc-1a-3}

\textbf{Attori}
\begin{itemize}
    \item HR manager/Redattore
\end{itemize}

\textbf{Pre-condizioni}
\begin{itemize}
    \item L'utente si trova nell'interfaccia di generazione contenuti AI
\end{itemize}

\textbf{Post-condizioni}
\begin{itemize}
    \item Il parametro "Stile" risulta impostato sulla scelta effettuata (es. Articolo di blog, Post social, Email)
\end{itemize}

\textbf{Scenario principale}
\begin{enumerate}
    \item L'utente interagisce con il selettore dello stile
    \item L'utente sceglie la tipologia di formato desiderata per il testo in output
\end{enumerate}

\vspace{0.5cm}


\subsubsection{UC-1B – Visualizzazione lista elementi storico}\label{sec:uc-1b}

\begin{figure}[H]
    \centering
    \includegraphics[width=1\textwidth]{Diagrammi casi d'uso/UC1B.jpg}
    \caption{Diagramma del caso d'uso UC-1B – Visualizzazione lista elementi storico}
\end{figure}

\textbf{Attori}
\begin{itemize}
    \item HR manager/Redattore
\end{itemize}

\textbf{Pre-condizioni}
\begin{itemize}
    \item L'utente si trova all'interno della sezione dedicata allo storico
\end{itemize}

\textbf{Post-condizioni}
\begin{itemize}
    \item L'utente visualizza la lista di elementi salvati nello storico nel suo insieme
\end{itemize}

\textbf{Scenario principale}
\begin{enumerate}
    \item L'utente si trova nel modulo storico dell'AI Assistant Generativo
    \item L'utente vede l'elenco di tutti gli elementi salvati con i relativi dati principali (prompt parziale, tono, stile, data)
\end{enumerate}


\textbf{Relazioni con altri casi d'uso (\textit{include} / \textit{extend})}
\begin{itemize}
    \item \textit{include}: 
    \begin{itemize}
        \item UC-1B.2 – Visualizzazione informazioni elemento
    \end{itemize}
    \item \textit{extend}: 
    \begin{itemize}
        \item UC-1B.1 – Nessun prompt salvato
    \end{itemize}
\end{itemize}

\vspace{0.5cm}

\subsubsection{UC-1B.1– Nessun prompt salvato}\label{sec:uc-1b-1}

\textbf{Attori}
\begin{itemize}
    \item HR manager/Redattore
\end{itemize}

\textbf{Pre-condizioni}
\begin{itemize}
    \item L'utente si trova all'interno della sezione dedicata allo storico
    \item Lo storico non ha neanche un elemento
\end{itemize}

\textbf{Post-condizioni}
\begin{itemize}
    \item Viene visualizzato un messaggio informativo (placeholder) che indica l'assenza di dati
\end{itemize}

\textbf{Scenario principale}
\begin{enumerate}
    \item Il sistema interroga il database per recuperare lo storico
    \item Il sistema non trova record associati all'utente e mostra un messaggio del tipo "Nessun prompt presente nello storico"
\end{enumerate}


\vspace{0.5cm}

\subsubsection{UC-1B.2 – Visualizzazione informazioni elemento}\label{sec:uc-1b-2}

\textbf{Attori}
\begin{itemize}
    \item HR manager/Redattore
\end{itemize}

\textbf{Pre-condizioni}
\begin{itemize}
    \item La lista dello storico contiene almeno un elemento
\end{itemize}

\textbf{Post-condizioni}
\begin{itemize}
    \item L'utente visualizza i dettagli completi di una specifica generazione (prompt completo, parametri usati, output ottenuto, data)
\end{itemize}

\textbf{Scenario principale}
\begin{enumerate}
    \item L'utente seleziona un elemento dalla lista dello storico
    \item Il sistema espande l'elemento o apre una modale mostrando tutti i metadati associati a quella generazione
\end{enumerate}


\textbf{Relazioni con altri casi d'uso (\textit{include} / \textit{extend})}
\begin{itemize}
    \item \textit{include}: 
    \begin{itemize}
        \item UC-1B.7– Visualizzazione prompt
        \item UC-1B.8 – Visualizzazione tono
        \item UC-1B.3 – Visualizzazione stile
        \item UC-1B.4 – Visualizzazione risultato
        \item UC-1B.5 – Visualizzazione timestamp
        \item UC-1B.6 – Visualizzazione valutazione
    \end{itemize}
\end{itemize}

\vspace{0.5cm}

\subsubsection{UC-1B.3 – Visualizzazione stile}\label{sec:uc-1b-3}

\textbf{Attori}
\begin{itemize}
    \item HR manager/Redattore
\end{itemize}

\textbf{Pre-condizioni}
\begin{itemize}
    \item L'utente ha aperto la visualizzazione di dettaglio di un elemento dello storico
\end{itemize}

\textbf{Post-condizioni}
\begin{itemize}
    \item Viene visualizzato il parametro "Stile" utilizzato per la generazione
\end{itemize}

\textbf{Scenario principale}
\begin{enumerate}
    \item Il sistema recupera e visualizza la tipologia di contenuto o stile (es. Articolo, Post Social) salvato con la generazione
\end{enumerate}


\vspace{0.5cm}

\subsubsection{UC-1B.4 – Visualizzazione risultato}\label{sec:uc-1b-4}

\textbf{Attori}
\begin{itemize}
    \item HR manager/Redattore
\end{itemize}

\textbf{Pre-condizioni}
\begin{itemize}
    \item L'utente ha aperto la visualizzazione di dettaglio di un elemento dello storico
\end{itemize}

\textbf{Post-condizioni}
\begin{itemize}
    \item Viene visualizzato il testo generato dall'intelligenza artificiale
\end{itemize}

\textbf{Scenario principale}
\begin{enumerate}
    \item Il sistema mostra il contenuto testuale prodotto dall'AI in risposta al prompt, mantenendo la formattazione originale se presente
\end{enumerate}


\vspace{0.5cm}

\subsubsection{UC-1B.5 – Visualizzazione timestamp}\label{sec:uc-1b-5}

\textbf{Attori}
\begin{itemize}
    \item HR manager/Redattore
\end{itemize}

\textbf{Pre-condizioni}
\begin{itemize}
    \item L'utente ha aperto la visualizzazione di dettaglio di un elemento dello storico
\end{itemize}

\textbf{Post-condizioni}
\begin{itemize}
    \item Viene visualizzata la data e l'ora in cui è stata effettuata la generazione
\end{itemize}

\textbf{Scenario principale}
\begin{enumerate}
    \item Il sistema mostra il timestamp (data e ora) di creazione del contenuto
\end{enumerate}


\vspace{0.5cm}

\subsubsection{UC-1B.6 – Visualizzazione valutazione}\label{sec:uc-1b-6}

\textbf{Attori}
\begin{itemize}
    \item HR manager/Redattore
\end{itemize}

\textbf{Pre-condizioni}
\begin{itemize}
    \item L'utente ha aperto la visualizzazione di dettaglio di un elemento dello storico
\end{itemize}

\textbf{Post-condizioni}
\begin{itemize}
    \item Viene visualizzato il feedback o il voto assegnato dall'utente (se presente)
\end{itemize}

\textbf{Scenario principale}
\begin{enumerate}
    \item Il sistema verifica se esiste una valutazione associata al contenuto
    \item Se presente, mostra l'indicatore grafico (es. stelle, thumbs up/down); in caso contrario mostra un campo vuoto o l'opzione per aggiungere una valutazione
\end{enumerate}


\vspace{0.5cm}

\subsubsection{UC-1B.7 – Visualizzazione prompt}\label{sec:uc-1b-7}

\textbf{Attori}
\begin{itemize}
    \item HR manager/Redattore
\end{itemize}

\textbf{Pre-condizioni}
\begin{itemize}
    \item L'utente ha aperto la visualizzazione di dettaglio di un elemento dello storico
\end{itemize}

\textbf{Post-condizioni}
\begin{itemize}
    \item Viene visualizzato il testo completo del prompt inserito originariamente dall'utente
\end{itemize}

\textbf{Scenario principale}
\begin{enumerate}
    \item Il sistema recupera dal database il testo della richiesta (prompt) associato alla generazione selezionata e lo mostra a video
\end{enumerate}


\vspace{0.5cm}

\subsubsection{UC-1B.8 – Visualizzazione tono}\label{sec:uc-1b-8}

\textbf{Attori}
\begin{itemize}
    \item HR manager/Redattore
\end{itemize}

\textbf{Pre-condizioni}
\begin{itemize}
    \item L'utente ha aperto la visualizzazione di dettaglio di un elemento dello storico
\end{itemize}

\textbf{Post-condizioni}
\begin{itemize}
    \item Viene visualizzato il parametro "Tono" utilizzato per la generazione
\end{itemize}

\textbf{Scenario principale}
\begin{enumerate}
    \item Il sistema recupera e visualizza l'etichetta del tono (es. Professionale, Amichevole) salvato con la generazione
\end{enumerate}

\vspace{0.5cm}

\subsubsection{UC-1C – Visualizzazione anteprima contenuto generato}\label{sec:uc-1c}

\begin{figure}[H]
    \centering
    \includegraphics[width=0.7\textwidth]{Diagrammi casi d'uso/UC1C.jpg}
    \caption{Diagramma del caso d'uso UC-1C – Visualizzazione anteprima contenuto generato}
\end{figure}

\textbf{Attori}
\begin{itemize}
    \item HR manager/Redattore
\end{itemize}

\textbf{Pre-condizioni}
\begin{itemize}
    \item Il sistema AI ha completato l'elaborazione della richiesta
\end{itemize}

\textbf{Post-condizioni}
\begin{itemize}
    \item Il testo generato è visibile a schermo e formattato correttamente
\end{itemize}

\textbf{Scenario principale}
\begin{enumerate}
    \item Il sistema riceve l'output dall'AI e lo renderizza nell'area di visualizzazione principale, permettendo all'utente di leggerlo
\end{enumerate}


\vspace{0.5cm}


\textbf{Relazioni con altri casi d'uso (\textit{include} / \textit{extend})}
\begin{itemize}
    \item \textit{include}: 
    \begin{itemize}
        \item Nessuna
    \end{itemize}
    \item \textit{extend}: 
    \begin{itemize}
        \item Nessuna
    \end{itemize}
\end{itemize}

\vspace{0.5cm}

\subsubsection{UC-1D – Modifica contenuto generato}\label{sec:uc-1d}

\begin{figure}[H]
    \centering
    \includegraphics[width=1\textwidth]{Diagrammi casi d'uso/UC1D.jpg}
    \caption{Diagramma del caso d'uso UC-1D – Modifica contenuto generato}
\end{figure}

\textbf{Attori}
\begin{itemize}
    \item HR manager/Redattore
\end{itemize}

\textbf{Pre-condizioni}
\begin{itemize}
    \item L'utente ha generato un contenuto tramite UC-1A e lo sta visualizzando in anteprima tramite UC-1C.
\end{itemize}

\textbf{Post-condizioni}
\begin{itemize}
    \item L'utente si trova nella pagina di modifica e può eseguire le opzioni mostrate.
\end{itemize}


\textbf{Scenario principale}
\begin{enumerate}
    \item L'utente è nella pagina del contenuto generato.
    \item L'utente seleziona l'opzione di modifica del contenuto generato.
    \item Vengono effettuate delle modifiche e successivamente salvate
\end{enumerate}

\textbf{Scenario secondario}
\begin{enumerate}
    \item L'utente dopo aver effettuato delle modifiche non le salva
\end{enumerate}



\textbf{Relazioni con altri casi d'uso (\textit{include} / \textit{extend})}
\begin{itemize}
    \item \textit{include}: 
    \begin{itemize}
        \item UC-1C.1 – Visualizzazione anteprima contenuto generato
    \end{itemize}
    \item \textit{extend}: 
    \begin{itemize}
        \item UC-1D.1 – Modifica immagine
        \item UC-1D.2 – Modifica titolo
        \item UC-1D.3 – Modifica testo
        \item UC-1D.4 – Salva modifiche
        \item UC-1D.5 – Annulla modifiche
        \item UC-1D.6 - File immagine non valido
    \end{itemize}
\end{itemize}

\vspace{0.5cm}

\subsubsection{UC-1D.1 – Modifica immagine}\label{sec:uc-1d-1}

\textbf{Attori}
\begin{itemize}
    \item HR manager/Redattore
\end{itemize}

\textbf{Pre-condizioni}
\begin{itemize}
    \item L'utente si trova nella pagina di modifica del contenuto (UC-1D)
\end{itemize}

\textbf{Post-condizioni}
\begin{itemize}
    \item L'immagine associata al contenuto viene aggiornata con quella caricata dall'utente
\end{itemize}

\textbf{Scenario principale}
\begin{enumerate}
    \item L'utente clicca sull'opzione per cambiare l'immagine (upload o selezione da libreria)
    \item L'utente seleziona un file valido dal proprio dispositivo
    \item Il sistema sostituisce l'immagine generata dall'AI con quella fornita dall'utente
\end{enumerate}

\textbf{Scenario principale}
\begin{enumerate}
    \item Il file immagine caricato dall'utente non è valido
\end{enumerate}

\textbf{Relazioni con altri casi d'uso (\textit{include} / \textit{extend})}
\begin{itemize}
    \item \textit{include}: 
    \begin{itemize}
        \item Nessuna
    \end{itemize}
    \item \textit{extend}: 
    \begin{itemize}
        \item UC-1D.6 - File immagine non valido
    \end{itemize}
\end{itemize}

\vspace{0.5cm}



\vspace{0.5cm}

\subsubsection{UC-1D.2 – Modifica titolo}\label{sec:uc-1d-2}

\textbf{Attori}
\begin{itemize}
    \item HR manager/Redattore
\end{itemize}

\textbf{Pre-condizioni}
\begin{itemize}
    \item L'utente si trova nella pagina di modifica del contenuto
\end{itemize}

\textbf{Post-condizioni}
\begin{itemize}
    \item Il titolo del contenuto risulta modificato secondo l'input dell'utente
\end{itemize}

\textbf{Scenario principale}
\begin{enumerate}
    \item L'utente modifica il testo presente nella casella di input del titolo
\end{enumerate}


\vspace{0.5cm}

\subsubsection{UC-1D.3 – Modifica testo}\label{sec:uc-1d-3}

\textbf{Attori}
\begin{itemize}
    \item HR manager/Redattore
\end{itemize}

\textbf{Pre-condizioni}
\begin{itemize}
    \item L'utente si trova nella pagina di modifica del contenuto
\end{itemize}

\textbf{Post-condizioni}
\begin{itemize}
    \item Il corpo del testo risulta modificato secondo l'input dell'utente
\end{itemize}

\textbf{Scenario principale}
\begin{enumerate}
    \item L'utente agisce sull'editor di testo (corpo del contenuto) aggiungendo, rimuovendo o formattando il testo generato
\end{enumerate}


\vspace{0.5cm}

\subsubsection{UC-1D.4 – Salva modifiche}\label{sec:uc-1d-4}

\textbf{Attori}
\begin{itemize}
    \item HR manager/Redattore
\end{itemize}

\textbf{Pre-condizioni}
\begin{itemize}
    \item L'utente ha apportato delle modifiche ai campi del contenuto
\end{itemize}

\textbf{Post-condizioni}
\begin{itemize}
    \item Le modifiche vengono salvate nel database e il contenuto aggiornato è pronto per l'uso
\end{itemize}

\textbf{Scenario principale}
\begin{enumerate}
    \item L'utente conferma le operazioni di modifica premendo il pulsante di salvataggio
    \item Il sistema valida i dati e sovrascrive la versione precedente del contenuto
\end{enumerate}


\vspace{0.5cm}

\subsubsection{UC-1D.5 – Annulla modifiche}\label{sec:uc-1d-5}

\textbf{Attori}
\begin{itemize}
    \item HR manager/Redattore
\end{itemize}

\textbf{Pre-condizioni}
\begin{itemize}
    \item L'utente si trova nella pagina di modifica
\end{itemize}

\textbf{Post-condizioni}
\begin{itemize}
    \item Le modifiche non salvate vengono scartate e l'utente ritorna alla visualizzazione precedente (anteprima originale)
\end{itemize}

\textbf{Scenario principale}
\begin{enumerate}
    \item L'utente decide di non applicare le modifiche correnti
    \item Il sistema ripristina lo stato del contenuto a quello precedente l'apertura dell'editor
\end{enumerate}


\vspace{0.5cm}

\subsubsection{UC-1D.6 – File immagine non valido}\label{sec:uc-1d-6}

\textbf{Attori}
\begin{itemize}
    \item HR manager/Redattore
\end{itemize}

\textbf{Pre-condizioni}
\begin{itemize}
    \item Un file non idoneo è stato caricato
\end{itemize}

\textbf{Post-condizioni}
\begin{itemize}
    \item Viene mostrato un messaggio di errore e l'immagine precedente rimane invariata
\end{itemize}

\textbf{Scenario principale}
\begin{enumerate}
    \item L'utente seleziona un file che non rispetta i requisiti del sistema (formato non supportato, dimensione eccessiva, file corrotto)
    \item Il sistema rileva l'anomalia durante il caricamento e blocca l'operazione
\end{enumerate}


\vspace{0.5cm}

\subsubsection{UC-1E – Riutilizza}\label{sec:uc-1e}

\begin{figure}[H]
    \centering
    \includegraphics[width=0.7\textwidth]{Diagrammi casi d'uso/UC1E.jpg}
    \caption{Diagramma del caso d'uso UC-1E – Riutilizza}
\end{figure}


\textbf{Attori}
\begin{itemize}
    \item HR manager/Redattore
\end{itemize}

\textbf{Attori secondari}
\begin{itemize}
    \item AI Post Generator
\end{itemize}

\textbf{Pre-condizioni}
\begin{itemize}
    \item L'utente si trova all'interno della sezione dedicata allo storico
    \item Lo storico ha almeno un elemento
\end{itemize}

\textbf{Post-condizioni}
\begin{itemize}
    \item Il sistema avvia una nuova generazione utilizzando esattamente gli stessi parametri (prompt, tono, stile) dell'elemento selezionato
\end{itemize}

\textbf{Scenario principale}
\begin{enumerate}
    \item L'utente richiede di riutilizzare un elemento dello storico
    \item Il sistema invia direttamente i dati al motore AI per generare un nuovo output senza richiedere modifiche all'utente
\end{enumerate}


\vspace{0.5cm}

\subsubsection{UC-1F – Duplica}\label{sec:uc-1f}

\begin{figure}[H]
    \centering
    \includegraphics[width=0.7\textwidth]{Diagrammi casi d'uso/UC1F.jpg}
    \caption{Diagramma del caso d'uso UC-1F – Duplica}
\end{figure}

\textbf{Attori}
\begin{itemize}
    \item HR manager/Redattore
\end{itemize}

\textbf{Pre-condizioni}
\begin{itemize}
    \item L'utente si trova all'interno della sezione dedicata allo storico
    \item Lo storico ha almeno un elemento
\end{itemize}

\textbf{Post-condizioni}
\begin{itemize}
    \item L'utente viene riportato al modulo di generazione (UC-1A) con i campi prompt, tono e stile già compilati con i dati storici, pronti per essere modificati
\end{itemize}

\textbf{Scenario principale}
\begin{enumerate}
    \item L'utente clicca sul comando "Duplica" associato ad un elemento dello storico
    \item Il sistema reindirizza l'utente alla pagina di creazione, pre-popolando i campi con i valori recuperati dallo storico
    \item L'utente può ora modificare il prompt o i parametri prima di generare
\end{enumerate}


\vspace{0.5cm}

\subsubsection{UC-1G – Ricerca generazione}\label{sec:uc-1g}

\begin{figure}[H]
    \centering
    \includegraphics[width=1\textwidth]{Diagrammi casi d'uso/UC1G.jpg}
    \caption{Diagramma del caso d'uso UC-1G – Ricerca generazione}
\end{figure}

\textbf{Attori}
\begin{itemize}
    \item HR manager/Redattore
\end{itemize}

\textbf{Pre-condizioni}
\begin{itemize}
    \item L'utente si trova all'interno della sezione dedicata allo storico
    \item Lo storico ha almeno un elemento
\end{itemize}

\textbf{Post-condizioni}
\begin{itemize}
    \item La lista visualizzata viene filtrata mostrando solo gli elementi che corrispondono ai criteri di ricerca
\end{itemize}

\textbf{Scenario principale}
\begin{enumerate}
    \item L'utente inserisce una parola chiave (es. parte del prompt o data) nella barra di ricerca
    \item Il sistema filtra gli elementi dello storico in tempo reale o alla conferma
\end{enumerate}


\textbf{Relazioni con altri casi d'uso (\textit{include} / \textit{extend})}
\begin{itemize}
    \item \textit{include}: 
    \begin{itemize}
        \item UC-1G.1 – Visualizzazione lista elementi storico filtrata
    \end{itemize}
\end{itemize}

\vspace{0.5cm}

\subsubsection{UC-1G.1 – Visualizzazione lista elementi storico filtrata}\label{sec:uc-1g-1}

\textbf{Attori}
\begin{itemize}
    \item HR manager/Redattore
\end{itemize}

\textbf{Pre-condizioni}
\begin{itemize}
    \item La lista dello storico contiene almeno un elemento 
    \item L'utente ha inserito un criterio di ricerca valido
\end{itemize}

\textbf{Post-condizioni}
\begin{itemize}
    \item L'utente visualizza la lista di elementi (coerenti con la ricerca) salvati nello storico nel suo insieme
\end{itemize}

\textbf{Scenario principale}
\begin{enumerate}
    \item L'utente si trova nel modulo storico dell'AI Assistant Generativo
    \item L'utente vede l'elenco di tutti gli elementi salvati con i relativi dati principali (prompt parziale, tono, stile, data)
\end{enumerate}


\textbf{Relazioni con altri casi d'uso (\textit{include} / \textit{extend})}
\begin{itemize}
    \item \textit{include}: 
    \begin{itemize}
        \item UC-1B.2 – Visualizzazione informazioni elemento
    \end{itemize}
\end{itemize}

\vspace{0.5cm}

\subsubsection{UC-1H – Rigenera contenuto tramite AI}\label{sec:uc-1h}

\begin{figure}[H]
    \centering
    \includegraphics[width=0.7\textwidth]{Diagrammi casi d'uso/UC1H.jpg}
    \caption{Diagramma del caso d'uso UC-1H – Rigenera contenuto tramite AI}
\end{figure}

\textbf{Attori}
\begin{itemize}
    \item HR manager/Redattore
\end{itemize}

\textbf{Attori secondari}
\begin{itemize}
    \item AI Post Generator
\end{itemize}

\textbf{Pre-condizioni}
\begin{itemize}
    \item L'utente si trova nel modulo AI assistant generativo
    \item È stato generato un contenuto tramite l'AI assistant
\end{itemize}

\textbf{Post-condizioni}
\begin{itemize}
    \item Il contenuto precedente viene sostituito da una nuova versione generata con gli stessi parametri (o parametri lievemente aggiustati)
\end{itemize}

\textbf{Scenario principale}
\begin{enumerate}
    \item L'utente richiede al sistema di provare nuovamente a generare il contenuto
    \item Il sistema invia una nuova richiesta all'AI
    \item Il nuovo risultato sovrascrive quello visualizzato in anteprima
\end{enumerate}


\vspace{0.5cm}

\subsubsection{UC-1I – Valuta contenuto generato}\label{sec:uc-1i}

\begin{figure}[H]
    \centering
    \includegraphics[width=0.7\textwidth]{Diagrammi casi d'uso/UC1I.jpg}
    \caption{Diagramma del caso d'uso UC-1I – Valuta contenuto generato}
\end{figure}

\textbf{Attori}
\begin{itemize}
    \item HR manager/Redattore
\end{itemize}

\textbf{Pre-condizioni}
\begin{itemize}
    \item L'utente si trova nel modulo AI assistant generativo
    \item È stato generato un contenuto tramite l'AI assistant
\end{itemize}

\textbf{Post-condizioni}
\begin{itemize}
    \item Il feedback dell'utente viene registrato e associato al contenuto nello storico
\end{itemize}

\textbf{Scenario principale}
\begin{enumerate}
    \item L'utente esprime un giudizio sulla qualità del contenuto (es. pollice in su/giù o stelle)
    \item Il sistema salva la valutazione
\end{enumerate}


\vspace{0.5cm}

\subsubsection{UC-1J – Scarta contenuto generato}\label{sec:uc-1j}

\begin{figure}[H]
    \centering
    \includegraphics[width=0.7\textwidth]{Diagrammi casi d'uso/UC1J.jpg}
    \caption{Diagramma del caso d'uso UC-1J – Scarta contenuto generato}
\end{figure}

\textbf{Attori}
\begin{itemize}
    \item HR manager/Redattore
\end{itemize}

\textbf{Pre-condizioni}
\begin{itemize}
    \item L'utente si trova nel modulo AI assistant generativo
    \item È stato generato un contenuto tramite l'AI assistant
\end{itemize}

\textbf{Post-condizioni}
\begin{itemize}
    \item Il contenuto attuale viene rimosso dalla vista e l'utente ritorna alla fase di inserimento Prompt
    \item L'area di anteprima viene pulita
    \item La generazione viene salvata nello storico con valutazione minima
\end{itemize}

\textbf{Scenario principale}
\begin{enumerate}
    \item L'utente decide che il contenuto non è utile e vuole ricominciare da zero
    \item Il sistema pulisce l'area di anteprima e riporta il focus sui campi di input
\end{enumerate}


\vspace{0.5cm}

\subsection{Sezione 2 – Modulo AI Co-Pilot per i Consulenti del Lavoro (CdL)}

\subsubsection{UC-2A – Analisi e gestione documentale}\label{sec:uc-2a}

\begin{figure}[H]
    \centering
    \includegraphics[width=0.7\textwidth]{Diagrammi casi d'uso/UC2A.jpg}
    \caption{Diagramma del caso d'uso UC-2A – Analisi e gestione documentale}
\end{figure}

\textbf{Attori}
\begin{itemize}
    \item Operatore CdL
\end{itemize}

\textbf{Attori secondari}
\begin{itemize}
    \item AI Analyst
\end{itemize}


\textbf{Pre-condizioni}
\begin{itemize}
    \item L'utente è entrato nel modulo AI Co-Pilot per i CdL dalla dashboard principale.
    \item L'utente dispone dei permessi necessari per utilizzare il modulo AI Co-Pilot per i CdL.
\end{itemize}

\textbf{Post-condizioni}
\begin{itemize}
    \item Il documento è stato analizzato tramite l'AI e i dati sono stati inviati al sistema
    \item L'utente visualizza l'interfaccia di upload documenti, pronta per il caricamento e la gestione dei file.
\end{itemize}

\textbf{Scenario principale}
\begin{enumerate}
    \item L'utente dalla dashboard principale accede al Modulo AI Co-Pilot per i Consulenti del Lavoro.
    \item L'utente procede a caricare il file che si desidera far analizzare e se dispone dei dati richiesti puó scegliere di compilare i campi (categoria, mese/anno di competenza, azienda, reparto)
    \item Il file viene analizzato
\end{enumerate}

\textbf{Scenario secondario}
\begin{enumerate}
    \item 
\end{enumerate}



\textbf{Relazioni con altri casi d'uso (\textit{include} / \textit{extend})}
\begin{itemize}
    \item \textit{include}: 
    \begin{itemize}
        \item UC-2A.1 - Caricamento file 
    \end{itemize}
    \item \textit{extend}: 
    \begin{itemize}
        \item UC-2A.2 - Inserimento categoria
        \item UC-2A.3 - Inserimento mese/anno di competenza
        \item UC-2A.4 - Inserimento azienda
        \item UC-2A.5 - Inserimento reparto
    \end{itemize}
\end{itemize}

\vspace{0.5cm}

\subsubsection{UC-2A.1 – Caricamento file}\label{sec:uc-2a-1}

\textbf{Attori}
\begin{itemize}
    \item Operatore CdL
\end{itemize}

\textbf{Pre-condizioni}
\begin{itemize}
    \item L'utente si trova nel modulo AI Co-Pilot
    \item L'utente dispone di file idonei al caricamento
\end{itemize}

\textbf{Post-condizioni}
\begin{itemize}
    \item Il file è stato selezionato e pre-caricato nel browser, pronto per l'associazione dei metadati (oppure viene mostrato un errore)
\end{itemize}

\textbf{Scenario principale}
\begin{enumerate}
    \item L'utente clicca sull'area di upload o trascina un file dal proprio computer
    \item Il sistema verifica che il file rispetti i requisiti di formato (es. PDF, Excel) e dimensione massima
    \item Se la verifica è positiva, il file viene mostrato in lista d'attesa
\end{enumerate}

\textbf{Eccezioni}
\begin{itemize}
    \item \textbf{Formato non valido:} L'utente tenta di caricare un'estensione non supportata. Il sistema blocca il caricamento e mostra un messaggio di errore.
    \item \textbf{Dimensione eccessiva:} Il file supera il limite di megabyte consentiti. Il sistema rifiuta il file e notifica l'utente.
\end{itemize}


\vspace{0.5cm}

\subsubsection{UC-2A.2 – Inserimento categoria}\label{sec:uc-2a-2}

\textbf{Attori}
\begin{itemize}
    \item Operatore CdL
\end{itemize}

\textbf{Pre-condizioni}
\begin{itemize}
    \item L'utente si trova nel modulo AI Co-Pilot
\end{itemize}

\textbf{Post-condizioni}
\begin{itemize}
    \item Al file viene associata una categoria documentale (es. Cedolino, F24, Contratto)
\end{itemize}

\textbf{Scenario principale}
\begin{enumerate}
    \item L'utente interagisce con il menu a tendina "Categoria"
    \item L'utente seleziona la tipologia corretta per il documento in lavorazione
\end{enumerate}


\vspace{0.5cm}

\subsubsection{UC-2A.3 – Inserimento mese/anno di competenza}\label{sec:uc-2a-3}

\textbf{Attori}
\begin{itemize}
    \item Operatore CdL
\end{itemize}

\textbf{Pre-condizioni}
\begin{itemize}
    \item L'utente si trova nel modulo AI Co-Pilot
\end{itemize}

\textbf{Post-condizioni}
\begin{itemize}
    \item Al file viene associato il periodo temporale di riferimento
\end{itemize}

\textbf{Scenario principale}
\begin{enumerate}
    \item L'utente inserisce o seleziona tramite date-picker il mese e l'anno a cui il documento fa riferimento
\end{enumerate}


\vspace{0.5cm}

\subsubsection{UC-2A.4 – Inserimento azienda}\label{sec:uc-2a-4}

\textbf{Attori}
\begin{itemize}
    \item Operatore CdL
\end{itemize}

\textbf{Pre-condizioni}
\begin{itemize}
    \item L'utente si trova nel modulo AI Co-Pilot
\end{itemize}

\textbf{Post-condizioni}
\begin{itemize}
    \item Al file viene associata l'azienda cliente proprietaria del documento
\end{itemize}

\textbf{Scenario principale}
\begin{enumerate}
    \item L'utente inizia a digitare il nome dell'azienda o la seleziona da un elenco predefinito
    \item Il sistema collega il documento all'anagrafica aziendale selezionata
\end{enumerate}


\vspace{0.5cm}

\subsubsection{UC-2A.5 – Inserimento reparto}\label{sec:uc-2a-5}

\textbf{Attori}
\begin{itemize}
    \item Operatore CdL
\end{itemize}

\textbf{Pre-condizioni}
\begin{itemize}
    \item L'utente si trova nel modulo AI Co-Pilot
\end{itemize}

\textbf{Post-condizioni}
\begin{itemize}
    \item Al file viene associato uno specifico reparto o dipartimento (opzionale)
\end{itemize}

\textbf{Scenario principale}
\begin{enumerate}
    \item L'utente specifica il reparto di riferimento (es. Amministrazione, Produzione) se pertinente per il documento
\end{enumerate}


\vspace{0.5cm}

\subsubsection{UC-2B – Visualizzazione lista documenti analisi}\label{sec:uc-2b}

\begin{figure}[H]
    \centering
    \includegraphics[width=0.7\textwidth]{Diagrammi casi d'uso/UC2B.jpg}
    \caption{Diagramma del caso d'uso UC-2B – Visualizzazione lista documenti analisi}
\end{figure}

\textbf{Attori}
\begin{itemize}
    \item Operatore CdL
\end{itemize}

\textbf{Pre-condizioni}
\begin{itemize}
    \item La lista dei contiene almeno un elemento
\end{itemize}

\textbf{Post-condizioni}
\begin{itemize}
    \item L'utente visualizza la lista di elementi salvati nel suo insieme
\end{itemize}

\textbf{Scenario principale}
\begin{enumerate}
    \item L'utente si trova nel modulo storico dell'AI Co-pilot
    \item L'utente vede l'elenco di tutti gli elementi recuperati dall'analisi
\end{enumerate}


\textbf{Relazioni con altri casi d'uso (\textit{include} / \textit{extend})}
\begin{itemize}
    \item \textit{include}: 
    \begin{itemize}
        \item UC-2B.2 – Visualizzazione elemento da lista documenti
    \end{itemize}
    \item \textit{extend}: 
    \begin{itemize}
        \item UC-2B.1 – Nessun documento riconosciuto
    \end{itemize}
\end{itemize}

\vspace{0.5cm}

\subsubsection{UC-2B.1 – Nessun documento riconosciuto}\label{sec:uc-2b-1}

\textbf{Attori}
\begin{itemize}
    \item Operatore CdL
\end{itemize}

\textbf{Pre-condizioni}
\begin{itemize}
    \item L'utente accede alla lista o effettua una ricerca/filtro
\end{itemize}

\textbf{Post-condizioni}
\begin{itemize}
    \item Viene mostrato un messaggio che informa l'utente dell'assenza di documenti
\end{itemize}

\textbf{Scenario principale}
\begin{enumerate}
    \item L'utente ha eseguito un'analisi di un file
    \item L'analisi non ha riconosciuto alcun documento
\end{enumerate}


\vspace{0.5cm}

\subsubsection{UC-2B.2 – Visualizzazione elemento da lista documenti}\label{sec:uc-2b-2}
\textbf{Attori}
\begin{itemize}
    \item Operatore CdL
\end{itemize}

\textbf{Pre-condizioni}
\begin{itemize}
    \item La lista dello storico contiene almeno un elemento
\end{itemize}

\textbf{Post-condizioni}
\begin{itemize}
    \item L'utente visualizza i dettagli di una tupla
\end{itemize}

\textbf{Scenario principale}
\begin{enumerate}
    \item L'utente visualizza un elemento dalla lista dello storico
\end{enumerate}


\textbf{Relazioni con altri casi d'uso (\textit{include} / \textit{extend})}
\begin{itemize}
    \item \textit{include}: 
    \begin{itemize}
        \item UC-2B.3 – Visualizzazione competenza documento
        \item UC-2B.4 – Visualizzazione azienda documento
        \item UC-2B.5 – Visualizzazione causale
        \item UC-2B.6 – Visualizzazione lingua
        \item UC-2B.7 – Visualizzazione numero pagine documento
        \item UC-2B.8 – Visualizzazione nome documento originale
        \item UC-2B.9 – Visualizzazione data redazione documento
        \item UC-2B.10 – Visualizzazione codice documento
        \item UC-2B.11 – Visualizzazione tipologia documento
    \end{itemize}
\end{itemize}

\vspace{0.5cm}

\subsubsection{UC-2B.3 – Visualizzazione competenza documento}\label{sec:uc-2b-3}

\textbf{Attori}
\begin{itemize}
    \item Operatore CdL
\end{itemize}

\textbf{Pre-condizioni}
\begin{itemize}
    \item L'utente visualizza la lista dei documenti
\end{itemize}

\textbf{Post-condizioni}
\begin{itemize}
    \item Viene visualizzato il periodo di competenza (Mese/Anno)
\end{itemize}

\textbf{Scenario principale}
\begin{enumerate}
    \item Il sistema mostra il mese e l'anno di riferimento fiscale/contabile del documento
\end{enumerate}


\vspace{0.5cm}

\subsubsection{UC-2B.4 – Visualizzazione azienda documento}\label{sec:uc-2b-4}

\textbf{Attori}
\begin{itemize}
    \item Operatore CdL
\end{itemize}

\textbf{Pre-condizioni}
\begin{itemize}
    \item L'utente visualizza la lista dei documenti
\end{itemize}

\textbf{Post-condizioni}
\begin{itemize}
    \item Viene visualizzata la ragione sociale dell'azienda a cui il documento appartiene
\end{itemize}

\textbf{Scenario principale}
\begin{enumerate}
    \item Il sistema recupera il nome dell'azienda collegata e lo mostra in elenco
\end{enumerate}


\vspace{0.5cm}

\subsubsection{UC-2B.5 – Visualizzazione causale}\label{sec:uc-2b-5}

\textbf{Attori}
\begin{itemize}
    \item Operatore CdL
\end{itemize}

\textbf{Pre-condizioni}
\begin{itemize}
    \item L'utente visualizza la lista dei documenti
\end{itemize}

\textbf{Post-condizioni}
\begin{itemize}
    \item Viene visualizzata la descrizione o causale specifica del documento
\end{itemize}

\textbf{Scenario principale}
\begin{enumerate}
    \item Il sistema mostra eventuali note o causali inserite in fase di upload o rilevate automaticamente
\end{enumerate}


\vspace{0.5cm}

\subsubsection{UC-2B.6 – Visualizzazione lingua}\label{sec:uc-2b-6}

\textbf{Attori}
\begin{itemize}
    \item Operatore CdL
\end{itemize}

\textbf{Pre-condizioni}
\begin{itemize}
    \item L'utente visualizza la lista dei documenti
\end{itemize}

\textbf{Post-condizioni}
\begin{itemize}
    \item Viene indicata la lingua principale del contenuto del documento
\end{itemize}

\textbf{Scenario principale}
\begin{enumerate}
    \item Il sistema mostra un indicatore (es. bandiera o codice ISO) della lingua rilevata
\end{enumerate}

\vspace{0.5cm}

\subsubsection{UC-2B.7 – Visualizzazione numero pagine documento}\label{sec:uc-2b-7}

\textbf{Attori}
\begin{itemize}
    \item Operatore CdL
\end{itemize}

\textbf{Pre-condizioni}
\begin{itemize}
    \item L'utente visualizza la lista dei documenti
\end{itemize}

\textbf{Post-condizioni}
\begin{itemize}
    \item Viene mostrato il conteggio totale delle pagine che compongono il file
\end{itemize}

\textbf{Scenario principale}
\begin{enumerate}
    \item Il sistema calcola e visualizza il numero di pagine del documento PDF o immagine
\end{enumerate}


\vspace{0.5cm}

\subsubsection{UC-2B.8 – Visualizzazione nome documento originale}\label{sec:uc-2b-8}

\textbf{Attori}
\begin{itemize}
    \item Operatore CdL
\end{itemize}

\textbf{Pre-condizioni}
\begin{itemize}
    \item L'utente visualizza la lista dei documenti
\end{itemize}

\textbf{Post-condizioni}
\begin{itemize}
    \item Viene visualizzato il nome del file originale caricato dall'utente
\end{itemize}

\textbf{Scenario principale}
\begin{enumerate}
    \item Il sistema mostra il filename originale per facilitare il riconoscimento
\end{enumerate}


\vspace{0.5cm}

\subsubsection{UC-2B.9 – Visualizzazione data redazione documento}\label{sec:uc-2b-9}

\textbf{Attori}
\begin{itemize}
    \item Operatore CdL
\end{itemize}

\textbf{Pre-condizioni}
\begin{itemize}
    \item L'utente visualizza la lista dei documenti
\end{itemize}

\textbf{Post-condizioni}
\begin{itemize}
    \item Viene visualizzata la data in cui il documento è stato creato o caricato a sistema
\end{itemize}

\textbf{Scenario principale}
\begin{enumerate}
    \item Il sistema recupera il timestamp di creazione o upload e lo mostra formattato
\end{enumerate}


\vspace{0.5cm}

\subsubsection{UC-2B.10 – Visualizzazione codice documento}\label{sec:uc-2b-10}

\textbf{Attori}
\begin{itemize}
    \item Operatore CdL
\end{itemize}

\textbf{Pre-condizioni}
\begin{itemize}
    \item L'utente visualizza la lista dei documenti (incluso in UC-2B)
\end{itemize}

\textbf{Post-condizioni}
\begin{itemize}
    \item Viene visualizzato l'identificativo univoco (ID) assegnato dal sistema al documento
\end{itemize}

\textbf{Scenario principale}
\begin{enumerate}
    \item Il sistema recupera il codice identificativo dal database e lo mostra nella riga corrispondente al documento
\end{enumerate}


\vspace{0.5cm}

\subsubsection{UC-2B.11 – Visualizzazione tipologia documento}\label{sec:uc-2b-11}

\textbf{Attori}
\begin{itemize}
    \item Operatore CdL
\end{itemize}

\textbf{Pre-condizioni}
\begin{itemize}
    \item L'utente visualizza la lista dei documenti
\end{itemize}

\textbf{Post-condizioni}
\begin{itemize}
    \item Viene visualizzata la categoria del documento (es. Cedolino, LUL, F24)
\end{itemize}

\textbf{Scenario principale}
\begin{enumerate}
    \item Il sistema mostra l'etichetta della tipologia documentale associata al file
\end{enumerate}


\vspace{0.5cm}

\subsubsection{UC-2C – Gestione documento}\label{sec:uc-2c}

\begin{figure}[H]
    \centering
    \includegraphics[width=1\textwidth]{Diagrammi casi d'uso/UC2C.jpg}
    \caption{Diagramma del caso d'uso UC-2C – Gestione documento}
\end{figure}


\textbf{Attori}
\begin{itemize}
    \item Operatore CdL
\end{itemize}


\textbf{Pre-condizioni}
\begin{itemize}
    \item Ci si trova in una delle pagine precedenti (UC-2B o UC-2D o UC-2E).
    \item Si è selezionato un documento da una delle lista.
    \item Il documento è disponibile per la visualizzazione.
\end{itemize}

\textbf{Post-condizioni}
\begin{itemize}
    \item L'utente può apportare le modifiche desiderate al documento e può procedere con le azioni successive 
    \item L'utente è già soddisfatto delle informazioni del documento e può tornare alla pagina precedente.
\end{itemize}



\textbf{Scenario principale}
\begin{enumerate}
    \item L'utente ha caricato vari documenti e vuole ricontrollare le informazioni prima di inviarli.
    \item L'utente seleziona un documento dalla lista per visualizzarne i dettagli.
\end{enumerate}

\textbf{Relazioni con altri casi d'uso (\textit{include} / \textit{extend})}
\begin{itemize}
    \item \textit{include}: 
    \begin{itemize}
        \item UC-2B.10 – Visualizzazione codice documento
        \item UC-2B.11 - Visualizzazione tipologia documento
        \item UC-2D.5 – Visualizzazione destinatario
        \item UC-2B.8 - Visualizzazione nome documento originale
        \item UC-2C.1 - Visualizzazione anteprima documento
        \item UC-2E.1 - Visualizzazione percentuale confidenza
    \end{itemize}
    \item \textit{extend}: 
    \begin{itemize}
        \item UC-2C.2 – Modifica destinatario
        \item UC-2C.3 – Modifica tipologia documento
        \item UC-2C.4 – Rivaluta percentuale confidenza
        \item UC-2C.5 – Salva modifiche documento
    \end{itemize}
\end{itemize}

\vspace{0.5cm}

\subsubsection{UC-2C.1 – Visualizzazione anteprima documento}\label{sec:uc-2c-1}

\textbf{Attori}
\begin{itemize}
    \item Operatore CdL
\end{itemize}

\textbf{Pre-condizioni}
\begin{itemize}
    \item L'utente ha effettuato l'accesso al dettaglio del documento (UC-2C).
    \item Il file fisico o digitale del documento è accessibile
\end{itemize}

\textbf{Post-condizioni}
\begin{itemize}
    \item L'anteprima del documento viene mostrata correttamente a schermo permettendo il confronto con i dati estratti.
\end{itemize}

\textbf{Scenario principale}
\begin{enumerate}
    \item Il sistema recupera il file sorgente (PDF o immagine).
    \item Il sistema renderizza l'anteprima del documento nell'area dedicata della pagina di dettaglio.
\end{enumerate}


\vspace{0.5cm}

\subsubsection{UC-2C.2 – Modifica destinatario}\label{sec:uc-2c-2}

\textbf{Attori}
\begin{itemize}
    \item Operatore CdL
\end{itemize}

\textbf{Pre-condizioni}
\begin{itemize}
    \item L'utente ha effettuato l'accesso al dettaglio del documento (UC-2C).
\end{itemize}

\textbf{Post-condizioni}
\begin{itemize}
    \item Il documento è associato provvisoriamente a un nuovo destinatario (in attesa di salvataggio).
\end{itemize}

\textbf{Scenario principale}
\begin{enumerate}
    \item L'utente nota che il destinatario assegnato automaticamente o precedentemente è errato o mancante.
    \item L'utente attiva il campo di modifica del destinatario.
    \item L'utente ricerca e seleziona il destinatario corretto dall'anagrafica.
    \item Il sistema aggiorna il campo visualizzato con il nuovo valore.
\end{enumerate}


\vspace{0.5cm}

\subsubsection{UC-2C.3 – Modifica tipologia documento}\label{sec:uc-2c-3}

\textbf{Attori}
\begin{itemize}
    \item Operatore CdL
\end{itemize}

\textbf{Pre-condizioni}
\begin{itemize}
    \item L'utente ha effettuato l'accesso al dettaglio del documento (UC-2C).
\end{itemize}

\textbf{Post-condizioni}
\begin{itemize}
    \item La tipologia del documento è stata modificata.
\end{itemize}

\textbf{Scenario principale}
\begin{enumerate}
    \item L'utente rileva che la classificazione automatica del documento è errata (es. classificato come "Fattura" invece di "Contratto").
    \item L'utente seleziona la nuova tipologia da un menu a tendina.
    \item Il sistema aggiorna la tipologia.
\end{enumerate}


\vspace{0.5cm}

\subsubsection{UC-2C.4 – Rivaluta percentuale confidenza}\label{sec:uc-2c-4}

\textbf{Attori}
\begin{itemize}
    \item Operatore CdL
\end{itemize}

\textbf{Pre-condizioni}
\begin{itemize}
    \item L'utente ha effettuato l'accesso al dettaglio del documento
    \item L'utente ha apportato modifiche al documento
\end{itemize}

\textbf{Post-condizioni}
\begin{itemize}
    \item La percentuale di confidenza viene ricalcolata o impostata al 100\% (validazione manuale).
\end{itemize}

\textbf{Scenario principale}
\begin{enumerate}
    \item L'utente ha modificato la tipologia del documento (UC-2C.3).
    \item Il sistema ri-analizza i campi in base alla nuova struttura dati prevista.
    \item Il sistema aggiorna la percentuale di confidenza per riflettere la nuova coerenza dei dati o la validazione manuale dell'utente.
\end{enumerate}

\vspace{0.5cm}

\subsubsection{UC-2C.5 – Salva modifiche documento}\label{sec:uc-2c-5}

\textbf{Attori}
\begin{itemize}
    \item Operatore CdL
\end{itemize}

\textbf{Pre-condizioni}
\begin{itemize}
    \item L'utente ha effettuato l'accesso al dettaglio del documento
    \item Sono state apportate modifiche ai metadati del documento (destinatario, tipologia, note, ecc.).
    \item I dati inseriti rispettano i vincoli di validazione.
\end{itemize}

\textbf{Post-condizioni}
\begin{itemize}
    \item Le modifiche sono persistite nel database.
    \item Il documento viene marcato come "Validato" o "Processato".
    \item L'utente viene riportato alla lista documenti o rimane sulla pagina con notifica di successo.
\end{itemize}

\textbf{Scenario principale}
\begin{enumerate}
    \item L'utente clicca sul pulsante di salvataggio/conferma.
    \item Il sistema valida la coerenza dei dati inseriti.
    \item Il sistema salva le modifiche nel database.
    \item Il sistema notifica all'utente l'avvenuto salvataggio.
\end{enumerate}

\textbf{Scenario secondario}
\begin{enumerate}
    \item Il sistema rileva dati mancanti o incongruenti (Errore validazione).
    \item Il salvataggio viene bloccato e viene mostrato un messaggio di errore all'utente.
\end{enumerate}

\subsubsection{UC-2D – Visualizzazione lista informazioni destinatari analisi}\label{sec:uc-2d}

\begin{figure}[H]
    \centering
    \includegraphics[width=0.7\textwidth]{Diagrammi casi d'uso/UC2D.jpg}
    \caption{Diagramma del caso d'uso UC-2D – Visualizzazione lista informazioni destinatari}
\end{figure}

\textbf{Attori}
\begin{itemize}
    \item Operatore CdL
\end{itemize}

\textbf{Pre-condizioni}
\begin{itemize}
    \item La lista dei destinatari contiene almeno un elemento
\end{itemize}

\textbf{Post-condizioni}
\begin{itemize}
    \item L'utente visualizza la lista di elementi salvati nel suo insieme
\end{itemize}

\textbf{Scenario principale}
\begin{enumerate}
    \item L'utente si trova nel modulo storico dell'AI Co-pilot
    \item L'utente vede l'elenco di tutti gli elementi relativi ai destinatari recuperati dall'analisi
\end{enumerate}

\textbf{Relazioni con altri casi d'uso (\textit{include} / \textit{extend})}
\begin{itemize}
    \item \textit{include}: 
    \begin{itemize}
        \item UC-2D.1 – Visualizzazione elemento da lista informazioni destinatari
    \end{itemize}
\end{itemize}

\vspace{0.5cm}

\subsubsection{UC-2D.1 – Visualizzazione elemento da lista informazioni destinatari}\label{sec:uc-2d-1}

\textbf{Attori}
\begin{itemize}
    \item Operatore CdL
\end{itemize}


\textbf{Pre-condizioni}
\begin{itemize}
    \item La lista dei destinatari contiene almeno un elemento
\end{itemize}


\textbf{Post-condizioni}
\begin{itemize}
    \item L'utente può visualizzare le informazioni riguardanti un elemento dalla lista.
\end{itemize}

\textbf{Scenario principale}
\begin{enumerate}
    \item L'utente dopo aver caricato uno o più documenti accede alla sezione "Lista documenti".
    \item Il sistema mostra l'elenco dei documenti caricati, con le relative informazioni (tipologia, competenza, azienda, causale, lingua, numero pagine, nome originale).
\end{enumerate}

\textbf{Scenario secondario}
\begin{enumerate}
    \item Il sistema non riconosce alcun documento caricato.
\end{enumerate}

\textbf{Relazioni con altri casi d'uso (\textit{include} / \textit{extend})}
\begin{itemize}
    \item \textit{include}: 
    \begin{itemize}
        \item UC-2B.10 – Visualizzazione codice documento
        \item UC-2D.5 – Visualizzazione destinatario
        \item UC-2D.6 - Visualizzazione codice fiscale
        \item UC-2D.3 - Visualizzazione matricola
        \item UC-2D.4 - Visualizzazione reparto
    \end{itemize}
    \item \textit{extend}: 
    \begin{itemize}
        \item Nessuno
    \end{itemize}
\end{itemize}

\subsubsection{UC-2D.2 – Visualizzazione codice fiscale}\label{sec:uc-2d-2}

\textbf{Attori}
\begin{itemize}
    \item Operatore CdL
\end{itemize}

\textbf{Pre-condizioni}
\begin{itemize}
    \item La lista dei destinatari contiene almeno un elemento
\end{itemize}

\textbf{Post-condizioni}
\begin{itemize}
    \item Il codice fiscale del destinatario è visibile per la verifica dell'identità.
\end{itemize}

\textbf{Scenario principale}
\begin{enumerate}
    \item Il sistema recupera il codice fiscale dal record anagrafico del destinatario.
    \item Il sistema formatta e visualizza il codice fiscale nel campo dedicato.
\end{enumerate}

\vspace{0.5cm}

\subsubsection{UC-2D.3 – Visualizzazione matricola}\label{sec:uc-2d-3}

\textbf{Attori}
\begin{itemize}
    \item Operatore CdL
\end{itemize}

\textbf{Pre-condizioni}
\begin{itemize}
    \item La lista dei destinatari contiene almeno un elemento
\end{itemize}

\textbf{Post-condizioni}
\begin{itemize}
    \item Il numero di matricola è visualizzato correttamente.
\end{itemize}

\textbf{Scenario principale}
\begin{enumerate}
    \item Il sistema verifica se al destinatario è associato un numero di matricola aziendale.
    \item Il sistema mostra il numero di matricola per permettere la correlazione con i sistemi paghe.
\end{enumerate}

\textbf{Scenario secondario}
\begin{enumerate}
    \item Il destinatario non possiede una matricola (es. collaboratore esterno).
    \item Il campo viene mostrato vuoto o con dicitura "N/D".
\end{enumerate}

\vspace{0.5cm}

\subsubsection{UC-2D.4 – Visualizzazione reparto}\label{sec:uc-2d-4}

\textbf{Attori}
\begin{itemize}
    \item Operatore CdL
\end{itemize}

\textbf{Pre-condizioni}
\begin{itemize}
    \item La lista dei destinatari contiene almeno un elemento
\end{itemize}

\textbf{Post-condizioni}
\begin{itemize}
    \item L'utente visualizza il reparto 
\end{itemize}

\textbf{Scenario principale}
\begin{enumerate}
    \item Il sistema recupera le informazioni sull'organigramma o l'assegnazione del destinatario.
    \item Il sistema visualizza il nome del reparto di afferenza.
\end{enumerate}

\subsubsection{UC-2D.5 – Visualizzazione destinatario}\label{sec:uc-2d-5}

\textbf{Attori}
\begin{itemize}
    \item Operatore CdL
\end{itemize}

\textbf{Pre-condizioni}
\begin{itemize}
    \item La lista dei destinatari contiene almeno un elemento
\end{itemize}

\textbf{Post-condizioni}
\begin{itemize}
    \item Il nome e cognome del destinatario sono visibili a schermo.
\end{itemize}

\textbf{Scenario principale}
\begin{enumerate}
    \item Il sistema recupera l'identificativo del destinatario associato al documento selezionato.
    \item Il sistema interroga l'anagrafica per ottenere i dati anagrafici principali.
    \item Il sistema mostra il nome del destinatario nell'intestazione della scheda.
\end{enumerate}

\vspace{0.5cm}

\subsubsection{UC-2D.6 – Nessun destionatario riconosciuto}\label{sec:uc-2b-6}

\textbf{Attori}
\begin{itemize}
    \item Operatore CdL
\end{itemize}

\textbf{Pre-condizioni}
\begin{itemize}
    \item L'utente accede alla lista o effettua una ricerca/filtro
\end{itemize}

\textbf{Post-condizioni}
\begin{itemize}
    \item Viene mostrato un messaggio che informa l'utente dell'assenza di destinatari
\end{itemize}

\textbf{Scenario principale}
\begin{enumerate}
    \item L'utente ha eseguito un'analisi di un file
    \item L'analisi non ha riconosciuto alcun destinatario
\end{enumerate}


\vspace{0.5cm}


\subsubsection{UC-2E – Visualizzazione lista storico documenti}\label{sec:uc-2e}

\begin{figure}[H]
    \centering
    \includegraphics[width=0.7\textwidth]{Diagrammi casi d'uso/UC2E.jpg}
    \caption{Diagramma del caso d'uso UC-2E – Visualizzazione lista storico documenti}
\end{figure}

\textbf{Attori}
\begin{itemize}
    \item Operatore CdL
\end{itemize}

\textbf{Pre-condizioni}
\begin{itemize}
    \item La lista dei contiene almeno un elemento
    \item L'utente si trova nella pagina dello storico documenti (UC-2E)
\end{itemize}

\textbf{Post-condizioni}
\begin{itemize}
    \item L'utente visualizza la lista di documenti nello storico salvati nel suo insieme
\end{itemize}

\textbf{Scenario principale}
\begin{enumerate}
    \item L'utente si trova nel modulo storico dell'AI Co-pilot
    \item L'utente vede l'elenco di tutti gli elementi relativi a tutte le analisi
\end{enumerate}

\textbf{Relazioni con altri casi d'uso (\textit{include} / \textit{extend})}
\begin{itemize}
    \item \textit{include}: 
    \begin{itemize}
        \item UC-2E.4 – Visualizzazione elemento da lista storico documenti
    \end{itemize}
    \item \textit{extend}: 
    \begin{itemize}
        \item UC-2E.1 – Documenti assenti
    \end{itemize}
\end{itemize}

\vspace{0.5cm}

\subsubsection{UC-2E.1 – Documenti assenti}\label{sec:uc-2e-1}

\textbf{Attori}
\begin{itemize}
    \item Operatore CdL
\end{itemize}

\textbf{Pre-condizioni}
\begin{itemize}
    \item Non esistono documenti salvati nello storico
\end{itemize}

\textbf{Post-condizioni}
\begin{itemize}
    \item Viene mostrato un messaggio informativo che indica l'assenza di risultati.
    \item La lista appare vuota.
\end{itemize}

\textbf{Scenario principale}
\begin{enumerate}
    \item Il sistema interroga il database per recuperare i documenti.
    \item La ricerca non produce risultati (0 record).
    \item Il sistema visualizza un placeholder o un messaggio "Nessun documento trovato" per informare l'utente.
\end{enumerate}

\vspace{0.5cm}

\subsubsection{UC-2E.2 – Visualizzazione appartenenza lista distribuzione}\label{sec:uc-2e-2}

\textbf{Attori}
\begin{itemize}
    \item Operatore CdL
\end{itemize}

\textbf{Pre-condizioni}
\begin{itemize}
    \item La lista contiene almeno un elemento
\end{itemize}

\textbf{Post-condizioni}
\begin{itemize}
    \item L'utente visualizza chi riceverà o ha ricevuto il documento.
\end{itemize}

\textbf{Scenario principale}
\begin{enumerate}
    \item Il sistema recupera i collegamenti tra il documento e le anagrafiche destinatari.
    \item Il sistema mostra i nomi dei destinatari o, in caso di lista lunga, un riepilogo (es. "Mario Rossi + 2 altri") espandibile al passaggio del mouse o al click.
\end{enumerate}

\vspace{0.5cm}

\subsubsection{UC-2E.3 – Visualizzazione percentuale confidenza}\label{sec:uc-2e-3}

\textbf{Attori}
\begin{itemize}
    \item Operatore CdL
\end{itemize}

\textbf{Pre-condizioni}
\begin{itemize}
    \item La lista contiene almeno un elemento
    \item È stata calcolata la percentuale di confidenza media del documento
\end{itemize}

\textbf{Post-condizioni}
\begin{itemize}
    \item Viene mostrato un indicatore numerico o visivo dell'affidabilità dei dati estratti.
\end{itemize}

\textbf{Scenario principale}
\begin{enumerate}
    \item Il sistema recupera il punteggio di confidenza calcolato durante l'importazione.
    \item Il sistema formatta il dato come percentuale.
    \item Il sistema visualizza il dato, evidenziandolo (es. in rosso) se la confidenza è sotto una soglia di sicurezza, suggerendo un controllo manuale.
\end{enumerate}

\vspace{0.5cm}

\subsubsection{UC-2E.4 – Visualizzazione elemento da lista storico documenti}\label{sec:uc-2e-4}

\textbf{Attori}
\begin{itemize}
    \item Operatore CdL
\end{itemize}

\textbf{Pre-condizioni}
\begin{itemize}
    \item La lista contiene almeno un elemento
\end{itemize}

\textbf{Post-condizioni}
\begin{itemize}
    \item Per ogni riga della lista vengono mostrati i dati essenziali del documento.
\end{itemize}

\textbf{Scenario principale}
\begin{enumerate}
    \item Il sistema recupera i metadati principali per ogni documento in elenco.
    \item Il sistema renderizza le colonne della tabella mostrando: Codice documento, Data caricamento, Stato elaborazione e Percentuale di confidenza.
\end{enumerate}

\textbf{Relazioni con altri casi d'uso (\textit{include} / \textit{extend})}
\begin{itemize}
    \item \textit{include}: 
    \begin{itemize}
        \item UC-2B.10 – Visualizzazione codice documento
        \item UC-2E.5 - Visualizzazione stato
        \item UC-2E.3 - Visualizzazione percentuale confidenza
        \item UC-2E.2 - Visualizzazione appartenenza lista distribuzione
    \end{itemize}
    \item \textit{extend}: 
    \begin{itemize}
        \item Nessuna
    \end{itemize}
\end{itemize}

\vspace{0.5cm}

\subsubsection{UC-2E.5 – Visualizzazione stato}\label{sec:uc-2e-5}

\textbf{Attori}
\begin{itemize}
    \item Operatore CdL
\end{itemize}

\textbf{Pre-condizioni}
\begin{itemize}
    \item La lista contiene almeno un elemento
\end{itemize}

\textbf{Post-condizioni}
\begin{itemize}
    \item L'utente conosce lo stato attuale del ciclo di vita del documento (es. "Da validare", "Pronto per l'invio", "Inviato", "Errore").
\end{itemize}

\textbf{Scenario principale}
\begin{enumerate}
    \item Il sistema interroga il workflow del documento per determinarne lo stato corrente.
    \item Il sistema visualizza un'etichetta testuale o un'icona colorata rappresentativa dello stato (es. Verde per inviato, Giallo per in attesa).
\end{enumerate}

\vspace{0.5cm}

\subsubsection{UC-2F – Gestione messaggio}\label{sec:uc-2f}

\begin{figure}[H]
    \centering
    \includegraphics[width=1\textwidth]{Diagrammi casi d'uso/UC2Fa.jpg}
    \caption{Diagramma del caso d'uso UC-2F – Gestione messaggio (include)}
\end{figure}

\begin{figure}[H]
    \centering
    \includegraphics[width=1\textwidth]{Diagrammi casi d'uso/UC2Fb.jpg}
    \caption{Diagramma del caso d'uso UC-2F – Gestione messaggio (extend)}
\end{figure}


\textbf{Attori}
\begin{itemize}
    \item Operatore CdL
\end{itemize}

\textbf{Pre-condizioni}
\begin{itemize}
    \item Almeno un documento è stato selezionato per l'invio.
\end{itemize}

\textbf{Post-condizioni}
\begin{itemize}
    \item Il messaggio è pronto per essere inviato assieme ai documenti selezionati.
\end{itemize}

\textbf{Scenario principale}
\begin{enumerate}
    \item Un utente ha selezionato uno o piú documenti da inviare a uno o piú destinatari e sta prepara in messaggio da inviare assieme ai documenti.
\end{enumerate}

\textbf{Scenario secondario}
\begin{enumerate}
    \item Un utente esce dalla pagina di gestione messaggio senza salvare le modifiche apportate.
\end{enumerate}


\textbf{Relazioni con altri casi d'uso (\textit{include} / \textit{extend})}
\begin{itemize}
    \item \textit{include}: 
    \begin{itemize}
        \item UC-2B.10 – Visualizzazione codice documento
        \item UC-2B.4 – Visualizzazione azienda documento
        \item UC-2B.9 - Visualizzazione data redazione documento
        \item UC-2F.6 - Visualizzazione oggetto messaggio
        \item UC-2F.7 - Visualizzazione testo messaggio
    \end{itemize}
    \item \textit{extend}: 
    \begin{itemize}
        \item UC-2F.1 - Salva template messaggio
        \item UC-2F.2 - Modifica oggetto messaggio
        \item UC-2F.3 - Modifica testo messaggio
        \item UC-2F.4 - Conferma modifica messaggio
        \item UC-2F.5 - Uscita senza salvare
    \end{itemize}
\end{itemize}

\vspace{0.5cm}

\subsubsection{UC-2F.1 – Salva template messaggio}\label{sec:uc-2f-1}

\textbf{Attori}
\begin{itemize}
    \item Operatore CdL
\end{itemize}

\textbf{Pre-condizioni}
\begin{itemize}
    \item L'utente ha modificato o generato un messaggio che ritiene utile per usi futuri.
    \item Il campo testo del messaggio non è vuoto.
\end{itemize}

\textbf{Post-condizioni}
\begin{itemize}
    \item Il contenuto del messaggio viene salvato nel database dei template personali o globali.
\end{itemize}

\textbf{Scenario principale}
\begin{enumerate}
    \item L'utente, soddisfatto del messaggio attuale, clicca sul pulsante "Salva come template".
    \item Il sistema richiede di assegnare un nome al nuovo template.
    \item L'utente inserisce il nome e conferma.
    \item Il sistema salva il template rendendolo disponibile per futuri utilizzi (vedi UC-2G).
\end{enumerate}

\vspace{0.5cm}

\subsubsection{UC-2F.2 – Modifica oggetto messaggio}\label{sec:uc-2f-2}

\textbf{Attori}
\begin{itemize}
    \item Operatore CdL
\end{itemize}

\textbf{Pre-condizioni}
\begin{itemize}
    \item L'utente si trova nella pagina di gestione messaggio (UC-2F).
\end{itemize}

\textbf{Post-condizioni}
\begin{itemize}
    \item L'oggetto della mail/messaggio è aggiornato con il testo inserito dall'utente.
\end{itemize}

\textbf{Scenario principale}
\begin{enumerate}
    \item L'utente seleziona il campo di input relativo all'oggetto del messaggio.
    \item L'utente digita o modifica il testo esistente (es. aggiungendo riferimenti specifici).
\end{enumerate}

\vspace{0.5cm}

\subsubsection{UC-2F.3 – Modifica testo messaggio}\label{sec:uc-2f-3}

\textbf{Attori}
\begin{itemize}
    \item Operatore CdL
\end{itemize}

\textbf{Pre-condizioni}
\begin{itemize}
    \item L'utente si trova nella pagina di gestione messaggio (UC-2F).
\end{itemize}

\textbf{Post-condizioni}
\begin{itemize}
    \item Il corpo del messaggio è aggiornato.
\end{itemize}

\textbf{Scenario principale}
\begin{enumerate}
    \item L'utente seleziona l'area di testo contenente il corpo del messaggio.
    \item L'utente apporta modifiche manuali, aggiunge note o corregge il testo generato/preimpostato.
\end{enumerate}

\vspace{0.5cm}

\subsubsection{UC-2F.4 – Conferma modifica messaggio}\label{sec:uc-2f-4}

\textbf{Attori}
\begin{itemize}
    \item Operatore CdL
\end{itemize}

\textbf{Pre-condizioni}
\begin{itemize}
    \item Sono state apportate modifiche ai campi oggetto o testo (UC-2F.2 o UC-2F.3).
\end{itemize}

\textbf{Post-condizioni}
\begin{itemize}
    \item Le modifiche vengono validate in preparazione all'invio.
\end{itemize}

\textbf{Scenario principale}
\begin{enumerate}
    \item L'utente termina la digitazione.
    \item L'utente clicca fuori dall'area di testo o preme un pulsante di conferma parziale.
    \item Il sistema mantiene in memoria la versione aggiornata del messaggio pronta per l'invio (UC-2H).
\end{enumerate}

\vspace{0.5cm}

\subsubsection{UC-2F.5 – Conferma modifica messaggio}\label{sec:uc-2f-}

\textbf{Attori}
\begin{itemize}
    \item Operatore CdL
\end{itemize}

\textbf{Pre-condizioni}
\begin{itemize}
    \item Sono state apportate modifiche ai campi oggetto o testo (UC-2F.2 o UC-2F.3).
\end{itemize}

\textbf{Post-condizioni}
\begin{itemize}
    \item Le modifiche non vengono salvate e il messaggio ritorna alla versione precedente.
\end{itemize}

\textbf{Scenario principale}
\begin{enumerate}
    \item L'utente effettua delle modifiche al messaggio.
    \item L'utente decide di uscire dalla pagina senza salvare.
    \item Il sistema non salva le modifiche e ripristina il messaggio alla versione precedente.
\end{enumerate}

\vspace{0.5cm}

\subsubsection{UC-2F.6 – Visualizzazione oggetto messaggio}\label{sec:uc-2f-6}

\textbf{Attori}
\begin{itemize}
    \item Operatore CdL
\end{itemize}

\textbf{Pre-condizioni}
\begin{itemize}
    \item Ci si trova nella pagina di gestione messaggio (UC-2F)
\end{itemize}

\textbf{Post-condizioni}
\begin{itemize}
    \item L'oggetto corrente (vuoto, di default o generato) è visibile.
\end{itemize}

\textbf{Scenario principale}
\begin{enumerate}
    \item Il sistema recupera l'eventuale oggetto predefinito basato sulla tipologia di documento.
    \item Il sistema mostra il contenuto nel campo dedicato.
\end{enumerate}

\vspace{0.5cm}

\subsubsection{UC-2F.7 – Visualizzazione testo messaggio}\label{sec:uc-2f-7}

\textbf{Attori}
\begin{itemize}
    \item Operatore CdL
\end{itemize}

\textbf{Pre-condizioni}
\begin{itemize}
    \item Ci si trova nella pagina di gestione messaggio (UC-2F)
\end{itemize}

\textbf{Post-condizioni}
\begin{itemize}
    \item Il corpo del messaggio è visibile e leggibile.
\end{itemize}

\textbf{Scenario principale}
\begin{enumerate}
    \item Il sistema recupera il testo (vuoto o pre-generato).
    \item Il sistema renderizza l'area di testo permettendo la lettura del contenuto.
\end{enumerate}

\subsubsection{UC-2G - Gestione template salvati }\label{sec:uc-2g}

\begin{figure}[H]
    \centering
    \includegraphics[width=1\textwidth]{Diagrammi casi d'uso/UC2G.jpg}
    \caption{Diagramma del caso d'uso UC-2G – Gestione template salvati}
\end{figure}

\textbf{Attori}
\begin{itemize}
    \item Operatore CdL
\end{itemize}

\textbf{Pre-condizioni}
\begin{itemize}
    \item Esistono uno o più template di messaggi salvati nel sistema.
\end{itemize}

\textbf{Post-condizioni}
\begin{itemize}
    \item L'utente visualizza la pagina di gestione messaggio con il template selezionato caricato nei campi oggetto e testo del messaggio.
\end{itemize}


\textbf{Scenario principale}
\begin{enumerate}
    \item L'utente ha bisogno di utilizzare un template di messaggio salvato per preparare il messaggio da inviare assieme ai documenti selezionati.
\end{enumerate}

\textbf{Relazioni con altri casi d'uso (\textit{include} / \textit{extend})}
\begin{itemize}
    \item \textit{include}: 
    \begin{itemize}
        \item UC-2G.4 – Visualizzazione lista template salvati
    \end{itemize}
    \item \textit{extend}: 
    \begin{itemize}
        \item UC-2G.1 – Elimina template
        \item UC-2G.2 – Nessun template salvato
        \item UC-2G.3 – Carica template
    \end{itemize}
\end{itemize}

\vspace{0.5cm}

\subsubsection{UC-2G.1 – Elimina template}\label{sec:uc-2g-1}

\textbf{Attori}
\begin{itemize}
    \item Operatore CdL
\end{itemize}

\textbf{Pre-condizioni}
\begin{itemize}
    \item L'utente visualizza la lista dei template salvati (UC-2G).
    \item Il template da eliminare è presente in lista.
\end{itemize}

\textbf{Post-condizioni}
\begin{itemize}
    \item Il template viene rimosso definitivamente dal sistema.
    \item La lista viene aggiornata e non mostra più il template eliminato.
\end{itemize}

\textbf{Scenario principale}
\begin{enumerate}
    \item L'utente identifica un template obsoleto o errato nella tabella.
    \item L'utente clicca sull'icona di eliminazione (es. cestino) in corrispondenza della riga.
    \item Il sistema richiede conferma dell'operazione.
    \item L'utente conferma e il sistema cancella il record dal database.
\end{enumerate}


\vspace{0.5cm}

\subsubsection{UC-2G.2 – Nessun template salvato}\label{sec:uc-2g-2}

\textbf{Attori}
\begin{itemize}
    \item Operatore CdL
\end{itemize}

\textbf{Pre-condizioni}
\begin{itemize}
    \item L'utente ha richiesto l'apertura della tabella template (UC-2G).
    \item Non sono presenti template nel database per l'utente o per lo studio.
\end{itemize}

\textbf{Post-condizioni}
\begin{itemize}
    \item Viene visualizzato un messaggio informativo ("Nessun template trovato").
    \item La tabella appare vuota o nascosta.
\end{itemize}

\textbf{Scenario principale}
\begin{enumerate}
    \item Il sistema interroga il database per recuperare la lista dei template disponibili.
    \item La query restituisce un risultato vuoto.
    \item Il sistema notifica all'utente che non ci sono template da caricare.
\end{enumerate}

\vspace{0.5cm}

\subsubsection{UC-2G.3 – Carica template}\label{sec:uc-2g-3}

\textbf{Attori}
\begin{itemize}
    \item Operatore CdL
\end{itemize}

\textbf{Pre-condizioni}
\begin{itemize}
    \item L'utente ha selezionato un template valido dalla lista (UC-2G).
\end{itemize}

\textbf{Post-condizioni}
\begin{itemize}
    \item I dati del template (Oggetto e Corpo del messaggio) vengono trasferiti nei campi di input della pagina di gestione messaggio (UC-2F).
    \item Il modale o la pagina di selezione template si chiude.
\end{itemize}

\textbf{Scenario principale}
\begin{enumerate}
    \item L'utente clicca sul pulsante "Carica" o sulla riga del template desiderato.
    \item Il sistema recupera il contenuto del template (testo e oggetto).
    \item Il sistema sovrascrive o compila i campi corrispondenti nell'interfaccia di composizione del messaggio.
    \item L'utente può ora procedere a modificare o inviare il messaggio (ritorno a UC-2F).
\end{enumerate}

\subsubsection{UC-2G.4 – Visualizzazione lista template salvati}\label{sec:uc-2g-4}

\textbf{Attori}
\begin{itemize}
    \item Operatore CdL
\end{itemize}

\textbf{Pre-condizioni}
\begin{itemize}
    \item La lista contiene almeno un template salvato
\end{itemize}

\textbf{Post-condizioni}
\begin{itemize}
    \item L'utente visualizza la lista di template salvati nel suo insieme
\end{itemize}

\textbf{Scenario principale}
\begin{enumerate}
    \item L'utente si trova nella sezione di gestione messaggi e vuole utilizzare un template salvato
    \item L'utente vede l'elenco di tutti i template relativi ai messaggi salvati
\end{enumerate}

\textbf{Relazioni con altri casi d'uso (\textit{include} / \textit{extend})}
\begin{itemize}
    \item \textit{include}: 
    \begin{itemize}
        \item UC-2G.5 – Visualizzazione elemento da lista template salvati
    \end{itemize}
\end{itemize}

\vspace{0.5cm}

\subsubsection{UC-2G.5 – Visualizzazione elemento da lista template salvati}\label{sec:uc-2g-5}

\textbf{Attori}
\begin{itemize}
    \item Operatore CdL
\end{itemize}


\textbf{Pre-condizioni}
\begin{itemize}
    \item è stato salvato almeno un template di messaggio.
\end{itemize}


\textbf{Post-condizioni}
\begin{itemize}
    \item L'utente può visualizzare le informazioni riguardanti un elemento dalla lista.
\end{itemize}

\textbf{Scenario principale}
\begin{enumerate}
    \item L'utente vuole visualizzare i dettagli di un template salvato.
\end{enumerate}

\textbf{Scenario secondario}
\begin{enumerate}
    \item nessuno
\end{enumerate}

\textbf{Relazioni con altri casi d'uso (\textit{include} / \textit{extend})}
\begin{itemize}
    \item \textit{include}: 
    \begin{itemize}
        \item UC-1B.8 – Visualizzazione tono
        \item UC-1B.3 – Visualizzazione stile 
        \item UC-2F.6 – Visualizzazione oggetto messaggio
        \item UC-2F.7 – Visualizzazione testo messaggio
    \end{itemize}
    \item \textit{extend}: 
    \begin{itemize}
        \item Nessuno
    \end{itemize}
\end{itemize}

\subsubsection{UC-2H – Invio documento e messaggio}\label{sec:uc-2h}

\begin{figure}[H]
    \centering
    \includegraphics[width=1\textwidth]{Diagrammi casi d'uso/UC2H.jpg}
    \caption{Diagramma del caso d'uso UC-2H – Invio documento e messaggio}
\end{figure}


\textbf{Attori}
\begin{itemize}
    \item Operatore CdL
\end{itemize}

\textbf{Pre-condizioni}
\begin{itemize}
    \item L'utente si trova all'interno del modulo di invio di documenti e messaggio
    \item Il messaggio è stato completato con oggetto, testo e documenti allegati.
\end{itemize}

\textbf{Post-condizioni}
\begin{itemize}
    \item Il sistema avvia il processo di spedizione delle email/messaggi.
    \item Il messaggio e i documenti sono stati inviati con successo ai destinatari previsti.
    \item L'utente riceve un feedback di successo.
\end{itemize}

\textbf{Scenario principale}
\begin{enumerate}
    \item L'utente ha completato la preparazione del messaggio e dei documenti da inviare.
    \item L'utente seleziona l'opzione per inviare il messaggio e i documenti ai destinatari.
\end{enumerate}

\textbf{Scenario secondario}
\begin{enumerate}
    \item 
\end{enumerate}

\textbf{Relazioni con altri casi d'uso (\textit{include} / \textit{extend})}
\begin{itemize}
    \item \textit{include}: 
    \begin{itemize}
        \item UC-2F.6 – Visualizzazione oggetto messaggio
        \item UC-2F.7 – Visualizzazione testo messaggio
        \item UC-2H.4 – Visualizzazione lista destinatari
        \item UC-2H.5 – Visualizzazione lista documenti
    \end{itemize}
    \item \textit{extend}: 
    \begin{itemize}
        \item UC-2H.1 – Allega file
        \item UC-2H.2 – Pianifica invio
    \end{itemize}
\end{itemize}

\subsubsection{UC-2H.1 – Allega file}\label{sec:uc-2h-1}

\textbf{Attori}
\begin{itemize}
    \item Operatore CdL
\end{itemize}

\textbf{Pre-condizioni}
\begin{itemize}
    \item L'utente si trova nella pagina di riepilogo invio (UC-2H).
\end{itemize}

\textbf{Post-condizioni}
\begin{itemize}
    \item Il nuovo file viene aggiunto alla lista degli allegati pronti per l'invio.
\end{itemize}

\textbf{Scenario principale}
\begin{enumerate}
    \item L'utente clicca sul pulsante "Aggiungi allegato".
    \item Il sistema apre la finestra di selezione file del sistema operativo.
    \item L'utente seleziona il file desiderato.
    \item Il sistema carica il file e lo mostra nella lista allegati (UC-2H.5).
\end{enumerate}

\vspace{0.5cm}

\subsubsection{UC-2H.2 – Pianifica invio}\label{sec:uc-2h-2}

\textbf{Attori}
\begin{itemize}
    \item Operatore CdL
\end{itemize}

\textbf{Pre-condizioni}
\begin{itemize}
    \item L'utente si trova nella pagina di riepilogo invio (UC-2H)
\end{itemize}

\textbf{Post-condizioni}
\begin{itemize}
    \item L'invio viene schedulato per la data e ora selezionate.
    \item Lo stato dei documenti passa a "Pianificato".
\end{itemize}

\textbf{Scenario principale}
\begin{enumerate}
    \item L'utente seleziona l'opzione "Pianifica invio".
    \item Il sistema mostra un selettore di data e ora.
    \item L'utente imposta il momento desiderato per l'invio e conferma.
    \item Il sistema prende in carico la richiesta e la accoda per l'elaborazione futura.
\end{enumerate}

\vspace{0.5cm}

\subsubsection{UC-2I – Filtraggio documenti analisi}\label{sec:uc-2b-i}

\begin{figure}[H]
    \centering
    \includegraphics[width=0.8\textwidth]{Diagrammi casi d'uso/UC2I.jpg}
    \caption{Diagramma del caso d'uso UC-2I – Filtraggio documenti analisi}
\end{figure}

\textbf{Attori}
\begin{itemize}
    \item Operatore CdL
\end{itemize}


\textbf{Pre-condizioni}
\begin{itemize}
    \item è stato caricato almeno un documento tramite UC-2A.6.
    \item L'utente è entrato nel modulo AI Co-Pilot per i CdL dalla dashboard principale.
    \item L'utente dispone dei permessi necessari per utilizzare il modulo AI Co-Pilot
\end{itemize}


\textbf{Post-condizioni}
\begin{itemize}
    \item L'utente può visualizzare le informazioni riguardanti gli elementi concordanti con il filtro
\end{itemize}

\textbf{Scenario principale}
\begin{enumerate}
    \item L'utente dopo aver caricato uno o più documenti accede alla sezione "Lista documenti".
    \item Il sistema mostra l'elenco dei documenti caricati, con le relative informazioni (tipologia, competenza, azienda, causale, lingua, numero pagine, nome originale).
\end{enumerate}

\textbf{Scenario secondario}
\begin{enumerate}
    \item Il sistema non riconosce alcun documento caricato.
\end{enumerate}


\textbf{Relazioni con altri casi d'uso (\textit{include} / \textit{extend})}
\begin{itemize}
    \item \textit{include}: 
    \begin{itemize}
        \item UC-2I.1 – Visualizzazione lista documenti filtrata
    \end{itemize}
    \item \textit{extend}: 
    \begin{itemize}
        \item Nessuno
    \end{itemize}
\end{itemize}

\subsubsection{UC-2I.1 – Visualizzazione lista documenti filtrata}\label{sec:uc-2b-i-1}

\textbf{Attori}
\begin{itemize}
    \item Operatore CdL
\end{itemize}

\textbf{Pre-condizioni}
\begin{itemize}
    \item La lista dello storico contiene almeno un elemento 
    \item L'utente ha inserito un criterio di ricerca valido (testo libero)
\end{itemize}

\textbf{Post-condizioni}
\begin{itemize}
    \item L'utente visualizza la lista di elementi (coerenti con la ricerca) trovati dall'analisi
\end{itemize}

\textbf{Scenario principale}
\begin{enumerate}
    \item L'utente si trova nel modulo storico dell'AI Assistant Generativo
    \item L'utente vede l'elenco di tutti gli elementi salvati con i relativi dati principali (prompt parziale, tono, stile, data)
\end{enumerate}


\textbf{Relazioni con altri casi d'uso (\textit{include} / \textit{extend})}
\begin{itemize}
    \item \textit{include}: 
    \begin{itemize}
        \item UC-2B.2 – Visualizzazione elemento da lista documenti
    \end{itemize}
\end{itemize}

\vspace{0.5cm}

\subsubsection{UC-2J – Filtraggio destinatari analisi}\label{sec:uc-2d-J}

\begin{figure}[H]
    \centering
    \includegraphics[width=0.8\textwidth]{Diagrammi casi d'uso/UC2J.jpg}
    \caption{Diagramma del caso d'uso UC-2J – Filtraggio destinatari analisi}
\end{figure}

\textbf{Attori}
\begin{itemize}
    \item Operatore CdL
\end{itemize}


\textbf{Pre-condizioni}
\begin{itemize}
    \item Si è nella pagina lista documenti (UC-2B)
    \item La lista dei destinatari contiene almeno un elemento
\end{itemize}


\textbf{Post-condizioni}
\begin{itemize}
    \item L'utente ha filtrato la lista dei destinatari in base ai criteri selezionati.
\end{itemize}

\textbf{Scenario principale}
\begin{enumerate}
    \item L'utente dopo aver caricato uno o più documenti accede alla sezione "Lista documenti".
    \item Il sistema mostra l'elenco dei documenti caricati, con le relative informazioni.
    \item L'utente applica filtri specifici per visualizzare solo i destinatari di interesse (es. per nome, codice fiscale, reparto).
\end{enumerate}

\textbf{Scenario secondario}
\begin{enumerate}
    \item Il sistema non riconosce alcun destinatario.
\end{enumerate}

\textbf{Relazioni con altri casi d'uso (\textit{include} / \textit{extend})}
\begin{itemize}
    \item \textit{include}: 
    \begin{itemize}
        \item UC-2J.1 – Visualizzazione lista destinatari filtrata
    \end{itemize}
    \item \textit{extend}: 
    \begin{itemize}
        \item Nessuno
    \end{itemize}
\end{itemize}

\subsubsection{UC-2J.1 – Visualizzazione lista destinatari filtrata}\label{sec:uc-2j-1}

\textbf{Attori}
\begin{itemize}
    \item Operatore CdL
\end{itemize}

\textbf{Pre-condizioni}
\begin{itemize}
    \item La lista dei destinatari contiene almeno un elemento 
    \item L'utente ha inserito un criterio di ricerca valido (testo libero)
\end{itemize}

\textbf{Post-condizioni}
\begin{itemize}
    \item L'utente visualizza la lista di elementi (coerenti con la ricerca) trovati dall'analisi
\end{itemize}

\textbf{Scenario principale}
\begin{enumerate}
    \item L'utente si trova nel modulo storico dell'AI Assistant Generativo
    \item L'utente vede l'elenco di tutti gli elementi salvati con i relativi dati principali 
\end{enumerate}

\textbf{Relazioni con altri casi d'uso (\textit{include} / \textit{extend})}
\begin{itemize}
    \item \textit{include}: 
    \begin{itemize}
        \item UC-2D.2 – Visualizzazione elemento da lista documenti
    \end{itemize}
\end{itemize}

\vspace{0.5cm}

\subsubsection{UC-2K – Filtraggio documenti storico}\label{sec:uc-2k}

\begin{figure}[H]
    \centering
    \includegraphics[width=0.8\textwidth]{Diagrammi casi d'uso/UC2K.jpg}
    \caption{Diagramma del caso d'uso UC-2K – Filtraggio documenti storico}
\end{figure}

\textbf{Attori}
\begin{itemize}
    \item Operatore CdL
\end{itemize}

\textbf{Pre-condizioni}
\begin{itemize}
    \item La lista dei documenti è visualizzata e popolata.
\end{itemize}

\textbf{Post-condizioni}
\begin{itemize}
    \item L''utente ha inserito i criteri di ricerca i criteri di ricerca inseriti.
\end{itemize}

\textbf{Scenario principale}
\begin{enumerate}
    \item L'utente desidera restringere la ricerca (es. per data, per tipologia o per stato invio).
    \item L'utente utilizza la barra di ricerca o i filtri laterali.
    \item Il sistema aggiorna la vista in tempo reale o dopo la conferma, nascondendo i documenti non pertinenti.
\end{enumerate}

\textbf{Scenario secondario}
\begin{enumerate}
    \item L'utente rimuove i filtri.
    \item Il sistema ripristina la visualizzazione completa dello storico.
\end{enumerate}

\textbf{Relazioni con altri casi d'uso (\textit{include} / \textit{extend})}
\begin{itemize}
    \item \textit{include}: 
    \begin{itemize}
        \item UC-2K.1 – Visualizzazione lista storico filtrata
    \end{itemize}
    \item \textit{extend}: 
    \begin{itemize}
        \item Nessuno
    \end{itemize}
\end{itemize}

\vspace{0.5cm}

\subsubsection{UC-2K.1 – Visualizzazione lista storico documenti filtrata}\label{sec:uc-2k-1}

\textbf{Attori}
\begin{itemize}
    \item Operatore CdL
\end{itemize}

\textbf{Pre-condizioni}
\begin{itemize}
    \item La lista dei contiene almeno un elemento
\end{itemize}

\textbf{Post-condizioni}
\begin{itemize}
    \item L'utente visualizza la lista filtrata di documenti nello storico salvati nel suo insieme
\end{itemize}

\textbf{Scenario principale}
\begin{enumerate}
    \item L'utente si trova nel modulo storico dell'AI Co-pilot
    \item L'utente vede l'elenco filtrato di tutti gli elementi relativi a tutte le analisi
\end{enumerate}

\textbf{Relazioni con altri casi d'uso (\textit{include} / \textit{extend})}
\begin{itemize}
    \item \textit{include}: 
    \begin{itemize}
        \item UC-2E.4 – Visualizzazione elemento da lista storico documenti
    \end{itemize}
\end{itemize}

\vspace{0.5cm}


\subsection{Sezione 3 - Modulo Analytics e Monitoraggio Trasversale}

\subsubsection{UC-3A - Analisi e Monitoraggio AI}\label{sec:uc-3a}

\begin{figure}[H]
    \centering
    \includegraphics[width=1\textwidth]{Diagrammi casi d'uso/UC3A.jpg}
    \caption{Diagramma del caso d'uso UC-3A – Analisi e Monitoraggio AI}
\end{figure}

\textbf{Attori}
\begin{itemize}
    \item Auditor/Data Analyst
\end{itemize}

\textbf{Pre-condizioni}
\begin{itemize}
    \item L'utente ha effettuato il login ed è autenticato come Auditor, Data Analyst o Amministratore.
    \item L'utente accede alla sezione ``Analytics e Monitoraggio'' dalla dashboard principale.
\end{itemize}

\textbf{Post-condizioni}
\begin{itemize}
    \item L'utente visualizza le metriche aggiornate in base ai filtri applicati (o di default).
\end{itemize}

\textbf{Scenario principale}
\begin{enumerate}
    \item L'utente accede al modulo centralizzato di Analytics.
    \item Il sistema calcola e mostra una panoramica generale (Dashboard) applicando i filtri di default (es. Periodo: Ultimo Mese, Utenti: Tutti).
    \item Il sistema visualizza i grafici e le tabelle relative alle metriche (confidenza, rating, volumi) per il contesto di default.
\end{enumerate}

\textbf{Scenario secondario}
\begin{enumerate}
    \item Il sistema non dispone di dati sufficienti nel periodo di default.
    \item Il sistema mostra un messaggio ``Nessun dato disponibile per il periodo selezionato''.
\end{enumerate}

\textbf{Relazioni con altri casi d'uso (\textit{include} / \textit{extend})}
\begin{itemize}
    \item \textit{include}: 
    \begin{itemize}
        \item UC-3A.1 - Visualizzazione elenco dati AI assistant
        \item UC-3A.7 - Visualizzazione elenco dati AI Co-Pilot
    \end{itemize}
    \item \textit{extend}: 
    \begin{itemize}
        \item UC-3A.12 - Visualizzazione da data inizio
        \item UC-3A.13 - Visualizzazione fino a data fine
    \end{itemize}
\end{itemize} 

\vspace{0.5cm}

\subsubsection{UC-3A.1 - Visualizzazione elenco dati AI assistant}\label{sec:uc-3a-1}

\begin{figure}[H]
    \centering
   
    \caption{Diagramma del caso d'uso UC-3A}
    \label{fig:uc3a}
\end{figure}

\textbf{Attori}
\begin{itemize}
    \item Auditor/Data Analyst
\end{itemize}

\textbf{Pre-condizioni}
\begin{itemize}
    \item Il sistema ha accumulato dati storici.
    \item L'utente accede alla sezione dell'elenco dati AI assistant
\end{itemize}

\textbf{Post-condizioni}
\begin{itemize}
    \item L'utente visualizza le metriche in base ai filtri applicati (o di default).
\end{itemize}

\textbf{Scenario principale}
\begin{enumerate}
    \item L'utente accede al modulo centralizzato di Analytics.
    \item Il sistema calcola e mostra una panoramica generale (Dashboard) applicando i filtri di data.
    \item Il sistema visualizza i grafici e le tabelle relative alle metriche (confidenza, rating, volumi) per il contesto di default.
\end{enumerate}

\textbf{Scenario secondario}
\begin{enumerate}
    \item Il sistema non dispone di dati sufficienti nel periodo di default.
    \item Il sistema mostra un messaggio ``Nessun dato disponibile per il periodo selezionato''.
\end{enumerate}

\textbf{Relazioni con altri casi d'uso (\textit{include} / \textit{extend})}
\begin{itemize}
    \item \textit{include}: 
    \begin{itemize}
        \item UC-3A.2 - Visualizzazione numero prompt generati
        \item UC-3A.3 - Visualizzazione rating medio prompt generati
        \item UC-3A.4 - Visualizzazione numero rigenerazioni prompt
        \item UC-3A.5 - Visualizzazione toni più usati
        \item UC-3A.6 - Visualizzazione stili più usati
    \end{itemize}
    \item \textit{extend}: 
    \begin{itemize}
        \item Nessuna
    \end{itemize}
\end{itemize} 

\vspace{0.5cm}

\subsubsection{UC-3A.2 – Visualizzazione numero prompt generati}\label{sec:uc-3a-2}

\textbf{Attori}
\begin{itemize}
    \item Auditor/Data Analyst
\end{itemize}

\textbf{Pre-condizioni}
\begin{itemize}
    \item L'utente accede alla sezione dell'elenco dati AI assistant
\end{itemize}

\textbf{Post-condizioni}
\begin{itemize}
    \item L'utente visualizza il volume totale di richieste (prompt) inviate all'AI nel periodo selezionato.
\end{itemize}

\textbf{Scenario principale}
\begin{enumerate}
    \item Il sistema interroga il database storico filtrando per l'intervallo temporale corrente.
    \item Il sistema conteggia il numero totale di interazioni avviate dagli utenti.
    \item Il sistema visualizza il dato numerico (KPI) in un widget dedicato della dashboard (es. "Totale Prompt: 1.250").
\end{enumerate}

\vspace{0.5cm}

\subsubsection{UC-3A.3 – Visualizzazione rating medio prompt generati}\label{sec:uc-3a-3}

\textbf{Attori}
\begin{itemize}
    \item Auditor/Data Analyst
\end{itemize}

\textbf{Pre-condizioni}
\begin{itemize}
    \item L'utente accede alla sezione dell'elenco dati AI assistant
    \item Gli utenti finali hanno fornito feedback (es. stelle o pollici su/giù) sui messaggi generati.
\end{itemize}

\textbf{Post-condizioni}
\begin{itemize}
    \item L'utente visualizza un indicatore della qualità percepita delle risposte dell'AI.
\end{itemize}

\textbf{Scenario principale}
\begin{enumerate}
    \item Il sistema aggrega i feedback ricevuti sulle risposte generate nel periodo di riferimento.
    \item Il sistema calcola la media aritmetica dei voti (es. su scala 1-5 o percentuale di approvazione).
    \item Il sistema mostra il valore medio (es. "Rating Medio: 4.2/5") permettendo di valutare la soddisfazione dell'utenza.
\end{enumerate}

\vspace{0.5cm}

\subsubsection{UC-3A.4 – Visualizzazione numero rigenerazioni prompt}\label{sec:uc-3a-4}

\textbf{Attori}
\begin{itemize}
    \item Auditor/Data Analyst
\end{itemize}

\textbf{Pre-condizioni}
\begin{itemize}
    \item L'utente accede alla sezione dell'elenco dati AI assistant
\end{itemize}

\textbf{Post-condizioni}
\begin{itemize}
    \item L'utente visualizza quante volte è stata utilizzata la funzione "Rigenera risposta".
\end{itemize}

\textbf{Scenario principale}
\begin{enumerate}
    \item Il sistema conta le occorrenze in cui un utente ha richiesto una nuova versione di un messaggio già generato (evento di "retry").
    \item Il sistema visualizza il totale delle rigenerazioni, dato utile per capire se la prima risposta dell'AI è spesso insoddisfacente.
\end{enumerate}

\vspace{0.5cm}

\subsubsection{UC-3A.5 – Visualizzazione toni più usati}\label{sec:uc-3a-5}

\textbf{Attori}
\begin{itemize}
    \item Auditor/Data Analyst
\end{itemize}

\textbf{Pre-condizioni}
\begin{itemize}
    \item L'utente accede alla sezione dell'elenco dati AI assistant
\end{itemize}

\textbf{Post-condizioni}
\begin{itemize}
    \item L'utente visualizza la distribuzione o la classifica dei "Toni" selezionati per la generazione dei messaggi.
\end{itemize}

\textbf{Scenario principale}
\begin{enumerate}
    \item Il sistema raggruppa i prompt generati in base all'attributo "Tono" (es. Formale, Empatico, Assertivo).
    \item Il sistema calcola la frequenza di utilizzo per ciascuna categoria.
    \item Il sistema visualizza un grafico (es. a torta o a barre) o una lista ordinata che evidenzia i toni prediletti dagli utenti.
\end{enumerate}

\vspace{0.5cm}

\subsubsection{UC-3A.6 – Visualizzazione stili più usati}\label{sec:uc-3a-6}

\textbf{Attori}
\begin{itemize}
    \item Auditor/Data Analyst
\end{itemize}

\textbf{Pre-condizioni}
\begin{itemize}
    \item L'utente accede alla sezione dell'elenco dati AI assistant
\end{itemize}

\textbf{Post-condizioni}
\begin{itemize}
    \item L'utente visualizza la distribuzione o la classifica degli "Stili" selezionati per la generazione dei messaggi.
\end{itemize}

\textbf{Scenario principale}
\begin{enumerate}
    \item Il sistema raggruppa i prompt generati in base all'attributo "Stile" (es. Sintetico, Dettagliato, Elenco puntato).
    \item Il sistema calcola la frequenza di utilizzo per ciascuna categoria.
    \item Il sistema visualizza un grafico o una classifica che mostra quali formati di risposta sono più richiesti.
\end{enumerate}

\subsubsection{UC-3A.7 - Visualizzazione elenco dati AI Co-pilot}\label{sec:uc-3a-7}

\begin{figure}[H]
    \centering
   
    \caption{Diagramma del caso d'uso UC-3A}
    \label{fig:uc3a}
\end{figure}

\textbf{Attori}
\begin{itemize}
    \item Auditor/Data Analyst
\end{itemize}

\textbf{Pre-condizioni}
\begin{itemize}
    \item L'utente ha effettuato il login ed è autenticato come Auditor, Data Analyst o Amministratore.
    \item L'utente accede alla sezione AI Co-Pilot
\end{itemize}

\textbf{Post-condizioni}
\begin{itemize}
    \item L'utente visualizza le metriche aggiornate in base ai filtri applicati (o di default).
\end{itemize}

\textbf{Scenario principale}
\begin{enumerate}
    \item L'utente accede al modulo centralizzato di Analytics.
    \item Il sistema calcola e mostra una panoramica generale (Dashboard) applicando i filtri di default (es. Periodo: Ultimo Mese, Utenti: Tutti).
    \item Il sistema visualizza i grafici e le tabelle relative alle metriche (confidenza, rating, volumi) per il contesto di default.
\end{enumerate}

\textbf{Scenario secondario}
\begin{enumerate}
    \item Il sistema non dispone di dati sufficienti nel periodo di default.
    \item Il sistema mostra un messaggio ``Nessun dato disponibile per il periodo selezionato''.
\end{enumerate}

\textbf{Relazioni con altri casi d'uso (\textit{include} / \textit{extend})}
\begin{itemize}
    \item \textit{include}: 
    \begin{itemize}
        \item UC-3A.8 - Visualizzazione confidenza media
        \item UC-3A.9 - Visualizzazione percentuale interventi manuali
        \item UC-3A.10 - Visualizzazione accuratezza mapping
        \item UC-3A.11 - Visualizzazione tempi medi analisi
    \end{itemize}
    \item \textit{extend}: 
    \begin{itemize}
        \item Nessuna
    \end{itemize}
\end{itemize} 

\vspace{0.5cm}

\subsubsection{UC-3A.8 – Visualizzazione confidenza media}\label{sec:uc-3a-8}

\textbf{Attori}
\begin{itemize}
    \item Auditor/Data Analyst
\end{itemize}

\textbf{Pre-condizioni}
\begin{itemize}
    \item L'utente ha fatto l'accesso alla dashboard di Analytics AI Co-pilot
\end{itemize}

\textbf{Post-condizioni}
\begin{itemize}
    \item L'utente visualizza il punteggio medio di affidabilità (confidence score) che l'AI ha attribuito alle sue estrazioni nel periodo selezionato.
\end{itemize}

\textbf{Scenario principale}
\begin{enumerate}
    \item Il sistema aggrega i punteggi di confidenza di tutti i documenti elaborati nell'intervallo temporale.
    \item Il sistema calcola la media pesata o aritmetica.
    \item Il sistema mostra il valore (es. 85\%) indicando se è in linea, superiore o inferiore rispetto ai trend storici.
\end{enumerate}

\vspace{0.5cm}

\subsubsection{UC-3A.9 – Visualizzazione percentuale interventi manuali}\label{sec:uc-3a-9}

\textbf{Attori}
\begin{itemize}
    \item Auditor/Data Analyst
\end{itemize}

\textbf{Pre-condizioni}
\begin{itemize}
    \item L'utente ha fatto l'accesso alla dashboard di Analytics AI Co-pilot
\end{itemize}

\textbf{Post-condizioni}
\begin{itemize}
    \item L'utente visualizza la frequenza con cui gli operatori umani devono correggere i dati estratti dall'AI.
\end{itemize}

\textbf{Scenario principale}
\begin{enumerate}
    \item Il sistema analizza i log di audit per contare quanti documenti hanno subito modifiche manuali ai campi estratti prima del salvataggio.
    \item Il sistema confronta questo dato con il volume totale dei documenti.
    \item Il sistema visualizza la percentuale di "Human-in-the-loop" (es. 12\% di documenti corretti manualmente).
\end{enumerate}

\vspace{0.5cm}

\subsubsection{UC-3A.10 – Visualizzazione accuratezza mapping}\label{sec:uc-3a-10}

\textbf{Attori}
\begin{itemize}
    \item Auditor/Data Analyst
\end{itemize}

\textbf{Pre-condizioni}
\begin{itemize}
    \item L'utente ha fatto l'accesso alla dashboard di Analytics AI Co-pilot
\end{itemize}

\textbf{Post-condizioni}
\begin{itemize}
    \item L'utente visualizza un indicatore che rappresenta la precisione con cui l'AI associa i dati trovati ai campi corretti del database.
\end{itemize}

\textbf{Scenario principale}
\begin{enumerate}
    \item Il sistema valuta la coerenza dei dati (es. verifica se i campi mappati sono stati successivamente spostati o rimossi dall'operatore).
    \item Il sistema calcola un indice di accuratezza strutturale.
    \item Il sistema visualizza il dato, permettendo di identificare se l'AI sta fallendo nel riconoscere specifici layout o tipologie di documento.
\end{enumerate}

\vspace{0.5cm}

\subsubsection{UC-3A.11 – Visualizzazione tempi medi analisi}\label{sec:uc-3a-11}

\textbf{Attori}
\begin{itemize}
    \item Auditor/Data Analyst
\end{itemize}

\textbf{Pre-condizioni}
\begin{itemize}
    \item L'utente ha fatto l'accesso alla dashboard di Analytics AI Co-pilot
\end{itemize}

\textbf{Post-condizioni}
\begin{itemize}
    \item L'utente visualizza il tempo medio impiegato dal sistema per processare un documento.
\end{itemize}

\textbf{Scenario principale}
\begin{enumerate}
    \item Il sistema recupera i timestamp di inizio (upload) e fine (disponibilità dati) elaborazione per i documenti del periodo.
    \item Il sistema calcola la durata media dell'elaborazione.
    \item Il sistema mostra il tempo medio (es. "1.5 secondi per pagina"), utile per monitorare le performance dell'infrastruttura e la latenza del servizio.
\end{enumerate}

\subsubsection{UC-3A.12 – Inserimento data inizio}\label{sec:uc-3a-12}

\textbf{Attori}
\begin{itemize}
    \item Auditor/Data Analyst
\end{itemize}

\textbf{Pre-condizioni}
\begin{itemize}
    \item L'utente ha fatto l'accesso alla dashboard di Analytics
    \item Il filtro temporale è attivo e modificabile.
\end{itemize}

\textbf{Post-condizioni}
\begin{itemize}
    \item La data di inizio del periodo di analisi è impostata.
    \item Il sistema è pronto ad aggiornare i grafici in base al nuovo intervallo (o li aggiorna automaticamente).
\end{itemize}

\textbf{Scenario principale}
\begin{enumerate}
    \item L'utente interagisce con il selettore di data "Dal" (Data inizio).
    \item Il sistema mostra un calendario interattivo.
    \item L'utente seleziona un giorno specifico.
    \item Il sistema verifica che la data selezionata sia antecedente o uguale alla data di fine (se impostata).
    \item Il sistema aggiorna il campo con la data scelta.
\end{enumerate}

\textbf{Scenario secondario}
\begin{enumerate}
    \item L'utente seleziona una data successiva alla data di fine attuale.
    \item Il sistema mostra un avviso di incongruenza temporale o resetta automaticamente la data di fine.
\end{enumerate}

\vspace{0.5cm}

\subsubsection{UC-3A.13 – Inserimento data fine}\label{sec:uc-3a-13}

\textbf{Attori}
\begin{itemize}
    \item Auditor/Data Analyst
\end{itemize}

\textbf{Pre-condizioni}
\begin{itemize}
    \item L'utente ha fatto l'accesso alla dashboard di Analytics
    \item Il filtro temporale è attivo e modificabile.
\end{itemize}

\textbf{Post-condizioni}
\begin{itemize}
    \item La data di fine del periodo di analisi è impostata.
    \item I dati mostrati nella dashboard vengono ricalcolati in base al nuovo intervallo chiuso (Start Date - End Date).
\end{itemize}

\textbf{Scenario principale}
\begin{enumerate}
    \item L'utente interagisce con il selettore di data "Al" (Data fine).
    \item Il sistema mostra un calendario interattivo.
    \item L'utente seleziona il giorno conclusivo dell'analisi.
    \item Il sistema verifica che la data sia successiva o uguale alla data di inizio.
    \item Il sistema avvia l'aggiornamento delle metriche visualizzate (refresh della dashboard).
\end{enumerate}

\textbf{Scenario secondario}
\begin{enumerate}
    \item L'utente seleziona una data antecedente alla data di inizio.
    \item Il sistema impedisce la selezione o mostra un messaggio di errore.
\end{enumerate}

\section{Requisiti}

Questa sezione classifica i requisiti del sistema in quattro categorie principali: funzionali, di qualità, di vincolo e prestazionali.

\subsection{Requisiti Funzionali}
Descrivono i servizi specifici e le funzioni che il sistema deve fornire. Rappresentano il "cosa" il sistema deve fare in risposta a determinati input o situazioni.
\vspace{0.5cm}

\textbf{Caratteristiche}
\begin{enumerate}
    \item Descrivono le interazioni tra l'utente (o altri sistemi) e il software.
    \item Sono diretti: se il requisito non è soddisfatto, il sistema non funziona come previsto.
    \item Derivano direttamente dai casi d'uso e dalle user stories.
\end{enumerate}


\vspace{0.5cm}


\renewcommand{\arraystretch}{1.2}
\setlength{\tabcolsep}{4pt}

\newcommand{\wCod}{2.2cm}
\newcommand{\wFonti}{2.6cm}
\newcommand{\wPrio}{2.8cm}
\newcommand{\wDesc}{%
  \dimexpr\textwidth-\wCod-\wFonti-\wPrio-6\tabcolsep-5\arrayrulewidth\relax
}

\begin{longtable}{|L{\wCod}|L{\wDesc}|L{\wFonti}|L{\wPrio}|}
\caption{Tabella dei Requisiti Funzionali}\label{tab:reqfunzionali}\\
\hline
\textbf{Codice} & \textbf{Descrizione} & \textbf{Fonti} & \textbf{Priorità} \\
\hline
\endfirsthead

\multicolumn{4}{c}{{\bfseries \tablename\ \thetable{} -- continua dalla pagina precedente}}\\
\hline
\textbf{Codice} & \textbf{Descrizione} & \textbf{Fonti} & \textbf{Priorità} \\
\hline
\endhead

\hline
\multicolumn{4}{|r|}{{Continua nella prossima pagina...}}\\
\hline
\endfoot

\hline
\endlastfoot


    RF- & Un utente deve potersi registrare all'interno del sistema & \hyperref[sec:uc-0a]{UC-0A} & Opzionale \\
    \hline
    RF- & L'utente deve poter inserire la propria email & \hyperref[sec:uc-0a-1]{UC-0A.1} & Opzionale \\
    \hline
    RF- & L'utente deve poter inserire password & \hyperref[sec:uc-0a-2]{UC-0A.2} & Opzionale \\
    \hline
    RF- & L'utente deve poter inserire username & \hyperref[sec:uc-0a-3]{UC-0A.3} & Opzionale \\
    \hline
    RF- & L'utente deve poter inserire nome & \hyperref[sec:uc-0a-4]{UC-0A.4} & Opzionale \\
    \hline
    RF- & L'utente deve poter inserire cognome & \hyperref[sec:uc-0a-5]{UC-0A.5} & Opzionale \\
    \hline
    RF- & L'utente deve poter inserire matricola & \hyperref[sec:uc-0a-6]{UC-0A.6} & Opzionale \\
    \hline
    RF- & Deve esserci un controllo per "Email non valida" & \hyperref[sec:uc-0a-7]{UC-0A.7} & Opzionale \\
    \hline
    RF- & Deve esserci un controllo per "Password non valida" & \hyperref[sec:uc-0a-8]{UC-0A.8} & Opzionale \\
    \hline
    RF- & Deve esserci un controllo per "Email già registrata" & \hyperref[sec:uc-0a-9]{UC-0A.9} & Opzionale \\
    \hline
    RF- & Deve esserci un controllo per "Username già registrato" & \hyperref[sec:uc-0a-10]{UC-0A.10} & Opzionale \\
    \hline
    RF- & Deve esserci un controllo per "Matricola già registrata" & \hyperref[sec:uc-0a-11]{UC-0A.11} & Opzionale \\
    \hline
    RF- & Deve esserci un controllo per "Matricola non valida" & \hyperref[sec:uc-0a-12]{UC-0A.12} & Opzionale \\
    \hline
    RF- & L'utente deve avere la possibilità di autenticarsi & \hyperref[sec:uc-0b]{UC-0B} & Opzionale \\
    \hline
    RF- & Deve esserci un controllo per "Email non registrata" & \hyperref[sec:uc-0b-1]{UC-0B.1} & Opzionale \\
    \hline
    RF- & Deve esserci un controllo per "Password errata" & \hyperref[sec:uc-0b-2]{UC-0B.2} & Opzionale \\
    \hline
    RF- & L'utente a seguito del login deve avere la possibilità di raggiungere la dashboard & \hyperref[sec:uc-0c]{UC-0C} & Obbligatorio \\
    \hline
    RF- & L'utente deve avere la possibilità di gestire il proprio profilo utente & \hyperref[sec:uc-0d]{UC-0D} & Opzionale \\
    \hline
    RF- & L'utente deve avere la possibilità di visualizzare la propria email & \hyperref[sec:uc-0d-1]{UC-0D.1} & Opzionale \\
    \hline
    RF- & L'utente deve avere la possibilità di visualizzare la propria password & \hyperref[sec:uc-0d-2]{UC-0D.2} & Opzionale \\
    \hline
    RF- & L'utente deve avere la possibilità di visualizzare il proprio username & \hyperref[sec:uc-0d-3]{UC-0D.3} & Opzionale \\
    \hline
    RF- & L'utente deve avere la possibilità di visualizzare il proprio nome & \hyperref[sec:uc-0d-4]{UC-0D.4} & Opzionale \\
    \hline
    RF- & L'utente deve avere la possibilità di visualizzare il proprio cognome & \hyperref[sec:uc-0d-5]{UC-0D.5} & Opzionale \\
    \hline
    RF- & L'utente deve avere la possibilità di visualizzare la propria matricola & \hyperref[sec:uc-0d-6]{UC-0D.6} & Opzionale \\
    \hline
    RF- & L'utente deve avere la possibilità di modificare le informazioni nel proprio profilo utente & \hyperref[sec:uc-0d-7]{UC-0D.7} & Opzionale \\
    \hline
    RF- & L'utente deve avere la possibilità di non salvare le modifiche al profilo utente & \hyperref[sec:uc-0d-8]{UC-0D.8} & Opzionale \\
    \hline
    RF- & L'utente deve avere la possibilità di salvare le modifiche al profilo utente & \hyperref[sec:uc-0d-9]{UC-0D.9} & Opzionale \\
    \hline
    RF- & Un utente con titolo di admin deve avere la possibilità di gestire i ruoli degli altri utenti & \hyperref[sec:uc-0e]{UC-0E} & Opzionale \\
    \hline
    RF- & Un utente con titolo di admin deve poter visualizzare lista utenti registrati & \hyperref[sec:uc-0e-1]{UC-0E.1} & Opzionale \\
    \hline
    RF- & Un utente con titolo di admin deve poter visualizzare un elemento della lista utenti registrati & \hyperref[sec:uc-0e-2]{UC-0E.2} & Opzionale \\
    \hline
    RF- & Un utente con titolo di admin deve poter visualizzare il ruolo di utente registrato & \hyperref[sec:uc-0e-3]{UC-0E.3} & Opzionale \\
    \hline
    RF- & Un utente con titolo di admin deve poter visualizzare il nome di utente registrato & \hyperref[sec:uc-0e-4]{UC-0E.4} & Opzionale \\
    \hline
    RF- & Un utente con titolo di admin deve poter visualizzare il cognome di utente registrato & \hyperref[sec:uc-0e-5]{UC-0E.5} & Opzionale \\
    \hline
    RF- & Un utente con titolo di admin deve poter modificare il ruolo di utente registrato & \hyperref[sec:uc-0e-6]{UC-0E.6} & Opzionale \\
    \hline
    RF- & Deve esserci un controllo "Utente non autorizzato" quando esso prova ad accedere alla sezione "Gestione ruoli utenti" & \hyperref[sec:uc-0e-7]{UC-0E.7} & Opzionale \\
    \hline
    RF- & Un utente con titolo di admin deve poter annullare la modifica al ruolo utente registrato & \hyperref[sec:uc-0e-8]{UC-0E.8} & Opzionale \\
    \hline
    RF- & Un utente con titolo di admin deve poter salvare le modifiche al ruolo di un utente registrato & \hyperref[sec:uc-0e-9]{UC-0E.9} & Opzionale \\
    \hline
    RF- & Un utente deve aver la possibilità di uscire al proprio profilo (Logout) & \hyperref[sec:uc-0f]{UC-0F} & Opzionale \\
    \hline
    RF- & Un utente deve avere la possibilità di accedere al modulo di generazione contenuti AI & \hyperref[sec:uc-1a]{UC-1A} & Obbligatorio \\
    \hline
    RF- & Un utente deve avere la possibilità di inserire un prompt per la propria generazione & \hyperref[sec:uc-1a-1]{UC-1A.1} & Obbligatorio \\
    \hline
    RF- & Un utente deve avere la possibilità di selezionare un tono per la propria generazione & \hyperref[sec:uc-1a-2]{UC-1A.2} & Obbligatorio \\
    \hline
    RF- & Un utente deve avere la possibilità di selezionare uno stile per la propria generazione & \hyperref[sec:uc-1a-3]{UC-1A.3} & Obbligatorio \\
    \hline
    RF- & Un utente deve avere la possibilità di aprire lo storico dei prompt & \hyperref[sec:uc-1b]{UC-1B} & Obbligatorio \\
    \hline
    RF- & Un utente deve avere la possibilità di visualizzare la lista degli elementi presenti nello storico & \hyperref[sec:uc-1b-1]{UC-1B.1} & Obbligatorio \\
    \hline
    RF- & Un utente deve avere la possibilità di visualizzare un elemento dalla lista mostrata nello storico & \hyperref[sec:uc-1b-2]{UC-1B.2} & Obbligatorio \\
    \hline
    RF- & Un utente deve avere la possibilità di visualizzare lo stile di una generazione & \hyperref[sec:uc-1b-3]{UC-1B.3} & Obbligatorio \\
    \hline
    RF- & Un utente deve avere la possibilità di visualizzare il risultato della generazione & \hyperref[sec:uc-1b-4]{UC-1B.4} & Obbligatorio \\
    \hline
    RF- & Un utente deve avere la possibilità di visualizzare il timestamp di una generazione & \hyperref[sec:uc-1b-5]{UC-1B.5} & Obbligatorio \\
    \hline
    RF- & Un utente deve avere la possibilità di visualizzare la valutazione di una generazione & \hyperref[sec:uc-1b-6]{UC-1B.6} & Obbligatorio \\
    \hline
    RF- & Un utente deve avere la possibilità di visualizzare il prompt id una generazione & \hyperref[sec:uc-1b-7]{UC-1B.7} & Obbligatorio \\
    \hline
    RF- & Un utente deve avere la possibilità di visualizzare il tono di una generazione & \hyperref[sec:uc-1b-8]{UC-1B.8} & Obbligatorio \\
    \hline
    RF- & Un utente deve avere la possibilità di duplicare una generazione & \hyperref[sec:uc-1f]{UC-1F} & Obbligatorio \\
    \hline
    RF- & C'è la possibilità che il prompt non abbia neanche un elemento data la mancanza di generazioni & \hyperref[sec:uc-1b-1]{UC-1B.1} & Obbligatorio \\
    \hline
    RF- & Un utente deve avere la possibilità di riutilizzare una generazione & \hyperref[sec:uc-1e]{UC-1E} & Obbligatorio \\
    \hline
    RF- & Un utente deve avere la possibilità applicare un filtro alla lista dello storico & \hyperref[sec:uc-1g]{UC-1G} & Obbligatorio \\
    \hline
    RF- & Un utente deve avere la possibilità di visualizzare la lista dopo aver applicato il filtro & \hyperref[sec:uc-1g-1]{UC-1G.1} & Obbligatorio \\
    \hline
    RF- & Un utente deve avere la possibilità di generare un contenuto tramite AI & \hyperref[sec:uc-1c]{UC-1C} & Obbligatorio \\
    \hline
    RF- & Un utente deve avere la possibilità di visualizzare un contenuto prima di pubblicarlo & \hyperref[sec:uc-1c-1]{UC-1C.1} & Obbligatorio \\
    \hline
    RF- & Un utente deve avere la possibilità di rigenerare un contenuto tramite AI & \hyperref[sec:uc-1h]{UC-1H} & Obbligatorio \\
    \hline
    RF- & Deve avvenire il salvataggio di ogni generazione nello storico & \hyperref[sec:uc-1c-3]{UC-1C.3} & Obbligatorio \\
    \hline
    RF- & Deve avvenire una valutazione di ogni contenuto generato & \hyperref[sec:uc-1i]{UC-1i} & Obbligatorio \\
    \hline
    RF- & Un utente deve avere la possibilità di scartare un contenuto generato & \hyperref[sec:uc-1j]{UC-1J} & Obbligatorio \\
    \hline
    RF- & Un utente deve avere la possibilità di effettuare modifiche al contenuto generato & \hyperref[sec:uc-1d]{UC-1D} & Obbligatorio \\
    \hline
    RF- & Un utente deve avere la possibilità di modificare l'immagine di un contenuto generato & \hyperref[sec:uc-1d-1]{UC-1D.1} & Obbligatorio \\
    \hline
    RF- & Un utente deve avere la possibilità di modificare il titolo di un contenuto generato & \hyperref[sec:uc-1d-2]{UC-1D.2} & Obbligatorio \\
    \hline
    RF- & Un utente deve avere la possibilità di modificare il testo di un contenuto generato & \hyperref[sec:uc-1d-3]{UC-1D.3} & Obbligatorio \\
    \hline
    RF- & Un utente deve avere la possibilità di salvare le modifiche effettuate a un contenuto generato & \hyperref[sec:uc-1d-4]{UC-1D.4} & Obbligatorio \\
    \hline
    RF- & Un utente deve avere la possibilità di annullare le modifiche effettuate a un contenuto generato & \hyperref[sec:uc-1d-5]{UC-1D.5} & Obbligatorio \\
    \hline
    RF- & Deve esserci un controllo per "File immagine non valido" & \hyperref[sec:uc-1d-6]{UC-1D.6} & Obbligatorio \\
    \hline
    RF- & Un utente deve avere la possibilità di accedere al modulo di upload e gestione documentale. & \hyperref[sec:uc-2a]{UC-2A} & Obbligatorio \\
    \hline
    RF- & Un utente deve avere la possibilità di caricare un file all'interno del modulo di upload e gestione documentale & \hyperref[sec:uc-2a-1]{UC-2A.1} & Obbligatorio \\
    \hline
    RF- & Un utente deve avere la possibilità di inserire la categoria del documento all'interno del modulo di upload e gestione documentale & \hyperref[sec:uc-2a-2]{UC-2A.2} & Obbligatorio \\
    \hline
    RF- & Un utente deve avere la possibilità di inserire il mese/anno di competenza di un documento all'interno del modulo di upload e gestione documentale & \hyperref[sec:uc-2a-3]{UC-2A.3} & Obbligatorio \\
    \hline
    RF- & Un utente deve avere la possibilità di inserire l'azienda di appartenenza del documento all'interno del modulo di upload e gestione documentale & \hyperref[sec:uc-2a-4]{UC-2A.4} & Obbligatorio \\
    \hline
    RF- & Un utente deve avere la possibilità di inserire il reparto di appartenenza del documento all'interno del modulo di upload e gestione documentale & \hyperref[sec:uc-2a-5]{UC-2A.5} & Obbligatorio \\
    \hline
    RF- & Un utente deve avere la possibilità di visualizzare la lista delle informazioni dei documenti partizionati dall'AI in seguito all'upload & \hyperref[sec:uc-2b-1]{UC-2B.1} & Obbligatorio \\
    \hline
    RF- & Un utente deve avere la possibilità di visualizzare un elemento dalla lista delle informazioni dei documenti partizionati dall'AI in seguito all'upload & \hyperref[sec:uc-2b-2]{UC-2B.2} & Obbligatorio \\
    \hline
    RF- & Un utente deve avere la possibilità di visualizzare il campo di competenza di un documento & \hyperref[sec:uc-2b-3]{UC-2B.3} & Obbligatorio \\
    \hline
    RF- & Un utente deve avere la possibilità di visualizzare l'azienda di appartenenza di un documento & \hyperref[sec:uc-2b-4]{UC-2B.4} & Obbligatorio \\
    \hline
    RF- & Un utente deve avere la possibilità di visualizzare la causale di un documento & \hyperref[sec:uc-2b-5]{UC-2B.5} & Obbligatorio \\
    \hline
    RF- & Un utente deve avere la possibilità di visualizzare la lingua di un documento & \hyperref[sec:uc-2b-6]{UC-2B.6} & Obbligatorio \\
    \hline
    RF- & Un utente deve avere la possibilità di visualizzare il numero di pagine documento & \hyperref[sec:uc-2b-7]{UC-2B.7} & Obbligatorio \\
    \hline
    RF- & Un utente deve avere la possibilità di visualizzare il nome del documento originale & \hyperref[sec:uc-2b-8]{UC-2B.8} & Obbligatorio \\
    \hline
    RF- & Un utente deve avere la possibilità di visualizzare la data i redazione documento & \hyperref[sec:uc-2b-9]{UC-2B.9} & Obbligatorio \\
    \hline
    RF- & Un utente deve avere la possibilità di visualizzare il codice di un documento & \hyperref[sec:uc-2b-10]{UC-2B.10} & Obbligatorio \\
    \hline
    RF- & Un utente deve avere la possibilità di visualizzare la tipologia di un documento & \hyperref[sec:uc-2b-11]{UC-2B.11} & Obbligatorio \\
    \hline
    RF- & Un utente deve avere la possibilità di filtrare la lista dei documenti & \hyperref[sec:uc-2b-12]{UC-2B.12} & Obbligatorio \\
    \hline
    RF- & Deve esserci un controllo per "Nessun documento riconosciuto" & \hyperref[sec:uc-2b-1]{UC-2B.1} & Obbligatorio \\
    \hline
    RF- & Un utente deve avere la possibilità di visualizzare la lista filtrata delle informazioni dei documenti partizionati dall'AI in seguito all'upload & \hyperref[sec:uc-2b-14]{UC-2B.14} & Obbligatorio \\
    \hline
    RF- & Un utente deve avere la possibilità di aprire il modulo di gestione del documento & \hyperref[sec:uc-2c]{UC-2C} & Obbligatorio \\
    \hline
    RF- & Un utente deve avere la possibilità di visualizzare l'anteprima di un documento prima di inviarlo & \hyperref[sec:uc-2c-1]{UC-2C.1} & Obbligatorio \\
    \hline
    RF- & Un utente deve avere la possibilità di modificare il destinatario a un documento & \hyperref[sec:uc-2c-2]{UC-2C.2} & Obbligatorio \\
    \hline
    RF- & Un utente deve avere la possibilità di modificare la tipologia a un documento & \hyperref[sec:uc-2c-3]{UC-2C.3} & Obbligatorio \\
    \hline
    RF- & Un utente deve avere la possibilità di rivalutare la percentuale di confidenza di un documento& \hyperref[sec:uc-2c-4]{UC-2C.4} & Obbligatorio \\
    \hline
    RF- & Un utente deve avere la possibilità di salvare le modifiche apportate a un documento & \hyperref[sec:uc-2c-5]{UC-2C.5} & Obbligatorio \\
    \hline
    RF- & Un utente deve avere la possibilità di aprire il modulo riguardante le informazioni dei destinatari & \hyperref[sec:uc-2d]{UC-2D} & Obbligatorio \\
    \hline
    RF- & Un utente deve avere la possibilità di visualizzare la lista informazioni destinatari & \hyperref[sec:uc-2d]{UC-2D} & Obbligatorio \\
    \hline
    RF- & Un utente deve avere la possibilità di visualizzare un elemento dalla lista informazioni destinatari & \hyperref[sec:uc-2d-1]{UC-2D.1} & Obbligatorio \\
    \hline
    RF- & Un utente deve avere la possibilità di visualizzare la matricola di un destinatario & \hyperref[sec:uc-2d-3]{UC-2D.3} & Obbligatorio \\
    \hline
    RF- & Un utente deve avere la possibilità di visualizzare il reparto di un destinatario & \hyperref[sec:uc-2d-4]{UC-2D.4} & Obbligatorio \\
    \hline
    RF- & Un utente deve avere la possibilità di visualizzare il nome di un destinatario & \hyperref[sec:uc-2d-5]{UC-2D.5} & Obbligatorio \\
    \hline
    RF- & Deve esserci un controllo per "Nessun destinatario riconosciuto" & \hyperref[sec:uc-2b-6]{UC-2B.6} & Obbligatorio \\
    \hline
    RF- & Un utente deve avere la possibilità di visualizzare il codice fiscale di un destinatario & \hyperref[sec:uc-2d-2]{UC-2D.2} & Obbligatorio \\
    \hline
    RF- & Un utente deve avere la possibilità di filtrare la lista dei destinatari & \hyperref[sec:uc-2d-7]{UC-2D.7} & Obbligatorio \\
    \hline
    RF- & Un utente deve avere la possibilità di visualizzare la lista filtrata dei destinatari & \hyperref[sec:uc-2d-8]{UC-2D.8} & Obbligatorio \\
    \hline
    RF- & Un utente deve avere la possibilitá di accedere allo storico dei documenti esaminati & \hyperref[sec:uc-2e]{UC-2E} & Obbligatorio \\
    \hline
    RF- & Un utente deve avere la possibilitá di selezionare un documento per poterlo poi inviare & \hyperref[sec:uc-2e-1]{UC-2E.1} & Obbligatorio \\
    \hline
    RF- & C'è la possibilitá che non siano mai stati registrati documenti e che quindi l'applicativo agisca di conseguenza & \hyperref[sec:uc-2e-2]{UC-2E.1} & Obbligatorio \\
    \hline
    RF- & Un utente deve avere la possibilità di visualizzare la lista storico documenti & \hyperref[sec:uc-2e-4]{UC-2E} & Obbligatorio \\
    \hline
    RF- & Un utente deve avere la possibilità di visualizzare un elemento dalla lista storico documenti & \hyperref[sec:uc-2e-5]{UC-2E.4} & Obbligatorio \\
    \hline
    RF- & Un utente deve avere la possibilità di visualizzare lo stato da un elemento& \hyperref[sec:uc-2e-3]{UC-2E.5} & Obbligatorio \\
    \hline
    RF- & Un utente deve avere la possibilità di visualizzare la percentuale di confidenza di un elemento & \hyperref[sec:uc-2e-6]{UC-2E.3} & Obbligatorio \\
    \hline
    RF- & Un utente deve avere la possibilità di visualizzare la lista di appartenenza di un elemento & \hyperref[sec:uc-2e-7]{UC-2E.2} & Obbligatorio \\
    \hline
    RF- & Un utente deve avere la possibilitá di filtrare i documenti per favorire la ricerca & \hyperref[sec:uc-2e-8]{UC-2E.8} & Obbligatorio \\
    \hline
    RF- & Un utente deve avere la possibilità di visualizzare la lista filtrata storico documenti & \hyperref[sec:uc-2e-9]{UC-2K.1} & Obbligatorio \\
    \hline
    RF- & Un utente deve avere la possibilità di accedere a un menú di gestione del messaggio da inviare con i documenti prima di inviarlo & \hyperref[sec:uc-2f]{UC-2F} & Obbligatorio \\
    \hline
    RF- & Un utente deve avere la possibilità di salvare il template di un messaaggio in uno storico per poterlo riutilizzare in futuro & \hyperref[sec:uc-2f-1]{UC-2F.1} & Obbligatorio \\
    \hline
    RF- & Un utente deve avere la possibilità di modificare l'oggetto di un messaggio & \hyperref[sec:uc-2f-2]{UC-2F.2} & Obbligatorio \\
    \hline
    RF- & Un utente deve avere la possibilità di modificare il testo di messaggio & \hyperref[sec:uc-2f-3]{UC-2F.3} & Obbligatorio \\
    \hline
    RF- & Un utente deve avere la possibilità di confermare le modifiche apportate ad un messaggio & \hyperref[sec:uc-2f-4]{UC-2F.4} & Obbligatorio \\
    \hline
    RF- & Un utente deve avere la possibilità di generare un messaggio tramite AI senza doverlo scrivere a mano & \hyperref[sec:uc-2f-5]{UC-2F.5} & Obbligatorio \\
    \hline
    RF- & Un utente deve avere la possibilità di visualizzare l'oggetto di un messaggio & \hyperref[sec:uc-2f-6]{UC-2F.6} & Obbligatorio \\
    \hline
    RF- & Un utente deve avere la possibilità di visualizzare il testo di messaggio & \hyperref[sec:uc-2f-7]{UC-2F.7} & Obbligatorio \\
    \hline
    RF- & Un utente deve avere la possibilità di accedere alla pagina dei template salvati & \hyperref[sec:uc-2g]{UC-2G} & Obbligatorio \\
    \hline
    RF- & Un utente deve avere la possibilità di eliminare template dalla tabella dei template salvati& \hyperref[sec:uc-2g-1]{UC-2G.1} & Obbligatorio \\
    \hline
    RF- & Il sistema non deve mostrare niente se nessun template è stato salvato all'interno dello storico & \hyperref[sec:uc-2g-2]{UC-2G.2} & Obbligatorio \\
    \hline
    RF- & Un utente deve avere la possibilità di caricare un template per riutilizzarlo & \hyperref[sec:uc-2g-3]{UC-2G.3} & Obbligatorio \\
    \hline
    RF- & Un utente deve avere la possibilità di visualizzare la lista dei template salvati & \hyperref[sec:uc-2g-4]{UC-2G.4} & Obbligatorio \\
    \hline
    RF- & Un utente deve avere la possibilità di visualizzare un elemento dall lista template salvati & \hyperref[sec:uc-2g-5]{UC-2G.5} & Obbligatorio \\
    \hline
    RF- & Un utente deve avere la possibilità di inviare un documento con il relativo messaggio & \hyperref[sec:uc-2h]{UC-2H} & Obbligatorio \\
    \hline
    RF- & Un utente deve avere la possibilità di visualizzare allegare un file con il documento che sta inviando & \hyperref[sec:uc-2h-1]{UC-2H.1} & Obbligatorio \\
    \hline
    RF- & Un utente deve avere la possibilità di pianificare l'invio di un documento & \hyperref[sec:uc-2h-2]{UC-2H.2} & Obbligatorio \\
    \hline
    RF- & Un utente deve avere la possibilità di dare conferma all'invio di un documento & \hyperref[sec:uc-2h-3]{UC-2H.3} & Obbligatorio \\
    \hline
    RF- & Un utente deve avere la possibilità di accedere alla dashboard di analisi e Monitoraggio AI & \hyperref[sec:uc-3a]{UC-3A} & Obbligatorio \\
    \hline
    RF- & Un utente deve avere la possibilità di visualizzare l'elenco dati AI assistant & \hyperref[sec:uc-3a-1]{UC-3A.1} & Obbligatorio \\
    \hline
    RF- & Un utente deve avere la possibilità di visualizzare il numero di prompt generati & \hyperref[sec:uc-3a-2]{UC-3A.2} & Obbligatorio \\
    \hline
    RF- & Un utente deve avere la possibilità di visualizzare il rating medio prompt generati & \hyperref[sec:uc-3a-3]{UC-3A.3} & Obbligatorio \\
    \hline
    RF- & Un utente deve avere la possibilità di visualizzare il numero rigenerazioni prompt & \hyperref[sec:uc-3a-4]{UC-3A.4} & Obbligatorio \\
    \hline
    RF- & Un utente deve avere la possibilità di visualizzare i toni più usati & \hyperref[sec:uc-3a-5]{UC-3A.5} & Obbligatorio \\
    \hline
    RF- & Un utente deve avere la possibilità di visualizzare gli stili più usati & \hyperref[sec:uc-3a-6]{UC-3A.6} & Obbligatorio \\
    \hline
    RF-  & Un utente deve avere la possibilità di visualizzare l'elenco dati AI Co-pilot & \hyperref[sec:uc-3a-7]{UC-3A.7} & Obbligatorio \\
    \hline
    RF-  & Un utente deve avere la possibilità di visualizzare la confidenza media & \hyperref[sec:uc-3a-8]{UC-3A.8} & Obbligatorio \\
    \hline
    RF-  & Un utente deve avere la possibilità di visualizzare la percentuale interventi manuali & \hyperref[sec:uc-3a-9]{UC-3A.9} & Obbligatorio \\
    \hline
    RF-  & Un utente deve avere la possibilità di visualizzare l'accuratezza mapping & \hyperref[sec:uc-3a-10]{UC-3A.10} & Obbligatorio \\
    \hline
    RF-  & Un utente deve avere la possibilità di visualizzare i tempi medi analisi & \hyperref[sec:uc-3a-11]{UC-3A.11} & Obbligatorio \\
    \hline
    RF-  & Un utente deve avere la possibilità di visualizzare inserire la data d'inizio delle statistiche & \hyperref[sec:uc-3a-12]{UC-3A.12} & Obbligatorio \\
    \hline
    RF-  & Un utente deve avere la possibilità di visualizzare inserire la data di fine delle statistiche & \hyperref[sec:uc-3a-13]{UC-3A.13} & Obbligatorio \\
    \hline

\end{longtable}


\subsection{Requisiti di Qualità}
Definiscono gli attributi qualitativi del software che influenzano l'esperienza d'uso e la manutenibilità del progetto. Spesso indicati come "attributi di qualità" (es. usabilità, affidabilità, manutenibilità).
\vspace{0.5cm}

\textbf{Caratteristiche}
\begin{enumerate}
    \item Determinano il livello di soddisfazione dell'utente e la facilità di evoluzione del software.
    \item Devono essere misurabili tramite metriche specifiche o feedback utente.
\end{enumerate}

\vspace{0.5cm}
\begin{longtable}{|L{\wCod}|L{\wDesc}|L{\wFonti}|L{\wPrio}|}
\caption{Tabella dei Requisiti di Qualità}\label{tab:reqqualita}\\
\hline
\textbf{Codice} & \textbf{Descrizione} & \textbf{Fonti} & \textbf{Priorità} \\
\hline
\endfirsthead

\multicolumn{4}{c}{{\bfseries \tablename\ \thetable{} -- continua dalla pagina precedente}}\\
\hline
\textbf{Codice} & \textbf{Descrizione} & \textbf{Fonti} & \textbf{Priorità} \\
\hline
\endhead

\hline
\multicolumn{4}{|r|}{{Continua nella prossima pagina...}}\\
\hline
\endfoot

\hline
\endlastfoot


    RQ-01 & Presentare documento Analisi dei Requisiti contenente diagrammi e descrizioni Use Case & Capitolato & Obbligatorio \\
    \hline
    RQ-02 & Il way of working descritto in Norme di Progetto deve essere rispettato & Interna & Obbligatorio \\
    \hline
    RQ-03 & Il Prodotto deve passare tutti i test con la copertura concordata con la proponente & Capitolato, Piano di Qualifica & Obbligatorio \\
    \hline
    RQ-04 & Il codice deve essere documentato secondo le linee guida descritte in Norme di Progetto & Capitolato, Interna & Obbligatorio \\
    \hline
    RQ-05 & È necessario versionare il codice con appositi strumenti di controllo versione, compreso di istruzioni di setup & Capitolato & Obbligatorio \\
    \hline
    RQ-06 & Report finale di integrazione e suggerimenti di evoluzione & Capitolato & Obbligatorio \\
    \hline

\end{longtable}

\subsection{Requisiti di Vincolo}
Rappresentano le limitazioni e le restrizioni entro cui il sistema deve essere sviluppato o operare. Questi vincoli restringono lo spazio delle soluzioni possibili.
\vspace{0.5cm}

\textbf{Caratteristiche}
\begin{enumerate}
    \item Possono essere di natura tecnologica (hardware, linguaggio), normativa (GDPR) o di business (budget).
    \item Sono mandatori e non negoziabili.
\end{enumerate}

\vspace{0.5cm}
\begin{longtable}{|L{\wCod}|L{\wDesc}|L{\wFonti}|L{\wPrio}|}
\caption{Tabella dei Requisiti Vincolo}\label{tab:reqvincolo}\\
\hline
\textbf{Codice} & \textbf{Descrizione} & \textbf{Fonti} & \textbf{Priorità} \\
\hline
\endfirsthead

\multicolumn{4}{c}{{\bfseries \tablename\ \thetable{} -- continua dalla pagina precedente}}\\
\hline
\textbf{Codice} & \textbf{Descrizione} & \textbf{Fonti} & \textbf{Priorità} \\
\hline
\endhead

\hline
\multicolumn{4}{|r|}{{Continua nella prossima pagina...}}\\
\hline
\endfoot

\hline
\endlastfoot


    RV-01 & Git come sistema di controllo versione & Capitolato & Obbligatorio \\
    \hline
    RV-02 & API \& Backend devono essere sviluppati in Ruby on Rails & Capitolato & Obbligatorio \\
    \hline
    RV-03 & Il database deve essere PostgreSQL & Capitolato & Obbligatorio \\
    \hline
    RV-04 & Il frontend deve essere sviluppato in Angular & Capitolato & Obbligatorio \\
    \hline
    RV-05 & La gestione dei modelli AI deve essere implementata utilizzando AWS Bedrock & Capitolato & Obbligatorio \\
    \hline
    RV-06 & Eventuali background jobs gestiti con Sidekiq e PWA con Next.js & Capitolato, trattativa con la proponente & Opzionale \\
    \hline

\end{longtable}

\subsection{Requisiti Prestazionali}
Specificano i parametri numerici relativi all'efficienza del sistema. Sebbene siano tecnicamente un sottoinsieme della qualità, vengono trattati separatamente per la loro criticità e misurabilità quantitativa.
\vspace{0.5cm}

\textbf{Caratteristiche}
\begin{enumerate}
    \item Definiscono limiti su tempi di risposta, throughput e utilizzo delle risorse.
    \item Sono sempre espressi con valori numerici e soglie precise.
\end{enumerate}

\vspace{0.5cm}
\begin{longtable}{|L{\wCod}|L{\wDesc}|L{\wFonti}|L{\wPrio}|}
\caption{Tabella dei Requisiti Prestazionali}\label{tab:reqprestazionali}\\
\hline
\textbf{Codice} & \textbf{Descrizione} & \textbf{Fonti} & \textbf{Priorità} \\
\hline
\endfirsthead

\multicolumn{4}{c}{{\bfseries \tablename\ \thetable{} -- continua dalla pagina precedente}}\\
\hline
\textbf{Codice} & \textbf{Descrizione} & \textbf{Fonti} & \textbf{Priorità} \\
\hline
\endhead

\hline
\multicolumn{4}{|r|}{{Continua nella prossima pagina...}}\\
\hline
\endfoot

\hline
\endlastfoot


    RP-01 & Il sistema deve generare contenuti testuali tramite AI (modulo Assistant) entro 5 secondi per testi fino a 500 parole & Interna & Obbligatorio \\
    \hline
    RP-02 & Il sistema deve classificare e partizionare documenti PDF (modulo Co-Pilot) entro 3 secondi per pagina & Interna & Obbligatorio \\
    \hline
    RP-03 & Il tempo di risposta dell'interfaccia utente per operazioni standard deve essere inferiore a 2 secondi & Interna & Obbligatorio \\
    \hline
    RP-04 & Il sistema deve supportare l'upload di file PDF fino a 20 MB & Interna & Obbligatorio \\
    \hline
    RP-05 & La dashboard di Analytics deve caricare le statistiche entro 3 secondi per dataset fino a 1000 documenti & Interna & Desiderabile \\
    \hline
    RP-06 & Il sistema deve garantire una disponibilità del 99\% durante l'orario lavorativo (8:00-18:00) & Interna & Desiderabile \\
    \hline
    RP-07 & Il sistema deve essere in grado di processare almeno 50 documenti in parallelo senza degrado delle performance & Interna & Desiderabile \\
    \hline
    RP-08 & Il tempo di estrazione OCR per documenti scansionati deve essere inferiore a 5 secondi per pagina & Interna & Obbligatorio \\
    \hline

\end{longtable}


\end{document}