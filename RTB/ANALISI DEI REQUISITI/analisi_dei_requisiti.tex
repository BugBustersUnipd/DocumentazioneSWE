\documentclass[a4paper,11pt]{article}

\usepackage[utf8]{inputenc}
\usepackage[T1]{fontenc}
\usepackage[italian]{babel}
\usepackage[margin=2.5cm]{geometry}
\usepackage{graphicx}
\usepackage{booktabs}
\usepackage{setspace}
\usepackage{titlesec}
\usepackage{float}
\usepackage[table]{xcolor}
\usepackage{tabularx}
\usepackage{tcolorbox}
\usepackage{enumitem}
\usepackage[titles]{tocloft}
\usepackage[colorlinks=true,linkcolor=black,urlcolor=blue,citecolor=blue]{hyperref}
\usepackage{fancyhdr}
\usepackage{lastpage}
\usepackage{amsmath}

\pagestyle{fancy}
\fancyhf{}
\fancyhead[L]{BugBusters}
\fancyhead[R]{Analisi dei Requisiti}
\fancyfoot[L]{\thepage\ di \pageref{LastPage}}
\renewcommand{\headrulewidth}{0pt}
\renewcommand{\footrulewidth}{0pt}

\setlength{\headheight}{14pt}

\setlength{\parskip}{4pt}
\setlength{\parindent}{0pt}

\titleformat{\section}{\large\bfseries}{\thesection}{1em}{}
\titleformat{\subsection}{\normalsize\bfseries}{\thesubsection}{1em}{}

\begin{document}

\begin{center}
  \thispagestyle{empty}
  \IfFileExists{../../assets/Logo.jpg}{%
    \includegraphics[width=6cm,height=3cm,keepaspectratio]{../../assets/Logo.jpg} \\[0.8cm]
  }{%
    \fbox{\parbox[c][2.5cm][c]{6cm}{\centering Logo non trovato\\(Logo.jpg)}}\\[0.5cm]
  }
  {\LARGE\bfseries BugBusters}\\[0.8cm]
  
  \rule{\textwidth}{0.5pt}\\[0.5cm]
  {\Large\bfseries Analisi dei Requisiti}\\[0.3cm]
  {\large Versione 0.0.1}\\[0.5cm]
  \rule{\textwidth}{0.5pt}\\[0.8cm]
\end{center}

\begin{center}
\begin{tcolorbox}[colback=gray!10,width=0.8\textwidth,arc=3mm,boxrule=0.5pt]
\begin{tabular}{ll}
\textbf{Stato} & In redazione \\
\textbf{Redattori} & ----- \\
\textbf{Destinatari} & BugBusters \\
 & Prof. Vardanega Tullio \\
 & Prof. Cardin Riccardo \\
 & Eggon \\
\end{tabular}
\end{tcolorbox}
\end{center}

\vspace{1cm}

\begin{center}
\textbf{Descrizione}
\end{center}

\begin{center}
\begin{minipage}{0.9\textwidth}
\small
Questo documento contiene le Norme di Progetto seguite dal team \textbf{BugBusters} per il progetto\textsubscript{\scalebox{0.6}{\textbf{G}}} C5 proposto dall'azienda Eggon
\end{minipage}
\end{center}

\newpage

\section*{Registro delle Modifiche}

{\footnotesize
\begin{center}
\begin{tabularx}{\textwidth}{|l|l|X|l|l|}
\hline
\textbf{Versione} & \textbf{Data} & \textbf{Descrizione} & \textbf{Autore} & \textbf{Ruolo} \\
\hline
0.0.8 & 05/12/2025 & 
\begin{minipage}[t]{\linewidth}
Riscrittura casi d'uso con aggiornato grado di precisione (sezione 2).
\end{minipage} 
& Leonardo Salviato & - \\
\hline
0.0.7 & 04/12/2025 & 
\begin{minipage}[t]{\linewidth}
Riscrittura casi d'uso con aggiornato grado di precisione (sezioni 0 e 1).
\end{minipage} 
& Leonardo Salviato & - \\
\hline
0.0.6 & 03/12/2025 & 
\begin{minipage}[t]{\linewidth}
Aggiunti casi d'uso sezione 2, aggiunta varianti/exceptions sezione 2, scritta possibile struttura requisiti.
\end{minipage} 
& Leonardo Salviato & - \\
\hline
0.0.5 & 02/12/2025 & 
\begin{minipage}[t]{\linewidth}
Aggiunto schema attori
\end{minipage} 
& Marco Piro & - \\
\hline
0.0.4 & 30/11/2025 & 
\begin{minipage}[t]{\linewidth}
    Sistemazione Attori.
\end{minipage} 
& Marco Piro & - \\
\hline
0.0.3 & 29/11/2025 & 
\begin{minipage}[t]{\linewidth}
Correzione casi d'uso e aggiunta schemi.
\end{minipage} 
& Leonardo Salviato & - \\
\hline
0.0.2 & 25/11/2025 & 
\begin{minipage}[t]{\linewidth}
Riscrittura della prima stesura e modifica casi d'uso.
\end{minipage} 
& Leonardo Salviato & - \\
\hline
0.0.1 & 16/11/2025 & 
\begin{minipage}[t]{\linewidth}
Prima stesura della struttura del documento.
\end{minipage} 
& Leonardo Salviato e Marco Piro & - \\
\hline
\end{tabularx}
\end{center}
}

\vfill
\begin{center}
2 di \pageref{LastPage}
\end{center}

\newpage

\section*{Indice}

\noindent
\begin{minipage}[t]{0.8\textwidth}
\subsection*{1 Introduzione}
1.1 Scopo del documento \\
1.2 Prospettiva del prodotto \\
1.3 Funzioni del prodotto \\
1.4 Caratterisitiche dell'utente \\
1.5 Definizioni, acronimi e abbreviazioni \\
1.6 Riferimenti \\
\quad 1.6.1 Riferimenti normativi \\
\quad 1.6.2 Riferimenti informativi \\
\subsection*{2 Casi d'uso}
2.1 Introduzione \\
2.2 Attori \\
2.3 Lista casi d'uso \\
2.4 Sezione 0 – Applicazione standalone\\
\quad 2.4.1 UC-0A: \\
\quad 2.4.2 UC-0B: \\
\quad 2.4.3 UC-0C: \\
\quad 2.4.4 UC-0D: \\
\quad 2.4.5 UC-0E: \\
2.5 Sezione 1 – Modulo “AI Assistant Generativo\\
\quad 2.5.1 UC-1A: \\
\quad 2.5.2 UC-1B: \\
\quad 2.5.3 UC-1C: \\
\quad 2.5.4 UC-1D: \\
\quad 2.5.5 UC-1E: \\
\quad 2.5.6 UC-1F: \\
\quad 2.5.7 UC-1G: \\
\quad 2.5.8 UC-1H: \\


2.6 Sezione 2 – Modulo "AI Co-Pilot per i CdL"\\
\quad 2.6.1 UC-2A: \\
\quad 2.6.2 UC-2B: \\
\quad 2.6.3 UC-2C: \\
\quad 2.6.4 UC-2D: \\
\quad 2.6.5 UC-2E: \\
\quad 2.6.5 UC-2F: \\
\quad 2.6.5 UC-2G: \\




\end{minipage}
\begin{minipage}[t]{0.2\textwidth}
\vspace{1.65\baselineskip}
9 \\
9 \\
9 \\
10 \\
10 \\
10 \\
\end{minipage}

\newpage

\section{Introduzione}

\subsection{Scopo del documento}
Questo documento di Analisi dei Requisiti\textsubscript{\scalebox{0.6}{\textbf{G}}}, adottato da parte di BugBusters durante lo svolgimento del progetto\textsubscript{\scalebox{0.6}{\textbf{G}}} didattico, ha lo scopo di definire in maniera precisa e dettagliata i requisiti funzionali\textsubscript{\scalebox{0.6}{\textbf{G}}} e non funzionali del Sistema software da sviluppare.

A seguito delle nuove decisioni progettuali rispetto alle proposte del capitolato, il Sistema non sarà inizialmente integrato nella piattaforma NEXUM, ma verrà realizzato come \textbf{applicazione standalone}, autonoma e indipendente. Tale applicazione implementerà i moduli “AI Assistant Generativo” e "AI Co-Pilot per i CdL" in un ambiente isolato, così da consentire una fase di sviluppo, test e validazione più controllata. Solo in una fase successiva si valuterà l’\textbf{integrazione con la piattaforma NEXUM}, garantendo continuità architetturale e coerenza con i moduli già presenti.

Il documento include una descrizione approfondita dei Casi d’Uso, che costituiscono la principale fonte dei requisiti finali. Per agevolare la comprensione, verranno utilizzati anche i \textbf{Diagrammi dei Casi d’Uso}, che visualizzano le interazioni tra utenti e Sistema.

Questo documento rappresenta il riferimento fondamentale per la progettazione, l’implementazione e il collaudo dell’applicazione standalone, assicurando che essa soddisfi pienamente le esigenze del Committente e gli obiettivi formativi del progetto.

I requisiti identificati sono classificati nelle seguenti categorie:
\begin{itemize}
    \item \textbf{Obbligatori}: necessari e imprescindibili per garantire il corretto funzionamento dell’applicazione standalone;
    \item \textbf{Desiderabili}: non strettamente necessari, ma capaci di migliorare l’esperienza utente o l’efficienza del Sistema;
    \item \textbf{Opzionali}: funzionalità aggiuntive utili per estensioni future, in particolare in vista della possibile integrazione con NEXUM.
\end{itemize}

Il documento è rivolto ai seguenti destinatari:
\begin{itemize}
    \item Il \textbf{Committente}, che potrà verificare che i requisiti siano stati compresi e documentati correttamente;
    \item Il \textbf{Team di Progettisti e Programmatori}, che utilizzerà questa analisi come base per la realizzazione del Sistema;
    \item Il \textbf{Team di Verificatori}, che impiegherà il presente documento per definire i casi di Test e validare il comportamento del prodotto.
\end{itemize}

\subsection{Prospettiva del prodotto}

Il prodotto che BugBusters si propone di sviluppare è una versione standalone dei moduli “AI Assistant Generativo” e "AI Co-Pilot per i CdL", inizialmente svincolata dalla piattaforma NEXUM. Tale applicazione costituirà un prototipo funzionale in grado di operare autonomamente e di implementare le principali funzionalità richieste dal Committente, senza dipendere dagli altri moduli della piattaforma.

L’app standalone permetterà di testare e consolidare le funzionalità richieste, offrendo un ambiente controllato che faciliti la sperimentazione e lo sviluppo incrementale. Questa fase costituirà la base per un’eventuale integrazione futura con la piattaforma NEXUM, la quale fornirà un ecosistema HR completo e dotato di servizi quali la messaggistica top-down, la timbratura digitale, la gestione delle anagrafiche e dei ruoli, e la collaborazione con gli studi dei Consulenti del Lavoro (CdL).

L’integrazione futura con NEXUM sarà concepita in modo modulare, consentendo alla nuova applicazione di inserirsi nell’architettura esistente come componente riutilizzabile e scalabile. L’integrazione includerà l’adattamento delle API, l’allineamento della gestione utenti e la centralizzazione dei dati all’interno dell’infrastruttura NEXUM.

\subsection{Funzioni del prodotto}

Dal punto di vista dell’utilizzatore finale, l’applicazione standalone dovrà fornire le seguenti funzionalità:

\begin{itemize}
    \item \textbf{Generazione di contenuti tramite AI (Modulo AI Assistant Generativo)}: 
    generazione di titolo, testo e immagine di copertina a partire da un prompt, 
    con possibilità di selezionare tono, stile e configurazioni avanzate del modello AI.

    \item \textbf{Salvataggio locale}: 
    gestione interna di prompt, contenuti generati, immagini e valutazioni, 
    tramite archivio locale dedicato all’app standalone, indipendente dalla piattaforma NEXUM.

    \item \textbf{Sistema di rating}: 
    valutazione della qualità dei contenuti generati dall’AI, utile per analisi interne e miglioramento continuo.

    \item \textbf{Gestione dei prompt}: 
    storico dei prompt utilizzati, con possibilità di riutilizzo, duplicazione e ricerca interna.

    \item \textbf{Esportazione dei contenuti}: 
    esportazione in formati standard (PDF, testo, immagine) per permettere anche un’integrazione manuale con sistemi esterni.

    \item \textbf{Dashboard standalone}: 
    visualizzazione e gestione di storico, filtri, ricerca e analisi delle interazioni con l’AI generativa.

    \item \textbf{Gestione delle immagini}: 
    possibilità di caricare immagini dall’utente o di generarle tramite AI, con salvataggio locale.

    \item \textbf{Gestione utenti}: 
    registrazione, autenticazione, gestione del profilo e configurazione dei parametri AI 
    (per utenti privilegiati come amministratori o editor avanzati).

    \item \textbf{Upload e gestione documentale (Modulo AI Co-Pilot per i CdL)}: 
    possibilità di caricare documenti (PDF, ZIP, etc), salvarli localmente e gestirne lo stato di elaborazione.

    \item \textbf{Riconoscimento automatico della tipologia di documento}: 
    classificazione tramite AI (cedolini, CU, comunicazioni, lettere, moduli da firmare, ecc.) 
    sfruttando modelli OCR e classificatori addestrati.

    \item \textbf{Estrazione dei destinatari}: 
    riconoscimento automatico di informazioni contenute nei documenti 
    (nome, cognome, codice fiscale, matricola, reparto) tramite tecniche AI di entity extraction.

    \item \textbf{Split dei documenti massivi}: 
    suddivisione automatica dei documenti multi-destinatario (es. cedolini massivi) 
    in documenti singoli, ognuno associato al proprio destinatario riconosciuto.

    \item \textbf{Revisione manuale (Human-in-the-Loop)}: 
    interfaccia dedicata per verificare, correggere o confermare i risultati ottenuti dall’AI 
    in ogni fase (classificazione, destinatari, split).

    \item \textbf{Creazione di messaggi e liste di distribuzione}: 
    generazione automatica di bozze di messaggi e liste di destinatari derivanti dai documenti processati.

    \item \textbf{Tracciamento locale}: 
    storico delle operazioni effettuate (upload, riconoscimento, revisioni, esportazioni), 
    utile per audit interni e analisi del flusso documentale.

    \item \textbf{Esportazione documentale}: 
    generazione di pacchetti ZIP contenenti documenti processati, metadati, liste di destinatari e log di lavorazione.
\end{itemize}

Queste funzionalità permetteranno all’app standalone di essere completamente operativa e autonoma nei due moduli (AI Assistant Generativo e AI Co-Pilot per i CdL). 
In una fase successiva, tali componenti saranno progettati per essere integrati nella piattaforma NEXUM, 
consentendo così un’evoluzione verso un ecosistema HR completo, scalabile e basato su automazioni intelligenti.

\subsection{Caratterisitiche dell'utente}

Gli utilizzatori finali dell'applicazione standalone non appartengono a un’unica categoria specifica: 
l’obiettivo del progetto è quello di progettare moduli intelligenti e interoperabili 
che possano essere integrati all’interno dell’ecosistema NEXUM o utilizzati autonomamente durante la fase standalone.

In generale, è possibile affermare che gli utenti finali sono coloro che necessitano di uno strumento scalabile, 
intelligente e semplice da utilizzare per generare contenuti tramite AI e per gestire flussi documentali complessi con il supporto del modulo Co-Pilot.
Rientrano in questa categoria:

\begin{itemize}
    \item \textbf{Responsabili e amministratori HR}, che necessitano di strumenti avanzati per la creazione di comunicazioni interne, 
    la gestione dei contenuti generativi e l’analisi delle produzioni.

    \item \textbf{Consulenti del Lavoro (CdL) e personale amministrativo}, che richiedono un sistema in grado di caricare, riconoscere, suddividere e preparare documenti per la distribuzione ai destinatari.

    \item \textbf{Dipendenti e collaboratori} (in fase integrata), che potranno interagire con la piattaforma NEXUM per consultare documenti e comunicazioni, 
    pur non essendo utenti della versione standalone.

    \item \textbf{Manager aziendali}, interessati a monitorare la consistenza delle comunicazioni e l’efficienza dei processi documentali, 
    sia nella versione standalone che nella futura integrazione.
\end{itemize}

In sintesi, il prodotto è rivolto a organizzazioni di varie dimensioni — in particolare aziende medio-grandi e studi professionali — 
che necessitano di strumenti intelligenti per la creazione di contenuti, 
la gestione automatizzata dei documenti e la collaborazione con gli studi dei Consulenti del Lavoro.
L’app standalone funge da primo passo verso una piattaforma HR completa, modulare e potenziata dall’AI.


\subsection{Definizioni, acronimi e abbreviazioni}
Per tutte le definizioni, acronimi e abbreviazioni utilizzati in questo documento, si faccia
riferimento al \textbf{Glossario}, fornito come documento separato, che contiene tutte le spiegazioni
necessarie per garantire una comprensione uniforme dei termini tecnici e dei concetti
rilevanti per il progetto.

\newpage

\subsection{Riferimenti}

\subsubsection{Riferimenti normativi}
\begin{itemize}
\item \textbf{Capitolato\textsubscript{\scalebox{0.6}{\textbf{G}}}
 d'appalto C5: Nexum - Piattaforma di consulenza e documentazione previdenziale}\\
\url{https://www.math.unipd.it/~tullio/IS-1/2025/Progetto/C5.pdf}
\end{itemize}

\subsubsection{Riferimenti informativi}
\begin{itemize}
\item \textbf{Glossario\textsubscript{\scalebox{0.6}{\textbf{G}}}
:}\\
\url{https://github.com/BugBustersUnipd/DocumentazioneSWE/blob/main//RTB/GLOSSARIO/Glossario.pdf}
\end{itemize}



\section{Casi d'uso}
\subsection{Introduzione}
I casi d’uso si compongono di un grafico UML e una descrizione testuale che permetta di
comprendere al meglio cosa il prodotto deve fornire. La descrizione testuale, in particolar
modo, dovrà contenere le informazioni sotto presenti, salvo i casi in cui lo
specifico campo non risulti rilevante (ad esempio, un Caso d’Uso\textsubscript{\scalebox{0.6}{\textbf{G}}} che non prevede la
possibilità di errori non avrà Scenari secondari):

\begin{itemize}
    \item \textbf{Attori}: Sono coloro che interagiscono attivamente con il Sistema e
    svolgono l’azione indicata dal Caso d’Uso
    \item \textbf{Precondizioni}: Lista di elementi che sono necessari affinché l’Attore possa
    compiere l’azione indicata dal caso d’uso
    \item \textbf{Postcondizioni}: Lista di elementi che descrivono come il Sistema risulta
    essere internamente cambiato dopo che l’Attore ha effettuato
    l’azione prevista dal Caso d’Uso
    \item \textbf{Scenario principale}: Descrizione ragionevole delle operazioni che l’attore deve
    fare per compiere l’azione descritta dal Caso d’Uso
    \item \textbf{Scenario secondario}: Descrizione ragionevole degli eventi che possono accadere
    qualora una delle operazioni descritte nello Scenario
    principale non vada a buon fine
    \item \textbf{Trigger}: Evento che innesca l’inizio del Caso d’Uso
    \item \textbf{Inclusioni}: Casi d’Uso ulteriori che l’Attore deve compiere per realizzare
    il Caso d’Uso attualmente descritto
    \item \textbf{Estensioni}: Casi d’Uso ulteriori che possono realizzarsi durante
    l’esecuzione delle operazioni del Caso d’Uso principale
    
\end{itemize}
Motivazioni che portano l’Attore a svolgere l’azione descritta
dal Caso d’Uso. Non sempre disponibile in quanto il Caso
d’Uso potrebbe essere incluso da un altro caso d’uso «principale».

\subsection{Attori}
Di seguito sono esposti gli attori utilizzati:
\begin{figure}[H]
    \centering
    \includegraphics[width=0.8\textwidth]{Diagrammi casi d'uso/diagramma_attori.jpg}
    \caption{Diagramma degli attori principali}
\end{figure}
\begin{itemize}
    \item \textbf{Utente}: Rappresenta un utente che vuole accedere al Sistema.
    \item \textbf{HR Manager}: È la figura responsabile della comunicazione interna. Utilizza il modulo AI Assistant per generare, revisionare e pubblicare messaggi o avvisi rivolti ai dipendenti, definendone tono e stile.
    \item Redattore: 
    \item \textbf{Data Analyst}: Figura incaricata di monitorare le prestazioni. Accede alle dashboard di analisi per consultare le statistiche di utilizzo, i rating di qualità dei contenuti generati e i KPI del riconoscimento documentale.
    \item \textbf{Amministratore}: Gestisce la configurazione tecnica dell'applicazione standalone. Si occupa della creazione degli utenti, della gestione dei ruoli e della configurazione dei parametri globali dell'AI (es. prompt di sistema o soglie di confidenza).
    \item \textbf{Operatore Studio CdL}: È l'utente principale del modulo AI Co-Pilot. Si occupa di caricare i flussi documentali (es. cedolini massivi), supervisionare il riconoscimento automatico (validazione Human-in-the-Loop) e gestire le liste di distribuzione.
    \item \textbf{Sistema NEXUM (AI Doc Classifier)}: Il modulo intelligente incaricato di analizzare visivamente il documento, applicare l'OCR e classificarne la tipologia (es. "Cedolino", "CUD").
    \item \textbf{Sistema NEXUM (Entity Resolver)}: Il componente che analizza il testo estratto per identificare univocamente i destinatari (es. Nome, Cognome, CF) confrontandoli con l'anagrafica.
    \item \textbf{Sistema NEXUM (Splitter)}: L'agente automatico che scansiona i documenti massivi (es. PDF multipagina) e li suddivide in singoli file, uno per ciascun destinatario individuato.
    \item \textbf{Sistema NEXUM (Dispatcher \& Tracking)}: Il modulo responsabile della creazione dei pacchetti di invio e della generazione delle ricevute di consegna (simulata in ambiente standalone).
    \item \textbf{Destinatario finale}: Rappresenta il dipendente a cui sono indirizzati i documenti o i messaggi. Nell'applicazione standalone, la sua interazione (ricezione e lettura) è simulata per verificare il corretto funzionamento del dispaccio.
    \item \textbf{Auditor interno}: Utente con permessi di sola lettura focalizzato sul controllo. Verifica lo storico delle operazioni (audit trail) per garantire la tracciabilità e la sicurezza dei flussi documentali.
    \item \textbf{Sistema}: L'applicazione standalone nel suo complesso, che gestisce autenticazione, database e interfaccia.
    \item \textbf{Admin Cliente, Admin Eggon}: Figure di alto livello responsabili, rispettivamente, della gestione dell'organizzazione cliente e della supervisione tecnica del progetto per conto di Eggon.
\end{itemize}


\subsection{Lista casi d'uso}

\text{L'elenco dei casi d'uso sará diviso in tre parti:}
\begin{itemize}
    \item 0 - Casi d'uso per la gestione utenti e autenticazione
    \item 1 - Casi d'uso per il modulo "AI Assistant Generativo"
    \item 2 - Casi d'uso per il modulo "AI Doc Classifier"
\end{itemize}


\subsection{Sezione 0 – Applicazione standalone}

\subsubsection{UC-0A – Registrazione nuovo utente}

\begin{figure}[H]
    \centering
    \includegraphics[width=1\textwidth]{Diagrammi casi d'uso/UC0A.jpg}
    \caption{Didascalia dell'immagine}
\end{figure}


\textbf{Attori}
\begin{itemize}
    \item Utente non autenticato (nuovo utente).
    \item Sistema di autenticazione dell’applicazione standalone.
\end{itemize}

\textbf{Pre-condizioni}
\begin{itemize}
    \item L’utente non ha una sessione attiva.
    \item L’utente non è ancora registrato nel Sistema (l’e-mail inserita non risulta già presente).
\end{itemize}

\textbf{Post-condizioni}
\begin{itemize}
    \item Esiste un nuovo account utente registrato nel Sistema.
    \item L’utente può effettuare il login utilizzando le credenziali appena create.
\end{itemize}

\textbf{Scenario principale}
\begin{enumerate}
    \item L’utente accede alla schermata di registrazione dell’applicazione standalone.
    \item L’utente inserisce i dati richiesti (ad esempio: nome, cognome, e-mail, password).
    \item Il Sistema verifica la correttezza formale dei dati inseriti (es. formato e-mail, forza della password).
    \item Il Sistema controlla che l’indirizzo e-mail non sia già associato a un account esistente.
    \item In caso di esito positivo, il Sistema crea un nuovo account utente e lo memorizza nel proprio archivio.
    \item Il Sistema conferma l’avvenuta registrazione e può opzionalmente eseguire il login automatico del nuovo utente.
\end{enumerate}

\textbf{Scenario secondario}
\begin{enumerate}
    \item Nell'inserimento dei dati, uno o più campi non rispettano i requisiti di validità (es. e-mail non valida, password debole).
    \item Il Sistema mostra un messaggio di errore specifico per il campo non valido.
    \item L’utente corregge i dati e ripete l’inserimento.
\end{enumerate}

\textbf{Relazioni con altri casi d'uso (\textit{include} / \textit{extend})}
\begin{itemize}
    \item \textit{include}:
    \begin{itemize}
        \item UC-0A.1 - Inserimento email
        \item UC-0A.2 - Inserimento password
        \item UC-0A.3 - Inserimento username
        \item UC-0A.4 - Inserimento nome 
        \item UC-0A.5 - Inserimento cognome
        \item UC-0A.6 - Inserimento matricola
    \end{itemize}
    \item \textit{extend}: 
    \begin{itemize}
        \item UC-0B – Login / Autenticazione utente (in caso di login automatico al termine della registrazione).
        \item UC-0A.7 – email non valida.
        \item UC-0A.8 – password non valida.
        \item UC-0A.9 – email già registrata.
        \item UC-0A.10 – username già registrato.
        \item UC-0A.11 – matricola già registrata.
        \item UC-0A.12 – matricola non valida.
    \end{itemize}
\end{itemize}

\vspace{0.5cm}

\subsubsection{UC-0B – Login / Autenticazione utente}

\begin{figure}[H]
    \centering
    \includegraphics[width=0.7\textwidth]{Diagrammi casi d'uso/UC0B.jpg}
    \caption{Didascalia dell'immagine}
\end{figure}

\textbf{Attori}
\begin{itemize}
    \item Utente registrato.
    \item Sistema di autenticazione dell’applicazione standalone.
\end{itemize}

\textbf{Pre-condizioni}
\begin{itemize}
    \item L’utente è già registrato nel Sistema.
    \item Non esiste una sessione attiva associata all’utente sul dispositivo corrente.
\end{itemize}

\textbf{Post-condizioni}
\begin{itemize}
    \item L’utente risulta autenticato nel Sistema.
    \item È attiva una sessione associata all’utente, che consente l’accesso alle funzionalità riservate (es. generazione contenuti, upload documenti).
\end{itemize}

\textbf{Scenario principale}
\begin{enumerate}
    \item L’utente accede alla schermata di login.
    \item L’utente inserisce le proprie credenziali (e-mail e password).
    \item Il Sistema verifica la correttezza delle credenziali.
    \item In caso di credenziali valide, il Sistema crea una nuova sessione autenticata per l’utente.
    \item Il Sistema reindirizza l’utente alla dashboard principale dell’applicazione standalone.
\end{enumerate}

\textbf{Scenario secondario}
\begin{enumerate}
    \item L'utente inserisce un’e-mail non registrata nel Sistema.
    \item L'utente inserisce una password errata.
    \item L'utente inserisce credenziali non valide.
\end{enumerate}



\textbf{Relazioni con altri casi d'uso (\textit{include} / \textit{extend})}
\begin{itemize}
    \item \textit{include}:
    \begin{itemize}
        \item UC-0A.1 - Inserimento email
        \item UC-0A.2 - Inserimento password
    \end{itemize}
    \item \textit{extend}: 
    \begin{itemize}
        \item UC-0A.7 – email non valida.
        \item UC-0A.8 – password non valida.
        \item UC-0B.1 – email non registrata.
        \item UC-0B.2 – password errata.
    \end{itemize}
\end{itemize}

\vspace{0.5cm}



\subsubsection{UC-0C – Visualizzazione pagina principale / dashboard}

\begin{figure}[H]
    \centering
    \includegraphics[width=1\textwidth]{Diagrammi casi d'uso/UC0C.jpg}
    \caption{Didascalia dell'immagine}
\end{figure}

\textbf{Attori}
\begin{itemize}
    \item Utente autenticato.
\end{itemize}

\textbf{Pre-condizioni}
\begin{itemize}
    \item L’utente ha effettuato il login ed è autenticato.
    \item Esiste un profilo associato all’utente nel Sistema (dati anagrafici e preferenze).
\end{itemize}


\textbf{Post-condizioni}
\begin{itemize}
    \item L'utente autenticato puó scegliere tra diverse azioni.
\end{itemize}

\textbf{Scenario principale}
\begin{enumerate}
    \item L’utente dopo aver effettuato il login si ritrova in una schermata con la scelta di diversi moduli.
\end{enumerate}

\textbf{Scenario secondario}
\begin{enumerate}
    \item 
\end{enumerate}



\textbf{Relazioni con altri casi d'uso (\textit{include} / \textit{extend})}
\begin{itemize}
    \item \textit{include}: 
    \begin{itemize}
        \item 
    \end{itemize}
    \item \textit{extend}: 
    \begin{itemize}
        \item UC-1A - Visualizzazione modulo di generazione contenuti AI.
        \item UC-2A - Visualizzazione modulo di upload e gestione documentale.
        \item UC-0D - Visualizzazione modulo gestione profilo utente.
        \item UC-0E - Visualizzazione modulo gestione ruoli.
    \end{itemize}
\end{itemize}


\vspace{0.5cm}

\subsubsection{UC-0D – Visualizzazione gestione profilo utente}

\begin{figure}[H]
    \centering
    \includegraphics[width=1\textwidth]{Diagrammi casi d'uso/UC0C.jpg}
    \caption{Didascalia dell'immagine}
\end{figure}

\textbf{Attori}
\begin{itemize}
    \item Utente autenticato.
    \item Sistema di gestione profilo dell’applicazione standalone.
\end{itemize}


\textbf{Pre-condizioni}
\begin{itemize}
    \item L’utente ha effettuato il login ed è autenticato.
    \item Esiste un profilo associato all’utente nel Sistema (dati anagrafici e preferenze).
    \item L'utente speciale é entrato nel modulo di gestione profilo utente dalla dashboard principale.

\end{itemize}

\textbf{Post-condizioni}
\begin{itemize}
    \item Le informazioni del profilo utente risultano aggiornate nel Sistema.
    \item Le nuove preferenze (ad esempio tono/stile predefinito) verranno utilizzate nelle interazioni successive con i moduli AI.
\end{itemize}

\textbf{Scenario principale}
\begin{enumerate}
    \item L’utente accede alla sezione “Profilo” dalla dashboard dell’applicazione.
    \item Il Sistema mostra i dati correnti del profilo (es. nome, cognome, e-mail, ruolo, preferenze AI come tono/stile predefinito).
    \item L’utente puó uno o più campi del profilo (es. nome visualizzato, preferenze di tono, lingua).
    \item L’utente conferma le modifiche.
    \item Il Sistema valida i dati inseriti (ad esempio formato dell’e-mail, campi obbligatori).
    \item Il Sistema salva le modifiche nel proprio archivio.
    \item Il Sistema conferma l’avvenuto aggiornamento del profilo.
\end{enumerate}

\textbf{Scenario secondario}
\begin{enumerate}
    \item 
\end{enumerate}



\textbf{Relazioni con altri casi d'uso (\textit{include} / \textit{extend})}
\begin{itemize}
    \item \textit{include}: 
    \begin{itemize}
        \item UC-0D.1 - Visualizzazione email
        \item UC-0D.2 - Visualizzazione password
        \item UC-0D.3 - Visualizzazione username
        \item UC-0D.4 - Visualizzazione nome 
        \item UC-0D.5 - Visualizzazione cognome
        \item UC-0D.6 - Visualizzazione matricola
    \end{itemize}
    \item \textit{extend}: 
    \begin{itemize}
        \item UC-0A.1 - Inserimento email
        \item UC-0A.2 - Inserimento password
        \item UC-0A.3 - Inserimento username
        \item UC-0A.4 - Inserimento nome 
        \item UC-0A.5 - Inserimento cognome
        \item UC-0A.6 - Inserimento matricola
        \item UC-0D.7 – Salva profilo utente. %in questo caso d'uso ci saranno tutti gli errori su inserimenti sbaliati (uguali a quelli della registrazione)
        \item UC-0D.8 – Uscita senza salvare profilo utente.
    \end{itemize}
\end{itemize}

\vspace{0.5cm}

\subsubsection{UC-0E – Visualizzazione gestione ruoli (Admin / Editor)}

\begin{figure}[H]
    \centering
    \includegraphics[width=0.7\textwidth]{Diagrammi casi d'uso/UC0B.jpg}
    \caption{Didascalia dell'immagine}
\end{figure}

\textbf{Attori}
\begin{itemize}
    \item Utenti speciali (Admin o altri ruoli definiti).
    \item Sistema di gestione ruoli e permessi.
\end{itemize}


\textbf{Pre-condizioni}
\begin{itemize}
    \item L’utente amministratore ha effettuato il login ed è autenticato come Admin.
    \item Esistono uno o più account utente registrati nel Sistema.
    \item L'utente speciale é entrato nel modulo di gestione ruoli dalla dashboard principale.
\end{itemize}

\textbf{Post-condizioni}
\begin{itemize}
    \item I ruoli e i permessi degli utenti risultano aggiornati nel Sistema.
    \item Le funzionalità accessibili a ciascun utente dipendono dal nuovo ruolo assegnato (es. solo Admin può modificare i parametri AI globali).
\end{itemize}

\textbf{Scenario principale}
\begin{enumerate}
    \item L’Amministratore accede alla sezione di amministrazione utenti.
    \item Il Sistema mostra l’elenco degli utenti registrati, con i rispettivi ruoli correnti.
    \item L’Amministratore seleziona un utente da modificare.
    \item L’Amministratore assegna o modifica il ruolo dell’utente (es. da Editor a Admin, oppure rimozione privilegi).
    \item L’Amministratore conferma le modifiche.
    \item Il Sistema aggiorna i ruoli e i permessi associati all’utente.
    \item Il Sistema registra l’operazione per finalità di audit interno.
\end{enumerate}

\textbf{Scenario secondario}
\begin{enumerate}
    \item Un utente non autorizzato tenta di accedere al modulo di gestione ruoli.
\end{enumerate}



\textbf{Relazioni con altri casi d'uso (\textit{include} / \textit{extend})}
\begin{itemize}
    \item \textit{include}: 
    \begin{itemize}
        \item UC-0E.1 - Visualizzazione nome utenti registrati.
        \item UC-0E.2 - Visualizzazione cognome utenti registrati.
        \item UC-0E.3 - Visualizzazione ruolo utenti registrati.
    \end{itemize}
    \item \textit{extend}: 
    \begin{itemize}
        \item UC-0E.4 - Modifica ruolo utente registrato.
        \item UC-0E.5 - Salva modifica ruolo utente registrato.
        \item UC-0E.6 - Annulla modifica ruolo utente registrato
        \item UC-0E.7 - Utente non autorizzato.
    \end{itemize}
    
\end{itemize}

\vspace{0.5cm}

\subsubsection{UC-0F – Logout}

\begin{figure}[H]
    \centering
    \includegraphics[width=1\textwidth]{Diagrammi casi d'uso/UC0E.jpg}
    \caption{Didascalia dell'immagine}
\end{figure}

\textbf{Attori}
\begin{itemize}
    \item Utente autenticato.
    \item Sistema di gestione sessione dell’applicazione standalone.
\end{itemize}

\textbf{Pre-condizioni}
\begin{itemize}
    \item L’utente ha una sessione attiva nel Sistema.
\end{itemize}

\textbf{Post-condizioni}
\begin{itemize}
    \item Non esiste più una sessione attiva associata all’utente sul dispositivo corrente.
    \item Per accedere nuovamente alle funzionalità riservate è necessario eseguire un nuovo login.
\end{itemize}

\textbf{Scenario principale}
\begin{enumerate}
    \item L’utente seleziona l’opzione di logout (ad esempio dal menu della dashboard).
    \item Il Sistema invalida la sessione corrente associata all’utente (es. rimozione token di sessione).
    \item Il Sistema reindirizza l’utente alla schermata di login o alla schermata iniziale pubblica.
\end{enumerate}

\textbf{Scenario secondario}
\begin{enumerate}
    \item 
\end{enumerate}


\textbf{Relazioni con altri casi d'uso (\textit{include} / \textit{extend})}
\begin{itemize}
    \item \textit{include}: 
    \begin{itemize}
        \item Nessuno.
    \end{itemize}
    \item \textit{extend}:
    \begin{itemize}
        \item Nessuno.
    \end{itemize}
\end{itemize}

\vspace{0.5cm}

\subsection{Sezione 1 – Modulo AI Assistant Generativo}

\subsubsection{UC-1A – Visualizzazione modulo di generazione contenuti AI}


\begin{figure}[H]
    \centering
    \includegraphics[width=1\textwidth]{Diagrammi casi d'uso/UC1A.jpg}
    \caption{Diagramma del caso d'uso UC-1A – Visualizzazione modulo di generazione contenuti AI}
\end{figure}

\textbf{Attori}
\begin{itemize}
    \item Utente autorizzato (Editor o Admin).
\end{itemize}

\textbf{Pre-condizioni}
\begin{itemize}
    \item L’utente é entrato nel modulo AI Assistant Generativo dalla dashboard principale.
    \item L’utente dispone dei permessi necessari per utilizzare il modulo AI Assistant.
\end{itemize}

\textbf{Post-condizioni}
\begin{itemize}
    \item L'utente visualizza l’interfaccia di generazione contenuti AI, pronta per l’inserimento del prompt e la selezione delle opzioni.
\end{itemize}

\textbf{Scenario principale}
\begin{enumerate}
    \item L’utente accede alla sezione “AI Assistant Generativo”.
    \item Il Sistema mostra il campo per l’inserimento del prompt e le azione che possono essere eseguite dall'utente.
\end{enumerate}

\textbf{Scenario secondario}
\begin{enumerate}
    \item 
\end{enumerate}


\textbf{Relazioni con altri casi d'uso (\textit{include} / \textit{extend})}
\begin{itemize}
    \item \textit{include}: 
    \begin{itemize}
        \item Nessuna.
    \end{itemize}
    \item \textit{extend}: 
    \begin{itemize}
        \item UC-1A.1 - Inserimento prompt
        \item UC-1A.2 - Selezione tono
        \item UC-1A.3 - Selezione stile
        \item UC-1C - Generazione contenuto tramite AI
        \item UC-1B - Visualizzazione storico prompt
    \end{itemize}
\end{itemize}

\vspace{0.5cm}

\subsubsection{UC-1B – Visualizzazione storico prompt}

\begin{figure}[H]
    \centering
    \includegraphics[width=0.7\textwidth]{Diagrammi casi d'uso/UC0B.jpg}
    \caption{Didascalia dell'immagine}
\end{figure}

\textbf{Attori}
\begin{itemize}
    \item Utente autorizzato (Editor o Admin).
    \item Sistema di persistenza locale.
\end{itemize}

\textbf{Pre-condizioni}
\begin{itemize}
    \item Ci sono stati precedenti utilizzi del modulo AI Assistant Generativo.
    \item L’utente é entrato nel modulo AI Assistant Generativo dalla dashboard principale.
\end{itemize}

\textbf{Post-condizioni}
\begin{itemize}
    \item L'utente visualizza lo storico dei prompt utilizzati e puó interagire con essi (es. riutilizzo, duplicazione, ricerca).
\end{itemize}

\textbf{Scenario principale}
\begin{enumerate}
    \item L’utente accede alla sezione “Storico Prompt” all’interno del modulo AI Assistant Generativo.
    \item Il Sistema mostra l’elenco dei prompt precedentemente utilizzati, con i relativi dettagli (tono, stile, risultato generato, data/ora, valutazione).
\end{enumerate}

\textbf{Scenario secondario}
\begin{enumerate}
    \item Non ci sono prompt salvati nello storico.
\end{enumerate}


\textbf{Relazioni con altri casi d'uso (\textit{include} / \textit{extend})}
\begin{itemize}
    \item \textit{include}: 
    \begin{itemize}
        \item UC-1B.1 – Visualizzazione prompt
        \item UC-1B.2 – Visualizzazione tono
        \item UC-1B.3 – Visualizzazione stile
        \item UC-1B.4 – Visualizzazione risultato
        \item UC-1B.5 – Visualizzazione timestamp
        \item UC-1B.6 – Visualizzazione Valutazione
    \end{itemize}
    \item \textit{extend}: 
    \begin{itemize}
        \item UC-1B.7 – Ricerca
        \item UC-1B.8 – Riutilizza (rigenera con lo stesso prompt)
        \item UC-1B.9 – Duplica (Ti riporta al modulo di generazione con prompt, tono e stile precompilati)
        \item UC-1B.10 – Nessun prompt salvato.
    \end{itemize}
\end{itemize}

\vspace{0.5cm}


\subsubsection{UC-1C – Visualizzazione pagina contenuto generato}

\begin{figure}[H]
    \centering
    \includegraphics[width=1\textwidth]{Diagrammi casi d'uso/UC1C.jpg}
    \caption{Diagramma del caso d'uso UC-1C – Visualizzazione pagina contenuto generato}
\end{figure}

\textbf{Attori}
\begin{itemize}
    \item Utente autorizzato.
    \item Sistema AI (in caso di nuove generazioni parziali).
\end{itemize}

\textbf{Pre-condizioni}
\begin{itemize}
    \item Un contenuto è stato generato tramite UC-1A.
\end{itemize}

\textbf{Post-condizioni}
\begin{itemize}
    \item L'utente puó eseguire le varie azioni mostrate a schermo
\end{itemize}


\textbf{Scenario principale}
\begin{enumerate}
    \item Il contenuto viene generato e mostrato in anteprima all’utente.
\end{enumerate}

\textbf{Scenario secondario}
\begin{enumerate}
    \item 
\end{enumerate}


\textbf{Relazioni con altri casi d'uso (\textit{include} / \textit{extend})}
\begin{itemize}
    \item \textit{include}: 
    \begin{itemize}
        \item UC-1C.1 – Visualizzazione anteprima contenuto generato.
    \end{itemize}
    \item \textit{extend}: 
    \begin{itemize}
        \item UC-1C.2 – Rigenera contenuto tramite AI 
        \item UC-1C.3 – Salva post generato
        \item UC-1C.4 – Valuta contenuto generato
        \item UC-1C.5 – Scarta contenuto generato
        \item UC-1C.6 – Pubblica contenuto generato
        \item UC-1D – Visualizzazione pagina modifica contenuto generato
    \end{itemize}
\end{itemize}

\vspace{0.5cm}

\subsubsection{UC-1D – Visualizzazione pagina modifica contenuto generato}

\begin{figure}[H]
    \centering
    \includegraphics[width=1\textwidth]{Diagrammi casi d'uso/UC1D.jpg}
    \caption{Diagramma del caso d'uso UC-1D – Visualizzazione pagina modifica contenuto generato}
\end{figure}

\textbf{Attori}
\begin{itemize}
    \item Utente autorizzato (Editor o Admin).
    \item Sistema di persistenza locale.
\end{itemize}

\textbf{Pre-condizioni}
\begin{itemize}
    \item L'utente ha generato un contenuto tramite UC-1A e lo sta visualizzando in anteprima tramite UC-1C.
\end{itemize}

\textbf{Post-condizioni}
\begin{itemize}
    \item L'utente si trova nella pagina di modifica e puó eseguire le opzioni mostrate.
\end{itemize}


\textbf{Scenario principale}
\begin{enumerate}
    \item L'utente é nella pagina del contenuto generato.
    \item L'utente seleziona l’opzione di modifica del contenuto generato.
\end{enumerate}

\textbf{Scenario secondario}
\begin{enumerate}
    \item 
\end{enumerate}



\textbf{Relazioni con altri casi d'uso (\textit{include} / \textit{extend})}
\begin{itemize}
    \item \textit{include}: 
    \begin{itemize}
        \item UC-1C.1 – Visualizzazione anteprima contenuto generato
    \end{itemize}
    \item \textit{extend}: 
    \begin{itemize}
        \item UC-1D.1 – Modifica immagine
        \item UC-1D.2 – Modifica titolo
        \item UC-1D.3 – Modifica testo
        \item UC-1D.4 – Salva modifiche
        \item UC-1D.5 – Annulla modifiche
        \item UC-1D.6 - File immagine non valido
    \end{itemize}
\end{itemize}

\vspace{0.5cm}



\subsection{Sezione 2 – Modulo AI Co-Pilot per i Consulenti del Lavoro (CdL)}

\subsubsection{UC-2A – Visualizzazione modulo di upload e gestione documentale.}

\begin{figure}[H]
    \centering
    \includegraphics[width=0.7\textwidth]{Diagrammi casi d'uso/UC0B.jpg}
    \caption{Didascalia dell'immagine}
\end{figure}

\textbf{Attori}
\begin{itemize}
    \item Utente autorizzato (Operatore di studio CdL).
\end{itemize}


\textbf{Pre-condizioni}
\begin{itemize}
    \item L’utente é entrato nel modulo AI Co-Pilot per i CdL dalla dashboard principale.
    \item L’utente dispone dei permessi necessari per utilizzare il modulo AI Co-Pilot per i CdL.
\end{itemize}

\textbf{Post-condizioni}
\begin{itemize}
    \item L'utente visualizza l’interfaccia di upload documenti, pronta per il caricamento e la gestione dei file.
\end{itemize}

\textbf{Scenario principale}
\begin{enumerate}
    \item L'utente dalla dashboard principale accede al Modulo AI Co-Pilot per i Consulenti del Lavoro.
    \item Il Sistema mostra l’interfaccia di upload documenti, con le azioni
\end{enumerate}

\textbf{Scenario secondario}
\begin{enumerate}
    \item 
\end{enumerate}



\textbf{Relazioni con altri casi d'uso (\textit{include} / \textit{extend})}
\begin{itemize}
    \item \textit{include}: 
    \begin{itemize}
        \item Nessuna.
    \end{itemize}
    \item \textit{extend}: 
    \begin{itemize}
        \item UC-2A.1 - Caricamento file %Errore nel caricamento file (format non valido, dimensione eccessiva)
        \item UC-2A.2 - Inserimento categoria
        \item UC-2A.3 - Inserimento mese/anno di competenza
        \item UC-2A.4 - Inserimento azienda
        \item UC-2A.5 - Inserimento reparto
        \item UC-2A.6 - Avvio upload
    \end{itemize}
\end{itemize}

\vspace{0.5cm}


\subsubsection{UC-2B – Visualizzazione pagina lista documenti.}

\begin{figure}[H]
    \centering
    \includegraphics[width=0.7\textwidth]{Diagrammi casi d'uso/UC0B.jpg}
    \caption{Didascalia dell'immagine}
\end{figure}

\textbf{Attori}
\begin{itemize}
    \item Utente autorizzato (Operatore di studio CdL).
    \item Sistema AI Doc Classifier.
\end{itemize}


\textbf{Pre-condizioni}
\begin{itemize}
    \item É stato caricato almeno un documento tramite UC-2A.6.
    \item L’utente é entrato nel modulo AI Co-Pilot per i CdL dalla dashboard principale.
    \item L’utente dispone dei permessi necessari per utilizzare il modulo AI Co-Pilot
\end{itemize}


\textbf{Post-condizioni}
\begin{itemize}
    \item L'utente puó visualizzare la lista dei documenti caricati e le relative informazioni.
\end{itemize}

\textbf{Scenario principale}
\begin{enumerate}
    \item L'utente dopo aver caricato uno o più documenti accede alla sezione "Lista documenti".
    \item Il Sistema mostra l’elenco dei documenti caricati, con le relative informazioni (tipologia, competenza, azienda, causale, lingua, numero pagine, nome originale).
\end{enumerate}

\textbf{Scenario secondario}
\begin{enumerate}
    \item Il sistema non riconosce alcun documento caricato.
\end{enumerate}


\textbf{Relazioni con altri casi d'uso (\textit{include} / \textit{extend})}
\begin{itemize}
    \item \textit{include}: 
    \begin{itemize}
        \item UC-2B.1 – Visualizzazione codice documento
        \item UC-2B.2 - Visualizzazione tipologia documento
        \item UC-2B.3 - Visualizzazione competenza documento
        \item UC-2B.4 – Visualizzazione azienda documento
        \item UC-2B.5 - Visualizzazione causale
        \item UC-2B.6 - Visualizzazione lingua
        \item UC-2B.7 - Visualizzazione numero pagine documento
        \item UC-2B.8 - Visualizzazione nome documento originale
        \item UC-2B.9 - Visualizzazione data redazione documento
    \end{itemize}
    \item \textit{extend}: 
    \begin{itemize}
        \item UC-2B.10 - Filtraggio documenti
        \item UC-2B.11 - Nessun documento riconosciuto.
        \item UC-2C – Visualizzazione pagina documento
        \item UC-2D - Visualizzazione pagina destinatario
    \end{itemize}
\end{itemize}


\vspace{0.5cm}

\subsubsection{UC-2C – Visualizzazione pagina documento}

\begin{figure}[H]
    \centering
    \includegraphics[width=0.7\textwidth]{Diagrammi casi d'uso/UC0B.jpg}
    \caption{Didascalia dell'immagine}
\end{figure}

\textbf{Attori}
\begin{itemize}
    \item Utente autorizzato (Operatore di studio CdL).
\end{itemize}


\textbf{Pre-condizioni}
\begin{itemize}
    \item Ci si trova in una delle pagine precedenti (UC-2B o UC-2D o UC-2E).
    \item Si é selezionato un documento da una delle lista.
    \item Il documento é disponibile per la visualizzazione.
\end{itemize}

\textbf{Post-condizioni}
\begin{itemize}
    \item L'utente ha apportato le modifiche desiderate al documento e puó procedere con le azioni successive (salvataggio, invio, ecc.).
    \item L'utente é giá soddisfatto delle informazioni del documento e puó tornare alla pagina precedente.
\end{itemize}



\textbf{Scenario principale}
\begin{enumerate}
    \item L'utente ha caricato vari documenti e vuole ricontrollare le informazioni prima di inviarli.
    \item L'utente seleziona un documento dalla lista per visualizzarne i dettagli.
\end{enumerate}

\textbf{Scenario secondario}
\begin{enumerate}
    \item 
\end{enumerate}



\textbf{Relazioni con altri casi d'uso (\textit{include} / \textit{extend})}
\begin{itemize}
    \item \textit{include}: 
    \begin{itemize}
        \item UC-2B.1 – Visualizzazione codice documento
        \item UC-2B.2 - Visualizzazione tipologia documento
        \item UC-2D.1 – Visualizzazione destinatario
        \item UC-2B.8 - Visualizzazione nome documento originale
        \item UC-2C.1 - Visualizzazione anteprima documento
        \item UC-2E.2 - Visualizzazione percentuale confidenza
    \end{itemize}
    \item \textit{extend}: 
    \begin{itemize}
        \item UC-2C.2 – Modifica destinatario
        \item UC-2C.3 – Modifica tipologia documento
        \item UC-2C.4 – Rivaluta percentuale confidenza
        \item UC-2C.5 – Salva modifiche documento
        \item UC-2B – Visualizzazione pagina lista documenti.
        \item UC-2D - Visualizzazione pagina destinatario
        \item UC-2E - Visualizzazione pagina storico documenti
    \end{itemize}
\end{itemize}

\vspace{0.5cm}

\subsubsection{UC-2D - Visualizzazione pagina destinatario }

\begin{figure}[H]
    \centering
    \includegraphics[width=0.7\textwidth]{Diagrammi casi d'uso/UC0B.jpg}
    \caption{Didascalia dell'immagine}
\end{figure}

\textbf{Attori}
\begin{itemize}
    \item Utente autorizzato (Operatore di studio CdL).
\end{itemize}


\textbf{Pre-condizioni}
\begin{itemize}
    \item Si é nella pagina lista documenti tramite UC-2B.
\end{itemize}

\textbf{Post-condizioni}
\begin{itemize}
    \item L'utente visualizza le informazioni dettagliate del destinatario associate al documento selezionato.
    \item L'utente puó nuovamente accedere alla pagina del documento tramite UC-2C.
\end{itemize}

\textbf{Scenario principale}
\begin{enumerate}
    \item L'utente ha finito di visualizzare le informazioni del documento e seleziona l’opzione per visualizzare i dettagli del destinatario.
\end{enumerate}

\textbf{Scenario secondario}
\begin{enumerate}
    \item 
\end{enumerate}



\textbf{Relazioni con altri casi d'uso (\textit{include} / \textit{extend})}
\begin{itemize}
    \item \textit{include}: 
    \begin{itemize}
        \item UC-2B.1 – Visualizzazione codice documento
        \item UC-2D.1 – Visualizzazione destinatario
        \item UC-2D.2 - Visualizzazione codice fiscale
        \item UC-2D.3 - Visualizzazione matricola
        \item UC-2D.4 - Visualizzazione reparto
    \end{itemize}
    \item \textit{extend}: 
    \begin{itemize}
        \item UC-2B – Visualizzazione pagina lista documenti.
        \item UC-2C – Visualizzazione pagina documento
        \item UC-2E - Visualizzazione pagina storico documenti
    \end{itemize}
\end{itemize}

\vspace{0.5cm}

\subsubsection{UC-2E – Visualizzazione pagina storico documenti}

\begin{figure}[H]
    \centering
    \includegraphics[width=0.7\textwidth]{Diagrammi casi d'uso/UC0B.jpg}
    \caption{Didascalia dell'immagine}
\end{figure}

\textbf{Attori}
\begin{itemize}
    \item Utente autorizzato (Operatore di studio CdL).
\end{itemize}

\textbf{Pre-condizioni}
\begin{itemize}
    \item L'utente si trova nella pagina destinatario tramite UC-2D.
    \item L'utente ha selezionato l’opzione per inviare il documento al destinatario.
\end{itemize}

\textbf{Post-condizioni}
\begin{itemize}
    \item L'utente va alla pagina di gestione dell'invio e del rispettivo messaggio.
\end{itemize}

\textbf{Scenario principale}
\begin{enumerate}
    \item L'utente dopo aver fatto le opportune verifiche sul documento e sul destinatario seleziona l’opzione per inviare il documento.
\end{enumerate}

\textbf{Scenario secondario}
\begin{enumerate}
    \item Non sono mai stati caricati documenti, quindi la visualizzazione di dati non è possibile.
\end{enumerate}


\textbf{Relazioni con altri casi d'uso (\textit{include} / \textit{extend})}
\begin{itemize}
    \item \textit{include}: 
    \begin{itemize}
        \item UC-2B.1 – Visualizzazione codice documento
        \item UC-2E.1 - Visualizzazione stato
        \item UC-2E.2 - Visualizzazione percentuale confidenza
        \item UC-2E.3 - Visualizzazione lista distribuzione
    \end{itemize}
    \item \textit{extend}: 
    \begin{itemize}
        \item UC-2C – Visualizzazione pagina documento
        \item UC-2D - Visualizzazione pagina destinatario
        \item UC-2E.4 - Selezione documento
        \item UC-2F - Gestione messaggio
        \item UC-2E.4 - Documenti assenti.
    \end{itemize}
\end{itemize}

\vspace{0.5cm}

\subsubsection{UC-2F – Gestione messaggio}

\begin{figure}[H]
    \centering
    \includegraphics[width=0.7\textwidth]{Diagrammi casi d'uso/UC0B.jpg}
    \caption{Didascalia dell'immagine}
\end{figure}

\textbf{Attori}
\begin{itemize}
    \item Utente autorizzato (Operatore di studio CdL).
    \item Sistema di invio documenti.
\end{itemize}

\textbf{Pre-condizioni}
\begin{itemize}
    \item Ci si trova nella pagina gestione messaggio tramite UC-2E.
    \item Almeno un documento è stato selezionato per l'invio.
\end{itemize}

\textbf{Post-condizioni}
\begin{itemize}
    \item Il messaggio é pronto per essere inviato assieme ai documenti selezionati.
\end{itemize}

\textbf{Scenario principale}
\begin{enumerate}
    \item Un utente ha selezionato uno o piú documenti da inviare a uno o piú destinatari e sta prepara in messaggio da inviare assieme ai documenti.
\end{enumerate}

\textbf{Scenario secondario}
\begin{enumerate}
    \item 
\end{enumerate}



\textbf{Relazioni con altri casi d'uso (\textit{include} / \textit{extend})}
\begin{itemize}
    \item \textit{include}: 
    \begin{itemize}
        \item UC-2B.1 – Visualizzazione codice documento
        \item UC-2B.4 – Visualizzazione azienda documento
        \item UC-2B.9 - Visualizzazione data redazione documento
        \item UC-2F.6 - Visualizzazione oggetto messaggio
        \item UC-2F.7 - Visualizzazione testo messaggio
    \end{itemize}
    \item \textit{extend}: 
    \begin{itemize}
        \item UC-2F.1 - Salva template messaggio
        \item UC-2G - Tabella template salvati
        \item UC-1A.2 - Selezione tono
        \item UC-1A.3 - Selezione stile
        \item UC-2F.2 - Modifica oggetto messaggio
        \item UC-2F.3 - Modifica testo messaggio
        \item UC-2F.4 - Conferma modifica messaggio
        \item UC-2F.5 - Genera messaggio
        \item UC-2E – Visualizzazione pagina storico documenti
        \item UC-2H – Invio documento e messaggio
    \end{itemize}
\end{itemize}

\vspace{0.5cm}


\subsubsection{UC-2G - Tabella template salvati }

\begin{figure}[H]
    \centering
    \includegraphics[width=0.7\textwidth]{Diagrammi casi d'uso/UC0B.jpg}
    \caption{Didascalia dell'immagine}
\end{figure}

\textbf{Attori}
\begin{itemize}
    \item Utente autorizzato (Operatore di studio CdL).
    \item Sistema di gestione template messaggi.
\end{itemize}

\textbf{Pre-condizioni}
\begin{itemize}
    \item L'utente si trova nella pagina di gestione messaggio tramite UC-2F.
    \item Esistono uno o più template di messaggi salvati nel sistema.
\end{itemize}

\textbf{Post-condizioni}
\begin{itemize}
    \item L'utente si ritrova nella pagina di gestione messaggio con il template selezionato caricato nei campi oggetto e testo del messaggio.
\end{itemize}


\textbf{Scenario principale}
\begin{enumerate}
    \item L'utente ha bisogno di utilizzare un template di messaggio salvato per preparare il messaggio da inviare assieme ai documenti selezionati.
\end{enumerate}

\textbf{Scenario secondario}
\begin{enumerate}
    \item 
\end{enumerate}



\textbf{Relazioni con altri casi d'uso (\textit{include} / \textit{extend})}
\begin{itemize}
    \item \textit{include}: 
    \begin{itemize}
        \item UC-1B.2 – Visualizzazione tono
        \item UC-1B.3 – Visualizzazione stile 
        \item UC-2F.6 – Visualizzazione oggetto messaggio
        \item UC-2F.7 – Visualizzazione testo messaggio
    \end{itemize}
    \item \textit{extend}: 
    \begin{itemize}
        \item UC-2F – Gestione messaggio %Sarebbe carica template messaggio
        \item UC-2G.1 – Elimina template
        \item UC-2G.2 – Nessun template salvato
    \end{itemize}
\end{itemize}

\vspace{0.5cm}

\subsubsection{UC-2H – Invio documento e messaggio}

\begin{figure}[H]
    \centering
    \includegraphics[width=0.7\textwidth]{Diagrammi casi d'uso/UC0B.jpg}
    \caption{Didascalia dell'immagine}
\end{figure}

\textbf{Attori}
\begin{itemize}
    \item Utente autorizzato (Operatore di studio CdL).
    \item Sistema di invio documenti e messaggi.
\end{itemize}

\textbf{Pre-condizioni}
\begin{itemize}
    \item L'utente si trova nella pagina di gestione messaggio tramite UC-2F.
    \item Il messaggio é stato completato con oggetto, testo e documenti allegati.
\end{itemize}

\textbf{Post-condizioni}
\begin{itemize}
    \item Il messaggio e i documenti sono stati inviati con successo ai destinatari previsti.
\end{itemize}

\textbf{Scenario principale}
\begin{enumerate}
    \item L'utente ha completato la preparazione del messaggio e dei documenti da inviare.
    \item L'utente seleziona l’opzione per inviare il messaggio e i documenti ai destinatari.
\end{enumerate}

\textbf{Scenario secondario}
\begin{enumerate}
    \item 
\end{enumerate}



\textbf{Relazioni con altri casi d'uso (\textit{include} / \textit{extend})}
\begin{itemize}
    \item \textit{include}: 
    \begin{itemize}
        \item UC-2F.6 – Visualizzazione oggetto messaggio
        \item UC-2F.7 – Visualizzazione testo messaggio
        \item UC-2H.4 – Visualizzazione lista destinatari
        \item UC-2H.5 – Visualizzazione lista documenti
    \end{itemize}
    \item \textit{extend}: 
    \begin{itemize}
        \item UC-2H.1 – Allega file
        \item UC-2H.2 – Pianifica invio
        \item UC-2H.3 – Conferma invio
        \item UC-2F – Gestione messaggio %Torna a gestione messaggio
    \end{itemize}
\end{itemize}

\vspace{0.5cm}

\section{Requisiti}

\subsection{Requisiti funzionali}
Descrivono le funzionalità che il sistema deve offrire, ovvero ciò che il sistema deve fare.
Rappresentano i comportamenti osservabili del sistema in risposta alle azioni degli utenti o ad altri eventi.
\vspace{0.5cm}

\textbf{Caratterisitiche}
\begin{enumerate}
    \item Derivano direttamente dai casi d'uso.
    \item Sono specifici, misurabili e testabili.
    \item Rispondono alla domanda "Cosa deve fare il sistema?".
\end{enumerate}


\subsection{Requisiti non funzionali}
Descrivono come il sistema deve comportarsi, cioè le qualità, i vincoli e le caratteristiche del software che non riguardano le funzionalità, ma le prestazioni,
l’usabilità, la sicurezza, ecc.
\vspace{0.5cm}

\textbf{Caratterisitiche}
\begin{enumerate}
    \item Un requisito non funzionale deve essere espresso in modo tale da poter essere misurato oggettivamente.
    \item Deve essere possibile verificare tramite test, ispezioni o metriche se il requisito è stato soddisfatto.
    \item I requisiti non funzionali non devono essere in conflitto fra loro o con quelli funzionali.
\end{enumerate}

\subsection{Requisiti di dominio}
Derivano dal contesto specifico del problema che si vuole risolvere.
Sono requisiti legati all’ambiente applicativo e alle regole del dominio.
\vspace{0.5cm}

\textbf{Caratterisitiche}
\begin{enumerate}
    \item Non dipendono dalle specifiche implementazioni tecniche.
    \item Sono legati alle regole di business.
\end{enumerate}

\subsection{Requisiti utente}
Descrivono ciò che l’utente si aspetta dal sistema, spesso in termini non tecnici.
Sono più astratti rispetto ai requisiti funzionali.
\vspace{0.5cm}

\textbf{Caratterisitiche}
\begin{enumerate}
    \item Vengono trasformati in requisiti funzionali.
\end{enumerate}

\subsection{Requisiti di sistema}
Descrivono cosa il sistema nel suo complesso deve garantire.
Sono più ampi dei requisiti funzionali e includono aspetti architetturali.
\vspace{0.5cm}

\textbf{Caratterisitiche}
\begin{enumerate}
    \item Spesso derivano dai requisiti utente e dai requisiti non funzionali.
\end{enumerate}

\subsection{Requisiti opzionali, desiderabili, obbligatori}
Questa classificazione viene usata per stabilire la priorità dei requisiti.

\vspace{0.5cm}

\textbf{Caratterisitiche}
\begin{enumerate}
    \item Descrivono il livello di importanza del requisito.
\end{enumerate}


\end{document}