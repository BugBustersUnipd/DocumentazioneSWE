\documentclass[a4paper,11pt]{article}
\newcommand{\CurrentVersion}{0.2.1} % ultima versione, da cambiare ad ogni push significativo


\usepackage[utf8]{inputenc}
\usepackage[T1]{fontenc}
\usepackage[italian]{babel}
\usepackage[margin=2.5cm]{geometry}
\usepackage{graphicx}
\usepackage{booktabs}
\usepackage{setspace}
\usepackage{titlesec}
\usepackage{float}
\usepackage[table]{xcolor}
\usepackage{tabularx}
\usepackage{tcolorbox}
\usepackage{enumitem}
\usepackage[titles]{tocloft}
\usepackage[colorlinks=true,linkcolor=black,urlcolor=blue,citecolor=blue]{hyperref}
\usepackage{fancyhdr}
\usepackage{lastpage}
\usepackage{amsmath}
\usepackage{longtable}
\usepackage{tabularx}
\usepackage{array}
\usepackage{ragged2e}



\definecolor{primaryblue}{RGB}{0,102,204}
\definecolor{secondaryblue}{RGB}{51,153,255}
\definecolor{lightgray}{RGB}{245,245,245}
\definecolor{darkgray}{RGB}{100,100,100}

% TODO è il caso di inserire queste regole in un file e includerlo in tutti i documenti
\titleformat{\section}
 {\Large\bfseries\color{primaryblue}}
 {\thesection}{1em}{}

\titleformat{\subsection}
 {\large\bfseries\color{primaryblue}} % Sottosezione: colore secondaryblue
 {\thesubsection}{1em}{}

\titleformat{\subsubsection}
 {\normalsize\bfseries\color{secondaryblue}} % Sotto-sottosezione: colore secondaryblue
 {\thesubsubsection}{1em}{}

 \titleformat{\paragraph}
 {\normalsize\bfseries\color{secondaryblue}} % Sotto-sotto-sottosezione: colore secondaryblue
 {\paragraph}{1em}{}

\setlength{\LTpre}{0pt}
\setlength{\LTpost}{0pt}
\setlength{\LTleft}{0pt}
\setlength{\LTright}{0pt}
\raggedbottom

\newcolumntype{L}[1]{>{\RaggedRight\arraybackslash}p{#1}}


\pagestyle{fancy}
\fancyhf{}
\fancyhead[L]{BugBusters}
\fancyhead[R]{Analisi dei Requisiti\textsubscript{\scalebox{0.6}{\textbf{G}}}}
\fancyfoot[L]{\thepage\ di \pageref{LastPage}}
\renewcommand{\headrulewidth}{0pt}
\renewcommand{\footrulewidth}{0pt}

\setlength{\headheight}{14pt}

\setlength{\parskip}{4pt}
\setlength{\parindent}{0pt}

% --- NEL PREAMBOLO (Prima di \begin{document}) ---
\makeatletter
\renewcommand\paragraph{\@startsection{paragraph}{4}{\z@}%
            {-2.5ex\@plus -1ex \@minus -.25ex}%
            {1.25ex \@plus .25ex}%
            {\normalfont\normalsize\bfseries}}
\makeatother
\setcounter{tocdepth}{4}     % Serve per mostrare i sotto-casi nell'indice
\setcounter{secnumdepth}{4}  

% --- INIZIO DEFINIZIONI TEMPLATE DINAMICO ---

\newcounter{usecase}[subsection]
\renewcommand{\theusecase}{\Alph{usecase}}

\newcounter{subusecase}[usecase]

\newcommand{\currentUCID}{}
\newcommand{\UCPrefix}{0} 

% --- DEFINIZIONE DEL CASO D'USO (PADRE) ---
% Mantiene \subsubsection perché è comodo per la struttura principale
\newcommand{\useCase}[1]{%
    \refstepcounter{usecase}%
    \renewcommand{\currentUCID}{UC-\UCPrefix\theusecase}%
    \subsubsection{\currentUCID\ – #1}\label{\currentUCID}%
}

\newcounter{fullsubusecase}[usecase]
\renewcommand{\thefullsubusecase}{\arabic{fullsubusecase}}
\newcommand{\currentSubUCID}{}

% --- DEFINIZIONE DEL SOTTO-CASO (FIGLIO) ---
% QUESTA È LA PARTE MODIFICATA
% Non usa più \paragraph, ma costruisce il titolo manualmente
\newcommand{\subUseCase}[1]{%
    \refstepcounter{fullsubusecase}%
    \renewcommand{\currentSubUCID}{\currentUCID.\thefullsubusecase}%
    
    \par\vspace{10pt}%
    \phantomsection
    
    % Indice (Resta nero standard o segue lo stile dei link)
    \addcontentsline{toc}{paragraph}{\numberline{\thesubsubsection.\arabic{fullsubusecase}}\currentSubUCID\ – #1}%
    
    % --- MODIFICA QUI ---
    % Applica colore primaryblue e grassetto a tutto il blocco (Codice + Titolo)
    \noindent\textcolor{primaryblue}{\textbf{\currentSubUCID\ – #1}}%
    
    \label{\currentSubUCID}%
    \par\vspace{2pt}%
}

% --- FINE DEFINIZIONI TEMPLATE DINAMICO ---

\begin{document}

\begin{center}
  \thispagestyle{empty}
  \IfFileExists{../../assets/Logo.jpg}{%
    \includegraphics[width=6cm,height=3cm,keepaspectratio]{../../assets/Logo.jpg} \\[0.8cm]
  }{%
    \fbox{\parbox[c][2.5cm][c]{6cm}{\centering Logo non trovato\\(Logo.jpg)}}\\[0.5cm]
  }
  {\LARGE\bfseries BugBusters}\\[0.8cm]
  
  \rule{\textwidth}{0.5pt}\\[0.5cm]
  {\Large\bfseries Analisi dei Requisiti\textsubscript{\scalebox{0.6}{\textbf{G}}}}\\[0.3cm]
  {\large Versione \CurrentVersion}\\[0.5cm]
  \rule{\textwidth}{0.5pt}\\[0.8cm]
\end{center}

\begin{center}
\begin{tcolorbox}[colback=gray!10,width=0.8\textwidth,arc=3mm,boxrule=0.5pt]
\begin{tabular}{ll}
\textbf{Stato} & In redazione \\
\textbf{Redattori} & Leonardo Salviato \\
\textbf{Destinatari} & BugBusters \\
 & Prof. Vardanega Tullio \\
 & Prof. Cardin Riccardo \\
 & Eggon \\
\end{tabular}
\end{tcolorbox}
\end{center}

\vspace{1cm}

\begin{center}
\textbf{Descrizione}
\end{center}

\begin{center}
\begin{minipage}{0.9\textwidth}
\small
Questo documento contiene le Norme di Progetto\textsubscript{\scalebox{0.6}{\textbf{G}}} seguite dal team \textbf{BugBusters} per il progetto\textsubscript{\scalebox{0.6}{\textbf{G}}} C5 proposto dall'azienda Eggon
\end{minipage}
\end{center}



\newpage

\section*{Registro delle Modifiche}

\setlength{\extrarowheight}{2pt}
\renewcommand{\arraystretch}{1.5}
\arrayrulecolor{primaryblue}

{\footnotesize
\begin{longtable}{|>{\raggedright\arraybackslash}p{1.5cm}|>{\raggedright\arraybackslash}p{2cm}|>{\raggedright\arraybackslash}p{3cm}|>{\raggedright\arraybackslash}p{2cm}|>{\raggedright\arraybackslash}p{2cm}|>{\raggedright\arraybackslash}p{2cm}|}
\hline
\rowcolor{primaryblue!40}
\textbf{\color{white} Versione} & \textbf{\color{white} Data} & \textbf{\color{white} Descrizione} & \textbf{\color{white} Redatto} & \textbf{\color{white} Verificato\textsubscript{\scalebox{0.6}{\textbf{G}}}} & \textbf{\color{white} Approvato} \\
\hline
\endfirsthead

\hline
\rowcolor{primaryblue!40}
\textbf{\color{white} Versione} & \textbf{\color{white} Data} & \textbf{\color{white} Descrizione} & \textbf{\color{white} Redatto} & \textbf{\color{white} Verificato\textsubscript{\scalebox{0.6}{\textbf{G}}}} & \textbf{\color{white} Approvato} \\
\hline
\endhead
\rowcolor{secondaryblue!10} 0.2.4 & 14/01/2026 & Riscrittura tabella requisiti funzionali & Leonardo Salviato &  & - \\
\hline
\rowcolor{secondaryblue!10} 0.2.3 & 12/01/2026 & Riscrittura a seguito del colloqui con il professor Cardin & Leonardo Salviato &  & - \\
\hline
\rowcolor{secondaryblue!10} 0.2.2 & 09/01/2026 & Correzioni errori riportati & Leonardo Salviato &  & - \\
\hline
\rowcolor{secondaryblue!10} 0.2.1 & 06/01/2026 & Aggiunti ai termini presenti nel Glossario la G & Alberto Autiero &  & - \\
\hline
\rowcolor{secondaryblue!10} 0.2.0 & 04/01/2026 & Verifica\textsubscript{\scalebox{0.6}{\textbf{G}}} & - & Luca Slongo & - \\
\hline
\rowcolor{secondaryblue!10} 0.1.11 & 04/01/2026 & Inserimento diagrammi Casi d'uso\textsubscript{\scalebox{0.6}{\textbf{G}}} sezioni 2 e 3 & Leonardo Salviato & - & - \\
\hline
\rowcolor{secondaryblue!10} 0.1.10 & 02/01/2026 & Inserimento diagrammi Casi d'uso\textsubscript{\scalebox{0.6}{\textbf{G}}} sezioni 0 e 1 & Alberto Autiero & - & - \\
\hline
\rowcolor{secondaryblue!10} 0.1.9 & 29/12/2025 & Correzioni a Casi d'uso\textsubscript{\scalebox{0.6}{\textbf{G}}} e a paragrafi nel documento & Leonardo Salviato & - & - \\
\hline
\rowcolor{secondaryblue!10} 0.1.8 & 27/12/2025 & Aggiunti Requisiti Prestazionali\textsubscript{\scalebox{0.6}{\textbf{G}}}, di qualità\textsubscript{\scalebox{0.6}{\textbf{G}}} e di vincolo & Marco Favero & - & - \\
\hline
\rowcolor{secondaryblue!10} 0.1.7 & 23/12/2025 & Correzioni e aggiunte minime nel contenuto e nel registro delle modifiche & Linor Sadè & - & - \\
\hline
\rowcolor{secondaryblue!10} 0.1.6 & 21/12/2025 & Correzione attori, precondizioni\textsubscript{\scalebox{0.6}{\textbf{G}}}, postcondizioni\textsubscript{\scalebox{0.6}{\textbf{G}}} & Leonardo Salviato & - & - \\
\hline
\rowcolor{secondaryblue!10} 0.1.5 & 19/12/2025 & Scritta sezione 3, rimozione trigger, finita tabella requisisti funzionali & Marco Piro & - & - \\
\hline
\rowcolor{secondaryblue!10} 0.1.4 & 16/12/2025 & Correzione errori, aggiunta index e sistemata visualizzazione tabelle nei Casi d'uso\textsubscript{\scalebox{0.6}{\textbf{G}}}, sistemazione indicizzazione & Leonardo Salviato & - & - \\
\hline
\rowcolor{secondaryblue!10} 0.1.3 & 13/12/2025 & Aggiunta sotto-Casi d'uso\textsubscript{\scalebox{0.6}{\textbf{G}}} sezione due, riscritta struttura Requisiti e relative tabelle & Marco Favero & - & - \\
\hline
\rowcolor{secondaryblue!10} 0.1.2 & 11/12/2025 & Aggiunto Modulo\textsubscript{\scalebox{0.6}{\textbf{G}}} 3 & Marco Piro & - & - \\
\hline
\rowcolor{secondaryblue!10} 0.1.1 & 10/12/2025 & Rinominazione Casi d'uso\textsubscript{\scalebox{0.6}{\textbf{G}}}, aggiunta sotto-Casi d'uso\textsubscript{\scalebox{0.6}{\textbf{G}}} alle sezioni 0 e 1. & Leonardo Salviato & - & - \\
\hline
\rowcolor{secondaryblue!10} 0.1.0 & 07/12/2025 & verifica\textsubscript{\scalebox{0.6}{\textbf{G}}} dei primi Casi d'uso\textsubscript{\scalebox{0.6}{\textbf{G}}} & - & Linor Sadè & - \\
\hline
\rowcolor{secondaryblue!10} 0.0.9 & 06/12/2025 & Aggiunti a ogni sezione scenari secondari, relativi Casi d'uso\textsubscript{\scalebox{0.6}{\textbf{G}}} (eccezioni e varianti) e trigger. & Alberto Autiero & - & - \\
\hline
\rowcolor{secondaryblue!10} 0.0.8 & 05/12/2025 & Riscrittura Casi d'uso\textsubscript{\scalebox{0.6}{\textbf{G}}} con aggiornato grado di precisione (sezione 2). & Leonardo Salviato & - & - \\
\hline
\rowcolor{secondaryblue!10} 0.0.7 & 04/12/2025 & Riscrittura Casi d'uso\textsubscript{\scalebox{0.6}{\textbf{G}}} con aggiornato grado di precisione (sezioni 0 e 1). & Alberto Pignat & - & - \\
\hline
\rowcolor{secondaryblue!10} 0.0.6 & 03/12/2025 & Aggiunti Casi d'uso\textsubscript{\scalebox{0.6}{\textbf{G}}} sezione 2, aggiunta varianti/exceptions sezione 2, scritta possibile struttura Requisiti. & Leonardo Salviato & - & - \\
\hline
\rowcolor{secondaryblue!10} 0.0.5 & 02/12/2025 & Aggiunto schema attori & Marco Piro & - & - \\
\hline
\rowcolor{secondaryblue!10} 0.0.4 & 30/11/2025 & Sistemazione Attori\textsubscript{\scalebox{0.6}{\textbf{G}}}. & Marco Piro & - & - \\
\hline
\rowcolor{secondaryblue!10} 0.0.3 & 29/11/2025 & Correzione Casi d'uso\textsubscript{\scalebox{0.6}{\textbf{G}}} e aggiunta schemi. & Leonardo Salviato & - & - \\
\hline
\rowcolor{secondaryblue!10} 0.0.2 & 25/11/2025 & Riscrittura della prima stesura e modifica Casi d'uso\textsubscript{\scalebox{0.6}{\textbf{G}}}. & Luca Slongo & - & - \\
\hline
\rowcolor{secondaryblue!10} 0.0.1 & 16/11/2025 & Prima stesura della struttura del documento. & Leonardo Salviato e Linor Sadè & - & - \\
\hline
\end{longtable}
}

\vfill
\begin{center}
2 di \pageref{LastPage}
\end{center}

\newpage

\tableofcontents

\newpage
\listoftables
\listoffigures


\newpage

\section{Introduzione}

\subsection{Scopo del documento}
Il documento di analisi dei Requisiti\textsubscript{\scalebox{0.6}{\textbf{G}}} ha lo scopo di definire in maniera precisa e dettagliata i Requisiti funzionali\textsubscript{\scalebox{0.6}{\textbf{G}}} e non funzionali del sistema software da sviluppare.
A seguito delle nuove decisioni progettuali rispetto alle proposte del capitolato\textsubscript{\scalebox{0.6}{\textbf{G}}}, il sistema non sarà inizialmente integrato nella piattaforma NEXUM, ma verrà realizzato come 
\textbf{applicazione standalone}, cioè indipendente dal sistema già esistente. 
Tale applicazione implementerà i moduli "AI Assistant\textsubscript{\scalebox{0.6}{\textbf{G}}} Generativo" e "AI Co-Pilot\textsubscript{\scalebox{0.6}{\textbf{G}}} per i CdL\textsubscript{\scalebox{0.6}{\textbf{G}}}" in un ambiente isolato, così da consentire una fase di sviluppo, test\textsubscript{\scalebox{0.6}{\textbf{G}}} e validazione\textsubscript{\scalebox{0.6}{\textbf{G}}} più controllata. 
Eggon avrà la possibilità di valutare il prototipo standalone ed eventualmente procedere con l'integrazione nella piattaforma NEXUM.
Il documento include una descrizione approfondita dei Casi d'uso\textsubscript{\scalebox{0.6}{\textbf{G}}}, che costituiscono la principale fonte dei Requisiti finali. Per agevolare la comprensione, verranno utilizzati anche i \textbf{diagrammi dei Casi d'uso\textsubscript{\scalebox{0.6}{\textbf{G}}}}, che visualizzano le interazioni tra utenti e sistema.
Questo documento rappresenta il riferimento fondamentale per la progettazione, l'implementazione\textsubscript{\scalebox{0.6}{\textbf{G}}} e il collaudo dell'applicazione standalone, 
assicurando che essa soddisfi pienamente le esigenze del committente\textsubscript{\scalebox{0.6}{\textbf{G}}} e gli obiettivi formativi del progetto\textsubscript{\scalebox{0.6}{\textbf{G}}}.

I Requisiti identificati sono classificati nelle seguenti categorie:
\begin{itemize}
    \item \textbf{Obbligatori}: necessari e imprescindibili per garantire il corretto funzionamento dell'applicazione standalone;
    \item \textbf{Desiderabili}: non strettamente necessari, ma capaci di migliorare l'esperienza utente o l'efficienza\textsubscript{\scalebox{0.6}{\textbf{G}}} del sistema;
    \item \textbf{Opzionali}: funzionalità\textsubscript{\scalebox{0.6}{\textbf{G}}} aggiuntive utili per estensioni future, in particolare in vista della possibile integrazione con NEXUM.
\end{itemize}

Il documento è rivolto ai seguenti destinatari:
\begin{itemize}
    \item Il \textbf{committente\textsubscript{\scalebox{0.6}{\textbf{G}}}}, che potrà verifica\textsubscript{\scalebox{0.6}{\textbf{G}}}re che i Requisiti siano stati compresi e documentati correttamente;
    \item La \textbf{proponente\textsubscript{\scalebox{0.6}{\textbf{G}}}} (Eggon), che potrà utilizzare questo documento per monitorare l'aderenza del progetto\textsubscript{\scalebox{0.6}{\textbf{G}}} alle specifiche concordate.
    \item Il \textbf{team di progettisti e programmatori}, che utilizzerà questa analisi come base per la realizzazione del sistema;
    \item Il \textbf{team di verifica\textsubscript{\scalebox{0.6}{\textbf{G}}}tori}, che impiegherà il presente documento per definire i casi di test\textsubscript{\scalebox{0.6}{\textbf{G}}} e validare il comportamento del prodotto.
\end{itemize}

\newpage

\subsection{Prospettiva del prodotto}

Il prodotto che BugBusters intende sviluppare è una versione standalone dei moduli "AI Assistant\textsubscript{\scalebox{0.6}{\textbf{G}}} Generativo" e "AI Co-Pilot\textsubscript{\scalebox{0.6}{\textbf{G}}} per i CdL\textsubscript{\scalebox{0.6}{\textbf{G}}}", svincolata dalla piattaforma NEXUM. Tale applicazione costituirà un prototipo funzionale in grado di operare autonomamente e di implementare le principali funzionalità\textsubscript{\scalebox{0.6}{\textbf{G}}} richieste dal committente\textsubscript{\scalebox{0.6}{\textbf{G}}}, senza dipendere dalla piattaforma esistente.

Tale scelta deriva dalla necessità di rispettare le politiche di privacy e sicurezza dei dati della proponente\textsubscript{\scalebox{0.6}{\textbf{G}}}, le quali risultano incompatibili con la richiesta del committente\textsubscript{\scalebox{0.6}{\textbf{G}}} di rendere il prodotto accessibile al pubblico.

L'app standalone permetterà di testare e consolidare le funzionalità\textsubscript{\scalebox{0.6}{\textbf{G}}} richieste, offrendo un ambiente controllato che faciliti la sperimentazione e lo sviluppo incrementale. Questa fase costituirà la base per un'eventuale integrazione futura con la piattaforma NEXUM, la quale fornirà un ecosistema HR completo e dotato di servizi quali la messaggistica top-down, la timbratura digitale, la gestione delle anagrafiche e dei ruoli, e la collaborazione con gli studi dei Consulenti del Lavoro (CdL)\textsubscript{\scalebox{0.6}{\textbf{G}}}.

Per questo motivo, l'architettura software prevede l'utilizzo di mock o simulazioni delle componenti NEXUM. Tale approccio garantisce alla proponente\textsubscript{\scalebox{0.6}{\textbf{G}}} una futura e agevole integrazione del sistema, evitando al contempo che il team di sviluppo debba reimplementare funzionalità\textsubscript{\scalebox{0.6}{\textbf{G}}} già esistenti ma non strettamente necessarie nella fase attuale.

I moduli saranno quindi concepiti in modo modulare per permettere all'applicazione, in un'eventuale integrazione futura con NEXUM, di inserirsi nell'architettura esistente come componente riutilizzabile e scalabile. L'integrazione includerà l'adattamento delle API, l'allineamento della gestione utenti e la centralizzazione dei dati all'interno dell'infrastruttura NEXUM.

\subsection{Funzioni del prodotto}

Qui di seguito le funzionalità\textsubscript{\scalebox{0.6}{\textbf{G}}} del prodotto descritte in breve e divise per i due moduli che andranno sviluppati:

\textbf{Modulo\textsubscript{\scalebox{0.6}{\textbf{G}}} AI Assistant\textsubscript{\scalebox{0.6}{\textbf{G}}} Generativo}
\begin{itemize}
    \item \textbf{Generazione di contenuti tramite AI}: 
    generazione di titolo, testo e immagine di copertina a partire da un prompt\textsubscript{\scalebox{0.6}{\textbf{G}}}, 
    con possibilità di selezionare tono e stile.
    \item \textbf{Salvataggio locale}:
    gestione interna di prompt\textsubscript{\scalebox{0.6}{\textbf{G}}}, contenuti generati, immagini e valutazioni, 
    tramite archivio locale dedicato all'app standalone, indipendente dalla piattaforma NEXUM.
    
    \item \textbf{Sistema di rating}: 
    valutazione della qualità\textsubscript{\scalebox{0.6}{\textbf{G}}} dei contenuti generati dall'AI, utile per analisi interne e miglioramento continuo.

    \item \textbf{Gestione dei prompt\textsubscript{\scalebox{0.6}{\textbf{G}}}}: 
    storico dei prompt\textsubscript{\scalebox{0.6}{\textbf{G}}} utilizzati con possibilità di riutilizzo, duplicazione e ricerca interna.

    \item \textbf{Dashboard\textsubscript{\scalebox{0.6}{\textbf{G}}} standalone}: 
    visualizzazione e gestione di storico, filtri, ricerca e analisi delle interazioni con l'AI generativa.

    \item \textbf{Gestione delle immagini}: 
    possibilità di caricare immagini dall'utente o di generarle tramite AI, con salvataggio locale.

\end{itemize}

\newpage

\textbf{Modulo\textsubscript{\scalebox{0.6}{\textbf{G}}} AI Co-Pilot\textsubscript{\scalebox{0.6}{\textbf{G}}} per i CdL\textsubscript{\scalebox{0.6}{\textbf{G}}}}
\begin{itemize}
    \item \textbf{Upload e gestione documentale}: 
    possibilità di caricare documenti (PDF, ZIP, ecc.), salvarli localmente e gestirne lo stato di elaborazione.
    
    \item \textbf{Riconoscimento automatico della tipologia di documento}: 
    classificazione tramite AI (cedolini\textsubscript{\scalebox{0.6}{\textbf{G}}}, CU, comunicazioni, lettere, moduli da firmare, ecc.) 
    sfruttando modelli OCR e classificatori addestrati.
    
    \item \textbf{Estrazione dei destinatari}: 
    riconoscimento automatico di informazioni contenute nei documenti 
    (nome, cognome, codice fiscale, matricola, reparto) tramite tecniche AI di entity extraction.
    
    \item \textbf{Split dei documenti massivi}: 
    suddivisione automatica dei documenti multi-destinatario (es. cedolini massivi\textsubscript{\scalebox{0.6}{\textbf{G}}}) 
    in documenti singoli, ognuno associato al proprio destinatario riconosciuto.
    
    \item \textbf{Revisione manuale (Human-in-the-Loop)}: 
    interfaccia\textsubscript{\scalebox{0.6}{\textbf{G}}} dedicata per verifica\textsubscript{\scalebox{0.6}{\textbf{G}}}re, correggere o confermare i risultati ottenuti dall'AI 
    in ogni fase (classificazione, destinatari, split).
    
    \item \textbf{Creazione di messaggi e liste di distribuzione}: 
    generazione automatica di bozze di messaggi e liste di destinatari derivanti dai documenti processati.
    
    \item \textbf{Tracciamento locale}: 
    storico delle operazioni effettuate (upload, riconoscimento, revisioni, esportazioni), 
    utile per audit interni e analisi del flusso documentale.
\end{itemize}  
\textbf{Altre funzionalità\textsubscript{\scalebox{0.6}{\textbf{G}}} dell'applicazione:}
\begin{itemize}
    \item \textbf{Gestione utenti}: 
    registrazione, autenticazione, gestione del profilo e configurazione dei parametri AI 
    (per utenti privilegiati come Amministratore\textsubscript{\scalebox{0.6}{\textbf{G}}} o editor avanzati).
    \item \textbf{Analisi e reportistica}:
    Dashboard\textsubscript{\scalebox{0.6}{\textbf{G}}} di monitoraggio delle performance dei moduli AI per consentire valutazioni semplici su efficacia\textsubscript{\scalebox{0.6}{\textbf{G}}} ed efficienza\textsubscript{\scalebox{0.6}{\textbf{G}}} dei moduli. L'appicazione consentirà inoltre
    la possibilità di rilasciare miglioramenti in seguito ad uun intervento umano. 
\end{itemize}

Queste funzionalità\textsubscript{\scalebox{0.6}{\textbf{G}}} permetteranno all'applicazione standalone di essere completamente operativa e autonoma nei due moduli (AI Assistant\textsubscript{\scalebox{0.6}{\textbf{G}}} Generativo e AI Co-Pilot\textsubscript{\scalebox{0.6}{\textbf{G}}} per i CdL\textsubscript{\scalebox{0.6}{\textbf{G}}}). 
In una fase successiva, tali componenti saranno progettati per essere integrati nella piattaforma NEXUM, 
consentendo così un'evoluzione verso un ecosistema HR completo, scalabile e basato su automazioni intelligenti.

\newpage

\subsection{Caratterisitiche dell'utente}


Gli utilizzatori finali dell'applicazione standalone non appartengono a un'unica categoria specifica: 
l'obiettivo è quello di progettare moduli intelligenti e interoperabili per essere in futuro integrati nella piattaforma NEXUM. Tale ecosistema è concepito per rispondere alle esigenze di un ampio spettro di organizzazioni e professionisti nel settore delle risorse umane e della consulenza del lavoro.

In generale, è possibile affermare che gli utenti finali sono coloro che necessitano di uno strumento scalabile, 
intelligente e semplice da utilizzare per generare contenuti tramite AI e per gestire flussi documentali complessi con il supporto del Modulo\textsubscript{\scalebox{0.6}{\textbf{G}}} Co-Pilot\textsubscript{\scalebox{0.6}{\textbf{G}}}.
Rientrano in questa categoria:

\begin{itemize}
    \item \textbf{Responsabili e Amministratore\textsubscript{\scalebox{0.6}{\textbf{G}}} HR}, che necessitano di strumenti avanzati per la creazione di comunicazioni interne, 
    la gestione dei contenuti generativi e l'analisi delle produzioni.

    \item \textbf{Consulenti del Lavoro (CdL)\textsubscript{\scalebox{0.6}{\textbf{G}}} e personale amministrativo}, che richiedono un sistema in grado di caricare, riconoscere, suddividere e preparare documenti per la distribuzione ai destinatari.

    \item \textbf{Dipendenti e collaboratori} (in fase integrata), che potranno interagire con la piattaforma NEXUM per consultare documenti e comunicazioni, 
    pur non essendo utenti della versione standalone.

    \item \textbf{Manager aziendali}, interessati a monitorare la consistenza delle comunicazioni e l'efficienza\textsubscript{\scalebox{0.6}{\textbf{G}}} dei processi documentali, 
    sia nella versione standalone che nella futura integrazione.
\end{itemize}

In sintesi, il prodotto è rivolto a organizzazioni di varie dimensioni — in particolare aziende medio-grandi e studi professionali — 
che necessitano di strumenti intelligenti per la creazione di contenuti, 
la gestione automatizzata dei documenti e la collaborazione con gli studi dei Consulenti del Lavoro.
L'app standalone funge da primo passo verso una piattaforma HR completa, modulare e potenziata dall'AI.


\subsection{Definizioni, acronimi e abbreviazioni (Glossario\textsubscript{\scalebox{0.6}{\textbf{G}}})}
Per tutte le definizioni, acronimi e abbreviazioni utilizzati in questo documento, si faccia
riferimento al \textbf{Glossario\textsubscript{\scalebox{0.6}{\textbf{G}}}}, fornito come documento separato, che contiene tutte le spiegazioni
necessarie per garantire una comprensione uniforme dei termini tecnici e dei concetti
rilevanti per il progetto\textsubscript{\scalebox{0.6}{\textbf{G}}}.

\newpage

\subsection{Riferimenti}

\subsubsection{Riferimenti normativi}
\begin{itemize}
\item \textbf{capitolatoverifica\textsubscript{\scalebox{0.6}{\textbf{G}}}
 d'appalto C5: Nexum - Piattaforma di consulenza e documentazione previdenziale}\\
\url{https://www.math.unipd.it/~tullio/IS-1/2025/progetto/C5.pdf}
\end{itemize}

\subsubsection{Riferimenti informativi}
\begin{itemize}
\item \textbf{Glossario\textsubscript{\scalebox{0.6}{\textbf{G}}}
:}\\
\url{https://github.com/BugBustersUnipd/DocumentazioneSWE/blob/main//RTB/GLOSSARIO/Glossario.pdf}
\end{itemize}



\section{Casi d'uso\textsubscript{\scalebox{0.6}{\textbf{G}}}}
\subsection{Introduzione}
I Casi d'uso\textsubscript{\scalebox{0.6}{\textbf{G}}} si compongono di un grafico UML e una descrizione testuale che permette di
comprendere al meglio cosa il prodotto deve fornire. La descrizione testuale, in particolar
modo, dovrà contenere le informazioni sotto presenti, salvo i casi in cui lo
specifico campo non risulti rilevante (ad esempio, un caso d'uso\textsubscript{\scalebox{0.6}{\textbf{G}}} che non prevede la
possibilità di errori non avrà scenari alternativi):

\begin{itemize}
    \item \textbf{Attori\textsubscript{\scalebox{0.6}{\textbf{G}}}}: Sono coloro che interagiscono attivamente con il sistema e
    svolgono l'azione indicata dal caso d'uso
    \item \textbf{Attori\textsubscript{\scalebox{0.6}{\textbf{G}}} secondari}: Sono coloro che interagiscono passivamente con il sistema
    \item \textbf{Precondizioni\textsubscript{\scalebox{0.6}{\textbf{G}}}}: Lista di elementi che sono necessari affinchè l'attore\textsubscript{\scalebox{0.6}{\textbf{G}}} possa
    compiere l'azione indicata dal caso d'uso
    \item \textbf{Postcondizioni\textsubscript{\scalebox{0.6}{\textbf{G}}}}: Lista di elementi che descrivono come il sistema risulta
    essere internamente cambiato dopo che l'attore\textsubscript{\scalebox{0.6}{\textbf{G}}} ha effettuato
    l'azione prevista dal caso d'uso
    \item \textbf{Scenario principale\textsubscript{\scalebox{0.6}{\textbf{G}}}}: Descrizione ragionevole delle operazioni che l'attore\textsubscript{\scalebox{0.6}{\textbf{G}}} deve
    fare per compiere l'azione descritta dal caso d'uso
    \item \textbf{Scenario secondario\textsubscript{\scalebox{0.6}{\textbf{G}}}} (scenari alternativi): Descrizione ragionevole degli eventi che possono accadere
    qualora una delle operazioni descritte nello scenario
    principale non vada a buon fine
    \item \textbf{Inclusioni}: Casi d'uso\textsubscript{\scalebox{0.6}{\textbf{G}}} ulteriori che l'attore\textsubscript{\scalebox{0.6}{\textbf{G}}} deve compiere per realizzare
    il caso d'uso attualmente descritto
    \item \textbf{Estensioni}: Casi d'uso\textsubscript{\scalebox{0.6}{\textbf{G}}} ulteriori che possono realizzarsi durante
    l'esecuzione delle operazioni del caso d'uso principale e rappresentano uno scenario alternativo
    \item \textbf{Generalizzazioni}: Casi d'uso\textsubscript{\scalebox{0.6}{\textbf{G}}} meno specifici che racchiudono il caso d'uso attualmente descritto
    
\end{itemize}
Motivazioni che portano l'attore\textsubscript{\scalebox{0.6}{\textbf{G}}} a svolgere l'azione descritta
dal caso d'uso. Non sempre disponibile in quanto il caso
d'uso potrebbe essere incluso da un altro caso d'uso principale.

\subsection{Attori\textsubscript{\scalebox{0.6}{\textbf{G}}}}
Nella \hyperref[fig:1]{figura~1} sono riportati gli attori considerati in questa analisi dei Requisiti\textsubscript{\scalebox{0.6}{\textbf{G}}}.

Si fa presente che alcuni attori hanno senso nel contesto dell'applicazione che implementa l'autenticazione e la gestione dei ruoli. Tuttavia tale funzionalità\textsubscript{\scalebox{0.6}{\textbf{G}}}, come verrà descritto più avanti, è requisito\textsubscript{\scalebox{0.6}{\textbf{G}}} opzionale, pertanto gli attori devono essere considerati semplice utente se letti in un contesto in cui tale funzionalità\textsubscript{\scalebox{0.6}{\textbf{G}}} non è implementata. 
\begin{figure}[H]
    \centering
    \includegraphics[width=0.9\textwidth]{Diagrammi casi d'uso/diagramma_attori.jpg}
    \caption{Diagramma degli attori principali}
    \label{fig:1}
\end{figure}
\begin{itemize}
    \item \textbf{Utente}: Rappresenta un utente che utilizza il sistema.
    \item \textbf{Utente non autenticato}: Rappresenta un utente che non ha effettuato l'accesso al sistema.
    \item \textbf{Redattore\textsubscript{\scalebox{0.6}{\textbf{G}}}}: Utente con la capacitá di generare post
    \item \textbf{Data Analyst\textsubscript{\scalebox{0.6}{\textbf{G}}}}: Figura incaricata di monitorare le prestazioni. Accede alle Dashboard\textsubscript{\scalebox{0.6}{\textbf{G}}} di analisi per consultare le statistiche di utilizzo, i rating di qualità\textsubscript{\scalebox{0.6}{\textbf{G}}} dei contenuti generati e i KPI del riconoscimento documentale.
    \item \textbf{Amministratore\textsubscript{\scalebox{0.6}{\textbf{G}}}}: Gestisce la configurazione tecnica dell'applicazione standalone. Si occupa della creazione degli utenti, della gestione dei ruoli e della configurazione dei parametri globali dell'AI (es. prompt\textsubscript{\scalebox{0.6}{\textbf{G}}} di sistema o soglie di confidenza).
    \item \textbf{Operatore Studio CdL}: È l'utente principale del Modulo\textsubscript{\scalebox{0.6}{\textbf{G}}} AI Co-Pilot\textsubscript{\scalebox{0.6}{\textbf{G}}}. Si occupa di caricare i flussi documentali (es. cedolini massivi\textsubscript{\scalebox{0.6}{\textbf{G}}}), supervisionare il riconoscimento automatico (validazione\textsubscript{\scalebox{0.6}{\textbf{G}}} Human-in-the-Loop) e gestire le liste di distribuzione.
    \item \textbf{AI Post Generator}: Modello AI esterno utilizzato per la creazione di contenuti all'interno del Modulo\textsubscript{\scalebox{0.6}{\textbf{G}}} AI-assistant
    \item \textbf{AI Analyst}: Modello AI esterno utilizzato per l'analisi dei documenti all'interno del Modulo\textsubscript{\scalebox{0.6}{\textbf{G}}} AI-Copilot
\end{itemize}


\subsection{Lista Casi d'uso\textsubscript{\scalebox{0.6}{\textbf{G}}}}

\text{L'elenco dei Casi d'uso\textsubscript{\scalebox{0.6}{\textbf{G}}} sarà diviso in tre parti:}
\begin{itemize}
    \item 0 - Casi d'uso\textsubscript{\scalebox{0.6}{\textbf{G}}} per la gestione utenti e autenticazione
    \item 1 - Casi d'uso\textsubscript{\scalebox{0.6}{\textbf{G}}} per il Modulo\textsubscript{\scalebox{0.6}{\textbf{G}}} "AI Assistant\textsubscript{\scalebox{0.6}{\textbf{G}}} Generativo"
    \item 2 - Casi d'uso\textsubscript{\scalebox{0.6}{\textbf{G}}} per il Modulo\textsubscript{\scalebox{0.6}{\textbf{G}}} "AI Co-Pilot\textsubscript{\scalebox{0.6}{\textbf{G}}}"
\end{itemize}


\subsection{Sezione 0 – Applicazione standalone}

\renewcommand{\UCPrefix}{0}

\useCase{Registrazione nuovo utente}{

\begin{figure}[H]
    \centering
    \includegraphics[width=1\textwidth]{Diagrammi casi d'uso/UC0Aa.jpg}
    \caption{Diagramma del caso d'uso UC-0A – Registrazione nuovo utente (extend)}
\end{figure}

\begin{figure}[H]
    \centering
    \includegraphics[width=1\textwidth]{Diagrammi casi d'uso/UC0Ab.jpg}
    \caption{Diagramma del caso d'uso UC-0A – Registrazione nuovo utente (include)}
\end{figure}


\textbf{Attori\textsubscript{\scalebox{0.6}{\textbf{G}}}}
\begin{itemize}
    \item Utente non autenticato
\end{itemize}

\textbf{Pre-condizioni\textsubscript{\scalebox{0.6}{\textbf{G}}}}
\begin{itemize}
    \item L'utente non ha una sessione attiva.
    \item L'utente non è ancora registrato nel sistema (l'e-mail inserita non risulta già presente).
\end{itemize}

\textbf{Post-condizioni\textsubscript{\scalebox{0.6}{\textbf{G}}}}
\begin{itemize}
    \item Esiste un nuovo account utente registrato nel sistema.
    \item L'utente può effettuare il login utilizzando le credenziali appena create.
\end{itemize}

\textbf{Scenario principale\textsubscript{\scalebox{0.6}{\textbf{G}}}}
\begin{enumerate}
    \item L'utente accede alla schermata di registrazione dell'applicazione standalone.
    \item L'utente inserisce i dati richiesti (ad esempio: nome, cognome, e-mail, password).
    \item Il sistema verifica\textsubscript{\scalebox{0.6}{\textbf{G}}} la correttezza formale dei dati inseriti (es. formato e-mail, forza della password).
    \item Il sistema controlla che l'indirizzo e-mail non sia già associato a un account esistente.
    \item In caso di esito positivo, il sistema crea un nuovo account utente e lo memorizza nel proprio archivio.
    \item Il sistema conferma l'avvenuta registrazione e può opzionalmente eseguire il login automatico del nuovo utente.
\end{enumerate}

\textbf{Scenario secondario\textsubscript{\scalebox{0.6}{\textbf{G}}}}
\begin{enumerate}
    \item Nell'inserimento dei dati, uno o più campi non rispettano i Requisiti di validità (es. e-mail non valida, password debole).
    \item Il sistema mostra un messaggio di errore specifico per il campo non valido.
    \item L'utente può correggere i dati e ripetere l'inserimento.
\end{enumerate}

\textbf{Relazioni con altri Casi d'uso\textsubscript{\scalebox{0.6}{\textbf{G}}}}
\begin{itemize}
    \item \textit{include}:
    \begin{itemize}
        \item UC-0A.1 - Inserimento email
        \item UC-0A.2 - Inserimento password
        \item UC-0A.3 - Inserimento username
        \item UC-0A.4 - Inserimento nome 
        \item UC-0A.5 - Inserimento cognome
        \item UC-0A.6 - Inserimento matricola
    \end{itemize}
    \item \textit{extend}: 
    \begin{itemize}
        \item UC-0B – Login / Autenticazione utente 
        \item UC-0A.7 – email non valida.
        \item UC-0A.8 – password non valida.
        \item UC-0A.9 – email già registrata.
        \item UC-0A.10 – username già registrato.
        \item UC-0A.11 – matricola già registrata.
        \item UC-0A.12 – matricola non valida.
    \end{itemize}
\end{itemize}

\vspace{0.5cm}

}

\subUseCase{Inserimento email}

\textbf{Attori\textsubscript{\scalebox{0.6}{\textbf{G}}}}
\begin{itemize}
    \item Utente
\end{itemize}

\textbf{Pre-condizioni\textsubscript{\scalebox{0.6}{\textbf{G}}}}
\begin{itemize}
    \item Casella di testo vuota
\end{itemize}

\textbf{Post-condizioni\textsubscript{\scalebox{0.6}{\textbf{G}}}}
\begin{itemize}
    \item Casella di testo contenente l'email inserita
\end{itemize}

\textbf{Scenario principale\textsubscript{\scalebox{0.6}{\textbf{G}}}}
\begin{enumerate}
    \item L'utente inserisce la propria email nell'apposita casella di testo
\end{enumerate}

\vspace{0.5cm}

\subUseCase{Inserimento password}

\textbf{Attori\textsubscript{\scalebox{0.6}{\textbf{G}}}}
\begin{itemize}
    \item Utente
\end{itemize}

\textbf{Pre-condizioni\textsubscript{\scalebox{0.6}{\textbf{G}}}}
\begin{itemize}
    \item Casella di testo vuota
\end{itemize}

\textbf{Post-condizioni\textsubscript{\scalebox{0.6}{\textbf{G}}}}
\begin{itemize}
    \item Casella di testo contenente la password inserita
\end{itemize}

\textbf{Scenario principale\textsubscript{\scalebox{0.6}{\textbf{G}}}}
\begin{enumerate}
    \item L'utente inserisce la propria password nell'apposita casella di testo
\end{enumerate}


\vspace{0.5cm}

\subUseCase{Inserimento username}

\textbf{Attori\textsubscript{\scalebox{0.6}{\textbf{G}}}}
\begin{itemize}
    \item Utente
\end{itemize}

\textbf{Pre-condizioni\textsubscript{\scalebox{0.6}{\textbf{G}}}}
\begin{itemize}
    \item Casella di testo vuota
\end{itemize}

\textbf{Post-condizioni\textsubscript{\scalebox{0.6}{\textbf{G}}}}
\begin{itemize}
    \item Casella di testo contenente l'username inserito
\end{itemize}

\textbf{Scenario principale\textsubscript{\scalebox{0.6}{\textbf{G}}}}
\begin{enumerate}
    \item L'utente inserisce il proprio username nell'apposita casella di testo
\end{enumerate}

\vspace{0.5cm}

\subUseCase{Inserimento nome}

\textbf{Attori\textsubscript{\scalebox{0.6}{\textbf{G}}}}
\begin{itemize}
    \item Utente
\end{itemize}

\textbf{Pre-condizioni\textsubscript{\scalebox{0.6}{\textbf{G}}}}
\begin{itemize}
    \item Casella di testo vuota
\end{itemize}

\textbf{Post-condizioni\textsubscript{\scalebox{0.6}{\textbf{G}}}}
\begin{itemize}
    \item Casella di testo contenente il nome inserito
\end{itemize}

\textbf{Scenario principale\textsubscript{\scalebox{0.6}{\textbf{G}}}}
\begin{enumerate}
    \item L'utente inserisce il proprio nome nell'apposita casella di testo
\end{enumerate}

\vspace{0.5cm}

\newpage

\subUseCase{Inserimento cognome}

\textbf{Attori\textsubscript{\scalebox{0.6}{\textbf{G}}}}
\begin{itemize}
    \item Utente
\end{itemize}

\textbf{Pre-condizioni\textsubscript{\scalebox{0.6}{\textbf{G}}}}
\begin{itemize}
    \item Casella di testo vuota
\end{itemize}

\textbf{Post-condizioni\textsubscript{\scalebox{0.6}{\textbf{G}}}}
\begin{itemize}
    \item Casella di testo contenente il cognome inserito
\end{itemize}

\textbf{Scenario principale\textsubscript{\scalebox{0.6}{\textbf{G}}}}
\begin{enumerate}
    \item L'utente inserisce il proprio cognome nell'apposita casella di testo
\end{enumerate}


\vspace{0.5cm}

\subUseCase{Inserimento matricola}

\textbf{Attori\textsubscript{\scalebox{0.6}{\textbf{G}}}}
\begin{itemize}
    \item Utente
\end{itemize}

\textbf{Pre-condizioni\textsubscript{\scalebox{0.6}{\textbf{G}}}}
\begin{itemize}
    \item Casella di testo vuota
\end{itemize}

\textbf{Post-condizioni\textsubscript{\scalebox{0.6}{\textbf{G}}}}
\begin{itemize}
    \item Casella di testo contenente la matricola inserita
\end{itemize}

\textbf{Scenario principale\textsubscript{\scalebox{0.6}{\textbf{G}}}}
\begin{enumerate}
    \item L'utente inserisce la propria matricola nell'apposita casella di testo
\end{enumerate}

\vspace{0.5cm}

\subUseCase{Email non valida}

\textbf{Attori\textsubscript{\scalebox{0.6}{\textbf{G}}}}
\begin{itemize}
    \item Utente
\end{itemize}

\textbf{Pre-condizioni\textsubscript{\scalebox{0.6}{\textbf{G}}}}
\begin{itemize}
    \item L'utente ha inserito dei caratteri nel campo email
\end{itemize}

\textbf{Post-condizioni\textsubscript{\scalebox{0.6}{\textbf{G}}}}
\begin{itemize}
    \item Viene visualizzato un messaggio di errore relativo al formato non valido dell'email
\end{itemize}

\newpage

\textbf{Scenario principale\textsubscript{\scalebox{0.6}{\textbf{G}}}}
\begin{enumerate}
    \item L'utente inserisce un indirizzo email che non rispetta il formato standard (es. manca la chiocciola o il dominio)
    \item Il sistema rileva che il formato non è corretto
\end{enumerate}

\vspace{0.5cm}

\subUseCase{Password non valida}

\textbf{Attori\textsubscript{\scalebox{0.6}{\textbf{G}}}}
\begin{itemize}
    \item Utente
\end{itemize}

\textbf{Pre-condizioni\textsubscript{\scalebox{0.6}{\textbf{G}}}}
\begin{itemize}
    \item L'utente ha inserito dei caratteri nel campo password
\end{itemize}

\textbf{Post-condizioni\textsubscript{\scalebox{0.6}{\textbf{G}}}}
\begin{itemize}
    \item Viene visualizzato un messaggio di errore relativo ai Requisiti di sicurezza della password
\end{itemize}

\textbf{Scenario principale\textsubscript{\scalebox{0.6}{\textbf{G}}}}
\begin{enumerate}
    \item L'utente inserisce una password che non soddisfa i criteri minimi di sicurezza (es. lunghezza minima, caratteri speciali)
    \item Il sistema rileva che la password è troppo debole
\end{enumerate}

\vspace{0.5cm}

\subUseCase{Email già registrata}

\textbf{Attori\textsubscript{\scalebox{0.6}{\textbf{G}}}}
\begin{itemize}
    \item Utente
\end{itemize}

\textbf{Pre-condizioni\textsubscript{\scalebox{0.6}{\textbf{G}}}}
\begin{itemize}
    \item L'indirizzo email inserito è già presente all'interno del sistema
\end{itemize}

\textbf{Post-condizioni\textsubscript{\scalebox{0.6}{\textbf{G}}}}
\begin{itemize}
    \item Viene visualizzato un messaggio di errore che notifica l'esistenza dell'account
\end{itemize}

\textbf{Scenario principale\textsubscript{\scalebox{0.6}{\textbf{G}}}}
\begin{enumerate}
    \item L'utente inserisce un'email formalmente valida ma già associata ad un altro utente registrato
    \item Il sistema verifica\textsubscript{\scalebox{0.6}{\textbf{G}}} la presenza dell'email nel database e blocca l'operazione
\end{enumerate}

\vspace{0.5cm}

\newpage

\subUseCase{Username già registrato}

\textbf{Attori\textsubscript{\scalebox{0.6}{\textbf{G}}}}
\begin{itemize}
    \item Utente
\end{itemize}

\textbf{Pre-condizioni\textsubscript{\scalebox{0.6}{\textbf{G}}}}
\begin{itemize}
    \item Lo username inserito è già presente all'interno del sistema
\end{itemize}

\textbf{Post-condizioni\textsubscript{\scalebox{0.6}{\textbf{G}}}}
\begin{itemize}
    \item Viene visualizzato un messaggio di errore che indica che lo username non è disponibile
\end{itemize}

\textbf{Scenario principale\textsubscript{\scalebox{0.6}{\textbf{G}}}}
\begin{enumerate}
    \item L'utente inserisce uno username già utilizzato da un altro utente
    \item Il sistema verifica\textsubscript{\scalebox{0.6}{\textbf{G}}} l'univocità dello username e ne segnala l'indisponibilità
\end{enumerate}

\vspace{0.5cm}

\subUseCase{Matricola già registrata}

\textbf{Attori\textsubscript{\scalebox{0.6}{\textbf{G}}}}
\begin{itemize}
    \item Utente
\end{itemize}

\textbf{Pre-condizioni\textsubscript{\scalebox{0.6}{\textbf{G}}}}
\begin{itemize}
    \item La matricola inserita è già associata ad un account esistente
\end{itemize}

\textbf{Post-condizioni\textsubscript{\scalebox{0.6}{\textbf{G}}}}
\begin{itemize}
    \item Viene visualizzato un messaggio di errore che impedisce la registrazione multipla con la stessa matricola
\end{itemize}

\textbf{Scenario principale\textsubscript{\scalebox{0.6}{\textbf{G}}}}
\begin{enumerate}
    \item L'utente inserisce un numero di matricola già presente nel sistema
    \item Il sistema rileva la duplicazione e impedisce il proseguimento
\end{enumerate}

\vspace{0.5cm}
\newpage

\subUseCase{Matricola non valida}

\textbf{Attori\textsubscript{\scalebox{0.6}{\textbf{G}}}}
\begin{itemize}
    \item Utente
\end{itemize}

\textbf{Pre-condizioni\textsubscript{\scalebox{0.6}{\textbf{G}}}}
\begin{itemize}
    \item L'utente ha inserito dei dati nel campo matricola
\end{itemize}

\textbf{Post-condizioni\textsubscript{\scalebox{0.6}{\textbf{G}}}}
\begin{itemize}
    \item Viene visualizzato un messaggio di errore sul formato della matricola
\end{itemize}

\textbf{Scenario principale\textsubscript{\scalebox{0.6}{\textbf{G}}}}
\begin{enumerate}
    \item L'utente inserisce una matricola che non rispetta il formato atteso (es. contiene lettere dove non previste o ha una lunghezza errata)
    \item Il sistema invalida il dato inserito
\end{enumerate}

\vspace{0.5cm}


\useCase{Login / Autenticazione utente}

\begin{figure}[H]
    \centering
    \includegraphics[width=1\textwidth]{Diagrammi casi d'uso/UC0Ba.jpg}
    \caption{Diagramma del caso d'uso UC-0B – Login / Autenticazione utente (include)}
\end{figure}

\begin{figure}[H]
    \centering
    \includegraphics[width=0.8\textwidth]{Diagrammi casi d'uso/UC0Bb.jpg}
    \caption{Diagramma del caso d'uso UC-0B – Login / Autenticazione utente (extend)}
\end{figure}

\textbf{Attori\textsubscript{\scalebox{0.6}{\textbf{G}}}}
\begin{itemize}
    \item Utente non autenticato 
\end{itemize}

\textbf{Pre-condizioni\textsubscript{\scalebox{0.6}{\textbf{G}}}}
\begin{itemize}
    \item L'utente è già registrato nel sistema.
    \item Non esiste una sessione attiva associata all'utente sul dispositivo corrente.
\end{itemize}

\textbf{Post-condizioni\textsubscript{\scalebox{0.6}{\textbf{G}}}}
\begin{itemize}
    \item L'utente risulta autenticato nel sistema.
    \item È attiva una sessione associata all'utente, che consente l'accesso alle funzionalità\textsubscript{\scalebox{0.6}{\textbf{G}}} riservate (es. generazione contenuti, upload documenti).
\end{itemize}

\textbf{Scenario principale\textsubscript{\scalebox{0.6}{\textbf{G}}}}
\begin{enumerate}
    \item L'utente accede alla schermata di login.
    \item L'utente inserisce le proprie credenziali (e-mail e password).
    \item Il sistema verifica\textsubscript{\scalebox{0.6}{\textbf{G}}} la correttezza delle credenziali.
    \item In caso di credenziali valide, il sistema crea una nuova sessione autenticata per l'utente.
    \item Il sistema reindirizza l'utente alla Dashboard\textsubscript{\scalebox{0.6}{\textbf{G}}} principale dell'applicazione standalone.
\end{enumerate}

\textbf{Scenario secondario\textsubscript{\scalebox{0.6}{\textbf{G}}}}
\begin{itemize}
    \item L'utente inserisce un'e-mail non registrata nel sistema.
    \item L'utente inserisce una password errata.
    \item L'utente inserisce credenziali non valide.
\end{itemize}

\textbf{Relazioni con altri Casi d'uso\textsubscript{\scalebox{0.6}{\textbf{G}}}}
\begin{itemize}
    \item \textit{include}:
    \begin{itemize}
        \item UC-0A.1 - Inserimento email
        \item UC-0A.2 - Inserimento password
    \end{itemize}
    \item \textit{extend}: 
    \begin{itemize}
        \item UC-0B.1 – email non registrata.
        \item UC-0B.2 – password errata.
    \end{itemize}
\end{itemize}

\vspace{0.5cm}

\subUseCase{Email non registrata}

\textbf{Attori\textsubscript{\scalebox{0.6}{\textbf{G}}}}
\begin{itemize}
    \item Utente
\end{itemize}

\textbf{Pre-condizioni\textsubscript{\scalebox{0.6}{\textbf{G}}}}
\begin{itemize}
    \item L'email inserita nel Modulo\textsubscript{\scalebox{0.6}{\textbf{G}}} di accesso non è presente nel database del sistema
\end{itemize}

\textbf{Post-condizioni\textsubscript{\scalebox{0.6}{\textbf{G}}}}
\begin{itemize}
    \item Viene visualizzato un messaggio di errore e l'accesso viene negato
\end{itemize}

\textbf{Scenario principale\textsubscript{\scalebox{0.6}{\textbf{G}}}}
\begin{enumerate}
    \item L'utente tenta di effettuare il login inserendo un indirizzo email non associato ad alcun account
    \item Il sistema verifica\textsubscript{\scalebox{0.6}{\textbf{G}}} l'esistenza dell'email e non trova corrispondenze
\end{enumerate}

\vspace{0.5cm}

\subUseCase{Password errata}

\textbf{Attori\textsubscript{\scalebox{0.6}{\textbf{G}}}}
\begin{itemize}
    \item Utente
\end{itemize}

\textbf{Pre-condizioni\textsubscript{\scalebox{0.6}{\textbf{G}}}}
\begin{itemize}
    \item L'email inserita è corretta, ma la password non corrisponde a quella salvata nel sistema
\end{itemize}

\textbf{Post-condizioni\textsubscript{\scalebox{0.6}{\textbf{G}}}}
\begin{itemize}
    \item Viene visualizzato un messaggio di errore relativo alle credenziali non valide
\end{itemize}

\textbf{Scenario principale\textsubscript{\scalebox{0.6}{\textbf{G}}}}
\begin{enumerate}
    \item L'utente inserisce la propria email (corretta) e una password errata
    \item Il sistema verifica\textsubscript{\scalebox{0.6}{\textbf{G}}} la corrispondenza delle credenziali e rileva l'errore
\end{enumerate}

\vspace{0.5cm}



\useCase{Visualizzazione dati profilo utente}

\begin{figure}[H]
    \centering
    \includegraphics[width=1\textwidth]{Diagrammi casi d'uso/UC0C.jpg}
    \caption{Diagramma del caso d'uso UC-0C – Visualizzazione dati profilo utente}
\end{figure}

\textbf{Attori\textsubscript{\scalebox{0.6}{\textbf{G}}}}
\begin{itemize}
    \item Utente 
\end{itemize}


\textbf{Pre-condizioni\textsubscript{\scalebox{0.6}{\textbf{G}}}}
\begin{itemize}
    \item L'utente ha effettuato il login ed è autenticato.
    \item Esiste un profilo associato all'utente nel sistema
    \item L'utente è entrato nel Modulo\textsubscript{\scalebox{0.6}{\textbf{G}}} di gestione profilo utente dalla Dashboard\textsubscript{\scalebox{0.6}{\textbf{G}}} principale

\end{itemize}
\newpage

\textbf{Post-condizioni\textsubscript{\scalebox{0.6}{\textbf{G}}}}
\begin{itemize}
    \item Sono ottenibili le informazioni dal profilo utente e si possono modificare
\end{itemize}

\textbf{Scenario principale\textsubscript{\scalebox{0.6}{\textbf{G}}}}
\begin{enumerate}
    \item L'utente accede alla sezione “Profilo” dalla Dashboard\textsubscript{\scalebox{0.6}{\textbf{G}}} dell'applicazione.
    \item Il sistema mostra i dati correnti del profilo
\end{enumerate}


\textbf{Relazioni con altri Casi d'uso\textsubscript{\scalebox{0.6}{\textbf{G}}}}
\begin{itemize}
    \item \textit{include}: 
    \begin{itemize}
        \item UC-0C.1 - Visualizzazione email
        \item UC-0C.2 - Visualizzazione password
        \item UC-0C.3 - Visualizzazione username
        \item UC-0C.4 - Visualizzazione nome 
        \item UC-0C.5 - Visualizzazione cognome
        \item UC-0C.6 - Visualizzazione matricola
    \end{itemize}
\end{itemize}

\vspace{0.5cm}

\subUseCase{Visualizzazione email}

\textbf{Attori\textsubscript{\scalebox{0.6}{\textbf{G}}}}
\begin{itemize}
    \item Utente
\end{itemize}

\textbf{Pre-condizioni\textsubscript{\scalebox{0.6}{\textbf{G}}}}
\begin{itemize}
    \item L'utente ha effettuato l'accesso e si trova nella pagina del profilo
\end{itemize}

\textbf{Post-condizioni\textsubscript{\scalebox{0.6}{\textbf{G}}}}
\begin{itemize}
    \item Casella di testo precompilata con l'email attuale è mostrata all'utente
\end{itemize}

\textbf{Scenario principale\textsubscript{\scalebox{0.6}{\textbf{G}}}}
\begin{enumerate}
    \item Il sistema recupera l'email associata all'account e la mostra nell'apposita casella
\end{enumerate}

\vspace{0.5cm}

\subUseCase{Visualizzazione password}

\textbf{Attori\textsubscript{\scalebox{0.6}{\textbf{G}}}}
\begin{itemize}
    \item Utente
\end{itemize}

\textbf{Pre-condizioni\textsubscript{\scalebox{0.6}{\textbf{G}}}}
\begin{itemize}
    \item L'utente ha effettuato l'accesso e si trova nella pagina del profilo
\end{itemize}

\newpage


\textbf{Post-condizioni\textsubscript{\scalebox{0.6}{\textbf{G}}}}
\begin{itemize}
    \item Casella di testo contenente la password attuale (tipicamente oscurata) è mostrata all'utente
\end{itemize}

\textbf{Scenario principale\textsubscript{\scalebox{0.6}{\textbf{G}}}}
\begin{enumerate}
    \item Il sistema predispone il campo password permettendo all'utente di visualizzarne lo stato o modificarla
\end{enumerate}

\vspace{0.5cm}

\subUseCase{Visualizzazione username}

\textbf{Attori\textsubscript{\scalebox{0.6}{\textbf{G}}}}
\begin{itemize}
    \item Utente
\end{itemize}

\textbf{Pre-condizioni\textsubscript{\scalebox{0.6}{\textbf{G}}}}
\begin{itemize}
    \item L'utente ha effettuato l'accesso e si trova nella pagina del profilo
\end{itemize}

\textbf{Post-condizioni\textsubscript{\scalebox{0.6}{\textbf{G}}}}
\begin{itemize}
    \item Casella di testo precompilata con lo username è mostrata all'utente
\end{itemize}

\textbf{Scenario principale\textsubscript{\scalebox{0.6}{\textbf{G}}}}
\begin{enumerate}
    \item Il sistema recupera lo username associato all'account e lo mostra nell'apposita casella
\end{enumerate}

\vspace{0.5cm}

\subUseCase{Visualizzazione nome}

\textbf{Attori\textsubscript{\scalebox{0.6}{\textbf{G}}}}
\begin{itemize}
    \item Utente
\end{itemize}

\textbf{Pre-condizioni\textsubscript{\scalebox{0.6}{\textbf{G}}}}
\begin{itemize}
    \item L'utente ha effettuato l'accesso e si trova nella pagina del profilo
\end{itemize}

\textbf{Post-condizioni\textsubscript{\scalebox{0.6}{\textbf{G}}}}
\begin{itemize}
    \item Casella di testo precompilata con il nome attuale è mostrata all'utente
\end{itemize}

\textbf{Scenario principale\textsubscript{\scalebox{0.6}{\textbf{G}}}}
\begin{enumerate}
    \item Il sistema recupera il nome dell'utente e lo mostra nell'apposita casella
\end{enumerate}

\vspace{0.5cm}

\subUseCase{Visualizzazione cognome}

\textbf{Attori\textsubscript{\scalebox{0.6}{\textbf{G}}}}
\begin{itemize}
    \item Utente
\end{itemize}

\textbf{Pre-condizioni\textsubscript{\scalebox{0.6}{\textbf{G}}}}
\begin{itemize}
    \item L'utente ha effettuato l'accesso e si trova nella pagina del profilo
\end{itemize}

\textbf{Post-condizioni\textsubscript{\scalebox{0.6}{\textbf{G}}}}
\begin{itemize}
    \item Casella di testo precompilata con il cognome attuale è mostrata all'utente
\end{itemize}

\textbf{Scenario principale\textsubscript{\scalebox{0.6}{\textbf{G}}}}
\begin{enumerate}
    \item Il sistema recupera il cognome dell'utente e lo mostra nell'apposita casella
\end{enumerate}

\vspace{0.5cm}

\subUseCase{Visualizzazione matricola}

\textbf{Attori\textsubscript{\scalebox{0.6}{\textbf{G}}}}
\begin{itemize}
    \item Utente
\end{itemize}

\textbf{Pre-condizioni\textsubscript{\scalebox{0.6}{\textbf{G}}}}
\begin{itemize}
    \item L'utente ha effettuato l'accesso e si trova nella pagina del profilo
\end{itemize}

\textbf{Post-condizioni\textsubscript{\scalebox{0.6}{\textbf{G}}}}
\begin{itemize}
    \item Casella di testo precompilata con la matricola attuale è mostrata all'utente
\end{itemize}

\textbf{Scenario principale\textsubscript{\scalebox{0.6}{\textbf{G}}}}
\begin{enumerate}
    \item Il sistema recupera la matricola associata all'account e la mostra nell'apposita casella
\end{enumerate}

\vspace{0.5cm}

\useCase{Modifica informazioni profilo utente}

\begin{figure}[H]
    \centering
    \includegraphics[width=1\textwidth]{Diagrammi casi d'uso/UC0D.jpg}
    \caption{Diagramma del caso d'uso UC-0D – Modifica informazioni profilo utente}
\end{figure}

\newpage

\textbf{Attori\textsubscript{\scalebox{0.6}{\textbf{G}}}}
\begin{itemize}
    \item Utente
\end{itemize}

\textbf{Pre-condizioni\textsubscript{\scalebox{0.6}{\textbf{G}}}}
\begin{itemize}
    \item L'utente si trova nella pagina di modifica profilo
    \item L'utente conferma le modifiche
\end{itemize}

\textbf{Post-condizioni\textsubscript{\scalebox{0.6}{\textbf{G}}}}
\begin{itemize}
    \item L'utente ha cambiato i dati del proprio profilo
    \item I dati vengono salvati nel sistema
\end{itemize}

\textbf{Scenario principale\textsubscript{\scalebox{0.6}{\textbf{G}}}}
\begin{enumerate}
    \item L'utente cambia i dati del proprio profilo
\end{enumerate}

\textbf{Relazioni con altri Casi d'uso\textsubscript{\scalebox{0.6}{\textbf{G}}}}
\begin{itemize}
    \item \textit{extend}: 
    \begin{itemize}
        \item UC-0D.1 - Uscita senza salvare modifiche profilo utente
    \end{itemize}
\end{itemize}

\vspace{0.5cm}

\subUseCase{Uscita senza salvare modifiche profilo utente}

\textbf{Attori\textsubscript{\scalebox{0.6}{\textbf{G}}}}
\begin{itemize}
    \item Utente
\end{itemize}

\textbf{Pre-condizioni\textsubscript{\scalebox{0.6}{\textbf{G}}}}
\begin{itemize}
    \item L'utente si trova nella pagina di modifica profilo
\end{itemize}

\textbf{Post-condizioni\textsubscript{\scalebox{0.6}{\textbf{G}}}}
\begin{itemize}
    \item L'utente viene reindirizzato alla pagina precedente o alla home senza che alcuna modifica venga applicata al database
\end{itemize}

\textbf{Scenario principale\textsubscript{\scalebox{0.6}{\textbf{G}}}}
\begin{enumerate}
    \item L'utente decide di annullare l'operazione di modifica
    \item Il sistema scarta le modifiche pendenti nei campi di testo e chiude la schermata
\end{enumerate}

\vspace{0.5cm}



\useCase{Visualizzazione lista utenti registrati}

\begin{figure}[H]
    \centering
    \includegraphics[width=0.7\textwidth]{Diagrammi casi d'uso/UC0E.jpg}
    \caption{Diagramma del caso d'uso UC-0E – Visualizzazione lista utenti registrati}
\end{figure}

\textbf{Attori\textsubscript{\scalebox{0.6}{\textbf{G}}}}
\begin{itemize}
    \item Amministratore\textsubscript{\scalebox{0.6}{\textbf{G}}}
\end{itemize}

\textbf{Pre-condizioni\textsubscript{\scalebox{0.6}{\textbf{G}}}}
\begin{itemize}
    \item L'Amministratore\textsubscript{\scalebox{0.6}{\textbf{G}}} ha effettuato l'accesso e si trova nella sezione di gestione utenti
\end{itemize}

\textbf{Post-condizioni\textsubscript{\scalebox{0.6}{\textbf{G}}}}
\begin{itemize}
    \item Viene visualizzata la lista degli utenti registrati nel sistema
\end{itemize}

\textbf{Scenario principale\textsubscript{\scalebox{0.6}{\textbf{G}}}}
\begin{enumerate}
    \item Il sistema recupera dal database la lista degli utenti e ne mostra i dati in formato tabellare o a lista
\end{enumerate}


\textbf{Relazioni con altri Casi d'uso\textsubscript{\scalebox{0.6}{\textbf{G}}}}
\begin{itemize}
    \item \textit{include}: 
    \begin{itemize}
        \item UC-0E.1 - Visualizzazione elemento lista utenti registrati.
    \end{itemize}
    \item \textit{extend}: 
    \begin{itemize}
        \item nessuna
    \end{itemize}
    
\end{itemize}

\vspace{0.5cm}

\newpage

\subUseCase{Visualizzazione elemento lista utenti registrati}

\begin{figure}[H]
    \centering
    \includegraphics[width=0.7\textwidth]{Diagrammi casi d'uso/UC0E-1.jpg}
    \caption{Diagramma del caso d'uso UC-0E.1 – Visualizzazione elemento lista utenti registrati}
\end{figure}

\textbf{Attori\textsubscript{\scalebox{0.6}{\textbf{G}}}}
\begin{itemize}
    \item Amministratore\textsubscript{\scalebox{0.6}{\textbf{G}}}
\end{itemize}

\textbf{Pre-condizioni\textsubscript{\scalebox{0.6}{\textbf{G}}}}
\begin{itemize}
    \item L'Amministratore\textsubscript{\scalebox{0.6}{\textbf{G}}} ha effettuato l'accesso e si trova nella sezione di gestione utenti
    \item Esiste almeno un utente registrato nel sistema
\end{itemize}

\textbf{Post-condizioni\textsubscript{\scalebox{0.6}{\textbf{G}}}}
\begin{itemize}
    \item Viene visualizzato un elemento dalla lista degli utenti registrati nel sistema
\end{itemize}

\textbf{Scenario principale\textsubscript{\scalebox{0.6}{\textbf{G}}}}
\begin{enumerate}
    \item L'applicazione mostra un singolo utente con i relativi dettagli (nome, cognome, ruolo, ecc.) all'interno della lista
\end{enumerate}

\textbf{Relazioni con altri Casi d'uso\textsubscript{\scalebox{0.6}{\textbf{G}}}}
\begin{itemize}
    \item \textit{include}: 
    \begin{itemize}
        \item UC-0E.2 - Visualizzazione ruolo utente registrato.
        \item UC-0E.3 - Visualizzazione nome utente registrato.
        \item UC-0E.4 - Visualizzazione cognome utente registrato.
    \end{itemize}
    \item \textit{extend}: 
    \begin{itemize}
        \item Nessuna
    \end{itemize}
    
\end{itemize}

\vspace{0.5cm}

\newpage

\subUseCase{Visualizzazione ruolo utente registrato}

\textbf{Attori\textsubscript{\scalebox{0.6}{\textbf{G}}}}
\begin{itemize}
    \item Amministratore\textsubscript{\scalebox{0.6}{\textbf{G}}}
\end{itemize}

\textbf{Pre-condizioni\textsubscript{\scalebox{0.6}{\textbf{G}}}}
\begin{itemize}
    \item L'Amministratore\textsubscript{\scalebox{0.6}{\textbf{G}}} ha effettuato l'accesso e si trova nella sezione di gestione utenti
\end{itemize}

\textbf{Post-condizioni\textsubscript{\scalebox{0.6}{\textbf{G}}}}
\begin{itemize}
    \item Viene visualizzato il ruolo attuale (es. Utente, Admin, Moderatore) associato a ciascun utente
\end{itemize}

\textbf{Scenario principale\textsubscript{\scalebox{0.6}{\textbf{G}}}}
\begin{enumerate}
    \item Il sistema identifica i permessi di ogni utente e mostra l'etichetta del ruolo corrispondente
\end{enumerate}

\vspace{0.5cm}

\subUseCase{Visualizzazione nome utente registrato}

\textbf{Attori\textsubscript{\scalebox{0.6}{\textbf{G}}}}
\begin{itemize}
    \item Amministratore\textsubscript{\scalebox{0.6}{\textbf{G}}}
\end{itemize}

\textbf{Pre-condizioni\textsubscript{\scalebox{0.6}{\textbf{G}}}}
\begin{itemize}
    \item L'Amministratore\textsubscript{\scalebox{0.6}{\textbf{G}}} ha effettuato l'accesso e si trova nella sezione di gestione utenti
\end{itemize}

\textbf{Post-condizioni\textsubscript{\scalebox{0.6}{\textbf{G}}}}
\begin{itemize}
    \item Viene visualizzatil nome di un utente 
\end{itemize}

\textbf{Scenario principale\textsubscript{\scalebox{0.6}{\textbf{G}}}}
\begin{enumerate}
    \item Il sistema recupera dal database la lista degli utenti e ne mostra i nomi in formato tabellare o a lista
\end{enumerate}

\vspace{0.5cm}

\subUseCase{Visualizzazione cognome utente registrato}

\textbf{Attori\textsubscript{\scalebox{0.6}{\textbf{G}}}}
\begin{itemize}
    \item Amministratore\textsubscript{\scalebox{0.6}{\textbf{G}}}
\end{itemize}

\textbf{Pre-condizioni\textsubscript{\scalebox{0.6}{\textbf{G}}}}
\begin{itemize}
    \item L'Amministratore\textsubscript{\scalebox{0.6}{\textbf{G}}} ha effettuato l'accesso e si trova nella sezione di gestione utenti
\end{itemize}

\textbf{Post-condizioni\textsubscript{\scalebox{0.6}{\textbf{G}}}}
\begin{itemize}
    \item Viene visualizzato il cognome di un utente
\end{itemize}

\textbf{Scenario principale\textsubscript{\scalebox{0.6}{\textbf{G}}}}
\begin{enumerate}
    \item Il sistema recupera dal database la lista degli utenti e ne mostra i cognomi in corrispondenza dei rispettivi nomi
\end{enumerate}

\vspace{0.5cm}

\useCase{Modifica ruolo utente registrato}

\begin{figure}[H]
    \centering
    \includegraphics[width=0.7\textwidth]{Diagrammi casi d'uso/UC0F.jpg}
    \caption{Diagramma del caso d'uso UC-0F – Modifica ruolo utente registrato}
\end{figure}

\textbf{Attori\textsubscript{\scalebox{0.6}{\textbf{G}}}}
\begin{itemize}
    \item Amministratore\textsubscript{\scalebox{0.6}{\textbf{G}}}
\end{itemize}

\textbf{Pre-condizioni\textsubscript{\scalebox{0.6}{\textbf{G}}}}
\begin{itemize}
    \item L'Amministratore\textsubscript{\scalebox{0.6}{\textbf{G}}} ha selezionato un utente specifico dalla lista
\end{itemize}

\textbf{Post-condizioni\textsubscript{\scalebox{0.6}{\textbf{G}}}}
\begin{itemize}
    \item Il ruolo dell'utente è stato modificato
\end{itemize}

\textbf{Scenario principale\textsubscript{\scalebox{0.6}{\textbf{G}}}}
\begin{enumerate}
    \item L'Amministratore\textsubscript{\scalebox{0.6}{\textbf{G}}} apporta delle modifiche al ruolo di un utente
    \item Le modifiche vengono salvate all'interno del sistema
\end{enumerate}

\textbf{Relazioni con altri Casi d'uso\textsubscript{\scalebox{0.6}{\textbf{G}}}}
\begin{itemize}
    \item \textit{extend}: 
    \begin{itemize}
        \item UC-0F.1 - Annulla modifica ruolo utente registrato
    \end{itemize}
    
\end{itemize}

\vspace{0.5cm}

\subUseCase{Annulla modifica ruolo utente registrato}

\textbf{Attori\textsubscript{\scalebox{0.6}{\textbf{G}}}}
\begin{itemize}
    \item Amministratore\textsubscript{\scalebox{0.6}{\textbf{G}}}
\end{itemize}

\textbf{Pre-condizioni\textsubscript{\scalebox{0.6}{\textbf{G}}}}
\begin{itemize}
    \item L'Amministratore\textsubscript{\scalebox{0.6}{\textbf{G}}} sta modificando il ruolo di un utente
\end{itemize}

\textbf{Post-condizioni\textsubscript{\scalebox{0.6}{\textbf{G}}}}
\begin{itemize}
    \item Il ruolo dell'utente rimane invariato e l'interfaccia\textsubscript{\scalebox{0.6}{\textbf{G}}} torna allo stato precedente
\end{itemize}

\textbf{Scenario principale\textsubscript{\scalebox{0.6}{\textbf{G}}}}
\begin{enumerate}
    \item L'Amministratore\textsubscript{\scalebox{0.6}{\textbf{G}}} decide di non applicare le modifiche al ruolo
    \item Il sistema ripristina il valore originale visualizzato
\end{enumerate}

\vspace{0.5cm}



\useCase{Logout}

\begin{figure}[H]
    \centering
    \includegraphics[width=0.5\textwidth]{Diagrammi casi d'uso/UC0G.jpg}
    \caption{Diagramma del caso d'uso UC-0G – Logout}
\end{figure}

\textbf{Attori\textsubscript{\scalebox{0.6}{\textbf{G}}}}
\begin{itemize}
    \item Utente
\end{itemize}

\textbf{Pre-condizioni\textsubscript{\scalebox{0.6}{\textbf{G}}}}
\begin{itemize}
    \item L'utente ha una sessione attiva nel sistema.
\end{itemize}

\textbf{Post-condizioni\textsubscript{\scalebox{0.6}{\textbf{G}}}}
\begin{itemize}
    \item Non esiste più una sessione attiva associata all'utente sul dispositivo corrente.
\end{itemize}

\textbf{Scenario principale\textsubscript{\scalebox{0.6}{\textbf{G}}}}
\begin{enumerate}
    \item L'utente seleziona l'opzione di logout (ad esempio dal menu della Dashboard\textsubscript{\scalebox{0.6}{\textbf{G}}}).
    \item Il sistema invalida la sessione corrente associata all'utente (es. rimozione token di sessione).
    \item Il sistema reindirizza l'utente alla schermata di login o alla schermata iniziale pubblica.
\end{enumerate}

\vspace{0.5cm}




\subsection{Sezione 1 – Modulo\textsubscript{\scalebox{0.6}{\textbf{G}}} AI Assistant\textsubscript{\scalebox{0.6}{\textbf{G}}} Generativo}

\renewcommand{\UCPrefix}{1}


\useCase{Generazione contenuti AI}


\begin{figure}[H]
    \centering
    \includegraphics[width=1\textwidth]{Diagrammi casi d'uso/UC1A.jpg}
    \caption{Diagramma del caso d'uso UC-1A – Generazione contenuti AI}
\end{figure}

\textbf{Attori\textsubscript{\scalebox{0.6}{\textbf{G}}}}
\begin{itemize}
    \item Redattore\textsubscript{\scalebox{0.6}{\textbf{G}}}
\end{itemize}

\textbf{Attori\textsubscript{\scalebox{0.6}{\textbf{G}}} secondari}
\begin{itemize}
    \item AI Post Generator
\end{itemize}

\textbf{Pre-condizioni\textsubscript{\scalebox{0.6}{\textbf{G}}}}
\begin{itemize}
    \item L'utente è entrato nel Modulo\textsubscript{\scalebox{0.6}{\textbf{G}}} AI Assistant\textsubscript{\scalebox{0.6}{\textbf{G}}} Generativo dalla Dashboard\textsubscript{\scalebox{0.6}{\textbf{G}}} principale.
    \item L'utente dispone dei permessi necessari per utilizzare il Modulo\textsubscript{\scalebox{0.6}{\textbf{G}}} AI Assistant\textsubscript{\scalebox{0.6}{\textbf{G}}}.
\end{itemize}

\textbf{Post-condizioni\textsubscript{\scalebox{0.6}{\textbf{G}}}}
\begin{itemize}
    \item L'utente ha generato un contenuto testuale basato sul prompt\textsubscript{\scalebox{0.6}{\textbf{G}}} e sui parametri selezionati
    \item Il contenuto generato è visualizzato nell'interfaccia\textsubscript{\scalebox{0.6}{\textbf{G}}} utente
\end{itemize}

\textbf{Scenario principale\textsubscript{\scalebox{0.6}{\textbf{G}}}}
\begin{enumerate}
    \item L'utente accede alla sezione "AI Assistant\textsubscript{\scalebox{0.6}{\textbf{G}}} Generativo".
    \item Il sistema mostra il campo per l'inserimento del prompt\textsubscript{\scalebox{0.6}{\textbf{G}}} e le azioni che possono essere eseguite dall'utente.
    \item L'utente inserisce il prompt\textsubscript{\scalebox{0.6}{\textbf{G}}} descrittivo (UC-1A.1).
    \item L'utente seleziona il tono desiderato per il contenuto (UC-1A.2).
    \item L'utente seleziona lo stile del contenuto (UC-1A.3).
    \item L'utente avvia il processo di generazione del contenuto.
    \item Il sistema invia il prompt\textsubscript{\scalebox{0.6}{\textbf{G}}} e i parametri selezionati all'AI Post Generator.
    \item L'AI Post Generator elabora la richiesta e genera il contenuto testuale.
    \item Il sistema riceve il contenuto generato dall'AI Post Generator.
\end{enumerate}


\textbf{Relazioni con altri Casi d'uso\textsubscript{\scalebox{0.6}{\textbf{G}}}}
\begin{itemize}
    \item \textit{include}: 
    \begin{itemize}
        \item UC-1A.1 - Inserimento prompt\textsubscript{\scalebox{0.6}{\textbf{G}}}
        \item UC-1A.2 - Selezione tono
        \item UC-1A.3 - Selezione stile
    \end{itemize}
\end{itemize}

\vspace{0.5cm}

\subUseCase{Inserimento prompt\textsubscript{\scalebox{0.6}{\textbf{G}}}}

\textbf{Attori\textsubscript{\scalebox{0.6}{\textbf{G}}}}
\begin{itemize}
    \item Redattore\textsubscript{\scalebox{0.6}{\textbf{G}}}
\end{itemize}

\textbf{Pre-condizioni\textsubscript{\scalebox{0.6}{\textbf{G}}}}
\begin{itemize}
    \item L'utente si trova nel modulo di generazione contenuti AI (UC-1A)
\end{itemize}

\textbf{Post-condizioni\textsubscript{\scalebox{0.6}{\textbf{G}}}}
\begin{itemize}
    \item Il prompt\textsubscript{\scalebox{0.6}{\textbf{G}}} è inserito
\end{itemize}

\textbf{Scenario principale\textsubscript{\scalebox{0.6}{\textbf{G}}}}
\begin{enumerate}
    \item L'utente inserisce o incolla il testo descrittivo per la generazione del contenuto nell'apposita area di testo
\end{enumerate}

\vspace{0.5cm}

\subUseCase{Selezione tono}

\textbf{Attori\textsubscript{\scalebox{0.6}{\textbf{G}}}}
\begin{itemize}
    \item Redattore\textsubscript{\scalebox{0.6}{\textbf{G}}}
\end{itemize}

\textbf{Pre-condizioni\textsubscript{\scalebox{0.6}{\textbf{G}}}}
\begin{itemize}
    \item L'utente si trova nel modulo di generazione contenuti AI (UC-1A)
\end{itemize}

\textbf{Post-condizioni\textsubscript{\scalebox{0.6}{\textbf{G}}}}
\begin{itemize}
    \item Il parametro "Tono" risulta impostato sulla scelta effettuata (es. Professionale, Informale, Spiritoso)
\end{itemize}

\textbf{Scenario principale\textsubscript{\scalebox{0.6}{\textbf{G}}}}
\begin{enumerate}
    \item L'utente interagisce con il selettore del tono (es. menu a tendina)
    \item L'utente sceglie l'opzione desiderata tra quelle disponibili
\end{enumerate}

\vspace{0.5cm}

\subUseCase{Selezione stile}

\textbf{Attori\textsubscript{\scalebox{0.6}{\textbf{G}}}}
\begin{itemize}
    \item Redattore\textsubscript{\scalebox{0.6}{\textbf{G}}}
\end{itemize}

\textbf{Pre-condizioni\textsubscript{\scalebox{0.6}{\textbf{G}}}}
\begin{itemize}
    \item L'utente si trova nel modulo di generazione contenuti AI (UC-1A)
\end{itemize}

\textbf{Post-condizioni\textsubscript{\scalebox{0.6}{\textbf{G}}}}
\begin{itemize}
    \item Il parametro "Stile" risulta impostato sulla scelta effettuata (es. Articolo di blog, Post social, Email)
\end{itemize}

\textbf{Scenario principale\textsubscript{\scalebox{0.6}{\textbf{G}}}}
\begin{enumerate}
    \item L'utente interagisce con il selettore dello stile
    \item L'utente sceglie la tipologia di formato desiderata per il testo in output
\end{enumerate}

\vspace{0.5cm}


\useCase{Visualizzazione lista elementi storico}

\begin{figure}[H]
    \centering
    \includegraphics[width=1\textwidth]{Diagrammi casi d'uso/UC1B.jpg}
    \caption{Diagramma del caso d'uso UC-1B – Visualizzazione lista elementi storico}
\end{figure}

\newpage

\textbf{Attori\textsubscript{\scalebox{0.6}{\textbf{G}}}}
\begin{itemize}
    \item Redattore\textsubscript{\scalebox{0.6}{\textbf{G}}}
\end{itemize}

\textbf{Pre-condizioni\textsubscript{\scalebox{0.6}{\textbf{G}}}}
\begin{itemize}
    \item L'utente si trova all'interno della sezione dedicata allo storico
\end{itemize}

\textbf{Post-condizioni\textsubscript{\scalebox{0.6}{\textbf{G}}}}
\begin{itemize}
    \item L'utente visualizza la lista di elementi salvati nello storico nel suo insieme
\end{itemize}

\textbf{Scenario principale\textsubscript{\scalebox{0.6}{\textbf{G}}}}
\begin{enumerate}
    \item L'utente si trova nel Modulo\textsubscript{\scalebox{0.6}{\textbf{G}}} storico dell'AI Assistant\textsubscript{\scalebox{0.6}{\textbf{G}}} Generativo
    \item L'utente vede l'elenco di tutti gli elementi salvati con i relativi dati principali (prompt\textsubscript{\scalebox{0.6}{\textbf{G}}} parziale, tono, stile, data)
\end{enumerate}


\textbf{Relazioni con altri Casi d'uso\textsubscript{\scalebox{0.6}{\textbf{G}}}}
\begin{itemize}
    \item \textit{include}: 
    \begin{itemize}
        \item UC-1B.2 – Visualizzazione informazioni elemento
    \end{itemize}
    \item \textit{extend}: 
    \begin{itemize}
        \item UC-1B.1 – Nessun prompt\textsubscript{\scalebox{0.6}{\textbf{G}}} salvato
    \end{itemize}
\end{itemize}

\vspace{0.5cm}

\subUseCase{Nessun prompt\textsubscript{\scalebox{0.6}{\textbf{G}}} salvato}

\textbf{Attori\textsubscript{\scalebox{0.6}{\textbf{G}}}}
\begin{itemize}
    \item Redattore\textsubscript{\scalebox{0.6}{\textbf{G}}}
\end{itemize}

\textbf{Pre-condizioni\textsubscript{\scalebox{0.6}{\textbf{G}}}}
\begin{itemize}
    \item L'utente si trova all'interno della sezione dedicata allo storico
    \item Lo storico non ha neanche un elemento
\end{itemize}

\textbf{Post-condizioni\textsubscript{\scalebox{0.6}{\textbf{G}}}}
\begin{itemize}
    \item Non viene visualizzato alcun elemento
\end{itemize}

\textbf{Scenario principale\textsubscript{\scalebox{0.6}{\textbf{G}}}}
\begin{enumerate}
    \item Il sistema interroga il database per recuperare lo storico
    \item Il sistema non trova record associati all'utente e mostra un messaggio di avviso
\end{enumerate}


\vspace{0.5cm}

\subUseCase{Visualizzazione informazioni elemento}

\begin{figure}[H]
    \centering
    \includegraphics[width=1\textwidth]{Diagrammi casi d'uso/UC1B-2.jpg}
    \caption{Diagramma del caso d'uso UC-1B.2 – Visualizzazione informazioni elemento}
\end{figure}

\textbf{Attori\textsubscript{\scalebox{0.6}{\textbf{G}}}}
\begin{itemize}
    \item Redattore\textsubscript{\scalebox{0.6}{\textbf{G}}}
\end{itemize}

\textbf{Pre-condizioni\textsubscript{\scalebox{0.6}{\textbf{G}}}}
\begin{itemize}
    \item La lista dello storico contiene almeno un elemento
\end{itemize}

\textbf{Post-condizioni\textsubscript{\scalebox{0.6}{\textbf{G}}}}
\begin{itemize}
    \item L'utente visualizza i dettagli completi di una specifica generazione (prompt\textsubscript{\scalebox{0.6}{\textbf{G}}} completo, parametri usati, output ottenuto, data)
\end{itemize}

\textbf{Scenario principale\textsubscript{\scalebox{0.6}{\textbf{G}}}}
\begin{enumerate}
    \item L'utente seleziona un elemento dalla lista dello storico
    \item Il sistema espande l'elemento o apre una modale mostrando tutti i metadati associati a quella generazione
\end{enumerate}

\newpage

\textbf{Relazioni con altri Casi d'uso\textsubscript{\scalebox{0.6}{\textbf{G}}}}
\begin{itemize}
    \item \textit{include}: 
    \begin{itemize}
        \item UC-1B.3 – Visualizzazione stile
        \item UC-1B.4 – Visualizzazione risultato
        \item UC-1B.5 – Visualizzazione timestamp
        \item UC-1B.6 – Visualizzazione valutazione
        \item UC-1B.7– Visualizzazione prompt\textsubscript{\scalebox{0.6}{\textbf{G}}}
        \item UC-1B.8 – Visualizzazione tono
    \end{itemize}
\end{itemize}

\vspace{0.5cm}

\subUseCase{Visualizzazione stile}

\textbf{Attori\textsubscript{\scalebox{0.6}{\textbf{G}}}}
\begin{itemize}
    \item Redattore\textsubscript{\scalebox{0.6}{\textbf{G}}}
\end{itemize}

\textbf{Pre-condizioni\textsubscript{\scalebox{0.6}{\textbf{G}}}}
\begin{itemize}
    \item L'utente ha aperto la visualizzazione di dettaglio di un elemento dello storico
\end{itemize}

\textbf{Post-condizioni\textsubscript{\scalebox{0.6}{\textbf{G}}}}
\begin{itemize}
    \item Viene visualizzato il parametro "Stile" utilizzato per la generazione
\end{itemize}

\textbf{Scenario principale\textsubscript{\scalebox{0.6}{\textbf{G}}}}
\begin{enumerate}
    \item Il sistema recupera e visualizza la tipologia di contenuto o stile (es. Articolo, Post Social) salvato con la generazione
\end{enumerate}


\vspace{0.5cm}

\subUseCase{Visualizzazione risultato}

\textbf{Attori\textsubscript{\scalebox{0.6}{\textbf{G}}}}
\begin{itemize}
    \item Redattore\textsubscript{\scalebox{0.6}{\textbf{G}}}
\end{itemize}

\textbf{Pre-condizioni\textsubscript{\scalebox{0.6}{\textbf{G}}}}
\begin{itemize}
    \item L'utente ha aperto la visualizzazione di dettaglio di un elemento dello storico
\end{itemize}

\textbf{Post-condizioni\textsubscript{\scalebox{0.6}{\textbf{G}}}}
\begin{itemize}
    \item Viene visualizzato il risultato della generazione prodotta dall'AI
\end{itemize}

\textbf{Scenario principale\textsubscript{\scalebox{0.6}{\textbf{G}}}}
\begin{enumerate}
    \item Il sistema mostra il risultato prodotto dall'AI in risposta al prompt\textsubscript{\scalebox{0.6}{\textbf{G}}}, mantenendo la formattazione originale se presente
\end{enumerate}


\vspace{0.5cm}

\subUseCase{Visualizzazione timestamp}

\textbf{Attori\textsubscript{\scalebox{0.6}{\textbf{G}}}}
\begin{itemize}
    \item Redattore\textsubscript{\scalebox{0.6}{\textbf{G}}}
\end{itemize}

\textbf{Pre-condizioni\textsubscript{\scalebox{0.6}{\textbf{G}}}}
\begin{itemize}
    \item L'utente ha aperto la visualizzazione di dettaglio di un elemento dello storico
\end{itemize}

\textbf{Post-condizioni\textsubscript{\scalebox{0.6}{\textbf{G}}}}
\begin{itemize}
    \item Viene visualizzata la data e l'ora in cui è stata effettuata la generazione
\end{itemize}

\textbf{Scenario principale\textsubscript{\scalebox{0.6}{\textbf{G}}}}
\begin{enumerate}
    \item Il sistema mostra il timestamp (data e ora) di creazione del contenuto
\end{enumerate}


\vspace{0.5cm}

\subUseCase{Visualizzazione valutazione}

\textbf{Attori\textsubscript{\scalebox{0.6}{\textbf{G}}}}
\begin{itemize}
    \item Redattore\textsubscript{\scalebox{0.6}{\textbf{G}}}
\end{itemize}

\textbf{Pre-condizioni\textsubscript{\scalebox{0.6}{\textbf{G}}}}
\begin{itemize}
    \item L'utente ha aperto la visualizzazione di dettaglio di un elemento dello storico
\end{itemize}

\textbf{Post-condizioni\textsubscript{\scalebox{0.6}{\textbf{G}}}}
\begin{itemize}
    \item Viene visualizzato il feedback o il voto assegnato dall'utente (se presente)
\end{itemize}

\textbf{Scenario principale\textsubscript{\scalebox{0.6}{\textbf{G}}}}
\begin{enumerate}
    \item Il sistema verifica\textsubscript{\scalebox{0.6}{\textbf{G}}} se esiste una valutazione associata al contenuto
    \item Se presente, mostra l'indicatore grafico (es. stelle, thumbs up/down); in caso contrario mostra un campo vuoto o l'opzione per aggiungere una valutazione
\end{enumerate}


\vspace{0.5cm}

\subUseCase{Visualizzazione prompt\textsubscript{\scalebox{0.6}{\textbf{G}}}}

\textbf{Attori\textsubscript{\scalebox{0.6}{\textbf{G}}}}
\begin{itemize}
    \item Redattore\textsubscript{\scalebox{0.6}{\textbf{G}}}
\end{itemize}

\textbf{Pre-condizioni\textsubscript{\scalebox{0.6}{\textbf{G}}}}
\begin{itemize}
    \item L'utente ha aperto la visualizzazione di dettaglio di un elemento dello storico
\end{itemize}

\textbf{Post-condizioni\textsubscript{\scalebox{0.6}{\textbf{G}}}}
\begin{itemize}
    \item Viene visualizzato il testo completo del prompt\textsubscript{\scalebox{0.6}{\textbf{G}}} inserito originariamente dall'utente
\end{itemize}

\textbf{Scenario principale\textsubscript{\scalebox{0.6}{\textbf{G}}}}
\begin{enumerate}
    \item Il sistema recupera dal database il testo della richiesta (prompt\textsubscript{\scalebox{0.6}{\textbf{G}}}) associato alla generazione selezionata e lo mostra a video
\end{enumerate}


\vspace{0.5cm}

\subUseCase{Visualizzazione tono}

\textbf{Attori\textsubscript{\scalebox{0.6}{\textbf{G}}}}
\begin{itemize}
    \item Redattore\textsubscript{\scalebox{0.6}{\textbf{G}}}
\end{itemize}

\textbf{Pre-condizioni\textsubscript{\scalebox{0.6}{\textbf{G}}}}
\begin{itemize}
    \item L'utente ha aperto la visualizzazione di dettaglio di un elemento dello storico
\end{itemize}

\textbf{Post-condizioni\textsubscript{\scalebox{0.6}{\textbf{G}}}}
\begin{itemize}
    \item Viene visualizzato il parametro "Tono" utilizzato per la generazione
\end{itemize}

\textbf{Scenario principale\textsubscript{\scalebox{0.6}{\textbf{G}}}}
\begin{enumerate}
    \item Il sistema recupera e visualizza l'etichetta del tono (es. Professionale, Amichevole) salvato con la generazione
\end{enumerate}

\vspace{0.5cm}

\useCase{Visualizzazione anteprima contenuto generato}

\begin{figure}[H]
    \centering
    \includegraphics[width=1\textwidth]{Diagrammi casi d'uso/UC1C.jpg}
    \caption{Diagramma del caso d'uso UC-1C – Visualizzazione anteprima contenuto generato}
\end{figure}

\textbf{Attori\textsubscript{\scalebox{0.6}{\textbf{G}}}}
\begin{itemize}
    \item Redattore\textsubscript{\scalebox{0.6}{\textbf{G}}}
\end{itemize}

\textbf{Pre-condizioni\textsubscript{\scalebox{0.6}{\textbf{G}}}}
\begin{itemize}
    \item Il sistema AI ha completato l'elaborazione della richiesta
\end{itemize}

\textbf{Post-condizioni\textsubscript{\scalebox{0.6}{\textbf{G}}}}
\begin{itemize}
    \item La generazione è visibile e salvata nello storico dell'utente
\end{itemize}

\textbf{Scenario principale\textsubscript{\scalebox{0.6}{\textbf{G}}}}
\begin{enumerate}
    \item Il sistema riceve l'output dall'AI e lo renderizza nell'area di visualizzazione principale, permettendo all'utente di leggerlo
\end{enumerate}

\textbf{Relazioni con altri Casi d'uso\textsubscript{\scalebox{0.6}{\textbf{G}}}}
\begin{itemize}
    \item \textit{include}: 
    \begin{itemize}
        \item UC-1C.1 - Visualizzazione titolo
        \item UC-1C.2 - Visualizzazione testo
        \item UC-1C.3 - Visualizzazione immagine
    \end{itemize}
\end{itemize}

\vspace{0.5cm}

\subUseCase{Visualizzazione titolo}

\textbf{Attori\textsubscript{\scalebox{0.6}{\textbf{G}}}}
\begin{itemize}
    \item Redattore\textsubscript{\scalebox{0.6}{\textbf{G}}}
\end{itemize}

\textbf{Pre-condizioni\textsubscript{\scalebox{0.6}{\textbf{G}}}}
\begin{itemize}
    \item Il sistema AI ha completato l'elaborazione della richiesta
\end{itemize}

\textbf{Post-condizioni\textsubscript{\scalebox{0.6}{\textbf{G}}}}
\begin{itemize}
    \item Il titolo generato è visibile dall'utente
\end{itemize}

\textbf{Scenario principale\textsubscript{\scalebox{0.6}{\textbf{G}}}}
\begin{enumerate}
    \item L'utente ha fatto la generazione di un contenuto e all'interno dell'anteprima è visibile il titolo generato
\end{enumerate}

\vspace{0.5cm}

\subUseCase{Visualizzazione testo}

\textbf{Attori\textsubscript{\scalebox{0.6}{\textbf{G}}}}
\begin{itemize}
    \item Redattore\textsubscript{\scalebox{0.6}{\textbf{G}}}
\end{itemize}

\textbf{Pre-condizioni\textsubscript{\scalebox{0.6}{\textbf{G}}}}
\begin{itemize}
    \item Il sistema AI ha completato l'elaborazione della richiesta
\end{itemize}

\textbf{Post-condizioni\textsubscript{\scalebox{0.6}{\textbf{G}}}}
\begin{itemize}
    \item Il testo generato è visibile dall'utente
\end{itemize}

\textbf{Scenario principale\textsubscript{\scalebox{0.6}{\textbf{G}}}}
\begin{enumerate}
    \item L'utente ha fatto la generazione di un contenuto e all'interno dell'anteprima è visibile il testo generato
\end{enumerate}

\vspace{0.5cm}

\subUseCase{Visualizzazione immagine}

\textbf{Attori\textsubscript{\scalebox{0.6}{\textbf{G}}}}
\begin{itemize}
    \item Redattore\textsubscript{\scalebox{0.6}{\textbf{G}}}
\end{itemize}

\textbf{Pre-condizioni\textsubscript{\scalebox{0.6}{\textbf{G}}}}
\begin{itemize}
    \item Il sistema AI ha completato l'elaborazione della richiesta
\end{itemize}

\textbf{Post-condizioni\textsubscript{\scalebox{0.6}{\textbf{G}}}}
\begin{itemize}
    \item L'immagine generata è visibile dall'utente
\end{itemize}

\textbf{Scenario principale\textsubscript{\scalebox{0.6}{\textbf{G}}}}
\begin{enumerate}
    \item L'utente ha fatto la generazione di un contenuto e all'interno dell'anteprima è visibile l'immagine generata
\end{enumerate}

\vspace{0.5cm}




\useCase{Modifica immagine contenuto generato}

\begin{figure}[H]
    \centering
    \includegraphics[width=1\textwidth]{Diagrammi casi d'uso/UC1D.jpg}
    \caption{Diagramma del caso d'uso UC-1D – Modifica immagine contenuto generato}
\end{figure}

\textbf{Attori\textsubscript{\scalebox{0.6}{\textbf{G}}}}
\begin{itemize}
    \item Redattore\textsubscript{\scalebox{0.6}{\textbf{G}}}
\end{itemize}

\textbf{Pre-condizioni\textsubscript{\scalebox{0.6}{\textbf{G}}}}
\begin{itemize}
    \item L'utente ha generato un contenuto con l'AI Assistant\textsubscript{\scalebox{0.6}{\textbf{G}}}
    \item L'utente si trova nella pagina di modifica del contenuto 
\end{itemize}

\textbf{Post-condizioni\textsubscript{\scalebox{0.6}{\textbf{G}}}}
\begin{itemize}
    \item L'immagine associata al contenuto viene aggiornata con quella caricata dall'utente
\end{itemize}

\textbf{Scenario principale\textsubscript{\scalebox{0.6}{\textbf{G}}}}
\begin{enumerate}
    \item L'utente clicca sull'opzione per cambiare l'immagine (upload o selezione da libreria)
    \item L'utente seleziona un file valido dal proprio dispositivo
    \item Il sistema sostituisce l'immagine generata dall'AI con quella fornita dall'utente
\end{enumerate}

\textbf{Scenario secondario\textsubscript{\scalebox{0.6}{\textbf{G}}}}
\begin{enumerate}
    \item Il file immagine caricato dall'utente non è valido
\end{enumerate}

\textbf{Relazioni con altri Casi d'uso\textsubscript{\scalebox{0.6}{\textbf{G}}}}
\begin{itemize}
    \item \textit{extend}: 
    \begin{itemize}
        \item UC-1D.1 - File immagine non valido
    \end{itemize}
\end{itemize}

\vspace{0.5cm}


\subUseCase{File immagine non valido}

\textbf{Attori\textsubscript{\scalebox{0.6}{\textbf{G}}}}
\begin{itemize}
    \item Redattore\textsubscript{\scalebox{0.6}{\textbf{G}}}
\end{itemize}

\textbf{Pre-condizioni\textsubscript{\scalebox{0.6}{\textbf{G}}}}
\begin{itemize}
    \item L'utente si trova nella pagina di modifica del contenuto 
    \item L'utente ha selezionato un file immagine non valido (formato non supportato, dimensioni eccessive, file corrotto)
\end{itemize}

\textbf{Post-condizioni\textsubscript{\scalebox{0.6}{\textbf{G}}}}
\begin{itemize}
    \item L'immagine non viene aggiornata e l'utente riceve un messaggio di errore che indica il problema riscontrato
\end{itemize}

\textbf{Scenario principale\textsubscript{\scalebox{0.6}{\textbf{G}}}}
\begin{enumerate}
    \item L'utente clicca sull'opzione per cambiare l'immagine (upload o selezione da libreria)
    \item L'utente seleziona un file non valido dal proprio dispositivo
    \item Il sistema rimane invariato e mostra un messaggio di errore che indica il problema riscontrato (es. "Formato non supportato", "Dimensioni eccessive")
\end{enumerate}

\vspace{0.5cm}


\useCase{Modifica titolo contenuto generato}

\begin{figure}[H]
    \centering
    \includegraphics[width=0.8\textwidth]{Diagrammi casi d'uso/UC1E.jpg}
    \caption{Diagramma del caso d'uso UC-1E – Modifica titolo contenuto generato}
\end{figure}

\textbf{Attori\textsubscript{\scalebox{0.6}{\textbf{G}}}}
\begin{itemize}
    \item Redattore\textsubscript{\scalebox{0.6}{\textbf{G}}}
\end{itemize}

\textbf{Pre-condizioni\textsubscript{\scalebox{0.6}{\textbf{G}}}}
\begin{itemize}
    \item L'utente ha generato un contenuto con l'AI Assistant\textsubscript{\scalebox{0.6}{\textbf{G}}}
    \item L'utente si trova nella pagina di modifica del contenuto
\end{itemize}

\textbf{Post-condizioni\textsubscript{\scalebox{0.6}{\textbf{G}}}}
\begin{itemize}
    \item Il titolo del contenuto risulta modificato secondo l'input dell'utente
\end{itemize}

\textbf{Scenario principale\textsubscript{\scalebox{0.6}{\textbf{G}}}}
\begin{enumerate}
    \item L'utente modifica il testo presente nella casella di input del titolo
\end{enumerate}


\vspace{0.5cm}

\useCase{Modifica testo contenuto generato}

\begin{figure}[H]
    \centering
    \includegraphics[width=0.8\textwidth]{Diagrammi casi d'uso/UC1F.jpg}
    \caption{Diagramma del caso d'uso UC-1F – Modifica testo contenuto generato}
\end{figure}

\textbf{Attori\textsubscript{\scalebox{0.6}{\textbf{G}}}}
\begin{itemize}
    \item Redattore\textsubscript{\scalebox{0.6}{\textbf{G}}}
\end{itemize}

\textbf{Pre-condizioni\textsubscript{\scalebox{0.6}{\textbf{G}}}}
\begin{itemize}
    \item L'utente si trova nella pagina di modifica del contenuto
\end{itemize}

\textbf{Post-condizioni\textsubscript{\scalebox{0.6}{\textbf{G}}}}
\begin{itemize}
    \item Il corpo del testo risulta modificato secondo l'input dell'utente
\end{itemize}

\textbf{Scenario principale\textsubscript{\scalebox{0.6}{\textbf{G}}}}
\begin{enumerate}
    \item L'utente agisce sull'editor di testo (corpo del contenuto) aggiungendo, rimuovendo o formattando il testo generato
\end{enumerate}


\vspace{0.5cm}



\useCase{Annulla modifiche a contenuto generato}

\begin{figure}[H]
    \centering
    \includegraphics[width=0.7\textwidth]{Diagrammi casi d'uso/UC1G.jpg}
    \caption{Diagramma del caso d'uso UC-1G – Annulla modifiche a contenuto generato}
\end{figure}

\textbf{Attori\textsubscript{\scalebox{0.6}{\textbf{G}}}}
\begin{itemize}
    \item Redattore\textsubscript{\scalebox{0.6}{\textbf{G}}}
\end{itemize}

\textbf{Pre-condizioni\textsubscript{\scalebox{0.6}{\textbf{G}}}}
\begin{itemize}
    \item L'utente si trova nella pagina di modifica
\end{itemize}

\textbf{Post-condizioni\textsubscript{\scalebox{0.6}{\textbf{G}}}}
\begin{itemize}
    \item Le modifiche non salvate vengono scartate e l'utente ritorna alla visualizzazione precedente
\end{itemize}

\textbf{Scenario principale\textsubscript{\scalebox{0.6}{\textbf{G}}}}
\begin{enumerate}
    \item L'utente decide di non applicare le modifiche correnti
    \item Il sistema ripristina lo stato del contenuto a quello precedente l'apertura dell'editor
\end{enumerate}


\vspace{0.5cm}


\useCase{Riutilizza contenuto salvato nello storico}

\begin{figure}[H]
    \centering
    \includegraphics[width=0.7\textwidth]{Diagrammi casi d'uso/UC1H.jpg}
    \caption{Diagramma del caso d'uso UC-1H – Riutilizza contenuto salvato nello storico}
\end{figure}


\textbf{Attori\textsubscript{\scalebox{0.6}{\textbf{G}}}}
\begin{itemize}
    \item Redattore\textsubscript{\scalebox{0.6}{\textbf{G}}}
\end{itemize}

\textbf{Attori\textsubscript{\scalebox{0.6}{\textbf{G}}} secondari}
\begin{itemize}
    \item AI Post Generator
\end{itemize}

\textbf{Pre-condizioni\textsubscript{\scalebox{0.6}{\textbf{G}}}}
\begin{itemize}
    \item L'utente si trova all'interno della sezione dedicata allo storico
    \item Lo storico ha almeno un elemento
\end{itemize}

\textbf{Post-condizioni\textsubscript{\scalebox{0.6}{\textbf{G}}}}
\begin{itemize}
    \item Il sistema avvia una nuova generazione utilizzando esattamente gli stessi parametri (prompt\textsubscript{\scalebox{0.6}{\textbf{G}}}, tono, stile) dell'elemento selezionato
\end{itemize}

\textbf{Scenario principale\textsubscript{\scalebox{0.6}{\textbf{G}}}}
\begin{enumerate}
    \item L'utente richiede di riutilizzare un elemento dello storico
    \item Il sistema invia direttamente i dati al motore AI per generare un nuovo output senza richiedere modifiche all'utente
\end{enumerate}


\vspace{0.5cm}

\useCase{Duplica contenuto salvato nello storico}

\begin{figure}[H]
    \centering
    \includegraphics[width=0.7\textwidth]{Diagrammi casi d'uso/UC1I.jpg}
    \caption{Diagramma del caso d'uso UC-1I – Duplica contenuto salvato nello storico}
\end{figure}

\textbf{Attori\textsubscript{\scalebox{0.6}{\textbf{G}}}}
\begin{itemize}
    \item Redattore\textsubscript{\scalebox{0.6}{\textbf{G}}}
\end{itemize}

\textbf{Pre-condizioni\textsubscript{\scalebox{0.6}{\textbf{G}}}}
\begin{itemize}
    \item L'utente si trova all'interno della sezione dedicata allo storico
    \item Lo storico ha almeno un elemento
\end{itemize}

\textbf{Post-condizioni\textsubscript{\scalebox{0.6}{\textbf{G}}}}
\begin{itemize}
    \item L'utente viene riportato al Modulo\textsubscript{\scalebox{0.6}{\textbf{G}}} di generazione (UC-1A) con i campi prompt\textsubscript{\scalebox{0.6}{\textbf{G}}}, tono e stile già compilati con i dati storici, pronti per essere modificati
\end{itemize}

\textbf{Scenario principale\textsubscript{\scalebox{0.6}{\textbf{G}}}}
\begin{enumerate}
    \item L'utente clicca sul comando "Duplica" associato ad un elemento dello storico
    \item Il sistema reindirizza l'utente alla pagina di creazione, pre-popolando i campi con i valori recuperati dallo storico
    \item L'utente può ora modificare il prompt\textsubscript{\scalebox{0.6}{\textbf{G}}} o i parametri prima di generare
\end{enumerate}


\vspace{0.5cm}

\useCase{Filtraggio lista generazioni storico}

\begin{figure}[H]
    \centering
    \includegraphics[width=0.8\textwidth]{Diagrammi casi d'uso/UC1J.jpg}
    \caption{Diagramma del caso d'uso UC-1J – Filtraggio lista generazioni storico}
\end{figure}

\textbf{Attori\textsubscript{\scalebox{0.6}{\textbf{G}}}}
\begin{itemize}
    \item Redattore\textsubscript{\scalebox{0.6}{\textbf{G}}}
\end{itemize}

\textbf{Pre-condizioni\textsubscript{\scalebox{0.6}{\textbf{G}}}}
\begin{itemize}
    \item L'utente si trova all'interno della sezione dedicata allo storico
    \item Lo storico ha almeno un elemento
\end{itemize}

\textbf{Post-condizioni\textsubscript{\scalebox{0.6}{\textbf{G}}}}
\begin{itemize}
    \item Viene applicato un filtro alla lista delle generazioni dello storico
\end{itemize}

\textbf{Scenario principale\textsubscript{\scalebox{0.6}{\textbf{G}}}}
\begin{enumerate}
    \item L'utente inserisce un testo chiave (es. parte del prompt\textsubscript{\scalebox{0.6}{\textbf{G}}} o data) nella barra di ricerca
    \item Il sistema filtra gli elementi dello storico in tempo reale
\end{enumerate}


\vspace{0.5cm}

\useCase{Visualizzazione lista elementi storico filtrata}

% \begin{figure}[H]
%     \centering
%     \includegraphics[width=1\textwidth]{Diagrammi casi d'uso/UC1K.jpg}
%     \caption{Diagramma del caso d'uso UC-1K – Visualizzazione lista elementi storico filtrata}
% \end{figure}

\textbf{Attori\textsubscript{\scalebox{0.6}{\textbf{G}}}}
\begin{itemize}
    \item Redattore\textsubscript{\scalebox{0.6}{\textbf{G}}}
\end{itemize}

\textbf{Pre-condizioni\textsubscript{\scalebox{0.6}{\textbf{G}}}}
\begin{itemize}
    \item La lista dello storico contiene almeno un elemento 
    \item L'utente ha inserito un criterio di ricerca valido
\end{itemize}

\textbf{Post-condizioni\textsubscript{\scalebox{0.6}{\textbf{G}}}}
\begin{itemize}
    \item L'utente visualizza la lista di elementi (coerenti con la ricerca) salvati nello storico nel suo insieme
\end{itemize}

\textbf{Scenario principale\textsubscript{\scalebox{0.6}{\textbf{G}}}}
\begin{enumerate}
    \item L'utente si trova nel Modulo\textsubscript{\scalebox{0.6}{\textbf{G}}} storico dell'AI Assistant\textsubscript{\scalebox{0.6}{\textbf{G}}} Generativo
    \item L'utente vede l'elenco di tutti gli elementi salvati con i relativi dati principali (prompt\textsubscript{\scalebox{0.6}{\textbf{G}}} parziale, tono, stile, data)
\end{enumerate}


\textbf{Relazioni con altri Casi d'uso\textsubscript{\scalebox{0.6}{\textbf{G}}}}
\begin{itemize}
    % \item \textit{include}: 
    % \begin{itemize}
    %     \item UC-1B.2 – Visualizzazione informazioni elemento
    % \end{itemize}
    \item \textit{generalizzazione}: 
    \begin{itemize}
        \item UC-1B – Visualizzazione lista elementi storico
    \end{itemize}
\end{itemize}

\vspace{0.5cm}

\useCase{Rigenera contenuto tramite AI}

\begin{figure}[H]
    \centering
    \includegraphics[width=0.7\textwidth]{Diagrammi casi d'uso/UC1L.jpg}
    \caption{Diagramma del caso d'uso UC-1L – Rigenera contenuto tramite AI}
\end{figure}

\textbf{Attori\textsubscript{\scalebox{0.6}{\textbf{G}}}}
\begin{itemize}
    \item Redattore\textsubscript{\scalebox{0.6}{\textbf{G}}}
\end{itemize}

\textbf{Attori\textsubscript{\scalebox{0.6}{\textbf{G}}} secondari}
\begin{itemize}
    \item AI Post Generator
\end{itemize}

\textbf{Pre-condizioni\textsubscript{\scalebox{0.6}{\textbf{G}}}}
\begin{itemize}
    \item L'utente si trova nel Modulo\textsubscript{\scalebox{0.6}{\textbf{G}}} AI Assistant\textsubscript{\scalebox{0.6}{\textbf{G}}} generativo
    \item È stato generato un contenuto tramite l'AI Assistant\textsubscript{\scalebox{0.6}{\textbf{G}}}
\end{itemize}

\textbf{Post-condizioni\textsubscript{\scalebox{0.6}{\textbf{G}}}}
\begin{itemize}
    \item Il contenuto precedente viene sostituito da una nuova versione generata con gli stessi parametri
\end{itemize}

\textbf{Scenario principale\textsubscript{\scalebox{0.6}{\textbf{G}}}}
\begin{enumerate}
    \item L'utente richiede al sistema di provare nuovamente a generare il contenuto
    \item Il sistema invia una nuova richiesta all'AI
    \item Il nuovo risultato sovrascrive quello visualizzato in anteprima
\end{enumerate}


\vspace{0.5cm}

\useCase{Valuta contenuto generato}

\begin{figure}[H]
    \centering
    \includegraphics[width=0.7\textwidth]{Diagrammi casi d'uso/UC1M.jpg}
    \caption{Diagramma del caso d'uso UC-1M – Valuta contenuto generato}
\end{figure}

\textbf{Attori\textsubscript{\scalebox{0.6}{\textbf{G}}}}
\begin{itemize}
    \item Redattore\textsubscript{\scalebox{0.6}{\textbf{G}}}
\end{itemize}

\textbf{Pre-condizioni\textsubscript{\scalebox{0.6}{\textbf{G}}}}
\begin{itemize}
    \item L'utente si trova nel Modulo\textsubscript{\scalebox{0.6}{\textbf{G}}} AI Assistant\textsubscript{\scalebox{0.6}{\textbf{G}}} generativo
    \item È stato generato un contenuto tramite l'AI Assistant\textsubscript{\scalebox{0.6}{\textbf{G}}}
\end{itemize}

\textbf{Post-condizioni\textsubscript{\scalebox{0.6}{\textbf{G}}}}
\begin{itemize}
    \item Il feedback dell'utente viene registrato e associato al contenuto nello storico
\end{itemize}

\textbf{Scenario principale\textsubscript{\scalebox{0.6}{\textbf{G}}}}
\begin{enumerate}
    \item L'utente esprime un giudizio sulla qualità\textsubscript{\scalebox{0.6}{\textbf{G}}} del contenuto (es. pollice in su/giù o stelle)
    \item Il sistema salva la valutazione
\end{enumerate}


\vspace{0.5cm}

\useCase{Scarta contenuto generato}

\begin{figure}[H]
    \centering
    \includegraphics[width=0.7\textwidth]{Diagrammi casi d'uso/UC1N.jpg}
    \caption{Diagramma del caso d'uso UC-1N – Scarta contenuto generato}
\end{figure}

\textbf{Attori\textsubscript{\scalebox{0.6}{\textbf{G}}}}
\begin{itemize}
    \item Redattore\textsubscript{\scalebox{0.6}{\textbf{G}}}
\end{itemize}

\textbf{Pre-condizioni\textsubscript{\scalebox{0.6}{\textbf{G}}}}
\begin{itemize}
    \item L'utente si trova nel Modulo\textsubscript{\scalebox{0.6}{\textbf{G}}} AI Assistant\textsubscript{\scalebox{0.6}{\textbf{G}}} generativo
    \item È stato generato un contenuto tramite l'AI Assistant\textsubscript{\scalebox{0.6}{\textbf{G}}}
\end{itemize}

\textbf{Post-condizioni\textsubscript{\scalebox{0.6}{\textbf{G}}}}
\begin{itemize}
    \item La generazione viene salvata nello storico con valutazione minima
\end{itemize}

\textbf{Scenario principale\textsubscript{\scalebox{0.6}{\textbf{G}}}}
\begin{enumerate}
    \item L'utente decide che il contenuto non è utile e vuole ricominciare da zero
    \item Il sistema pulisce l'area di anteprima e riporta il focus sui campi di input
\end{enumerate}


\vspace{0.5cm}

\useCase{Salvataggio contenuto generato}

\begin{figure}[H]
    \centering
    \includegraphics[width=0.7\textwidth]{Diagrammi casi d'uso/UC1O.jpg}
    \caption{Diagramma del caso d'uso UC-1O – Salva contenuto generato}
\end{figure}

\textbf{Attori\textsubscript{\scalebox{0.6}{\textbf{G}}}}
\begin{itemize}
    \item Redattore\textsubscript{\scalebox{0.6}{\textbf{G}}}
\end{itemize}

\textbf{Pre-condizioni\textsubscript{\scalebox{0.6}{\textbf{G}}}}
\begin{itemize}
    \item L'utente si trova nel Modulo\textsubscript{\scalebox{0.6}{\textbf{G}}} AI Assistant\textsubscript{\scalebox{0.6}{\textbf{G}}} generativo
    \item È stato generato un contenuto tramite l'AI Assistant\textsubscript{\scalebox{0.6}{\textbf{G}}}
\end{itemize}

\textbf{Post-condizioni\textsubscript{\scalebox{0.6}{\textbf{G}}}}
\begin{itemize}
    \item Il contenuto viene salvato all'interno del sistema
\end{itemize}

\textbf{Scenario principale\textsubscript{\scalebox{0.6}{\textbf{G}}}}
\begin{enumerate}
    \item L'utente decide che il contenuto è appropriato e vuole salvarlo nel sistema
    \item Il sistema salva il contenuto generato nel database, rendendolo accessibile per usi futuri
\end{enumerate}


\vspace{0.5cm}

\useCase{Aggiunta tono}

\begin{figure}[H]
    \centering
    \includegraphics[width=0.7\textwidth]{Diagrammi casi d'uso/UC1P.jpg}
    \caption{Diagramma del caso d'uso UC-1P – Aggiunta tono}
\end{figure}

\textbf{Attori\textsubscript{\scalebox{0.6}{\textbf{G}}}}
\begin{itemize}
    \item Redattore\textsubscript{\scalebox{0.6}{\textbf{G}}}
\end{itemize}

\textbf{Pre-condizioni\textsubscript{\scalebox{0.6}{\textbf{G}}}}
\begin{itemize}
    \item L'utente si trova nel Modulo\textsubscript{\scalebox{0.6}{\textbf{G}}} AI Assistant\textsubscript{\scalebox{0.6}{\textbf{G}}} generativo
\end{itemize}

\textbf{Post-condizioni\textsubscript{\scalebox{0.6}{\textbf{G}}}}
\begin{itemize}
    \item Tra le opzioni disponibili nella scelta dei toni per la generazione di un contenuto adesso è presente il nuovo tono inserito
\end{itemize}

\textbf{Scenario principale\textsubscript{\scalebox{0.6}{\textbf{G}}}}
\begin{enumerate}
    \item L'utente inserisce un nuovo tono tramite lo strumento apposito
    \item Il tono viene salvato all'interno del sistema ed è utilizzabile per generazioni future
\end{enumerate}


\vspace{0.5cm}

\useCase{Eliminazione tono}

\begin{figure}[H]
    \centering
    \includegraphics[width=0.7\textwidth]{Diagrammi casi d'uso/UC1Q.jpg}
    \caption{Diagramma del caso d'uso UC-1Q – Eliminazione tono}
\end{figure}

\textbf{Attori\textsubscript{\scalebox{0.6}{\textbf{G}}}}
\begin{itemize}
    \item Redattore\textsubscript{\scalebox{0.6}{\textbf{G}}}
\end{itemize}

\textbf{Pre-condizioni\textsubscript{\scalebox{0.6}{\textbf{G}}}}
\begin{itemize}
    \item L'utente si trova nel Modulo\textsubscript{\scalebox{0.6}{\textbf{G}}} AI Assistant\textsubscript{\scalebox{0.6}{\textbf{G}}} generativo
\end{itemize}

\textbf{Post-condizioni\textsubscript{\scalebox{0.6}{\textbf{G}}}}
\begin{itemize}
    \item Tra le opzioni disponibili nella scelta dei toni per la generazione di un contenuto adesso non si trova più un tono precedentemente inserito
\end{itemize}

\textbf{Scenario principale\textsubscript{\scalebox{0.6}{\textbf{G}}}}
\begin{enumerate}
    \item L'utente elimina un tono precedentemente inserito tramite lo strumento apposito
    \item Il tono viene rimosso dal sistema e non è più utilizzabile per generazioni future
\end{enumerate}


\vspace{0.5cm}

\useCase{Aggiunta stile}

\begin{figure}[H]
    \centering
    \includegraphics[width=0.7\textwidth]{Diagrammi casi d'uso/UC1R.jpg}
    \caption{Diagramma del caso d'uso UC-1R – Aggiunta stile}
\end{figure}

\textbf{Attori\textsubscript{\scalebox{0.6}{\textbf{G}}}}
\begin{itemize}
    \item Redattore\textsubscript{\scalebox{0.6}{\textbf{G}}}
\end{itemize}

\textbf{Pre-condizioni\textsubscript{\scalebox{0.6}{\textbf{G}}}}
\begin{itemize}
    \item L'utente si trova nel Modulo\textsubscript{\scalebox{0.6}{\textbf{G}}} AI Assistant\textsubscript{\scalebox{0.6}{\textbf{G}}} generativo
\end{itemize}

\textbf{Post-condizioni\textsubscript{\scalebox{0.6}{\textbf{G}}}}
\begin{itemize}
    \item Tra le opzioni disponibili nella scelta degli stili per la generazione di un contenuto adesso è presente il nuovo stile inserito
\end{itemize}

\textbf{Scenario principale\textsubscript{\scalebox{0.6}{\textbf{G}}}}
\begin{enumerate}
    \item L'utente inserisce un nuovo stile tramite lo strumento apposito
    \item Il stile viene salvato all'interno del sistema ed è utilizzabile per generazioni future
\end{enumerate}


\vspace{0.5cm}

\useCase{Eliminazione stile}

\begin{figure}[H]
    \centering
    \includegraphics[width=0.7\textwidth]{Diagrammi casi d'uso/UC1S.jpg}
    \caption{Diagramma del caso d'uso UC-1S – Eliminazione stile}
\end{figure}

\textbf{Attori\textsubscript{\scalebox{0.6}{\textbf{G}}}}
\begin{itemize}
    \item Redattore\textsubscript{\scalebox{0.6}{\textbf{G}}}
\end{itemize}

\textbf{Pre-condizioni\textsubscript{\scalebox{0.6}{\textbf{G}}}}
\begin{itemize}
    \item L'utente si trova nel Modulo\textsubscript{\scalebox{0.6}{\textbf{G}}} AI Assistant\textsubscript{\scalebox{0.6}{\textbf{G}}} generativo
\end{itemize}

\textbf{Post-condizioni\textsubscript{\scalebox{0.6}{\textbf{G}}}}
\begin{itemize}
    \item Tra le opzioni disponibili nella scelta degli stili per la generazione di un contenuto adesso non si trova più uno stile precedentemente inserito
\end{itemize}

\textbf{Scenario principale\textsubscript{\scalebox{0.6}{\textbf{G}}}}
\begin{enumerate}
    \item L'utente elimina uno stile precedentemente inserito tramite lo strumento apposito
    \item Lo stile viene rimosso dal sistema e non è più utilizzabile per generazioni future
\end{enumerate}


\vspace{0.5cm}


\subsection{Sezione 2 – Modulo\textsubscript{\scalebox{0.6}{\textbf{G}}} AI Co-Pilot\textsubscript{\scalebox{0.6}{\textbf{G}}} per i Consulenti del Lavoro (CdL)\textsubscript{\scalebox{0.6}{\textbf{G}}}}

\renewcommand{\UCPrefix}{2}


\useCase{Analisi documento}

\begin{figure}[H]
    \centering
    \includegraphics[width=1\textwidth]{Diagrammi casi d'uso/UC2A.jpg}
    \caption{Diagramma del caso d'uso UC-2A – Analisi documento}
\end{figure}

\textbf{Attori\textsubscript{\scalebox{0.6}{\textbf{G}}}}
\begin{itemize}
    \item Operatore CdL\textsubscript{\scalebox{0.6}{\textbf{G}}}
\end{itemize}

\textbf{Attori\textsubscript{\scalebox{0.6}{\textbf{G}}} secondari}
\begin{itemize}
    \item AI Analyst
\end{itemize}


\textbf{Pre-condizioni\textsubscript{\scalebox{0.6}{\textbf{G}}}}
\begin{itemize}
    \item L'utente è entrato nel Modulo\textsubscript{\scalebox{0.6}{\textbf{G}}} AI Co-Pilot\textsubscript{\scalebox{0.6}{\textbf{G}}} per i CdL\textsubscript{\scalebox{0.6}{\textbf{G}}} dalla Dashboard\textsubscript{\scalebox{0.6}{\textbf{G}}} principale.
    \item L'utente dispone dei permessi necessari per utilizzare il Modulo\textsubscript{\scalebox{0.6}{\textbf{G}}} AI Co-Pilot\textsubscript{\scalebox{0.6}{\textbf{G}}} per i CdL\textsubscript{\scalebox{0.6}{\textbf{G}}}.
    \item Un file è stato caricato per l'analisi.
\end{itemize}

\textbf{Post-condizioni\textsubscript{\scalebox{0.6}{\textbf{G}}}}
\begin{itemize}
    \item Il documento è stato analizzato tramite l'AI e i dati sono stati inviati al sistema
\end{itemize}

\textbf{Scenario principale\textsubscript{\scalebox{0.6}{\textbf{G}}}}
\begin{enumerate}
    \item L'utente dalla Dashboard\textsubscript{\scalebox{0.6}{\textbf{G}}} principale accede al Modulo\textsubscript{\scalebox{0.6}{\textbf{G}}} AI Co-Pilot\textsubscript{\scalebox{0.6}{\textbf{G}}} per i Consulenti del Lavoro.
    \item L'utente procede a caricare il file che si desidera far analizzare
    \item Il file viene analizzato
\end{enumerate}

\textbf{Scenario secondario\textsubscript{\scalebox{0.6}{\textbf{G}}}}
\begin{enumerate}
    \item Se l'utente dispone dei dati necessari, procede ad associare i metadati al file caricato: categoria, mese/anno di competenza, azienda, reparto.
\end{enumerate}



\textbf{Relazioni con altri Casi d'uso\textsubscript{\scalebox{0.6}{\textbf{G}}}}
\begin{itemize}
    \item \textit{include}: 
    \begin{itemize}
        \item UC-2A.1 - Inserimento categoria
        \item UC-2A.2 - Inserimento mese/anno di competenza
        \item UC-2A.3 - Inserimento azienda
        \item UC-2A.4 - Inserimento reparto
    \end{itemize}
\end{itemize}

\vspace{0.5cm}

\subUseCase{Inserimento categoria}

\textbf{Attori\textsubscript{\scalebox{0.6}{\textbf{G}}}}
\begin{itemize}
    \item Operatore CdL\textsubscript{\scalebox{0.6}{\textbf{G}}}
\end{itemize}

\textbf{Pre-condizioni\textsubscript{\scalebox{0.6}{\textbf{G}}}}
\begin{itemize}
    \item L'utente si trova nel Modulo\textsubscript{\scalebox{0.6}{\textbf{G}}} AI Co-Pilot\textsubscript{\scalebox{0.6}{\textbf{G}}}
\end{itemize}

\textbf{Post-condizioni\textsubscript{\scalebox{0.6}{\textbf{G}}}}
\begin{itemize}
    \item Al file viene associata una categoria documentale (es. cedolino\textsubscript{\scalebox{0.6}{\textbf{G}}}, F24, Contratto)
\end{itemize}

\textbf{Scenario principale\textsubscript{\scalebox{0.6}{\textbf{G}}}}
\begin{enumerate}
    \item L'utente seleziona la categoria corretta per il documento che deve essere analizzato
\end{enumerate}


\vspace{0.5cm}

\subUseCase{Inserimento mese/anno di competenza}

\textbf{Attori\textsubscript{\scalebox{0.6}{\textbf{G}}}}
\begin{itemize}
    \item Operatore CdL\textsubscript{\scalebox{0.6}{\textbf{G}}}
\end{itemize}

\textbf{Pre-condizioni\textsubscript{\scalebox{0.6}{\textbf{G}}}}
\begin{itemize}
    \item L'utente si trova nel Modulo\textsubscript{\scalebox{0.6}{\textbf{G}}} AI Co-Pilot\textsubscript{\scalebox{0.6}{\textbf{G}}}
\end{itemize}

\textbf{Post-condizioni\textsubscript{\scalebox{0.6}{\textbf{G}}}}
\begin{itemize}
    \item Al file viene associato il periodo temporale di riferimento
\end{itemize}

\textbf{Scenario principale\textsubscript{\scalebox{0.6}{\textbf{G}}}}
\begin{enumerate}
    \item L'utente inserisce o seleziona tramite date-picker il mese e l'anno a cui il documento fa riferimento
\end{enumerate}


\vspace{0.5cm}

\subUseCase{Inserimento azienda}

\textbf{Attori\textsubscript{\scalebox{0.6}{\textbf{G}}}}
\begin{itemize}
    \item Operatore CdL\textsubscript{\scalebox{0.6}{\textbf{G}}}
\end{itemize}

\textbf{Pre-condizioni\textsubscript{\scalebox{0.6}{\textbf{G}}}}
\begin{itemize}
    \item L'utente si trova nel Modulo\textsubscript{\scalebox{0.6}{\textbf{G}}} AI Co-Pilot\textsubscript{\scalebox{0.6}{\textbf{G}}}
\end{itemize}

\textbf{Post-condizioni\textsubscript{\scalebox{0.6}{\textbf{G}}}}
\begin{itemize}
    \item Al file viene associata l'azienda cliente proprietaria del documento
\end{itemize}

\textbf{Scenario principale\textsubscript{\scalebox{0.6}{\textbf{G}}}}
\begin{enumerate}
    \item L'utente inizia a digitare il nome dell'azienda o la seleziona da un elenco predefinito
    \item Il sistema collega il documento all'anagrafica aziendale selezionata
\end{enumerate}


\vspace{0.5cm}

\subUseCase{Inserimento reparto}

\textbf{Attori\textsubscript{\scalebox{0.6}{\textbf{G}}}}
\begin{itemize}
    \item Operatore CdL\textsubscript{\scalebox{0.6}{\textbf{G}}}
\end{itemize}

\textbf{Pre-condizioni\textsubscript{\scalebox{0.6}{\textbf{G}}}}
\begin{itemize}
    \item L'utente si trova nel Modulo\textsubscript{\scalebox{0.6}{\textbf{G}}} AI Co-Pilot\textsubscript{\scalebox{0.6}{\textbf{G}}}
\end{itemize}

\textbf{Post-condizioni\textsubscript{\scalebox{0.6}{\textbf{G}}}}
\begin{itemize}
    \item Al file viene associato uno specifico reparto o dipartimento (opzionale)
\end{itemize}

\textbf{Scenario principale\textsubscript{\scalebox{0.6}{\textbf{G}}}}
\begin{enumerate}
    \item L'utente specifica il reparto di riferimento (es. Amministrazione, Produzione) se pertinente per il documento
\end{enumerate}


\vspace{0.5cm}

\useCase{Visualizzazione lista documenti analisi}

\begin{figure}[H]
    \centering
    \includegraphics[width=1\textwidth]{Diagrammi casi d'uso/UC2B.jpg}
    \caption{Diagramma del caso d'uso UC-2B – Visualizzazione lista documenti analisi}
\end{figure}

\textbf{Attori\textsubscript{\scalebox{0.6}{\textbf{G}}}}
\begin{itemize}
    \item Operatore CdL\textsubscript{\scalebox{0.6}{\textbf{G}}}
\end{itemize}

\textbf{Pre-condizioni\textsubscript{\scalebox{0.6}{\textbf{G}}}}
\begin{itemize}
    \item La lista dei contiene almeno un elemento
\end{itemize}

\textbf{Post-condizioni\textsubscript{\scalebox{0.6}{\textbf{G}}}}
\begin{itemize}
    \item L'utente visualizza la lista di elementi salvati nel suo insieme
\end{itemize}

\textbf{Scenario principale\textsubscript{\scalebox{0.6}{\textbf{G}}}}
\begin{enumerate}
    \item L'utente si trova nel Modulo\textsubscript{\scalebox{0.6}{\textbf{G}}} storico dell'AI Co-Pilot\textsubscript{\scalebox{0.6}{\textbf{G}}}
    \item L'utente vede l'elenco di tutti gli elementi recuperati dall'analisi
\end{enumerate}


\textbf{Relazioni con altri Casi d'uso\textsubscript{\scalebox{0.6}{\textbf{G}}}}
\begin{itemize}
    \item \textit{include}: 
    \begin{itemize}
        \item UC-2B.2 – Visualizzazione elemento da lista documenti
    \end{itemize}
    \item \textit{extend}: 
    \begin{itemize}
        \item UC-2B.1 – Nessun documento riconosciuto
    \end{itemize}
\end{itemize}

\vspace{0.5cm}

\subUseCase{Nessun documento riconosciuto}

\textbf{Attori\textsubscript{\scalebox{0.6}{\textbf{G}}}}
\begin{itemize}
    \item Operatore CdL\textsubscript{\scalebox{0.6}{\textbf{G}}}
\end{itemize}

\textbf{Pre-condizioni\textsubscript{\scalebox{0.6}{\textbf{G}}}}
\begin{itemize}
    \item L'utente accede alla lista o effettua una ricerca/filtro
\end{itemize}

\textbf{Post-condizioni\textsubscript{\scalebox{0.6}{\textbf{G}}}}
\begin{itemize}
    \item Viene mostrato un messaggio che informa l'utente dell'assenza di documenti
\end{itemize}

\textbf{Scenario principale\textsubscript{\scalebox{0.6}{\textbf{G}}}}
\begin{enumerate}
    \item L'utente ha eseguito un'analisi di un file
    \item L'analisi non ha riconosciuto alcun documento
\end{enumerate}


\vspace{0.5cm}

\subUseCase{Visualizzazione elemento da lista documenti}

\begin{figure}[H]
    \centering
    \includegraphics[width=0.7\textwidth]{Diagrammi casi d'uso/UC2B-2.jpg}
    \caption{Diagramma del caso d'uso UC-2B.2 – Visualizzazione elemento da lista documenti}
\end{figure}

\textbf{Attori\textsubscript{\scalebox{0.6}{\textbf{G}}}}
\begin{itemize}
    \item Operatore CdL\textsubscript{\scalebox{0.6}{\textbf{G}}}
\end{itemize}

\textbf{Pre-condizioni\textsubscript{\scalebox{0.6}{\textbf{G}}}}
\begin{itemize}
    \item La lista dello storico contiene almeno un elemento
\end{itemize}

\textbf{Post-condizioni\textsubscript{\scalebox{0.6}{\textbf{G}}}}
\begin{itemize}
    \item L'utente visualizza i dettagli di una tupla
\end{itemize}

\textbf{Scenario principale\textsubscript{\scalebox{0.6}{\textbf{G}}}}
\begin{enumerate}
    \item L'utente visualizza un elemento dalla lista dello storico
\end{enumerate}


\textbf{Relazioni con altri Casi d'uso\textsubscript{\scalebox{0.6}{\textbf{G}}}}
\begin{itemize}
    \item \textit{include}: 
    \begin{itemize}
        \item UC-2B.3 – Visualizzazione competenza documento
        \item UC-2B.4 – Visualizzazione azienda documento
        \item UC-2B.5 – Visualizzazione causale
        \item UC-2B.6 – Visualizzazione lingua
        \item UC-2B.7 – Visualizzazione numero pagine documento
        \item UC-2B.8 – Visualizzazione nome documento originale
        \item UC-2B.9 – Visualizzazione data redazione documento
        \item UC-2B.10 – Visualizzazione codice documento
        \item UC-2B.11 – Visualizzazione tipologia documento
    \end{itemize}
\end{itemize}

\vspace{0.5cm}

\subUseCase{Visualizzazione competenza documento}

\textbf{Attori\textsubscript{\scalebox{0.6}{\textbf{G}}}}
\begin{itemize}
    \item Operatore CdL\textsubscript{\scalebox{0.6}{\textbf{G}}}
\end{itemize}

\textbf{Pre-condizioni\textsubscript{\scalebox{0.6}{\textbf{G}}}}
\begin{itemize}
    \item L'utente visualizza la lista dei documenti
\end{itemize}

\textbf{Post-condizioni\textsubscript{\scalebox{0.6}{\textbf{G}}}}
\begin{itemize}
    \item Viene visualizzato il periodo di competenza (Mese/Anno)
\end{itemize}

\textbf{Scenario principale\textsubscript{\scalebox{0.6}{\textbf{G}}}}
\begin{enumerate}
    \item Il sistema mostra il mese e l'anno di riferimento fiscale/contabile del documento
\end{enumerate}


\vspace{0.5cm}

\subUseCase{Visualizzazione azienda documento}

\textbf{Attori\textsubscript{\scalebox{0.6}{\textbf{G}}}}
\begin{itemize}
    \item Operatore CdL\textsubscript{\scalebox{0.6}{\textbf{G}}}
\end{itemize}

\textbf{Pre-condizioni\textsubscript{\scalebox{0.6}{\textbf{G}}}}
\begin{itemize}
    \item L'utente visualizza la lista dei documenti
\end{itemize}

\textbf{Post-condizioni\textsubscript{\scalebox{0.6}{\textbf{G}}}}
\begin{itemize}
    \item Viene visualizzata la ragione sociale dell'azienda a cui il documento appartiene
\end{itemize}

\textbf{Scenario principale\textsubscript{\scalebox{0.6}{\textbf{G}}}}
\begin{enumerate}
    \item Il sistema recupera il nome dell'azienda collegata e lo mostra in elenco
\end{enumerate}


\vspace{0.5cm}

\subUseCase{Visualizzazione causale}

\textbf{Attori\textsubscript{\scalebox{0.6}{\textbf{G}}}}
\begin{itemize}
    \item Operatore CdL\textsubscript{\scalebox{0.6}{\textbf{G}}}
\end{itemize}

\textbf{Pre-condizioni\textsubscript{\scalebox{0.6}{\textbf{G}}}}
\begin{itemize}
    \item L'utente visualizza la lista dei documenti
\end{itemize}

\textbf{Post-condizioni\textsubscript{\scalebox{0.6}{\textbf{G}}}}
\begin{itemize}
    \item Viene visualizzata la descrizione o causale specifica del documento
\end{itemize}

\textbf{Scenario principale\textsubscript{\scalebox{0.6}{\textbf{G}}}}
\begin{enumerate}
    \item Il sistema mostra eventuali note o causali inserite in fase di upload o rilevate automaticamente
\end{enumerate}


\vspace{0.5cm}

\subUseCase{Visualizzazione lingua}

\textbf{Attori\textsubscript{\scalebox{0.6}{\textbf{G}}}}
\begin{itemize}
    \item Operatore CdL\textsubscript{\scalebox{0.6}{\textbf{G}}}
\end{itemize}

\textbf{Pre-condizioni\textsubscript{\scalebox{0.6}{\textbf{G}}}}
\begin{itemize}
    \item L'utente visualizza la lista dei documenti
\end{itemize}

\textbf{Post-condizioni\textsubscript{\scalebox{0.6}{\textbf{G}}}}
\begin{itemize}
    \item Viene indicata la lingua principale del contenuto del documento
\end{itemize}

\textbf{Scenario principale\textsubscript{\scalebox{0.6}{\textbf{G}}}}
\begin{enumerate}
    \item Il sistema mostra un indicatore (es. bandiera o codice ISO) della lingua rilevata
\end{enumerate}

\vspace{0.5cm}

\subUseCase{Visualizzazione numero pagine documento}

\textbf{Attori\textsubscript{\scalebox{0.6}{\textbf{G}}}}
\begin{itemize}
    \item Operatore CdL\textsubscript{\scalebox{0.6}{\textbf{G}}}
\end{itemize}

\textbf{Pre-condizioni\textsubscript{\scalebox{0.6}{\textbf{G}}}}
\begin{itemize}
    \item L'utente visualizza la lista dei documenti
\end{itemize}

\textbf{Post-condizioni\textsubscript{\scalebox{0.6}{\textbf{G}}}}
\begin{itemize}
    \item Viene mostrato il conteggio totale delle pagine che compongono il file
\end{itemize}

\textbf{Scenario principale\textsubscript{\scalebox{0.6}{\textbf{G}}}}
\begin{enumerate}
    \item Il sistema calcola e visualizza il numero di pagine del documento PDF o immagine
\end{enumerate}


\vspace{0.5cm}

\subUseCase{Visualizzazione nome documento originale}

\textbf{Attori\textsubscript{\scalebox{0.6}{\textbf{G}}}}
\begin{itemize}
    \item Operatore CdL\textsubscript{\scalebox{0.6}{\textbf{G}}}
\end{itemize}

\textbf{Pre-condizioni\textsubscript{\scalebox{0.6}{\textbf{G}}}}
\begin{itemize}
    \item L'utente visualizza la lista dei documenti
\end{itemize}

\textbf{Post-condizioni\textsubscript{\scalebox{0.6}{\textbf{G}}}}
\begin{itemize}
    \item Viene visualizzato il nome del file originale caricato dall'utente
\end{itemize}

\textbf{Scenario principale\textsubscript{\scalebox{0.6}{\textbf{G}}}}
\begin{enumerate}
    \item Il sistema mostra il filename originale per facilitare il riconoscimento
\end{enumerate}


\vspace{0.5cm}

\subUseCase{Visualizzazione data redazione documento}

\textbf{Attori\textsubscript{\scalebox{0.6}{\textbf{G}}}}
\begin{itemize}
    \item Operatore CdL\textsubscript{\scalebox{0.6}{\textbf{G}}}
\end{itemize}

\textbf{Pre-condizioni\textsubscript{\scalebox{0.6}{\textbf{G}}}}
\begin{itemize}
    \item L'utente visualizza la lista dei documenti
\end{itemize}

\textbf{Post-condizioni\textsubscript{\scalebox{0.6}{\textbf{G}}}}
\begin{itemize}
    \item Viene visualizzata la data in cui il documento è stato creato o caricato a sistema
\end{itemize}

\textbf{Scenario principale\textsubscript{\scalebox{0.6}{\textbf{G}}}}
\begin{enumerate}
    \item Il sistema recupera il timestamp di creazione o upload e lo mostra formattato
\end{enumerate}


\vspace{0.5cm}

\subUseCase{Visualizzazione codice documento}

\textbf{Attori\textsubscript{\scalebox{0.6}{\textbf{G}}}}
\begin{itemize}
    \item Operatore CdL\textsubscript{\scalebox{0.6}{\textbf{G}}}
\end{itemize}

\textbf{Pre-condizioni\textsubscript{\scalebox{0.6}{\textbf{G}}}}
\begin{itemize}
    \item L'utente visualizza la lista dei documenti (incluso in UC-2B)
\end{itemize}

\textbf{Post-condizioni\textsubscript{\scalebox{0.6}{\textbf{G}}}}
\begin{itemize}
    \item Viene visualizzato l'identificativo univoco (ID) assegnato dal sistema al documento
\end{itemize}

\textbf{Scenario principale\textsubscript{\scalebox{0.6}{\textbf{G}}}}
\begin{enumerate}
    \item Il sistema recupera il codice identificativo dal database e lo mostra nella riga corrispondente al documento
\end{enumerate}


\vspace{0.5cm}

\subUseCase{Visualizzazione tipologia documento}

\textbf{Attori\textsubscript{\scalebox{0.6}{\textbf{G}}}}
\begin{itemize}
    \item Operatore CdL\textsubscript{\scalebox{0.6}{\textbf{G}}}
\end{itemize}

\textbf{Pre-condizioni\textsubscript{\scalebox{0.6}{\textbf{G}}}}
\begin{itemize}
    \item L'utente visualizza la lista dei documenti
\end{itemize}

\textbf{Post-condizioni\textsubscript{\scalebox{0.6}{\textbf{G}}}}
\begin{itemize}
    \item Viene visualizzata la categoria del documento (es. cedolino\textsubscript{\scalebox{0.6}{\textbf{G}}}, LUL, F24)
\end{itemize}

\textbf{Scenario principale\textsubscript{\scalebox{0.6}{\textbf{G}}}}
\begin{enumerate}
    \item Il sistema mostra l'etichetta della tipologia documentale associata al file
\end{enumerate}


\vspace{0.5cm}

\useCase{Visualizzazione anteprima documento}

\begin{figure}[H]
    \centering
    \includegraphics[width=0.7\textwidth]{Diagrammi casi d'uso/UC2C.jpg}
    \caption{Diagramma del caso d'uso UC-2C – Visualizzazione anteprima documento}
\end{figure}

\textbf{Attori\textsubscript{\scalebox{0.6}{\textbf{G}}}}
\begin{itemize}
    \item Operatore CdL\textsubscript{\scalebox{0.6}{\textbf{G}}}
\end{itemize}

\textbf{Pre-condizioni\textsubscript{\scalebox{0.6}{\textbf{G}}}}
\begin{itemize}
    \item L'utente ha effettuato l'accesso al dettaglio del documento
    \item Il file digitale del documento è accessibile
    \item Il documento è stato analizzato e i dati sono stati estratti
\end{itemize}

\textbf{Post-condizioni\textsubscript{\scalebox{0.6}{\textbf{G}}}}
\begin{itemize}
    \item L'anteprima del documento viene mostrata correttamente permettendo il confronto con i dati estratti
\end{itemize}

\textbf{Scenario principale\textsubscript{\scalebox{0.6}{\textbf{G}}}}
\begin{enumerate}
    \item Il sistema recupera il file sorgente (PDF o immagine).
    \item Il sistema renderizza l'anteprima del documento nell'area dedicata della pagina di dettaglio.
\end{enumerate}


\vspace{0.5cm}

\useCase{Modifica destinatario documento}

\begin{figure}[H]
    \centering
    \includegraphics[width=0.7\textwidth]{Diagrammi casi d'uso/UC2D.jpg}
    \caption{Diagramma del caso d'uso UC-2D – Modifica destinatario documento}
\end{figure}

\textbf{Attori\textsubscript{\scalebox{0.6}{\textbf{G}}}}
\begin{itemize}
    \item Operatore CdL\textsubscript{\scalebox{0.6}{\textbf{G}}}
\end{itemize}

\textbf{Pre-condizioni\textsubscript{\scalebox{0.6}{\textbf{G}}}}
\begin{itemize}
    \item Il documento è stato analizzato e i dati sono stati estratti
    \item L'utente ha effettuato l'accesso al dettaglio del documento
\end{itemize}

\textbf{Post-condizioni\textsubscript{\scalebox{0.6}{\textbf{G}}}}
\begin{itemize}
    \item Il documento è associato a un nuovo destinatario
\end{itemize}

\textbf{Scenario principale\textsubscript{\scalebox{0.6}{\textbf{G}}}}
\begin{enumerate}
    \item L'utente nota che il destinatario assegnato automaticamente o precedentemente è errato o mancante.
    \item L'utente attiva il campo di modifica del destinatario.
    \item L'utente ricerca e seleziona il destinatario corretto dall'anagrafica.
    \item Il sistema aggiorna il campo visualizzato con il nuovo valore.
\end{enumerate}


\vspace{0.5cm}

\useCase{Modifica tipologia documento}

\begin{figure}[H]
    \centering
    \includegraphics[width=0.7\textwidth]{Diagrammi casi d'uso/UC2E.jpg}
    \caption{Diagramma del caso d'uso UC-2E – Modifica tipologia documento}
\end{figure}

\textbf{Attori\textsubscript{\scalebox{0.6}{\textbf{G}}}}
\begin{itemize}
    \item Operatore CdL\textsubscript{\scalebox{0.6}{\textbf{G}}}
\end{itemize}

\textbf{Pre-condizioni\textsubscript{\scalebox{0.6}{\textbf{G}}}}
\begin{itemize}
    \item Il documento è stato analizzato e i dati sono stati estratti
    \item L'utente ha effettuato l'accesso al dettaglio del documento
\end{itemize}

\textbf{Post-condizioni\textsubscript{\scalebox{0.6}{\textbf{G}}}}
\begin{itemize}
    \item La tipologia del documento è stata modificata.
\end{itemize}

\textbf{Scenario principale\textsubscript{\scalebox{0.6}{\textbf{G}}}}
\begin{enumerate}
    \item L'utente rileva che la classificazione automatica del documento è errata (es. classificato come "Fattura" invece di "Contratto").
    \item L'utente seleziona la nuova tipologia da un menu a tendina.
    \item Il sistema aggiorna la tipologia.
\end{enumerate}


\vspace{0.5cm}

\useCase{Rivaluta percentuale confidenza}

\begin{figure}[H]
    \centering
    \includegraphics[width=0.7\textwidth]{Diagrammi casi d'uso/UC2F.jpg}
    \caption{Diagramma del caso d'uso UC-2F – Rivaluta percentuale confidenza}
\end{figure}

\textbf{Attori\textsubscript{\scalebox{0.6}{\textbf{G}}}}
\begin{itemize}
    \item Operatore CdL\textsubscript{\scalebox{0.6}{\textbf{G}}}
\end{itemize}

\textbf{Pre-condizioni\textsubscript{\scalebox{0.6}{\textbf{G}}}}
\begin{itemize}
    \item L'utente ha effettuato l'accesso al dettaglio del documento
    \item L'utente ha apportato modifiche al documento
\end{itemize}

\textbf{Post-condizioni\textsubscript{\scalebox{0.6}{\textbf{G}}}}
\begin{itemize}
    \item La percentuale di confidenza viene ricalcolata (validazione\textsubscript{\scalebox{0.6}{\textbf{G}}} manuale).
\end{itemize}

\textbf{Scenario principale\textsubscript{\scalebox{0.6}{\textbf{G}}}}
\begin{enumerate}
    \item L'utente ha modificato dati del documento
    \item Il sistema ri-analizza i campi in base alla nuova struttura dati prevista.
    \item Il sistema aggiorna la percentuale di confidenza per riflettere la nuova coerenza dei dati o la validazione\textsubscript{\scalebox{0.6}{\textbf{G}}} manuale dell'utente.
\end{enumerate}

\vspace{0.5cm}

\useCase{Visualizzazione lista informazioni destinatari analisi}
\begin{figure}[H]
    \centering
    \includegraphics[width=0.9\textwidth]{Diagrammi casi d'uso/UC2G.jpg}
    \caption{Diagramma del caso d'uso UC-2G – Visualizzazione lista informazioni destinatari}
\end{figure}

\textbf{Attori\textsubscript{\scalebox{0.6}{\textbf{G}}}}
\begin{itemize}
    \item Operatore CdL\textsubscript{\scalebox{0.6}{\textbf{G}}}
\end{itemize}

\textbf{Pre-condizioni\textsubscript{\scalebox{0.6}{\textbf{G}}}}
\begin{itemize}
    \item La lista dei destinatari contiene almeno un elemento
\end{itemize}

\textbf{Post-condizioni\textsubscript{\scalebox{0.6}{\textbf{G}}}}
\begin{itemize}
    \item L'utente visualizza la lista di elementi salvati nel suo insieme
\end{itemize}

\textbf{Scenario principale\textsubscript{\scalebox{0.6}{\textbf{G}}}}
\begin{enumerate}
    \item L'utente si trova nel Modulo\textsubscript{\scalebox{0.6}{\textbf{G}}} storico dell'AI Co-Pilot\textsubscript{\scalebox{0.6}{\textbf{G}}}
    \item L'utente vede l'elenco di tutti gli elementi relativi ai destinatari recuperati dall'analisi
\end{enumerate}

\textbf{Relazioni con altri Casi d'uso\textsubscript{\scalebox{0.6}{\textbf{G}}}}
\begin{itemize}
    \item \textit{include}: 
    \begin{itemize}
        \item UC-2G.1 – Visualizzazione elemento da lista informazioni destinatari
    \end{itemize}
    \item \textit{extend}: 
    \begin{itemize}
        \item UC-2G.6 – Nessun destinatario riconosciuto
    \end{itemize}
\end{itemize}

\vspace{0.5cm}

\subUseCase{Visualizzazione elemento da lista informazioni destinatari}

\begin{figure}[H]
    \centering
    \includegraphics[width=0.9\textwidth]{Diagrammi casi d'uso/UC2G-1.jpg}
    \caption{Diagramma del caso d'uso UC-2G.1 – Visualizzazione elemento da lista informazioni destinatari}
\end{figure}

\textbf{Attori\textsubscript{\scalebox{0.6}{\textbf{G}}}}
\begin{itemize}
    \item Operatore CdL\textsubscript{\scalebox{0.6}{\textbf{G}}}
\end{itemize}


\textbf{Pre-condizioni\textsubscript{\scalebox{0.6}{\textbf{G}}}}
\begin{itemize}
    \item La lista dei destinatari contiene almeno un elemento
\end{itemize}


\textbf{Post-condizioni\textsubscript{\scalebox{0.6}{\textbf{G}}}}
\begin{itemize}
    \item L'utente può visualizzare le informazioni riguardanti un elemento dalla lista.
\end{itemize}

\textbf{Scenario principale\textsubscript{\scalebox{0.6}{\textbf{G}}}}
\begin{enumerate}
    \item L'utente dopo aver caricato uno o più documenti accede alla sezione "Lista documenti".
    \item Il sistema mostra l'elenco dei documenti caricati, con le relative informazioni (tipologia, competenza, azienda, causale, lingua, numero pagine, nome originale).
\end{enumerate}

\textbf{Scenario secondario\textsubscript{\scalebox{0.6}{\textbf{G}}}}
\begin{enumerate}
    \item Il sistema non riconosce alcun documento caricato.
\end{enumerate}

\textbf{Relazioni con altri Casi d'uso\textsubscript{\scalebox{0.6}{\textbf{G}}}}
\begin{itemize}
    \item \textit{include}: 
    \begin{itemize}
        \item UC-2B.10 – Visualizzazione codice documento
        \item UC-2G.2 - Visualizzazione codice fiscale
        \item UC-2G.3 - Visualizzazione matricola
        \item UC-2G.4 - Visualizzazione reparto
        \item UC-2G.5 – Visualizzazione destinatario
    \end{itemize}
    \item \textit{extend}: 
    \begin{itemize}
        \item Nessuno
    \end{itemize}
\end{itemize}

\subUseCase{Visualizzazione codice fiscale}

\textbf{Attori\textsubscript{\scalebox{0.6}{\textbf{G}}}}
\begin{itemize}
    \item Operatore CdL\textsubscript{\scalebox{0.6}{\textbf{G}}}
\end{itemize}

\textbf{Pre-condizioni\textsubscript{\scalebox{0.6}{\textbf{G}}}}
\begin{itemize}
    \item La lista dei destinatari contiene almeno un elemento
\end{itemize}

\textbf{Post-condizioni\textsubscript{\scalebox{0.6}{\textbf{G}}}}
\begin{itemize}
    \item Il codice fiscale del destinatario è visibile per la verifica\textsubscript{\scalebox{0.6}{\textbf{G}}} dell'identità.
\end{itemize}

\textbf{Scenario principale\textsubscript{\scalebox{0.6}{\textbf{G}}}}
\begin{enumerate}
    \item Il sistema recupera il codice fiscale dal record anagrafico del destinatario.
    \item Il sistema formatta e visualizza il codice fiscale nel campo dedicato.
\end{enumerate}

\vspace{0.5cm}

\subUseCase{Visualizzazione matricola}

\textbf{Attori\textsubscript{\scalebox{0.6}{\textbf{G}}}}
\begin{itemize}
    \item Operatore CdL\textsubscript{\scalebox{0.6}{\textbf{G}}}
\end{itemize}

\textbf{Pre-condizioni\textsubscript{\scalebox{0.6}{\textbf{G}}}}
\begin{itemize}
    \item La lista dei destinatari contiene almeno un elemento
\end{itemize}

\textbf{Post-condizioni\textsubscript{\scalebox{0.6}{\textbf{G}}}}
\begin{itemize}
    \item Il numero di matricola è visualizzato correttamente.
\end{itemize}

\textbf{Scenario principale\textsubscript{\scalebox{0.6}{\textbf{G}}}}
\begin{enumerate}
    \item Il sistema verifica\textsubscript{\scalebox{0.6}{\textbf{G}}} se al destinatario è associato un numero di matricola aziendale.
    \item Il sistema mostra il numero di matricola per permettere la correlazione con i sistemi paghe.
\end{enumerate}

\textbf{Scenario secondario\textsubscript{\scalebox{0.6}{\textbf{G}}}}
\begin{enumerate}
    \item Il destinatario non possiede una matricola (es. collaboratore esterno).
    \item Il campo viene mostrato vuoto o con dicitura "N/D".
\end{enumerate}

\vspace{0.5cm}

\subUseCase{Visualizzazione reparto}

\textbf{Attori\textsubscript{\scalebox{0.6}{\textbf{G}}}}
\begin{itemize}
    \item Operatore CdL\textsubscript{\scalebox{0.6}{\textbf{G}}}
\end{itemize}

\textbf{Pre-condizioni\textsubscript{\scalebox{0.6}{\textbf{G}}}}
\begin{itemize}
    \item La lista dei destinatari contiene almeno un elemento
\end{itemize}

\textbf{Post-condizioni\textsubscript{\scalebox{0.6}{\textbf{G}}}}
\begin{itemize}
    \item L'utente visualizza il reparto 
\end{itemize}

\textbf{Scenario principale\textsubscript{\scalebox{0.6}{\textbf{G}}}}
\begin{enumerate}
    \item Il sistema recupera le informazioni sull'organigramma o l'assegnazione del destinatario.
    \item Il sistema visualizza il nome del reparto di afferenza.
\end{enumerate}

\subUseCase{Visualizzazione destinatario}

\textbf{Attori\textsubscript{\scalebox{0.6}{\textbf{G}}}}
\begin{itemize}
    \item Operatore CdL\textsubscript{\scalebox{0.6}{\textbf{G}}}
\end{itemize}

\textbf{Pre-condizioni\textsubscript{\scalebox{0.6}{\textbf{G}}}}
\begin{itemize}
    \item La lista dei destinatari contiene almeno un elemento
\end{itemize}

\textbf{Post-condizioni\textsubscript{\scalebox{0.6}{\textbf{G}}}}
\begin{itemize}
    \item Il nome e cognome del destinatario sono visibili a schermo.
\end{itemize}

\textbf{Scenario principale\textsubscript{\scalebox{0.6}{\textbf{G}}}}
\begin{enumerate}
    \item Il sistema recupera l'identificativo del destinatario associato al documento selezionato.
    \item Il sistema interroga l'anagrafica per ottenere i dati anagrafici principali.
    \item Il sistema mostra il nome del destinatario nell'intestazione della scheda.
\end{enumerate}

\vspace{0.5cm}

\subUseCase{Nessun destinatario riconosciuto}

\textbf{Attori\textsubscript{\scalebox{0.6}{\textbf{G}}}}
\begin{itemize}
    \item Operatore CdL\textsubscript{\scalebox{0.6}{\textbf{G}}}
\end{itemize}

\textbf{Pre-condizioni\textsubscript{\scalebox{0.6}{\textbf{G}}}}
\begin{itemize}
    \item L'utente accede alla lista o effettua una ricerca/filtro
\end{itemize}

\textbf{Post-condizioni\textsubscript{\scalebox{0.6}{\textbf{G}}}}
\begin{itemize}
    \item Viene mostrato un messaggio che informa l'utente dell'assenza di destinatari
\end{itemize}

\textbf{Scenario principale\textsubscript{\scalebox{0.6}{\textbf{G}}}}
\begin{enumerate}
    \item L'utente ha eseguito un'analisi di un file
    \item L'analisi non ha riconosciuto alcun destinatario
\end{enumerate}

\vspace{0.5cm}


\useCase{Visualizzazione lista storico documenti}
\begin{figure}[H]
    \centering
    \includegraphics[width=0.9\textwidth]{Diagrammi casi d'uso/UC2H.jpg}
    \caption{Diagramma del caso d'uso UC-2H – Visualizzazione lista storico documenti}
\end{figure}

\textbf{Attori\textsubscript{\scalebox{0.6}{\textbf{G}}}}
\begin{itemize}
    \item Operatore CdL\textsubscript{\scalebox{0.6}{\textbf{G}}}
\end{itemize}

\textbf{Pre-condizioni\textsubscript{\scalebox{0.6}{\textbf{G}}}}
\begin{itemize}
    \item La lista dei contiene almeno un elemento
    \item L'utente si trova nella pagina dello storico documenti (UC-2E)
\end{itemize}

\textbf{Post-condizioni\textsubscript{\scalebox{0.6}{\textbf{G}}}}
\begin{itemize}
    \item L'utente visualizza la lista di documenti nello storico salvati nel suo insieme
\end{itemize}

\textbf{Scenario principale\textsubscript{\scalebox{0.6}{\textbf{G}}}}
\begin{enumerate}
    \item L'utente si trova nel Modulo\textsubscript{\scalebox{0.6}{\textbf{G}}} storico dell'AI Co-Pilot\textsubscript{\scalebox{0.6}{\textbf{G}}}
    \item L'utente vede l'elenco di tutti gli elementi relativi a tutte le analisi
\end{enumerate}

\textbf{Relazioni con altri Casi d'uso\textsubscript{\scalebox{0.6}{\textbf{G}}}}
\begin{itemize}
    \item \textit{include}: 
    \begin{itemize}
        \item UC-2H.2 – Visualizzazione elemento da lista storico documenti
    \end{itemize}
    \item \textit{extend}: 
    \begin{itemize}
        \item UC-2H.1 – Documenti assenti
    \end{itemize}
\end{itemize}

\vspace{0.5cm}

\subUseCase{Documenti assenti}

\textbf{Attori\textsubscript{\scalebox{0.6}{\textbf{G}}}}
\begin{itemize}
    \item Operatore CdL\textsubscript{\scalebox{0.6}{\textbf{G}}}
\end{itemize}

\textbf{Pre-condizioni\textsubscript{\scalebox{0.6}{\textbf{G}}}}
\begin{itemize}
    \item Non esistono documenti salvati nello storico
\end{itemize}

\textbf{Post-condizioni\textsubscript{\scalebox{0.6}{\textbf{G}}}}
\begin{itemize}
    \item Viene mostrato un messaggio informativo che indica l'assenza di risultati.
    \item La lista appare vuota.
\end{itemize}

\textbf{Scenario principale\textsubscript{\scalebox{0.6}{\textbf{G}}}}
\begin{enumerate}
    \item Il sistema interroga il database per recuperare i documenti.
    \item La ricerca non produce risultati (0 record).
    \item Il sistema visualizza un placeholder o un messaggio "Nessun documento trovato" per informare l'utente.
\end{enumerate}

\vspace{0.5cm}

\subUseCase{Visualizzazione elemento da lista storico documenti}


\begin{figure}[H]
    \centering
    \includegraphics[width=0.9\textwidth]{Diagrammi casi d'uso/UC2H-2.jpg}
    \caption{Diagramma del caso d'uso UC-2H.2 – Visualizzazione elemento da lista storico documenti}
\end{figure}


\textbf{Attori\textsubscript{\scalebox{0.6}{\textbf{G}}}}
\begin{itemize}
    \item Operatore CdL\textsubscript{\scalebox{0.6}{\textbf{G}}}
\end{itemize}

\textbf{Pre-condizioni\textsubscript{\scalebox{0.6}{\textbf{G}}}}
\begin{itemize}
    \item La lista contiene almeno un elemento
\end{itemize}

\textbf{Post-condizioni\textsubscript{\scalebox{0.6}{\textbf{G}}}}
\begin{itemize}
    \item Per ogni riga della lista vengono mostrati i dati essenziali del documento.
\end{itemize}

\textbf{Scenario principale\textsubscript{\scalebox{0.6}{\textbf{G}}}}
\begin{enumerate}
    \item Il sistema recupera i metadati principali per ogni documento in elenco.
    \item Il sistema renderizza le colonne della tabella mostrando: Codice documento, Data caricamento, Stato elaborazione e Percentuale di confidenza.
\end{enumerate}

\textbf{Relazioni con altri Casi d'uso\textsubscript{\scalebox{0.6}{\textbf{G}}}}
\begin{itemize}
    \item \textit{include}: 
    \begin{itemize}
        \item UC-2B.10 – Visualizzazione codice documento
        \item UC-2H.3 - Visualizzazione percentuale confidenza
        \item UC-2H.4 - Visualizzazione appartenenza lista distribuzione
        \item UC-2H.5 - Visualizzazione stato
    \end{itemize}
    \item \textit{extend}: 
    \begin{itemize}
        \item Nessuna
    \end{itemize}
\end{itemize}

\vspace{0.5cm}

\subUseCase{Visualizzazione percentuale confidenza}


\textbf{Attori\textsubscript{\scalebox{0.6}{\textbf{G}}}}
\begin{itemize}
    \item Operatore CdL\textsubscript{\scalebox{0.6}{\textbf{G}}}
\end{itemize}

\textbf{Pre-condizioni\textsubscript{\scalebox{0.6}{\textbf{G}}}}
\begin{itemize}
    \item La lista contiene almeno un elemento
    \item È stata calcolata la percentuale di confidenza media del documento
\end{itemize}

\textbf{Post-condizioni\textsubscript{\scalebox{0.6}{\textbf{G}}}}
\begin{itemize}
    \item Viene mostrato un indicatore numerico o visivo dell'affidabilità dei dati estratti.
\end{itemize}

\textbf{Scenario principale\textsubscript{\scalebox{0.6}{\textbf{G}}}}
\begin{enumerate}
    \item Il sistema recupera il punteggio di confidenza calcolato durante l'importazione.
    \item Il sistema formatta il dato come percentuale.
    \item Il sistema visualizza il dato, evidenziandolo (es. in rosso) se la confidenza è sotto una soglia di sicurezza, suggerendo un controllo manuale.
\end{enumerate}

\vspace{0.5cm}

\subUseCase{Visualizzazione appartenenza lista distribuzione}


\textbf{Attori\textsubscript{\scalebox{0.6}{\textbf{G}}}}
\begin{itemize}
    \item Operatore CdL\textsubscript{\scalebox{0.6}{\textbf{G}}}
\end{itemize}

\textbf{Pre-condizioni\textsubscript{\scalebox{0.6}{\textbf{G}}}}
\begin{itemize}
    \item La lista contiene almeno un elemento
\end{itemize}

\textbf{Post-condizioni\textsubscript{\scalebox{0.6}{\textbf{G}}}}
\begin{itemize}
    \item L'utente visualizza chi riceverà o ha ricevuto il documento.
\end{itemize}

\textbf{Scenario principale\textsubscript{\scalebox{0.6}{\textbf{G}}}}
\begin{enumerate}
    \item Il sistema recupera i collegamenti tra il documento e le anagrafiche destinatari.
    \item Il sistema mostra i nomi dei destinatari o, in caso di lista lunga, un riepilogo (es. "Mario Rossi + 2 altri") espandibile al passaggio del mouse o al click.
\end{enumerate}

\vspace{0.5cm}

\subUseCase{Visualizzazione stato}


\textbf{Attori\textsubscript{\scalebox{0.6}{\textbf{G}}}}
\begin{itemize}
    \item Operatore CdL\textsubscript{\scalebox{0.6}{\textbf{G}}}
\end{itemize}

\textbf{Pre-condizioni\textsubscript{\scalebox{0.6}{\textbf{G}}}}
\begin{itemize}
    \item La lista contiene almeno un elemento
\end{itemize}

\textbf{Post-condizioni\textsubscript{\scalebox{0.6}{\textbf{G}}}}
\begin{itemize}
    \item L'utente conosce lo stato attuale del ciclo di vita del documento (es. "Da validare", "Pronto per l'invio", "Inviato", "Errore").
\end{itemize}

\textbf{Scenario principale\textsubscript{\scalebox{0.6}{\textbf{G}}}}
\begin{enumerate}
    \item Il sistema interroga il workflow\textsubscript{\scalebox{0.6}{\textbf{G}}} del documento per determinarne lo stato corrente.
    \item Il sistema visualizza un'etichetta testuale o un'icona colorata rappresentativa dello stato (es. Verde per inviato, Giallo per in attesa).
\end{enumerate}

\vspace{0.5cm}

\useCase{Carica template messaggio}

\begin{figure}[H]
    \centering
    \includegraphics[width=0.7\textwidth]{Diagrammi casi d'uso/UC2I.jpg}
    \caption{Diagramma del caso d'uso UC-2I – Carica template messaggio}
\end{figure}

\textbf{Attori\textsubscript{\scalebox{0.6}{\textbf{G}}}}
\begin{itemize}
    \item Operatore CdL\textsubscript{\scalebox{0.6}{\textbf{G}}}
\end{itemize}

\textbf{Pre-condizioni\textsubscript{\scalebox{0.6}{\textbf{G}}}}
\begin{itemize}
    \item L'utente ha selezionato un template dalla lista dei template disponibili
\end{itemize}

\textbf{Post-condizioni\textsubscript{\scalebox{0.6}{\textbf{G}}}}
\begin{itemize}
    \item I dati del template (Oggetto e Corpo del messaggio) vengono trasferiti nei campi di input della pagina di gestione messaggio
\end{itemize}

\textbf{Scenario principale\textsubscript{\scalebox{0.6}{\textbf{G}}}}
\begin{enumerate}
    \item L'utente clicca sul pulsante "Carica" o sulla riga del template desiderato.
    \item Il sistema recupera il contenuto del template (testo e oggetto).
    \item Il sistema sovrascrive o compila i campi corrispondenti nell'interfaccia\textsubscript{\scalebox{0.6}{\textbf{G}}} di composizione del messaggio.
    \item L'utente può ora procedere a modificare o inviare il messaggio
\end{enumerate}


\useCase{Modifica oggetto messaggio}

\begin{figure}[H]
    \centering
    \includegraphics[width=0.7\textwidth]{Diagrammi casi d'uso/UC2J.jpg}
    \caption{Diagramma del caso d'uso UC-2J – Modifica oggetto messaggio}
\end{figure}

\textbf{Attori\textsubscript{\scalebox{0.6}{\textbf{G}}}}
\begin{itemize}
    \item Operatore CdL\textsubscript{\scalebox{0.6}{\textbf{G}}}
\end{itemize}

\textbf{Pre-condizioni\textsubscript{\scalebox{0.6}{\textbf{G}}}}
\begin{itemize}
    \item L'utente si trova nella pagina di gestione messaggio (UC-2F).
\end{itemize}

\textbf{Post-condizioni\textsubscript{\scalebox{0.6}{\textbf{G}}}}
\begin{itemize}
    \item L'oggetto della mail/messaggio è aggiornato con il testo inserito dall'utente.
\end{itemize}

\textbf{Scenario principale\textsubscript{\scalebox{0.6}{\textbf{G}}}}
\begin{enumerate}
    \item L'utente seleziona il campo di input relativo all'oggetto del messaggio.
    \item L'utente digita o modifica il testo esistente (es. aggiungendo riferimenti specifici).
\end{enumerate}

\vspace{0.5cm}

\useCase{Modifica testo messaggio}

\begin{figure}[H]
    \centering
    \includegraphics[width=0.7\textwidth]{Diagrammi casi d'uso/UC2K.jpg}
    \caption{Diagramma del caso d'uso UC-2K – Modifica testo messaggio}
\end{figure}

\textbf{Attori\textsubscript{\scalebox{0.6}{\textbf{G}}}}
\begin{itemize}
    \item Operatore CdL\textsubscript{\scalebox{0.6}{\textbf{G}}}
\end{itemize}

\textbf{Pre-condizioni\textsubscript{\scalebox{0.6}{\textbf{G}}}}
\begin{itemize}
    \item L'utente si trova nella pagina di gestione messaggio (UC-2F).
\end{itemize}

\textbf{Post-condizioni\textsubscript{\scalebox{0.6}{\textbf{G}}}}
\begin{itemize}
    \item Il corpo del messaggio è aggiornato.
\end{itemize}

\textbf{Scenario principale\textsubscript{\scalebox{0.6}{\textbf{G}}}}
\begin{enumerate}
    \item L'utente seleziona l'area di testo contenente il corpo del messaggio.
    \item L'utente apporta modifiche manuali, aggiunge note o corregge il testo generato/preimpostato.
\end{enumerate}

\vspace{0.5cm}

\useCase{Salva nuovo template messaggio}

\begin{figure}[H]
    \centering
    \includegraphics[width=0.7\textwidth]{Diagrammi casi d'uso/UC2L.jpg}
    \caption{Diagramma del caso d'uso UC-2L – Salva nuovo template messaggio}
\end{figure}

\textbf{Attori\textsubscript{\scalebox{0.6}{\textbf{G}}}}
\begin{itemize}
    \item Operatore CdL\textsubscript{\scalebox{0.6}{\textbf{G}}}
\end{itemize}

\textbf{Pre-condizioni\textsubscript{\scalebox{0.6}{\textbf{G}}}}
\begin{itemize}
    \item L'utente ha modificato un messaggio l'oggetto e/o il corpo del messaggio.
    \item Il campo testo del messaggio non è vuoto.
\end{itemize}

\textbf{Post-condizioni\textsubscript{\scalebox{0.6}{\textbf{G}}}}
\begin{itemize}
    \item Il contenuto del messaggio viene salvato nel database dei template personali o globali.
\end{itemize}

\textbf{Scenario principale\textsubscript{\scalebox{0.6}{\textbf{G}}}}
\begin{enumerate}
    \item L'utente, soddisfatto del messaggio attuale, clicca sul pulsante "Salva come template".
    \item Il sistema richiede di assegnare un nome al nuovo template.
    \item L'utente inserisce il nome e conferma.
    \item Il sistema salva il template rendendolo disponibile per futuri utilizzi (vedi UC-2G).
\end{enumerate}

\vspace{0.5cm}


\useCase{Elimina template messaggio}

\begin{figure}[H]
    \centering
    \includegraphics[width=0.7\textwidth]{Diagrammi casi d'uso/UC2M.jpg}
    \caption{Diagramma del caso d'uso UC-2M – Elimina template messaggio}
\end{figure}

\textbf{Attori\textsubscript{\scalebox{0.6}{\textbf{G}}}}
\begin{itemize}
    \item Operatore CdL\textsubscript{\scalebox{0.6}{\textbf{G}}}
\end{itemize}

\textbf{Pre-condizioni\textsubscript{\scalebox{0.6}{\textbf{G}}}}
\begin{itemize}
    \item L'utente visualizza la lista dei template salvati (UC-2G).
    \item Il template da eliminare è presente in lista.
\end{itemize}

\textbf{Post-condizioni\textsubscript{\scalebox{0.6}{\textbf{G}}}}
\begin{itemize}
    \item Il template viene rimosso definitivamente dal sistema.
    \item La lista viene aggiornata e non mostra più il template eliminato.
\end{itemize}

\textbf{Scenario principale\textsubscript{\scalebox{0.6}{\textbf{G}}}}
\begin{enumerate}
    \item L'utente identifica un template obsoleto o errato nella tabella.
    \item L'utente clicca sull'icona di eliminazione (es. cestino) in corrispondenza della riga.
    \item Il sistema richiede conferma dell'operazione.
    \item L'utente conferma e il sistema cancella il record dal database.
\end{enumerate}


\vspace{0.5cm}



\useCase{Visualizzazione lista template}

\begin{figure}[H]
    \centering
    \includegraphics[width=0.8\textwidth]{Diagrammi casi d'uso/UC2N.jpg}
    \caption{Diagramma del caso d'uso UC-2N – Visualizzazione lista template}
\end{figure}

\textbf{Attori\textsubscript{\scalebox{0.6}{\textbf{G}}}}
\begin{itemize}
    \item Operatore CdL\textsubscript{\scalebox{0.6}{\textbf{G}}}
\end{itemize}

\textbf{Pre-condizioni\textsubscript{\scalebox{0.6}{\textbf{G}}}}
\begin{itemize}
    \item La lista contiene almeno un template salvato
\end{itemize}

\textbf{Post-condizioni\textsubscript{\scalebox{0.6}{\textbf{G}}}}
\begin{itemize}
    \item L'utente visualizza la lista di template salvati nel suo insieme
\end{itemize}

\textbf{Scenario principale\textsubscript{\scalebox{0.6}{\textbf{G}}}}
\begin{enumerate}
    \item L'utente si trova nella sezione di gestione messaggi e vuole utilizzare un template salvato
    \item L'utente vede l'elenco di tutti i template relativi ai messaggi salvati
\end{enumerate}

\textbf{Relazioni con altri Casi d'uso\textsubscript{\scalebox{0.6}{\textbf{G}}}}
\begin{itemize}
    \item \textit{include}: 
    \begin{itemize}
        \item UC-2N.1 – Visualizzazione elemento da lista template
    \end{itemize}
\end{itemize}

\vspace{0.5cm}



\subUseCase{Visualizzazione elemento da lista template}
\textbf{Attori\textsubscript{\scalebox{0.6}{\textbf{G}}}}
\begin{itemize}
    \item Operatore CdL\textsubscript{\scalebox{0.6}{\textbf{G}}}
\end{itemize}


\textbf{Pre-condizioni\textsubscript{\scalebox{0.6}{\textbf{G}}}}
\begin{itemize}
    \item è stato salvato almeno un template di messaggio.
\end{itemize}


\textbf{Post-condizioni\textsubscript{\scalebox{0.6}{\textbf{G}}}}
\begin{itemize}
    \item L'utente può visualizzare le informazioni riguardanti un elemento dalla lista.
\end{itemize}

\textbf{Scenario principale\textsubscript{\scalebox{0.6}{\textbf{G}}}}
\begin{enumerate}
    \item L'utente vuole visualizzare i dettagli di un template salvato.
\end{enumerate}

\textbf{Scenario secondario\textsubscript{\scalebox{0.6}{\textbf{G}}}}
\begin{enumerate}
    \item nessuno
\end{enumerate}

\textbf{Relazioni con altri Casi d'uso\textsubscript{\scalebox{0.6}{\textbf{G}}}}
\begin{itemize}
    \item \textit{include}: 
    \begin{itemize}
        \item UC-2N.2 – Visualizzazione oggetto template
        \item UC-2N.3 – Visualizzazione testo template
        \item UC-2N.4 – Visualizzazione codice template
    \end{itemize}
    \item \textit{extend}: 
    \begin{itemize}
        \item Nessuno
    \end{itemize}
\end{itemize}


\subUseCase{Visualizzazione oggetto template}

\textbf{Attori\textsubscript{\scalebox{0.6}{\textbf{G}}}}
\begin{itemize}
    \item Operatore CdL\textsubscript{\scalebox{0.6}{\textbf{G}}}
\end{itemize}

\textbf{Pre-condizioni\textsubscript{\scalebox{0.6}{\textbf{G}}}}
\begin{itemize}
    \item Si sta visualizzando la tabella dei template messaggio
\end{itemize}

\textbf{Post-condizioni\textsubscript{\scalebox{0.6}{\textbf{G}}}}
\begin{itemize}
    \item Recupero dell'oggetto del messaggio nel template
\end{itemize}

\textbf{Scenario principale\textsubscript{\scalebox{0.6}{\textbf{G}}}}
\begin{enumerate}
    \item Il sistema recupera l'oggetto del template.
    \item Il sistema mostra il contenuto nel campo dedicato.
\end{enumerate}

\vspace{0.5cm}

\subUseCase{Visualizzazione testo messaggio}

\textbf{Attori\textsubscript{\scalebox{0.6}{\textbf{G}}}}
\begin{itemize}
    \item Operatore CdL\textsubscript{\scalebox{0.6}{\textbf{G}}}
\end{itemize}

\textbf{Pre-condizioni\textsubscript{\scalebox{0.6}{\textbf{G}}}}
\begin{itemize}
    \item Si sta visualizzando la tabella dei template messaggio
\end{itemize}

\textbf{Post-condizioni\textsubscript{\scalebox{0.6}{\textbf{G}}}}
\begin{itemize}
    \item Recupero del testo del messaggio nel template
\end{itemize}

\textbf{Scenario principale\textsubscript{\scalebox{0.6}{\textbf{G}}}}
\begin{enumerate}
    \item Il sistema recupera il testo del template.
    \item Il sistema mostra il contenuto nel campo dedicato.
\end{enumerate}

\vspace{0.5cm}

\subUseCase{Visualizzazione codice template}

\textbf{Attori\textsubscript{\scalebox{0.6}{\textbf{G}}}}
\begin{itemize}
    \item Operatore CdL\textsubscript{\scalebox{0.6}{\textbf{G}}}
\end{itemize}

\textbf{Pre-condizioni\textsubscript{\scalebox{0.6}{\textbf{G}}}}
\begin{itemize}
    \item Si sta visualizzando la tabella dei template messaggio
\end{itemize}

\textbf{Post-condizioni\textsubscript{\scalebox{0.6}{\textbf{G}}}}
\begin{itemize}
    \item Recupero del codice del template
\end{itemize}

\textbf{Scenario principale\textsubscript{\scalebox{0.6}{\textbf{G}}}}
\begin{enumerate}
    \item Il sistema recupera il codice del template.
    \item Il sistema mostra il contenuto nel campo dedicato.
\end{enumerate}

\vspace{0.5cm}


\useCase{Invio documento e messaggio}

\begin{figure}[H]
    \centering
    \includegraphics[width=1\textwidth]{Diagrammi casi d'uso/UC2O.jpg}
    \caption{Diagramma del caso d'uso UC-2O – Invio documento e messaggio}
\end{figure}


\textbf{Attori\textsubscript{\scalebox{0.6}{\textbf{G}}}}
\begin{itemize}
    \item Operatore CdL\textsubscript{\scalebox{0.6}{\textbf{G}}}
\end{itemize}

\textbf{Pre-condizioni\textsubscript{\scalebox{0.6}{\textbf{G}}}}
\begin{itemize}
    \item L'utente si trova all'interno del Modulo\textsubscript{\scalebox{0.6}{\textbf{G}}} di invio di documenti e messaggio
    \item Il messaggio è stato completato con oggetto, testo e documenti allegati.
\end{itemize}

\textbf{Post-condizioni\textsubscript{\scalebox{0.6}{\textbf{G}}}}
\begin{itemize}
    \item Il sistema avvia il processo di spedizione delle email/messaggi.
    \item Il messaggio e i documenti sono stati inviati con successo ai destinatari previsti.
    \item L'utente riceve un feedback di successo.
\end{itemize}

\textbf{Scenario principale\textsubscript{\scalebox{0.6}{\textbf{G}}}}
\begin{enumerate}
    \item L'utente ha completato la preparazione del messaggio e dei documenti da inviare.
    \item L'utente seleziona l'opzione per inviare il messaggio e i documenti ai destinatari.
\end{enumerate}

\textbf{Scenario secondario\textsubscript{\scalebox{0.6}{\textbf{G}}}}
\begin{enumerate}
    \item 
\end{enumerate}

\textbf{Relazioni con altri Casi d'uso\textsubscript{\scalebox{0.6}{\textbf{G}}}}
\begin{itemize}
    \item \textit{include}: 
    \begin{itemize}
        \item UC-2O.1 – Allega file
        \item UC-2O.2 – Pianifica invio
    \end{itemize}
\end{itemize}

\subUseCase{Allega file}

\textbf{Attori\textsubscript{\scalebox{0.6}{\textbf{G}}}}
\begin{itemize}
    \item Operatore CdL\textsubscript{\scalebox{0.6}{\textbf{G}}}
\end{itemize}

\textbf{Pre-condizioni\textsubscript{\scalebox{0.6}{\textbf{G}}}}
\begin{itemize}
    \item L'utente si trova nella pagina di riepilogo invio (UC-2H).
\end{itemize}

\textbf{Post-condizioni\textsubscript{\scalebox{0.6}{\textbf{G}}}}
\begin{itemize}
    \item Il nuovo file viene aggiunto alla lista degli allegati pronti per l'invio.
\end{itemize}

\textbf{Scenario principale\textsubscript{\scalebox{0.6}{\textbf{G}}}}
\begin{enumerate}
    \item L'utente clicca sul pulsante "Aggiungi allegato".
    \item Il sistema apre la finestra di selezione file del sistema operativo.
    \item L'utente seleziona il file desiderato.
    \item Il sistema carica il file e lo mostra nella lista allegati (UC-2H.5).
\end{enumerate}

\vspace{0.5cm}

\subUseCase{Pianifica invio}

\textbf{Attori\textsubscript{\scalebox{0.6}{\textbf{G}}}}
\begin{itemize}
    \item Operatore CdL\textsubscript{\scalebox{0.6}{\textbf{G}}}
\end{itemize}

\textbf{Pre-condizioni\textsubscript{\scalebox{0.6}{\textbf{G}}}}
\begin{itemize}
    \item L'utente si trova nella pagina di riepilogo invio (UC-2H)
\end{itemize}

\textbf{Post-condizioni\textsubscript{\scalebox{0.6}{\textbf{G}}}}
\begin{itemize}
    \item L'invio viene schedulato per la data e ora selezionate.
    \item Lo stato dei documenti passa a "Pianificato".
\end{itemize}

\textbf{Scenario principale\textsubscript{\scalebox{0.6}{\textbf{G}}}}
\begin{enumerate}
    \item L'utente seleziona l'opzione "Pianifica invio".
    \item Il sistema mostra un selettore di data e ora.
    \item L'utente imposta il momento desiderato per l'invio e conferma.
    \item Il sistema prende in carico la richiesta e la accoda per l'elaborazione futura.
\end{enumerate}

\vspace{0.5cm}

\useCase{Filtraggio documenti analisi}

\begin{figure}[H]
    \centering
    \includegraphics[width=0.8\textwidth]{Diagrammi casi d'uso/UC2P.jpg}
    \caption{Diagramma del caso d'uso UC-2P – Filtraggio documenti analisi}
\end{figure}

\textbf{Attori\textsubscript{\scalebox{0.6}{\textbf{G}}}}
\begin{itemize}
    \item Operatore CdL\textsubscript{\scalebox{0.6}{\textbf{G}}}
\end{itemize}


\textbf{Pre-condizioni\textsubscript{\scalebox{0.6}{\textbf{G}}}}
\begin{itemize}
    \item è stato caricato almeno un documento tramite UC-2A.6.
    \item L'utente è entrato nel Modulo\textsubscript{\scalebox{0.6}{\textbf{G}}} AI Co-Pilot\textsubscript{\scalebox{0.6}{\textbf{G}}} per i CdL dalla Dashboard\textsubscript{\scalebox{0.6}{\textbf{G}}} principale.
    \item L'utente dispone dei permessi necessari per utilizzare il Modulo\textsubscript{\scalebox{0.6}{\textbf{G}}} AI Co-Pilot\textsubscript{\scalebox{0.6}{\textbf{G}}}
\end{itemize}


\textbf{Post-condizioni\textsubscript{\scalebox{0.6}{\textbf{G}}}}
\begin{itemize}
    \item L'utente può visualizzare le informazioni riguardanti gli elementi concordanti con il filtro
\end{itemize}

\textbf{Scenario principale\textsubscript{\scalebox{0.6}{\textbf{G}}}}
\begin{enumerate}
    \item L'utente dopo aver caricato uno o più documenti accede alla sezione "Lista documenti".
    \item Il sistema mostra l'elenco dei documenti caricati, con le relative informazioni (tipologia, competenza, azienda, causale, lingua, numero pagine, nome originale).
\end{enumerate}

\textbf{Scenario secondario\textsubscript{\scalebox{0.6}{\textbf{G}}}}
\begin{enumerate}
    \item Il sistema non riconosce alcun documento caricato.
\end{enumerate}


\useCase{Visualizzazione lista documenti filtrata}

% \begin{figure}[H]
%     \centering
%     \includegraphics[width=0.8\textwidth]{Diagrammi casi d'uso/UC2Q.jpg}
%     \caption{Diagramma del caso d'uso UC-2Q – Visualizzazione lista documenti filtrata}
% \end{figure}

\textbf{Attori\textsubscript{\scalebox{0.6}{\textbf{G}}}}
\begin{itemize}
    \item Operatore CdL\textsubscript{\scalebox{0.6}{\textbf{G}}}
\end{itemize}

\textbf{Pre-condizioni\textsubscript{\scalebox{0.6}{\textbf{G}}}}
\begin{itemize}
    \item La lista dello storico contiene almeno un elemento 
    \item L'utente ha inserito un criterio di ricerca valido (testo libero)
\end{itemize}

\textbf{Post-condizioni\textsubscript{\scalebox{0.6}{\textbf{G}}}}
\begin{itemize}
    \item L'utente visualizza la lista di elementi (coerenti con la ricerca) trovati dall'analisi
\end{itemize}

\textbf{Scenario principale\textsubscript{\scalebox{0.6}{\textbf{G}}}}
\begin{enumerate}
    \item L'utente si trova nel Modulo\textsubscript{\scalebox{0.6}{\textbf{G}}} storico dell'AI Assistant\textsubscript{\scalebox{0.6}{\textbf{G}}} Generativo
    \item L'utente vede l'elenco di tutti gli elementi salvati con i relativi dati principali (prompt\textsubscript{\scalebox{0.6}{\textbf{G}}} parziale, tono, stile, data)
\end{enumerate}


\textbf{Relazioni con altri Casi d'uso\textsubscript{\scalebox{0.6}{\textbf{G}}}}
\begin{itemize}
    % \item \textit{include}: 
    % \begin{itemize}
    %     \item UC-2B.2 – Visualizzazione elemento da lista documenti
    % \end{itemize}
    \item \textit{generalizzazione}: 
    \begin{itemize}
        \item UC-2B – Visualizzazione lista documenti analisi
    \end{itemize}
\end{itemize}

\vspace{0.5cm}

\useCase{Filtraggio destinatari analisi}

\begin{figure}[H]
    \centering
    \includegraphics[width=0.8\textwidth]{Diagrammi casi d'uso/UC2R.jpg}
    \caption{Diagramma del caso d'uso UC-2R – Filtraggio destinatari analisi}
\end{figure}

\textbf{Attori\textsubscript{\scalebox{0.6}{\textbf{G}}}}
\begin{itemize}
    \item Operatore CdL\textsubscript{\scalebox{0.6}{\textbf{G}}}
\end{itemize}


\textbf{Pre-condizioni\textsubscript{\scalebox{0.6}{\textbf{G}}}}
\begin{itemize}
    \item Si è nella pagina lista documenti (UC-2B)
    \item La lista dei destinatari contiene almeno un elemento
\end{itemize}


\textbf{Post-condizioni\textsubscript{\scalebox{0.6}{\textbf{G}}}}
\begin{itemize}
    \item L'utente ha filtrato la lista dei destinatari in base ai criteri selezionati.
\end{itemize}

\textbf{Scenario principale\textsubscript{\scalebox{0.6}{\textbf{G}}}}
\begin{enumerate}
    \item L'utente dopo aver caricato uno o più documenti accede alla sezione "Lista documenti".
    \item Il sistema mostra l'elenco dei documenti caricati, con le relative informazioni.
    \item L'utente applica filtri specifici per visualizzare solo i destinatari di interesse (es. per nome, codice fiscale, reparto).
\end{enumerate}

\textbf{Scenario secondario\textsubscript{\scalebox{0.6}{\textbf{G}}}}
\begin{enumerate}
    \item Il sistema non riconosce alcun destinatario.
\end{enumerate}


\useCase{Visualizzazione lista destinatari filtrata}

% \begin{figure}[H]
%     \centering
%     \includegraphics[width=0.8\textwidth]{Diagrammi casi d'uso/UC2S.jpg}
%     \caption{Diagramma del caso d'uso UC-2S – Visualizzazione lista destinatari filtrata}
% \end{figure}

\textbf{Attori\textsubscript{\scalebox{0.6}{\textbf{G}}}}
\begin{itemize}
    \item Operatore CdL\textsubscript{\scalebox{0.6}{\textbf{G}}}
\end{itemize}

\textbf{Pre-condizioni\textsubscript{\scalebox{0.6}{\textbf{G}}}}
\begin{itemize}
    \item La lista dei destinatari contiene almeno un elemento 
    \item L'utente ha inserito un criterio di ricerca valido (testo libero)
\end{itemize}

\textbf{Post-condizioni\textsubscript{\scalebox{0.6}{\textbf{G}}}}
\begin{itemize}
    \item L'utente visualizza la lista di elementi (coerenti con la ricerca) trovati dall'analisi
\end{itemize}

\textbf{Scenario principale\textsubscript{\scalebox{0.6}{\textbf{G}}}}
\begin{enumerate}
    \item L'utente si trova nel Modulo\textsubscript{\scalebox{0.6}{\textbf{G}}} storico dell'AI Co-Pilot\textsubscript{\scalebox{0.6}{\textbf{G}}} Generativo
    \item L'utente vede l'elenco di tutti gli elementi salvati con i relativi dati principali 
\end{enumerate}

\textbf{Relazioni con altri Casi d'uso\textsubscript{\scalebox{0.6}{\textbf{G}}}}
\begin{itemize}
    % \item \textit{include}: 
    % \begin{itemize}
    %     \item UC-2G.1 – Visualizzazione elemento da lista destinatari
    % \end{itemize}
    \item \textit{generalizzazione}: 
    \begin{itemize}
        \item UC-2G – Visualizzazione lista informazioni destinatari analisi
    \end{itemize}
\end{itemize}

\vspace{0.5cm}

\useCase{Filtraggio documenti storico}

\begin{figure}[H]
    \centering
    \includegraphics[width=0.8\textwidth]{Diagrammi casi d'uso/UC2T.jpg}
    \caption{Diagramma del caso d'uso UC-2T – Filtraggio documenti storico}
\end{figure}

\textbf{Attori\textsubscript{\scalebox{0.6}{\textbf{G}}}}
\begin{itemize}
    \item Operatore CdL\textsubscript{\scalebox{0.6}{\textbf{G}}}
\end{itemize}

\textbf{Pre-condizioni\textsubscript{\scalebox{0.6}{\textbf{G}}}}
\begin{itemize}
    \item La lista dei documenti è visualizzata e popolata.
\end{itemize}

\textbf{Post-condizioni\textsubscript{\scalebox{0.6}{\textbf{G}}}}
\begin{itemize}
    \item La ricerca è stata eseguita in base ai criteri selezionati.
\end{itemize}

\textbf{Scenario principale\textsubscript{\scalebox{0.6}{\textbf{G}}}}
\begin{enumerate}
    \item L'utente desidera restringere la ricerca (es. per data, per tipologia o per stato invio).
    \item L'utente utilizza la barra di ricerca o i filtri laterali.
    \item Il sistema aggiorna la vista in tempo reale o dopo la conferma, nascondendo i documenti non pertinenti.
\end{enumerate}

\textbf{Scenario secondario\textsubscript{\scalebox{0.6}{\textbf{G}}}}
\begin{enumerate}
    \item L'utente rimuove i filtri.
    \item Il sistema ripristina la visualizzazione completa dello storico.
\end{enumerate}


\vspace{0.5cm}

\useCase{Visualizzazione lista storico documenti filtrata}

% \begin{figure}[H]
%     \centering
%     \includegraphics[width=0.8\textwidth]{Diagrammi casi d'uso/UC2U.jpg}
%     \caption{Diagramma del caso d'uso UC-2U – Visualizzazione lista storico documenti filtrata}
% \end{figure}

\textbf{Attori\textsubscript{\scalebox{0.6}{\textbf{G}}}}
\begin{itemize}
    \item Operatore CdL\textsubscript{\scalebox{0.6}{\textbf{G}}}
\end{itemize}

\textbf{Pre-condizioni\textsubscript{\scalebox{0.6}{\textbf{G}}}}
\begin{itemize}
    \item La lista dei contiene almeno un elemento
\end{itemize}

\textbf{Post-condizioni\textsubscript{\scalebox{0.6}{\textbf{G}}}}
\begin{itemize}
    \item L'utente visualizza la lista filtrata di documenti nello storico salvati nel suo insieme
\end{itemize}

\textbf{Scenario principale\textsubscript{\scalebox{0.6}{\textbf{G}}}}
\begin{enumerate}
    \item L'utente si trova nel Modulo\textsubscript{\scalebox{0.6}{\textbf{G}}} storico dell'AI Co-Pilot\textsubscript{\scalebox{0.6}{\textbf{G}}}
    \item L'utente vede l'elenco filtrato di tutti gli elementi relativi a tutte le analisi
\end{enumerate}

\textbf{Relazioni con altri Casi d'uso\textsubscript{\scalebox{0.6}{\textbf{G}}}}
\begin{itemize}
    % \item \textit{include}: 
    % \begin{itemize}
    %     \item UC-2H.2 – Visualizzazione elemento da lista storico documenti
    % \end{itemize}
    \item \textit{generalizzazione}: 
    \begin{itemize}
        \item UC-2H – Visualizzazione lista storico documenti
    \end{itemize}
\end{itemize}

\vspace{0.5cm}


\subsection{Sezione 3 - Modulo\textsubscript{\scalebox{0.6}{\textbf{G}}} Analytics e Monitoraggio Trasversale}

\renewcommand{\UCPrefix}{3}


\useCase{Visualizzazione elenco dati AI Assistant\textsubscript{\scalebox{0.6}{\textbf{G}}}}

\begin{figure}[H]
    \centering
    \includegraphics[width=1\textwidth]{Diagrammi casi d'uso/UC3A.jpg}
    \caption{Diagramma del caso d'uso UC-3A – Visualizzazione elenco dati AI Assistant\textsubscript{\scalebox{0.6}{\textbf{G}}}}
\end{figure}


\textbf{Attori\textsubscript{\scalebox{0.6}{\textbf{G}}}}
\begin{itemize}
    \item Data Analyst\textsubscript{\scalebox{0.6}{\textbf{G}}}
\end{itemize}

\textbf{Pre-condizioni\textsubscript{\scalebox{0.6}{\textbf{G}}}}
\begin{itemize}
    \item Il sistema ha accumulato dati storici.
    \item L'utente accede alla sezione dell'elenco dati AI Assistant\textsubscript{\scalebox{0.6}{\textbf{G}}}
\end{itemize}

\textbf{Post-condizioni\textsubscript{\scalebox{0.6}{\textbf{G}}}}
\begin{itemize}
    \item L'utente visualizza le metriche in base ai filtri applicati (o di default).
\end{itemize}

\textbf{Scenario principale\textsubscript{\scalebox{0.6}{\textbf{G}}}}
\begin{enumerate}
    \item L'utente accede al Modulo\textsubscript{\scalebox{0.6}{\textbf{G}}} centralizzato di Analytics.
    \item Il sistema calcola e mostra una panoramica generale (Dashboard\textsubscript{\scalebox{0.6}{\textbf{G}}}) applicando i filtri di data.
    \item Il sistema visualizza i grafici e le tabelle relative alle metriche (confidenza, rating, volumi) per il contesto di default.
\end{enumerate}

\textbf{Scenario secondario\textsubscript{\scalebox{0.6}{\textbf{G}}}}
\begin{enumerate}
    \item Il sistema non dispone di dati sufficienti nel periodo di default.
    \item Il sistema mostra un messaggio ``Nessun dato disponibile per il periodo selezionato''.
\end{enumerate}

\textbf{Relazioni con altri Casi d'uso\textsubscript{\scalebox{0.6}{\textbf{G}}}}
\begin{itemize}
    \item \textit{include}: 
    \begin{itemize}
        \item UC-3A.2 - Visualizzazione numero prompt\textsubscript{\scalebox{0.6}{\textbf{G}}} generati
        \item UC-3A.3 - Visualizzazione rating medio prompt\textsubscript{\scalebox{0.6}{\textbf{G}}} generati
        \item UC-3A.4 - Visualizzazione numero rigenerazioni prompt\textsubscript{\scalebox{0.6}{\textbf{G}}}
        \item UC-3A.5 - Visualizzazione toni più usati
        \item UC-3A.6 - Visualizzazione stili più usati
    \end{itemize}
    \item \textit{extend}: 
    \begin{itemize}
        \item Nessuna
    \end{itemize}
\end{itemize} 

\vspace{0.5cm}

\subUseCase{Visualizzazione numero prompt\textsubscript{\scalebox{0.6}{\textbf{G}}} generati}

\textbf{Attori\textsubscript{\scalebox{0.6}{\textbf{G}}}}
\begin{itemize}
    \item Data Analyst\textsubscript{\scalebox{0.6}{\textbf{G}}}
\end{itemize}

\textbf{Pre-condizioni\textsubscript{\scalebox{0.6}{\textbf{G}}}}
\begin{itemize}
    \item L'utente accede alla sezione dell'elenco dati AI Assistant\textsubscript{\scalebox{0.6}{\textbf{G}}}
\end{itemize}

\textbf{Post-condizioni\textsubscript{\scalebox{0.6}{\textbf{G}}}}
\begin{itemize}
    \item L'utente visualizza il volume totale di richieste (prompt\textsubscript{\scalebox{0.6}{\textbf{G}}}) inviate all'AI nel periodo selezionato.
\end{itemize}

\textbf{Scenario principale\textsubscript{\scalebox{0.6}{\textbf{G}}}}
\begin{enumerate}
    \item Il sistema interroga il database storico filtrando per l'intervallo temporale corrente.
    \item Il sistema conteggia il numero totale di interazioni avviate dagli utenti.
    \item Il sistema visualizza il dato numerico (KPI) in un widget dedicato della Dashboard\textsubscript{\scalebox{0.6}{\textbf{G}}} (es. "Totale prompt\textsubscript{\scalebox{0.6}{\textbf{G}}}: 1.250").
\end{enumerate}

\vspace{0.5cm}

\subUseCase{Visualizzazione rating medio prompt\textsubscript{\scalebox{0.6}{\textbf{G}}} generati}

\textbf{Attori\textsubscript{\scalebox{0.6}{\textbf{G}}}}
\begin{itemize}
    \item Data Analyst\textsubscript{\scalebox{0.6}{\textbf{G}}}
\end{itemize}

\textbf{Pre-condizioni\textsubscript{\scalebox{0.6}{\textbf{G}}}}
\begin{itemize}
    \item L'utente accede alla sezione dell'elenco dati AI Assistant\textsubscript{\scalebox{0.6}{\textbf{G}}}
    \item Gli utenti finali hanno fornito feedback (es. stelle o pollici su/giù) sui messaggi generati.
\end{itemize}

\textbf{Post-condizioni\textsubscript{\scalebox{0.6}{\textbf{G}}}}
\begin{itemize}
    \item L'utente visualizza un indicatore della qualità\textsubscript{\scalebox{0.6}{\textbf{G}}} percepita delle risposte dell'AI.
\end{itemize}

\textbf{Scenario principale\textsubscript{\scalebox{0.6}{\textbf{G}}}}
\begin{enumerate}
    \item Il sistema aggrega i feedback ricevuti sulle risposte generate nel periodo di riferimento.
    \item Il sistema calcola la media aritmetica dei voti (es. su scala 1-5 o percentuale di approvazione\textsubscript{\scalebox{0.6}{\textbf{G}}}).
    \item Il sistema mostra il valore medio (es. "Rating Medio: 4.2/5") permettendo di valutare la soddisfazione dell'utenza.
\end{enumerate}

\vspace{0.5cm}

\subUseCase{Visualizzazione numero rigenerazioni prompt\textsubscript{\scalebox{0.6}{\textbf{G}}}}

\textbf{Attori\textsubscript{\scalebox{0.6}{\textbf{G}}}}
\begin{itemize}
    \item Data Analyst\textsubscript{\scalebox{0.6}{\textbf{G}}}
\end{itemize}

\textbf{Pre-condizioni\textsubscript{\scalebox{0.6}{\textbf{G}}}}
\begin{itemize}
    \item L'utente accede alla sezione dell'elenco dati AI Assistant\textsubscript{\scalebox{0.6}{\textbf{G}}}
\end{itemize}

\textbf{Post-condizioni\textsubscript{\scalebox{0.6}{\textbf{G}}}}
\begin{itemize}
    \item L'utente visualizza quante volte è stata utilizzata la funzione "Rigenera risposta".
\end{itemize}

\textbf{Scenario principale\textsubscript{\scalebox{0.6}{\textbf{G}}}}
\begin{enumerate}
    \item Il sistema conta le occorrenze in cui un utente ha richiesto una nuova versione di un messaggio già generato (evento di "retry").
    \item Il sistema visualizza il totale delle rigenerazioni, dato utile per capire se la prima risposta dell'AI è spesso insoddisfacente.
\end{enumerate}

\vspace{0.5cm}

\subUseCase{Visualizzazione toni più usati}

\textbf{Attori\textsubscript{\scalebox{0.6}{\textbf{G}}}}
\begin{itemize}
    \item Data Analyst\textsubscript{\scalebox{0.6}{\textbf{G}}}
\end{itemize}

\textbf{Pre-condizioni\textsubscript{\scalebox{0.6}{\textbf{G}}}}
\begin{itemize}
    \item L'utente accede alla sezione dell'elenco dati AI Assistant\textsubscript{\scalebox{0.6}{\textbf{G}}}
\end{itemize}

\textbf{Post-condizioni\textsubscript{\scalebox{0.6}{\textbf{G}}}}
\begin{itemize}
    \item L'utente visualizza la distribuzione o la classifica dei "Toni" selezionati per la generazione dei messaggi.
\end{itemize}

\textbf{Scenario principale\textsubscript{\scalebox{0.6}{\textbf{G}}}}
\begin{enumerate}
    \item Il sistema raggruppa i prompt\textsubscript{\scalebox{0.6}{\textbf{G}}} generati in base all'attributo "Tono" (es. Formale, Empatico, Assertivo).
    \item Il sistema calcola la frequenza di utilizzo per ciascuna categoria.
    \item Il sistema visualizza un grafico (es. a torta o a barre) o una lista ordinata che evidenzia i toni prediletti dagli utenti.
\end{enumerate}

\vspace{0.5cm}

\subUseCase{Visualizzazione stili più usati}

\textbf{Attori\textsubscript{\scalebox{0.6}{\textbf{G}}}}
\begin{itemize}
    \item Data Analyst\textsubscript{\scalebox{0.6}{\textbf{G}}}
\end{itemize}

\textbf{Pre-condizioni\textsubscript{\scalebox{0.6}{\textbf{G}}}}
\begin{itemize}
    \item L'utente accede alla sezione dell'elenco dati AI Assistant\textsubscript{\scalebox{0.6}{\textbf{G}}}
\end{itemize}

\textbf{Post-condizioni\textsubscript{\scalebox{0.6}{\textbf{G}}}}
\begin{itemize}
    \item L'utente visualizza la distribuzione o la classifica degli "Stili" selezionati per la generazione dei messaggi.
\end{itemize}

\textbf{Scenario principale\textsubscript{\scalebox{0.6}{\textbf{G}}}}
\begin{enumerate}
    \item Il sistema raggruppa i prompt\textsubscript{\scalebox{0.6}{\textbf{G}}} generati in base all'attributo "Stile" (es. Sintetico, Dettagliato, Elenco puntato).
    \item Il sistema calcola la frequenza di utilizzo per ciascuna categoria.
    \item Il sistema visualizza un grafico o una classifica che mostra quali formati di risposta sono più richiesti.
\end{enumerate}

\useCase{Visualizzazione elenco dati AI Co-Pilot\textsubscript{\scalebox{0.6}{\textbf{G}}}}

\begin{figure}[H]
    \centering
    \includegraphics[width=1\textwidth]{Diagrammi casi d'uso/UC3B.jpg}
    \caption{Diagramma del caso d'uso UC-3B – Visualizzazione elenco dati AI Co-Pilot\textsubscript{\scalebox{0.6}{\textbf{G}}}}
\end{figure}

\textbf{Attori\textsubscript{\scalebox{0.6}{\textbf{G}}}}
\begin{itemize}
    \item Data Analyst\textsubscript{\scalebox{0.6}{\textbf{G}}}
\end{itemize}

\textbf{Pre-condizioni\textsubscript{\scalebox{0.6}{\textbf{G}}}}
\begin{itemize}
    \item L'utente ha effettuato il login ed è autenticato come Auditor, Data Analyst\textsubscript{\scalebox{0.6}{\textbf{G}}} o Amministratore\textsubscript{\scalebox{0.6}{\textbf{G}}}.
    \item L'utente accede alla sezione AI Co-Pilot\textsubscript{\scalebox{0.6}{\textbf{G}}}
\end{itemize}

\textbf{Post-condizioni\textsubscript{\scalebox{0.6}{\textbf{G}}}}
\begin{itemize}
    \item L'utente visualizza le metriche aggiornate in base ai filtri applicati (o di default).
\end{itemize}

\textbf{Scenario principale\textsubscript{\scalebox{0.6}{\textbf{G}}}}
\begin{enumerate}
    \item L'utente accede al Modulo\textsubscript{\scalebox{0.6}{\textbf{G}}} centralizzato di Analytics.
    \item Il sistema calcola e mostra una panoramica generale (Dashboard\textsubscript{\scalebox{0.6}{\textbf{G}}}) applicando i filtri di default (es. Periodo: Ultimo Mese, Utenti: Tutti).
    \item Il sistema visualizza i grafici e le tabelle relative alle metriche (confidenza, rating, volumi) per il contesto di default.
\end{enumerate}

\textbf{Scenario secondario\textsubscript{\scalebox{0.6}{\textbf{G}}}}
\begin{enumerate}
    \item Il sistema non dispone di dati sufficienti nel periodo di default.
    \item Il sistema mostra un messaggio ``Nessun dato disponibile per il periodo selezionato''.
\end{enumerate}

\textbf{Relazioni con altri Casi d'uso\textsubscript{\scalebox{0.6}{\textbf{G}}}}
\begin{itemize}
    \item \textit{include}: 
    \begin{itemize}
        \item UC-3B.1 - Visualizzazione confidenza media
        \item UC-3B.2 - Visualizzazione percentuale interventi manuali
        \item UC-3B.3 - Visualizzazione accuratezza mapping
        \item UC-3B.4 - Visualizzazione tempi medi analisi
    \end{itemize}
    \item \textit{extend}: 
    \begin{itemize}
        \item Nessuna
    \end{itemize}
\end{itemize} 

\vspace{0.5cm}

\subUseCase{Visualizzazione confidenza media}

\textbf{Attori\textsubscript{\scalebox{0.6}{\textbf{G}}}}
\begin{itemize}
    \item Data Analyst\textsubscript{\scalebox{0.6}{\textbf{G}}}
\end{itemize}

\textbf{Pre-condizioni\textsubscript{\scalebox{0.6}{\textbf{G}}}}
\begin{itemize}
    \item L'utente ha fatto l'accesso alla Dashboard\textsubscript{\scalebox{0.6}{\textbf{G}}} di Analytics AI Co-Pilot\textsubscript{\scalebox{0.6}{\textbf{G}}}
\end{itemize}

\textbf{Post-condizioni\textsubscript{\scalebox{0.6}{\textbf{G}}}}
\begin{itemize}
    \item L'utente visualizza il punteggio medio di affidabilità (confidence score) che l'AI ha attribuito alle sue estrazioni nel periodo selezionato.
\end{itemize}

\textbf{Scenario principale\textsubscript{\scalebox{0.6}{\textbf{G}}}}
\begin{enumerate}
    \item Il sistema aggrega i punteggi di confidenza di tutti i documenti elaborati nell'intervallo temporale.
    \item Il sistema calcola la media pesata o aritmetica.
    \item Il sistema mostra il valore (es. 85\%) indicando se è in linea, superiore o inferiore rispetto ai trend storici.
\end{enumerate}

\vspace{0.5cm}

\subUseCase{Visualizzazione percentuale interventi manuali}

\textbf{Attori\textsubscript{\scalebox{0.6}{\textbf{G}}}}
\begin{itemize}
    \item Data Analyst\textsubscript{\scalebox{0.6}{\textbf{G}}}
\end{itemize}

\textbf{Pre-condizioni\textsubscript{\scalebox{0.6}{\textbf{G}}}}
\begin{itemize}
    \item L'utente ha fatto l'accesso alla Dashboard\textsubscript{\scalebox{0.6}{\textbf{G}}} di Analytics AI Co-Pilot\textsubscript{\scalebox{0.6}{\textbf{G}}}
\end{itemize}

\textbf{Post-condizioni\textsubscript{\scalebox{0.6}{\textbf{G}}}}
\begin{itemize}
    \item L'utente visualizza la frequenza con cui gli operatori umani devono correggere i dati estratti dall'AI.
\end{itemize}

\textbf{Scenario principale\textsubscript{\scalebox{0.6}{\textbf{G}}}}
\begin{enumerate}
    \item Il sistema analizza i log di audit per contare quanti documenti hanno subito modifiche manuali ai campi estratti prima del salvataggio.
    \item Il sistema confronta questo dato con il volume totale dei documenti.
    \item Il sistema visualizza la percentuale di "Human-in-the-loop" (es. 12\% di documenti corretti manualmente).
\end{enumerate}

\vspace{0.5cm}

\subUseCase{Visualizzazione accuratezza mapping}

\textbf{Attori\textsubscript{\scalebox{0.6}{\textbf{G}}}}
\begin{itemize}
    \item Data Analyst\textsubscript{\scalebox{0.6}{\textbf{G}}}
\end{itemize}

\textbf{Pre-condizioni\textsubscript{\scalebox{0.6}{\textbf{G}}}}
\begin{itemize}
    \item L'utente ha fatto l'accesso alla Dashboard\textsubscript{\scalebox{0.6}{\textbf{G}}} di Analytics AI Co-Pilot\textsubscript{\scalebox{0.6}{\textbf{G}}}
\end{itemize}

\textbf{Post-condizioni\textsubscript{\scalebox{0.6}{\textbf{G}}}}
\begin{itemize}
    \item L'utente visualizza un indicatore che rappresenta la precisione con cui l'AI associa i dati trovati ai campi corretti del database.
\end{itemize}

\textbf{Scenario principale\textsubscript{\scalebox{0.6}{\textbf{G}}}}
\begin{enumerate}
    \item Il sistema valuta la coerenza dei dati (es. verifica\textsubscript{\scalebox{0.6}{\textbf{G}}} se i campi mappati sono stati successivamente spostati o rimossi dall'operatore).
    \item Il sistema calcola un indice di accuratezza strutturale.
    \item Il sistema visualizza il dato, permettendo di identificare se l'AI sta fallendo nel riconoscere specifici layout o tipologie di documento.
\end{enumerate}

\vspace{0.5cm}

\subUseCase{Visualizzazione tempi medi analisi}

\textbf{Attori\textsubscript{\scalebox{0.6}{\textbf{G}}}}
\begin{itemize}
    \item Data Analyst\textsubscript{\scalebox{0.6}{\textbf{G}}}
\end{itemize}

\textbf{Pre-condizioni\textsubscript{\scalebox{0.6}{\textbf{G}}}}
\begin{itemize}
    \item L'utente ha fatto l'accesso alla Dashboard\textsubscript{\scalebox{0.6}{\textbf{G}}} di Analytics AI Co-Pilot\textsubscript{\scalebox{0.6}{\textbf{G}}}
\end{itemize}

\textbf{Post-condizioni\textsubscript{\scalebox{0.6}{\textbf{G}}}}
\begin{itemize}
    \item L'utente visualizza il tempo medio impiegato dal sistema per processare un documento.
\end{itemize}

\textbf{Scenario principale\textsubscript{\scalebox{0.6}{\textbf{G}}}}
\begin{enumerate}
    \item Il sistema recupera i timestamp di inizio (upload) e fine (disponibilità dati) elaborazione per i documenti del periodo.
    \item Il sistema calcola la durata media dell'elaborazione.
    \item Il sistema mostra il tempo medio (es. "1.5 secondi per pagina"), utile per monitorare le performance dell'infrastruttura e la latenza del servizio.
\end{enumerate}

\useCase{Filtraggio periodo temporale}

\begin{figure}[H]
    \centering
    \includegraphics[width=0.7\textwidth]{Diagrammi casi d'uso/UC3C.jpg}
    \caption{Diagramma del caso d'uso UC-3C – Filtraggio periodo temporale\textsubscript{\scalebox{0.6}{\textbf{G}}}}
\end{figure}

\textbf{Attori\textsubscript{\scalebox{0.6}{\textbf{G}}}}
\begin{itemize}
    \item Data Analyst\textsubscript{\scalebox{0.6}{\textbf{G}}}
\end{itemize}

\textbf{Pre-condizioni\textsubscript{\scalebox{0.6}{\textbf{G}}}}
\begin{itemize}
    \item L'utente ha fatto l'accesso alla Dashboard\textsubscript{\scalebox{0.6}{\textbf{G}}} di Analytics
    \item Il filtro temporale è attivo e modificabile.
\end{itemize}

\textbf{Post-condizioni\textsubscript{\scalebox{0.6}{\textbf{G}}}}
\begin{itemize}
    \item La data di inizio del periodo di analisi è impostata.
    \item Il sistema è pronto ad aggiornare i grafici in base al nuovo intervallo (o li aggiorna automaticamente).
\end{itemize}

\textbf{Scenario principale\textsubscript{\scalebox{0.6}{\textbf{G}}}}
\begin{enumerate}
    \item L'utente interagisce con il selettore di data "Dal" (Data inizio) e "Al" (Data fine).
    \item Il sistema mostra un calendario interattivo.
    \item L'utente seleziona un giorno specifico.
    \item Il sistema verifica\textsubscript{\scalebox{0.6}{\textbf{G}}} che la data selezionata sia antecedente o uguale alla data di fine (se impostata).
    \item Il sistema aggiorna il campo con la data scelta.
\end{enumerate}

\textbf{Scenario secondario\textsubscript{\scalebox{0.6}{\textbf{G}}}}
\begin{enumerate}
    \item L'utente seleziona una data successiva alla data di fine attuale.
    \item Il sistema mostra un avviso di incongruenza temporale o resetta automaticamente la data di fine.
\end{enumerate}

\vspace{0.5cm}

\section{Requisiti}

Questa sezione classifica i Requisiti del sistema in quattro categorie principali: funzionali, di qualità\textsubscript{\scalebox{0.6}{\textbf{G}}}, di vincolo e prestazionali.

\subsection{Requisiti funzionali\textsubscript{\scalebox{0.6}{\textbf{G}}}}
Descrivono i servizi specifici e le funzioni che il sistema deve fornire. Rappresentano il "cosa" il sistema deve fare in risposta a determinati input o situazioni.
\vspace{0.5cm}

\textbf{Caratteristiche}
\begin{enumerate}
    \item Descrivono le interazioni tra l'utente (o altri sistemi) e il software.
    \item Sono diretti: se il requisito\textsubscript{\scalebox{0.6}{\textbf{G}}} non è soddisfatto, il sistema non funziona come previsto.
    \item Derivano direttamente dai Casi d'uso\textsubscript{\scalebox{0.6}{\textbf{G}}} e dalle user stories.
\end{enumerate}


\vspace{0.5cm}


\renewcommand{\arraystretch}{1.2}
\setlength{\tabcolsep}{4pt}

\newcommand{\wCod}{2.2cm}
\newcommand{\wFonti}{2.6cm}
\newcommand{\wPrio}{2.8cm}
\newcommand{\wDesc}{%
  \dimexpr\textwidth-\wCod-\wFonti-\wPrio-6\tabcolsep-5\arrayrulewidth\relax
}

\newcounter{rfCounter}

\newcommand{\autoRF}{%
  \stepcounter{rfCounter}% Incrementa il contatore
  RF-%
  \ifnum\value{rfCounter}<100 0\fi % Aggiunge uno 0 se < 100
  \ifnum\value{rfCounter}<10 0\fi  % Aggiunge un altro 0 se < 10
  \arabic{rfCounter}% Stampa il numero
}

\setcounter{rfCounter}{0}

\begin{longtable}{|L{\wCod}|L{\wDesc}|L{\wFonti}|L{\wPrio}|}
\caption{Tabella dei Requisiti funzionali\textsubscript{\scalebox{0.6}{\textbf{G}}}}\label{tab:reqfunzionali}\\
\hline
\textbf{Codice} & \textbf{Descrizione} & \textbf{Fonti} & \textbf{Priorità} \\
\hline
\endfirsthead

\multicolumn{4}{c}{{\bfseries \tablename\ \thetable{} -- continua dalla pagina precedente}}\\
\hline
\textbf{Codice} & \textbf{Descrizione} & \textbf{Fonti} & \textbf{Priorità} \\
\hline
\endhead

\hline
\multicolumn{4}{|r|}{{Continua nella prossima pagina...}}\\
\hline
\endfoot

\hline
\endlastfoot


\autoRF & Il sistema deve permettere all'utente non autenticato di effettuare la registrazione di un nuovo account. & \hyperref[UC-0A]{UC-0A} & Opzionale \\
\hline
\autoRF & Il sistema deve permettere all'utente di inserire il proprio indirizzo email in fase di registrazione. & \hyperref[UC-0A.1]{UC-0A.1} & Opzionale \\
\hline
\autoRF & Il sistema deve permettere all'utente di inserire una password in fase di registrazione. & \hyperref[UC-0A.2]{UC-0A.2} & Opzionale \\
\hline
\autoRF & Il sistema deve permettere all'utente di inserire un username in fase di registrazione. & \hyperref[UC-0A.3]{UC-0A.3} & Opzionale \\
\hline
\autoRF & Il sistema deve permettere all'utente di inserire il proprio nome in fase di registrazione. & \hyperref[UC-0A.4]{UC-0A.4} & Opzionale \\
\hline
\autoRF & Il sistema deve permettere all'utente di inserire il proprio cognome in fase di registrazione. & \hyperref[UC-0A.5]{UC-0A.5} & Opzionale \\
\hline
\autoRF & Il sistema deve permettere all'utente di inserire la propria matricola in fase di registrazione. & \hyperref[UC-0A.6]{UC-0A.6} & Opzionale \\
\hline
\autoRF & Il sistema deve visualizzare un messaggio di errore se il formato dell'email inserita non è valido. & \hyperref[UC-0A.7]{UC-0A.7} & Opzionale \\
\hline
\autoRF & Il sistema deve visualizzare un messaggio di errore se la password non rispetta i criteri di sicurezza. & \hyperref[UC-0A.8]{UC-0A.8} & Opzionale \\
\hline
\autoRF & Il sistema deve impedire la registrazione se l'email inserita è già associata a un account esistente. & \hyperref[UC-0A.9]{UC-0A.9} & Opzionale \\
\hline
\autoRF & Il sistema deve impedire la registrazione se lo username inserito è già utilizzato. & \hyperref[UC-0A.10]{UC-0A.10} & Opzionale \\
\hline
\autoRF & Il sistema deve impedire la registrazione se la matricola inserita è già presente nel sistema. & \hyperref[UC-0A.11]{UC-0A.11} & Opzionale \\
\hline
\autoRF & Il sistema deve visualizzare un messaggio di errore se il formato della matricola non è valido. & \hyperref[UC-0A.12]{UC-0A.12} & Opzionale \\
\hline
\autoRF & Il sistema deve permettere all'utente registrato di effettuare il login (autenticazione). & \hyperref[UC-0B]{UC-0B} & Opzionale \\
\hline
\autoRF & Il sistema deve notificare l'errore in caso di tentativo di login con email non registrata. & \hyperref[UC-0B.1]{UC-0B.1} & Opzionale \\
\hline
\autoRF & Il sistema deve notificare l'errore in caso di tentativo di login con password errata. & \hyperref[UC-0B.2]{UC-0B.2} & Opzionale \\
\hline
\autoRF & Il sistema deve permettere all'utente di visualizzare i dati del proprio profilo. & \hyperref[UC-0C]{UC-0C} & Obbligatorio \\
\hline
\autoRF & Il sistema deve mostrare l'email associata al profilo utente. & \hyperref[UC-0C.1]{UC-0C.1} & Obbligatorio \\
\hline
\autoRF & Il sistema deve mostrare (o permettere la gestione del) la password del profilo utente. & \hyperref[UC-0C.2]{UC-0C.2} & Obbligatorio \\
\hline
\autoRF & Il sistema deve mostrare lo username associato al profilo utente. & \hyperref[UC-0C.3]{UC-0C.3} & Obbligatorio \\
\hline
\autoRF & Il sistema deve mostrare il nome associato al profilo utente. & \hyperref[UC-0C.4]{UC-0C.4} & Obbligatorio \\
\hline
\autoRF & Il sistema deve mostrare il cognome associato al profilo utente. & \hyperref[UC-0C.5]{UC-0C.5} & Obbligatorio \\
\hline
\autoRF & Il sistema deve mostrare la matricola associata al profilo utente. & \hyperref[UC-0C.6]{UC-0C.6} & Obbligatorio \\
\hline
\autoRF & Il sistema deve permettere all'utente di modificare le informazioni del proprio profilo. & \hyperref[UC-0D]{UC-0D} & Opzionale \\
\hline
\autoRF & Il sistema deve gestire l'uscita dalla modifica profilo senza salvare i cambiamenti. & \hyperref[UC-0D.1]{UC-0D.1} & Opzionale \\
\hline
\autoRF & Il sistema deve permettere all'Amministratore di visualizzare la lista degli utenti registrati. & \hyperref[UC-0E]{UC-0E} & Opzionale \\
\hline
\autoRF & Il sistema deve permettere la visualizzazione del dettaglio di un singolo utente dalla lista. & \hyperref[UC-0E.1]{UC-0E.1} & Opzionale \\
\hline
\autoRF & Il sistema deve mostrare il ruolo associato a un utente registrato. & \hyperref[UC-0E.2]{UC-0E.2} & Opzionale \\
\hline
\autoRF & Il sistema deve mostrare il nome di un utente registrato. & \hyperref[UC-0E.3]{UC-0E.3} & Opzionale \\
\hline
\autoRF & Il sistema deve mostrare il cognome di un utente registrato. & \hyperref[UC-0E.4]{UC-0E.4} & Opzionale \\
\hline
\autoRF & Il sistema deve permettere all'Amministratore di modificare il ruolo di un utente registrato. & \hyperref[UC-0F]{UC-0F} & Opzionale \\
\hline
\autoRF & Il sistema deve permettere all'utente di effettuare il logout (terminare la sessione). & \hyperref[UC-0G]{UC-0G} & Opzionale \\
\hline
\autoRF & Il sistema deve generare contenuti testuali tramite AI Assistant in base a prompt e parametri. & \hyperref[UC-1A]{UC-1A} & Obbligatorio \\
\hline
\autoRF & Il sistema deve permettere l'inserimento di un prompt testuale per la generazione. & \hyperref[UC-1A.1]{UC-1A.1} & Obbligatorio \\
\hline
\autoRF & Il sistema deve permettere la selezione del tono per la generazione del contenuto. & \hyperref[UC-1A.2]{UC-1A.2} & Obbligatorio \\
\hline
\autoRF & Il sistema deve permettere la selezione dello stile per la generazione del contenuto. & \hyperref[UC-1A.3]{UC-1A.3} & Obbligatorio \\
\hline
\autoRF & Il sistema deve permettere la visualizzazione dello storico delle generazioni AI. & \hyperref[UC-1B]{UC-1B} & Obbligatorio \\
\hline
\autoRF & Il sistema deve notificare l'assenza di elementi se lo storico delle generazioni è vuoto. & \hyperref[UC-1B.1]{UC-1B.1} & Obbligatorio \\
\hline
\autoRF & Il sistema deve mostrare i dettagli completi di un elemento selezionato dallo storico. & \hyperref[UC-1B.2]{UC-1B.2} & Obbligatorio \\
\hline
\autoRF & Il sistema deve visualizzare lo stile utilizzato per un contenuto nello storico. & \hyperref[UC-1B.3]{UC-1B.3} & Obbligatorio \\
\hline
\autoRF & Il sistema deve visualizzare il testo del risultato generato nello storico. & \hyperref[UC-1B.4]{UC-1B.4} & Obbligatorio \\
\hline
\autoRF & Il sistema deve visualizzare il timestamp (data/ora) della generazione nello storico. & \hyperref[UC-1B.5]{UC-1B.5} & Obbligatorio \\
\hline
\autoRF & Il sistema deve visualizzare la valutazione assegnata dall'utente al contenuto nello storico. & \hyperref[UC-1B.6]{UC-1B.6} & Obbligatorio \\
\hline
\autoRF & Il sistema deve visualizzare il prompt originale utilizzato per un contenuto nello storico. & \hyperref[UC-1B.7]{UC-1B.7} & Obbligatorio \\
\hline
\autoRF & Il sistema deve visualizzare il tono utilizzato per un contenuto nello storico. & \hyperref[UC-1B.8]{UC-1B.8} & Obbligatorio \\
\hline
\autoRF & Il sistema deve mostrare un'anteprima del contenuto generato dall'AI. & \hyperref[UC-1C]{UC-1C} & Obbligatorio \\
\hline
\autoRF & Il sistema deve permettere di modificare l'immagine associata al contenuto generato. & \hyperref[UC-1D]{UC-1D} & Obbligatorio \\
\hline
\autoRF & Il sistema deve notificare l'utente quando tenta di caricare un file immagine non valido. & \hyperref[UC-1D.1]{UC-1D.1} & Obbligatorio \\
\hline
\autoRF & Il sistema deve permettere di modificare il titolo del contenuto generato. & \hyperref[UC-1E]{UC-1E} & Obbligatorio \\
\hline
\autoRF & Il sistema deve permettere di modificare il testo del corpo del contenuto generato. & \hyperref[UC-1F]{UC-1F} & Obbligatorio \\
\hline
\autoRF & Il sistema deve permettere di annullare le modifiche apportate al contenuto generato. & \hyperref[UC-1G]{UC-1G} & Obbligatorio \\
\hline
\autoRF & Il sistema deve permettere di riutilizzare i parametri di un contenuto dello storico per una nuova generazione. & \hyperref[UC-1H]{UC-1H} & Obbligatorio \\
\hline
\autoRF & Il sistema deve permettere di duplicare un contenuto dallo storico per modificarne i parametri. & \hyperref[UC-1I]{UC-1I} & Obbligatorio \\
\hline
\autoRF & Il sistema deve permettere di filtrare la lista delle generazioni nello storico. & \hyperref[UC-1J]{UC-1J} & Obbligatorio \\
\hline
\autoRF & Il sistema deve visualizzare la lista dello storico aggiornata in base ai filtri applicati. & \hyperref[UC-1K]{UC-1K} & Obbligatorio \\
\hline
\autoRF & Il sistema deve permettere di rigenerare un contenuto tramite AI mantenendo i parametri. & \hyperref[UC-1L]{UC-1L} & Obbligatorio \\
\hline
\autoRF & Il sistema deve permettere all'utente di valutare (rating) il contenuto generato. & \hyperref[UC-1M]{UC-1M} & Obbligatorio \\
\hline
\autoRF & Il sistema deve permettere di scartare il contenuto generato e pulire l'interfaccia. & \hyperref[UC-1N]{UC-1N} & Obbligatorio \\
\hline
\autoRF & Il sistema deve permettere di salvare il contenuto generato nel database. & \hyperref[UC-1O]{UC-1O} & Obbligatorio \\
\hline
\autoRF & Il sistema deve permettere all'utente l'inserimento di un nuovo tono per la generazione di contenuti & \hyperref[UC-1P]{UC-1P} & Obbligatorio \\
\hline
\autoRF & Il sistema deve permettere all'utente l'eliminazione di un tono per la generazione di contenuti & \hyperref[UC-1Q]{UC-1Q} & Obbligatorio \\
\hline
\autoRF & Il sistema deve permettere all'utente l'inserimento di un nuovo stile per la generazione di contenuti & \hyperref[UC-1R]{UC-1R} & Obbligatorio \\
\hline
\autoRF & Il sistema deve permettere all'utente l'eliminazione di uno stile per la generazione di contenuti & \hyperref[UC-1S]{UC-1S} & Obbligatorio \\
\hline
\autoRF & Il sistema deve permettere l'analisi di documenti tramite il modulo AI Co-Pilot. & \hyperref[UC-2A]{UC-2A} & Obbligatorio \\
\hline
\autoRF & Il sistema deve permettere l'inserimento/selezione della categoria del documento. & \hyperref[UC-2A.1]{UC-2A.1} & Obbligatorio \\
\hline
\autoRF & Il sistema deve permettere l'inserimento del mese/anno di competenza del documento. & \hyperref[UC-2A.2]{UC-2A.2} & Obbligatorio \\
\hline
\autoRF & Il sistema deve permettere l'inserimento dell'azienda associata al documento. & \hyperref[UC-2A.3]{UC-2A.3} & Obbligatorio \\
\hline
\autoRF & Il sistema deve permettere l'inserimento del reparto associato al documento. & \hyperref[UC-2A.4]{UC-2A.4} & Obbligatorio \\
\hline
\autoRF & Il sistema deve visualizzare la lista dei documenti analizzati. & \hyperref[UC-2B]{UC-2B} & Obbligatorio \\
\hline
\autoRF & Il sistema deve notificare l'utente se nessun documento è stato riconosciuto dall'analisi. & \hyperref[UC-2B.1]{UC-2B.1} & Obbligatorio \\
\hline
\autoRF & Il sistema deve permettere di visualizzare i dettagli di un singolo documento dalla lista. & \hyperref[UC-2B.2]{UC-2B.2} & Obbligatorio \\
\hline
\autoRF & Il sistema deve visualizzare la competenza (periodo) del documento analizzato. & \hyperref[UC-2B.3]{UC-2B.3} & Obbligatorio \\
\hline
\autoRF & Il sistema deve visualizzare l'azienda associata al documento analizzato. & \hyperref[UC-2B.4]{UC-2B.4} & Obbligatorio \\
\hline
\autoRF & Il sistema deve visualizzare la causale del documento analizzato. & \hyperref[UC-2B.5]{UC-2B.5} & Obbligatorio \\
\hline
\autoRF & Il sistema deve visualizzare la lingua rilevata nel documento. & \hyperref[UC-2B.6]{UC-2B.6} & Obbligatorio \\
\hline
\autoRF & Il sistema deve visualizzare il numero di pagine del documento. & \hyperref[UC-2B.7]{UC-2B.7} & Obbligatorio \\
\hline
\autoRF & Il sistema deve visualizzare il nome originale del file del documento. & \hyperref[UC-2B.8]{UC-2B.8} & Obbligatorio \\
\hline
\autoRF & Il sistema deve visualizzare la data di redazione/caricamento del documento. & \hyperref[UC-2B.9]{UC-2B.9} & Obbligatorio \\
\hline
\autoRF & Il sistema deve visualizzare il codice identificativo del documento. & \hyperref[UC-2B.10]{UC-2B.10} & Obbligatorio \\
\hline
\autoRF & Il sistema deve visualizzare la tipologia del documento. & \hyperref[UC-2B.11]{UC-2B.11} & Obbligatorio \\
\hline
\autoRF & Il sistema deve mostrare l'anteprima visiva del documento analizzato. & \hyperref[UC-2C]{UC-2C} & Obbligatorio \\
\hline
\autoRF & Il sistema deve permettere di modificare il destinatario associato al documento. & \hyperref[UC-2D]{UC-2D} & Obbligatorio \\
\hline
\autoRF & Il sistema deve permettere di modificare la tipologia del documento. & \hyperref[UC-2E]{UC-2E} & Obbligatorio \\
\hline
\autoRF & Il sistema deve ricalcolare la percentuale di confidenza dopo modifiche manuali. & \hyperref[UC-2F]{UC-2F} & Obbligatorio \\
\hline
\autoRF & Il sistema deve visualizzare la lista delle informazioni sui destinatari estratti. & \hyperref[UC-2G]{UC-2G} & Obbligatorio \\
\hline
\autoRF & Il sistema deve permettere di visualizzare i dettagli di un singolo destinatario in lista. & \hyperref[UC-2G.1]{UC-2G.1} & Obbligatorio \\
\hline
\autoRF & Il sistema deve visualizzare il codice fiscale del destinatario. & \hyperref[UC-2G.2]{UC-2G.2} & Obbligatorio \\
\hline
\autoRF & Il sistema deve visualizzare la matricola del destinatario. & \hyperref[UC-2G.3]{UC-2G.3} & Obbligatorio \\
\hline
\autoRF & Il sistema deve visualizzare il reparto del destinatario. & \hyperref[UC-2G.4]{UC-2G.4} & Obbligatorio \\
\hline
\autoRF & Il sistema deve visualizzare il nome/cognome del destinatario. & \hyperref[UC-2G.5]{UC-2G.5} & Obbligatorio \\
\hline
\autoRF & Il sistema deve notificare se nessun destinatario è stato riconosciuto. & \hyperref[UC-2G.6]{UC-2G.6} & Obbligatorio \\
\hline
\autoRF & Il sistema deve visualizzare lo storico completo dei documenti processati. & \hyperref[UC-2H]{UC-2H} & Obbligatorio \\
\hline
\autoRF & Il sistema deve notificare l'assenza di documenti nello storico. & \hyperref[UC-2H.1]{UC-2H.1} & Obbligatorio \\
\hline
\autoRF & Il sistema deve visualizzare i dettagli di un elemento nello storico documenti. & \hyperref[UC-2H.2]{UC-2H.2} & Obbligatorio \\
\hline
\autoRF & Il sistema deve visualizzare la percentuale di confidenza dell'analisi nello storico. & \hyperref[UC-2H.3]{UC-2H.3} & Obbligatorio \\
\hline
\autoRF & Il sistema deve visualizzare l'appartenenza alle liste di distribuzione. & \hyperref[UC-2H.4]{UC-2H.4} & Obbligatorio \\
\hline
\autoRF & Il sistema deve visualizzare lo stato di elaborazione del documento. & \hyperref[UC-2H.5]{UC-2H.5} & Obbligatorio \\
\hline
\autoRF & Il sistema deve permettere di caricare un template di messaggio esistente. & \hyperref[UC-2I]{UC-2I} & Obbligatorio \\
\hline
\autoRF & Il sistema deve permettere di modificare l'oggetto del messaggio. & \hyperref[UC-2J]{UC-2J} & Obbligatorio \\
\hline
\autoRF & Il sistema deve permettere di modificare il testo del corpo del messaggio. & \hyperref[UC-2K]{UC-2K} & Obbligatorio \\
\hline
\autoRF & Il sistema deve permettere di salvare il messaggio corrente come nuovo template. & \hyperref[UC-2L]{UC-2L} & Obbligatorio \\
\hline
\autoRF & Il sistema deve permettere di eliminare un template di messaggio. & \hyperref[UC-2M]{UC-2M} & Obbligatorio \\
\hline
\autoRF & Il sistema deve permettere di visualizzare la lista dei template di messaggio disponibili. & \hyperref[UC-2N]{UC-2N} & Obbligatorio \\
\hline
\autoRF & Il sistema deve permettere di visualizzare un elemento della lista dei template di messaggio disponibili. & \hyperref[UC-2N.1]{UC-2N.1} & Obbligatorio \\
\hline
\autoRF & Il sistema deve permettere di visualizzare l'oggetto del template. & \hyperref[UC-2N.2]{UC-2N.2} & Obbligatorio \\
\hline
\autoRF & Il sistema deve permettere di visualizzare il testo del template. & \hyperref[UC-2N.3]{UC-2N.3} & Obbligatorio \\
\hline
\autoRF & Il sistema deve permettere di visualizzare il codice del template. & \hyperref[UC-2N.4]{UC-2N.4} & Obbligatorio \\
\hline
\autoRF & Il sistema deve permettere l'invio del documento e del messaggio associato. & \hyperref[UC-2O]{UC-2O} & Obbligatorio \\
\hline
\autoRF &  Il sistema deve permettere di allegare ulteriore contenuto al messaggio. & \hyperref[UC-2O.1]{UC-2O.1} & Obbligatorio \\
\hline
\autoRF & Il sistema deve permettere di pianificare l'invio del documento e del messaggio associato. & \hyperref[UC-2O.2]{UC-2O.2} & Obbligatorio \\
\hline
\autoRF & Il sistema deve permettere il filtraggio della lista dei documenti analizzati. & \hyperref[UC-2P]{UC-2P} & Obbligatorio \\
\hline
\autoRF & Il sistema deve mostrare la lista dei documenti aggiornata in base ai filtri. & \hyperref[UC-2Q]{UC-2Q} & Obbligatorio \\
\hline
\autoRF & Il sistema deve permettere il filtraggio della lista dei destinatari. & \hyperref[UC-2R]{UC-2R} & Obbligatorio \\
\hline
\autoRF & Il sistema deve mostrare la lista dei destinatari aggiornata in base ai filtri. & \hyperref[UC-2S]{UC-2S} & Obbligatorio \\
\hline
\autoRF & Il sistema deve permettere il filtraggio della lista dello storico documenti. & \hyperref[UC-2T]{UC-2T} & Obbligatorio \\
\hline
\autoRF & Il sistema deve mostrare la lista dello storico documenti aggiornata in base ai filtri. & \hyperref[UC-2U]{UC-2U} & Obbligatorio \\
\hline
\autoRF & Il sistema deve visualizzare la dashboard con i dati di analytics per l'AI Assistant. & \hyperref[UC-3A]{UC-3A} & Obbligatorio \\
\hline
\autoRF & Il sistema deve visualizzare il numero totale di prompt generati. & \hyperref[UC-3A.1]{UC-3A.1} & Obbligatorio \\
\hline
\autoRF & Il sistema deve visualizzare il rating medio dei prompt generati. & \hyperref[UC-3A.2]{UC-3A.2} & Obbligatorio \\
\hline
\autoRF & Il sistema deve visualizzare il numero di rigenerazioni effettuate. & \hyperref[UC-3A.3]{UC-3A.3} & Obbligatorio \\
\hline
\autoRF & Il sistema deve visualizzare le statistiche sui toni più utilizzati. & \hyperref[UC-3A.4]{UC-3A.4} & Obbligatorio \\
\hline
\autoRF & Il sistema deve visualizzare le statistiche sugli stili più utilizzati. & \hyperref[UC-3A.5]{UC-3A.5} & Obbligatorio \\
\hline
\autoRF & Il sistema deve visualizzare la dashboard con i dati di analytics per l'AI Co-Pilot. & \hyperref[UC-3B]{UC-3B} & Obbligatorio \\
\hline
\autoRF & Il sistema deve visualizzare la confidenza media delle analisi documenti. & \hyperref[UC-3B.1]{UC-3B.1} & Obbligatorio \\
\hline
\autoRF & Il sistema deve visualizzare la percentuale di interventi manuali necessari. & \hyperref[UC-3B.2]{UC-3B.2} & Obbligatorio \\
\hline
\autoRF & Il sistema deve visualizzare l'accuratezza del mapping dei dati. & \hyperref[UC-3B.3]{UC-3B.3} & Obbligatorio \\
\hline
\autoRF & Il sistema deve visualizzare i tempi medi di analisi dei documenti. & \hyperref[UC-3B.4]{UC-3B.4} & Obbligatorio \\
\hline
\autoRF & Il sistema deve permettere di filtrare i dati di analytics per periodo temporale. & \hyperref[UC-3C]{UC-3C} & Obbligatorio \\
\hline


\end{longtable}


\subsection{Requisiti di qualità\textsubscript{\scalebox{0.6}{\textbf{G}}}}
Definiscono gli attributi qualitativi del software che influenzano l'esperienza d'uso e la manutenibilità\textsubscript{\scalebox{0.6}{\textbf{G}}} del progetto\textsubscript{\scalebox{0.6}{\textbf{G}}}. Spesso indicati come "attributi di qualità\textsubscript{\scalebox{0.6}{\textbf{G}}}" (es. usabilità, affidabilità, manutenibilità\textsubscript{\scalebox{0.6}{\textbf{G}}}).
\vspace{0.5cm}

\textbf{Caratteristiche}
\begin{enumerate}
    \item Determinano il livello di soddisfazione dell'utente e la facilità di evoluzione del software.
    \item Devono essere misurabili tramite metriche specifiche o feedback utente.
\end{enumerate}

\vspace{0.5cm}
\begin{longtable}{|L{\wCod}|L{\wDesc}|L{\wFonti}|L{\wPrio}|}
\caption{Tabella dei requisiti di qualità}\label{tab:reqqualita}\\
\hline
\textbf{Codice} & \textbf{Descrizione} & \textbf{Fonti} & \textbf{Priorità} \\
\hline
\endfirsthead

\multicolumn{4}{c}{{\bfseries \tablename\ \thetable{} -- continua dalla pagina precedente}}\\
\hline
\textbf{Codice} & \textbf{Descrizione} & \textbf{Fonti} & \textbf{Priorità} \\
\hline
\endhead

\hline
\multicolumn{4}{|r|}{{Continua nella prossima pagina...}}\\
\hline
\endfoot

\hline
\endlastfoot


    RQ-01 & Presentare documento analisi dei Requisiti\textsubscript{\scalebox{0.6}{\textbf{G}}} contenente diagrammi e descrizioni Use Case & capitolato\textsubscript{\scalebox{0.6}{\textbf{G}}} & Obbligatorio \\
    \hline
    RQ-02 & Il way of working\textsubscript{\scalebox{0.6}{\textbf{G}}} descritto in Norme di Progetto\textsubscript{\scalebox{0.6}{\textbf{G}}} deve essere rispettato & Interna & Obbligatorio \\
    \hline
    RQ-03 & Il Prodotto deve passare tutti i test\textsubscript{\scalebox{0.6}{\textbf{G}}} con la copertura concordata con la proponente\textsubscript{\scalebox{0.6}{\textbf{G}}} & capitolato\textsubscript{\scalebox{0.6}{\textbf{G}}}, Piano di Qualifica\textsubscript{\scalebox{0.6}{\textbf{G}}} & Obbligatorio \\
    \hline
    RQ-04 & Il codice deve essere documentato secondo le linee guida descritte in Norme di Progetto\textsubscript{\scalebox{0.6}{\textbf{G}}} & capitolato\textsubscript{\scalebox{0.6}{\textbf{G}}}, Interna & Obbligatorio \\
    \hline
    RQ-05 & È necessario versionare il codice con appositi strumenti di controllo versione, compreso di istruzioni di setup & capitolato\textsubscript{\scalebox{0.6}{\textbf{G}}} & Obbligatorio \\
    \hline
    RQ-06 & Report finale di integrazione e suggerimenti di evoluzione & capitolato\textsubscript{\scalebox{0.6}{\textbf{G}}} & Obbligatorio \\
    \hline

\end{longtable}

\subsection{Requisiti di Vincolo\textsubscript{\scalebox{0.6}{\textbf{G}}}}
Rappresentano le limitazioni e le restrizioni entro cui il sistema deve essere sviluppato o operare. Questi vincoli restringono lo spazio delle soluzioni possibili.
\vspace{0.5cm}

\textbf{Caratteristiche}
\begin{enumerate}
    \item Possono essere di natura tecnologica (hardware, linguaggio), normativa (GDPR) o di business (budget).
    \item Sono mandatori e non negoziabili.
\end{enumerate}

\vspace{0.5cm}
\begin{longtable}{|L{\wCod}|L{\wDesc}|L{\wFonti}|L{\wPrio}|}
\caption{Tabella dei Requisiti Vincolo}\label{tab:reqvincolo}\\
\hline
\textbf{Codice} & \textbf{Descrizione} & \textbf{Fonti} & \textbf{Priorità} \\
\hline
\endfirsthead

\multicolumn{4}{c}{{\bfseries \tablename\ \thetable{} -- continua dalla pagina precedente}}\\
\hline
\textbf{Codice} & \textbf{Descrizione} & \textbf{Fonti} & \textbf{Priorità} \\
\hline
\endhead

\hline
\multicolumn{4}{|r|}{{Continua nella prossima pagina...}}\\
\hline
\endfoot

\hline
\endlastfoot


    RV-01 & Git come sistema di controllo versione & capitolato\textsubscript{\scalebox{0.6}{\textbf{G}}} & Obbligatorio \\
    \hline
    RV-02 & API \& Backend devono essere sviluppati in Ruby on Rails\textsubscript{\scalebox{0.6}{\textbf{G}}} & capitolato\textsubscript{\scalebox{0.6}{\textbf{G}}} & Obbligatorio \\
    \hline
    RV-03 & Il database deve essere PostgreSQL\textsubscript{\scalebox{0.6}{\textbf{G}}} & capitolato\textsubscript{\scalebox{0.6}{\textbf{G}}} & Obbligatorio \\
    \hline
    RV-04 & Il frontend deve essere sviluppato in Angular\textsubscript{\scalebox{0.6}{\textbf{G}}} & capitolato\textsubscript{\scalebox{0.6}{\textbf{G}}} & Obbligatorio \\
    \hline
    RV-05 & La gestione dei modelli AI deve essere implementata utilizzando AWS Bedrock\textsubscript{\scalebox{0.6}{\textbf{G}}} & capitolato\textsubscript{\scalebox{0.6}{\textbf{G}}} & Obbligatorio \\
    \hline
    RV-06 & Eventuali background jobs gestiti con Sidekiq e PWA con Next.js & capitolato\textsubscript{\scalebox{0.6}{\textbf{G}}}, trattativa con la proponente\textsubscript{\scalebox{0.6}{\textbf{G}}} & Opzionale \\
    \hline

\end{longtable}

\subsection{Requisiti Prestazionali\textsubscript{\scalebox{0.6}{\textbf{G}}}}
Specificano i parametri numerici relativi all'efficienza\textsubscript{\scalebox{0.6}{\textbf{G}}} del sistema. Sebbene siano tecnicamente un sottoinsieme della qualità\textsubscript{\scalebox{0.6}{\textbf{G}}}, vengono trattati separatamente per la loro criticità e misurabilità quantitativa.
\vspace{0.5cm}

\textbf{Caratteristiche}
\begin{enumerate}
    \item Definiscono limiti su tempi di risposta, throughput e utilizzo delle risorse.
    \item Sono sempre espressi con valori numerici e soglie precise.
\end{enumerate}

\vspace{0.5cm}
\begin{longtable}{|L{\wCod}|L{\wDesc}|L{\wFonti}|L{\wPrio}|}
\caption{Tabella dei Requisiti Prestazionali\textsubscript{\scalebox{0.6}{\textbf{G}}}}\label{tab:reqprestazionali}\\
\hline
\textbf{Codice} & \textbf{Descrizione} & \textbf{Fonti} & \textbf{Priorità} \\
\hline
\endfirsthead

\multicolumn{4}{c}{{\bfseries \tablename\ \thetable{} -- continua dalla pagina precedente}}\\
\hline
\textbf{Codice} & \textbf{Descrizione} & \textbf{Fonti} & \textbf{Priorità} \\
\hline
\endhead

\hline
\multicolumn{4}{|r|}{{Continua nella prossima pagina...}}\\
\hline
\endfoot

\hline
\endlastfoot


    RP-01 & Il sistema deve generare contenuti testuali tramite AI (Modulo\textsubscript{\scalebox{0.6}{\textbf{G}}} Assistant) entro 5 secondi per testi fino a 500 parole & Interna & Obbligatorio \\
    \hline
    RP-02 & Il sistema deve classificare e partizionare documenti PDF (Modulo\textsubscript{\scalebox{0.6}{\textbf{G}}} Co-Pilot) entro 3 secondi per pagina & Interna & Obbligatorio \\
    \hline
    RP-03 & Il tempo di risposta dell'interfaccia\textsubscript{\scalebox{0.6}{\textbf{G}}} utente per operazioni standard deve essere inferiore a 2 secondi & Interna & Obbligatorio \\
    \hline
    RP-04 & Il sistema deve supportare l'upload di file PDF fino a 20 MB & Interna & Obbligatorio \\
    \hline
    RP-05 & La Dashboard\textsubscript{\scalebox{0.6}{\textbf{G}}} di Analytics deve caricare le statistiche entro 3 secondi per dataset fino a 1000 documenti & Interna & Desiderabile \\
    \hline
    RP-06 & Il sistema deve garantire una disponibilità del 99\% durante l'orario lavorativo (8:00-18:00) & Interna & Desiderabile \\
    \hline
    RP-07 & Il sistema deve essere in grado di processare almeno 50 documenti in parallelo senza degrado delle performance & Interna & Desiderabile \\
    \hline
    RP-08 & Il tempo di estrazione OCR per documenti scansionati deve essere inferiore a 5 secondi per pagina & Interna & Obbligatorio \\
    \hline

\end{longtable}


\end{document}