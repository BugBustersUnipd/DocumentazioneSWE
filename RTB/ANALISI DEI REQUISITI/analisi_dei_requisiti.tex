\documentclass[a4paper,11pt]{article}

\usepackage[utf8]{inputenc}
\usepackage[T1]{fontenc}
\usepackage[italian]{babel}
\usepackage[margin=2.5cm]{geometry}
\usepackage{graphicx}
\usepackage{booktabs}
\usepackage{setspace}
\usepackage{titlesec}
\usepackage{float}
\usepackage[table]{xcolor}
\usepackage{tabularx}
\usepackage{tcolorbox}
\usepackage{enumitem}
\usepackage[titles]{tocloft}
\usepackage[colorlinks=true,linkcolor=black,urlcolor=blue,citecolor=blue]{hyperref}
\usepackage{fancyhdr}
\usepackage{lastpage}
\usepackage{amsmath}

\pagestyle{fancy}
\fancyhf{}
\fancyhead[L]{BugBusters}
\fancyhead[R]{Analisi dei Requisiti}
\fancyfoot[L]{\thepage\ di \pageref{LastPage}}
\renewcommand{\headrulewidth}{0pt}
\renewcommand{\footrulewidth}{0pt}

\setlength{\headheight}{14pt}

\setlength{\parskip}{4pt}
\setlength{\parindent}{0pt}

\titleformat{\section}{\large\bfseries}{\thesection}{1em}{}
\titleformat{\subsection}{\normalsize\bfseries}{\thesubsection}{1em}{}

\begin{document}

\begin{center}
  \thispagestyle{empty}
  \IfFileExists{../../assets/Logo.jpg}{%
    \includegraphics[width=6cm,height=3cm,keepaspectratio]{../../assets/Logo.jpg} \\[0.8cm]
  }{%
    \fbox{\parbox[c][2.5cm][c]{6cm}{\centering Logo non trovato\\(Logo.jpg)}}\\[0.5cm]
  }
  {\LARGE\bfseries BugBusters}\\[0.8cm]
  
  \rule{\textwidth}{0.5pt}\\[0.5cm]
  {\Large\bfseries Analisi dei Requisiti}\\[0.3cm]
  {\large Versione 0.0.1}\\[0.5cm]
  \rule{\textwidth}{0.5pt}\\[0.8cm]
\end{center}

\begin{center}
\begin{tcolorbox}[colback=gray!10,width=0.8\textwidth,arc=3mm,boxrule=0.5pt]
\begin{tabular}{ll}
\textbf{Stato} & In redazione \\
\textbf{Redattori} & ----- \\
\textbf{Destinatari} & BugBusters \\
 & Prof. Vardanega Tullio \\
 & Prof. Cardin Riccardo \\
 & Eggon \\
\end{tabular}
\end{tcolorbox}
\end{center}

\vspace{1cm}

\begin{center}
\textbf{Descrizione}
\end{center}

\begin{center}
\begin{minipage}{0.9\textwidth}
\small
Questo documento contiene le Norme di Progetto seguite dal team \textbf{BugBusters} per il progetto\textsubscript{\scalebox{0.6}{\textbf{G}}} C5 proposto dall'azienda Eggon
\end{minipage}
\end{center}

\newpage

\section*{Registro delle Modifiche}

{\footnotesize
\begin{center}
\begin{tabularx}{\textwidth}{|l|l|X|l|l|}
\hline
\textbf{Versione} & \textbf{Data} & \textbf{Descrizione} & \textbf{Autore} & \textbf{Ruolo} \\
\hline
0.0.3 & 29/11/2025 & 
\begin{minipage}[t]{\linewidth}
Correzione casi d'uso e aggiunta schemi.
\end{minipage} 
& Leonardo Salviato& - \\
\hline
0.0.2 & 25/11/2025 & 
\begin{minipage}[t]{\linewidth}
Riscrittura della prima stesura e modifica casi d'uso.
\end{minipage} 
& Leonardo Salviato& - \\
\hline
0.0.1 & 16/11/2025 & 
\begin{minipage}[t]{\linewidth}
Prima stesura della struttura del documento.
\end{minipage} 
& Leonardo Salviato e Marco Piro & - \\
\hline
\end{tabularx}
\end{center}
}

\vfill
\begin{center}
2 di \pageref{LastPage}
\end{center}

\newpage

\section*{Indice}

\noindent
\begin{minipage}[t]{0.8\textwidth}
\subsection*{1 Introduzione}
1.1 Scopo del documento \\
1.2 Prospettiva del prodotto \\
1.3 Funzioni del prodotto \\
1.4 Caratterisitiche dell'utente \\
1.5 Definizioni, acronimi e abbreviazioni \\
1.6 Riferimenti \\
\quad 1.6.1 Riferimenti normativi \\
\quad 1.6.2 Riferimenti informativi \\
\subsection*{2 Casi d'uso}
2.1 Introduzione \\
2.2 Attori \\
2.3 Lista casi d'uso \\
2.4 Sezione 0 – Applicazione standalone\\
\quad 2.4.1 UC-0A: \\
\quad 2.4.2 UC-0B: \\
\quad 2.4.3 UC-0C: \\
\quad 2.4.4 UC-0D: \\
\quad 2.4.5 UC-0E: \\
2.5 Sezione 1 – Modulo “AI Assistant Generativo\\
\quad 2.5.1 UC-1A: \\
\quad 2.5.2 UC-1B: \\
\quad 2.5.3 UC-1C: \\
\quad 2.5.4 UC-1D: \\
\quad 2.5.5 UC-1E: \\
\quad 2.5.6 UC-1F: \\
\quad 2.5.7 UC-1G: \\
\quad 2.5.8 UC-1H: \\


2.6 Sezione 2 – Modulo "AI Co-Pilot per i CdL"\\
\quad 2.6.1 UC-2A: \\
\quad 2.6.2 UC-2B: \\
\quad 2.6.3 UC-2C: \\
\quad 2.6.4 UC-2D: \\
\quad 2.6.5 UC-2E: \\
\quad 2.6.5 UC-2F: \\
\quad 2.6.5 UC-2G: \\




\end{minipage}
\begin{minipage}[t]{0.2\textwidth}
\vspace{1.65\baselineskip}
9 \\
9 \\
9 \\
10 \\
10 \\
10 \\
\end{minipage}

\newpage

\section{Introduzione}

\subsection{Scopo del documento}
Questo documento di Analisi dei Requisiti\textsubscript{\scalebox{0.6}{\textbf{G}}}, adottato da parte di BugBusters durante lo svolgimento del progetto\textsubscript{\scalebox{0.6}{\textbf{G}}} didattico, ha lo scopo di definire in maniera precisa e dettagliata i requisiti funzionali\textsubscript{\scalebox{0.6}{\textbf{G}}} e non funzionali del Sistema software da sviluppare.

A seguito delle nuove decisioni progettuali rispetto alle proposte del capitolato, il Sistema non sarà inizialmente integrato nella piattaforma NEXUM, ma verrà realizzato come \textbf{applicazione standalone}, autonoma e indipendente. Tale applicazione implementerà i moduli “AI Assistant Generativo” e "AI Co-Pilot per i CdL" in un ambiente isolato, così da consentire una fase di sviluppo, test e validazione più controllata. Solo in una fase successiva si valuterà l’\textbf{integrazione con la piattaforma NEXUM}, garantendo continuità architetturale e coerenza con i moduli già presenti.

Il documento include una descrizione approfondita dei Casi d’Uso, che costituiscono la principale fonte dei requisiti finali. Per agevolare la comprensione, verranno utilizzati anche i \textbf{Diagrammi dei Casi d’Uso}, che visualizzano le interazioni tra utenti e Sistema.

Questo documento rappresenta il riferimento fondamentale per la progettazione, l’implementazione e il collaudo dell’applicazione standalone, assicurando che essa soddisfi pienamente le esigenze del Committente e gli obiettivi formativi del progetto.

I requisiti identificati sono classificati nelle seguenti categorie:
\begin{itemize}
    \item \textbf{Obbligatori}: necessari e imprescindibili per garantire il corretto funzionamento dell’applicazione standalone;
    \item \textbf{Desiderabili}: non strettamente necessari, ma capaci di migliorare l’esperienza utente o l’efficienza del Sistema;
    \item \textbf{Opzionali}: funzionalità aggiuntive utili per estensioni future, in particolare in vista della possibile integrazione con NEXUM.
\end{itemize}

Il documento è rivolto ai seguenti destinatari:
\begin{itemize}
    \item Il \textbf{Committente}, che potrà verificare che i requisiti siano stati compresi e documentati correttamente;
    \item Il \textbf{Team di Progettisti e Programmatori}, che utilizzerà questa analisi come base per la realizzazione del Sistema;
    \item Il \textbf{Team di Verificatori}, che impiegherà il presente documento per definire i casi di Test e validare il comportamento del prodotto.
\end{itemize}

\subsection{Prospettiva del prodotto}

Il prodotto che BugBusters si propone di sviluppare è una versione standalone dei moduli “AI Assistant Generativo” e "AI Co-Pilot per i CdL", inizialmente svincolata dalla piattaforma NEXUM. Tale applicazione costituirà un prototipo funzionale in grado di operare autonomamente e di implementare le principali funzionalità richieste dal Committente, senza dipendere dagli altri moduli della piattaforma.

L’app standalone permetterà di testare e consolidare le funzionalità richieste, offrendo un ambiente controllato che faciliti la sperimentazione e lo sviluppo incrementale. Questa fase costituirà la base per un’eventuale integrazione futura con la piattaforma NEXUM, la quale fornirà un ecosistema HR completo e dotato di servizi quali la messaggistica top-down, la timbratura digitale, la gestione delle anagrafiche e dei ruoli, e la collaborazione con gli studi dei Consulenti del Lavoro (CdL).

L’integrazione futura con NEXUM sarà concepita in modo modulare, consentendo alla nuova applicazione di inserirsi nell’architettura esistente come componente riutilizzabile e scalabile. L’integrazione includerà l’adattamento delle API, l’allineamento della gestione utenti e la centralizzazione dei dati all’interno dell’infrastruttura NEXUM.

\subsection{Funzioni del prodotto}

Dal punto di vista dell’utilizzatore finale, l’applicazione standalone dovrà fornire le seguenti funzionalità:

\begin{itemize}
    \item \textbf{Generazione di contenuti tramite AI (Modulo AI Assistant Generativo)}: 
    generazione di titolo, testo e immagine di copertina a partire da un prompt, 
    con possibilità di selezionare tono, stile e configurazioni avanzate del modello AI.

    \item \textbf{Salvataggio locale}: 
    gestione interna di prompt, contenuti generati, immagini e valutazioni, 
    tramite archivio locale dedicato all’app standalone, indipendente dalla piattaforma NEXUM.

    \item \textbf{Sistema di rating}: 
    valutazione della qualità dei contenuti generati dall’AI, utile per analisi interne e miglioramento continuo.

    \item \textbf{Gestione dei prompt}: 
    storico dei prompt utilizzati, con possibilità di riutilizzo, duplicazione e ricerca interna.

    \item \textbf{Esportazione dei contenuti}: 
    esportazione in formati standard (PDF, testo, immagine) per permettere anche un’integrazione manuale con sistemi esterni.

    \item \textbf{Dashboard standalone}: 
    visualizzazione e gestione di storico, filtri, ricerca e analisi delle interazioni con l’AI generativa.

    \item \textbf{Gestione delle immagini}: 
    possibilità di caricare immagini dall’utente o di generarle tramite AI, con salvataggio locale.

    \item \textbf{Gestione utenti}: 
    registrazione, autenticazione, gestione del profilo e configurazione dei parametri AI 
    (per utenti privilegiati come amministratori o editor avanzati).

    \item \textbf{Upload e gestione documentale (Modulo AI Co-Pilot per i CdL)}: 
    possibilità di caricare documenti (PDF, ZIP, immagini), salvarli localmente e gestirne lo stato di elaborazione.

    \item \textbf{Riconoscimento automatico della tipologia di documento}: 
    classificazione tramite AI (cedolini, CU, comunicazioni, lettere, moduli da firmare, ecc.) 
    sfruttando modelli OCR e classificatori addestrati.

    \item \textbf{Estrazione dei destinatari}: 
    riconoscimento automatico di informazioni contenute nei documenti 
    (nome, cognome, codice fiscale, matricola, reparto) tramite tecniche AI di entity extraction.

    \item \textbf{Split dei documenti massivi}: 
    suddivisione automatica dei documenti multi-destinatario (es. cedolini massivi) 
    in documenti singoli, ognuno associato al proprio destinatario riconosciuto.

    \item \textbf{Revisione manuale (Human-in-the-Loop)}: 
    interfaccia dedicata per verificare, correggere o confermare i risultati ottenuti dall’AI 
    in ogni fase (classificazione, destinatari, split).

    \item \textbf{Creazione di messaggi e liste di distribuzione}: 
    generazione automatica di bozze di messaggi e liste di destinatari derivanti dai documenti processati.

    \item \textbf{Tracciamento locale}: 
    storico delle operazioni effettuate (upload, riconoscimento, revisioni, esportazioni), 
    utile per audit interni e analisi del flusso documentale.

    \item \textbf{Esportazione documentale}: 
    generazione di pacchetti ZIP contenenti documenti processati, metadati, liste di destinatari e log di lavorazione.
\end{itemize}

Queste funzionalità permetteranno all’app standalone di essere completamente operativa e autonoma nei due moduli (AI Assistant Generativo e AI Co-Pilot per i CdL). 
In una fase successiva, tali componenti saranno progettati per essere integrati nella piattaforma NEXUM, 
consentendo così un’evoluzione verso un ecosistema HR completo, scalabile e basato su automazioni intelligenti.

\subsection{Caratterisitiche dell'utente}

Gli utilizzatori finali dell'applicazione standalone non appartengono a un’unica categoria specifica: 
l’obiettivo del progetto è quello di progettare moduli intelligenti e interoperabili 
che possano essere integrati all’interno dell’ecosistema NEXUM o utilizzati autonomamente durante la fase standalone.

In generale, è possibile affermare che gli utenti finali sono coloro che necessitano di uno strumento scalabile, 
intelligente e semplice da utilizzare per generare contenuti tramite AI e per gestire flussi documentali complessi con il supporto del modulo Co-Pilot.
Rientrano in questa categoria:

\begin{itemize}
    \item \textbf{Responsabili e amministratori HR}, che necessitano di strumenti avanzati per la creazione di comunicazioni interne, 
    la gestione dei contenuti generativi e l’analisi delle produzioni.

    \item \textbf{Consulenti del Lavoro (CdL) e personale amministrativo}, che richiedono un sistema in grado di caricare, riconoscere, suddividere e preparare documenti per la distribuzione ai destinatari.

    \item \textbf{Dipendenti e collaboratori} (in fase integrata), che potranno interagire con la piattaforma NEXUM per consultare documenti e comunicazioni, 
    pur non essendo utenti della versione standalone.

    \item \textbf{Manager aziendali}, interessati a monitorare la consistenza delle comunicazioni e l’efficienza dei processi documentali, 
    sia nella versione standalone che nella futura integrazione.
\end{itemize}

In sintesi, il prodotto è rivolto a organizzazioni di varie dimensioni — in particolare aziende medio-grandi e studi professionali — 
che necessitano di strumenti intelligenti per la creazione di contenuti, 
la gestione automatizzata dei documenti e la collaborazione con gli studi dei Consulenti del Lavoro.
L’app standalone funge da primo passo verso una piattaforma HR completa, modulare e potenziata dall’AI.


\subsection{Definizioni, acronimi e abbreviazioni}
Per tutte le definizioni, acronimi e abbreviazioni utilizzati in questo documento, si faccia
riferimento al \textbf{Glossario}, fornito come documento separato, che contiene tutte le spiegazioni
necessarie per garantire una comprensione uniforme dei termini tecnici e dei concetti
rilevanti per il progetto.

\newpage

\subsection{Riferimenti}

\subsubsection{Riferimenti normativi}
\begin{itemize}
\item \textbf{Capitolato\textsubscript{\scalebox{0.6}{\textbf{G}}}
 d'appalto C5: Nexum - Piattaforma di consulenza e documentazione previdenziale}\\
\url{https://www.math.unipd.it/~tullio/IS-1/2025/Progetto/C5.pdf}
\end{itemize}

\subsubsection{Riferimenti informativi}
\begin{itemize}
\item \textbf{Glossario\textsubscript{\scalebox{0.6}{\textbf{G}}}
:}\\
\url{https://github.com/BugBustersUnipd/DocumentazioneSWE/blob/main//RTB/GLOSSARIO/Glossario.pdf}
\end{itemize}



\section{Casi d'uso}
\subsection{Introduzione}
I casi d’uso si compongono di un grafico UML e una descrizione testuale che permetta di
comprendere al meglio cosa il prodotto deve fornire. La descrizione testuale, in particolar
modo, dovrà contenere le informazioni sotto presenti, salvo i casi in cui lo
specifico campo non risulti rilevante (ad esempio, un Caso d’Uso\textsubscript{\scalebox{0.6}{\textbf{G}}} che non prevede la
possibilità di errori non avrà Scenari secondari):

\begin{itemize}
    \item Attori: Sono coloro che interagiscono attivamente con il Sistema e
    svolgono l’azione indicata dal Caso d’Uso
    \item Precondizioni: Lista di elementi che sono necessari affinché l’Attore possa
    compiere l’azione indicata dal caso d’uso
    \item Postcondizioni: Lista di elementi che descrivono come il Sistema risulta
    essere internamente cambiato dopo che l’Attore ha effettuato
    l’azione prevista dal Caso d’Uso
    \item Scenario principale: Descrizione ragionevole delle operazioni che l’attore deve
    fare per compiere l’azione descritta dal Caso d’Uso
    \item Scenario secondario: Descrizione ragionevole degli eventi che possono accadere
    qualora una delle operazioni descritte nello Scenario
    principale non vada a buon fine
    \item Inclusioni: Casi d’Uso ulteriori che l’Attore deve compiere per realizzare
    il Caso d’Uso attualmente descritto
    \item Estensioni: Casi d’Uso ulteriori che possono realizzarsi durante
    l’esecuzione delle operazioni del Caso d’Uso principale
    \item Motivazioni che portano l’Attore a svolgere l’azione descritta
    dal Caso d’Uso. Non sempre disponibile in quanto il Caso
    d’Uso potrebbe essere incluso da un altro caso d’uso
    «principale»
\end{itemize}

\subsection{Attori}
Di seguito sono esposti gli attori utilizzati:
\begin{itemize}
    \item DA SPIEGARE OGNUNO DI QUESTI ATTORI
    \item DA AGGIUNGERE DIAGRAMMA DEGLI ATTORI
    \item HR Manager
    \item Redattore
    \item Data Analyst (Eggon)
    \item Amministratore
    \item Operatore Studio CdL (Admin/Editor), Sistema NEXUM
    \item Sistema NEXUM (AI Doc Classifier), Operatore CdL
    \item Sistema NEXUM (Entity Resolver), Operatore CdL
    \item Sistema NEXUM (Splitter), Operatore CdL
    \item Operatore CdL, Sistema
    \item Sistema NEXUM (Dispatcher), Operatore CdL
    \item Sistema NEXUM (Dispatcher \& Tracking), Operatore CdL, Destinatario finale
    \item Operatore CdL, Auditor interno, Sistema
    \item Admin Cliente, Admin Eggon, Sistema
    \item Data Analyst, Admin Eggon, Sistema
\end{itemize}


\subsection{Lista casi d'uso}

\text{L'elenco dei casi d'uso sará diviso in tre parti:}
\begin{itemize}
    \item 0 - Casi d'uso per la gestione utenti e autenticazione
    \item 1 - Casi d'uso per il modulo "AI Assistant Generativo"
    \item 2 - Casi d'uso per il modulo "AI Doc Classifier"
\end{itemize}


\subsection{Sezione 0 – Applicazione standalone}

\subsubsection{UC-0A – Registrazione nuovo utente}

\begin{figure}[H]
    \centering
    \includegraphics[width=1\textwidth]{Diagrammi casi d'uso/UC0A.jpg}
    \caption{Didascalia dell'immagine}
\end{figure}


\textbf{Attori}
\begin{itemize}
    \item Utente non autenticato (nuovo utente).
    \item Sistema di autenticazione dell’applicazione standalone.
\end{itemize}

\textbf{Relazioni con altri casi d'uso (\textit{include} / \textit{extend})}
\begin{itemize}
    \item \textit{include}: nessuno (la registrazione è un punto di ingresso al Sistema).
    \item \textit{extend}: UC-0B – Login / Autenticazione utente (in caso di login automatico al termine della registrazione).
\end{itemize}

\textbf{Pre-condizioni}
\begin{itemize}
    \item L’utente non ha una sessione attiva.
    \item L’utente non è ancora registrato nel Sistema (l’e-mail inserita non risulta già presente).
\end{itemize}

\textbf{Scenario principale}
\begin{enumerate}
    \item L’utente accede alla schermata di registrazione dell’applicazione standalone.
    \item L’utente inserisce i dati richiesti (ad esempio: nome, cognome, e-mail, password).
    \item L’utente accetta le condizioni d’uso e l’informativa sulla privacy.
    \item Il Sistema verifica la correttezza formale dei dati inseriti (es. formato e-mail, forza della password).
    \item Il Sistema controlla che l’indirizzo e-mail non sia già associato a un account esistente.
    \item In caso di esito positivo, il Sistema crea un nuovo account utente e lo memorizza nel proprio archivio.
    \item Il Sistema conferma l’avvenuta registrazione e può opzionalmente eseguire il login automatico del nuovo utente.
\end{enumerate}

\textbf{Post-condizioni}
\begin{itemize}
    \item Esiste un nuovo account utente registrato nel Sistema.
    \item L’utente può effettuare il login utilizzando le credenziali appena create.
\end{itemize}

\vspace{0.5cm}

\subsubsection{UC-0B – Login / Autenticazione utente}

\begin{figure}[H]
    \centering
    \includegraphics[width=0.8\textwidth]{Diagrammi casi d'uso/UC0B.jpg}
    \caption{Didascalia dell'immagine}
\end{figure}

\textbf{Attori}
\begin{itemize}
    \item Utente registrato.
    \item Sistema di autenticazione dell’applicazione standalone.
\end{itemize}

\textbf{Relazioni con altri casi d'uso (\textit{include} / \textit{extend})}
\begin{itemize}
    \item \textit{include}: nessuno (caso d’uso base di autenticazione).
    \item \textit{extend}: nessuno (questo caso d’uso è base e viene richiamato da altri, ma non estende ulteriori flussi).
\end{itemize}

\textbf{Pre-condizioni}
\begin{itemize}
    \item L’utente è già registrato nel Sistema.
    \item Non esiste una sessione attiva associata all’utente sul dispositivo corrente.
\end{itemize}

\textbf{Scenario principale}
\begin{enumerate}
    \item L’utente accede alla schermata di login.
    \item L’utente inserisce le proprie credenziali (e-mail e password).
    \item Il Sistema verifica la correttezza delle credenziali.
    \item In caso di credenziali valide, il Sistema crea una nuova sessione autenticata per l’utente.
    \item Il Sistema reindirizza l’utente alla dashboard principale dell’applicazione standalone.
\end{enumerate}

\textbf{Post-condizioni}
\begin{itemize}
    \item L’utente risulta autenticato nel Sistema.
    \item È attiva una sessione associata all’utente, che consente l’accesso alle funzionalità riservate (es. generazione contenuti, upload documenti).
\end{itemize}

\vspace{0.5cm}

\subsubsection{UC-0C – Gestione profilo utente}

\begin{figure}[H]
    \centering
    \includegraphics[width=1\textwidth]{Diagrammi casi d'uso/UC0C.jpg}
    \caption{Didascalia dell'immagine}
\end{figure}

\textbf{Attori}
\begin{itemize}
    \item Utente autenticato.
    \item Sistema di gestione profilo dell’applicazione standalone.
\end{itemize}

\textbf{Relazioni con altri casi d'uso (\textit{include} / \textit{extend})}
\begin{itemize}
    \item \textit{include}: UC-0B – Login / Autenticazione utente (è necessario che l’utente sia autenticato).
    \item \textit{extend}: nessuno.
\end{itemize}

\textbf{Pre-condizioni}
\begin{itemize}
    \item L’utente ha effettuato il login ed è autenticato.
    \item Esiste un profilo associato all’utente nel Sistema (dati anagrafici e preferenze).
\end{itemize}

\textbf{Scenario principale}
\begin{enumerate}
    \item L’utente accede alla sezione “Profilo” dalla dashboard dell’applicazione.
    \item Il Sistema mostra i dati correnti del profilo (es. nome, cognome, e-mail, ruolo, preferenze AI come tono/stile predefinito).
    \item L’utente modifica uno o più campi del profilo (es. nome visualizzato, preferenze di tono, lingua).
    \item L’utente conferma le modifiche.
    \item Il Sistema valida i dati inseriti (ad esempio formato dell’e-mail, campi obbligatori).
    \item Il Sistema salva le modifiche nel proprio archivio.
    \item Il Sistema conferma l’avvenuto aggiornamento del profilo.
\end{enumerate}

\textbf{Post-condizioni}
\begin{itemize}
    \item Le informazioni del profilo utente risultano aggiornate nel Sistema.
    \item Le nuove preferenze (ad esempio tono/stile predefinito) verranno utilizzate nelle interazioni successive con i moduli AI.
\end{itemize}

\vspace{0.5cm}

\subsubsection{UC-0D – Gestione ruoli (Admin / Editor)}

\textbf{Attori}
\begin{itemize}
    \item Amministratore dell’applicazione standalone (Admin).
    \item Utenti registrati (Editor o altri ruoli definiti).
    \item Sistema di gestione ruoli e permessi.
\end{itemize}

\textbf{Relazioni con altri casi d'uso (\textit{include} / \textit{extend})}
\begin{itemize}
    \item \textit{include}: UC-0B – Login / Autenticazione utente (l’Amministratore deve essere autenticato come Admin).
    \item \textit{extend}: nessuno.
\end{itemize}

\textbf{Pre-condizioni}
\begin{itemize}
    \item L’utente amministratore ha effettuato il login ed è autenticato come Admin.
    \item Esistono uno o più account utente registrati nel Sistema.
\end{itemize}

\textbf{Scenario principale}
\begin{enumerate}
    \item L’Amministratore accede alla sezione di amministrazione utenti.
    \item Il Sistema mostra l’elenco degli utenti registrati, con i rispettivi ruoli correnti.
    \item L’Amministratore seleziona un utente da modificare.
    \item L’Amministratore assegna o modifica il ruolo dell’utente (es. da Editor a Admin, oppure rimozione privilegi).
    \item L’Amministratore conferma le modifiche.
    \item Il Sistema aggiorna i ruoli e i permessi associati all’utente.
    \item Il Sistema registra l’operazione per finalità di audit interno.
\end{enumerate}

\textbf{Post-condizioni}
\begin{itemize}
    \item I ruoli e i permessi degli utenti risultano aggiornati nel Sistema.
    \item Le funzionalità accessibili a ciascun utente dipendono dal nuovo ruolo assegnato (es. solo Admin può modificare i parametri AI globali).
\end{itemize}

\vspace{0.5cm}

\subsubsection{UC-0E – Logout}

\begin{figure}[H]
    \centering
    \includegraphics[width=1\textwidth]{Diagrammi casi d'uso/UC0E.jpg}
    \caption{Didascalia dell'immagine}
\end{figure}

\textbf{Attori}
\begin{itemize}
    \item Utente autenticato.
    \item Sistema di gestione sessione dell’applicazione standalone.
\end{itemize}

\textbf{Relazioni con altri casi d'uso (\textit{include} / \textit{extend})}
\begin{itemize}
    \item \textit{include}: UC-0B – Login / Autenticazione utente (per poter effettuare il logout è necessaria una sessione attiva).
    \item \textit{extend}: Nessuno.
\end{itemize}

\textbf{Pre-condizioni}
\begin{itemize}
    \item L’utente ha una sessione attiva nel Sistema.
\end{itemize}

\textbf{Scenario principale}
\begin{enumerate}
    \item L’utente seleziona l’opzione di logout (ad esempio dal menu della dashboard).
    \item Il Sistema invalida la sessione corrente associata all’utente (es. rimozione token di sessione).
    \item Il Sistema reindirizza l’utente alla schermata di login o alla schermata iniziale pubblica.
\end{enumerate}

\textbf{Post-condizioni}
\begin{itemize}
    \item Non esiste più una sessione attiva associata all’utente sul dispositivo corrente.
    \item Per accedere nuovamente alle funzionalità riservate è necessario eseguire un nuovo login.
\end{itemize}


\subsection{Sezione 1 – Modulo AI Assistant Generativo}

\subsubsection{UC-1A – Creazione di un nuovo contenuto tramite prompt}

\textbf{Attori}
\begin{itemize}
    \item Utente autenticato (Editor o Admin).
    \item Sistema AI di generazione contenuti.
\end{itemize}

\textbf{Pre-condizioni}
\begin{itemize}
    \item L’utente ha effettuato il login.
    \item L’utente dispone dei permessi necessari per utilizzare il modulo AI Assistant.
\end{itemize}

\textbf{Scenario principale}
\begin{enumerate}
    \item L’utente accede alla sezione “AI Assistant Generativo”.
    \item Il Sistema mostra il campo per l’inserimento del prompt.
    \item L’utente inserisce il prompt desiderato (ad esempio: “Comunicato di benvenuto ai nuovi dipendenti”).
    \item Il Sistema richiede la selezione di tono e stile (formale, informale, tecnico, neutro).
    \item Il Sistema invia la richiesta al motore AI.
    \item Il motore AI genera titolo, contenuto testuale e immagine di copertina coerenti.
    \item Il Sistema mostra l’anteprima del contenuto generato.
\end{enumerate}

\textbf{Post-condizioni}
\begin{itemize}
    \item Una bozza temporanea del contenuto risulta generata e visibile in anteprima.
\end{itemize}

\vspace{0.5cm}

\subsubsection{UC-1B – Modifica del contenuto generato}

\textbf{Attori}
\begin{itemize}
    \item Utente autenticato.
    \item Sistema AI (in caso di nuove generazioni parziali).
\end{itemize}

\textbf{Pre-condizioni}
\begin{itemize}
    \item Un contenuto è stato generato tramite UC-1A.
    \item L’utente ha i permessi per modificarlo.
\end{itemize}

\textbf{Scenario principale}
\begin{enumerate}
    \item L’utente apre l’anteprima del contenuto generato.
    \item Il Sistema mostra gli elementi modificabili (titolo, corpo del messaggio, immagine).
    \item L’utente modifica uno o più elementi.
    \item Il Sistema aggiorna l’anteprima in tempo reale.
    \item L’utente conferma le modifiche effettuate.
\end{enumerate}

\textbf{Post-condizioni}
\begin{itemize}
    \item La versione aggiornata del contenuto è pronta per essere salvata o scartata.
\end{itemize}

\vspace{0.5cm}

\subsubsection{UC-1C – Pubblicazione o scarto del contenuto}

\textbf{Attori}
\begin{itemize}
    \item Utente autenticato (Editor o Admin).
    \item Sistema di persistenza locale.
\end{itemize}

\textbf{Pre-condizioni}
\begin{itemize}
    \item Un contenuto è disponibile, generato tramite UC-1A e/o modificato tramite UC-1B.
\end{itemize}

\textbf{Scenario principale}
\begin{enumerate}
    \item L’utente decide se salvare la bozza, pubblicare il contenuto o scartarlo.
    \item Il Sistema esegue la validazione dei campi obbligatori.
    \item Se l’utente sceglie “Pubblica”, il Sistema memorizza il contenuto e lo marca come “pubblicato”.
    \item Se l’utente sceglie “Scarta”, il Sistema elimina la bozza e ritorna alla schermata iniziale.
\end{enumerate}

\textbf{Post-condizioni}
\begin{itemize}
    \item Il contenuto risulta pubblicato o eliminato, a seconda dell’azione scelta dall’utente.
\end{itemize}

\vspace{0.5cm}

\subsubsection{UC-1D – Salvataggio e gestione dei prompt}

\textbf{Attori}
\begin{itemize}
    \item Utente autenticato.
    \item Sistema di archivio locale.
\end{itemize}

\textbf{Pre-condizioni}
\begin{itemize}
    \item L’utente ha generato almeno un contenuto tramite UC-1A.
\end{itemize}

\textbf{Scenario principale}
\begin{enumerate}
    \item Il Sistema salva automaticamente il prompt utilizzato, il tono, lo stile e la data di utilizzo.
    \item L’utente accede alla sezione “Storico Prompt”.
    \item Il Sistema mostra l’elenco dei prompt salvati.
    \item L’utente può filtrare, ricercare, duplicare o riutilizzare un prompt.
\end{enumerate}

\textbf{Post-condizioni}
\begin{itemize}
    \item Lo storico dei prompt risulta aggiornato e consultabile.
\end{itemize}

\vspace{0.5cm}

\subsubsection{UC-1E – Valutazione della qualità del contenuto (Rating)}

\textbf{Attori}
\begin{itemize}
    \item Utente autenticato.
\end{itemize}

\textbf{Pre-condizioni}
\begin{itemize}
    \item L’utente ha generato un contenuto ed è in grado di visualizzarlo.
\end{itemize}

\textbf{Scenario principale}
\begin{enumerate}
    \item Il Sistema mostra un widget di valutazione (es. stelle 1–5).
    \item L’utente seleziona il voto desiderato.
    \item Il Sistema salva la valutazione e la associa al contenuto e/o al prompt originario.
\end{enumerate}

\textbf{Post-condizioni}
\begin{itemize}
    \item Il contenuto ha associata una valutazione salvata localmente.
\end{itemize}

\vspace{0.5cm}

\subsubsection{UC-1F – Rigenerazione del contenuto}

\textbf{Attori}
\begin{itemize}
    \item Utente autenticato.
    \item Sistema AI di generazione contenuti.
\end{itemize}

\textbf{Pre-condizioni}
\begin{itemize}
    \item L’utente ha effettuato il login.
    \item Un contenuto è stato generato tramite UC-1A.
\end{itemize}

\textbf{Scenario principale}
\begin{enumerate}
    \item L’utente ha generato un contenuto.
    \item Nell'anteprima del contenuto decide di volere un'alternativa.
    \item Il contenuto viene rigenerato e ci si ritrova nuovamente nell'anteprima.
    \item Si ha la possibilitá di rigenerare il contenuto.
\end{enumerate}

\textbf{Post-condizioni}
\begin{itemize}
    \item Le preferenze AI dell’utente vengono applicate automaticamente nelle generazioni successive.
\end{itemize}

\vspace{0.5cm}

\subsubsection{UC-1G – Revisione finale del contenuto}

\textbf{Attori}
\begin{itemize}
    \item Utente autenticato (Editor o Admin).
\end{itemize}

\textbf{Pre-condizioni}
\begin{itemize}
    \item Un contenuto è stato generato e/o modificato.
\end{itemize}

\textbf{Scenario principale}
\begin{enumerate}
    \item L’utente accede alla schermata di revisione.
    \item Il Sistema mostra il contenuto completo (titolo, testo, immagine).
    \item L’utente verifica la correttezza del contenuto.
    \item L’utente decide se procedere al salvataggio/pubblicazione (UC-1C) o tornare alla modifica (UC-1B).
\end{enumerate}

\textbf{Post-condizioni}
\begin{itemize}
    \item Il contenuto è pronto per essere gestito tramite UC-1C.
\end{itemize}

\vspace{0.5cm}

\subsubsection{UC-1H – Analisi e reportistica sull’utilizzo}

\textbf{Attori}
\begin{itemize}
    \item Utente autenticato (Admin o analista interno).
\end{itemize}

\textbf{Pre-condizioni}
\begin{itemize}
    \item Esistono dati salvati relativi a generazioni, rating e utilizzo del Sistema.
\end{itemize}

\textbf{Scenario principale}
\begin{enumerate}
    \item L’utente accede alla dashboard di analytics.
    \item Il Sistema mostra statistiche aggregate (numero generazioni, rating medi, prompt più usati, frequenza d’uso).
    \item L’utente applica filtri o intervalli temporali.
    \item Il Sistema aggiorna la visualizzazione dei dati in base ai criteri selezionati.
\end{enumerate}

\textbf{Post-condizioni}
\begin{itemize}
    \item Le informazioni statistiche sono consultabili per valutare l'efficacia del modulo generativo.
\end{itemize}

\subsection{Sezione 2 – Modulo AI Co-Pilot per i Consulenti del Lavoro (CdL)}

\subsubsection{UC-2A – Upload documento nel repository locale CdL}

\textbf{Attori}
\begin{itemize}
    \item Operatore di studio CdL (utente autenticato).
    \item Sistema di gestione documentale locale dell’applicazione standalone.
\end{itemize}

\textbf{Pre-condizioni}
\begin{itemize}
    \item L’utente ha effettuato il login nell’applicazione standalone.
    \item Il repository documentale locale è disponibile e correttamente inizializzato.
\end{itemize}

\textbf{Scenario principale}
\begin{enumerate}
    \item L’utente accede alla sezione ``Repository documentale CdL''.
    \item Il Sistema mostra l’elenco dei documenti già presenti e le azioni disponibili.
    \item L’utente seleziona l’azione di upload di uno o più file (es. PDF, ZIP, immagini).
    \item L’utente seleziona i file dal proprio dispositivo.
    \item L’utente, se necessario, inserisce alcuni metadati di base (es. mese/anno di competenza, azienda, categoria documento).
    \item Il Sistema verifica il formato dei file caricati.
    \item In caso di esito positivo, il Sistema memorizza i file nel repository locale e li marca come ``In attesa di elaborazione''.
\end{enumerate}

\textbf{Post-condizioni}
\begin{itemize}
    \item I documenti risultano archiviati nel repository locale con stato ``In attesa di elaborazione''.
    \item I documenti sono disponibili per le successive fasi di riconoscimento ed elaborazione AI.
\end{itemize}

\vspace{0.5cm}

\subsubsection{UC-2B – Riconoscimento automatico del documento (OCR + classificazione)}

\textbf{Attori}
\begin{itemize}
    \item Sistema AI di riconoscimento documentale (OCR + classificatore).
    \item Operatore di studio CdL (in qualità di supervisore).
\end{itemize}

\textbf{Pre-condizioni}
\begin{itemize}
    \item Uno o più documenti sono presenti nel repository con stato ``In attesa di elaborazione'' (UC-2A completato).
    \item Il motore OCR e il classificatore AI sono configurati e disponibili nell’applicazione standalone.
\end{itemize}

\textbf{Scenario principale}
\begin{enumerate}
    \item L’utente avvia l’elaborazione automatica dei documenti oppure il Sistema la esegue in modo asincrono.
    \item Il Sistema esegue l’OCR sul documento per estrarne il testo.
    \item Il Sistema applica il modello di classificazione per identificare la tipologia di documento (es. cedolino, CU, comunicazione, lettera).
    \item Il Sistema associa al documento la tipologia rilevata e un punteggio di confidenza.
    \item Il Sistema aggiorna lo stato del documento a ``Riconosciuto'' o ``Da verificare'' in base alla soglia di confidenza.
\end{enumerate}

\textbf{Post-condizioni}
\begin{itemize}
    \item Ogni documento elaborato possiede una tipologia identificata e un punteggio di confidenza.
    \item I documenti sono pronti per l’estrazione dei destinatari o per una eventuale revisione manuale.
\end{itemize}

\vspace{0.5cm}

\subsubsection{UC-2C – Estrazione e riconoscimento dei destinatari}

\textbf{Attori}
\begin{itemize}
    \item Sistema AI di estrazione entità (entity extraction).
    \item Operatore di studio CdL.
\end{itemize}

\textbf{Pre-condizioni}
\begin{itemize}
    \item Il documento è stato elaborato tramite OCR e classificazione (UC-2B completato).
    \item Il testo del documento è disponibile in forma strutturata o semi-strutturata.
\end{itemize}

\textbf{Scenario principale}
\begin{enumerate}
    \item L’utente seleziona un documento in stato ``Riconosciuto''.
    \item Il Sistema analizza il testo alla ricerca di entità rilevanti (es. nome e cognome, codice fiscale, matricola, reparto).
    \item Il Sistema propone uno o più possibili destinatari in base ai dati estratti.
    \item Il Sistema associa al documento il destinatario selezionato o quello con confidenza più alta.
    \item Il Sistema aggiorna lo stato del documento a ``Destinatario riconosciuto'' o ``Ambiguità da risolvere''.
\end{enumerate}

\textbf{Post-condizioni}
\begin{itemize}
    \item Per ciascun documento è stato individuato almeno un destinatario, con relativo punteggio di confidenza.
    \item I documenti sono pronti per l’eventuale fase di split (se multi-destinatario) o per la revisione manuale.
\end{itemize}

\vspace{0.5cm}

\subsubsection{UC-2D – Split dei documenti massivi per destinatario}

\textbf{Attori}
\begin{itemize}
    \item Sistema di splitting documentale.
    \item Operatore di studio CdL.
\end{itemize}

\textbf{Pre-condizioni}
\begin{itemize}
    \item È presente un documento multi-destinatario (es. file unico con più cedolini) nel repository.
    \item Il documento è stato elaborato tramite OCR (UC-2B) ed eventualmente sono stati identificati pattern utili alla separazione.
\end{itemize}

\textbf{Scenario principale}
\begin{enumerate}
    \item L’utente seleziona un documento identificato come ``massivo''.
    \item Il Sistema analizza la struttura del documento (pagine, sezioni, marcatori) per individuare i confini fra i diversi destinatari.
    \item Il Sistema propone una suddivisione del documento in più documenti singoli.
    \item Il Sistema associa a ciascun documento derivato uno o più possibili destinatari (se disponibili da UC-2C).
    \item L’utente conferma o corregge la suddivisione proposta.
    \item Il Sistema crea i documenti singoli nel repository, mantenendo un riferimento all’originale.
\end{enumerate}

\textbf{Post-condizioni}
\begin{itemize}
    \item Il documento massivo è stato suddiviso in documenti singoli, ciascuno associato al proprio destinatario.
    \item I documenti singoli risultano pronti per la revisione e per la successiva esportazione.
\end{itemize}

\vspace{0.5cm}

\subsubsection{UC-2E – Revisione manuale dei risultati (Human-in-the-Loop)}

\textbf{Attori}
\begin{itemize}
    \item Operatore di studio CdL.
\end{itemize}

\textbf{Pre-condizioni}
\begin{itemize}
    \item Esistono documenti in stato ``Da verificare'', ``Ambiguità da risolvere'' oppure documenti derivati dallo split (UC-2B, UC-2C o UC-2D completati).
\end{itemize}

\textbf{Scenario principale}
\begin{enumerate}
    \item L’utente accede alla sezione di revisione documenti.
    \item Il Sistema mostra l’elenco dei documenti che richiedono intervento umano (bassa confidenza, ambiguità, split da confermare).
    \item L’utente seleziona un documento da revisionare.
    \item Il Sistema mostra il documento, i metadati rilevati (tipologia, destinatari, competenza) e i relativi punteggi di confidenza.
    \item L’utente corregge, conferma o integra le informazioni (tipologia documento, destinatario, metadati).
    \item L’utente conferma le modifiche.
    \item Il Sistema aggiorna i dati del documento e imposta lo stato come ``Validato manualmente''.
\end{enumerate}

\textbf{Post-condizioni}
\begin{itemize}
    \item I documenti selezionati risultano revisionati e validati dall’operatore.
    \item Le informazioni corrette sono pronte per la creazione di liste di distribuzione o per l’esportazione.
\end{itemize}

\vspace{0.5cm}

\subsubsection{UC-2F – Creazione di liste di distribuzione ed esportazione dei documenti}

\textbf{Attori}
\begin{itemize}
    \item Operatore di studio CdL.
    \item Sistema di esportazione documentale dell’applicazione standalone.
\end{itemize}

\textbf{Pre-condizioni}
\begin{itemize}
    \item I documenti sono stati riconosciuti e validati (UC-2B, UC-2C, UC-2D e UC-2E, secondo necessità).
    \item Ogni documento è associato a uno o più destinatari.
\end{itemize}

\textbf{Scenario principale}
\begin{enumerate}
    \item L’utente accede alla sezione di generazione delle liste di distribuzione.
    \item Il Sistema propone una lista di documenti raggruppati per destinatario.
    \item L’utente seleziona un insieme di documenti da esportare (ad esempio tutti i cedolini di un mese).
    \item Il Sistema genera una lista di distribuzione contenente destinatari e documenti associati.
    \item L’utente sceglie la modalità di esportazione (es. cartella locale, archivio ZIP, esportazione metadati in CSV).
    \item Il Sistema produce il pacchetto di esportazione contenente documenti e metadati (lista di distribuzione).
\end{enumerate}

\textbf{Post-condizioni}
\begin{itemize}
    \item È stato generato un output (es. archivio ZIP o struttura di cartelle) che contiene i documenti pronti per l’invio tramite canali esterni all’applicazione standalone.
    \item La lista di distribuzione è salvata localmente per eventuale tracciamento e riuso.
\end{itemize}

\vspace{0.5cm}

\subsubsection{UC-2G – Consultazione storico documentale e audit locale}

\textbf{Attori}
\begin{itemize}
    \item Operatore di studio CdL.
    \item Amministratore dell’applicazione standalone.
\end{itemize}

\textbf{Pre-condizioni}
\begin{itemize}
    \item Sono presenti nel Sistema documenti caricati, elaborati, revisionati o esportati.
\end{itemize}

\textbf{Scenario principale}
\begin{enumerate}
    \item L’utente accede alla sezione ``Storico e audit'' dell’applicazione.
    \item Il Sistema mostra l’elenco dei documenti con i relativi stati (caricato, riconosciuto, validato, esportato).
    \item L’utente applica filtri di ricerca (per azienda, destinatario, tipologia documento, intervallo temporale).
    \item Il Sistema aggiorna i risultati in base ai filtri selezionati.
    \item L’utente seleziona un documento per visualizzare i dettagli (metadati, log delle operazioni, eventuali revisioni manuali).
\end{enumerate}

\textbf{Post-condizioni}
\begin{itemize}
    \item L’utente dispone di una visione completa e tracciata delle operazioni svolte sui documenti.
    \item Le informazioni storiche possono essere utilizzate per verifiche interne e per migliorare il flusso di lavoro.
\end{itemize}



\end{document}