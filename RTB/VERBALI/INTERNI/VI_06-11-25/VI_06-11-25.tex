\documentclass[a4paper,12pt]{article}

\usepackage[utf8]{inputenc}
\usepackage[T1]{fontenc}
\usepackage[italian]{babel}
\usepackage[margin=2.5cm]{geometry}
\usepackage{graphicx}
\usepackage{grffile}
\usepackage{booktabs}
\usepackage{setspace}
\usepackage{titlesec}
\usepackage{float}
\usepackage{ifthen}
\usepackage{tcolorbox}
\usepackage{enumitem}
\usepackage[colorlinks=true,linkcolor=black,urlcolor=primaryblue,citecolor=primaryblue]{hyperref}
\usepackage{changepage} % per definire un rientro (un margine) solo in una parte del documento

\usepackage[table]{xcolor}   % per i colori (celle, testo e righe)
\usepackage{tabularx}               % per tabelle a larghezza adattiva (X)
\usepackage{array}                  % per comandi di formattazione avanzata nelle colonne (p, >{\raggedright})

% Macro
% genera la stringa "\noindent (Riferimento alla tabella decisioni:
% \hyperref[RTB4]{RTB4})", si usa facendo \refDecisione{NomeLabel}{Testo}
\newcommand{\refDecisione}[2]{%
    \noindent (\textbf{Riferimento alla tabella decisioni: \hyperref[#1]{#2}})%
}



\definecolor{primaryblue}{RGB}{0,102,204}
\definecolor{secondaryblue}{RGB}{51,153,255}
\definecolor{lightgray}{RGB}{245,245,245}
\definecolor{darkgray}{RGB}{100,100,100}

\titleformat{\section}
  {\Large\bfseries\color{primaryblue}}
  {\thesection}{1em}{}

\setlength{\parskip}{4pt}
\setlength{\parindent}{0pt}

\setlist[itemize]{leftmargin=*,itemsep=3pt}
\setlist[enumerate]{leftmargin=*,itemsep=3pt}

\graphicspath{{./}{../assets/images/}{./images/}}

\begin{document}

\begin{center}  \IfFileExists{../../../../assets/Logo.jpg}{%
    \includegraphics[width=6cm,height=3cm,keepaspectratio]{../../../../assets/Logo.jpg} \\[0.8cm]
  }{%
    \fbox{\parbox[c][2.5cm][c]{6cm}{\centering Logo non trovato\\(Logo.jpg)}}\\[0.5cm]
  }
  
  {\Large\bfseries\color{primaryblue} BugBusters}\\[0.3cm]
  {\small\color{darkgray} Email: \texttt{bugbusters.unipd@gmail.com}} \\[0.1cm]
  {\small\color{darkgray} Gruppo: 4} \\[0.5cm]

  {\large\bfseries Università degli Studi di Padova}\\[0.3cm]
  {\small Laurea in Informatica}\\[0.2cm]
  {\small Corso: Ingegneria del Software}\\[0.2cm]
  {\small Anno Accademico: 2025/2026}\\[0.8cm]

  {\Huge\bfseries\color{primaryblue} Verbale Interno}\\[0.3cm]
  {\Large\color{secondaryblue} 6 novembre 2025}\\[0.8cm]
\end{center}

\begin{center}
\begin{tcolorbox}[colback=lightgray,colframe=primaryblue,width=0.85\textwidth,arc=3mm,boxrule=0.5pt]
\begin{tabular}{@{}ll@{}}
\textbf{Redattori}    & Linor Sadè \\
\textbf{Verificatore}    & Marco Piro \\
\textbf{Uso}          & Interno \\
\textbf{Destinatari}  & BugBusters \\
\textbf{Versione} & 0.0.2\\

\end{tabular}
\end{tcolorbox}
\end{center}

\vspace{0.5cm}

\begin{center}
\begin{tcolorbox}[colback=secondaryblue!10,colframe=secondaryblue,width=0.9\textwidth,arc=3mm,boxrule=0.8pt,title={\bfseries Abstract}]
Verbale interno del gruppo di progetto dedicato all'allineamento sulla lezione di documentazione, all'integrazione dei feedback del docente, alla definizione della tracciabilità di decisioni e azioni, alla riprogettazione del sito e della repository, e alla pianificazione delle prossime attività operative.
\end{tcolorbox}
\end{center}

\newpage

\tableofcontents
\newpage

\section{Informazioni generali}

\begin{itemize}
    \item \textbf{Tipo riunione:} interna
    \item \textbf{Piattaforma:} Discord
    \item \textbf{Data:} 6/11/2025
    \item \textbf{Orario di inizio:} 14:30
    \item \textbf{Orario di fine:} 16:00
    \item \textbf{Presenti:}
    \begin{itemize}[leftmargin=1.5em, itemsep=3pt, label={\rule[0.5ex]{0.4em}{0.4em}}]
        \item Linor Sadè
        \item Alberto Autiero
        \item Marco Favero
        \item Luca Slongo
        \item Alberto Pignat
        \item Leonardo Salviato
    \end{itemize}
    \item \textbf{Assenti:} Marco Piro
\end{itemize}

\section{Ordine del giorno}

\begin{enumerate}
    \item \label{itm:lezione} Allineamento sulla lezione di Ingegneria del Software del giorno corrente;
    \item \label{itm:feedback} Integrazione dei feedback del docente sulla documentazione di candidatura;
    \begin{enumerate}
        \item \label{itm:tracciabilita} Definizione della tracciabilità di decisioni e azioni nei verbali;
        \item \label{itm:rotazione_ruoli} Integrazione delle regole di rotazione dei ruoli nel documento di Dichiarazione degli impegni;
        \item \label{itm:sito} Riprogettazione del sito web e della repository GitHub;
    \end{enumerate}
    \item \label{itm:scelta_ruoli} Scelta dei ruoli da coprire e frequenza di rotazione.
    \item \label{itm:prossimeattivita} Pianificazione delle prossime attività.
    \begin{itemize}
        \item prosecuzione della redazione di Glossario e Norme di Progetto;
        \item creazione di un foglio di rendicontazione ore (preventivate vs. effettive) per ogni membro e ruolo;
        \item discussione sulla tempistica del prossimo colloquio con l'azienda e sulle attività preparatorie da svolgere in vista dell'incontro.
        \item avvio dello studio delle tecnologie richieste
    \end{itemize}
\end{enumerate}

\newpage
\section{Svolgimento}
Qui di seguito i punti discussi con più dettaglio:

\subsection*{\ref{itm:lezione}. Allineamento sulla lezione di Ingegneria del Software del giorno corrente}
I membri presenti alla lezione di Ingegneria del Software di data odierna, in sostituzione di chi era assente per il Career Day o altri impegni personali, hanno discusso i contenuti della lezione rovesciata dedicata alla documentazione. In particolare, ci siamo confrontati su glossario e norme di progetto, chiarendo obiettivi e modalità di redazione.

\subsection*{\ref{itm:feedback}. Integrazione dei feedback del docente sulla documentazione di candidatura}
\label{par:ruoli}
Abbiamo analizzato i commenti del docente relativi alla documentazione per la candidatura del progetto, valutando le modifiche da apportare per migliorare la coerenza e la completezza dei documenti.

\begin{adjustwidth}{1cm}{1cm} % per un rientro a sinistra e destra
    \subsubsection*{\ref{itm:tracciabilita}. Definizione della tracciabilità di decisioni e azioni nei verbali}
    \label{par:tracciabilita}
    È stato discusso come inserire nei verbali una sezione dedicata alle decisioni e alle azioni, assicurando che queste risultino tracciabili nel sistema di gestione (definizione di un ID collegato a github issue, persona/e incaricate, descrizione della decisione con relative azioni).\\
    Abbiamo deciso di iniziare la tracciabilità dal verbale corrente in quanto la modifica dei precedenti verabli sarebbe un'attività di poco valore.\\
    Quando nei dettagli del verbale è riportata una discussione su un tema specifico e questa discussione porta a una decisione con conseguente azione, deve essere creato un collegamento tra la discussione e la decisione. In particolare, il collegamento deve essere \textbf{bidirezionale}:

    \begin{itemize}[leftmargin=1cm, rightmargin=1cm]
        \item nella tabella delle decisioni (dove ogni decisione ha un ID) deve esserci un riferimento al paragrafo del verbale che ha portato alla decisione;
        \item nel paragrafo del verbale deve esserci un collegamento alla decisione corrispondente nella tabella.
    \end{itemize}
    \refDecisione{RTB1}{RTB1} %macro definita dentro assets/macro/macro-verbali.tex
    \subsubsection*{\ref{itm:rotazione_ruoli}. Integrazione delle regole di rotazione dei ruoli nel documento di Dichiarazione degli impegni}
    Si è valutato come integrare le regole di rotazione dei ruoli all'interno del documento di dichiarazione degli impegni.

    \subsubsection*{\ref{itm:sito}. Riprogettazione del sito web e della repository GitHub}
    \label{par:sito}
    È stato deciso di avviare una riprogettazione del sito web per migliorarne l'accessibilità, la navigazione e la completezza dei contenuti, in linea con i riscontri ricevuti dal professore. In parallelo, si prevede una ristrutturazione della repository GitHub per garantire maggiore chiarezza e uniformità.\\
    In particolare:
    \begin{itemize}
    \item È stata \textbf{rimossa l'automatizzazione della pubblicazione} di tutti i file presenti nel branch \texttt{main}.
    
    \item Per la \textbf{creazione di un nuovo documento} per tutti i documenti che non sono verbali si decide di:
    \begin{itemize}
        \item creare un branch dedicato allo sviluppo del documento;
        \item al termine della stesura, aprire una pull request verso il branch \texttt{verifica};
        \item i verificatori hanno il compito di controllare la correttezza e la conformità del documento;
        \item una volta approvato, il documento viene fuso nel branch \texttt{main}, segnando la versione \textbf1.0.0.
    \end{itemize}
    
    \item La pubblicazione sul sito web avviene solo con i file presenti nel branch \texttt{main} e viene fatto manualmente modificando l'html.
    \item Il file README.md viene modificato aggiungendo le istruzioni per la creazione di un documento LaTeX, la pubblicazione su Github e sul sito.
\end{itemize}

\refDecisione{RTB2}{RTB2} %macro definita dentro assets/macro/macro-verbali.tex


\end{adjustwidth}

\subsection*{\ref{itm:scelta_ruoli} Scelta dei ruoli da coprire e frequenza di rotazione}
Il gruppo ha stabilito che i ruoli saranno ruotati ogni due settimane, in corrispondenza degli sprint previsti dall’azienda proponente Eggon. Al momento della redazione del presente verbale, la data di inizio dello sprint corrente non era nota.
Nel frattempo, sono stati assegnati i seguenti ruoli per il prossimo sprint, che, salvo diversa indicazione da parte dell’azienda, è previsto iniziare il 10/11/2025:
\begin{center}


Alberto Autiero \(\rightarrow\) Responsabile \\
Marco Favero \(\rightarrow\) Amministratore \\
Alberto Pignat \(\rightarrow\) Analista \\
Marco Piro \(\rightarrow\) Verificatore \\
Linor Sadè \(\rightarrow\) Analista \\
Leonardo Saviato \(\rightarrow\) Analista \\
Luca Slongo \(\rightarrow\) Analista
\end{center}


Inoltre si è deciso che ogni file (esclusi i verbali) verrà redatto da uno specifico ruolo, a seconda dello scopo del documento.

\subsection*{\ref{itm:prossimeattivita} Pianificazione delle prossime attività}

\paragraph{Glossario}
\label{par:glossario}
Da iniziare uno studio più approfondito sulla redazione e sull'usabilità del glossario, le domande su cui ci si deve focalizzare sono:
\begin{itemize}[leftmargin=1cm, rightmargin=1cm]
    \item Quali termini deve contenere il glossario? Dove sono contenuti i termini?
    \item A chi è rivolto il glossario? Se a più destinatari, bisogna redigere più glossari? Lo stesso termine può essere ripetuto più volte in glossari diversi con definizioni coerenti al registro e forma più appropriata al destinatario?
    \item Come avviene il collegamento tra termine e glossario? è sufficientemente efficiente ed efficace?
    \item È opportuno inserire a piè di pagina una definizione più riassuntiva del termine ed eventualmente rimandare al glossario?
    \item Se un termine è ripetuto più volte in un documento, come deve avvenire il riferimento al glossario? 
\end{itemize}
Applicazione delle risposte alla versione corrente del glossario.\\

\refDecisione{RTB3}{RTB3} %macro definita dentro assets/macro/macro-verbali.tex

\paragraph{Foglio di calcolo per la rendicontazione ore}
\label{par:esempio}
Idea iniziale su come si può suddividere il foglio di calcolo, in particolare:
\begin{itemize}[leftmargin=1cm, rightmargin=1cm]
    \item Una scheda o sezione dedicata per ciascuno sprint;
    \item In ogni scheda, una tabella che riporti i membri del gruppo sull'asse delle y e i ruoli sull'asse delle x;
    \item Una colonna finale che indichi le ore rimanenti per ciascun membro rispetto al totale previsto;
    \item Una riga di riepilogo che mostri le ore rimanenti per ciascun ruolo rispetto al monte ore complessivo assegnato;
    \item Una riga aggiuntiva che indichi le ore consigliate per ogni ruolo in ciascuno sprint, accompagnata da un'analisi sul rispetto di tali valori o sugli eventuali sforamenti.
\end{itemize}

\paragraph{Studio tecnologie}
\label{par:studiotec}
È stato inoltre concordato di iniziare una fase preliminare di studio e approfondimento delle tecnologie indicate dall'azienda proponente, con l'obiettivo di acquisire familiarità con gli strumenti e i linguaggi che saranno impiegati nello sviluppo del progetto.\\
A questo proposito abbiamo discusso della necessità di dialogare con l'azienda proponente per assicurarci della corretta decisione di prosecuzione.\\
\refDecisione{RTB7}{RTB5} %macro definita dentro assets/macro/macro-verbali.tex


\paragraph{Prossimo colloquio con la proponente}
Durante la riunione è stato deciso che il prossimo incontro in via telematica con Eggon sarà possibilmente martedì della prossima settimana (10.11 -16.11)
e che in vista di tale incontro il gruppo si impegnerà ad analizzare con molta più attenzione il documento di capitolato per la creazione
di un file \textit{Google docs} con quanto analizzato. Nella riunione si discuterà dei requisiti funzionali e non funzionali dei moduli da integrare nella piattaforma Nexum.



\section{Tabella delle decisioni e azioni}

\setlength{\extrarowheight}{2pt} % padding extra verticale
\renewcommand{\arraystretch}{1.5} 


\arrayrulecolor{primaryblue}
\sloppy
\begin{tabularx}{\textwidth}{
    |>{\raggedright\arraybackslash}p{3cm}|
    >{\raggedright\arraybackslash}X|
    >{\raggedright\arraybackslash}p{3cm}|
}
\hline
\rowcolor{primaryblue!40}
\textbf{\color{white} ID Decisione} & \textbf{\color{white} Descrizione} & \textbf{\color{white} Incaricato} \\
\hline
\rowcolor{secondaryblue!10} \href{https://github.com/BugBustersUnipd/DocumentazioneSWE/issues/1}{C1} \label{C1} & Integrazione delle regole di rotazione nel documento di dichiarazione degli impegni (vedi \hyperref[par:ruoli]{dettagli}). & Luca Slongo \\
\hline
\rowcolor{secondaryblue!10} \href{https://github.com/BugBustersUnipd/DocumentazioneSWE/issues/1}{RTB1} \label{RTB1} & Redigere il verbale della corrente riunione e definire la sezione decisione~-~azioni tracciabili nel sistema (vedi \hyperref[par:tracciabilita]{dettagli}). & Linor Sadè \\
\hline
\rowcolor{secondaryblue!10} \href{https://github.com/BugBustersUnipd/DocumentazioneSWE/issues/3}{RTB2} \label{RTB2} & Riprogettazione sito web e repository GitHub in linea con il resocontro del professore e regole di accessibilità (vedi \hyperref[par:sito]{dettagli}). & Marco Favero \\
\hline
\rowcolor{secondaryblue!10} \href{https://github.com/BugBustersUnipd/DocumentazioneSWE/issues/4}{RTB3} \label{RTB3} & Studio e continuo della redazione del glossario. (vedi \hyperref[par:glossario]{dettagli}) & Alberto Autiero \\
\hline
\rowcolor{secondaryblue!10} \href{https://github.com/BugBustersUnipd/DocumentazioneSWE/issues/5}{RTB4} \label{RTB4} & Scrittura del foglio di calcolo per la rendicontazione ore, seguendo l'idea riportata \hyperref[par:esempio]{qui}.  & Leonardo Salviato \\
\hline
\rowcolor{secondaryblue!10} RTB5 \label{RTB5} & Scrittura all'azienda. & Alberto Autiero \\
\hline
\rowcolor{secondaryblue!10} RTB6 & creare la KanBan e le issue su GitHub a seconda di quanto scritto su questo verbale. & Marco Piro \\
\hline
\rowcolor{secondaryblue!10} RTB7 & Migliorie grafiche alla firma e-mail. & Alberto Pignat \\
\hline
\end{tabularx}
\fussy

\vfill
\begin{center}
    {\small\color{darkgray} Documento redatto e approvato dal gruppo BugBusters.}
\end{center}

\end{document}
