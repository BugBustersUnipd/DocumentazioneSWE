\documentclass[a4paper,12pt]{article}

\usepackage[utf8]{inputenc}
\usepackage[T1]{fontenc}
\usepackage[italian]{babel}
\usepackage[margin=2.5cm]{geometry}
\usepackage{graphicx}
\usepackage{grffile}
\usepackage{booktabs}
\usepackage{setspace}
\usepackage{titlesec}
\usepackage{float}
\usepackage{ifthen}
\usepackage{tcolorbox}
\usepackage{enumitem}
\usepackage[colorlinks=true,linkcolor=black,urlcolor=primaryblue,citecolor=primaryblue]{hyperref}
\usepackage{changepage} % per definire un rientro (un margine) solo in una parte del documento

\usepackage[table]{xcolor}   % per i colori (celle, testo e righe)
\usepackage{tabularx}               % per tabelle a larghezza adattiva (X)
\usepackage{array}                  % per comandi di formattazione avanzata nelle colonne (p, >{\raggedright})

% Macro
% genera la stringa "\noindent (Riferimento alla tabella decisioni:
% \hyperref[RTB4]{RTB4})", si usa facendo \refDecisione{NomeLabel}{Testo}
\newcommand{\refDecisione}[2]{%
    \noindent (\textbf{Riferimento alla tabella decisioni: \hyperref[#1]{#2}})%
}



\definecolor{primaryblue}{RGB}{0,102,204}
\definecolor{secondaryblue}{RGB}{51,153,255}
\definecolor{lightgray}{RGB}{245,245,245}
\definecolor{darkgray}{RGB}{100,100,100}

\titleformat{\section}
  {\Large\bfseries\color{primaryblue}}
  {\thesection}{1em}{}

\setlength{\parskip}{4pt}
\setlength{\parindent}{0pt}

\setlist[itemize]{leftmargin=*,itemsep=3pt}
\setlist[enumerate]{leftmargin=*,itemsep=3pt}

\graphicspath{{./}{../assets/images/}{./images/}}

\begin{document}

\begin{center}  \IfFileExists{../../../../assets/Logo.jpg}{%
    \includegraphics[width=6cm,height=3cm,keepaspectratio]{../../../../assets/Logo.jpg} \\[0.8cm]
  }{%
    \fbox{\parbox[c][2.5cm][c]{6cm}{\centering Logo non trovato\\(Logo.jpg)}}\\[0.5cm]
  }
  
  {\Large\bfseries\color{primaryblue} BugBusters}\\[0.3cm]
  {\small\color{darkgray} Email: \texttt{bugbusters.unipd@gmail.com}} \\[0.1cm]
  {\small\color{darkgray} Gruppo: 4} \\[0.5cm]

  {\large\bfseries Università degli Studi di Padova}\\[0.3cm]
  {\small Laurea in Informatica}\\[0.2cm]
  {\small Corso: Ingegneria del Software}\\[0.2cm]
  {\small Anno Accademico: 2025/2026}\\[0.8cm]

  {\Huge\bfseries\color{primaryblue} Verbale Interno}\\[0.3cm]
  {\Large\color{secondaryblue} 6 novembre 2025}\\[0.8cm]
\end{center}

\begin{center}
\begin{tcolorbox}[colback=lightgray,colframe=primaryblue,width=0.85\textwidth,arc=3mm,boxrule=0.5pt]
\begin{tabular}{@{}ll@{}}
\textbf{Redattori}    & Linor Sadè \\
\textbf{Verificatore}    & Alberto Pignat \\
\textbf{Uso}          & Interno \\
\textbf{Destinatari}  & BugBusters \\
\textbf{Versione} & 1.0.0\\

\end{tabular}
\end{tcolorbox}
\end{center}

\vspace{0.5cm}

\begin{center}
\begin{tcolorbox}[colback=secondaryblue!10,colframe=secondaryblue,width=0.9\textwidth,arc=3mm,boxrule=0.8pt,title={\bfseries Abstract}]
Verbale interno del gruppo di progetto\textsubscript{\textbf{G}} dedicato all'allineamento sulla lezione di documentazione\textsubscript{\textbf{G}}, all'integrazione dei feedback del docente, alla definizione della tracciabilità di decisioni e azioni, alla riprogettazione del sito e della repository\textsubscript{\textbf{G}}, e alla pianificazione delle prossime attività\textsubscript{\textbf{G}} operative.
\end{tcolorbox}
\end{center}

\newpage

\tableofcontents
\newpage

\section{Informazioni generali}

\begin{itemize}
    \item \textbf{Tipo riunione:} interna
    \item \textbf{Piattaforma:} Discord\textsubscript{\textbf{G}}
    \item \textbf{Data:} 6/11/2025
    \item \textbf{Orario di inizio:} 14:30
    \item \textbf{Orario di fine:} 16:00
    \item \textbf{Presenti:}
    \begin{itemize}[leftmargin=1.5em, itemsep=3pt, label={\rule[0.5ex]{0.4em}{0.4em}}]
        \item Linor Sadè
        \item Alberto Autiero
        \item Marco Favero
        \item Luca Slongo
        \item Alberto Pignat
        \item Leonardo Salviato
    \end{itemize}
    \item \textbf{Assenti:} Marco Piro
\end{itemize}

\section{Ordine del giorno}

\begin{enumerate}
    \item \label{itm:lezione} Allineamento sulla lezione di Ingegneria del Software del giorno corrente;
    \item \label{itm:feedback} Integrazione dei feedback del docente sulla documentazione\textsubscript{\textbf{G}} di candidatura;
    \begin{enumerate}
        \item \label{itm:tracciabilita} Definizione della tracciabilità di decisioni e azioni nei verbali;
        \item \label{itm:rotazione_ruoli} Integrazione delle regole di rotazione dei ruoli\textsubscript{\textbf{G}} nel documento\textsubscript{\textbf{G}} di Dichiarazione degli impegni;
        \item \label{itm:sito} Riprogettazione del sito web e della repository\textsubscript{\textbf{G}} GitHub\textsubscript{\textbf{G}};
    \end{enumerate}
    \item \label{itm:scelta_ruoli} Scelta dei ruoli\textsubscript{\textbf{G}} da coprire e frequenza di rotazione.
    \item \label{itm:prossimeattivita} Pianificazione delle prossime attività\textsubscript{\textbf{G}}.
    \begin{itemize}
        \item prosecuzione della redazione di Glossario\textsubscript{\textbf{G}} e Norme di Progetto\textsubscript{\textbf{G}};
        \item creazione di un foglio di rendicontazione ore (preventivate vs. effettive) per ogni membro e ruolo\textsubscript{\textbf{G}};
        \item discussione sulla tempistica del prossimo colloquio con l'azienda e sulle attività\textsubscript{\textbf{G}} preparatorie da svolgere in vista dell'incontro.
        \item avvio dello studio delle tecnologie\textsubscript{\textbf{G}} richieste
    \end{itemize}
\end{enumerate}

\newpage
\section{Svolgimento}
Qui di seguito i punti discussi con più dettaglio:

\subsection*{\ref{itm:lezione}. Allineamento sulla lezione di Ingegneria del Software del giorno corrente}
I membri presenti alla lezione di Ingegneria del Software di data odierna, in sostituzione di chi era assente per il Career Day o altri impegni personali, hanno discusso i contenuti della lezione rovesciata dedicata alla documentazione\textsubscript{\textbf{G}}. In particolare, ci siamo confrontati su glossario\textsubscript{\textbf{G}} e norme di progetto\textsubscript{\textbf{G}}, chiarendo obiettivi e modalità di redazione.

\subsection*{\ref{itm:feedback}. Integrazione dei feedback del docente sulla documentazione\textsubscript{\textbf{G}} di candidatura}
\label{par:ruoli}
Abbiamo analizzato i commenti del docente relativi alla documentazione\textsubscript{\textbf{G}} per la candidatura del progetto\textsubscript{\textbf{G}}, valutando le modifiche da apportare per migliorare la coerenza e la completezza dei documenti.

\begin{adjustwidth}{1cm}{1cm} % per un rientro a sinistra e destra
    \subsubsection*{\ref{itm:tracciabilita}. Definizione della tracciabilità di decisioni e azioni nei verbali}
    \label{par:tracciabilita}
    È stato discusso come inserire nei verbali una sezione dedicata alle decisioni e alle azioni, assicurando che queste risultino tracciabili nel sistema di gestione\textsubscript{\textbf{G}} (definizione di un ID collegato a github\textsubscript{\textbf{G}} issue, persona/e incaricate, descrizione della decisione con relative azioni).\\
    Abbiamo deciso di iniziare la tracciabilità dal verbale corrente in quanto la modifica dei precedenti verbali sarebbe un'attività\textsubscript{\textbf{G}} di poco valore.\\
    Quando nei dettagli del verbale è riportata una discussione su un tema specifico e questa discussione porta a una decisione con conseguente azione, deve essere creato un collegamento tra la discussione e la decisione. In particolare, il collegamento deve essere \textbf{bidirezionale}:

    \begin{itemize}[leftmargin=1cm, rightmargin=1cm]
        \item nella tabella delle decisioni (dove ogni decisione ha un ID) deve esserci un riferimento al paragrafo del verbale che ha portato alla decisione;
        \item nel paragrafo del verbale deve esserci un collegamento alla decisione corrispondente nella tabella.
    \end{itemize}
    \refDecisione{RTB2}{RTB2} %macro definita dentro assets/macro/macro-verbali.tex
    \subsubsection*{\ref{itm:rotazione_ruoli}. Integrazione delle regole di rotazione dei ruoli\textsubscript{\textbf{G}} nel documento\textsubscript{\textbf{G}} di Dichiarazione degli impegni}
    Si è valutato come integrare le regole di rotazione dei ruoli\textsubscript{\textbf{G}} all'interno del documento\textsubscript{\textbf{G}} di dichiarazione degli impegni.

    \subsubsection*{\ref{itm:sito}. Riprogettazione del sito web e della repository\textsubscript{\textbf{G}} GitHub\textsubscript{\textbf{G}}}
    \label{par:sito}
    È stato deciso di avviare una riprogettazione del sito web per migliorarne l'accessibilità, la navigazione e la completezza dei contenuti, in linea con i riscontri ricevuti dal professore. In parallelo, si prevede una ristrutturazione della repository\textsubscript{\textbf{G}} GitHub\textsubscript{\textbf{G}} per garantire maggiore chiarezza e uniformità.\\
    In particolare:
    \begin{itemize}
    \item È stata \textbf{rimossa l'automatizzazione della pubblicazione} di tutti i file presenti nel branch\textsubscript{\textbf{G}} \texttt{main}.
    
    \item Per la \textbf{creazione di un nuovo documento\textsubscript{\textbf{G}}} si decide di:
    \begin{itemize}
        \item creare un branch\textsubscript{\textbf{G}} di tipo \texttt{feature} dedicato allo sviluppo del documento\textsubscript{\textbf{G}};
        \item al termine della stesura, aprire una pull request\textsubscript{\textbf{G}} verso il branch\textsubscript{\textbf{G}} \texttt{dev};
        \item i verificatori\textsubscript{\textbf{G}} hanno il compito di controllare la correttezza e la conformità del documento\textsubscript{\textbf{G}};
        \item una volta approvato, il documento\textsubscript{\textbf{G}} viene fuso nel branch\textsubscript{\textbf{G}} \texttt{main}, segnando la versione \textbf{1.0.0}.
    \end{itemize}
    
    \item La pubblicazione sul sito web avviene solo con i file presenti nel branch\textsubscript{\textbf{G}} \texttt{main}.
    \item Il file README.md viene modificato aggiungendo le istruzioni per la creazione di un documento\textsubscript{\textbf{G}} Latex\textsubscript{\textbf{G}}, la pubblicazione su Github\textsubscript{\textbf{G}} e sul sito.
\end{itemize}

\refDecisione{RTB3}{RTB3} %macro definita dentro assets/macro/macro-verbali.tex


\end{adjustwidth}

\subsection*{\ref{itm:scelta_ruoli} Scelta dei ruoli\textsubscript{\textbf{G}} da coprire e frequenza di rotazione}
Il gruppo ha deciso che ...
Inoltre si è deciso che ogni file (esclusi i verbali) verrà redatto da uno specifico ruolo\textsubscript{\textbf{G}}, a seconda dello scopo del documento\textsubscript{\textbf{G}}. 

\subsection*{\ref{itm:prossimeattivita} Pianificazione delle prossime attività\textsubscript{\textbf{G}}}

\paragraph{Glossario\textsubscript{\textbf{G}}}
\label{par:glossario}
Da iniziare uno studio più approfondito sulla redazione e sull'usabilità del glossario\textsubscript{\textbf{G}}, le domande su cui ci si deve focalizzare sono:
\begin{itemize}[leftmargin=1cm, rightmargin=1cm]
    \item Quali termini deve contenere il glossario\textsubscript{\textbf{G}}? Dove sono contenuti i termini?
    \item A chi è rivolto il glossario\textsubscript{\textbf{G}}? Se a più destinatari, bisogna redigere più glossari\textsubscript{\textbf{G}}? Lo stesso termine può essere ripetuto più volte in glossari\textsubscript{\textbf{G}} diversi con definizioni coerenti al registro e forma più appropriata al destinatario?
    \item Come avviene il collegamento tra termine e glossario\textsubscript{\textbf{G}}? è sufficientemente efficiente ed efficace?
    \item È opportuno inserire a piè di pagina una definizione più riassuntiva del termine ed eventualmente rimandare al glossario\textsubscript{\textbf{G}}?
    \item Se un termine è ripetuto più volte in un documento\textsubscript{\textbf{G}}, come deve avvenire il riferimento al glossario\textsubscript{\textbf{G}}? 
\end{itemize}
Applicazione delle risposte alla versione corrente del glossario\textsubscript{\textbf{G}}.\\

\refDecisione{RTB4}{RTB4} %macro definita dentro assets/macro/macro-verbali.tex

\paragraph{Foglio di calcolo per la rendicontazione ore}
\label{par:esempio}
Idea iniziale su come si può suddividere il foglio di calcolo, in particolare:
\begin{itemize}[leftmargin=1cm, rightmargin=1cm]
    \item Una scheda dedicata per ciascuno sprint\textsubscript{\textbf{G}};
    \item In ogni scheda, una tabella che riporti i membri del gruppo sull'asse delle y e i ruoli\textsubscript{\textbf{G}} sull'asse delle x;
    \item Una colonna finale che indichi le ore rimanenti per ciascun membro rispetto al totale previsto;
    \item Una riga di riepilogo che mostri le ore rimanenti per ciascun ruolo\textsubscript{\textbf{G}} rispetto al monte ore complessivo assegnato;
    \item Una riga aggiuntiva che indichi le ore consigliate per ogni ruolo\textsubscript{\textbf{G}} in ciascuno sprint\textsubscript{\textbf{G}}, accompagnata da un'analisi\textsubscript{\textbf{G}} sul rispetto di tali valori o sugli eventuali sforamenti.
\end{itemize}

\paragraph{Studio tecnologie\textsubscript{\textbf{G}}}
\label{par:studiotec}
È stato inoltre concordato di iniziare una fase preliminare di studio e approfondimento delle tecnologie\textsubscript{\textbf{G}} indicate dall'azienda proponente\textsubscript{\textbf{G}}, con l'obiettivo di acquisire familiarità con gli strumenti e i linguaggi che saranno impiegati nello sviluppo del progetto\textsubscript{\textbf{G}}.\\
A questo proposito abbiamo discusso della necessità di dialogare con l'azienda proponente\textsubscript{\textbf{G}} per assicurarci della corretta decisione di prosecuzione.\\
\refDecisione{RTB7}{RTB7} %macro definita dentro assets/macro/macro-verbali.tex


\paragraph{Prossimo colloquio con la proponente\textsubscript{\textbf{G}}}
Durante la riunione è stato deciso che il prossimo incontro in via telematica con Eggon sarà possibilmente a metà della prossima settimana (10.11 -16.11)
e che in vista di tale incontro il gruppo si impegnerà ad analizzare con molta più attenzione il documento\textsubscript{\textbf{G}} di capitolato\textsubscript{\textbf{G}} per la creazione
di un file \textit{Google docs} con quanto analizzato. Nella riunione si discuterà dei requisiti\textsubscript{\textbf{G}} funzionali\textsubscript{\textbf{G}} e non funzionali\textsubscript{\textbf{G}} dei moduli da integrare nella piattaforma Nexum.



\section{Tabella delle decisioni e azioni}

\setlength{\extrarowheight}{2pt} % padding extra verticale
\renewcommand{\arraystretch}{1.5} 


\arrayrulecolor{primaryblue}
\sloppy
\begin{tabularx}{\textwidth}{
    |>{\raggedright\arraybackslash}p{3cm}|
    >{\raggedright\arraybackslash}X|
    >{\raggedright\arraybackslash}p{3cm}|
}
\hline
\rowcolor{primaryblue!40}
\textbf{\color{white} ID Decisione} & \textbf{\color{white} Descrizione} & \textbf{\color{white} Incaricato} \\
\hline
\rowcolor{secondaryblue!10} RTB1 \label{RTB1} & Integrazione delle regole di rotazione nel documento\textsubscript{\textbf{G}} di dichiarazione degli impegni (vedi \hyperref[par:ruoli]{dettagli}). & Luca Slongo \\
\hline
\rowcolor{secondaryblue!10} RTB2 \label{RTB2} & Redigere il verbale della corrente riunione e definire la sezione decisione~-~azioni tracciabili nel sistema\textsubscript{\textbf{G}} (vedi \hyperref[par:tracciabilita]{dettagli}). & Linor Sadè \\
\hline
\rowcolor{secondaryblue!10} RTB3 \label{RTB3} & Riprogettazione sito web e repository\textsubscript{\textbf{G}} GitHub\textsubscript{\textbf{G}} in linea con il resocontro del professore e regole di accessibilità (vedi \hyperref[par:sito]{dettagli}). & Marco Favero \\
\hline
\rowcolor{secondaryblue!10} RTB4 \label{RTB4} & Studio e continuo della redazione del glossario\textsubscript{\textbf{G}}. (vedi \hyperref[par:glossario]{dettagli}) & Alberto Autiero \\
\hline
\rowcolor{secondaryblue!10} RTB5 \label{RTB5} & Scrittura del foglio di calcolo per la rendicontazione ore, seguendo l'esempio definito (vedi \hyperref[par:esempio]{dettagli}).  & Leonardo Salviato \\
\hline
\rowcolor{secondaryblue!10} RTB6 \label{RTB6} & Approvazione del seguente verbale, scrittura all'azienda, migliorie grafiche alla firma e-mail. & Alberto Pignat \\
\hline
\rowcolor{secondaryblue!10} RTB7 \label{RTB7} & Inizio studio tecnologie\textsubscript{\textbf{G}} (vedi \hyperref[par:studiotec]{dettagli}). & Tutti \\
\hline
\end{tabularx}
\fussy


\vfill
\begin{center}
    {\small\color{darkgray} Documento redatto e approvato dal gruppo BugBusters.}
\end{center}

\end{document}