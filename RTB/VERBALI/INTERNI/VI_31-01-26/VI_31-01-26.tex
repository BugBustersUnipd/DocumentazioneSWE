\documentclass[a4paper,12pt]{article}

\usepackage[utf8]{inputenc}
\usepackage[T1]{fontenc}
\usepackage[italian]{babel}
\usepackage[margin=2.5cm]{geometry}
\usepackage{graphicx}
\usepackage{grffile}
\usepackage{booktabs}
\usepackage{setspace}
\usepackage{titlesec}
\usepackage{float}
\usepackage{ifthen}
\usepackage{tcolorbox}
\usepackage{enumitem}
\usepackage[colorlinks=true,linkcolor=black,urlcolor=primaryblue,citecolor=primaryblue]{hyperref}
\usepackage{changepage} % per definire un rientro (un margine) solo in una parte del documento

\usepackage[table]{xcolor}   % per i colori (celle, testo e righe)
\usepackage{tabularx}               % per tabelle a larghezza adattiva (X)
\usepackage{array}                  % per comandi di formattazione avanzata nelle colonne (p, >{\raggedright})

% Macro
% genera la stringa "\noindent (Riferimento alla tabella decisioni:
% \hyperref[RTB4]{RTB4})", si usa facendo \refDecisione{NomeLabel}{Testo}
\newcommand{\refDecisione}[2]{%
    \noindent (\textbf{Riferimento alla tabella decisioni: \hyperref[#1]{#2}})%
}



\definecolor{primaryblue}{RGB}{0,102,204}
\definecolor{secondaryblue}{RGB}{51,153,255}
\definecolor{lightgray}{RGB}{245,245,245}
\definecolor{darkgray}{RGB}{100,100,100}

\titleformat{\section}
  {\Large\bfseries\color{primaryblue}}
  {\thesection}{1em}{}

\setlength{\parskip}{4pt}
\setlength{\parindent}{0pt}

\setlist[itemize]{leftmargin=*,itemsep=3pt}
\setlist[enumerate]{leftmargin=*,itemsep=3pt}

\graphicspath{{./}{../assets/images/}{./images/}}

\begin{document}

\begin{center}
  \IfFileExists{../../../../assets/Logo.jpg}{%
    \includegraphics[width=6cm,height=3cm,keepaspectratio]{../../../../assets/Logo.jpg} \\[0.8cm]
  }{%
    \fbox{\parbox[c][2.5cm][c]{6cm}{\centering Logo non trovato\\(Logo.jpg)}}\\[0.5cm]
  }
  
  {\Large\bfseries\color{primaryblue} BugBusters}\\[0.3cm]
  {\small\color{darkgray} Email: \texttt{bugbusters.unipd@gmail.com}} \\[0.1cm]
  {\small\color{darkgray} Gruppo: 4} \\[0.5cm]

  {\large\bfseries Università degli Studi di Padova}\\[0.3cm]
  {\small Laurea in Informatica}\\[0.2cm]
  {\small Corso: Ingegneria del Software}\\[0.2cm]
  {\small Anno Accademico: 2025/2026}\\[0.8cm]

  {\Huge\bfseries\color{primaryblue} Verbale Interno\textsubscript{\scalebox{0.6}{\textbf{G}}}}\\[0.3cm]
  {\Large\color{secondaryblue} 31 gennaio 2026}\\[0.8cm]
\end{center}

\begin{center}
\begin{tcolorbox}[colback=lightgray,colframe=primaryblue,width=0.85\textwidth,arc=3mm,boxrule=0.5pt]
\begin{tabular}{@{}ll@{}}
\textbf{Redattore\textsubscript{\scalebox{0.6}{\textbf{G}}}}    & Luca Slongo \\
\textbf{Verificatore\textsubscript{\scalebox{0.6}{\textbf{G}}}}    &  Marco Piro  \\
\textbf{Uso}          & Interno \\
\textbf{Destinatari}  & BugBusters \\
\textbf{Versione} & 1.0.0\\
\end{tabular}
\end{tcolorbox}
\end{center}

\vspace{0.5cm}

\begin{center}
\begin{tcolorbox}[colback=secondaryblue!10,colframe=secondaryblue,width=0.9\textwidth,arc=3mm,boxrule=0.8pt,title={\bfseries Abstract}]
Verbale interno\textsubscript{\scalebox{0.6}{\textbf{G}}} relativo alla riunione del 31 Gennaio 2026, durante la quale sono state decise le prossime attività da svolgere per poter poi candidarsi alla RTB\textsubscript{\scalebox{0.6}{\textbf{G}}}.
\end{tcolorbox}
\end{center}

\newpage

\tableofcontents
\newpage

\section{Informazioni generali}

\begin{itemize}
    \item \textbf{Tipo riunione:} Interna
    \item \textbf{Piattaforma:} Discord\textsubscript{\scalebox{0.6}{\textbf{G}}}
    \item \textbf{Data:} 31/01/2026
    \item \textbf{Orario di inizio:} 15:15
    \item \textbf{Orario di fine:} 15:45
    \item \textbf{Presenti:}
    \begin{itemize}[leftmargin=1.5em, itemsep=3pt, label={\rule[0.5ex]{0.4em}{0.4em}}]
        \item Alberto Autiero
        \item Marco Favero
        \item Alberto Pignat
        \item Marco Piro
        \item Linor Sadè
        \item Leonardo Salviato
        \item Luca Slongo
    \end{itemize}
\end{itemize}

\section{Ordine del giorno}
\begin{enumerate}
    \item \label{itm:revisione_doc} Completamento e revisione finale dei documenti;
    \item \label{itm:verifica_poc} Verifica\textsubscript{\scalebox{0.6}{\textbf{G}}} del PoC\textsubscript{\scalebox{0.6}{\textbf{G}}};
\end{enumerate}


\newpage
\section{Svolgimento}
Qui di seguito i punti discussi con più dettaglio:

\subsection*{\ref{itm:revisione_doc}. Completamento e revisione finale dei documenti} Al fine di candidarsi all RTB\textsubscript{\scalebox{0.6}{\textbf{G}}} il prima possibile, si è discusso brevemente dei documenti da completare e revisionare per poter procedere. 
Si è deciso di creare delle issue\textsubscript{\scalebox{0.6}{\textbf{G}}} specifiche per ogni documento, in modo da poter assegnare le responsabilità e tenere traccia dei progressi.


\subsection*{\ref{itm:verifica_poc}. Verifica\textsubscript{\scalebox{0.6}{\textbf{G}}} del PoC\textsubscript{\scalebox{0.6}{\textbf{G}}}} Prima di procedere con la candidatura\textsubscript{\scalebox{0.6}{\textbf{G}}} alla RTB\textsubscript{\scalebox{0.6}{\textbf{G}}}, si è deciso di effettuare una verifica\textsubscript{\scalebox{0.6}{\textbf{G}}} del PoC\textsubscript{\scalebox{0.6}{\textbf{G}}}, in modo da assicurarsi che sia correttamente funzionante e che il codice scritto sia conforme agli standard qualitativi richiesti.
Tutti i membri del gruppo si occuperanno di questa verifica\textsubscript{\scalebox{0.6}{\textbf{G}}}, ognuno controllerà tutto ciò che riguarda il PoC\textsubscript{\scalebox{0.6}{\textbf{G}}} e non solo le parti specifiche di cui è responsabile\textsubscript{\scalebox{0.6}{\textbf{G}}}, in modo da avere tutti una conoscenza generale del funzionamento totale del sistema.

\section{Tabella delle decisioni e azioni}

\setlength{\extrarowheight}{2pt} % padding extra verticale
\renewcommand{\arraystretch}{1.5} 


\arrayrulecolor{primaryblue}
\sloppy
\begin{tabularx}{\textwidth}{
    |>{\raggedright\arraybackslash}p{3cm}|
    >{\raggedright\arraybackslash}X|
    >{\raggedright\arraybackslash}p{3cm}|
}
\hline
\rowcolor{primaryblue!40}
\textbf{\color{white} ID Decisione} & \textbf{\color{white} Descrizione} & \textbf{\color{white} Incaricato} \\

\hline
\rowcolor{secondaryblue!10} \href{https://github.com/BugBustersUnipd/DocumentazioneSWE/issues/150}{RTB83}  & Verificare Analisi dei Requisiti & Marco Piro \\
\hline
\rowcolor{secondaryblue!10} \href{https://github.com/BugBustersUnipd/DocumentazioneSWE/issues/167}{RTB84} & Verificare se la guida POC\textsubscript{\scalebox{0.6}{\textbf{G}}} va bene su Linux e MACOS & Marco Piro, Luca Slongo \\
\hline
\rowcolor{secondaryblue!10} \href{https://github.com/BugBustersUnipd/DocumentazioneSWE/issues/152}{RTB85} & Verifica\textsubscript{\scalebox{0.6}{\textbf{G}}} Norme di Progetto\textsubscript{\scalebox{0.6}{\textbf{G}}} & Alberto Pignat \\
\hline
\rowcolor{secondaryblue!10} \href{https://github.com/BugBustersUnipd/DocumentazioneSWE/issues/153}{RTB86}  & Verifica\textsubscript{\scalebox{0.6}{\textbf{G}}} Piano di Progetto\textsubscript{\scalebox{0.6}{\textbf{G}}} & Marco Favero \\
\hline
\rowcolor{secondaryblue!10} \href{https://github.com/BugBustersUnipd/DocumentazioneSWE/issues/154}{RTB87}  & Verifica\textsubscript{\scalebox{0.6}{\textbf{G}}} Piano di Qualifica\textsubscript{\scalebox{0.6}{\textbf{G}}} & Marco Favero \\
\hline
\rowcolor{secondaryblue!10} \href{https://github.com/BugBustersUnipd/DocumentazioneSWE/issues/155}{RTB88}  & Verifica\textsubscript{\scalebox{0.6}{\textbf{G}}} Glossario\textsubscript{\scalebox{0.6}{\textbf{G}}} & Alberto Pignat \\
\hline
\rowcolor{secondaryblue!10} \href{https://github.com/BugBustersUnipd/DocumentazioneSWE/issues/156}{RTB89}  & Aggiornare il Glossario\textsubscript{\scalebox{0.6}{\textbf{G}}} e i vari documenti & Alberto Autiero \\
\hline
\rowcolor{secondaryblue!10} \href{https://github.com/BugBustersUnipd/DocumentazioneSWE/issues/164}{RTB90}  & Aggiungere i documenti al sito ed aggiornare i documenti con i link
& Linor Sadè \\
\hline

\end{tabularx}
\fussy

\vfill
\begin{center}
    {\small\color{darkgray} Documento redatto e approvato dal gruppo BugBusters.}
\end{center}

\end{document}