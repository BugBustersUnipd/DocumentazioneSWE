\documentclass[a4paper,12pt]{article}

\usepackage[utf8]{inputenc}
\usepackage[T1]{fontenc}
\usepackage[italian]{babel}
\usepackage[margin=2.5cm]{geometry}
\usepackage{graphicx}
\usepackage{grffile}
\usepackage{booktabs}
\usepackage{setspace}
\usepackage{titlesec}
\usepackage{float}
\usepackage{ifthen}
\usepackage{tcolorbox}
\usepackage{enumitem}
\usepackage[colorlinks=true,linkcolor=black,urlcolor=primaryblue,citecolor=primaryblue]{hyperref}
\usepackage{changepage} % per definire un rientro (un margine) solo in una parte del documento

\usepackage[table]{xcolor}   % per i colori (celle, testo e righe)
\usepackage{tabularx}               % per tabelle a larghezza adattiva (X)
\usepackage{array}                  % per comandi di formattazione avanzata nelle colonne (p, >{\raggedright})

% Macro
% genera la stringa "\noindent (Riferimento alla tabella decisioni:
% \hyperref[RTB4]{RTB4})", si usa facendo \refDecisione{NomeLabel}{Testo}
\newcommand{\refDecisione}[2]{%
    \noindent (\textbf{Riferimento alla tabella decisioni: \hyperref[#1]{#2}})%
}



\definecolor{primaryblue}{RGB}{0,102,204}
\definecolor{secondaryblue}{RGB}{51,153,255}
\definecolor{lightgray}{RGB}{245,245,245}
\definecolor{darkgray}{RGB}{100,100,100}

\titleformat{\section}
  {\Large\bfseries\color{primaryblue}}
  {\thesection}{1em}{}

\setlength{\parskip}{4pt}
\setlength{\parindent}{0pt}

\setlist[itemize]{leftmargin=*,itemsep=3pt}
\setlist[enumerate]{leftmargin=*,itemsep=3pt}

\graphicspath{{./}{../assets/images/}{./images/}}

\begin{document}

\begin{center}  \IfFileExists{../../../../assets/Logo.jpg}{%
    \includegraphics[width=6cm,height=3cm,keepaspectratio]{../../../../assets/Logo.jpg} \\[0.8cm]
  }{%
    \fbox{\parbox[c][2.5cm][c]{6cm}{\centering Logo non trovato\\(Logo.jpg)}}\\[0.5cm]
  }
  
  {\Large\bfseries\color{primaryblue} BugBusters}\\[0.3cm]
  {\small\color{darkgray} Email: \texttt{bugbusters.unipd@gmail.com}} \\[0.1cm]
  {\small\color{darkgray} Gruppo: 4} \\[0.5cm]

  {\large\bfseries Università degli Studi di Padova}\\[0.3cm]
  {\small Laurea in Informatica}\\[0.2cm]
  {\small Corso: Ingegneria del Software}\\[0.2cm]
  {\small Anno Accademico: 2025/2026}\\[0.8cm]

  {\Huge\bfseries\color{primaryblue} Verbale Interno}\\[0.3cm]
  {\Large\color{secondaryblue} 1 Dicembre 2025}\\[0.8cm]
\end{center}

\begin{center}
\begin{tcolorbox}[colback=lightgray,colframe=primaryblue,width=0.85\textwidth,arc=3mm,boxrule=0.5pt]
\begin{tabular}{@{}ll@{}}
\textbf{Redattori}    & Marco Favero \\
\textbf{Verificatore}    & Linor Sadè \\
\textbf{Uso}          & Interno \\
\textbf{Destinatari}  & BugBusters \\
\textbf{Versione} & 1.0.0\\

\end{tabular}
\end{tcolorbox}
\end{center}

\vspace{0.5cm}

\begin{center}
\begin{tcolorbox}[colback=secondaryblue!10,colframe=secondaryblue,width=0.9\textwidth,arc=3mm,boxrule=0.8pt,title={\bfseries Abstract}]
Verbale dell'incontro interno del gruppo BugBusters.
Sono stati analizzati i casi d'uso e il mockup dell'applicazione, proponendo modifiche 
e nuove funzionalità; è stato preparato un documento Google condiviso per raccogliere domande da 
porre a Nexum in sede di colloquio. Si è deciso di automatizzare la pubblicazione dei verbali sul 
sito e di adottare workflow GitHub per chiudere automaticamente le issue tramite commit. 
È stato inoltre fatto un allineamento sullo stato delle conoscenze tecniche del gruppo. Sono 
state definite decisioni operative e assegnati compiti (RTB17-RTB25) con responsabili specifici. 
\end{tcolorbox}
\end{center}

\newpage

\tableofcontents
\newpage

\section{Informazioni generali}

\begin{itemize}
    \item \textbf{Tipo riunione:} interna
    \item \textbf{Piattaforma:} Discord
    \item \textbf{Data:} 01/12/2025
    \item \textbf{Orario di inizio:} 14:30
    \item \textbf{Orario di fine:} 16:30
    \item \textbf{Presenti:}
    \begin{itemize}[leftmargin=1.5em, itemsep=3pt, label={\rule[0.5ex]{0.4em}{0.4em}}]
        \item Alberto Autiero
        \item Marco Favero
        \item Alberto Pignat
        \item Marco Piro
        \item Linor Sadè
        \item Leonardo Salviato
        \item Luca Slongo
    \end{itemize}
\end{itemize}

\section{Ordine del giorno}

\begin{enumerate}
    \item \label{itm:casi d'uso} Discussione su use cases.
    \item \label{itm:mockup} Discussione su mock up della applicazione e funzionalità principali.
    \item \label{itm:colloquio} Preparazione per colloquio con Nexum, raccolta domande e dubbi.
    \item \label{itm:automazioni} Automazioni per aggiornare il sito, per chiudere automaticamente le issue nei commit.
    \item \label{itm:allineamento} Allineamento sullo stato di avanzamento delle conoscenze delle 
    tecnologie richieste. 
\end{enumerate}

\newpage
\section{Svolgimento}
\subsection*{\ref{itm:casi d'uso}. Discussione sui casi d'uso}
I membri del gruppo hanno discusso sui casi d'uso individuati fino ad ora, apportando alcune modifiche e 
suggerendo l'aggiunta di nuovi casi d'uso per coprire meglio le funzionalità richieste dal progetto.

\subsection*{\ref{itm:mockup}. Discussione su mock up della applicazione e funzionalità principali}
È stato visionato un mockup grafico dell'applicazione e sono state discusse le funzionalità principali 
che dovranno essere implementate. Questo mockup aiuterà a visualizzare meglio l'interfaccia utente e a 
pianificare lo sviluppo. Non è assolutamente definitivo e potrà essere modificato in futuro, è utile 
per avere un'idea di massima di come potrebbe essere l'applicazione e dunque ridurre l'astrazione durante l'analisi.

\subsection*{\ref{itm:colloquio}. Preparazione per colloquio con Nexum}
Sono state raccolte domande e dubbi da porre a Nexum durante il colloquio, in modo da chiarire eventuali
incertezze e ottenere maggiori dettagli sul progetto. Si è discusso anche di come strutturare il colloquio
stesso per massimizzare l'efficacia della comunicazione con Nexum. 
è stato creato un documento su Google Docs condiviso tra i membri del gruppo per raccogliere
tutte le domande e i dubbi da porre a Nexum.

\subsection*{\ref{itm:automazioni}. Automazioni per aggiornare il sito, per chiudere automaticamente le issue nei commit}
Si è deciso che i verbali li si vuole più automatici possibili, dunque verranno caricati automaticamente sul sito. \\
Inoltre si è deciso di utilizzare le automazioni di GitHub per chiudere automaticamente le issue quando
si effettuano commit che le risolvono, possibilmente usando parole chiave specifiche nei messaggi di commit.


\subsection*{\ref{itm:allineamento}. Allineamento sullo stato di avanzamento delle conoscenze delle tecnologie richieste}
Abbiamo discusso del nostro stato di avanzamento delle conoscenze delle tecnologie richieste per il progetto, utile per 
condividere spunti e risorse tra i membri del gruppo. 

\section{Tabella delle decisioni e azioni}

\setlength{\extrarowheight}{2pt} % padding extra verticale
\renewcommand{\arraystretch}{1.5} 

\arrayrulecolor{primaryblue}
\sloppy
\begin{tabularx}{\textwidth}{
    |>{\raggedright\arraybackslash}p{3cm}|
    >{\raggedright\arraybackslash}X|
    >{\raggedright\arraybackslash}p{3cm}|
}
\hline
\rowcolor{primaryblue!40}
\textbf{\color{white} ID Decisione} & \textbf{\color{white} Descrizione} & \textbf{\color{white} Incaricato} \\
\hline
\DecisionRow{\href{https://github.com/BugBustersUnipd/DocumentazioneSWE/issues/19}{RTB17} \label{RTB17}}{Definire domande tramite google docs condiviso da porre alla proponente.}{Tutti}
\DecisionRow{\href{https://github.com/BugBustersUnipd/DocumentazioneSWE/issues/20}{RTB18} \label{RTB18}}{Informarsi e implementare automazioni per aggiornare il sito web e per chiudere automaticamente le issue nei commit.}{Marco Favero}
\DecisionRow{\href{https://github.com/BugBustersUnipd/DocumentazioneSWE/issues/21}{RTB19} \label{RTB19}}{Avanzamento Analisi dei Requisiti.}{Leonardo Salviato, Marco Piro}
\DecisionRow{\href{https://github.com/BugBustersUnipd/DocumentazioneSWE/issues/22}{RTB20} \label{RTB20}}{Avanzamento Norme.}{Marco Favero, Albero Pignat}
\DecisionRow{\href{https://github.com/BugBustersUnipd/DocumentazioneSWE/issues/23}{RTB21} \label{RTB21}}{Informarsi riguardo al Piano di Qualifica.}{Luca Slongo}
\DecisionRow{\href{https://github.com/BugBustersUnipd/DocumentazioneSWE/issues/24}{RTB22} \label{RTB22}}{Informarsi riguardo al Piano di Progetto, eventualmente scrivere bozza.}{Alberto Autiero}
\DecisionRow{\href{https://github.com/BugBustersUnipd/DocumentazioneSWE/issues/25}{RTB23} \label{RTB23}}{Verificare questo verbale}{Linor Sadè}
\DecisionRow{\href{https://github.com/BugBustersUnipd/DocumentazioneSWE/issues/26}{RTB24} \label{RTB24}}{Verificare Norme}{Linor Sadè}
\DecisionRow{\href{https://github.com/BugBustersUnipd/DocumentazioneSWE/issues/27}{RTB25} \label{RTB25}}{Verificare Analisi dei Requisiti}{Linor Sadè}
\end{tabularx}
\fussy


\vfill
\begin{center}
    {\small\color{darkgray} Documento redatto e approvato dal gruppo BugBusters.}
\end{center}

\end{document}
