\documentclass[a4paper,12pt]{article}

\usepackage[utf8]{inputenc}
\usepackage[T1]{fontenc}
\usepackage[italian]{babel}
\usepackage[margin=2.5cm]{geometry}
\usepackage{graphicx}
\usepackage{grffile}
\usepackage{booktabs}
\usepackage{setspace}
\usepackage{titlesec}
\usepackage{float}
\usepackage{ifthen}
\usepackage{tcolorbox}
\usepackage{enumitem}
\usepackage[colorlinks=true,linkcolor=black,urlcolor=primaryblue,citecolor=primaryblue]{hyperref}
\usepackage{changepage} % per definire un rientro (un margine) solo in una parte del documento

\usepackage[table]{xcolor}   % per i colori (celle, testo e righe)
\usepackage{tabularx}               % per tabelle a larghezza adattiva (X)
\usepackage{array}                  % per comandi di formattazione avanzata nelle colonne (p, >{\raggedright})

% Macro
% genera la stringa "\noindent (Riferimento alla tabella decisioni:
% \hyperref[RTB4]{RTB4})", si usa facendo \refDecisione{NomeLabel}{Testo}
\newcommand{\refDecisione}[2]{%
    \noindent (\textbf{Riferimento alla tabella decisioni: \hyperref[#1]{#2}})%
}



\definecolor{primaryblue}{RGB}{0,102,204}
\definecolor{secondaryblue}{RGB}{51,153,255}
\definecolor{lightgray}{RGB}{245,245,245}
\definecolor{darkgray}{RGB}{100,100,100}

\titleformat{\section}
  {\Large\bfseries\color{primaryblue}}
  {\thesection}{1em}{}

\setlength{\parskip}{4pt}
\setlength{\parindent}{0pt}

\setlist[itemize]{leftmargin=*,itemsep=3pt}
\setlist[enumerate]{leftmargin=*,itemsep=3pt}

\graphicspath{{./}{../assets/images/}{./images/}}

\begin{document}

\begin{center}
  \IfFileExists{../../../../assets/Logo.jpg}{%
    \includegraphics[width=6cm,height=3cm,keepaspectratio]{../../../../assets/Logo.jpg} \\[0.8cm]
  }{%
    \fbox{\parbox[c][2.5cm][c]{6cm}{\centering Logo non trovato\\(Logo.jpg)}}\\[0.5cm]
  }
  
  {\Large\bfseries\color{primaryblue} BugBusters}\\[0.3cm]
  {\small\color{darkgray} Email: \texttt{bugbusters.unipd@gmail.com}} \\[0.1cm]
  {\small\color{darkgray} Gruppo: 4} \\[0.5cm]

  {\large\bfseries Università degli Studi di Padova}\\[0.3cm]
  {\small Laurea in Informatica}\\[0.2cm]
  {\small Corso: Ingegneria del Software}\\[0.2cm]
  {\small Anno Accademico: 2025/2026}\\[0.8cm]

  {\Huge\bfseries\color{primaryblue} Verbale Interno\textsubscript{\scalebox{0.6}{\textbf{G}}}}\\[0.3cm]
  {\Large\color{secondaryblue} 12 gennaio 2026}\\[0.8cm]
\end{center}

\begin{center}
\begin{tcolorbox}[colback=lightgray,colframe=primaryblue,width=0.85\textwidth,arc=3mm,boxrule=0.5pt]
\begin{tabular}{@{}ll@{}}
\textbf{Redattore\textsubscript{\scalebox{0.6}{\textbf{G}}}} & Linor Sadè \\
\textbf{Verificatore\textsubscript{\scalebox{0.6}{\textbf{G}}}} & Marco Favero \\
\textbf{Uso} & Interno \\
\textbf{Destinatari} & BugBusters \\
\textbf{Versione} & 1.0.0\\
\end{tabular}
\end{tcolorbox}
\end{center}

\vspace{0.5cm}

\begin{center}
\begin{tcolorbox}[colback=secondaryblue!10,colframe=secondaryblue,width=0.9\textwidth,arc=3mm,boxrule=0.8pt,title={\bfseries Abstract}]
    Verbale interno\textsubscript{\scalebox{0.6}{\textbf{G}}} della riunione del 12 gennaio 2026, principalmente a seguito del ricevimento con il professor Cardin, per discutere delle modifiche necessarie nell'Analisi dei requisiti\textsubscript{\scalebox{0.6}{\textbf{G}}} e per pianificare le attività per la candidatura\textsubscript{\scalebox{0.6}{\textbf{G}}} all'RTB\textsubscript{\scalebox{0.6}{\textbf{G}}}.
\end{tcolorbox}
\end{center}

\newpage

\tableofcontents
\newpage

\section{Informazioni generali}

\begin{itemize}
    \item \textbf{Tipo riunione:} Interna
    \item \textbf{Piattaforma:} Discord\textsubscript{\scalebox{0.6}{\textbf{G}}}
    \item \textbf{Data:} 12/01/2026
    \item \textbf{Orario di inizio:} 9:30
    \item \textbf{Orario di fine:} 10:30
    \item \textbf{Presenti:}
    \begin{itemize}[leftmargin=1.5em, itemsep=3pt, label={\rule[0.5ex]{0.4em}{0.4em}}]
        \item Alberto Autiero
        \item Marco Favero
        \item Alberto Pignat
        \item Marco Piro
        \item Linor Sadè
        \item Leonardo Salviato
        \item Luca Slongo
    \end{itemize}
\end{itemize}

\section{Ordine del giorno}
\begin{enumerate}
    \item \label{itm:modifiche-analisi-requisiti} Discussione di modifiche sostanziali nell'Analisi dei requisiti\textsubscript{\scalebox{0.6}{\textbf{G}}};
    \item \label{itm:prossimeattivitaPOC} Discussione di eventuali modifiche nel POC\textsubscript{\scalebox{0.6}{\textbf{G}}}:
    \begin{enumerate}
        \item \label{itm:backend} Aggiunta funzionalità\textsubscript{\scalebox{0.6}{\textbf{G}}} guardrails nel backend nel POC\textsubscript{\scalebox{0.6}{\textbf{G}}};
        \item \label{itm:frontend} Refactoring del codice frontend del POC\textsubscript{\scalebox{0.6}{\textbf{G}}};
    \end{enumerate}
    \item \label{itm:pdq-ndp} Allineamento sullo stato di avanzamento del Piano di Qualifica\textsubscript{\scalebox{0.6}{\textbf{G}}} e Norme di Progetto\textsubscript{\scalebox{0.6}{\textbf{G}}};
    \item \label{itm:autiero} Modifiche al Piano di Progetto\textsubscript{\scalebox{0.6}{\textbf{G}}};
\end{enumerate}


\newpage
\section{Svolgimento}
Qui di seguito i punti discussi con più dettaglio:

\subsection*{\ref{itm:modifiche-analisi-requisiti}. Discussione di modifiche sostanziali nell'Analisi dei requisiti\textsubscript{\scalebox{0.6}{\textbf{G}}}}
A seguito di un ricevimento organizzato con il professor Cardin, il gruppo necessita di porre alcuni cambiamenti nell'Analisi dei requisiti\textsubscript{\scalebox{0.6}{\textbf{G}}}. In particolare, gli errori riguardano principalmente l'uso di extend e include e la suddivisione di casi d'uso\textsubscript{\scalebox{0.6}{\textbf{G}}} che non rispettano la divisione logica delle funzionalità\textsubscript{\scalebox{0.6}{\textbf{G}}}.
\subsection*{\ref{itm:prossimeattivitaPOC}. Discussione di eventuali modifiche nel POC\textsubscript{\scalebox{0.6}{\textbf{G}}}}
In risposta al feedback ricevuto dal professor Cardin, che ha chiarito dubbi sul ruolo del POC\textsubscript{\scalebox{0.6}{\textbf{G}}}, il gruppo ha deciso che quest'ultimo è quasi terminato e necessita solo di alcune modifiche minori.\\
Tali modifiche sono state discusse e pianificate nei seguenti punti:
\begin{adjustwidth}{1cm}{1cm} % per un rientro a sinistra e destra
    \subsubsection*{\ref{itm:backend} Aggiunta funzionalità\textsubscript{\scalebox{0.6}{\textbf{G}}} guardrails nel backend del POC\textsubscript{\scalebox{0.6}{\textbf{G}}}}
    Rispetto a quanto riportato anche nel verbale\textsubscript{\scalebox{0.6}{\textbf{G}}} della riunione esterna con Eggon datata 07/01/2026 (verbale\textsubscript{\scalebox{0.6}{\textbf{G}}} reperibile al seguente \href{https://bugbustersunipd.github.io/DocumentazioneSWE/RTB/VERBALI/ESTERNI/VE_07-01-26/VE_07-01-26.pdf}{link}), il team ha deciso di migliorare l'AI\textsubscript{\scalebox{0.6}{\textbf{G}}} Assistant Generativo aggiungendo guardrails per evitare risposte inappropriate. Si prevede di completare l'attività entro il 14/01/2026.
    \subsubsection*{\ref{itm:frontend} Refactoring del codice frontend del POC\textsubscript{\scalebox{0.6}{\textbf{G}}}}
    Il team ha deciso di effettuare un refactoring del codice frontend del POC\textsubscript{\scalebox{0.6}{\textbf{G}}} per migliorarne la leggibilità. È stato inoltre valutato l'utilizzo del pattern \textit{Observer}, tuttavia si è deciso di non implementarlo in questa fase in quanto non rientra negli obiettivi di un POC\textsubscript{\scalebox{0.6}{\textbf{G}}}.
    Si prevede di completare l'attività entro il 14/01/2026.
    \end{adjustwidth}
\subsection*{\ref{itm:pdq-ndp}. Allineamento sullo stato di avanzamento del Piano di Qualifica\textsubscript{\scalebox{0.6}{\textbf{G}}} e Norme di Progetto\textsubscript{\scalebox{0.6}{\textbf{G}}}}
Il gruppo si è allineato sullo stato di avanzamento del Piano di Qualifica\textsubscript{\scalebox{0.6}{\textbf{G}}} e delle Norme di Progetto\textsubscript{\scalebox{0.6}{\textbf{G}}} in quanto l'avanzamento di quest'ultima richiede un buon stato di avanzamento del Piano di Qualifica\textsubscript{\scalebox{0.6}{\textbf{G}}}. Per le Norme di Progetto\textsubscript{\scalebox{0.6}{\textbf{G}}} il gruppo prevede il loro completamento entro il 13/01/2026.
\subsection*{\ref{itm:autiero}. Modifiche al Piano di Progetto\textsubscript{\scalebox{0.6}{\textbf{G}}}}
Il Piano di Progetto\textsubscript{\scalebox{0.6}{\textbf{G}}} necessita di alcune piccole modifiche grafiche, quali la ristrutturazione della tabella delle attività dell'RTB\textsubscript{\scalebox{0.6}{\textbf{G}}} e la modifica della tabella del registro delle modifiche per renderla coerente con quella presente negli altri documenti. \\
Inoltre, è stato deciso di inserire un diagramma di Gantt per rappresentare visivamente le tempistiche delle attività pianificate.
\setlength{\extrarowheight}{2pt} % padding extra verticale
\section{Tabella delle decisioni e azioni}
\renewcommand{\arraystretch}{1.5} 


\arrayrulecolor{primaryblue}
\sloppy
\begin{tabularx}{\textwidth}{
    |>{\raggedright\arraybackslash}p{3cm}|
    >{\raggedright\arraybackslash}X|
    >{\raggedright\arraybackslash}p{3cm}|
}
\hline
\rowcolor{primaryblue!40}
\textbf{\color{white} ID Decisione} & \textbf{\color{white} Descrizione} & \textbf{\color{white} Incaricato} \\
\hline
\rowcolor{secondaryblue!10} \href{https://github.com/BugBustersUnipd/DocumentazioneSWE/issues/107}{RTB68} & Modifiche all'Analisi dei requisiti\textsubscript{\scalebox{0.6}{\textbf{G}}} & Leonardo Salviato \\
\hline
\rowcolor{secondaryblue!10} \href{https://github.com/BugBustersUnipd/DocumentazioneSWE/issues/108}{RTB69}  & Aggiunta guardrails nel backend del POC\textsubscript{\scalebox{0.6}{\textbf{G}}} & Luca Slongo \\
\hline
\rowcolor{secondaryblue!10} \href{https://github.com/BugBustersUnipd/DocumentazioneSWE/issues/109}{RTB70}  & Refactoring del codice frontend del POC\textsubscript{\scalebox{0.6}{\textbf{G}}} & Alberto Pignat, Marco Piro, Linor Sadè \\
\hline
\rowcolor{secondaryblue!10} \href{https://github.com/BugBustersUnipd/DocumentazioneSWE/issues/110}{RTB71}  & Avanzamento Piano di Qualifica\textsubscript{\scalebox{0.6}{\textbf{G}}} & Marco Piro \\
\hline
\rowcolor{secondaryblue!10} \href{https://github.com/BugBustersUnipd/DocumentazioneSWE/issues/111}{RTB72}  & Avanzamento Norme di Progetto\textsubscript{\scalebox{0.6}{\textbf{G}}} (rispetto PdQ) & Linor Sadè \\
\hline
\rowcolor{secondaryblue!10} \href{https://github.com/BugBustersUnipd/DocumentazioneSWE/issues/112}{RTB73}  & Modifiche (anche estetiche) al Piano di Progetto\textsubscript{\scalebox{0.6}{\textbf{G}}} & Alberto Autiero \\
\hline
\rowcolor{secondaryblue!10} \href{https://github.com/BugBustersUnipd/DocumentazioneSWE/issues/113}{RTB74}  & Ricerca informazioni presentazione RTB\textsubscript{\scalebox{0.6}{\textbf{G}}} & Alberto Autiero \\
\hline
\rowcolor{secondaryblue!10} \href{https://github.com/BugBustersUnipd/DocumentazioneSWE/issues/114}{RTB75}  & Redazione verbale\textsubscript{\scalebox{0.6}{\textbf{G}}} corrente & Linor Sadè \\
\hline
\rowcolor{secondaryblue!10} \href{https://github.com/BugBustersUnipd/DocumentazioneSWE/issues/115}{RTB76}  & Verifica\textsubscript{\scalebox{0.6}{\textbf{G}}} verbale\textsubscript{\scalebox{0.6}{\textbf{G}}} corrente & Marco Favero \\
\hline
\end{tabularx}
\fussy

\vfill
\begin{center}
    {\small\color{darkgray} Documento redatto e approvato dal gruppo BugBusters.}
\end{center}

\end{document}