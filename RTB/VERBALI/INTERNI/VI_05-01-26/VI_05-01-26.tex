\documentclass[a4paper,12pt]{article}

\usepackage[utf8]{inputenc}
\usepackage[T1]{fontenc}
\usepackage[italian]{babel}
\usepackage[margin=2.5cm]{geometry}
\usepackage{graphicx}
\usepackage{grffile}
\usepackage{booktabs}
\usepackage{setspace}
\usepackage{titlesec}
\usepackage{float}
\usepackage{ifthen}
\usepackage{tcolorbox}
\usepackage{enumitem}
\usepackage[colorlinks=true,linkcolor=black,urlcolor=primaryblue,citecolor=primaryblue]{hyperref}
\usepackage{changepage} % per definire un rientro (un margine) solo in una parte del documento

\usepackage[table]{xcolor}   % per i colori (celle, testo e righe)
\usepackage{tabularx}               % per tabelle a larghezza adattiva (X)
\usepackage{array}                  % per comandi di formattazione avanzata nelle colonne (p, >{\raggedright})

% Macro
% genera la stringa "\noindent (Riferimento alla tabella decisioni:
% \hyperref[RTB4]{RTB4})", si usa facendo \refDecisione{NomeLabel}{Testo}
\newcommand{\refDecisione}[2]{%
    \noindent (\textbf{Riferimento alla tabella decisioni: \hyperref[#1]{#2}})%
}



\definecolor{primaryblue}{RGB}{0,102,204}
\definecolor{secondaryblue}{RGB}{51,153,255}
\definecolor{lightgray}{RGB}{245,245,245}
\definecolor{darkgray}{RGB}{100,100,100}

\titleformat{\section}
  {\Large\bfseries\color{primaryblue}}
  {\thesection}{1em}{}

\setlength{\parskip}{4pt}
\setlength{\parindent}{0pt}

\setlist[itemize]{leftmargin=*,itemsep=3pt}
\setlist[enumerate]{leftmargin=*,itemsep=3pt}

\graphicspath{{./}{../assets/images/}{./images/}}

\begin{document}

\begin{center}
  \IfFileExists{../../../../assets/Logo.jpg}{%
    \includegraphics[width=6cm,height=3cm,keepaspectratio]{../../../../assets/Logo.jpg} \\[0.8cm]
  }{%
    \fbox{\parbox[c][2.5cm][c]{6cm}{\centering Logo non trovato\\(Logo.jpg)}}\\[0.5cm]
  }
  
  {\Large\bfseries\color{primaryblue} BugBusters}\\[0.3cm]
  {\small\color{darkgray} Email: \texttt{bugbusters.unipd@gmail.com}} \\[0.1cm]
  {\small\color{darkgray} Gruppo: 4} \\[0.5cm]

  {\large\bfseries Università degli Studi di Padova}\\[0.3cm]
  {\small Laurea in Informatica}\\[0.2cm]
  {\small Corso: Ingegneria del Software}\\[0.2cm]
  {\small Anno Accademico: 2025/2026}\\[0.8cm]

  {\Huge\bfseries\color{primaryblue} Verbale Interno\textsubscript{\scalebox{0.6}{\textbf{G}}}}\\[0.3cm]
  {\Large\color{secondaryblue} 5 Gennaio 2026}\\[0.8cm]
\end{center}

\begin{center}
\begin{tcolorbox}[colback=lightgray,colframe=primaryblue,width=0.85\textwidth,arc=3mm,boxrule=0.5pt]
\begin{tabular}{@{}ll@{}}
\textbf{Redattori\textsubscript{\scalebox{0.6}{\textbf{G}}}}    & Linor Sadè \\
\textbf{Verificatore\textsubscript{\scalebox{0.6}{\textbf{G}}}}    & Marco Favero  \\
\textbf{Uso}          & Interno \\
\textbf{Destinatari}  & BugBusters \\
\textbf{Versione} & 0.0.1\\
\end{tabular}
\end{tcolorbox}
\end{center}

\vspace{0.5cm}

\begin{center}
\begin{tcolorbox}[colback=secondaryblue!10,colframe=secondaryblue,width=0.9\textwidth,arc=3mm,boxrule=0.8pt,title={\bfseries Abstract}]
Verbale interno\textsubscript{\scalebox{0.6}{\textbf{G}}} relativo alla riunione del 5 Gennaio 2026, durante la quale sono stati discussi i punti principali per il completamento del POC\textsubscript{\scalebox{0.6}{\textbf{G}}} in vista dell'incontro con l'azienda proponente\textsubscript{\scalebox{0.6}{\textbf{G}}} Eggon, previsto per il 7 Gennaio 2026.
\end{tcolorbox}
\end{center}

\newpage

\tableofcontents
\newpage

\section{Informazioni generali}

\begin{itemize}
    \item \textbf{Tipo riunione:} Interna
    \item \textbf{Piattaforma:} Discord\textsubscript{\scalebox{0.6}{\textbf{G}}}
    \item \textbf{Data:} 05/01/2026
    \item \textbf{Orario di inizio:} 10:30
    \item \textbf{Orario di fine:} 12:10
    \item \textbf{Presenti:}
    \begin{itemize}[leftmargin=1.5em, itemsep=3pt, label={\rule[0.5ex]{0.4em}{0.4em}}]
        \item Alberto Autiero
        \item Marco Favero
        \item Alberto Pignat
        \item Marco Piro
        \item Linor Sadè
        \item Leonardo Salviato
        \item Luca Slongo
    \end{itemize}
\end{itemize}

\section{Ordine del giorno}
\begin{enumerate}
    \item \label{itm:rotazione_ruoli} Comunicazione di inizio sprint\textsubscript{\scalebox{0.6}{\textbf{G}}} e della rotazione dei ruoli. Breve discussione dello sprint\textsubscript{\scalebox{0.6}{\textbf{G}}} precedente;
    \item \label{itm:pianificazione-incontro} Pianificazione dell'incontro con l'azienda proponente\textsubscript{\scalebox{0.6}{\textbf{G}}} in vista del colloquio \textbf{in presenza} in data 07/01/2026;
    \item \label{itm:feedback} Discussione su ultimi commit eseguiti nella repository\textsubscript{\scalebox{0.6}{\textbf{G}}} del POC\textsubscript{\scalebox{0.6}{\textbf{G}}} ed eventuali migliorie;
    \item \label{itm:prossimeattivita} Discussione approfondita sulla prima stesura bozza del POC\textsubscript{\scalebox{0.6}{\textbf{G}}}, che deve essere terminata entro il 07/01/2026.
    \begin{enumerate}
        \item \label{itm:frontend} frontend\textsubscript{\scalebox{0.6}{\textbf{G}}};
        \item \label{itm:backend} backend\textsubscript{\scalebox{0.6}{\textbf{G}}}.
    \end{enumerate}
    \item \label{itm:scrittura_diario_di_bordo} Inizio stesura dell'ultimo diario di bordo: discussione su difficoltà nell'inizio o avanzamento attività già iniziate;
\end{enumerate}


\newpage
\section{Svolgimento}
Qui di seguito i punti discussi con più dettaglio:

\subsection*{\ref{itm:rotazione_ruoli}. Comunicazione di inizio sprint\textsubscript{\scalebox{0.6}{\textbf{G}}} e della rotazione dei ruoli. Breve discussione dello sprint\textsubscript{\scalebox{0.6}{\textbf{G}}} precedente}
Si è discusso brevemente dello sprint\textsubscript{\scalebox{0.6}{\textbf{G}}} precedente, evidenziando il rallentamento subito dal periodo festivo e gli impegni che hanno influito sull'avanzamento del POC, rispetto ciò che era stato pianificato.\\
Questo nuovo sprint\textsubscript{\scalebox{0.6}{\textbf{G}}} sarà cruciale nel raggiungere gli obiettivi di consegna del POC\textsubscript{\scalebox{0.6}{\textbf{G}}}, sia per l'incontro con Eggon, sia per la candidatura\textsubscript{\scalebox{0.6}{\textbf{G}}} all'RTB\textsubscript{\scalebox{0.6}{\textbf{G}}}.
Abbiamo quindi concordato di concentrarci, nei restanti 2 giorni, sul POC\textsubscript{\scalebox{0.6}{\textbf{G}}} e nel terminare l'analisi dei requisiti\textsubscript{\scalebox{0.6}{\textbf{G}}}.
\subsection*{\ref{itm:pianificazione-incontro}. Pianificazione dell'incontro con l'azienda proponente\textsubscript{\scalebox{0.6}{\textbf{G}}} in vista del colloquio \textbf{in presenza} in data 07/01/2026}
L'incontro con eggon è già stato pianificato per il 07/01/2026 alle ore 16:00, presso la sede di Eggon a Padova.\\
Si è discusso brevemente degli argomenti da trattare durante l'incontro, con particolare attenzione alla presentazione del POC\textsubscript{\scalebox{0.6}{\textbf{G}}} (sperabilmente completo) e alla raccolta di feedback sull'analisi dei requisiti\textsubscript{\scalebox{0.6}{\textbf{G}}} (precedentemente già discussa ma in forma non ufficiale).\\
Qui di seguito punti o domande specifiche da non dimenticare di porre:
\begin{itemize}
    \item Richiesta di abilitazione bucket S3 per il modulo\textsubscript{\scalebox{0.6}{\textbf{G}}} AI-Copilot\textsubscript{\scalebox{0.6}{\textbf{G}}} per salvataggio dei file separando i tenant (al momento della presentazione del POC\textsubscript{\scalebox{0.6}{\textbf{G}}} ad Eggon non era ancora stata effettuata ma vorremmo che sia pronta per l'RTB\textsubscript{\scalebox{0.6}{\textbf{G}}});
    \item Accordarci sulle "conversazioni" per le modifiche del prompt\textsubscript{\scalebox{0.6}{\textbf{G}}} nel modulo\textsubscript{\scalebox{0.6}{\textbf{G}}} Ai Assistant\textsubscript{\scalebox{0.6}{\textbf{G}}} Generativo: le lasciamo nel POC\textsubscript{\scalebox{0.6}{\textbf{G}}} ma da analisi è prevista la modifica del prompt\textsubscript{\scalebox{0.6}{\textbf{G}}}, non un'iterazione conversazionale;  
\end{itemize}
\subsection*{\ref{itm:feedback}. Discussione su ultimi commit eseguiti nella repository\textsubscript{\scalebox{0.6}{\textbf{G}}} del POC\textsubscript{\scalebox{0.6}{\textbf{G}}} ed eventuali migliorie}
\begin{itemize}
    \item Eliminare test\textsubscript{\scalebox{0.6}{\textbf{G}}} non necessari e file inutili dalla repository\textsubscript{\scalebox{0.6}{\textbf{G}}} del POC\textsubscript{\scalebox{0.6}{\textbf{G}}}.
    \item Inserire al file env le credenziali al database, che al momento sono hardcodate nel file di connessione e vengono cambiate manualmente dal singolo sviluppatore.
    \item Buona l'idea di far in modo che la configurazione del progetto\textsubscript{\scalebox{0.6}{\textbf{G}}} sia il più possibile automatica e non richieda di avviare frontend\textsubscript{\scalebox{0.6}{\textbf{G}}} e backend\textsubscript{\scalebox{0.6}{\textbf{G}}} separatamente (file di configurazione già iniziati)
\end{itemize}
\subsection*{\ref{itm:prossimeattivita}. Discussione approfondita sulla prima stesura bozza del POC\textsubscript{\scalebox{0.6}{\textbf{G}}}, che deve essere terminata entro il 07/01/2026.}
A causa della scarsità di tempo a disposizione si è deciso che ogni membro si occuperà delle prossime attività, mantenendo il più possibile la divisione del ruoli dello sprint\textsubscript{\scalebox{0.6}{\textbf{G}}}.
Il POC\textsubscript{\scalebox{0.6}{\textbf{G}}} deve essere completato entro il 07/01/2026, data dell'incontro con Eggon.\\ Al momento della riunione il POC comprendeva: buon backend\textsubscript{\scalebox{0.6}{\textbf{G}}} con API\textsubscript{\scalebox{0.6}{\textbf{G}}} funzionanti per l'Ai assistant generativo ma senza la generazione di immagini, frontend\textsubscript{\scalebox{0.6}{\textbf{G}}} con inizio di interfaccia\textsubscript{\scalebox{0.6}{\textbf{G}}} troppo avanzata per le funzionalità\textsubscript{\scalebox{0.6}{\textbf{G}}} discusse per il POC\textsubscript{\scalebox{0.6}{\textbf{G}}}, collegamento tra frontend\textsubscript{\scalebox{0.6}{\textbf{G}}} e backend\textsubscript{\scalebox{0.6}{\textbf{G}}}.
Per dialogare efficacemente sulle scelte da fare sono stati condivisi in diretta da un membro alcuni mockup delle schermate principali del POC.\\
Abbiamo discusso anche della necessità di implementare (frontend\textsubscript{\scalebox{0.6}{\textbf{G}}} e backend\textsubscript{\scalebox{0.6}{\textbf{G}}}) la selezione dell'azienda tenant, necessaria principalmente per la selezione di toni (e stile) che si riferiscono ad una specifica azienda.
Inoltre la funzionalità di split e data analysis non saranno implementati nel POC\textsubscript{\scalebox{0.6}{\textbf{G}}} (verranno discussi con Eggon per capire se sono necessari per l'RTB\textsubscript{\scalebox{0.6}{\textbf{G}}}). 
\begin{adjustwidth}{1cm}{1cm} % per un rientro a sinistra e destra
    \subsubsection*{\ref{itm:backend} Backend\textsubscript{\scalebox{0.6}{\textbf{G}}}}
    Si è discusso delle funzionalità\textsubscript{\scalebox{0.6}{\textbf{G}}} mancanti del backend\textsubscript{\scalebox{0.6}{\textbf{G}}} e di come implementarle velocemente entro la scadenza del 07/01/2026.\\
    In particolare:
    \paragraph{Ai Assistant Generativo}
    \begin{itemize}
        \item controllare che esista una funzione che raccoglie tutte le richieste fatte all'Ai (per lo storico);
        \item aggiungere la funzione per restituire la singola richiesta fatta all'Ai (per lo storico);
        \item la generazione di immagini è importante per l'incontro con Eggon, è necessario capire quali sono i modelli che ci sono resi disponibili per questa funzionalità\textsubscript{\scalebox{0.6}{\textbf{G}}} e quali sono le loro performance. Eventualmente chiedere ad Eggon;
        \item la rigenerazione del prompt\textsubscript{\scalebox{0.6}{\textbf{G}}} in deve tenere conto delle conversazioni precedenti e riproporre un risultato basandosi sull'ultimo prompt\textsubscript{\scalebox{0.6}{\textbf{G}}} inserito (al momento non è possibile rigenerare senza aggiungere una conversazione);
        \item eventualmente implementare una funzione per scartare l'intera conversazione (eliminazione dallo storico);
    \end{itemize}
    \paragraph{Ai Co-pilot\textsubscript{\scalebox{0.6}{\textbf{G}}}}
    \begin{itemize}
        \item Ricercare un modello OCR\textsubscript{\scalebox{0.6}{\textbf{G}}} valido perché quello attuale non funziona bene (chiedere eventualmente ad Eggon);
        \item Correggere il problema del checksum: deve riconoscere che il documento e già stato caricato e deve ricaricare lo stesso;
        \item opzionalmente informarsi su Sidekiq per la gestione di code di processi asincroni (come l'OCR\textsubscript{\scalebox{0.6}{\textbf{G}}});
        \item rivedere come definire il json di output dell'Ai: quali sono le informazioni che devono essere restituite al frontend\textsubscript{\scalebox{0.6}{\textbf{G}}}.
        \item controllare che esista una funzione che raccoglie i risultati dei documenti caricati (per lo storico), eventualimente aggiungerla;
        \item Ricercare uso di Amazon Comprehend per l'analisi del testo estratto (chiedere eventualmente ad Eggon);
    \end{itemize}
    \subsubsection*{\ref{itm:frontend} frontend\textsubscript{\scalebox{0.6}{\textbf{G}}}}
    Si è discusso delle funzionalità\textsubscript{\scalebox{0.6}{\textbf{G}}} mancanti del frontend\textsubscript{\scalebox{0.6}{\textbf{G}}} e di come implementarle velocemente entro la scadenza del 07/01/2026.\\
    In particolare:
    \begin{itemize}
        \item implementare lo storico delle richieste fatte all'AI Assistant Generativo;
        \item inserire una tendina nella schermata di Ai Assistant Generativo per la selezione dell'azienda tenant (necessaria per raccolta dei toni presenti);
        \item implementare la ricerca nello storico secondo risultato, prompt\textsubscript{\scalebox{0.6}{\textbf{G}}}, tono e azienda;
        \item togliere il pulsante di "stile" che di fatto è implementato esattamente come "tono" quindi non ci serve a dimostrare nulla in più;
        \item togliere la valutazione del risultato;
        \item il salvataggio è automatico, non serve un pulsante apposito, aggiungere tuttavia un pulsante per uscire;
        \item togliere il pulsante di modifica e quindi aggiungere le conversazioni per la modifica del prompt\textsubscript{\scalebox{0.6}{\textbf{G}}}: il rigenera è già implementato tenendo conto delle conversazioni precedenti; 
        \item aggiungere l'immagine generata al risultato dell'AI Assistant Generativo; 
        \item togliere i pulsanti nella schermata di anteprima risultato dell'AI Co-pilot\textsubscript{\scalebox{0.6}{\textbf{G}}} (rigenera, modifica e salva);
    \end{itemize}
\end{adjustwidth}


\subsection*{\ref{itm:scrittura_diario_di_bordo}. Inizio stesura dell'ultimo diario di bordo: discussione su difficoltà nell'inizio o avanzamento attività già iniziate}
È stata iniziata una bozza del diario di bordo, aggiungendo la slide per le previsioni di candidatura\textsubscript{\scalebox{0.6}{\textbf{G}}}, richieste dal professore nella mail odierna.\\
Il gruppo ha deciso di aggiornarsi su nuove difficoltà o dubbi che verranno eventualmente riscontrati nei prossimi due giorni, in vista della riunione con Eggon e il completamento del POC\textsubscript{\scalebox{0.6}{\textbf{G}}}.


\section{Tabella delle decisioni e azioni}

\setlength{\extrarowheight}{2pt} % padding extra verticale
\renewcommand{\arraystretch}{1.5} 


\arrayrulecolor{primaryblue}
\sloppy
\begin{tabularx}{\textwidth}{
    |>{\raggedright\arraybackslash}p{3cm}|
    >{\raggedright\arraybackslash}X|
    >{\raggedright\arraybackslash}p{3cm}|
}
\hline
\rowcolor{primaryblue!40}
\textbf{\color{white} ID Decisione} & \textbf{\color{white} Descrizione} & \textbf{\color{white} Incaricato} \\
\hline
\rowcolor{secondaryblue!10} \href{https://github.com/BugBustersUnipd/DocumentazioneSWE/issues/80}{RTB46-49} & Completamento delle attività backend\textsubscript{\scalebox{0.6}{\textbf{G}}} per Ai Assistant\textsubscript{\scalebox{0.6}{\textbf{G}}} Generativo& Marco Favero \\
\hline
\rowcolor{secondaryblue!10} \href{https://github.com/BugBustersUnipd/DocumentazioneSWE/issues/80}{RTB50-53}  & Completamento delle attività backend\textsubscript{\scalebox{0.6}{\textbf{G}}} per Ai Co-pilot\textsubscript{\scalebox{0.6}{\textbf{G}}} per Cdl & Luca Slongo \\
\hline
\rowcolor{secondaryblue!10} \href{https://github.com/BugBustersUnipd/DocumentazioneSWE/issues/81}{RTB54-62}  & Completamento delle attività frontend\textsubscript{\scalebox{0.6}{\textbf{G}}} per Ai\textsubscript{\scalebox{0.6}{\textbf{G}}} Assitant Generativo e Ai Co-pilot\textsubscript{\scalebox{0.6}{\textbf{G}}} per Cdl & Alberto Pignat, Marco Piro, Linor Sadè\\
\hline
\rowcolor{secondaryblue!10} \href{https://github.com/BugBustersUnipd/DocumentazioneSWE/issues/99}{RTB63}& Ricerca e costruzione di documenti per testare AI Co-pilot\textsubscript{\scalebox{0.6}{\textbf{G}}} per Cdl & Alberto Autiero \\
\hline
\rowcolor{secondaryblue!10} \href{https://github.com/BugBustersUnipd/DocumentazioneSWE/issues/100}{RTB64} & Redazione verbale\textsubscript{\scalebox{0.6}{\textbf{G}}} corrente & Linor Sadè \\
\hline
\rowcolor{secondaryblue!10} \href{https://github.com/BugBustersUnipd/DocumentazioneSWE/issues/101}{RTB65}  & Verifica\textsubscript{\scalebox{0.6}{\textbf{G}}} del verbale\textsubscript{\scalebox{0.6}{\textbf{G}}} corrente & Marco Favero \\
\hline
\rowcolor{secondaryblue!10} \href{https://github.com/BugBustersUnipd/DocumentazioneSWE/issues/103}{RTB66}  & Terminare analisi dei requisiti\textsubscript{\scalebox{0.6}{\textbf{G}}} & Leonardo Salviato \\
\hline
\end{tabularx}
\fussy

\vfill
\begin{center}
    {\small\color{darkgray} Documento redatto e approvato dal gruppo BugBusters.}
\end{center}

\end{document}