\documentclass[a4paper,12pt]{article}

\usepackage[utf8]{inputenc}
\usepackage[T1]{fontenc}
\usepackage[italian]{babel}
\usepackage[margin=2.5cm]{geometry}
\usepackage{graphicx}
\usepackage{grffile}
\usepackage{booktabs}
\usepackage{setspace}
\usepackage{titlesec}
\usepackage{float}
\usepackage{ifthen}
\usepackage{tcolorbox}
\usepackage{enumitem}
\usepackage[colorlinks=true,linkcolor=black,urlcolor=primaryblue,citecolor=primaryblue]{hyperref}
\usepackage{changepage} % per definire un rientro (un margine) solo in una parte del documento

\usepackage[table]{xcolor}   % per i colori (celle, testo e righe)
\usepackage{tabularx}               % per tabelle a larghezza adattiva (X)
\usepackage{array}                  % per comandi di formattazione avanzata nelle colonne (p, >{\raggedright})

% Macro
% genera la stringa "\noindent (Riferimento alla tabella decisioni:
% \hyperref[RTB4]{RTB4})", si usa facendo \refDecisione{NomeLabel}{Testo}
\newcommand{\refDecisione}[2]{%
    \noindent (\textbf{Riferimento alla tabella decisioni: \hyperref[#1]{#2}})%
}



\definecolor{primaryblue}{RGB}{0,102,204}
\definecolor{secondaryblue}{RGB}{51,153,255}
\definecolor{lightgray}{RGB}{245,245,245}
\definecolor{darkgray}{RGB}{100,100,100}

\titleformat{\section}
  {\Large\bfseries\color{primaryblue}}
  {\thesection}{1em}{}

\setlength{\parskip}{4pt}
\setlength{\parindent}{0pt}

\setlist[itemize]{leftmargin=*,itemsep=3pt}
\setlist[enumerate]{leftmargin=*,itemsep=3pt}

\graphicspath{{./}{../assets/images/}{./images/}}

\begin{document}

\begin{center}  \IfFileExists{../../../../assets/Logo.jpg}{%
    \includegraphics[width=6cm,height=3cm,keepaspectratio]{../../../../assets/Logo.jpg} \\[0.8cm]
  }{%
    \fbox{\parbox[c][2.5cm][c]{6cm}{\centering Logo non trovato\\(Logo.jpg)}}\\[0.5cm]
  }
  
  {\Large\bfseries\color{primaryblue} BugBusters}\\[0.3cm]
  {\small\color{darkgray} Email: \texttt{bugbusters.unipd@gmail.com}} \\[0.1cm]
  {\small\color{darkgray} Gruppo: 4} \\[0.5cm]

  {\large\bfseries Università degli Studi di Padova}\\[0.3cm]
  {\small Laurea in Informatica}\\[0.2cm]
  {\small Corso: Ingegneria del Software}\\[0.2cm]
  {\small Anno Accademico: 2025/2026}\\[0.8cm]

  {\Huge\bfseries\color{primaryblue} Verbale Interno}\\[0.3cm]
  {\Large\color{secondaryblue} 17 novembre 2025}\\[0.8cm]
\end{center}

\begin{center}
\begin{tcolorbox}[colback=lightgray,colframe=primaryblue,width=0.85\textwidth,arc=3mm,boxrule=0.5pt]
\begin{tabular}{@{}ll@{}}
\textbf{Redattori}    & Luca Slongo \\
\textbf{Verificatore}    & Marco Piro \\
\textbf{Uso}          & Interno \\
\textbf{Destinatari}  & BugBusters \\
\textbf{Versione} & 1.0.0 \\

\end{tabular}
\end{tcolorbox}
\end{center}

\vspace{0.5cm}

\begin{center}
\begin{tcolorbox}[colback=secondaryblue!10,colframe=secondaryblue,width=0.9\textwidth,arc=3mm,boxrule=0.8pt,title={\bfseries Abstract}]
L'incontro ha avuto come oggetto principale la preparazione del colloquio con Eggon in programma per il 19/11, nonché la definizione delle prossime tappe relative alla documentazione e all'approfondimento tecnologico.\end{tcolorbox}
\end{center}

\newpage

\tableofcontents
\newpage

\section{Informazioni generali}

\begin{itemize}
    \item \textbf{Tipo riunione:} Interna
    \item \textbf{Piattaforma:} Discord\textsubscript{\textbf{G}}
    \item \textbf{Data:} 17/11/2025
    \item \textbf{Orario di inizio:} 14:30
    \item \textbf{Orario di fine:} 16:30
    \item \textbf{Presenti:}
    \begin{itemize}[leftmargin=1.5em, itemsep=3pt, label={\rule[0.5ex]{0.4em}{0.4em}}]
        \item Alberto Autiero
        \item Marco Favero
        \item Alberto Pignat
        \item Marco Piro
        \item Linor Sadè
        \item Leonardo Salviato
        \item Luca Slongo
    \end{itemize}
\end{itemize}

\section{Ordine del giorno}

\begin{enumerate}
    \item \label{itm:diario} Discussione dei commenti sul diario di bordo
    \item \label{itm:colloquio} Preparazione per colloquio con Eggon
    \item \label{itm:ore} Discussione sulla gestione delle ore tra attività preparatorie e lavoro da rendicontare.
    \item \label{itm:email} Invio email a Nexum

\end{enumerate}

\newpage
\section{Svolgimento}
\subsection*{\ref{itm:diario}. Discussione dei commenti sul diario di bordo}
I membri del team hanno discusso i commenti ricevuti durante il diario di bordo del 17 novembre e hanno definito le modalità di proseguimento del progetto\textsubscript{\textbf{G}}.
\subsection*{\ref{itm:colloquio}. Preparazione per colloquio con Nexum}
Sono stati definiti i prossimi obiettivi: continuare principalmente con le Norme di Progetto\textsubscript{\textbf{G}} e in via informale anche l'Analisi dei Requisiti\textsubscript{\textbf{G}}, studio individuale delle tecnologie richieste dal capitolato\textsubscript{\textbf{G}}.

\subsection*{\ref{itm:ore}. Discussione gestione delle ore tra palestra e da segnare}
Non è stata ancora raggiunta una soluzione definitiva sulla rendicontazione delle ore lavorative.

\subsection*{\ref{itm:email}. Invio email a Nexum}
In preparazione del colloquio con Eggon, è stata inviata loro un'email per condividere in anticipo l'ordine del giorno.

\section{Tabella delle decisioni e azioni}

\setlength{\extrarowheight}{2pt} % padding extra verticale
\renewcommand{\arraystretch}{1.5} 

\arrayrulecolor{primaryblue}
\sloppy
\begin{tabularx}{\textwidth}{
    |>{\raggedright\arraybackslash}p{3cm}|
    >{\raggedright\arraybackslash}X|
    >{\raggedright\arraybackslash}p{3cm}|
}
\hline
\rowcolor{primaryblue!40}
\textbf{\color{white} ID Decisione} & \textbf{\color{white} Descrizione} & \textbf{\color{white} Incaricato} \\
\hline
\rowcolor{secondaryblue!10} RTB13 \label{RTB13} & Continuare lo studio individuale delle tecnologie richieste & Tutti \\
\hline
\rowcolor{secondaryblue!10} RTB14 \label{RTB14} & Approfondire e proseguire la redazione dei documenti "Norme di Progetto" e "Analisi dei Requisiti"\textsubscript{\textbf{G}} & Tutti \\
\hline
\rowcolor{secondaryblue!10} RTB15 \label{RTB15} & Nei prossimi verbali esterni verrà inserita la tabella decisioni/azioni & Tutti \\
\hline
\rowcolor{secondaryblue!10} RTB16 \label{RTB16} & Sviluppare il progetto\textsubscript{\textbf{G}} come un'applicazione stand-alone, indipendente da quella già esistente di Nexum. & Tutti \\
\hline
\end{tabularx}
\fussy


\vfill
\begin{center}
    {\small\color{darkgray} Documento redatto e approvato dal gruppo BugBusters.}
\end{center}

\end{document}
