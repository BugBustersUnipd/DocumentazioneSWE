\documentclass[a4paper,12pt]{article}

\usepackage[utf8]{inputenc}
\usepackage[T1]{fontenc}
\usepackage[italian]{babel}
\usepackage[margin=2.5cm]{geometry}
\usepackage{graphicx}
\usepackage{grffile}
\usepackage{booktabs}
\usepackage{setspace}
\usepackage{titlesec}
\usepackage{float}
\usepackage{ifthen}
\usepackage{tcolorbox}
\usepackage{enumitem}
\usepackage[colorlinks=true,linkcolor=black,urlcolor=primaryblue,citecolor=primaryblue]{hyperref}
\usepackage{changepage} % per definire un rientro (un margine) solo in una parte del documento

\usepackage[table]{xcolor}   % per i colori (celle, testo e righe) 
\usepackage{tabularx}               % per tabelle a larghezza adattiva (X)
\usepackage{array}                  % per comandi di formattazione avanzata nelle colonne (p, >{\raggedright})

% Macro
% genera la stringa "\noindent (Riferimento alla tabella decisioni:
% \hyperref[RTB4]{RTB4})", si usa facendo \refDecisione{NomeLabel}{Testo}
\newcommand{\refDecisione}[2]{%
    \noindent (\textbf{Riferimento alla tabella decisioni: \hyperref[#1]{#2}})%
}



\definecolor{primaryblue}{RGB}{0,102,204}
\definecolor{secondaryblue}{RGB}{51,153,255}
\definecolor{lightgray}{RGB}{245,245,245}
\definecolor{darkgray}{RGB}{100,100,100}

\titleformat{\section}
  {\Large\bfseries\color{primaryblue}}
  {\thesection}{1em}{}

\setlength{\parskip}{4pt}
\setlength{\parindent}{0pt}

\setlist[itemize]{leftmargin=*,itemsep=3pt}
\setlist[enumerate]{leftmargin=*,itemsep=3pt}

\graphicspath{{./}{../assets/images/}{./images/}}

\begin{document}

\begin{center}
  \IfFileExists{../../../../assets/Logo.jpg}{%
    \includegraphics[width=6cm,height=3cm,keepaspectratio]{../../../../assets/Logo.jpg} \\[0.8cm]
  }{%
    \fbox{\parbox[c][2.5cm][c]{6cm}{\centering Logo non trovato\\(Logo.jpg)}}\\[0.5cm]
  }
  
  {\Large\bfseries\color{primaryblue} BugBusters}\\[0.3cm]
  {\small\color{darkgray} Email: \texttt{bugbusters.unipd@gmail.com}} \\[0.1cm]
  {\small\color{darkgray} Gruppo: 4} \\[0.5cm]

  {\large\bfseries Università degli Studi di Padova}\\[0.3cm]
  {\small Laurea in Informatica}\\[0.2cm]
  {\small Corso: Ingegneria del Software}\\[0.2cm]
  {\small Anno Accademico: 2025/2026}\\[0.8cm]

  {\Huge\bfseries\color{primaryblue} Verbale Interno\textsubscript{\scalebox{0.6}{\textbf{G}}}}\\[0.3cm]
  {\Large\color{secondaryblue} 15 gennaio 2026}\\[0.8cm]
\end{center}

\begin{center}
\begin{tcolorbox}[colback=lightgray,colframe=primaryblue,width=0.85\textwidth,arc=3mm,boxrule=0.5pt]
\begin{tabular}{@{}ll@{}}
\textbf{Redattore\textsubscript{\scalebox{0.6}{\textbf{G}}}} & Linor Sadè \\
\textbf{Verificatore\textsubscript{\scalebox{0.6}{\textbf{G}}}} & Marco Favero \\
\textbf{Uso} & Interno \\
\textbf{Destinatari} & BugBusters \\
\textbf{Versione} & 1.0.0\\
\end{tabular}
\end{tcolorbox}
\end{center}

\vspace{0.5cm}

\begin{center}
\begin{tcolorbox}[colback=secondaryblue!10,colframe=secondaryblue,width=0.9\textwidth,arc=3mm,boxrule=0.8pt,title={\bfseries Abstract}]
    Verbale\textsubscript{\scalebox{0.6}{\textbf{G}}} della riunione interna del gruppo BugBusters tenutasi il 15 gennaio 2026 tramite Discord\textsubscript{\scalebox{0.6}{\textbf{G}}}. Nella riunione è stato discusso il materiale per la presentazione dell'RTB\textsubscript{\scalebox{0.6}{\textbf{G}}}, una retrospettiva\textsubscript{\scalebox{0.6}{\textbf{G}}} sullo sprint\textsubscript{\scalebox{0.6}{\textbf{G}}} appena concluso, le ultime modifiche al POC\textsubscript{\scalebox{0.6}{\textbf{G}}} e le prossime attività da svolgere. Infine è stato fatto il punto della situazione sulle Norme di Progetto\textsubscript{\scalebox{0.6}{\textbf{G}}}.
\end{tcolorbox}
\end{center}

\newpage

\tableofcontents
\newpage

\section{Informazioni generali}

\begin{itemize}
    \item \textbf{Tipo riunione:} Interna
    \item \textbf{Piattaforma:} Discord\textsubscript{\scalebox{0.6}{\textbf{G}}}
    \item \textbf{Data:} 15/01/2026
    \item \textbf{Orario di inizio:} 10:30
    \item \textbf{Orario di fine:} 11:10
    \item \textbf{Presenti:}
    \begin{itemize}[leftmargin=1.5em, itemsep=3pt, label={\rule[0.5ex]{0.4em}{0.4em}}]
        \item Alberto Autiero
        \item Marco Favero
        \item Alberto Pignat
        \item Marco Piro
        \item Linor Sadè
        \item Leonardo Salviato
        \item Luca Slongo
    \end{itemize}
\end{itemize}

\section{Ordine del giorno}
\begin{enumerate}
    \item\label{itm:presentazioneRTB} Presentazione delle informazioni raccolte per la presentazione dell'RTB\textsubscript{\scalebox{0.6}{\textbf{G}}};
    \item\label{itm:sprint-retrospective} Discussione di quanto emerso in questo sprint\textsubscript{\scalebox{0.6}{\textbf{G}}} (retrospective);
    \item\label{itm:poc-novita} Discussione ultime modifiche al POC\textsubscript{\scalebox{0.6}{\textbf{G}}}
    \begin{enumerate}
        \item\label{itm:backend-novita} Backend\textsubscript{\scalebox{0.6}{\textbf{G}}}: aggiunta funzionalità\textsubscript{\scalebox{0.6}{\textbf{G}}} guardrails nel backend\textsubscript{\scalebox{0.6}{\textbf{G}}} del POC\textsubscript{\scalebox{0.6}{\textbf{G}}} e miglioramenti vari;
        \item\label{itm:frontend-novita} Frontend\textsubscript{\scalebox{0.6}{\textbf{G}}}: aggiunte immagini nello storico e miglioramenti vari
    \end{enumerate}
    \item\label{itm:prossimeattivitaPOC} Discussione di prossime modifiche nel POC\textsubscript{\scalebox{0.6}{\textbf{G}}}:
    \begin{enumerate}
        \item\label{itm:backend} Backend\textsubscript{\scalebox{0.6}{\textbf{G}}}: aggiungere API\textsubscript{\scalebox{0.6}{\textbf{G}}} per recupero delle analisi OCR\textsubscript{\scalebox{0.6}{\textbf{G}}} precedenti, modifica alla logica delle conversazioni;
        \item\label{itm:frontend} Frontend\textsubscript{\scalebox{0.6}{\textbf{G}}}: aggiungere storico per l'AI Co-pilot\textsubscript{\scalebox{0.6}{\textbf{G}}} per CdL\textsubscript{\scalebox{0.6}{\textbf{G}}};
        \item\label{itm:ricerca} Backend\textsubscript{\scalebox{0.6}{\textbf{G}}}: ricerca del cost–benefit ratio di modelli AI\textsubscript{\scalebox{0.6}{\textbf{G}}} più performanti rispetto quello attuale;
    \end{enumerate}
    \item\label{itm:ndp} Discussione sullo stato di avanzamento delle Norme di Progetto\textsubscript{\scalebox{0.6}{\textbf{G}}} e ultime sezioni da completare;
\end{enumerate}


\newpage
\section{Svolgimento}
Qui di seguito i punti discussi con più dettaglio:

\subsection*{\ref{itm:presentazioneRTB}. Presentazione delle informazioni raccolte per la presentazione del RTB\textsubscript{\scalebox{0.6}{\textbf{G}}}}
Alberto Autiero ha presentato le informazioni raccolte per la presentazione dell'RTB\textsubscript{\scalebox{0.6}{\textbf{G}}} e delle prime bozze delle slide. I PowerPoint e i pdf delle slide sono consultabili (e modificabili) nel branch \textit{Presentazione-RTB} e \textit{Tecnologie}.
Tutti i membri del gruppo sono invitati a contribuire alla stesura delle slide.
\subsection*{\ref{itm:sprint-retrospective}. Discussione di quanto emerso in questo sprint\textsubscript{\scalebox{0.6}{\textbf{G}}} (retrospective)}
Il team ha discusso di quanto emerso in questo sprint\textsubscript{\scalebox{0.6}{\textbf{G}}}, evidenziando le ore effettivamente lavorate rispetto a quelle pianificate e i ruoli effettivamente ricoperti rispetto a quelli pianificati.\\
Come emerso dalla precedente riunione interna (reperibile al seguente \href{https://bugbustersunipd.github.io/DocumentazioneSWE/RTB/VERBALI/INTERNI/VI_12-01-26/VI_12-01-26.pdf}{link}) è stato deciso di apportare alcune modifiche sostanziali all'Analisi dei Requisiti\textsubscript{\scalebox{0.6}{\textbf{G}}}. Tuttavia questo ha influito nella suddivisione dei ruoli e delle ore lavorate: in particolare Leonardo Salviato ha dovuto concentrare i suoi sforzi nell'Analisi dei Requisitianalisi dei requisiti e per sopperire alla mancanza del ruolo a lui pianificato, altri membri del team hanno dovuto dedicare parte delle proprie ore a svolgere tali compiti.\\
Quanto discusso verrà riportato nel documento di Piano di Progetto\textsubscript{\scalebox{0.6}{\textbf{G}}}, nella sezione del corrente sprint\textsubscript{\scalebox{0.6}{\textbf{G}}} (5).
\subsection*{\ref{itm:poc-novita}. Discussione ultime modifiche al POC\textsubscript{\scalebox{0.6}{\textbf{G}}}}
\begin{adjustwidth}{1cm}{1cm} % per un rientro a sinistra e destra
    \subsubsection*{\ref{itm:backend-novita} Backend: aggiunta funzionalità\textsubscript{\scalebox{0.6}{\textbf{G}}} guardrails nel backend del POC\textsubscript{\scalebox{0.6}{\textbf{G}}} e miglioramenti vari}
    È stata aggiunta la funzionalità\textsubscript{\scalebox{0.6}{\textbf{G}}} di guardrails nel backend\textsubscript{\scalebox{0.6}{\textbf{G}}} del POC\textsubscript{\scalebox{0.6}{\textbf{G}}}, che permette di limitare le risposte dell'AI\textsubscript{\scalebox{0.6}{\textbf{G}}} Assistant Generativo nella generazione di testo in base a regole predefinite. In particolare, come spiegato da Luca Slongo, sono stati implementati dei filtri in input, per evitare che l'utente inserisca richieste inappropriate o non conformi alle linee guida stabilite.\\
    Tale funzionalità\textsubscript{\scalebox{0.6}{\textbf{G}}} tuttavia, non permette di limitare le risposte sulla base del contesto specifico dell'azienda, ma solo in base a regole generali che specificano cosa non deve essere detto.\\
    Inoltre è presente anche un filtro in output (quindi lato cloud) per evitare che l'AI\textsubscript{\scalebox{0.6}{\textbf{G}}} generi risposte inappropriate.\\
    \subsubsection*{\ref{itm:frontend-novita} Frontend: aggiunte immagini nello storico e miglioramenti vari}
    Sono state esposte le ultime modifiche al frontend\textsubscript{\scalebox{0.6}{\textbf{G}}} del POC\textsubscript{\scalebox{0.6}{\textbf{G}}}, in particolare l'aggiunta delle immagini nello storico dei risultati dell'AI\textsubscript{\scalebox{0.6}{\textbf{G}}} Assistant Generativo, qualora l'utente abbia generato un'immagine in una determinata conversazione.
    Sono stati mostrati alcuni miglioramenti estetici e di usabilità dell'interfaccia\textsubscript{\scalebox{0.6}{\textbf{G}}}, soprattutto nel modulo\textsubscript{\scalebox{0.6}{\textbf{G}}} di AI Co-pilot\textsubscript{\scalebox{0.6}{\textbf{G}}} per CdL\textsubscript{\scalebox{0.6}{\textbf{G}}}. Infine è stata aggiunta la gestione di un messaggio di errore nell'inserimento di un documento già analizzato.
\end{adjustwidth}
\subsection*{\ref{itm:prossimeattivitaPOC} Discussione di prossime modifiche nel POC\textsubscript{\scalebox{0.6}{\textbf{G}}}}
\begin{adjustwidth}{1cm}{1cm} % per un rientro a sinistra e destra
    \subsubsection*{\ref{itm:backend} Backend: aggiungere API\textsubscript{\scalebox{0.6}{\textbf{G}}} per recupero delle analisi OCR\textsubscript{\scalebox{0.6}{\textbf{G}}} precedenti e modifica alla logica delle conversazioni}
    È stata discussa l'aggiunta di una nuova API\textsubscript{\scalebox{0.6}{\textbf{G}}} per il recupero delle analisi OCR precedenti effettuate dall'utente, in modo da poterle visualizzare nello storico delle analisi. \\
    Inoltre è stata discussa una possibile modifica alla logica delle conversazioni dell'AI\textsubscript{\scalebox{0.6}{\textbf{G}}} Assistant Generativo per non dover inviare sempre il tono di risposta desiderato ad ogni richiesta, ma solo all'inizio della conversazione.
    \subsubsection*{\ref{itm:frontend} Frontend: aggiungere storico per l'AI Co-pilot\textsubscript{\scalebox{0.6}{\textbf{G}}} per CdL\textsubscript{\scalebox{0.6}{\textbf{G}}}}
    In linea con la necessità di visualizzare lo storico delle analisi OCR\textsubscript{\scalebox{0.6}{\textbf{G}}} precedenti e la modifica che verrà fatta nel backend, la modifica verrà rispecchiata anche nel frontend.\\
    Infine dovrà essere modificata la logica delle richieste inviate all'AI\textsubscript{\scalebox{0.6}{\textbf{G}}} Assistant Generativo per rispecchiare la modifica discussa nel backend.
    \subsubsection*{\ref{itm:ricerca} Backend: ricerca del cost–benefit ratio di modelli AI\textsubscript{\scalebox{0.6}{\textbf{G}}} più performanti rispetto quello attuale}
    Marco Favero ha proposto di effettuare una ricerca sul cost–benefit ratio di modelli AI\textsubscript{\scalebox{0.6}{\textbf{G}}} più performanti rispetto a quello attualmente utilizzato (\textit{Nova lite v1}), in modo da valutare se l'adozione di un modello più avanzato possa portare benefici significativi in termini di qualità\textsubscript{\scalebox{0.6}{\textbf{G}}} delle risposte generate dall'AI\textsubscript{\scalebox{0.6}{\textbf{G}}} Assistant Generativo, dal momento che ancora il modello attuale presenta ancora alcune lacune (es. placeholder ancora presenti).
\end{adjustwidth}
\subsection*{\ref{itm:ndp} Discussione sullo stato di avanzamento delle Norme di Progetto\textsubscript{\scalebox{0.6}{\textbf{G}}} e ultime sezioni da completare}
Linor Sadè ha fatto il punto della situazione sullo stato di avanzamento delle Norme di Progetto\textsubscript{\scalebox{0.6}{\textbf{G}}}, evidenziando le ultime sezioni da completare: metriche di qualità\textsubscript{\scalebox{0.6}{\textbf{G}}} dei processi e del prodotto\textsubscript{\scalebox{0.6}{\textbf{G}}}.
\setlength{\extrarowheight}{2pt} % padding extra verticale
\section{Tabella delle decisioni e azioni}
\renewcommand{\arraystretch}{1.5} 


\arrayrulecolor{primaryblue}
\sloppy
\begin{tabularx}{\textwidth}{
    |>{\raggedright\arraybackslash}p{3cm}|
    >{\raggedright\arraybackslash}X|
    >{\raggedright\arraybackslash}p{3cm}|
}
\hline
\rowcolor{primaryblue!40}
\textbf{\color{white} ID Decisione} & \textbf{\color{white} Descrizione} & \textbf{\color{white} Incaricato} \\
\hline
\DecisionRow{\href{https://github.com/BugBustersUnipd/DocumentazioneSWE/issues/134}{RTB77}}{Effettuare modifiche al backend\textsubscript{\scalebox{0.6}{\textbf{G}}} del POC\textsubscript{\scalebox{0.6}{\textbf{G}}}}{Luca Slongo, Marco Favero}
\DecisionRow{\href{https://github.com/BugBustersUnipd/DocumentazioneSWE/issues/138}{RTB78}}{Effettuare modifiche al frontend\textsubscript{\scalebox{0.6}{\textbf{G}}} del POC\textsubscript{\scalebox{0.6}{\textbf{G}}}}{Linor Sadè, Marco Piro, Alberto Pignat}
\DecisionRow{\href{https://github.com/BugBustersUnipd/DocumentazioneSWE/issues/142}{RTB79}}{Ricerca sul cost–benefit ratio di modelli AI\textsubscript{\scalebox{0.6}{\textbf{G}}} più performanti rispetto a quello attuale}{Marco Favero}
\DecisionRow{\href{https://github.com/BugBustersUnipd/DocumentazioneSWE/issues/143}{RTB80}}{Completare la tabella nel Piano di Progetto\textsubscript{\scalebox{0.6}{\textbf{G}}} relativa al Piano di Qualifica\textsubscript{\scalebox{0.6}{\textbf{G}}} e Analisi dei Requisiti\textsubscript{\scalebox{0.6}{\textbf{G}}}}{Leonardo Salviato, Marco Piro}
\DecisionRow{\href{https://github.com/BugBustersUnipd/DocumentazioneSWE/issues/144}{RTB81}}{Redigere il verbale\textsubscript{\scalebox{0.6}{\textbf{G}}} corrente}{Linor Sadè}
\DecisionRow{\href{https://github.com/BugBustersUnipd/DocumentazioneSWE/issues/145}{RTB82}}{Verificare il verbale\textsubscript{\scalebox{0.6}{\textbf{G}}} corrente}{Marco Favero}
\hline
\end{tabularx}
\fussy

\vfill
\begin{center}
    {\small\color{darkgray} Documento redatto e approvato dal gruppo BugBusters.}
\end{center}

\end{document}