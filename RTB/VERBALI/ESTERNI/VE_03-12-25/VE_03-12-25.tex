\documentclass[a4paper,12pt]{article}

\usepackage[utf8]{inputenc}
\usepackage[T1]{fontenc}
\usepackage[italian]{babel}
\usepackage[margin=2.5cm]{geometry}
\usepackage{graphicx}
\usepackage{grffile}
\usepackage{booktabs}
\usepackage{setspace}
\usepackage{titlesec}
\usepackage{float}
\usepackage{ifthen}
\usepackage{tcolorbox}
\usepackage{enumitem}
\usepackage[colorlinks=true,linkcolor=black,urlcolor=primaryblue,citecolor=primaryblue]{hyperref}
\usepackage{changepage} % per definire un rientro (un margine) solo in una parte del documento

\usepackage[table]{xcolor}   % per i colori (celle, testo e righe)
\usepackage{tabularx}               % per tabelle a larghezza adattiva (X)
\usepackage{array}                  % per comandi di formattazione avanzata nelle colonne (p, >{\raggedright})

% Macro
% genera la stringa "\noindent (Riferimento alla tabella decisioni:
% \hyperref[RTB4]{RTB4})", si usa facendo \refDecisione{NomeLabel}{Testo}
\newcommand{\refDecisione}[2]{%
    \noindent (\textbf{Riferimento alla tabella decisioni: \hyperref[#1]{#2}})%
}




\definecolor{primaryblue}{RGB}{0,102,204}
\definecolor{secondaryblue}{RGB}{51,153,255}
\definecolor{lightgray}{RGB}{245,245,245}
\definecolor{darkgray}{RGB}{100,100,100}

\titleformat{\section}
  {\Large\bfseries\color{primaryblue}}
  {\thesection}{1em}{}

\setlength{\parskip}{4pt}
\setlength{\parindent}{0pt}

\setlist[itemize]{leftmargin=*,itemsep=3pt}
\setlist[enumerate]{leftmargin=*,itemsep=3pt}
\graphicspath{{./}{../assets/images/}{./images/}}

\begin{document}

\begin{center}  \IfFileExists{../../../../assets/Logo.jpg}{%
    \includegraphics[width=6cm,height=3cm,keepaspectratio]{../../../../assets/Logo.jpg} \\[0.8cm]
  }{%
    \fbox{\parbox[c][2.5cm][c]{6cm}{\centering Logo non trovato\\(Logo.jpg)}}\\[0.5cm]
  }
  
  {\Large\bfseries\color{primaryblue} BugBusters}\\[0.3cm]
  {\small\color{darkgray} Email: \texttt{bugbusters.unipd@gmail.com}} \\[0.1cm]
  {\small\color{darkgray} Gruppo: 4} \\[0.5cm]

  {\large\bfseries Università degli Studi di Padova}\\[0.3cm]
  {\small Laurea in Informatica}\\[0.2cm]
  {\small Corso: Ingegneria del Software}\\[0.2cm]
  {\small Anno Accademico: 2025/2026}\\[0.8cm]

  {\Huge\bfseries\color{primaryblue} Verbale Esterno\G{}}\\[0.3cm]
  {\Large\color{secondaryblue} 3 dicembre 2025}\\[0.8cm]
\end{center}

\begin{center}
\begin{tcolorbox}[colback=lightgray,colframe=primaryblue,width=0.85\textwidth,arc=3mm,boxrule=0.5pt]
\begin{tabularx}{\linewidth}{@{}lX@{}}
\textbf{Redattori}    & Marco Favero \\
\textbf{Verificatore}    & Linor Sadè \\
\textbf{Uso}          & Esterno \\
\textbf{Destinatari}  & Prof. Tullio Vardanega, Prof. Riccardo Cardin, Eggon, BugBusters\\
\textbf{Versione} & 1.0.0\\

\end{tabularx}
\end{tcolorbox}
\end{center}

\vspace{0.5cm}

\begin{center}
\begin{tcolorbox}[colback=secondaryblue!10,colframe=secondaryblue,width=0.9\textwidth,arc=3mm,boxrule=0.8pt,title={\bfseries Abstract}]
Questo verbale\G{} documenta la riunione esterna del 3 dicembre 2025 con l'azienda Eggon, durante la quale sono stati discussi 
aspetti tecnici e funzionali del progetto\G{}. Sono stati chiariti dubbi relativi all'autenticazione, alla gestione dei cedolini\G{},
 alla separazione dei documenti e al tono delle comunicazioni automatiche. L'incontro ha permesso di risolvere criticità e definire 
 più precisamente i requisiti dell'applicazione.
\end{tcolorbox}
\end{center}

\newpage

\tableofcontents
\newpage

\section{Informazioni generali}

\begin{itemize}
    \item \textbf{Tipo riunione:} Esterna
    \item \textbf{Piattaforma:} Google Meet\G{}
    \item \textbf{Data:} 03/12/2025
    \item \textbf{Orario di inizio:} 15:00
    \item \textbf{Orario di fine:} 15:30
    \item \textbf{Presenti:}
    \begin{itemize}[leftmargin=1.5em, itemsep=3pt, label={\rule[0.5ex]{0.4em}{0.4em}}]
        \item Alberto Autiero
        \item Marco Favero
        \item Alberto Pignat
        \item Marco Piro
        \item Linor Sadè
        \item Leonardo Salviato
        \item Luca Slongo
    \end{itemize}
    \item \textbf{Assenti:} 
    \begin{itemize}[leftmargin=1.5em, itemsep=3pt, label={\rule[0.5ex]{0.4em}{0.4em}}]
    \item Nessuno
    \end{itemize}
    \item \textbf{Presenti Esterni:}
    \begin{itemize}[leftmargin=1.5em, itemsep=3pt, label={\rule[0.5ex]{0.4em}{0.4em}}]
    \item Luca Iuzzolino
    \end{itemize}
\end{itemize}

\section{Ordine del giorno}

\begin{enumerate}
    \item \label{itm:autenticazione} Richiesta sull'autenticazione riguardo alle app stand-alone\G{}.
    \item \label{itm:SeparazioneDOc} Richiesta feature separazione documenti su CDL.
    \item \label{itm:Cedolini} Richiesta se dobbiamo implementare la funzione di visualizzazione dei cedolini\G{}.
    \item \label{itm:Tono} Come definire il tono aziendale.
    \item \label{itm:domande} Domande minori.

\end{enumerate}

\newpage
\section{Svolgimento}

\subsection*{\ref{itm:autenticazione}. Richiesta sull'autenticazione riguardo alle app stand-alone\G{}}
Visto che l'applicazione che verrà sviluppata dal gruppo BugBusters non sarà integrata direttamente nell'applicazione NEXUM, ma verrà sviluppata come app stand-alone\G{}, 
il team ha chiesto se l'autenticazione degli utenti potrà essere gestita tramite un sistema di autenticazione separato o se non sarà requisito\G{} necessario.
Luca Iuzzolino ha tenuto a ribadire che non è parte richiesta dal progetto\G{} e non interessante per noi.\\

\subsection*{\ref{itm:SeparazioneDOc}. Richiesta feature separazione documenti su CDL.}
È stata discussa la funzionalità\G{} di separazione (split) dei documenti caricati su CDL: il team ha chiesto che venga resa opzionale, 
poiché non sempre ne percepisce l'utilità. \\
Luca Iuzzolino ha precisato che la funzione è utile in quanto i software che generano i cedolini\G{} spesso
producono file PDF contenenti più cedolini\G{} in un unico documento. Si è concordato dunque che la funzionalità\G{} resterà primaria.

\subsection*{\ref{itm:Cedolini}. Richiesta se dobbiamo implementare la funzione di visualizzazione dei cedolini\G{}}
I cedolini\G{} sono gestiti all'interno di NEXUM. 
Poiché la nostra applicazione sarà stand-alone\G{} e non integrata in NEXUM, 
il team ha chiesto se sia necessario implementare una funzionalità\G{} per la 
visualizzazione dei cedolini\G{}. Luca Iuzzolino ha risposto che non è richiesto: 
l'applicazione dovrà occuparsi esclusivamente dell'invio dei cedolini\G{}Sistemate G tramite 
sistemi di notifica, mentre la visualizzazione rimane a carico di NEXUM.


\subsection*{\ref{itm:Tono}. Come definire il tono aziendale.}
Il team chiede se usare una knowledge base per definire il tono aziendale, in quanto non è stato specificato nulla a riguardo.
Luca Iuzzolino risponde che non è strettamente necessario, in quanto il tono aziendale si può definire tramite scelta fra toni predefiniti, 
può impararlo durante l'utilizzo o può ricercare su internet informazioni riguardo all'azienda e adattare il tono di conseguenza.
La ricerca e l'analisi di eventuali soluzioni viene dunque affidata al team BugBusters.

\subsection*{\ref{itm:domande}. Domande minori.}
Il team chiede se è possibile avere un account Bedrock per poter iniziare a sperimentare le tecnologie richieste. La risposta è stata positiva.\\
Il team domanda ulteriori informazioni riguardo alla funzionalità\G{} duplica/riutilizza prompt\G{}: la funzionalità\G{} di duplicazione permette di copiare un prompt\G{} esistente e modificarlo per creare un nuovo prompt\G{},
mentre la funzionalità\G{} di riutilizzo consente di utilizzare un prompt\G{} esistente senza modificarlo.\\


\section{Tabella delle decisioni e azioni}

\setlength{\extrarowheight}{2pt} % padding extra verticale
\renewcommand{\arraystretch}{1.5} 

\arrayrulecolor{primaryblue}
\sloppy
\begin{tabularx}{\textwidth}{
    |>{\raggedright\arraybackslash}p{3cm}|
    >{\raggedright\arraybackslash}X|
    >{\raggedright\arraybackslash}p{3cm}|
}
\hline
\rowcolor{primaryblue!40}
\textbf{\color{white} ID Decisione} & \textbf{\color{white} Descrizione} & \textbf{\color{white} Incaricato} \\
\hline
\DecisionRow{\href{https://github.com/BugBustersUnipd/DocumentazioneSWE/issues/29}{RTB26} \label{RTB26}}{Rivedere analisi requisiti dopo i chiarimenti con la proponente\G{}.}{Leonardo Salviato}
\end{tabularx}
\fussy


\section{Esito Riunione}
Sono stati discussi i punti riportati nell'ordine del giorno, aiutando il team a risolvere i propri dubbi. Il prossimo meeting è stato fissato il 17 dicembre 2025.\\
Si ringraziano l'azienda Eggon e i rappresentanti Gianpaolo Ferrarin e Luca Iuzzolino per la disponibilità dimostrata.

% SEZIONE FIRMA E DATA SOSTITUITA CON L'IMMAGINE
\vspace{1.0cm}
\noindent
\includegraphics[width=0.5\textwidth]{Data e firma.png}

\vfill
\begin{center}
    {\small\color{darkgray} Documento redatto e approvato dal gruppo BugBusters.}
\end{center}

\end{document}
