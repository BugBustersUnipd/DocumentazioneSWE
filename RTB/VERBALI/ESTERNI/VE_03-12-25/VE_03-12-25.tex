\documentclass[a4paper,12pt]{article}

\usepackage[utf8]{inputenc}
\usepackage[T1]{fontenc}
\usepackage[italian]{babel}
\usepackage[margin=2.5cm]{geometry}
\usepackage{graphicx}
\usepackage{grffile}
\usepackage{booktabs}
\usepackage{setspace}
\usepackage{titlesec}
\usepackage{float}
\usepackage{ifthen}
\usepackage{tcolorbox}
\usepackage{enumitem}
\usepackage[colorlinks=true,linkcolor=black,urlcolor=primaryblue,citecolor=primaryblue]{hyperref}
\usepackage{changepage} % per definire un rientro (un margine) solo in una parte del documento

\usepackage[table]{xcolor}   % per i colori (celle, testo e righe)
\usepackage{tabularx}               % per tabelle a larghezza adattiva (X)
\usepackage{array}                  % per comandi di formattazione avanzata nelle colonne (p, >{\raggedright})

% Macro
% genera la stringa "\noindent (Riferimento alla tabella decisioni:
% \hyperref[RTB4]{RTB4})", si usa facendo \refDecisione{NomeLabel}{Testo}
\newcommand{\refDecisione}[2]{%
    \noindent (\textbf{Riferimento alla tabella decisioni: \hyperref[#1]{#2}})%
}




\definecolor{primaryblue}{RGB}{0,102,204}
\definecolor{secondaryblue}{RGB}{51,153,255}
\definecolor{lightgray}{RGB}{245,245,245}
\definecolor{darkgray}{RGB}{100,100,100}

\titleformat{\section}
  {\Large\bfseries\color{primaryblue}}
  {\thesection}{1em}{}

\setlength{\parskip}{4pt}
\setlength{\parindent}{0pt}

\setlist[itemize]{leftmargin=*,itemsep=3pt}
\setlist[enumerate]{leftmargin=*,itemsep=3pt}
\graphicspath{{./}{../assets/images/}{./images/}}

\begin{document}

\begin{center}  \IfFileExists{../../../../assets/Logo.jpg}{%
    \includegraphics[width=6cm,height=3cm,keepaspectratio]{../../../../assets/Logo.jpg} \\[0.8cm]
  }{%
    \fbox{\parbox[c][2.5cm][c]{6cm}{\centering Logo non trovato\\(Logo.jpg)}}\\[0.5cm]
  }
  
  {\Large\bfseries\color{primaryblue} BugBusters}\\[0.3cm]
  {\small\color{darkgray} Email: \texttt{bugbusters.unipd@gmail.com}} \\[0.1cm]
  {\small\color{darkgray} Gruppo: 4} \\[0.5cm]

  {\large\bfseries Università degli Studi di Padova}\\[0.3cm]
  {\small Laurea in Informatica}\\[0.2cm]
  {\small Corso: Ingegneria del Software}\\[0.2cm]
  {\small Anno Accademico: 2025/2026}\\[0.8cm]

  {\Huge\bfseries\color{primaryblue} Verbale Esterno}\\[0.3cm]
  {\Large\color{secondaryblue} 19 novembre 2025}\\[0.8cm]
\end{center}

\begin{center}
\begin{tcolorbox}[colback=lightgray,colframe=primaryblue,width=0.85\textwidth,arc=3mm,boxrule=0.5pt]
\begin{tabularx}{\linewidth}{@{}lX@{}}
\textbf{Redattori}    & Marco Favero \\
\textbf{Verificatore}    & Linor Sadè \\
\textbf{Uso}          & Esterno \\
\textbf{Destinatari}  & Prof. Tullio Vardanega, Prof. Riccardo Cardin, Eggon, BugBusters\\
\textbf{Versione} & 0.0.1\\

\end{tabularx}
\end{tcolorbox}
\end{center}

\vspace{0.5cm}

\begin{center}
\begin{tcolorbox}[colback=secondaryblue!10,colframe=secondaryblue,width=0.9\textwidth,arc=3mm,boxrule=0.8pt,title={\bfseries Abstract}]
Questo verbale documenta la riunione esterna del 3 dicembre 2025 con l'azienda Eggon, durante la quale sono stati discussi 
aspetti tecnici e funzionali del progetto. Sono stati chiariti dubbi relativi all'autenticazione, alla gestione dei cedolini,
 alla separazione dei documenti e al tono delle comunicazioni automatiche. L'incontro ha permesso di risolvere criticità e definire 
 più precisamente i requisiti dell'applicazione.
\end{tcolorbox}
\end{center}

\newpage

\tableofcontents
\newpage

\section{Informazioni generali}

\begin{itemize}
    \item \textbf{Tipo riunione:} Esterna
    \item \textbf{Piattaforma:} Google Meet
    \item \textbf{Data:} 03/12/2025
    \item \textbf{Orario di inizio:} 15:00
    \item \textbf{Orario di fine:} 15:30
    \item \textbf{Presenti:}
    \begin{itemize}[leftmargin=1.5em, itemsep=3pt, label={\rule[0.5ex]{0.4em}{0.4em}}]
        \item Alberto Autiero
        \item Marco Favero
        \item Alberto Pignat
        \item Marco Piro
        \item Linor Sadè
        \item Leonardo Salviato
        \item Luca Slongo
    \end{itemize}
    \item \textbf{Assenti:} 
    \begin{itemize}[leftmargin=1.5em, itemsep=3pt, label={\rule[0.5ex]{0.4em}{0.4em}}]
    \item Nessuno
    \end{itemize}
    \item \textbf{Presenti Esterni:}
    \begin{itemize}[leftmargin=1.5em, itemsep=3pt, label={\rule[0.5ex]{0.4em}{0.4em}}]
    \item Luca Iuzzolino
    \end{itemize}
\end{itemize}

\section{Ordine del giorno}

\begin{enumerate}
    \item \label{itm:autenticazione} Richista sul autenticazione riguardo alle app stand alone.
    \item \label{itm:SeparazioneDOc} Richiesta feature separazione documenti su CDL.
    \item \label{itm:Cedolini} Richiesta se dobbiamo implementare la funzione di visualizzazione dei cedolini.
    \item \label{itm:Tono} Come definire il tono aziendale.
    \item \label{itm:domande} Richiesta domande minori.

\end{enumerate}

\newpage
\section{Svolgimento}

\subsection*{\ref{itm:autenticazione}. Richista sul autenticazione riguardo alle app stand alone}
Visto che l'applicazione che verrá sviluppata dal gruppo BugBusters non sarà integrata direttamente nell'applicazione NEXUM, ma verrà sviluppata come app stand alone, 
il team ha chiesto se l'autenticazione degli utenti potrà essere gestita tramite un sistema di autenticazione separato o se non sará requisito necessario.
Luca Iuzzolino ha tenuto a ribadire che non é parte richiesta dal progetto e non interessante per noi.\\

\subsection*{\ref{itm:SeparazioneDOc}. Richiesta feature separazione documenti su CDL.}
È stata discussa la funzionalità di separazione (split) dei documenti caricati su CDL: il team ha chiesto che venga resa opzionale, 
poiché non sempre ne percepisce l'utilità. \\
Luca Iuzzolino ha precisato che la funzione è utile quando i software che generano i cedolini 
producono file PDF contenenti più cedolini in un unico documento. Si è concordato dunque chela funzionalitá resterá primaria.

\subsection*{\ref{itm:Cedolini}. Richiesta se dobbiamo implementare la funzione di visualizzazione dei cedolini}
I cedolini sono gestiti all'interno di NEXUM. 
Poiché la nostra applicazione sarà standalone e non integrata in NEXUM, 
il team ha chiesto se sia necessario implementare una funzionalità per la 
visualizzazione dei cedolini. Luca Iuzzolino ha risposto che non è richiesto: 
l'applicazione dovrà occuparsi esclusivamente dell'invio dei cedolini tramite 
sistemi di notifica, mentre la visualizzazione rimane a carico di NEXUM.


\subsection*{\ref{itm:Tono}. Come definire il tono aziendale.}
Il team chiede se usare una knowledge base per definire il tono aziendale, in quanto non é stato specificato nulla a riguardo.
Luca Iuzzolino risponde che non é strettamente necessario, in quanto il tono aziendale si puó definire tramite scelta fra toni predefiniti, 
puó impararlo durante l'utilizzo o puó ricercare su internet informazioni riguardo all'azienda e adattare il tono di conseguenza.
La ricerca e l'analisi dunque viene affidata al team BugBusters.

\subsection*{\ref{itm:domande}. Domande minori.}
Il team chiede se é possibile avere un account Bedrock per poter iniziare a toccare con mano le tecnologie richieste. La risposta é stata positiva.\\
Il team domanda ulteriori iformazioni riguardo alla funzionalitá duplica/riutilizza prompt: la funzionalitá di duplicazione permette di copiare un prompt esistente e modificarlo per creare un nuovo prompt,
mentre la funzionalitá di riutilizzo consente di utilizzare un prompt esistente senza modificarlo.\\


\section{Tabella delle decisioni e azioni}

\setlength{\extrarowheight}{2pt} % padding extra verticale
\renewcommand{\arraystretch}{1.5} 

\arrayrulecolor{primaryblue}
\sloppy
\begin{tabularx}{\textwidth}{
    |>{\raggedright\arraybackslash}p{3cm}|
    >{\raggedright\arraybackslash}X|
    >{\raggedright\arraybackslash}p{3cm}|
}
\hline
\rowcolor{primaryblue!40}
\textbf{\color{white} ID Decisione} & \textbf{\color{white} Descrizione} & \textbf{\color{white} Incaricato} \\
\hline
\DecisionRow{\href{https://github.com/BugBustersUnipd/DocumentazioneSWE/issues/29}{RTB26} \label{RTB26}}{Rivedere analisi requisiti dopo i chiarimenti con la proponente.}{Leonardo Salviato}
\end{tabularx}
\fussy


\section{Esito Riunione}
Sono stati discussi i punti riportati nell'ordine del giorno, aiutando il team a risolvere i propri dubbi. Il prossimo meeting é stato fissato il 17 dicembre 2025.\\
Si ringraziano l'azienda Eggon e i rappresentanti Gianpaolo Ferrarin e Luca Iuzzolino per la disponibilità dimostrata.

\vspace{1.5cm}
\noindent\textbf{\Large Data}\\[0.4cm]
\underline{\hspace{4cm}}

\vspace{1.5cm}
\noindent\textbf{\Large Firma}\\[0.8cm]
\underline{\hspace{6cm}} \\[0.2cm]

\vfill
\begin{center}
    {\small\color{darkgray} Documento redatto e approvato dal gruppo BugBusters.}
\end{center}

\end{document}
