\documentclass[a4paper,12pt]{article}

\usepackage[utf8]{inputenc}
\usepackage[T1]{fontenc}
\usepackage[italian]{babel}
\usepackage[margin=2.5cm]{geometry}
\usepackage{graphicx}
\usepackage{grffile}
\usepackage{booktabs}
\usepackage{setspace}
\usepackage{titlesec}
\usepackage{float}
\usepackage{ifthen}
\usepackage{tcolorbox}
\usepackage{enumitem}
\usepackage[colorlinks=true,linkcolor=black,urlcolor=primaryblue,citecolor=primaryblue]{hyperref}
\usepackage{changepage} % per definire un rientro (un margine) solo in una parte del documento

\usepackage[table]{xcolor}   % per i colori (celle, testo e righe)
\usepackage{tabularx}               % per tabelle a larghezza adattiva (X)
\usepackage{array}                  % per comandi di formattazione avanzata nelle colonne (p, >{\raggedright})

% Macro
% genera la stringa "\noindent (Riferimento alla tabella decisioni:
% \hyperref[RTB4]{RTB4})", si usa facendo \refDecisione{NomeLabel}{Testo}
\newcommand{\refDecisione}[2]{%
    \noindent (\textbf{Riferimento alla tabella decisioni: \hyperref[#1]{#2}})%
}



\definecolor{primaryblue}{RGB}{0,102,204}
\definecolor{secondaryblue}{RGB}{51,153,255}
\definecolor{lightgray}{RGB}{245,245,245}
\definecolor{darkgray}{RGB}{100,100,100}

\titleformat{\section}
  {\Large\bfseries\color{primaryblue}}
  {\thesection}{1em}{}

\setlength{\parskip}{4pt}
\setlength{\parindent}{0pt}

\setlist[itemize]{leftmargin=*,itemsep=3pt}
\setlist[enumerate]{leftmargin=*,itemsep=3pt}

\graphicspath{{./}{../assets/images/}{./images/}}

\begin{document}

\begin{center}  \IfFileExists{../../../../assets/Logo.jpg}{%
    \includegraphics[width=6cm,height=3cm,keepaspectratio]{../../../../assets/Logo.jpg} \\[0.8cm]
  }{%
    \fbox{\parbox[c][2.5cm][c]{6cm}{\centering Logo non trovato\\(Logo.jpg)}}\\[0.5cm]
  }
  
  {\Large\bfseries\color{primaryblue} BugBusters}\\[0.3cm]
  {\small\color{darkgray} Email: \texttt{bugbusters.unipd@gmail.com}} \\[0.1cm]
  {\small\color{darkgray} Gruppo: 4} \\[0.5cm]

  {\large\bfseries Università degli Studi di Padova}\\[0.3cm]
  {\small Laurea in Informatica}\\[0.2cm]
  {\small Corso: Ingegneria del Software}\\[0.2cm]
  {\small Anno Accademico: 2025/2026}\\[0.8cm]

  {\Huge\bfseries\color{primaryblue} Verbale Esterno\textsubscript{\scalebox{0.6}{\textbf{G}}}}\\[0.3cm]
  {\Large\color{secondaryblue} 12 novembre 2025}\\[0.8cm]
\end{center}

\begin{center}
\begin{tcolorbox}[colback=lightgray,colframe=primaryblue,width=0.85\textwidth,arc=3mm,boxrule=0.5pt]
\begin{tabularx}{\linewidth}{@{}lX@{}}
\textbf{Redattori}    & Alberto Pignat \\
\textbf{Verificatore}    & Marco Piro \\
\textbf{Uso}          & Esterno \\
\textbf{Destinatari}  & Prof. Tullio Vardanega, Prof. Riccardo Cardin, Eggon, BugBusters\\
\textbf{Versione} & 1.0.0\\

\end{tabularx}
\end{tcolorbox}
\end{center}

\vspace{0.5cm}

\begin{center}
\begin{tcolorbox}[colback=secondaryblue!10,colframe=secondaryblue,width=0.9\textwidth,arc=3mm,boxrule=0.8pt,title={\bfseries Abstract}]
Verbale\textsubscript{\scalebox{0.6}{\textbf{G}}} riguardante il colloquio via Google Meet\textsubscript{\scalebox{0.6}{\textbf{G}}} con Eggon. Sono stati discussi i vincoli tecnici del capitolato\textsubscript{\scalebox{0.6}{\textbf{G}}}, sono stati definiti i prossimi meeting con Eggon ed è stato presentato Nexum nella sua versione attuale. 
\end{tcolorbox}
\end{center}

\newpage

\tableofcontents
\newpage

\section{Informazioni generali}

\begin{itemize}
    \item \textbf{Tipo riunione:} Esterna
    \item \textbf{Piattaforma:} Google Meet\textsubscript{\scalebox{0.6}{\textbf{G}}}
    \item \textbf{Data:} 12/11/2025
    \item \textbf{Orario di inizio:} 15:00
    \item \textbf{Orario di fine:} 16:00
    \item \textbf{Presenti:}
    \begin{itemize}[leftmargin=1.5em, itemsep=3pt, label={\rule[0.5ex]{0.4em}{0.4em}}]
        \item Linor Sadè
        \item Alberto Autiero
        \item Marco Favero
        \item Luca Slongo
        \item Alberto Pignat
        \item Leonardo Salviato
        \item Marco Piro
    \end{itemize}
    \item \textbf{Assenti:} Nessuno
    \item \textbf{Presenti Esterni:}
    \begin{itemize}[leftmargin=1.5em, itemsep=3pt, label={\rule[0.5ex]{0.4em}{0.4em}}]
    \item Gianpaolo Ferrarin
    \item Luca Iuzzolino
    \end{itemize}
\end{itemize}

\section{Ordine del giorno}

\begin{enumerate}
    \item \label{itm:comunicazione} Definizione delle modalità di collaborazione, mezzi di comunicazione e organizzazione del progetto\textsubscript{\scalebox{0.6}{\textbf{G}}};
    \item \label{itm:Nexum} Visione del funzionamento dell'applicativo Nexum;
    \item \label{itm:tecniche} Chiarimenti riguardanti i vincoli tecnici progettuali;
\end{enumerate}

\newpage
\section{Svolgimento}

\subsection*{\ref{itm:comunicazione}. Definizione delle modalità di collaborazione, mezzi di comunicazione e organizzazione del progetto\textsubscript{\scalebox{0.6}{\textbf{G}}}}
Luca Iuzzolino fornirà supporto al team dal punto di vista della parte tecnica del progetto\textsubscript{\scalebox{0.6}{\textbf{G}}}.\\
Viene presentata l'applicazione web GitLab dove sono già state predisposte le repository\textsubscript{\scalebox{0.6}{\textbf{G}}} contenenti il codice preesistente di Nexum.\\
Per poter usufruire di GitLab è richiesto ad ogni membro del team di fornire la propria email in modo tale da poter visionare le repository\textsubscript{\scalebox{0.6}{\textbf{G}}} relative a Nexum.\\
Oltre al poter contattare Eggon tramite email o al poter recarsi in presenza alla sede, verrà predisposto un gruppo Telegram per comunicazioni più rapide.\\
Dentro l'applicativo Nexum verrà predisposta una nuova azienda di prova, con al suo interno i membri del team, per poter testare l'applicativo al meglio.\\



\subsection*{\ref{itm:Nexum}. Visione del funzionamento dell'applicativo Nexum}
È stato presentato l'applicativo Nexum con tutte le sue funzionalità\textsubscript{\scalebox{0.6}{\textbf{G}}}, sia dal punto di vista dell'utente che dal punto di vista dell'amministratore\textsubscript{\scalebox{0.6}{\textbf{G}}}.


\subsection*{\ref{itm:tecniche}. Chiarimenti riguardanti i vincoli tecnici progettuali}
Riguardo alle tecnologie da utilizzare, quando si presenterà il POC, l'unica tecnologia richiesta è AWS\textsubscript{\scalebox{0.6}{\textbf{G}}} Bedrock, per il resto non è obbligatorio che le tecnologie utilizzate siano le stesse in uso già da Eggon.


\section{Esito Riunione}
Sono stati meglio definiti i vincoli tecnici del progetto\textsubscript{\scalebox{0.6}{\textbf{G}}}. Si è deciso di avere, per un primo periodo, meeting settimanali (Mercoledì alle ore 15) con Eggon, per poi procedere con un meeting ogni due settimane. Il team si impegna ad inviare i propri recapiti email in modo da poter prendere visione delle repository\textsubscript{\scalebox{0.6}{\textbf{G}}} su GitLab.\\
Si ringraziano l'azienda Eggon e i rappresentanti Gianpaolo Ferrarin e Luca Iuzzolino per la disponibilità dimostrata.
\newpage

% SEZIONE FIRMA E DATA SOSTITUITA CON L'IMMAGINE
\vspace{1.0cm}
\noindent
\includegraphics[width=0.5\textwidth]{Data e firma.png}

\vfill
\begin{center}
    {\small\color{darkgray} Documento redatto e approvato dal gruppo BugBusters.}
\end{center}

\end{document}