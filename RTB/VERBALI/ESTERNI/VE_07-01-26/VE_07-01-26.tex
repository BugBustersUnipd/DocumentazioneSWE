\documentclass[a4paper,12pt]{article}

\usepackage[utf8]{inputenc}
\usepackage[T1]{fontenc}
\usepackage[italian]{babel}
\usepackage[margin=2.5cm]{geometry}
\usepackage{graphicx}
\usepackage{grffile}
\usepackage{booktabs}
\usepackage{setspace}
\usepackage{titlesec}
\usepackage{float}
\usepackage{ifthen}
\usepackage{tcolorbox}
\usepackage{enumitem}
\usepackage[colorlinks=true,linkcolor=black,urlcolor=primaryblue,citecolor=primaryblue]{hyperref}
\usepackage{changepage} % per definire un rientro (un margine) solo in una parte del documento

\usepackage[table]{xcolor}   % per i colori (celle, testo e righe)
\usepackage{tabularx}               % per tabelle a larghezza adattiva (X)
\usepackage{array}                  % per comandi di formattazione avanzata nelle colonne (p, >{\raggedright})

% Macro
% genera la stringa "\noindent (Riferimento alla tabella decisioni:
% \hyperref[RTB4]{RTB4})", si usa facendo \refDecisione{NomeLabel}{Testo}
\newcommand{\refDecisione}[2]{%
    \noindent (\textbf{Riferimento alla tabella decisioni: \hyperref[#1]{#2}})%
}




\definecolor{primaryblue}{RGB}{0,102,204}
\definecolor{secondaryblue}{RGB}{51,153,255}
\definecolor{lightgray}{RGB}{245,245,245}
\definecolor{darkgray}{RGB}{100,100,100}

\titleformat{\section}
  {\Large\bfseries\color{primaryblue}}
  {\thesection}{1em}{}

\setlength{\parskip}{4pt}
\setlength{\parindent}{0pt}

\setlist[itemize]{leftmargin=*,itemsep=3pt}
\setlist[enumerate]{leftmargin=*,itemsep=3pt}
\graphicspath{{./}{../assets/images/}{./images/}}

\begin{document}

\begin{center}  \IfFileExists{../../../../assets/Logo.jpg}{%
    \includegraphics[width=6cm,height=3cm,keepaspectratio]{../../../../assets/Logo.jpg} \\[0.8cm]
  }{%
    \fbox{\parbox[c][2.5cm][c]{6cm}{\centering Logo non trovato\\(Logo.jpg)}}\\[0.5cm]
  }
  
  {\Large\bfseries\color{primaryblue} BugBusters}\\[0.3cm]
  {\small\color{darkgray} Email: \texttt{bugbusters.unipd@gmail.com}} \\[0.1cm]
  {\small\color{darkgray} Gruppo: 4} \\[0.5cm]

  {\large\bfseries Università degli Studi di Padova}\\[0.3cm]
  {\small Laurea in Informatica}\\[0.2cm]
  {\small Corso: Ingegneria del Software}\\[0.2cm]
  {\small Anno Accademico: 2025/2026}\\[0.8cm]

  {\Huge\bfseries\color{primaryblue} Verbale Esterno}\\[0.3cm]
  {\Large\color{secondaryblue} 7 gennaio 2026}\\[0.8cm]
\end{center}

\begin{center}
\begin{tcolorbox}[colback=lightgray,colframe=primaryblue,width=0.85\textwidth,arc=3mm,boxrule=0.5pt]
\begin{tabularx}{\linewidth}{@{}lX@{}}
\textbf{Redattori}    & Linor Sadè \\
\textbf{Verificatore}    & Marco Favero \\
\textbf{Uso}          & Esterno \\
\textbf{Destinatari}  & Prof. Tullio Vardanega, Prof. Riccardo Cardin, Eggon, BugBusters\\
\textbf{Versione} & 0.0.1\\

\end{tabularx}
\end{tcolorbox}
\end{center}

\vspace{0.5cm}

\begin{center}
\begin{tcolorbox}[colback=secondaryblue!10,colframe=secondaryblue,width=0.9\textwidth,arc=3mm,boxrule=0.8pt,title={\bfseries Abstract}]
\end{tcolorbox}
\end{center}

\newpage

\tableofcontents
\newpage

\section{Informazioni generali}

\begin{itemize}
    \item \textbf{Tipo riunione:} Esterna
    \item \textbf{Piattaforma:} Google Meet
    \item \textbf{Data:} 17/12/2025
    \item \textbf{Orario di inizio:} 16:00
    \item \textbf{Orario di fine:} 17:10
    \item \textbf{Presenti:}
    \begin{itemize}[leftmargin=1.5em, itemsep=3pt, label={\rule[0.5ex]{0.4em}{0.4em}}]
        \item Alberto Autiero
        \item Marco Favero
        \item Alberto Pignat
        \item Marco Piro
        \item Linor Sadè
        \item Leonardo Salviato
        \item Luca Slongo
    \end{itemize}
    \item \textbf{Assenti:} 
    \begin{itemize}[leftmargin=1.5em, itemsep=3pt, label={\rule[0.5ex]{0.4em}{0.4em}}]
    \item Nessuno
    \end{itemize}
    \item \textbf{Presenti Esterni:}
    \begin{itemize}[leftmargin=1.5em, itemsep=3pt, label={\rule[0.5ex]{0.4em}{0.4em}}]
    \item Luca Iuzzolino
    \item Gianpaolo Ferrarin
    \end{itemize}
\end{itemize}

\section{Ordine del giorno}

\begin{enumerate}
    \item \label{itm:poc} Presentazione POC aggiornato in data 7 gennaio 2026.
    \begin{enumerate}
      \item \label{itm:poc2} Dimostrazione funzionalità implementate.
      \item \label{itm:problematiche} Discussione sulle problematiche riscontrate.
    \end{enumerate}
    \item \label{itm:Analisi dei requisiti} Discussione su eventuali modifiche al documento relativo all'analisi dei requisiti.
    \item \label{itm:progettazione} Domande relative ai modelli in uso.

\end{enumerate}

\newpage
\section{Svolgimento}

\subsection*{\ref{itm:poc}. Presentazione POC aggiornato in data 7 gennaio 2026.}
In sede di riunione, il team ha presentato il POC aggiornato alla data corrente, illustrando le funzionalità implementate e le tecnologie utilizzate.
Qui di seguito sono riportati i punti salienti della presentazione.
\subsubsection*{\ref{itm:poc2}. Dimostrazione funzionalità implementate.}
\begin{itemize}
  \item Il modulo AI assistant generativo, che consente agli utenti di generare contenuti testuali a partire da prompt, selezionando un'azienda di riferimento tra quelle disponibili e un tono di comunicazione preferito tra quelli disponibili per azienda.
  \item Il modulo di Ai co-pilot per Cdl ancora in stato embrionale, che consente agli utenti di caricare documenti in vari formati (es PDF, png, jpeg, jpg) e di estrarre informazioni predefinite.
\end{itemize}
\paragraph{}
Durante la presentazione, è stato spiegato che il POC non ha alcuna pretesa di essere finito e di essere coerente con i requisiti definiti, ma vuole essere un primo passo verso l'uso delle tecnologie.\\
Per esempio, il modulo AI assistant generativo consente la modifica dei contenuti generati tramite prompt consecutivi, mentre da analisi dei requisiti emerge la necessità di un risultato più preciso in seguito al singolo prompt, con eventuali modifiche manuali qualora ve ne fosse bisogno. 
\subsubsection{\ref{itm:problematiche}. Discussione sulle problematiche riscontrate.}
\begin{itemize}
  \item P
\end{itemize}
\paragraph{} Sono stati riscontrati alcuni problemi nella generazione di contenuti, come la presenza di placeholder nel testo generato, qualora mancasse il contesto nel prompt inserito dall'utente. Il modello dovrebbe essere in grado di generare un testo completo senza placeholder, il problema potrebbe essere dovuto ad un'implementazione parziale o alla scelta del modello.\\
Infine inserendo prompt fuori dal dominio aziendale, il modello tende a generare risposte non pertinenti o messaggi di errore (per esempio "Non sono in grado di rispondere a questa domanda"). È stato suggerito di implementare un sistema di filtraggio e di controllo dei prompt in ingresso e in uscita (guardrail AI), in modo da evitare richieste non pertinenti e che le risposte siano sicure e appropriate.
\subsection*{\ref{itm:progettazione}. Domande relative ai modelli in uso.}
Entrambi i moduli sono stati implementati usando dei modelli \textit{Nova} di amazon: il modello \textit{Nova lite v1} e \textit{Nova Canvas v1} per il testo e per l'OCR dei documenti.\\\\
Il dubbio del team riguardava la scelta di questi modelli, in quanto non sicuri fossero i più adatti allo scopo. Eggon ha confermato che per il primo modulo i modelli sono adeguati, anche se è da capire se il problema dei placeholder è dovuto al modello o all'implementazione. \\
È stato suggerito di testare modelli ad uso pubblico (es GPT-4 di OpenAI, Gemini etc) per confrontare i risultati ottenuti.
Qualora il modello risultasse inadeguato, si potrà pensare di cambiare modello nella sua interezza o di definire un sistema ibrido: usare un modello più potente per la rigenerazione del prompt scritto dall'utente, che verrà successivamente passato al modello attuale per la generazione del testo. Questo permetterebbe di mantenere i costi bassi, in quanto i modelli più potenti hanno costi di utilizzo più elevati ma nello stesso tempo di garantire risultati migliori.\\

\paragraph{}

\section{Tabella delle decisioni e azioni}

\subsection*{\ref{itm:Analisi dei requisiti}. Discussione su eventuali modifiche al documento relativo all'analisi dei requisiti.}

\setlength{\extrarowheight}{2pt} % padding extra verticale
\renewcommand{\arraystretch}{1.5} 

\arrayrulecolor{primaryblue}
\sloppy
\begin{tabularx}{\textwidth}{
    |>{\raggedright\arraybackslash}p{3cm}|
    >{\raggedright\arraybackslash}X|
    >{\raggedright\arraybackslash}p{3cm}|
}
\hline
\rowcolor{primaryblue!40}
\textbf{\color{white} ID Decisione} & \textbf{\color{white} Descrizione} & \textbf{\color{white} Incaricato} \\
\hline
\DecisionRow{\href{https://github.com/BugBustersUnipd/DocumentazioneSWE/issues/58}{RTB36} \label{RTB35}}{Verificare la disponibilità del team e di Eggon relativa ad un incontro in presenza il 7 Gennaio.}{Alberto Pignat}
\end{tabularx}
\fussy


\section{Esito Riunione}
Sono stati discussi i punti riportati nell'ordine del giorno, aiutando il team a risolvere i propri dubbi.\\
Si ringraziano l'azienda Eggon e i rappresentanti Gianpaolo Ferrarin e Luca Iuzzolino per la disponibilità dimostrata.
\newpage 

% SEZIONE FIRMA E DATA SOSTITUITA CON L'IMMAGINE
\vspace{1.0cm}
\noindent
% \includegraphics[width=0.5\textwidth]{Data e firma.png}

\vfill
\begin{center}
    {\small\color{darkgray} Documento redatto e approvato dal gruppo BugBusters.}
\end{center}

\end{document}