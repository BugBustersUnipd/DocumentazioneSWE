\documentclass[a4paper,12pt]{article}

\usepackage[utf8]{inputenc}
\usepackage[T1]{fontenc}
\usepackage[italian]{babel}
\usepackage[margin=2.5cm]{geometry}
\usepackage{graphicx}
\usepackage{grffile}
\usepackage{booktabs}
\usepackage{setspace}
\usepackage{titlesec}
\usepackage{float}
\usepackage{ifthen}
\usepackage{tcolorbox}
\usepackage{enumitem}
\usepackage[colorlinks=true,linkcolor=black,urlcolor=primaryblue,citecolor=primaryblue]{hyperref}
\usepackage{changepage} % per definire un rientro (un margine) solo in una parte del documento

\usepackage[table]{xcolor}   % per i colori (celle, testo e righe)
\usepackage{tabularx}               % per tabelle a larghezza adattiva (X)
\usepackage{array}                  % per comandi di formattazione avanzata nelle colonne (p, >{\raggedright})

% Macro
% genera la stringa "\noindent (Riferimento alla tabella decisioni:
% \hyperref[RTB4]{RTB4})", si usa facendo \refDecisione{NomeLabel}{Testo}
\newcommand{\refDecisione}[2]{%
    \noindent (\textbf{Riferimento alla tabella decisioni: \hyperref[#1]{#2}})%
}




\definecolor{primaryblue}{RGB}{0,102,204}
\definecolor{secondaryblue}{RGB}{51,153,255}
\definecolor{lightgray}{RGB}{245,245,245}
\definecolor{darkgray}{RGB}{100,100,100}

\titleformat{\section}
  {\Large\bfseries\color{primaryblue}}
  {\thesection}{1em}{}

\setlength{\parskip}{4pt}
\setlength{\parindent}{0pt}

\setlist[itemize]{leftmargin=*,itemsep=3pt}
\setlist[enumerate]{leftmargin=*,itemsep=3pt}
\graphicspath{{./}{../assets/images/}{./images/}}

\begin{document}

\begin{center}  \IfFileExists{../../../../assets/Logo.jpg}{%
    \includegraphics[width=6cm,height=3cm,keepaspectratio]{../../../../assets/Logo.jpg} \\[0.8cm]
  }{%
    \fbox{\parbox[c][2.5cm][c]{6cm}{\centering Logo non trovato\\(Logo.jpg)}}\\[0.5cm]
  }
  
  {\Large\bfseries\color{primaryblue} BugBusters}\\[0.3cm]
  {\small\color{darkgray} Email: \texttt{bugbusters.unipd@gmail.com}} \\[0.1cm]
  {\small\color{darkgray} Gruppo: 4} \\[0.5cm]

  {\large\bfseries Università degli Studi di Padova}\\[0.3cm]
  {\small Laurea in Informatica}\\[0.2cm]
  {\small Corso: Ingegneria del Software}\\[0.2cm]
  {\small Anno Accademico: 2025/2026}\\[0.8cm]

  {\Huge\bfseries\color{primaryblue} Verbale Esterno}\\[0.3cm]
  {\Large\color{secondaryblue} 7 gennaio 2026}\\[0.8cm]
\end{center}

\begin{center}
\begin{tcolorbox}[colback=lightgray,colframe=primaryblue,width=0.85\textwidth,arc=3mm,boxrule=0.5pt]
\begin{tabularx}{\linewidth}{@{}lX@{}}
\textbf{Redattori}    & Linor Sadè \\
\textbf{Verificatore}    & Marco Favero \\
\textbf{Uso}          & Esterno \\
\textbf{Destinatari}  & Prof. Tullio Vardanega, Prof. Riccardo Cardin, Eggon, BugBusters\\
\textbf{Versione} & 0.0.1\\

\end{tabularx}
\end{tcolorbox}
\end{center}

\vspace{0.5cm}

\begin{center}
\begin{tcolorbox}[colback=secondaryblue!10,colframe=secondaryblue,width=0.9\textwidth,arc=3mm,boxrule=0.8pt,title={\bfseries Abstract}]
\end{tcolorbox}
\end{center}

\newpage

\tableofcontents
\newpage

\section{Informazioni generali}

\begin{itemize}
    \item \textbf{Tipo riunione:} Esterna
    \item \textbf{Piattaforma:} Google Meet
    \item \textbf{Data:} 17/12/2025
    \item \textbf{Orario di inizio:} 15:00
    \item \textbf{Orario di fine:} 15:30
    \item \textbf{Presenti:}
    \begin{itemize}[leftmargin=1.5em, itemsep=3pt, label={\rule[0.5ex]{0.4em}{0.4em}}]
        \item Alberto Autiero
        \item Marco Favero
        \item Alberto Pignat
        \item Marco Piro
        \item Linor Sadè
        \item Leonardo Salviato
        \item Luca Slongo
    \end{itemize}
    \item \textbf{Assenti:} 
    \begin{itemize}[leftmargin=1.5em, itemsep=3pt, label={\rule[0.5ex]{0.4em}{0.4em}}]
    \item Nessuno
    \end{itemize}
    \item \textbf{Presenti Esterni:}
    \begin{itemize}[leftmargin=1.5em, itemsep=3pt, label={\rule[0.5ex]{0.4em}{0.4em}}]
    \item Luca Iuzzolino
    \item Gianpaolo Ferrarin
    \end{itemize}
\end{itemize}

\section{Ordine del giorno}

\begin{enumerate}
    \item \label{itm:poc} Presentazione POC aggiornato in data 7 gennaio 2026.
    \item \label{itm:Analisi dei requisiti} Discussione su eventuali modifiche al documento relativo all'analisi dei requisiti.
    \item \label{itm:progettazione} Domande relative ai modelli in uso.

\end{enumerate}

\newpage
\section{Svolgimento}

\subsection*{\ref{itm:poc}. Presentazione interfaccia utente per il POC.}
Il team ha presentato a Eggon un'idea di una possibile interfaccia di quello che sarà poi il POC.
Riguardo alla confidenza di un file è stato fatto notare che il livello di confidenza misurato dovrebbe essere relativo a ogni entità estratta, e non solo una media complessiva della confidenza del file.

\subsection*{\ref{itm:Analisi dei requisiti}. Presentazione documento relativo all'analisi dei requisiti.}
Il team ha presentato e descritto la struttura del documento relativo all'analisi dei requisiti, mostrando come sono suddivisi i casi d'uso e descrivendone la granularità.\\
Come concordato durante le passate riunioni, viene mantenuto il caso d'uso relativo al login degli utenti, ma posto come opzionale.

\subsection*{\ref{itm:Piano di Qualifica}. Domande relative alle metriche riguardanti la qualità di prodotto e processo relative al piano di qualifica.}
Il team ha chiesto relativamente al piano di qualifica, se Eggon richiedesse delle particolari metriche di qualita di processo o prodotto da essere rispettate, oltre a quelle già definite nel capitolato
La risposta è stata che i criteri attualmente definiti nel capitolato sono sufficienti a dimostrare l'affidabilità del sistema, ma in ogni caso il team è libero, qualora lo ritenesse necessario, 
di aggiungere ulteriori criteri.

\section{Tabella delle decisioni e azioni}

\setlength{\extrarowheight}{2pt} % padding extra verticale
\renewcommand{\arraystretch}{1.5} 

\arrayrulecolor{primaryblue}
\sloppy
\begin{tabularx}{\textwidth}{
    |>{\raggedright\arraybackslash}p{3cm}|
    >{\raggedright\arraybackslash}X|
    >{\raggedright\arraybackslash}p{3cm}|
}
\hline
\rowcolor{primaryblue!40}
\textbf{\color{white} ID Decisione} & \textbf{\color{white} Descrizione} & \textbf{\color{white} Incaricato} \\
\hline
\DecisionRow{\href{https://github.com/BugBustersUnipd/DocumentazioneSWE/issues/58}{RTB36} \label{RTB35}}{Verificare la disponibilità del team e di Eggon relativa ad un incontro in presenza il 7 Gennaio.}{Alberto Pignat}
\end{tabularx}
\fussy


\section{Esito Riunione}
Sono stati discussi i punti riportati nell'ordine del giorno, aiutando il team a risolvere i propri dubbi.\\
Si ringraziano l'azienda Eggon e i rappresentanti Gianpaolo Ferrarin e Luca Iuzzolino per la disponibilità dimostrata.
\newpage 

% SEZIONE FIRMA E DATA SOSTITUITA CON L'IMMAGINE
\vspace{1.0cm}
\noindent
% \includegraphics[width=0.5\textwidth]{Data e firma.png}

\vfill
\begin{center}
    {\small\color{darkgray} Documento redatto e approvato dal gruppo BugBusters.}
\end{center}

\end{document}