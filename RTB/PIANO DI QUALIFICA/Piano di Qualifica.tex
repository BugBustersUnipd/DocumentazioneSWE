\documentclass[a4paper,11pt]{article}
   
\newcommand{\CurrentVersion}{1.0.0} % ultima versione, da cambiare ad ogni push significativo

\usepackage[utf8]{inputenc}
\usepackage[T1]{fontenc}
\usepackage[italian]{babel}
\usepackage[margin=2.5cm]{geometry}
\usepackage{graphicx}
\usepackage{grffile}
\usepackage{booktabs}
\usepackage{setspace}
\usepackage{titlesec}
\usepackage{float}
\usepackage{ifthen}
\usepackage[table]{xcolor}
\usepackage{tabularx}
\usepackage{needspace}
\usepackage{tcolorbox}
\usepackage{enumitem}
\usepackage[titles]{tocloft}
\usepackage[colorlinks=true,linkcolor=black,urlcolor=blue,citecolor=blue]{hyperref}
\usepackage{xspace}
% Macro
% genera la stringa "\noindent (Riferimento alla tabella decisioni:
% \hyperref[RTB4]{RTB4})", si usa facendo \refDecisione{NomeLabel}{Testo}
\newcommand{\refDecisione}[2]{%
    \noindent (\textbf{Riferimento alla tabella decisioni: \hyperref[#1]{#2}})%
}


\definecolor{primaryblue}{RGB}{0,102,204}
\definecolor{secondaryblue}{RGB}{51,153,255}
\definecolor{lightgray}{RGB}{245,245,245}
\definecolor{darkgray}{RGB}{100,100,100}

\titleformat{\section}
 {\Large\bfseries\color{primaryblue}}
 {\thesection}{1em}{}

\titleformat{\subsection}
 {\large\bfseries\color{primaryblue}} % Sottosezione: colore secondaryblue
 {\thesubsection}{1em}{}

\titleformat{\subsubsection}
 {\normalsize\bfseries\color{secondaryblue}} % Sotto-sottosezione: colore secondaryblue
 {\thesubsubsection}{1em}{}

% per il footer con il numero di pagina
\usepackage{fancyhdr}
\usepackage{lastpage} % per ottenere il numero dell'ultima pagina da mettere nel footer

\usepackage{ltablex} %per far andare a capo le tabelle
\keepXColumns

\renewcommand{\sectionmark}[1]{\markright{#1}}
\renewcommand{\G}{\textsubscript{\scalebox{0.6}{\textbf{G}}}\ }

% Macro tabella metrica (tabella uniforme, non spezzata e con stile coerente)
\newcommand{\MetricTable}[1]{
  \Needspace{10\baselineskip}% evita spezzamenti sgradevoli
  \begin{table}[H]
  \centering
  \arrayrulecolor{primaryblue}
  \begin{tabularx}{\linewidth}{|>{\raggedright\arraybackslash}p{2.5cm}|X|>{\raggedright\arraybackslash}p{3cm}|>{\raggedright\arraybackslash}p{3cm}|}
  \hline
  \rowcolor{primaryblue!40}
      \textbf{\color{white} Metrica} & \textbf{\color{white} Descrizione} & \textbf{\color{white} Valore accettazione} & \textbf{\color{white} Valore ideale} \\
  \hline
  \ignorespaces
  #1
  \end{tabularx}
  \end{table}
}



\setlength{\parskip}{4pt}
\setlength{\parindent}{0pt}

\setlist[itemize]{leftmargin=*,itemsep=3pt}
\setlist[enumerate]{leftmargin=*,itemsep=3pt}

\graphicspath{{./}{../assets/images/}{./images/}}

\begin{document}

%configurazione per il footer
\pagestyle{fancy}
\fancyhf{} % pulisce tutti i campi del header e footer

% Header: sinistra e destra
\fancyhead[L]{Gruppo 4 - BugBusters} % sinistra
\fancyhead[R]{Piano di Qualifica\G}   % destra

\fancyfoot[L]{ \thepage\ di \pageref{LastPage}} %definisce il formato del footer
\fancyfoot[R]{ \nouppercase{\rightmark}} % nome della sezione


\renewcommand{\headrulewidth}{0pt} % rimuove la linea dell'header
\renewcommand{\footrulewidth}{0pt} % se vuoi anche togliere eventuale linea del footer

% abilitare numerazione e TOC fino al livello "paragraph" (subsubsubsection)
\setcounter{secnumdepth}{4}
\setcounter{tocdepth}{4}
% formattazione del nuovo livello per avere aspetto coerente
\titleformat{\paragraph}[block]{\normalsize\bfseries\color{secondaryblue}}{\theparagraph}{1em}{}
% alias comodo per usare "subsubsubsection"
\newcommand{\subsubsubsection}{\paragraph}

\begin{center}
  \IfFileExists{../../assets/Logo.jpg}{%
    \includegraphics[width=6cm,height=3cm,keepaspectratio]{../../assets/Logo.jpg} \\[0.8cm]
  }{%
    \fbox{\parbox[c][2.5cm][c]{6cm}{\centering Logo non trovato\\(../../assets/Logo.jpg)}}\\[0.5cm]
  }
  
  {\Large\bfseries\color{primaryblue} BugBusters}\\[0.5cm]

  {\Huge\bfseries\color{primaryblue} Piano di Qualifica\G}\\[0.3cm]
  {\Large\color{secondaryblue} Versione \CurrentVersion}\\[0.8cm]
\end{center}

\begin{center}
\begin{tcolorbox}[colback=lightgray,colframe=primaryblue,width=0.85\textwidth,arc=3mm,boxrule=0.5pt]
% Usa tabularx con una colonna fissa per l'etichetta e una colonna X per il contenuto
\begin{tabularx}{\linewidth}{@{}>{\raggedright\arraybackslash}p{3.5cm}>{\raggedright\arraybackslash}X@{}}
  {Stato} & Approvato per RTB\G \\
  {Redattori\G} & Marco Favero, Linor Sadé, Marco Piro, Leonardo Salviato \\
  {Verificatori\G} & Alberto Pignat, Marco Favero \\
  {Distribuzione} & BugBusters, Eggon, Prof. Tullio Vardanega, Prof. Riccardo Cardin \\
\end{tabularx}
\end{tcolorbox}
\end{center}

\vspace{0.5cm}

\begin{center}
\begin{tcolorbox}[colback=secondaryblue!10,colframe=secondaryblue,width=0.9\textwidth,arc=3mm,boxrule=0.8pt,title={\bfseries Descrizione}]
Piano di Qualifica\G del Team BugBusters per il Capitolato\G C5 proposto da Eggon, che ha l'obiettivo di far rispettare uno standard di qualità\G per il codice e rispettare i requisiti funzionali\G prestabiliti.\end{tcolorbox}
\end{center}

\newpage

% Registro delle Modifiche (pagina 2)
\section*{Registro delle Modifiche}

\arrayrulecolor{primaryblue}
{\footnotesize
\begin{tabularx}{\textwidth}{|>{\raggedright\arraybackslash}p{1.5cm}|>{\raggedright\arraybackslash}p{2cm}|X|>{\raggedright\arraybackslash}p{2cm}|>{\raggedright\arraybackslash}p{2cm}|>{\raggedright\arraybackslash}p{2cm}|}
\hline
\rowcolor{primaryblue!40}
\textbf{\color{white} Versione} & \textbf{\color{white} Data} & \textbf{\color{white} Descrizione} & \textbf{\color{white} Redatto} & \textbf{\color{white} Verificato} & \textbf{\color{white} Approvato} \\
\hline
\rowcolor{secondaryblue!10}1.0.0 & 08/02/2026 & Effettuata approvazione\G & - & - & Luca Slongo \\
\hline
\rowcolor{secondaryblue!10}0.2.0 & 08/02/2026 & Effettuata verifica\G & - & Marco Favero & - \\
\hline
\rowcolor{secondaryblue!10}0.1.4 & 01/02/2026 & Aggiunti ai termini presenti nel Glossario\G la G & Luca Slongo & - & - \\
\hline
\rowcolor{secondaryblue!10}0.1.3 & 01/02/2026 & Aggiunta descrizione grafici metriche & Leonardo Salviato & - & - \\
\hline
\rowcolor{secondaryblue!10}0.1.2 & 04/02/2026 & Aggiunti grafici metriche, aggiornamento Test\G, rimossa matrice di Tracciamento & Leonardo Salviato & - & - \\
\hline
\rowcolor{secondaryblue!10}0.1.1 & 19/01/2026 & Aggiornamento Test\G, aggiunti test\G di sistema Casi Limite e integrazione, cambiato alcune metriche di prodotto\G. Aggiunta matrice di Tracciamento & Linor Sadé & - & - \\
\hline
\rowcolor{secondaryblue!10}0.1.0 & 16/01/2026 & Verifica\G intermedia & - & Alberto Pignat & - \\
\hline
\rowcolor{secondaryblue!10}0.0.5 & 15/01/2026 & Aggiornamento Test\G, aggiunti test\G di sistema prestazionali, eliminata metrica errori ortografici & Linor Sadé & - & - \\
\hline
\rowcolor{secondaryblue!10}0.0.4 & 11/01/2026 & Aggiunto contenuto alla sezione 5 & Linor Sadé & - & - \\
\hline
\rowcolor{secondaryblue!10}0.0.3 & 04/01/2026 & Aggiunte sezioni 4 e 5 & Marco Piro & - & - \\
\hline
\rowcolor{secondaryblue!10}0.0.2 & 29/12/2025 & Aggiunta Test\G di Sistema e di Accettazione & Marco Piro & - & - \\
\hline
\rowcolor{secondaryblue!10}0.0.1 & 03/12/2025 & Prima stesura del documento & Marco Favero & - & - \\
\hline
\end{tabularx}
}

\newpage

% Indice cliccabile
\setcounter{tocdepth}{3} % Mostra fino al livello di sottosezione (2)
\tableofcontents
\listoftables
\listoffigures

\newpage
\section{Introduzione}

\subsection{Scopo del documento}
Il presente documento, denominato \textit{Piano di Qualifica\G}, ha lo scopo di definire le strategie, le procedure e le metriche adottate dal gruppo \textit{BugBusters} per garantire la qualità\G del prodotto\G software e dei processi produttivi relativi al progetto\G C5 (NEXUM), proposto dall'azienda \textit{Eggon}.

In particolare, questo documento si prefigge di:
\begin{itemize}
    \item \textbf{Definire gli obiettivi di qualità\G:} specificare i target qualitativi per il processo di sviluppo (efficienza\G, stabilità) e per il prodotto\G software (funzionalità\G, affidabilità\G, manutenibilità\G), in conformità con gli standard ISO/IEC 12207 e ISO/IEC 9126;
    \item \textbf{Identificare le metriche:} selezionare gli indicatori quantitativi più idonei per monitorare il raggiungimento degli obiettivi, fissando per ciascuno le soglie di accettazione e di ottimalità;
    \item \textbf{Pianificare le attività di verifica\G e validazione\G:} descrivere le metodologie di test\G (unità, integrazione, sistema, accettazione) e le procedure di analisi statica del codice e della documentazione;
    \item \textbf{Monitorare l'andamento del progetto\G:} fornire un resoconto puntuale (cruscotto\G di valutazione) delle misurazioni effettuate durante le varie fasi del ciclo di vita, permettendo al team di individuare tempestivamente criticità e attuare azioni correttive (miglioramento continuo).
\end{itemize}

\subsection{Glossario\G}
Al fine di evitare ambiguità e garantire una comprensione uniforme della terminologia utilizzata, è stato redatto un documento esterno denominato \textit{Glossario\G}.
I termini tecnici, gli acronimi e le parole con un significato specifico all'interno del progetto\G sono contrassegnati nel testo da una "G" in pedice (es. parola). La loro definizione completa è consultabile nel \textit{Glossario\G}.

\subsection{Riferimenti}

\subsubsection{Riferimenti normativi}
\begin{itemize}
    \item \textbf{Capitolato\G d'appalto C5 - NEXUM (Eggon):}\\
    \url{https://www.math.unipd.it/~tullio/IS-1/2025/Progetto/C5.pdf}
    
    \item \textbf{Norme di Progetto\G (v1.0.0):}\\
    Documento interno del gruppo \textit{BugBusters} che definisce le regole, i ruoli e le procedure operative.\\
    \url{https://bugbustersunipd.github.io/DocumentazioneSWE/RTB/NORME%20DI%20PROGETTO/Norme%20di%20Progetto.pdf}
    
    \item \textbf{Regolamento del progetto\G didattico:} \\
    \url{https://www.math.unipd.it/~tullio/IS-1/2025/Dispense/PD1.pdf}
\end{itemize}

\subsubsection{Riferimenti informativi}
\begin{itemize}
    \item \textbf{Glossario\G (v1.0.0):}\\
    Documento interno del gruppo \textit{BugBusters} contenente le definizioni dei termini tecnici.\\
    \url{https://bugbustersunipd.github.io/DocumentazioneSWE/RTB/GLOSSARIO/Glossario.pdf}
    \item \textbf{Standard ISO/IEC 12207:1995:}\\
    \textit{Information technology - Software life cycle processes}.\\
    \url{https://www.math.unipd.it/~tullio/IS-1/2009/Approfondimenti/ISO_12207-1995.pdf}
    
    \item \textbf{Standard ISO/IEC 9126:}\\
    \textit{Software engineering - Product quality}.\\
    \url{https://it.wikipedia.org/wiki/ISO/IEC_9126}
    
    \item \textbf{Slide del corso di Ingegneria del Software:} \\
    Materiale didattico fornito dai docenti Prof. Tullio Vardanega e Prof. Riccardo Cardin.
\end{itemize}
\newpage

\section{Obiettivi stabiliti per la qualità\G}

È fondamentale stabilire degli obiettivi da raggiungere per assicurare la qualità\G prefissata del prodotto\G. 
Questo documento definisce i valori di accettazione e ottimalità delle metriche secondo gli standard definiti 
nelle Norme di Progetto\G.

\subsection{Qualità\G di processo}
Un indicatore della qualità\G di un prodotto\G è il metodo con cui è stato sviluppato. Se il processo di sviluppo segue 
delle linee guida ben definite, esso favorisce la buona riuscita del prodotto\G. Come stabilito nelle Norme di Progetto\G, 
nel nostro way of working\G abbiamo adottato lo Standard ISO/IEC 12207:1995 adattandolo alle nostre esigenze e a quelle 
del progetto\G. 

\subsubsection{Processi primari}

I processi primari sono quelle attività che iniziano o eseguono lo sviluppo, l'operazione o la manutenzione di prodotti software. Essi rappresentano le componenti fondamentali del ciclo di vita del progetto\G e sono suddivisi nelle seguenti categorie:
\subsubsubsection{Fornitura}

\MetricTable{
MPC01 & Earned value (EV)\G & $\geq 0$ & $\leq$ EAC \\
\hline
MPC02 & Planned value (PV)\G & $\geq 0$ & $\leq$ Budget at completion (BAC) \\
\hline
MPC03 & Actual cost (AC)\G & $\geq 0$ & $\leq$ EAC \\
\hline
MPC04 & Cost Performance Index (CPI)\G & $\geq 0.9$ & 1 \\
\hline
MPC05 & Schedule Performance Index (SPI)\G & $\geq 0.9$ & 1 \\
\hline
MPC06 & Estimated at completion (EAC) & $\pm 5\%$ rispetto al (BAC) & Budget at completion (BAC) \\
\hline
MPC07 & Estimate to complete (ETC)\G & $\geq 0$ & $\leq$ EAC \\
\hline
MPC08 & Time Estimate At Completion (TEAC)\G & $\geq 0$ & $\leq$ Durata pianificata \\
\hline}



\subsubsubsection{Sviluppo}
\MetricTable{
MPC09 & Requirements Stability Index\G & $\geq 70\%$ & 100\% \\
\hline}


\subsubsection{Processi di supporto}
\subsubsubsection{Documentazione}
\MetricTable{
MPC10 & Indice di Gulpease\G del documento & $\geq 60\%$ & $\geq 80\%$ \\
\hline
}
\subsubsubsection{Verifica\G}
\MetricTable{
MPC11 & Code Coverage\G & $\geq 80\%$ & $100\%$ \\
\hline
MPC12 & Test\G Success Rate & $100\%$ & $100\%$ \\
\hline}
\subsubsubsection{Gestione della qualità\G}
\MetricTable{
MPC13 & Quality metrics satisfied & $\geq 80\%$ & $100\%$ \\
\hline}

\subsubsection{Processi organizzativi}
\subsubsubsection{Gestione dei processi}
\MetricTable{
MPC14 & Time Efficiency & $\geq 50\%$ & $100\%$ \\
\hline}



\subsection{Qualità\G di prodotto\G}

Per qualità\G di prodotto\G si intende una valutazione complessiva del software sia dal punto di vista 
funzionale sia dal punto di vista strutturale. Il codice deve adempiere alle funzionalità\G prestabilite in 
modo efficiente e semplice, e al contempo essere manutenibile, affidabile e portabile. Il gruppo ha aderito allo 
standard ISO/IEC 9126 per garantire il rispetto di queste caratteristiche fondamentali, affinché il prodotto\G 
sviluppato sia di alta qualità\G.

\subsubsection{Funzionalità\G}
\MetricTable{
MPD01 & Requisiti obbligatori soddisfatti & $100\%$ & $100\%$ \\
\hline
MPD02 & Requisiti desiderabili soddisfatti & $0\%$ & $100\%$ \\
\hline
MPD03 & Requisiti opzionali soddisfatti & $0\%$ & $100\%$ \\
\hline
MPD04 & AI\G Acceptance Rate (Rating $\geq$ 3/5) & $\geq 60\%$ & $\geq 80\%$ \\
\hline}

\subsubsection{Affidabilità\G}
\MetricTable{
MPD05 & Branch Coverage\G & $\geq 70\%$ & $\geq 85\%$ \\
\hline
MPD06 & Defect Density & $\leq 3$ / KLOC & $\leq 1$ / KLOC \\
\hline}

\subsubsection{Efficienza\G}
\MetricTable{
MPD07 & UI Response Time (Interfaccia\G) & $\leq 2$ sec & $\leq 0.5$ sec \\
\hline
MPD08 & Core Response Time - AI\G Generativo testo & $\leq 5$ sec & $\leq 3$ sec \\
\hline
MPD09 & Core Response Time - AI\G Generativo immagini & $\leq 10$ sec & $\leq 5$ sec \\
\hline
MPD10 & Core Response Time - AI Co-Pilot\G & $\leq 10$ sec & $\leq 5$ sec \\
\hline}

\subsubsection{Usabilità}
\MetricTable{
MPD11 & Click Count (Funzioni principali) & $\leq 5$ click & $\leq 3$ click \\
\hline
MPD12 & User Error Rate (Errori validazione\G) & $\leq 10\%$ & $\leq 5\%$ \\
\hline}

\subsubsection{Mantenibilità}
\MetricTable{
MPD13 & Blocker Code Smells & $0$ & $0$ \\
\hline
MPD14 & Cyclomatic complexity\G (per metodo) & $\leq 15$ & $\leq 10$ \\
\hline
MPD15 & Comment Intensity & $\geq 10\%$ & $\geq 20\%$ \\
\hline}

\subsubsection{Portabilità}
\MetricTable{
MPD16 & Supported Browsers (Test\G passati) & $100\%$ (Desktop) & $100\%$ (All devices) \\
\hline}



\section{Metodi di testing}
La strategia di verifica\G e validazione\G adottata dal gruppo \textit{BugBusters} mira a garantire che ogni rilascio software sia conforme ai requisiti specificati e privo di difetti critici.
I test\G dinamici pianificati seguono un approccio incrementale (piramide dei test\G), partendo dalle singole unità logiche fino alla validazione\G dell'intero sistema integrato.



\subsection{Test\G di Integrazione}
I test\G di integrazione verificano la corretta comunicazione tra i sottosistemi e i moduli definiti nell'architettura, assicurando che le interfacce e lo scambio dati avvengano come previsto.

\begin{longtable}{|p{2.2cm}|p{7cm}|p{3.5cm}|}
  \hline
  \rowcolor{primaryblue!30} 
  \textbf{Codice} & \textbf{Descrizione Interfaccia\G} & \textbf{Moduli Coinvolti} \\
  \hline
  \endhead
  
  \textbf{TI-001} & Verifica\G scambio dati e gestione errori tramite chiamate API\G REST (formato JSON). & Frontend\G (Angular\G) $\leftrightarrow$ Backend\G (Ruby on Rails\G) \\
  \hline
  \textbf{TI-002} & Verifica\G invio del contesto/prompt\G e ricezione dello stream di risposta dal servizio AI\G. & Backend\G (Assistant) $\leftrightarrow$ External LLM\G API\G \\
  \hline
  \textbf{TI-003} & Verifica\G dell'integrità dei dati salvati e recuperati (utenti, documenti, chat log). & Backend\G Logic $\leftrightarrow$ Database (PostgreSQL\G) \\
  \hline
  \textbf{TI-004} & Verifica\G del caricamento file, estrazione testo (OCR\G) e validazione\G formato. & Upload Service $\leftrightarrow$ PDF Parser Module \\
  \hline
  \textbf{TI-005} & Verifica\G dell'aggregazione dei dati per la generazione delle statistiche visualizzate nella dashboard\G. & Analytics Module $\leftrightarrow$ Database \\
  \hline
  \textbf{TI-006} & Verifica\G del sistema di autenticazione e gestione sessioni utente. & Auth Controller $\leftrightarrow$ Session Manager \\
  \hline
  
  \caption{Test\G di Integrazione}
  \label{tab:test-integrazione}
\end{longtable}


\subsection{Test\G di Sistema}

\subsubsection{Test\G di Sistema - Requisiti Funzionali\G}
\begin{longtable}{|p{2.2cm}|p{8.3cm}|p{2cm}|p{1.5cm}|}
\hline
\rowcolor{primaryblue!30} 
\textbf{Codice} & \textbf{Descrizione} & \textbf{Riferimento} & \textbf{Stato} \\
\hline
\endhead

% --- AUTENTICAZIONE E UTENTI (RF 1-25) ---
TS-F-001 & Verifica\G che il sistema permetta all'utente non autenticato di effettuare la registrazione. & RF-1 & NI \\ \hline
TS-F-002 & Verifica\G che il sistema permetta all'utente di inserire l'indirizzo email durante la registrazione. & RF-2 & NI \\ \hline
TS-F-003 & Verifica\G che il sistema permetta all'utente di inserire la password durante la registrazione. & RF-3 & NI \\ \hline
TS-F-004 & Verifica\G che il sistema permetta all'utente di inserire l'username durante la registrazione. & RF-4 & NI \\ \hline
TS-F-005 & Verifica\G che il sistema permetta all'utente di inserire il nome durante la registrazione. & RF-5 & NI \\ \hline
TS-F-006 & Verifica\G che il sistema permetta all'utente di inserire il cognome durante la registrazione. & RF-6 & NI \\ \hline
TS-F-007 & Verifica\G che il sistema permetta all'utente di inserire la propria matricola durante la registrazione. & RF-7 & NI \\ \hline
TS-F-008 & Verifica\G che il sistema permetta la visualizzazione di un errore se l'email non è valida. & RF-8 & NI \\ \hline
TS-F-009 & Verifica\G che il sistema permetta la visualizzazione di un errore se la password non rispetta i criteri di sicurezza\G. & RF-9 & NI \\ \hline
TS-F-010 & Verifica\G che il sistema impedisca la registrazione se
l’email inserita è già associata a un account
esistente. & RF-10 & NI \\ \hline
TS-F-011 & Verifica\G che il sistema impedisca la registrazione se
lo username inserito è già utilizzato. & RF-11 & NI \\ \hline
TS-F-012 & Verifica\G che il sistema impedisca la registrazione
se la matricola inserita è già presente nel
sistema. & RF-12 & NI \\ \hline
TS-F-013 & Verifica\G che il sistema permetta la visualizzazione di un messaggio
di errore se il formato della matricola non è
valido. & RF-13 & NI \\ \hline
TS-F-014 & Verifica\G che il sistema permetta all'utente di effettuare il login (autenticazione). & RF-14 & NI \\ \hline
TS-F-015 & Verifica\G che il sistema notifichi l’errore in caso di
tentativo di login con email non registrata. & RF-15 & NI \\ \hline
TS-F-016 & Verifica\G che il sistema notifichi l’errore in caso di
tentativo di login con password errata. & RF-16 & NI \\ \hline
TS-F-017 & Verifica\G che il sistema permetta all'utente di visualizzare i dati del proprio profilo. & RF-17 & NI \\ \hline
TS-F-018 & Verifica\G che il sistema mostri l’email associata al
profilo utente. & RF-18 & NI \\ \hline
TS-F-019 & Verifica\G che il sistema mostri (o permetta la
gestione della) password del profilo utente. & RF-19 & NI \\ \hline
TS-F-020 & Verifica\G che il sistema mostri lo username
associato al profilo utente. & RF-20 & NI \\ \hline
TS-F-021 & Verifica\G che il sistema mostri il nome associato
al profilo utente. & RF-21 & NI \\ \hline
TS-F-022 & Verifica\G che il sistema mostri il cognome associato
al profilo utente. & RF-22 & NI \\ \hline
TS-F-023 & Verifica\G che il sistema mostri la matricola associata
al profilo utente. & RF-23 & NI \\ \hline
TS-F-024 & Verifica\G che il sistema permetta all'utente di modificare le informazioni del proprio profilo. & RF-24 & NI \\ \hline
TS-F-025 & Verifica\G che il sistema gestisca l’uscita dal-
la modifica profilo senza salvare i
cambiamenti. & RF-25 & NI \\ \hline

TS-F-026 & Verifica\G che il sistema permetta all'Ammini-
stratore di visualizzare la lista degli utenti
registrati. & RF-26 & NI \\ \hline
TS-F-027 & Verifica\G che il sistema permetta la visualizzazione
del dettaglio di un singolo utente dalla lista. & RF-27 & NI \\ \hline
TS-F-028 & Verifica\G che il sistema mostri il ruolo associato a
un utente registrato. & RF-28 & NI \\ \hline
TS-F-029 & Verifica\G che il sistema mostri il nome di un
utente registrato. & RF-29 & NI \\ \hline
TS-F-030 & Verifica\G che il sistema mostri il cognome di un
utente registrato. & RF-30 & NI \\ \hline
TS-F-031 & Verifica\G che il sistema permetta all'Amministratore di modificare il ruolo di un utente
registrato. & RF-31 & NI \\ \hline
TS-F-032 & Verifica\G che il sistema permetta all’utente di
effettuare il logout (terminare la sessione). & RF-32 & NI \\ \hline
TS-F-033 & Verifica\G che il sistema generi contenuti testuali
tramite AI\G Assistant in base a prompt\G e
parametri. & RF-33 & NI \\ \hline
TS-F-034 & Verifica\G che il sistema permetta l’inserimento di un prompt\G testuale per la generazione. & RF-34 & NI \\ \hline
TS-F-035 & Verifica\G che il sistema permetta la selezione del
tono per la generazione del contenuto. & RF-35 & NI \\ \hline
TS-F-036 & Verifica\G che il sistema permetta la selezione dello
stile per la generazione del contenuto. & RF-36 & NI \\ \hline
TS-F-037 & Verifica\G che il sistema permetta la visualizzazione
dello storico delle generazioni AI\G. & RF-37 & NI \\ \hline
TS-F-038 & Verifica\G che il sistema notifichi l’assenza di elementi se lo storico delle generazioni è
vuoto. & RF-38 & NI \\ \hline
TS-F-039 & Verifica\G che il sistema mostri i dettagli completi
di un elemento selezionato dallo storico. & RF-39 & NI \\ \hline

% --- CO-PILOT (DOCUMENTI) (RF 40-54) ---
TS-F-040 & Verifica\G che il sistema permetta la visualizzazione dello stile utilizzato
per un contenuto nello storico. & RF-40 & NI \\ \hline
TS-F-041 & Verifica\G che il sistema permetta la visualizzazione del testo del
risultato generato nello storico. & RF-41 & NI \\ \hline
TS-F-042 & Verifica\G che il sistema permetta la visualizzazione del timestamp
(data/ora) della generazione nello storico. & RF-42 & NI \\ \hline
TS-F-043 & Verifica\G che il sistema permetta la visualizzazione della valutazio-
ne assegnata dall’utente al contenuto nello
storico. & RF-43 & NI \\ \hline
TS-F-044 & Verifica\G che il sistema permetta la visualizzazione del prompt\G ori-
ginale utilizzato per un contenuto nello
storico. & RF-44 & NI \\ \hline
TS-F-045 & Verifica\G che il sistema permetta la visualizzazione del tono utilizzato
per un contenuto nello storico. & RF-45 & NI \\ \hline
TS-F-046 & Verifica\G che il sistema mostri un’anteprima del
contenuto generato dall’AI\G. & RF-46 & NI \\ \hline
TS-F-047 & Verifica\G che il sistema permetta di modificare
l’immagine associata al contenuto generato. & RF-47 & NI \\ \hline
TS-F-048 & Verifica\G che il sistema notifichi l’utente quan-
do tenta di caricare un file immagine non
valido. & RF-48 & NI \\ \hline
TS-F-049 & Verifica\G che il sistema permetta di modificare il
titolo del contenuto generato. & RF-49 & NI \\ \hline
TS-F-050 & Verifica\G che il sistema permetta di modificare il
testo del corpo del contenuto generato. & RF-50 & NI \\ \hline
TS-F-051 & Verifica\G che il sistema permetta di annullare le
modifiche apportate al contenuto generato & RF-51 & NI \\ \hline
TS-F-052 & Verifica\G che il sistema permetta di riutilizzare i
parametri di un contenuto dello storico per
una nuova generazione. & RF-52 & NI \\ \hline
TS-F-053 & Verifica\G che il sistema permetta di duplicare un
contenuto dallo storico per modificarne i
parametri. & RF-53 & NI \\ \hline
TS-F-054 & Verifica\G che il sistema permetta di filtrare la lista
delle generazioni nello storico. & RF-54 & NI \\ \hline

TS-F-055 & Verifica\G che il sistema permetta la visualizzazione della lista dello
storico aggiornata in base ai filtri applicati. & RF-55 & NI \\ \hline
TS-F-056 & Verifica\G che il sistema permetta di rigenera-
re un contenuto tramite AI\G mantenendo i
parametri. & RF-56 & NI \\ \hline
TS-F-057 & Verifica\G che il sistema permetta all’utente di
valutare (rating) il contenuto generato. & RF-57 & NI \\ \hline
TS-F-058 & Verifica\G che il sistema permetta di scartare il
contenuto generato e pulire l’interfaccia\G. & RF-58 & NI \\ \hline
TS-F-059 & Verifica\G che il sistema permetta di salvare il
contenuto generato nel database. & RF-59 & NI \\ \hline
TS-F-060 & Verifica\G che il sistema permetta all’utente l’inse-
rimento di un nuovo tono per la generazione
di contenuti. & RF-60 & NI \\ \hline
TS-F-061 & Verifica\G che il sistema permetta all’utente l’eli-
minazione di un tono per la generazione di
contenuti & RF-61 & NI \\ \hline

TS-F-062 & Verifica\G che il sistema permetta all’utente l’inse-
rimento di un nuovo stile per la generazione
di contenuti. & RF-62 & NI \\ \hline
TS-F-063 & Verifica\G che il sistema permetta all’utente l’eli-
minazione di uno stile per la generazione di
contenuti. & RF-63 & NI \\ \hline
TS-F-064 & Verifica\G che il sistema permetta l’analisi di
documenti tramite il modulo\G AI Co-Pilot\G. & RF-64 & NI \\ \hline
TS-F-065 & Verifica\G che il sistema permetta la selezione della categoria del
documento. & RF-65 & NI \\ \hline
TS-F-066 & Verifica\G che il sistema permetta l’inserimento del
mese/anno di competenza del documento. & RF-66 & NI \\ \hline
TS-F-067 & Verifica\G che il sistema permetta l’inserimento
dell’azienda associata al documento. & RF-67 & NI \\ \hline
TS-F-068 & Verifica\G che il sistema permetta l’inserimento del
reparto associato al documento. & RF-68 & NI \\ \hline
TS-F-069 & Verifica\G che il sistema permetta di controllare la
correttezza del formato del file inserito. & RF-69 & NI \\ \hline
TS-F-070 & Verifica\G che il sistema permetta di controllare che lo
stesso file non sia già stato analizzato. & RF-70 & NI \\ \hline
TS-F-071 & Verifica\G che il sistema permetta lo split di do-
cumenti diversi all’interno dello stesso
file. & RF-71 & NI \\ \hline
TS-F-072 & Verifica\G che il sistema permetta la visualizzazione della lista dei
documenti analizzati. & RF-72 & NI \\ \hline
TS-F-073 & Verifica\G che il sistema notifichi l’utente se nessun
documento è stato riconosciuto dall’analisi. & RF-73 & NI \\ \hline
TS-F-074 & Verifica\G che il sistema permetta di visualizzare i
dettagli di un singolo documento dalla lista. & RF-74 & NI \\ \hline
TS-F-075 & Verifica\G che il sistema permetta la visualizzazione della competenza
(periodo) del documento analizzato. & RF-75 & NI \\ \hline
TS-F-076 & Verifica\G che il sistema permetta la visualizzazione dell’azienda
associata al documento analizzato. & RF-76 & NI \\ \hline
TS-F-077 & Verifica\G che il sistema permetta la visualizzazione della causale del
documento analizzato. & RF-77 & NI \\ \hline
TS-F-078 & Verifica\G che il sistema permetta la visualizzazione della lingua rilevata
nel documento. & RF-78 & NI \\ \hline
TS-F-079 & Verifica\G che il sistema permetta la visualizzazione del numero di
pagine del documento. & RF-79 & NI \\ \hline
TS-F-080 & Verifica\G che il sistema permetta la visualizzazione del nome originale
del file del documento. & RF-80 & NI \\ \hline
TS-F-081 & Verifica\G che il sistema permetta la visualizzazione della data di
redazione/caricamento del documento. & RF-81 & NI \\ \hline
TS-F-082 & Verifica\G che il sistema permetta la visualizzazione del codice
identificativo del documento. & RF-82 & NI \\ \hline
TS-F-083 & Verifica\G che il sistema permetta la visualizzazione della tipologia del
documento. & RF-83 & NI \\ \hline
TS-F-084 & Verifica\G che il sistema mostri l’anteprima visiva
del documento analizzato. & RF-84 & NI \\ \hline
TS-F-085 & Verifica\G che il sistema permetta di modificare il
destinatario associato al documento. & RF-85 & NI \\ \hline
TS-F-086 & Verifica\G che il sistema permetta di modificare la
tipologia del documento. & RF-86 & NI \\ \hline
TS-F-087 & Verifica\G che il sistema ricalcoli la percentuale di
confidenza dopo modifiche manuali. & RF-87 & NI \\ \hline
TS-F-088 & Verifica\G che il sistema permetta la visualizzazione della lista delle
informazioni sui destinatari estratti. & RF-88 & NI \\ \hline
TS-F-089 & Verifica\G che il sistema permetta la visualizzazione dei
dettagli di un singolo destinatario in lista. & RF-89 & NI \\ \hline
TS-F-090 & Verifica\G che il sistema permetta la visualizzazione del codice fiscale
del destinatario. & RF-90 & NI \\ \hline
TS-F-091 & Verifica\G che il sistema permetta la visualizzazione della matricola del
destinatario. & RF-91 & NI \\ \hline
TS-F-092 & Verifica\G che il sistema permetta la visualizzazione del reparto del
destinatario. & RF-92 & NI \\ \hline
TS-F-093 & Verifica\G che il sistema permetta la visualizzazione del
nome/cognome del destinatario. & RF-93 & NI \\ \hline
TS-F-094 & Verifica\G che il sistema notifichi se nessun
destinatario è stato riconosciuto. & RF-94 & NI \\ \hline
TS-F-095 & Verifica\G che il sistema permetta la visualizzazione dello storico
completo dei documenti processati. & RF-95 & NI \\ \hline
TS-F-096 & Verifica\G che il sistema notifichi l’assenza di
documenti nello storico. & RF-96 & NI \\ \hline
TS-F-097 & Verifica\G che il sistema permetta la visualizzazione dei dettagli di un
elemento nello storico documenti. & RF-97 & NI \\ \hline
TS-F-098 & Verifica\G che il sistema permetta la visualizzazione della percentuale di
confidenza dell’analisi nello storico. & RF-98 & NI \\ \hline
TS-F-099 & Verifica\G che il sistema permetta la visualizzazione dell’appartenenza
alle liste di distribuzione. & RF-99 & NI \\ \hline
TS-F-100 & Verifica\G che il sistema permetta la visualizzazione dello stato di
elaborazione del documento. & RF-100 & NI \\ \hline
TS-F-101 & Verifica\G che il sistema permetta di caricare un
template\G di messaggio esistente. & RF-101 & NI \\ \hline
TS-F-102 & Verifica\G che il sistema permetta di modificare
l’oggetto del messaggio. & RF-102 & NI \\ \hline
TS-F-103 & Verifica\G che il sistema permetta di modificare il
testo del corpo del messaggio. & RF-103 & NI \\ \hline
TS-F-104 & Verifica\G che il sistema permetta di salvare il
messaggio corrente come nuovo template\G. & RF-104 & NI \\ \hline
TS-F-105 & Verifica\G che il sistema permetta di eliminare un
template\G di messaggio. & RF-105 & NI \\ \hline
TS-F-106 & Verifica\G che il sistema permetta di visualizzare la
lista dei template\G di messaggio disponibili. & RF-106 & NI \\ \hline
TS-F-107 & Verifica\G che il sistema permetta di visualizza-
re un elemento della lista dei template\G di
messaggio disponibili. & RF-107 & NI \\ \hline
TS-F-108 & Verifica\G che il sistema permetta di visualizzare
l’oggetto del template\G. & RF-108 & NI \\ \hline
TS-F-109 & Verifica\G che il sistema permetta di visualizzare il
testo del template\G. & RF-109 & NI \\ \hline
TS-F-110 & Verifica\G che il sistema permetta di visualizzare il
codice del template\G. & RF-110 & NI \\ \hline
TS-F-111 & Verifica\G che il sistema permetta l’invio del
documento e del messaggio associato. & RF-111 & NI \\ \hline
TS-F-112 & Verifica\G che il sistema permetta di allegare
ulteriore contenuto al messaggio. & RF-112 & NI \\ \hline
TS-F-113 & Verifica\G che il sistema permetta di pianifica-
re l’invio del documento e del messaggio
associato. & RF-113 & NI \\ \hline
TS-F-114 & Verifica\G che il sistema permetta il filtraggio della
lista dei documenti analizzati. & RF-114 & NI \\ \hline
TS-F-115 & Verifica\G che il sistema mostri la lista dei
documenti aggiornata in base ai filtri. & RF-115 & NI \\ \hline
TS-F-116 & Verifica\G che il sistema permetta il filtraggio della
lista dei destinatari. & RF-116 & NI \\ \hline
TS-F-117 & Verifica\G che il sistema mostri la lista dei
destinatari aggiornata in base ai filtri. & RF-117 & NI \\ \hline
TS-F-118 & Verifica\G che il sistema permetta il filtraggio della
lista dello storico documenti. & RF-118 & NI \\ \hline
TS-F-119 & Verifica\G che il sistema mostri la lista dello storico
documenti aggiornata in base ai filtri. & RF-119 & NI \\ \hline
TS-F-120 & Verifica\G che il sistema permetta di mostrare
l’audit di un documento nello storico. & RF-120 & NI \\ \hline
TS-F-121 & Verifica\G che il sistema permetta la visualizzazione della dashboard\G
con i dati di analytics per l’AI\G Assistant. & RF-121 & NI \\ \hline
TS-F-122 & Verifica\G che il sistema permetta la visualizzazione del numero totale
di prompt\G generati. & RF-122 & NI \\ \hline
TS-F-123 & Verifica\G che il sistema permetta la visualizzazione del rating medio
dei prompt\G generati. & RF-123 & NI \\ \hline
TS-F-124 & Verifica\G che il sistema permetta la visualizzazione del numero di
rigenerazioni effettuate. & RF-124 & NI \\ \hline
TS-F-125 & Verifica\G che il sistema permetta la visualizzazione delle statistiche sui
toni più utilizzati. & RF-125 & NI \\ \hline
TS-F-126 & Verifica\G che il sistema permetta la visualizzazione delle statistiche
sugli stili più utilizzati. & RF-126 & NI \\ \hline
TS-F-127 & Verifica\G che il sistema permetta la visualizzazione della dashboard\G
con i dati di analytics per l’AI Co-Pilot\G. & RF-127 & NI \\ \hline
TS-F-128 & Verifica\G che il sistema permetta la visualizzazione della confidenza
media delle analisi documenti. & RF-128 & NI \\ \hline
TS-F-129 & Verifica\G che il sistema permetta la visualizzazione della percentuale di
interventi manuali necessari. & RF-129 & NI \\ \hline
TS-F-130 & Verifica\G che il sistema permetta la visualizzazione dell’accuratezza del
mapping dei dati. & RF-130 & NI \\ \hline
TS-F-131 & Verifica\G che il sistema permetta la visualizzazione dei tempi medi di
analisi dei documenti. & RF-131 & NI \\ \hline
TS-F-132 & Verifica\G che il sistema permetta di filtrare i dati
di analytics per periodo temporale. & RF-132 & NI \\ \hline

\caption{Test\G di Sistema - Requisiti Funzionali\G}
\label{tab:test-sistema-funzionali}
\end{longtable}



\newpage

\subsubsection{Test\G di Sistema - Requisiti Prestazionali\G}
\begin{longtable}{|p{2.5cm}|p{8cm}|p{2.5cm}|p{1.5cm}|}
\hline
\rowcolor{primaryblue!30} 
\textbf{Codice} & \textbf{Descrizione} & \textbf{Riferimento} & \textbf{Stato} \\
\hline
\endhead

TS-P-001 & Verifica\G che il sistema generi contenuti testuali tramite AI\G (Assistant) entro 5 secondi per testi fino a 500 parole & RP-01 & NI \\ \hline
TS-P-002 & Verifica\G che il sistema classifichi e partizioni documenti PDF (Co-Pilot) entro 3 secondi per pagina & RP-02 & NI \\ \hline
TS-P-003 & Verifica\G che il tempo di risposta dell'interfaccia\G utente per operazioni standard sia inferiore a 2 secondi & RP-03 & NI \\ \hline
TS-P-004 & Verifica\G che il sistema supporti l'upload di file PDF fino a 20 MB & RP-04 & NI \\ \hline
TS-P-005 & Verifica\G che la Dashboard\G di Analytics carichi le statistiche entro 3 secondi per dataset fino a 1000 documenti & RP-05 & NI \\ \hline
TS-P-006 & Verifica\G che il sistema garantisca una disponibilità del 99\% durante l'orario lavorativo (8:00-18:00) & RP-06 & NI \\ \hline
TS-P-007 & Verifica\G che il sistema sia in grado di processare almeno 50 documenti in parallelo senza degrado prestazionale & RP-07 & NI \\ \hline
TS-P-008 & Verifica\G che il tempo di estrazione OCR\G per documenti
scansionati sia inferiore a 5 secondi
per pagina & RP-08 & NI \\ \hline

\caption{Test\G di Sistema per Requisiti Prestazionali\G}
\label{tab:test-sistema-prestazionali}
\end{longtable}

\subsubsection{Test\G di Sistema - Requisiti di Qualità\G}
\begin{longtable}{|p{2.5cm}|p{8cm}|p{2.5cm}|p{1.5cm}|}
\hline
\rowcolor{primaryblue!30} 
\textbf{Codice} & \textbf{Descrizione} & \textbf{Riferimento} & \textbf{Stato} \\
\hline
\endhead

TS-Q-001 & Verifica\G che sia presente la documentazione dell'analisi dei Requisiti\G completa (diagrammi e descrizioni Use Case) & RQ-01 & NI \\ \hline
TS-Q-002 & Verifica\G che il way of working\G sia rispettato secondo gli standard definiti nelle Norme di Progetto\G. & RQ-02 & NI \\ \hline
TS-Q-003 & Verifica\G che il prodotto\G passi tutti i test\G con la
copertura concordata con la proponente\G. & RQ-03 & NI \\ \hline
TS-Q-004 & Verifica\G che il codice sia documentato secon-
do le linee guida descritte in Norme di
Progetto\G. & RQ-04 & NI \\ \hline
TS-Q-005 & Verifica\G che il codice sia versionato con appositi strumenti di controllo versione, comprensivo di istruzioni di setup. & RQ-05 & NI \\ \hline
TS-Q-006 & Verifica\G che ci sia il report finale di integrazione e suggerimenti di evoluzione. & RQ-06 & NI \\ \hline

\caption{Test\G di Sistema per Requisiti di Qualità\G}
\label{tab:test-sistema-qualita}
\end{longtable}

\newpage

\subsubsection{Test\G di Sistema - Requisiti di Vincolo\G}
\begin{longtable}{|p{2.5cm}|p{8cm}|p{2.5cm}|p{1.5cm}|}
\hline
\rowcolor{primaryblue!30} 
\textbf{Codice} & \textbf{Descrizione} & \textbf{Riferimento} & \textbf{Stato} \\
\hline
\endhead

TS-V-001 & Verifica\G che il sistema utilizzi Git come sistema di controllo versione & RV-01 & NI \\ \hline
TS-V-002 & Verifica\G che API\G e Backend\G siano sviluppati in Ruby on Rails\G & RV-02 & NI \\ \hline
TS-V-003 & Verifica\G che il database sia PostgreSQL\G. & RV-03 & NI \\ \hline
TS-V-004 & Verifica\G che il Frontend sia sviluppato utilizzando il framework Angular\G. & RV-04 & NI \\ \hline
TS-V-005 & Verifica\G che la gestione dei modelli AI\G sia
implementata utilizzando AWS Bedrock\G. & RV-05 & NI \\ \hline
TS-V-006 & Verifica\G che eventuali background jobs siano gestiti con
Sidekiq e PWA\G con Next.js. & RV-06 & NI \\ \hline

\caption{Test\G di Sistema per Requisiti di Vincolo\G}
\label{tab:test-sistema-vincolo}
\end{longtable}

\subsection{Test\G di Accettazione}
I test\G di accettazione validano il sistema rispetto agli scenari d'uso (Use Case) previsti, assicurando che l'utente possa completare i flussi di lavoro principali.

\begin{longtable}{|p{2.0cm}|p{8.5cm}|p{2.5cm}|p{2.5cm}|}
  \hline
  \rowcolor{primaryblue!30} 
  \textbf{Codice} & \textbf{Descrizione} & \textbf{Riferimento} & \textbf{Stato} \\
  \hline
  \endhead
  
  % --- UTENTE GENERICO ---
  \textbf{TA-001} & Verifica\G che un utente non registrato possa completare la procedura di registrazione (Happy Path). & UC-0A & Non Impl. \\
  \hline
  \textbf{TA-002} & Verifica\G che il sistema impedisca la registrazione con dati non validi o email già esistente. & UC-0A (Scenari alternativi) & Non Impl. \\
  \hline
  \textbf{TA-003} & Verifica\G che l'utente possa effettuare il Login e il Logout. & UC-0B, UC-0G & Non Impl. \\
  \hline
  \textbf{TA-004} & Verifica\G che l'utente possa visualizzare e modificare il proprio profilo e cambiare la password. & UC-0C, UC-0D & Non Impl. \\
  \hline
  
  % --- AMMINISTRATORE ---
  \textbf{TA-005} & Verifica\G che l'Amministratore\G possa consultare la lista utenti e visualizzare i dettagli di un singolo utente. & UC-0E & Non Impl. \\
  \hline
  \textbf{TA-006} & Verifica\G che l'Amministratore\G possa modificare il ruolo di un utente. & UC-0F & Non Impl. \\
  \hline
  
  % --- HR MANAGER (ASSISTANT) ---
  \textbf{TA-007} & Verifica\G che l'HR Manager possa configurare una richiesta (Prompt\G, Tono, Lunghezza) e generare un contenuto. & UC-1A, UC-1B, UC-1C & Non Impl. \\
  \hline
  \textbf{TA-008} & Verifica\G che l'HR Manager possa visualizzare, copiare e modificare il testo generato dall'AI\G. & UC-1D, UC-1E & Non Impl. \\
  \hline
  \textbf{TA-009} & Verifica\G che l'HR Manager possa valutare (Feedback) o scartare un contenuto generato. & UC-1F, UC-1N & Non Impl. \\
  \hline
  \textbf{TA-010} & Verifica\G il salvataggio automatico nello storico e la possibilità di recuperare generazioni passate. & UC-1O & Non Impl. \\
  \hline
  
  % --- OPERATORE CDL (CO-PILOT) ---
  \textbf{TA-011} & Verifica\G che l'Operatore possa caricare un documento PDF e avviare l'analisi automatica. & UC-2A & Non Impl. \\
  \hline
  \textbf{TA-012} & Verifica\G che il sistema estragga correttamente i dati e li mostri all'operatore. & UC-2B & Non Impl. \\
  \hline
  \textbf{TA-013} & Verifica\G lo scenario\G "Human-in-the-loop": l'operatore corregge manualmente un dato estratto errato e conferma. & UC-2D, UC-2E & Non Impl. \\
  \hline
  \textbf{TA-014} & Verifica\G gestione template\G: creazione, modifica e utilizzo di un template\G di messaggio. & UC-2I & Non Impl. \\
  \hline
  \textbf{TA-015} & Verifica\G il flusso di invio: selezione destinatari, associazione documento e invio email (o pianificazione). & UC-2L, UC-2O & Non Impl. \\
  \hline
  
  % --- DATA ANALYST ---
  \textbf{TA-016} & Verifica\G che il Data Analyst\G possa consultare le Dashboard\G e filtrare le metriche per periodo temporale. & UC-3A, UC-3B & Non Impl. \\
  \hline

    \caption{Test\G di Accettazione}
    \label{tab:test-accettazione}
\end{longtable}

\section{Cruscotto\G di Valutazione}
Di seguito verranno mostrate le misurazioni effettuate durante il periodo che va 
dall'aggiudicazione del capitolato\G
sino alla Requirements and Technology Baseline (RTB)\G. 

\subsection{MPC01 e MPC02 - Earned Value (EV)\G e Planned Value (PV)\G}
 \begin{figure}[H]
     \centering
     \includegraphics[width=0.9\textwidth]{grafici/MPC01_02.png}
     \caption{Grafico per periodo di MPC01 e MPC02}
 \end{figure}

 Dal grafico si osserva che l'andamento del Valore Guadagnato (\textit{Earned Value} - EV\G) segue fedelmente quello del Valore Pianificato (\textit{Planned Value} - PV\G), con un trend crescente che culmina nel sesto sprint\G, in corrispondenza del completamento delle attività per la \textit{Requirements and Technology Baseline} (RTB)\G.

\newpage

\subsection{MPC03 e MPC07 - Actual cost (AC)\G e Estimate to complete (ETC)\G}
\begin{figure}[H]
  \centering
  \includegraphics[width=0.9\textwidth]{grafici/MPC03_07.png}
  \caption{Grafico per periodo di MPC03 e MPC07}
\end{figure}
L'andamento della metrica MPC03 (\textit{Actual Cost\G}) mostra una crescita costante dei costi sostenuti, in linea con l'intensificazione delle attività produttive durante la fase di \textit{Requirements and Technology Baseline} (RTB)\G. Tale incremento, culminato nel sesto sprint\G, rispecchia fedelmente la pianificazione temporale definita nel Piano di Progetto\G, dove il maggior carico di lavoro (e quindi di spesa) era previsto proprio nelle settimane antecedenti la consegna della candidatura\G.

Parallelamente, la metrica MPC07 (\textit{Estimate to Complete\G}) evidenzia una progressiva diminuzione del budget residuo necessario per il completamento del progetto\G. Questo trend inverso conferma che le risorse sono state consumate coerentemente con l'avanzamento dei lavori, avvicinando il progetto\G al traguardo della \textit{Product Baseline} (PB)\G senza generare extra-costi imprevisti.

\newpage

\subsection{MPC04 e MPC05 - Cost Performance Index (CPI)\G e Schedule performance Index\G}
\begin{figure}[H]
  \centering
  \includegraphics[width=0.9\textwidth]{grafici/MPC04_05.png}
  \caption{Grafico per periodo di MPC04 e MPC05}
\end{figure}

L'analisi del \textit{Cost Performance Index} (CPI)\G mostra un percorso di netta crescita. Il progetto\G è iniziato con un indice inferiore alle aspettative, a causa delle difficoltà iniziali.

Dopo l'investimento iniziale, il processo produttivo è diventato altamente sostenibile, permettendo di recuperare il budget consumato.
Parallelamente, lo \textit{Schedule Performance Index} (SPI)\G si è mantenuto stabile e vicino al valore ideale per tutto il periodo, garantendo il rispetto delle scadenze per la candidatura\G.

\newpage

\subsection{MPC06 - Estimated at completion (EAC)}
\begin{figure}[H]
  \centering
  \includegraphics[width=0.9\textwidth]{grafici/MPC06.png}
  \caption{Grafico per periodo di MPC06}
\end{figure}

L'andamento del costo stimato a finire (Estimated at Completion - EAC) racconta chiaramente il percorso di ottimizzazione intrapreso dal team. Il progetto\G ha attraversato una fase iniziale critica durante i primi sprint\G, in cui la stima dei costi\G finali superava sensibilmente il budget stanziato. Questa proiezione negativa era la diretta conseguenza delle difficoltà iniziali che avevano ridotto l'indice di efficienza\G CPI\G. Successivamente si è innescato un trend di recupero costante. 

\newpage

\subsection{MPC08 - Time Estimate At Completion\G}
\begin{figure}[H]
  \centering
  \includegraphics[width=0.9\textwidth]{grafici/MPC08.png}
  \caption{Grafico per periodo di MPC08}
\end{figure}
L'andamento della stima temporale a finire (\textit{Time Estimate At Completion\G}) conferma la solidità della pianificazione iniziale. La proiezione della data di completamento per la fase di \textit{Requirements and Technology Baseline} (RTB)\G è rimasta sostanzialmente invariata lungo tutto l'arco temporale osservato.

\newpage

\subsection{MPC09 - Requirements Stability Index (RSI)\G}
\begin{figure}[H]
  \centering
  \includegraphics[width=0.9\textwidth]{grafici/MPC09.png}
  \caption{Grafico per periodo di MPC09}
\end{figure}
L'indice di stabilità dei requisiti (\textit{Requirements Stability Index\G}) mostra un andamento che riflette fedelmente il ciclo di vita dell'Analisi dei Requisiti\G.
Nello Sprint\G 2 si registra un picco negativo significativo. Tale valore, apparentemente critico, è in realtà l'indicatore di una intensa attività produttiva. Partendo da un set iniziale di requisiti, il team ha effettuato un'opera di espansione e dettaglio massiccia. Matematicamente, ciò ha portato il numero delle modifiche a superare il numero dei requisiti iniziali, generando l'indice negativo. Superata la fase critica di definizione, l'indice è risalito rapidamente. 

\newpage

\subsection{MPC10 - Indice di Gulpease\G}
\begin{figure}[H]
  \centering
  \includegraphics[width=0.9\textwidth]{grafici/MPC10.png}
  \caption{Grafico per periodo di MPC10}
\end{figure}
In linea generale, il gruppo BugBusters ha posto grande attenzione alla redazione della documentazione: l'obiettivo primario è sempre stato quello di produrre elaborati che fossero non solo corretti tecnicamente, ma anche facilmente fruibili da tutti gli stakeholder\G.
Dall'analisi dei dati emerge una disparità nei valori di leggibilità tra le diverse tipologie di documenti, dovuta alla natura intrinseca del loro contenuto.
I dati sulla leggibilità mostrano una chiara differenza tra i documenti. L'Analisi dei Requisiti\G supera abbondantemente la soglia ottima grazie alla scelta di usare frasi brevi e semplici, ideali per farsi capire chiaramente dal cliente. Al contrario, il Glossario\G e le Norme di Progetto\G rimangono sotto la soglia minima per motivi strutturali: il primo è penalizzato dalla presenza di parole tecniche molto lunghe, mentre le seconde richiedono un linguaggio formale e rigoroso che non può essere semplificato oltre un certo limite senza perdere di precisione.

Il gruppo si impegna comunque, nelle prossime iterazioni, a raffinare ulteriormente la sintassi di tali documenti per migliorarne la leggibilità senza comprometterne il rigore formale.

\newpage


\subsection{MPC13 - Quality metrics satisfied}
\begin{figure}[H]
  \centering
  \includegraphics[width=0.9\textwidth]{grafici/MPC13.png}
  \caption{Grafico per periodo di MPC13}
\end{figure}
L'andamento della percentuale di metriche soddisfatte offre una sintesi efficace della maturazione qualitativa del progetto\G. Il primo sprint\G ha risentito della bassa efficienza\G economica iniziale (CPI\G sotto soglia), mentre nei successivi due sprint\G è stato l'Indice di Stabilità dei Requisiti (RSI)\G a mancare l'obiettivo, a causa della necessaria fase di espansione dell'Analisi dei Requisiti\G.  
Superata la fase di assestamento, il trend ha mostrato un miglioramento netto. A partire dal quarto sprint\G, il team ha raggiunto una stabilità su tutti i fronti monitorati (Costi, Tempi, Documentazione e Processi), mantenendo l'indicatore vicino al valore minimo accettabile fino al termine della fase RTB\G. Questo risultato conferma che le misure correttive adottate sono state risolutive, portando il processo produttivo a un livello di affidabilità\G ottimale proprio nel momento decisivo della candidatura\G.


\newpage

\subsection{MPC14 - Time Efficiency}
\begin{figure}[H]
  \centering
  \includegraphics[width=0.9\textwidth]{grafici/MPC14.png}
  \caption{Grafico per periodo di MPC14}
\end{figure}
L'analisi dell'efficienza\G temporale mostra un andamento notevolmente stabile. Questa costanza è un segnale positivo: indica che il team è riuscito a mantenere un rapporto equilibrato tra il lavoro produttivo (stesura documenti, sviluppo) e le ore di gestione (riunioni, auto-formazione), senza mai farsi sopraffare dall'overhead organizzativo.
Le lievi flessioni registrate nella fase centrale (Sprint\G 4 e 5) sono fisiologiche e riconducibili principalmente al rischio\G riguardante la sovrapposizione con sessione d'esami e alla necessità di maggiori confronti interni per la riorganizzazione dell'Analisi dei Requisiti\G.
Il picco positivo raggiunto nel sesto sprint\G testimonia la capacità del gruppo di massimizzare la produttività nelle settimane decisive per la chiusura della candidatura\G.






\end{document}