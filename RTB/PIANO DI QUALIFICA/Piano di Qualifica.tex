\documentclass[a4paper,11pt]{article}
   
\newcommand{\CurrentVersion}{0.0.6} % ultima versione, da cambiare ad ogni push significativo

\usepackage[utf8]{inputenc}
\usepackage[T1]{fontenc}
\usepackage[italian]{babel}
\usepackage[margin=2.5cm]{geometry}
\usepackage{graphicx}
\usepackage{grffile}
\usepackage{booktabs}
\usepackage{setspace}
\usepackage{titlesec}
\usepackage{float}
\usepackage{ifthen}
\usepackage[table]{xcolor}
\usepackage{tabularx}
\usepackage{needspace}
\usepackage{tcolorbox}
\usepackage{enumitem}
\usepackage[titles]{tocloft}
\usepackage[colorlinks=true,linkcolor=black,urlcolor=blue,citecolor=blue]{hyperref}

% Macro
% genera la stringa "\noindent (Riferimento alla tabella decisioni:
% \hyperref[RTB4]{RTB4})", si usa facendo \refDecisione{NomeLabel}{Testo}
\newcommand{\refDecisione}[2]{%
    \noindent (\textbf{Riferimento alla tabella decisioni: \hyperref[#1]{#2}})%
}


\definecolor{primaryblue}{RGB}{0,102,204}
\definecolor{secondaryblue}{RGB}{51,153,255}
\definecolor{lightgray}{RGB}{245,245,245}
\definecolor{darkgray}{RGB}{100,100,100}

\titleformat{\section}
 {\Large\bfseries\color{primaryblue}}
 {\thesection}{1em}{}

\titleformat{\subsection}
 {\large\bfseries\color{primaryblue}} % Sottosezione: colore secondaryblue
 {\thesubsection}{1em}{}

\titleformat{\subsubsection}
 {\normalsize\bfseries\color{secondaryblue}} % Sotto-sottosezione: colore secondaryblue
 {\thesubsubsection}{1em}{}

% per il footer con il numero di pagina
\usepackage{fancyhdr}
\usepackage{lastpage} % per ottenere il numero dell'ultima pagina da mettere nel footer


\usepackage{ltablex} %per far andare a capo le tabelle
\keepXColumns

\renewcommand{\sectionmark}[1]{\markright{#1}}
\newcommand{\G}{\textsubscript{\scalebox{0.6}{\textbf{G}}}}


% Macro tabella metrica (tabella uniforme, non spezzata e con stile coerente)
\newcommand{\MetricTable}[1]{
  \Needspace{10\baselineskip}% evita spezzamenti sgradevoli
  \begin{table}[H]
  \centering
  \arrayrulecolor{primaryblue}
  \begin{tabularx}{\linewidth}{|>{\raggedright\arraybackslash}p{2.5cm}|X|>{\raggedright\arraybackslash}p{3cm}|>{\raggedright\arraybackslash}p{3cm}|}
  \hline
  \rowcolor{primaryblue!40}
  		\textbf{\color{white} Metrica} & \textbf{\color{white} Descrizione} & \textbf{\color{white} Valore accettazione} & \textbf{\color{white} Valore ideale} \\
  \hline
  \ignorespaces
  #1
  \end{tabularx}
  \end{table}
}



\setlength{\parskip}{4pt}
\setlength{\parindent}{0pt}

\setlist[itemize]{leftmargin=*,itemsep=3pt}
\setlist[enumerate]{leftmargin=*,itemsep=3pt}

\graphicspath{{./}{../assets/images/}{./images/}}

\begin{document}

%configurazione per il footer
\pagestyle{fancy}
\fancyhf{} % pulisce tutti i campi del header e footer

% Header: sinistra e destra
\fancyhead[L]{Gruppo 4 - BugBusters} % sinistra
\fancyhead[R]{Piano di Qualifica\G}   % destra

\fancyfoot[L]{ \thepage\ di \pageref{LastPage}} %definisce il formato del footer
\fancyfoot[R]{ \nouppercase{\rightmark}} % nome della sezione


\renewcommand{\headrulewidth}{0pt} % rimuove la linea dell'header
\renewcommand{\footrulewidth}{0pt} % se vuoi anche togliere eventuale linea del footer

% abilitare numerazione e TOC fino al livello "paragraph" (subsubsubsection)
\setcounter{secnumdepth}{4}
\setcounter{tocdepth}{4}
% formattazione del nuovo livello per avere aspetto coerente
\titleformat{\paragraph}[block]{\normalsize\bfseries\color{secondaryblue}}{\theparagraph}{1em}{}
% alias comodo per usare "subsubsubsection"
\newcommand{\subsubsubsection}{\paragraph}

\begin{center}
  \IfFileExists{../../assets/Logo.jpg}{%
    \includegraphics[width=6cm,height=3cm,keepaspectratio]{../../assets/Logo.jpg} \\[0.8cm]
  }{%
    \fbox{\parbox[c][2.5cm][c]{6cm}{\centering Logo non trovato\\(../../assets/Logo.jpg)}}\\[0.5cm]
  }
  
  {\Large\bfseries\color{primaryblue} BugBusters}\\[0.5cm]

  {\Huge\bfseries\color{primaryblue} Piano di Qualifica\G}\\[0.3cm]
  {\Large\color{secondaryblue} Versione \CurrentVersion}\\[0.8cm]
\end{center}

\begin{center}
\begin{tcolorbox}[colback=lightgray,colframe=primaryblue,width=0.85\textwidth,arc=3mm,boxrule=0.5pt]
% Usa tabularx con una colonna fissa per l'etichetta e una colonna X per il contenuto
\begin{tabularx}{\linewidth}{@{}>{\raggedright\arraybackslash}p{3.5cm}>{\raggedright\arraybackslash}X@{}}
	{Stato} & In redazione \\
	{Responsabile} &  \\
	{Verificatore} &  \\
	{Redattori} & Luca Slongo \\
	{Distribuzione} & BugBusters, Eggon, Prof. Tullio Vardanega, Prof. Riccardo Cardin \\
\end{tabularx}
\end{tcolorbox}
\end{center}

\vspace{0.5cm}

\begin{center}
\begin{tcolorbox}[colback=secondaryblue!10,colframe=secondaryblue,width=0.9\textwidth,arc=3mm,boxrule=0.8pt,title={\bfseries Descrizione}]
Piano di Qualifica\G del Team BugBusters per il Capitolato C5 proposto da Eggon, che ha l'obiettivo di far rispettare uno standard di qualità per il codice e rispettare i requisiti funzionali\G prestabiliti.\end{tcolorbox}
\end{center}

\newpage

% Registro delle Modifiche (pagina 2)
\section*{Registro delle Modifiche}

\arrayrulecolor{primaryblue}
{\footnotesize
\begin{tabularx}{\textwidth}{|>{\raggedright\arraybackslash}p{1.5cm}|>{\raggedright\arraybackslash}p{2cm}|X|>{\raggedright\arraybackslash}p{2cm}|>{\raggedright\arraybackslash}p{2cm}|>{\raggedright\arraybackslash}p{2cm}|}
\hline
\rowcolor{primaryblue!40}
\textbf{\color{white} Versione} & \textbf{\color{white} Data} & \textbf{\color{white} Descrizione} & \textbf{\color{white} Redatto} & \textbf{\color{white} Verificato} & \textbf{\color{white} Approvato} \\
\hline
\CurrentVersion & 19/01/2026 & Aggiornamento Test, aggiunti test di sistema Casi Limite e integrazione, cambiato alcune metriche di prodotto & Marco Piro & - & - \\
\hline
0.0.5 & 15/01/2026 & Aggiornamento Test, aggiunti test di sistema prestazionali, eliminata metrica errori ortografici & Marco Piro & - & - \\
\hline
0.0.4 & 11/01/2026 & Aggiunto contenuto alla sezione 5 & Marco Piro & - & - \\
\hline
0.0.3 & 04/01/2026 & Aggiunte sezioni 4 e 5 & Marco Piro & - & - \\
\hline
0.0.2 & 29/12/2025 & Aggiunta Test di Sistema e di Accettazione & Marco Piro & - & - \\
\hline
0.0.1 & 03/12/2025 & Prima stesura del documento & Luca Slongo & - & - \\
\hline
\end{tabularx}
}

\newpage

% Indice cliccabile
\setcounter{tocdepth}{3} % Mostra fino al livello di sottosezione (2)
\tableofcontents
\listoftables
\listoffigures

\newpage
\section{Introduzione}

\subsection{Scopo del documento}
Il presente documento, denominato \textit{Piano di Qualifica}, ha lo scopo di definire le strategie, le procedure e le metriche adottate dal gruppo \textit{BugBusters} per garantire la qualità del prodotto software e dei processi produttivi relativi al progetto C5 (NEXUM), proposto dall'azienda \textit{Eggon}.

In particolare, questo documento si prefigge di:
\begin{itemize}
    \item \textbf{Definire gli obiettivi di qualità:} specificare i target qualitativi per il processo di sviluppo (efficienza, stabilità) e per il prodotto software (funzionalità, affidabilità, manutenibilità), in conformità con gli standard ISO/IEC 12207 e ISO/IEC 9126;
    \item \textbf{Identificare le metriche:} selezionare gli indicatori quantitativi più idonei per monitorare il raggiungimento degli obiettivi, fissando per ciascuno le soglie di accettazione e di ottimalità;
    \item \textbf{Pianificare le attività di verifica e validazione:} descrivere le metodologie di test (unità, integrazione, sistema, accettazione) e le procedure di analisi statica del codice e della documentazione;
    \item \textbf{Monitorare l'andamento del progetto:} fornire un resoconto puntuale (cruscotto di valutazione) delle misurazioni effettuate durante le varie fasi del ciclo di vita, permettendo al team di individuare tempestivamente criticità e attuare azioni correttive (miglioramento continuo).
\end{itemize}

\subsection{Glossario}
Al fine di evitare ambiguità e garantire una comprensione uniforme della terminologia utilizzata, è stato redatto un documento esterno denominato \textit{Glossario}.
I termini tecnici, gli acronimi e le parole con un significato specifico all'interno del progetto sono contrassegnati nel testo da una "G" in pedice (es. parola\G). La loro definizione completa è consultabile nel \textit{Glossario}.

\subsection{Riferimenti}

\subsubsection{Riferimenti normativi}
\begin{itemize}
    \item \textbf{Capitolato d'appalto C5 - NEXUM (Eggon):}\\
    \url{https://www.math.unipd.it/~tullio/IS-1/2025/Progetto/C5.pdf}
    
    \item \textbf{Norme di Progetto (vX.Y.Z):}\\
    Documento interno del gruppo \textit{BugBusters} che definisce le regole, i ruoli e le procedure operative.
    
    \item \textbf{Regolamento del progetto didattico:} \\
    \url{https://www.math.unipd.it/~tullio/IS-1/2025/Dispense/PD1.pdf}
\end{itemize}

\subsubsection{Riferimenti informativi}
\begin{itemize}
    \item \textbf{Glossario (vX.Y.Z):}\\
    Documento interno del gruppo \textit{BugBusters} contenente le definizioni dei termini tecnici.
    
    \item \textbf{Standard ISO/IEC 12207:1995:}\\
    \textit{Information technology - Software life cycle processes}.\\
    \url{https://www.math.unipd.it/~tullio/IS-1/2009/Approfondimenti/ISO_12207-1995.pdf}
    
    \item \textbf{Standard ISO/IEC 9126:}\\
    \textit{Software engineering - Product quality}.\\
    \url{https://it.wikipedia.org/wiki/ISO/IEC_9126}
    
    \item \textbf{Slide del corso di Ingegneria del Software:} \\
    Materiale didattico fornito dai docenti Prof. Tullio Vardanega e Prof. Riccardo Cardin.
\end{itemize}
\newpage

\section{Obiettivi stabiliti per la qualità}

È fondamentale stabilire degli obiettivi da raggiungere per assicurare la qualità prefissata del prodotto. 
Questo documento definisce i valori di accettazione e ottimalità delle metriche secondo gli standard definiti 
nelle Norme di Progetto.

\subsection{Qualità di processo}
Un indicatore della qualità di un prodotto è il metodo con cui è stato sviluppato. Se il processo di sviluppo segue 
delle linee guida ben definite, esso favorisce la buona riuscita del prodotto. Come stabilito nelle Norme di Progetto, 
nel nostro way of working abbiamo adottato lo Standard ISO/IEC 12207:1995 adattandolo alle nostre esigenze e a quelle 
del progetto. 

\subsubsection{Processi primari}

I processi primari sono quelle attività che iniziano o eseguono lo sviluppo, l'operazione o la manutenzione di prodotti software. Essi rappresentano le componenti fondamentali del ciclo di vita del progetto e sono suddivisi nelle seguenti categorie:
\subsubsubsection{Fornitura}

\MetricTable{
MPC01 & Earned value (EV) & $\geq 0$ & $\leq$ EAC \\
\hline
MPC02 & Planned value (PV) & $\geq 0$ & $\leq$ Budget at completion (BAC) \\
\hline
MPC03 & Actual cost (AC) & $\geq 0$ & $\leq$ EAC \\
\hline
MPC04 & Cost Performance Index (CPI) & $\geq 0.9$ & $\geq 1.0$ \\
\hline
MPC05 & Schedule Performance Index & $\geq 0$ & 1 \\
\hline
MPC06 & Estimated at completion (EAC) & $\pm 5\%$ rispetto al (BAC) & Budget at completion (BAC) \\
\hline
MPC07 & Estimate to complete (ETC) & $\geq 0$ & $\leq$ EAC \\
\hline
MPC08 & Time Estimate At Completion & $\geq 0$ & $\leq$ Durata pianificata \\
\hline}



\subsubsubsection{Sviluppo}
\MetricTable{
MPC09 & Requirements Stability Index (EAC) & $\geq 70\%$ & 100\% \\
\hline}


\subsubsection{Processi di supporto}
\subsubsubsection{Documentazione}
\MetricTable{
MPC10 & Indice di Gulpease del documento & $\geq 60\%$ & $\geq 80\%$ \\
\hline
}
\subsubsubsection{Verifica}
\MetricTable{
MPC11 & Code Coverage & $\geq 80\%$ & $100\%$ \\
\hline
MPC12 & Test Success Rate & $100\%$ & $100\%$ \\
\hline}
\subsubsubsection{Gestione della qualità}
\MetricTable{
MPC13 & Quality metrics satisfied & $\geq 80\%$ & $100\%$ \\
\hline}

\subsubsection{Processi organizzativi}
\subsubsubsection{Gestione dei processi}
\MetricTable{
MPC14 & Time Efficiency & $\geq 50\%$ & $100\%$ \\
\hline}



\subsection{Qualità di prodotto}

Per qualità di prodotto si intende una valutazione complessiva del software sia dal punto di vista 
funzionale sia dal punto di vista strutturale. Il codice deve adempiere alle funzionalità prestabilite in 
modo efficiente e semplice, e al contempo essere manutenibile, affidabile e portabile. Il gruppo ha aderito allo 
standard ISO/IEC 9126 per garantire il rispetto di queste caratteristiche fondamentali, affinchè il prodotto 
sviluppato sia di alta qualità.

\subsubsection{Funzionalità}
\MetricTable{
MPD01 & Requisiti obbligatori soddisfatti & $100\%$ & $100\%$ \\
\hline
MPD02 & Requisiti desiderabili soddisfatti & $0\%$ & $100\%$ \\
\hline
MPD03 & Requisiti opzionali soddisfatti & $0\%$ & $100\%$ \\
\hline
MPD04 & AI Acceptance Rate (Rating $\geq$ 3/5) & $\geq 60\%$ & $\geq 80\%$ \\
\hline}

\subsubsection{Affidabilità}
\MetricTable{
MPD05 & Branch Coverage & $\geq 70\%$ & $\geq 85\%$ \\
\hline
MPD06 & Defect Density & $\leq 3$ / KLOC & $\leq 1$ / KLOC \\
\hline}

\subsubsection{Efficienza}
\MetricTable{
MPD07 & UI Response Time (Interfaccia) & $\leq 2$ sec & $\leq 0.5$ sec \\
\hline
MPD08 & Core Response Time (AI/Upload) & $\leq 5$ sec & $\leq 3$ sec \\
\hline}

\subsubsection{Usabilità}
\MetricTable{
MPD09 & Click Count (Funzioni principali) & $\leq 5$ click & $\leq 3$ click \\
\hline
MPD10 & User Error Rate (Errori validazione) & $\leq 10\%$ & $\leq 5\%$ \\
\hline}

\subsubsection{Mantenibilità}
\MetricTable{
MPD11 & Blocker Code Smells & $0$ & $0$ \\
\hline
MPD12 & Cyclomatic complexity (per metodo) & $\leq 15$ & $\leq 10$ \\
\hline
MPD13 & Comment Intensity & $\geq 10\%$ & $\geq 20\%$ \\
\hline}

\subsubsection{Portabilità}
\MetricTable{
MPD14 & Supported Browsers (Test passati) & $100\%$ (Desktop) & $100\%$ (All devices) \\
\hline}



\section{Metodi di testing}
La strategia di verifica e validazione adottata dal gruppo \textit{BugBusters} è definita in dettaglio nel documento \textit{Norme di Progetto}. Tale strategia mira a garantire che ogni rilascio software sia conforme ai requisiti specificati e privo di difetti critici, adottando un approccio proattivo alla qualità.

In conformità con lo standard ISO/IEC 12207, il processo di verifica si articola in due metodologie complementari:
\begin{itemize}
    \item \textbf{Analisi Statica:} Revisione della documentazione e del codice sorgente (tramite tool di analisi automatica e code review) per individuare difetti prima dell'esecuzione.
    \item \textbf{Analisi Dinamica:} Esecuzione del software a diversi livelli di granularità per verificarne il comportamento rispetto ai risultati attesi.
\end{itemize}

I test dinamici pianificati per il progetto seguono un approccio incrementale (piramide dei test), partendo dalle singole unità logiche fino alla validazione dell'intero sistema integrato:
\begin{itemize}
    \item \textbf{Test di Unità (TU):} Verifica delle singole componenti software (classi, metodi).
    \item \textbf{Test di Integrazione (TI):} Verifica delle interfacce e del flusso dati tra i moduli.
    \item \textbf{Test di Sistema (TS):} Verifica del comportamento globale del sistema rispetto ai requisiti funzionali e prestazionali.
    \item \textbf{Test di Accettazione (TA):} Validazione finale del prodotto rispetto agli scenari d'uso attesi dall'utente (Usecase).
\end{itemize}

\subsection{Test di Integrazione}
I test di integrazione verificano la corretta comunicazione tra i sottosistemi e i moduli definiti nell'architettura, assicurando che i dati fluiscano correttamente tra Frontend, Backend e servizi AI.

\begin{longtable}{|p{2.2cm}|p{7cm}|p{3.5cm}|}
	\hline
	\rowcolor{primaryblue!30} 
	\textbf{Codice} & \textbf{Descrizione Interfaccia} & \textbf{Moduli Coinvolti} \\
	\hline
	\endhead
	
	\textbf{TI-001} & Verifica scambio dati JSON tra Frontend e Backend (API REST) e gestione codici errore HTTP. & Standalone $\leftrightarrow$ Server \\
	\hline
	\textbf{TI-002} & Verifica invio prompt e ricezione risposta dal servizio LLM esterno (gestione timeout e token). & AI Assistant $\leftrightarrow$ LLM API \\
	\hline
	\textbf{TI-003} & Verifica corretta persistenza dei documenti analizzati e relativo recupero per lo storico. & Co-Pilot $\leftrightarrow$ Database \\
	\hline
	\textbf{TI-004} & Verifica aggregazione dati per le dashboard statistiche (query su dataset analizzati). & Database $\leftrightarrow$ Analytics \\
	\hline
	\textbf{TI-005} & Verifica passaggio del testo estratto (OCR/Parser) al motore di analisi AI. & Parser $\leftrightarrow$ Co-Pilot Logic \\
	\hline
\end{longtable}

\subsection{Test di Sistema}
La seguente tabella dettaglia i test di sistema (System Tests) progettati per verificare la completa copertura dei requisiti funzionali (RF). I codici dei test seguono il formato \textbf{TS-F-XX} (Test Sistema Funzionale).

\begin{longtable}{|p{2.2cm}|p{7cm}|p{3.5cm}|p{2.3cm}|}
	\hline
	\rowcolor{primaryblue!30} 
	\textbf{Codice} & \textbf{Descrizione} & \textbf{Requisiti Coperti} & \textbf{Stato} \\
	\hline
	\endhead
	
	% --- MODULO 0: AUTH & USER ---
	\multicolumn{4}{|c|}{\cellcolor{lightgray}\textbf{Modulo 0: App Standalone e Gestione Utenti}} \\
	\hline
	\textbf{TS-F-001} & Verifica registrazione utente non autenticato (inserimento dati validi: email, password, anagrafica). & RF-001, RF-002, RF-003, RF-004, RF-005, RF-006, RF-007 & Non Impl. \\
	\hline
	\textbf{TS-F-002} & Verifica gestione errori in registrazione (email errata, password debole, utente duplicato). & RF-008, RF-009, RF-010, RF-011, RF-012, RF-013 & Non Impl. \\
	\hline
	\textbf{TS-F-003} & Verifica procedura di Login e gestione credenziali errate. & RF-014, RF-015, RF-016 & Non Impl. \\
	\hline
	\textbf{TS-F-004} & Verifica visualizzazione dati profilo utente. & RF-017, RF-018, RF-019, RF-020, RF-021, RF-022, RF-023 & Non Impl. \\
	\hline
	\textbf{TS-F-005} & Verifica modifica dati profilo e gestione "unsaved changes". & RF-024, RF-025 & Non Impl. \\
	\hline
	\textbf{TS-F-006} & Verifica funzionalità Admin (lista utenti, dettagli, cambio ruolo). & RF-026, RF-027, RF-028, RF-029, RF-030, RF-031 & Non Impl. \\
	\hline
	\textbf{TS-F-007} & Verifica procedura di Logout. & RF-032 & Non Impl. \\
	\hline
	
	% --- MODULO 1: AI ASSISTANT ---
	\multicolumn{4}{|c|}{\cellcolor{lightgray}\textbf{Modulo 1: AI Assistant Generativo}} \\
	\hline
	\textbf{TS-F-008} & Verifica generazione contenuto testuale (prompt, tono, stile). & RF-033, RF-034, RF-035, RF-036 & Non Impl. \\
	\hline
	\textbf{TS-F-009} & Verifica visualizzazione storico generazioni e dettagli entry. & RF-037, RF-038, RF-039, RF-040, RF-041, RF-042, RF-044, RF-045 & Non Impl. \\
	\hline
	\textbf{TS-F-010} & Verifica anteprima e modifica contenuto generato. & RF-046, RF-047, RF-048, RF-049, RF-050 & Non Impl. \\
	\hline
	\textbf{TS-F-011} & Verifica azioni su contenuto (annulla, rating, scarta, duplica). & RF-051, RF-052, RF-053, RF-057, RF-058 & Non Impl. \\
	\hline
	\textbf{TS-F-012} & Verifica persistenza dati (salvataggio nel DB). & RF-059 & Non Impl. \\
	\hline
	\textbf{TS-F-013} & Verifica filtri storico e rigenerazione contenuto. & RF-054, RF-055, RF-056 & Non Impl. \\
	\hline
	
	% --- MODULO 2: AI CO-PILOT ---
	\multicolumn{4}{|c|}{\cellcolor{lightgray}\textbf{Modulo 2: AI Co-Pilot (Consulenti del Lavoro)}} \\
	\hline
	\textbf{TS-F-014} & Verifica upload e analisi documento con metadati input. & RF-060, RF-061, RF-062, RF-063, RF-064 & Non Impl. \\
	\hline
	\textbf{TS-F-015} & Verifica visualizzazione lista documenti e metadati estratti. & RF-065, RF-066, RF-067, RF-068, RF-069, RF-070, RF-071, RF-072, RF-073, RF-074, RF-075, RF-076 & Non Impl. \\
	\hline
	\textbf{TS-F-016} & Verifica Human-in-the-loop (modifica campi e confidenza). & RF-077, RF-078, RF-079, RF-080 & Non Impl. \\
	\hline
	\textbf{TS-F-017} & Verifica estrazione dati destinatari (CF, matricola, reparto). & RF-081, RF-082, RF-083, RF-084, RF-085, RF-086, RF-087 & Non Impl. \\
	\hline
	\textbf{TS-F-018} & Verifica storico processamento e stati documento. & RF-088, RF-089, RF-090, RF-091, RF-092, RF-093 & Non Impl. \\
	\hline
	\textbf{TS-F-019} & Verifica gestione CRUD Template messaggi. & RF-094, RF-095, RF-096, RF-097, RF-098, RF-099, RF-100, RF-101, RF-102, RF-103 & Non Impl. \\
	\hline
	\textbf{TS-F-020} & Verifica invio/pianificazione documenti e messaggi. & RF-104, RF-105, RF-106 & Non Impl. \\
	\hline
	\textbf{TS-F-021} & Verifica funzionalità di filtro liste (doc/destinatari). & RF-107, RF-108, RF-109, RF-110, RF-111, RF-112 & Non Impl. \\
	\hline

	% --- MODULO 3: ANALYTICS ---
	\multicolumn{4}{|c|}{\cellcolor{lightgray}\textbf{Modulo 3: Analytics}} \\
	\hline
	\textbf{TS-F-022} & Verifica Dashboard KPI Assistant (prompt, rating, usage). & RF-113, RF-114, RF-115, RF-116, RF-117, RF-118 & Non Impl. \\
	\hline
	\textbf{TS-F-023} & Verifica Dashboard KPI Co-Pilot (confidenza, manual fix). & RF-119, RF-120, RF-121, RF-122, RF-123 & Non Impl. \\
	\hline
	\textbf{TS-F-024} & Verifica filtro temporale dashboard. & RF-124 & Non Impl. \\
	\hline

\end{longtable}

\subsection{Test di Accettazione}
La seguente tabella descrive i test di accettazione, volti a validare il sistema rispetto agli scenari d'uso (Use Case) previsti.

\begin{longtable}{|p{2.0cm}|p{8.5cm}|p{2.5cm}|p{2.5cm}|}
	\hline
	\rowcolor{primaryblue!30} 
	\textbf{Codice} & \textbf{Descrizione} & \textbf{Riferimento} & \textbf{Stato} \\
	\hline
	\endhead
	
	\textbf{TA-001} & L'utente non registrato riesce a creare un account e successivamente ad effettuare l'accesso al sistema standalone. & UC-0A, UC-0B & Non Implementato \\
	\hline
	\textbf{TA-002} & L'utente riesce a gestire il proprio profilo (visualizzazione e modifica) e ad effettuare il logout. & UC-0C, UC-0D, UC-0G & Non Implementato \\
	\hline
	\textbf{TA-003} & L'Amministratore riesce a visualizzare la lista utenti e a modificare i ruoli assegnati. & UC-0E, UC-0F & Non Implementato \\
	\hline
	\textbf{TA-004} & L'HR Manager riesce a generare un contenuto AI valido partendo da un prompt, visualizzarlo nello storico e modificarlo. & UC-1A...1F & Non Implementato \\
	\hline
	\textbf{TA-005} & L'HR Manager riesce a salvare un contenuto generato o a scartarlo se non soddisfacente. & UC-1O, UC-1N & Non Implementato \\
	\hline
	\textbf{TA-006} & L'Operatore CdL riesce a caricare un documento, analizzarlo con l'AI Co-Pilot e visualizzare i dati estratti. & UC-2A, UC-2B & Non Implementato \\
	\hline
	\textbf{TA-007} & L'Operatore CdL riesce a validare manualmente i dati estratti (Human-in-the-loop) modificando destinatari o tipologie errate. & UC-2D, UC-2E, UC-2F & Non Implementato \\
	\hline
	\textbf{TA-008} & L'Operatore CdL riesce a creare un messaggio usando un template e ad inviarlo (o pianificarlo) insieme al documento. & UC-2I, UC-2L, UC-2O & Non Implementato \\
	\hline
	\textbf{TA-009} & Il Data Analyst riesce a visualizzare le metriche di performance (KPI) sia per il modulo Assistant che per il Co-Pilot filtrando per data. & UC-3A, UC-3B, UC-3C & Non Implementato \\
	\hline

\end{longtable}

\subsection{Test di Sistema - Prestazionali}
La seguente tabella descrive i test progettati per verificare i requisiti prestazionali (RP) definiti nell'Analisi dei Requisiti. I codici seguono il formato \textbf{TS-P-XX}.

\begin{longtable}{|p{2.2cm}|p{7cm}|p{3.5cm}|p{2.3cm}|}
	\hline
	\rowcolor{primaryblue!30} 
	\textbf{Codice} & \textbf{Descrizione} & \textbf{Requisiti Coperti} & \textbf{Stato} \\
	\hline
	\endhead
	
	\textbf{TS-P-001} & Verifica che la generazione di contenuti testuali (Modulo Assistant) avvenga entro 5 secondi per testi fino a 500 parole. & RP-01 & Non Impl. \\
	\hline
	\textbf{TS-P-002} & Verifica che la classificazione e il partizionamento dei PDF (Modulo Co-Pilot) avvenga entro 3 secondi per pagina. & RP-02 & Non Impl. \\
	\hline
	\textbf{TS-P-003} & Verifica che il tempo di risposta dell'interfaccia utente per operazioni standard sia inferiore a 2 secondi. & RP-03 & Non Impl. \\
	\hline
	\textbf{TS-P-004} & Verifica il supporto all'upload di file PDF fino a una dimensione massima di 20 MB. & RP-04 & Non Impl. \\
	\hline
	\textbf{TS-P-005} & Verifica che la Dashboard di Analytics carichi le statistiche entro 3 secondi (dataset < 1000 doc). & RP-05 & Non Impl. \\
	\hline
	\textbf{TS-P-006} & Verifica (tramite monitoraggio o stress test) la disponibilità del sistema del 99\% in orario lavorativo (8:00-18:00). & RP-06 & Non Impl. \\
	\hline
	\textbf{TS-P-007} & Verifica la capacità del sistema di processare almeno 50 documenti in parallelo senza degrado delle performance. & RP-07 & Non Impl. \\
	\hline
  \textbf{TS-P-008} & Verifica che l'estrazione OCR per documenti scansionati avvenga entro 5 secondi per pagina. & RP-07 & Non Impl. \\
  \hline

\end{longtable}

\subsection{Test di Sistema - Casi Limite (Boundary)}
La seguente tabella descrive i test volti a verificare la robustezza del sistema ai limiti operativi definiti nei requisiti prestazionali (RP).

\begin{longtable}{|p{2.2cm}|p{7cm}|p{3.5cm}|p{2.3cm}|}
	\hline
	\rowcolor{primaryblue!30} 
	\textbf{Codice} & \textbf{Descrizione} & \textbf{Riferimento (RP)} & \textbf{Stato} \\
	\hline
	\endhead
	
	\textbf{TS-B-001} & Verifica il rifiuto controllato di un upload di file PDF superiore a 20 MB (es. 21 MB) con messaggio di errore user-friendly. & RP-04 & Non Impl. \\
	\hline
	\textbf{TS-B-002} & Verifica il comportamento del sistema con un prompt AI superiore a 500 parole (troncamento automatico o avviso bloccante). & RP-01 & Non Impl. \\
	\hline
	\textbf{TS-B-003} & Verifica gestione upload di file vuoti (0 KB) o file corrotti/non PDF. & RP-04 & Non Impl. \\
	\hline
	\textbf{TS-B-004} & Stress Test: Verifica stabilità del sistema con invio simultaneo di 55 documenti (superiore al limite di 50). & RP-07 & Non Impl. \\
	\hline

\end{longtable}

\section{Cruscotto di Valutazione}
Di seguito verranno mostrate le misurazioni effettuate durante il periodo che va 
dall'aggiudicazione del capitolatoG sino alla Requirements and Technology BaselineG (RTB). 
Le misurazioni presenti saranno prese durante lo svolgimento delle attività per la Product BaselineG (PB).

\subsection{MPC01 e MPC02 - Earned Value (EV) e Planned Value (PV)}
Grafico

\subsection{MPC03 e MPC07 - Actual cost (AC) e Estimate to complete (ETC)}
Grafico

\subsection{MPC04 e MP05 - Cost Performance Index (CPI) e Schedule performance Index}
Grafico

\subsection{MPC06 - Estimated at completion (EAC)}
Grafico

\subsection{MPC08 - Time Estimate At Completion}
Grafico

\subsection{MPC09 - Requirements Stability Index (RSI)}
Grafico

\subsection{MPC10 - Indice di Gulpease}
Grafico


\subsection{MPC13 - Quality metrics satisfied}
Grafico

\subsection{MPC14 - Time Efficiency}
Grafico



\section{Iniziative di miglioramento}
L'ottimizzazione costante dei processi costituisce un pilastro fondamentale per la riuscita del progetto. Di seguito vengono esposte le problematiche operative riscontrate e le relative strategie di risoluzione adottate per superare tali ostacoli.

\subsection{Valutazioni sull'organizzazione}
\begin{longtable}{|p{3cm}|p{6cm}|p{5cm}|}
  \hline
  \textbf{Area} & \textbf{Problema Riscontrato} & \textbf{Contromisura Adottata} \\
  \hline
  \textbf{Tracciabilità} & L'assenza di un sistema di monitoraggio puntuale delle attività ostacola il flusso produttivo e compromette l'efficacia della programmazione operativa. & Adozione della funzionalità 'Issues' di GitHub per ottimizzare il controllo operativo e la supervisione dei flussi di lavoro. \\
  \hline
  \textbf{Controllo delle modifiche} & Operare senza un flusso di Pull Request obbligatorio riduce la stabilità del software e la tracciabilità delle integrazioni. & Attivazione della Branch Protection per inibire i push diretti e rendere mandatorio il processo di Code Review tramite Pull Request. \\
  \hline
  \textbf{Rendicontazione delle ore} & La mancanza di un sistema strutturato per la rendicontazione delle ore lavorate limita la capacità di analisi dell'efficienza e della produttività del team. & Implementazione di un foglio di calcolo condiviso per la registrazione puntuale delle ore dedicate alle attività progettuali, facilitando così il monitoraggio e l'analisi delle performance. \\
  \hline

  \endhead 
\hline
\end{longtable}


\subsection{Valutazioni sui ruoli}
\begin{longtable}{|p{3cm}|p{6cm}|p{5cm}|}
  \hline
  \textbf{Ruolo} & \textbf{Problema Riscontrato} & \textbf{Contromisura Adottata} \\
  \hline
  \endhead
  \textbf{Tutti i ruoli} & Per ottimizzare le ore produttive nelle fasi avanzate, è necessario superare il blocco bisettimanale dei ruoli, che attualmente lascia lacune nella copertura delle attività. & L'assegnazione dei ruoli diviene flessibile su base settimanale, previo allineamento tra le parti, mantenendo l'incompatibilità nel ricoprire funzioni simultanee. \\
\hline
\end{longtable}


\subsection{Valutazioni sugli strumenti}
\begin{longtable}{|p{3cm}|p{6cm}|p{5cm}|}
  \hline
  \textbf{Strumento} & \textbf{Problema Riscontrato} & \textbf{Contromisura Adottata} \\
  \hline
  \endhead
  \textbf{Titolo Problema} & Problema da descrivere & Contromisura spiegata \\
\hline
\end{longtable}


\subsection{Considerazioni finali}
L'iterazione e l'apprendimento continuo guidano la qualità del nostro lavoro. Le retrospettive ci hanno permesso di affinare i processi e aumentare l'efficienza. 
Il team resta focalizzato sul problem-solving proattivo per mantenere alti gli standard produttivi.



\end{document}