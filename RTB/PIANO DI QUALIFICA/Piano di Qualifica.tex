\documentclass[a4paper,11pt]{article}
   
\newcommand{\CurrentVersion}{0.0.9} % ultima versione, da cambiare ad ogni push significativo

\usepackage[utf8]{inputenc}
\usepackage[T1]{fontenc}
\usepackage[italian]{babel}
\usepackage[margin=2.5cm]{geometry}
\usepackage{graphicx}
\usepackage{grffile}
\usepackage{booktabs}
\usepackage{setspace}
\usepackage{titlesec}
\usepackage{float}
\usepackage{ifthen}
\usepackage[table]{xcolor}
\usepackage{tabularx}
\usepackage{needspace}
\usepackage{tcolorbox}
\usepackage{enumitem}
\usepackage[titles]{tocloft}
\usepackage[colorlinks=true,linkcolor=black,urlcolor=blue,citecolor=blue]{hyperref}

% Macro
% genera la stringa "\noindent (Riferimento alla tabella decisioni:
% \hyperref[RTB4]{RTB4})", si usa facendo \refDecisione{NomeLabel}{Testo}
\newcommand{\refDecisione}[2]{%
    \noindent (\textbf{Riferimento alla tabella decisioni: \hyperref[#1]{#2}})%
}


\definecolor{primaryblue}{RGB}{0,102,204}
\definecolor{secondaryblue}{RGB}{51,153,255}
\definecolor{lightgray}{RGB}{245,245,245}
\definecolor{darkgray}{RGB}{100,100,100}

\titleformat{\section}
 {\Large\bfseries\color{primaryblue}}
 {\thesection}{1em}{}

\titleformat{\subsection}
 {\large\bfseries\color{primaryblue}} % Sottosezione: colore secondaryblue
 {\thesubsection}{1em}{}

\titleformat{\subsubsection}
 {\normalsize\bfseries\color{secondaryblue}} % Sotto-sottosezione: colore secondaryblue
 {\thesubsubsection}{1em}{}

% per il footer con il numero di pagina
\usepackage{fancyhdr}
\usepackage{lastpage} % per ottenere il numero dell'ultima pagina da mettere nel footer


\usepackage{ltablex} %per far andare a capo le tabelle
\keepXColumns

\renewcommand{\sectionmark}[1]{\markright{#1}}
\renewcommand{\G}{\textsubscript{\scalebox{0.6}{\textbf{G}}}}


% Macro tabella metrica (tabella uniforme, non spezzata e con stile coerente)
\newcommand{\MetricTable}[1]{
  \Needspace{10\baselineskip}% evita spezzamenti sgradevoli
  \begin{table}[H]
  \centering
  \arrayrulecolor{primaryblue}
  \begin{tabularx}{\linewidth}{|>{\raggedright\arraybackslash}p{2.5cm}|X|>{\raggedright\arraybackslash}p{3cm}|>{\raggedright\arraybackslash}p{3cm}|}
  \hline
  \rowcolor{primaryblue!40}
  		\textbf{\color{white} Metrica} & \textbf{\color{white} Descrizione} & \textbf{\color{white} Valore accettazione} & \textbf{\color{white} Valore ideale} \\
  \hline
  \ignorespaces
  #1
  \end{tabularx}
  \end{table}
}



\setlength{\parskip}{4pt}
\setlength{\parindent}{0pt}

\setlist[itemize]{leftmargin=*,itemsep=3pt}
\setlist[enumerate]{leftmargin=*,itemsep=3pt}

\graphicspath{{./}{../assets/images/}{./images/}}

\begin{document}

%configurazione per il footer
\pagestyle{fancy}
\fancyhf{} % pulisce tutti i campi del header e footer

% Header: sinistra e destra
\fancyhead[L]{Gruppo 4 - BugBusters} % sinistra
\fancyhead[R]{Piano di Qualifica\G}   % destra

\fancyfoot[L]{ \thepage\ di \pageref{LastPage}} %definisce il formato del footer
\fancyfoot[R]{ \nouppercase{\rightmark}} % nome della sezione


\renewcommand{\headrulewidth}{0pt} % rimuove la linea dell'header
\renewcommand{\footrulewidth}{0pt} % se vuoi anche togliere eventuale linea del footer

% abilitare numerazione e TOC fino al livello "paragraph" (subsubsubsection)
\setcounter{secnumdepth}{4}
\setcounter{tocdepth}{4}
% formattazione del nuovo livello per avere aspetto coerente
\titleformat{\paragraph}[block]{\normalsize\bfseries\color{secondaryblue}}{\theparagraph}{1em}{}
% alias comodo per usare "subsubsubsection"
\newcommand{\subsubsubsection}{\paragraph}

\begin{center}
  \IfFileExists{../../assets/Logo.jpg}{%
    \includegraphics[width=6cm,height=3cm,keepaspectratio]{../../assets/Logo.jpg} \\[0.8cm]
  }{%
    \fbox{\parbox[c][2.5cm][c]{6cm}{\centering Logo non trovato\\(../../assets/Logo.jpg)}}\\[0.5cm]
  }
  
  {\Large\bfseries\color{primaryblue} BugBusters}\\[0.5cm]

  {\Huge\bfseries\color{primaryblue} Piano di Qualifica\G}\\[0.3cm]
  {\Large\color{secondaryblue} Versione \CurrentVersion}\\[0.8cm]
\end{center}

\begin{center}
\begin{tcolorbox}[colback=lightgray,colframe=primaryblue,width=0.85\textwidth,arc=3mm,boxrule=0.5pt]
% Usa tabularx con una colonna fissa per l'etichetta e una colonna X per il contenuto
\begin{tabularx}{\linewidth}{@{}>{\raggedright\arraybackslash}p{3.5cm}>{\raggedright\arraybackslash}X@{}}
	{Stato} & In redazione \\
	{Verificatore\G} &  \\
	{Redattori} & Luca slongo, Marco Piro \\
	{Distribuzione} & BugBusters, Eggon, Prof. Tullio Vardanega, Prof. Riccardo Cardin \\
\end{tabularx}
\end{tcolorbox}
\end{center}

\vspace{0.5cm}

\begin{center}
\begin{tcolorbox}[colback=secondaryblue!10,colframe=secondaryblue,width=0.9\textwidth,arc=3mm,boxrule=0.8pt,title={\bfseries Descrizione}]
Piano di Qualifica\G del Team BugBusters per il Capitolato\G C5 proposto da Eggon, che ha l'obiettivo di far rispettare uno standard di qualità\G per il codice e rispettare i requisiti funzionali\G prestabiliti.\end{tcolorbox}
\end{center}

\newpage

% Registro delle Modifiche (pagina 2)
\section*{Registro delle Modifiche}

\arrayrulecolor{primaryblue}
{\footnotesize
\begin{tabularx}{\textwidth}{|>{\raggedright\arraybackslash}p{1.5cm}|>{\raggedright\arraybackslash}p{2cm}|X|>{\raggedright\arraybackslash}p{2cm}|>{\raggedright\arraybackslash}p{2cm}|>{\raggedright\arraybackslash}p{2cm}|}
\hline
\rowcolor{primaryblue!40}
\textbf{\color{white} Versione} & \textbf{\color{white} Data} & \textbf{\color{white} Descrizione} & \textbf{\color{white} Redatto} & \textbf{\color{white} Verificato} & \textbf{\color{white} Approvato} \\
\hline
0.0.9 & 06/02/2026 & Aggiunti ai termini presenti nel Glossario\textsubscript{\scalebox{0.6}{\textbf{G}}} la G & Alberto Autiero & - & - \\

\hline
0.0.8 & 06/02/2026 & Aggiunta descrizione grafici metriche & Marco Piro & - & - \\
\hline
0.0.7 & 04/02/2026 & Aggiunti grafici metriche, aggiornamento Test\G, rimossa matrice di Tracciamento & Marco Piro & - & - \\
\hline
0.0.6 & 19/01/2026 & Aggiornamento Test\G, aggiunti test\G di sistema Casi Limite e integrazione, cambiato alcune metriche di prodotto\G. Aggiunta matrice di Tracciamento & Marco Piro & - & - \\
\hline
0.0.5 & 15/01/2026 & Aggiornamento Test\G, aggiunti test\G di sistema prestazionali, eliminata metrica errori ortografici & Marco Piro & - & - \\
\hline
0.0.4 & 11/01/2026 & Aggiunto contenuto alla sezione 5 & Marco Piro & - & - \\
\hline
0.0.3 & 04/01/2026 & Aggiunte sezioni 4 e 5 & Marco Piro & - & - \\
\hline
0.0.2 & 29/12/2025 & Aggiunta Test\G di Sistema e di Accettazione & Marco Piro & - & - \\
\hline
0.0.1 & 03/12/2025 & Prima stesura del documento & Luca Slongo & - & - \\
\hline
\end{tabularx}
}

\newpage

% Indice cliccabile
\setcounter{tocdepth}{3} % Mostra fino al livello di sottosezione (2)
\tableofcontents
\listoftables
\listoffigures

\newpage
\section{Introduzione}

\subsection{Scopo del documento}
Il presente documento, denominato \textit{Piano di Qualifica\G}, ha lo scopo di definire le strategie, le procedure e le metriche adottate dal gruppo \textit{BugBusters} per garantire la qualità\G del prodotto\G software e dei processi produttivi relativi al progetto\G C5 (NEXUM), proposto dall'azienda \textit{Eggon}.

In particolare, questo documento si prefigge di:
\begin{itemize}
    \item \textbf{Definire gli obiettivi di qualità\G:} specificare i target qualitativi per il processo di sviluppo (efficienza\G, stabilità) e per il prodotto\G software (funzionalità\G, affidabilità\G, manutenibilità\G), in conformità con gli standard ISO/IEC 12207 e ISO/IEC 9126;
    \item \textbf{Identificare le metriche:} selezionare gli indicatori quantitativi più idonei per monitorare il raggiungimento degli obiettivi, fissando per ciascuno le soglie di accettazione e di ottimalità;
    \item \textbf{Pianificare le attività di verifica\G e validazione\G:} descrivere le metodologie di test\G (unità, integrazione, sistema, accettazione) e le procedure di analisi statica del codice e della documentazione;
    \item \textbf{Monitorare l'andamento del progetto\G:} fornire un resoconto puntuale (cruscotto\G di valutazione) delle misurazioni effettuate durante le varie fasi del ciclo di vita, permettendo al team di individuare tempestivamente criticità e attuare azioni correttive (miglioramento continuo).
\end{itemize}

\subsection{Glossario\G}
Al fine di evitare ambiguità e garantire una comprensione uniforme della terminologia utilizzata, è stato redatto un documento esterno denominato \textit{Glossario\G}.
I termini tecnici, gli acronimi e le parole con un significato specifico all'interno del progetto\G sono contrassegnati nel testo da una "G" in pedice (es. parola). La loro definizione completa è consultabile nel \textit{Glossario\G}.

\subsection{Riferimenti}

\subsubsection{Riferimenti normativi}
\begin{itemize}
    \item \textbf{Capitolato\G d'appalto C5 - NEXUM (Eggon):}\\
    \url{https://www.math.unipd.it/~tullio/IS-1/2025/Progetto/C5.pdf}
    
    \item \textbf{Norme di Progetto\G (vX.Y.Z):}\\
    Documento interno del gruppo \textit{BugBusters} che definisce le regole, i ruoli e le procedure operative.
    
    \item \textbf{Regolamento del progetto\G didattico:} \\
    \url{https://www.math.unipd.it/~tullio/IS-1/2025/Dispense/PD1.pdf}
\end{itemize}

\subsubsection{Riferimenti informativi}
\begin{itemize}
    \item \textbf{Glossario\G (vX.Y.Z):}\\
    Documento interno del gruppo \textit{BugBusters} contenente le definizioni dei termini tecnici.
    
    \item \textbf{Standard ISO/IEC 12207:1995:}\\
    \textit{Information technology - Software life cycle processes}.\\
    \url{https://www.math.unipd.it/~tullio/IS-1/2009/Approfondimenti/ISO_12207-1995.pdf}
    
    \item \textbf{Standard ISO/IEC 9126:}\\
    \textit{Software engineering - Product quality}.\\
    \url{https://it.wikipedia.org/wiki/ISO/IEC_9126}
    
    \item \textbf{Slide del corso di Ingegneria del Software:} \\
    Materiale didattico fornito dai docenti Prof. Tullio Vardanega e Prof. Riccardo Cardin.
\end{itemize}
\newpage

\section{Obiettivi stabiliti per la qualità\G}

È fondamentale stabilire degli obiettivi da raggiungere per assicurare la qualità\G prefissata del prodotto\G. 
Questo documento definisce i valori di accettazione e ottimalità delle metriche secondo gli standard definiti 
nelle Norme di Progetto\G.

\subsection{Qualità\G di processo}
Un indicatore della qualità\G di un prodotto\G è il metodo con cui è stato sviluppato. Se il processo di sviluppo segue 
delle linee guida ben definite, esso favorisce la buona riuscita del prodotto\G. Come stabilito nelle Norme di Progetto\G, 
nel nostro way of working\G abbiamo adottato lo Standard ISO/IEC 12207:1995 adattandolo alle nostre esigenze e a quelle 
del progetto\G. 

\subsubsection{Processi primari}

I processi primari sono quelle attività che iniziano o eseguono lo sviluppo, l'operazione o la manutenzione di prodotti software. Essi rappresentano le componenti fondamentali del ciclo di vita del progetto\G e sono suddivisi nelle seguenti categorie:
\subsubsubsection{Fornitura}

\MetricTable{
MPC01 & Earned value (EV)\G & $\geq 0$ & $\leq$ EAC \\
\hline
MPC02 & Planned value (PV)\G & $\geq 0$ & $\leq$ Budget at completion (BAC) \\
\hline
MPC03 & Actual cost (AC)\G & $\geq 0$ & $\leq$ EAC \\
\hline
MPC04 & Cost Performance Index (CPI)\G & $\geq 0.9$ & 1 \\
\hline
MPC05 & Schedule Performance Index (SPI)\G & $\geq 0.9$ & 1 \\
\hline
MPC06 & Estimated at completion (EAC) & $\pm 5\%$ rispetto al (BAC) & Budget at completion (BAC) \\
\hline
MPC07 & Estimate to complete (ETC)\G & $\geq 0$ & $\leq$ EAC \\
\hline
MPC08 & Time Estimate At Completion (TEAC)\G & $\geq 0$ & $\leq$ Durata pianificata \\
\hline}



\subsubsubsection{Sviluppo}
\MetricTable{
MPC09 & Requirements Stability Index\G & $\geq 80\%$ & 100\% \\
\hline}


\subsubsection{Processi di supporto}
\subsubsubsection{Documentazione}
\MetricTable{
MPC10 & Indice di Gulpease\G del documento & $\geq 60\%$ & $\geq 80\%$ \\
\hline
}
\subsubsubsection{Verifica\G}
\MetricTable{
MPC11 & Code Coverage & $\geq 80\%$ & $100\%$ \\
\hline
MPC12 & Test\G Success Rate & $100\%$ & $100\%$ \\
\hline}
\subsubsubsection{Gestione della qualità\G}
\MetricTable{
MPC13 & Quality metrics satisfied & $\geq 80\%$ & $100\%$ \\
\hline}

\subsubsection{Processi organizzativi}
\subsubsubsection{Gestione dei processi}
\MetricTable{
MPC14 & Time Efficiency & $\geq 50\%$ & $100\%$ \\
\hline}



\subsection{Qualità\G di prodotto\G}

Per qualità\G di prodotto\G si intende una valutazione complessiva del software sia dal punto di vista 
funzionale sia dal punto di vista strutturale. Il codice deve adempiere alle funzionalità\G prestabilite in 
modo efficiente e semplice, e al contempo essere manutenibile, affidabile e portabile. Il gruppo ha aderito allo 
standard ISO/IEC 9126 per garantire il rispetto di queste caratteristiche fondamentali, affinchè il prodotto\G 
sviluppato sia di alta qualità\G.

\subsubsection{Funzionalità\G}
\MetricTable{
MPD01 & Requisiti obbligatori soddisfatti & $100\%$ & $100\%$ \\
\hline
MPD02 & Requisiti desiderabili soddisfatti & $0\%$ & $100\%$ \\
\hline
MPD03 & Requisiti opzionali soddisfatti & $0\%$ & $100\%$ \\
\hline
MPD04 & AI\G Acceptance Rate (Rating $\geq$ 3/5) & $\geq 60\%$ & $\geq 80\%$ \\
\hline}

\subsubsection{Affidabilità\G}
\MetricTable{
MPD05 & Branch Coverage & $\geq 70\%$ & $\geq 85\%$ \\
\hline
MPD06 & Defect Density & $\leq 3$ / KLOC & $\leq 1$ / KLOC \\
\hline}

\subsubsection{Efficienza\G}
\MetricTable{
MPD07 & UI Response Time (Interfaccia\G) & $\leq 2$ sec & $\leq 0.5$ sec \\
\hline
MPD08 & Core Response Time (AI\G/Upload) & $\leq 5$ sec & $\leq 3$ sec \\
\hline}

\subsubsection{Usabilità}
\MetricTable{
MPD09 & Click Count (Funzioni principali) & $\leq 5$ click & $\leq 3$ click \\
\hline
MPD10 & User Error Rate (Errori validazione\G) & $\leq 10\%$ & $\leq 5\%$ \\
\hline}

\subsubsection{Mantenibilità}
\MetricTable{
MPD11 & Blocker Code Smells & $0$ & $0$ \\
\hline
MPD12 & Cyclomatic complexity (per metodo) & $\leq 15$ & $\leq 10$ \\
\hline
MPD13 & Comment Intensity & $\geq 10\%$ & $\geq 20\%$ \\
\hline}

\subsubsection{Portabilità}
\MetricTable{
MPD14 & Supported Browsers (Test\G passati) & $100\%$ (Desktop) & $100\%$ (All devices) \\
\hline}



\section{Metodi di testing}
La strategia di verifica\G e validazione\G adottata dal gruppo \textit{BugBusters} mira a garantire che ogni rilascio software sia conforme ai requisiti specificati e privo di difetti critici.
I test\G dinamici pianificati seguono un approccio incrementale (piramide dei test\G), partendo dalle singole unità logiche fino alla validazione\G dell'intero sistema integrato.

\subsection{Riepilogo dei Requisiti}
La seguente tabella riassume la distribuzione dei requisiti definiti nell'Analisi dei Requisiti\G, che costituiscono la base per la pianificazione dei test\G.

\begin{table}[H]
\centering
\caption{Riepilogo dei requisiti}
\label{tab:riepilogo-requisiti}
\arrayrulecolor{primaryblue}
\begin{tabularx}{0.8\textwidth}{|X|c|c|c|c|}
\hline
\rowcolor{primaryblue!40}
\textbf{\color{white} Tipologia} & \textbf{\color{white} Obbligatorio} & \textbf{\color{white} Desiderabile} & \textbf{\color{white} Opzionale} & \textbf{\color{white} Totale} \\
\hline
Funzionali & 103 & 0 & 25 & 128 \\
\hline
Prestazionali & 5 & 3 & 0 & 8 \\
\hline
Qualità\G & 6 & 0 & 0 & 6 \\
\hline
Vincolo & 5 & 0 & 1 & 6 \\
\hline
\textbf{Totale} & \textbf{119} & \textbf{3} & \textbf{26} & \textbf{148} \\
\hline
\end{tabularx}
\end{table}

\subsection{Test\G di Integrazione}
I test\G di integrazione verificano la corretta comunicazione tra i sottosistemi e i moduli definiti nell'architettura.

\begin{longtable}{|p{2.2cm}|p{7cm}|p{3.5cm}|}
	\hline
	\rowcolor{primaryblue!30} 
	\textbf{Codice} & \textbf{Descrizione Interfaccia\G} & \textbf{Moduli Coinvolti} \\
	\hline
	\endhead
	
	\textbf{TI-001} & Verifica\G scambio dati JSON tra Frontend e Backend (API\G REST). & Standalone $\leftrightarrow$ Server \\
	\hline
	\textbf{TI-002} & Verifica\G invio prompt\G e ricezione risposta dal servizio LLM\G esterno. & AI\G Assistant $\leftrightarrow$ LLM\G API\G \\
	\hline
	\textbf{TI-003} & Verifica\G persistenza e recupero documenti analizzati. & Co-Pilot $\leftrightarrow$ Database \\
	\hline
	\textbf{TI-004} & Verifica\G aggregazione dati per le dashboard\G statistiche. & Database $\leftrightarrow$ Analytics \\
	\hline
	\textbf{TI-005} & Verifica\G passaggio del testo estratto (OCR\G) al motore di analisi AI\G. & Parser $\leftrightarrow$ Co-Pilot Logic \\
	\hline
	\caption{Test\G di Integrazione}
    \label{tab:test-integrazione}
\end{longtable}

\subsection{Test\G di Sistema}

\subsubsection{Test\G di Sistema - Requisiti Funzionali\G}
Questa tabella mappa puntualmente ogni requisito funzionale (RF)\G al test\G di sistema progettato per verificarlo.

\begin{longtable}{|p{2.5cm}|p{8cm}|p{2.5cm}|p{1.5cm}|}
\hline
\rowcolor{primaryblue!30} 
\textbf{Codice} & \textbf{Descrizione} & \textbf{Riferimento} & \textbf{Stato} \\
\hline
\endhead

% --- BLOCCO AUTH & UTENTI ---
TS-F-001 & Verifica\G che il sistema permetta all'utente non autenticato di effettuare la registrazione di un nuovo account & RF-1 & NI \\ \hline
TS-F-002 & Verifica\G che il sistema permetta all'utente di inserire il proprio indirizzo email in fase di registrazione & RF-2 & NI \\ \hline
TS-F-003 & Verifica\G che il sistema permetta all'utente di inserire la propria password in fase di registrazione & RF-3 & NI \\ \hline
TS-F-004 & Verifica\G che il sistema permetta all'utente di inserire la conferma della password in fase di registrazione & RF-4 & NI \\ \hline
TS-F-005 & Verifica\G che il sistema permetta all'utente di inserire il proprio nome in fase di registrazione & RF-5 & NI \\ \hline
TS-F-006 & Verifica\G che il sistema permetta all'utente di inserire il proprio cognome in fase di registrazione & RF-6 & NI \\ \hline
TS-F-007 & Verifica\G che il sistema permetta all'utente di confermare la registrazione tramite pulsante dedicato & RF-7 & NI \\ \hline
TS-F-008 & Verifica\G che il sistema mostri un messaggio di errore se l'email inserita non è valida & RF-8 & NI \\ \hline
TS-F-009 & Verifica\G che il sistema mostri un messaggio di errore se la password non rispetta i requisiti di sicurezza\G & RF-9 & NI \\ \hline
TS-F-010 & Verifica\G che il sistema mostri un messaggio di errore se le password inserite non coincidono & RF-10 & NI \\ \hline
TS-F-011 & Verifica\G che il sistema mostri un messaggio di errore se l'email è già associata ad un altro account & RF-11 & NI \\ \hline
TS-F-012 & Verifica\G che il sistema mostri un messaggio di successo al termine della registrazione & RF-12 & NI \\ \hline
TS-F-013 & Verifica\G che il sistema reindirizzi l'utente alla pagina di login dopo la registrazione & RF-13 & NI \\ \hline
TS-F-014 & Verifica\G che il sistema permetta all'utente di effettuare il login inserendo email e password & RF-14 & NI \\ \hline
TS-F-015 & Verifica\G che il sistema mostri un errore in caso di credenziali non valide & RF-15 & NI \\ \hline
TS-F-016 & Verifica\G che il sistema permetta all'utente autenticato di effettuare il logout & RF-16 & NI \\ \hline
TS-F-017 & Verifica\G che il sistema permetta all'utente autenticato di visualizzare il proprio profilo & RF-17 & NI \\ \hline
TS-F-018 & Verifica\G che il sistema mostri i dati anagrafici dell'utente nel profilo & RF-18 & NI \\ \hline
TS-F-019 & Verifica\G che il sistema permetta all'utente di modificare i propri dati anagrafici & RF-19 & NI \\ \hline
TS-F-020 & Verifica\G che il sistema permetta all'utente di modificare la propria password & RF-20 & NI \\ \hline
TS-F-021 & Verifica\G che il sistema richieda la vecchia password per impostarne una nuova & RF-21 & NI \\ \hline
TS-F-022 & Verifica\G che il sistema permetta all'Amministratore\G di visualizzare la lista degli utenti registrati & RF-22 & NI \\ \hline
TS-F-023 & Verifica\G che il sistema permetta all'Amministratore\G di visualizzare i dettagli di un utente specifico & RF-23 & NI \\ \hline
TS-F-024 & Verifica\G che il sistema permetta all'Amministratore\G di modificare il ruolo di un utente & RF-24 & NI \\ \hline
TS-F-025 & Verifica\G che il sistema permetta all'Amministratore\G di eliminare un utente & RF-25 & NI \\ \hline

% --- BLOCCO ASSISTANT (AI) ---
TS-F-026 & Verifica\G che il sistema permetta all'HR Manager di accedere al modulo\G Assistant & RF-26 & NI \\ \hline
TS-F-027 & Verifica\G che il sistema permetta di inserire un prompt\G testuale per la generazione di contenuti & RF-27 & NI \\ \hline
TS-F-028 & Verifica\G che il sistema permetta di selezionare il tono della risposta (formale, informale, neutro) & RF-28 & NI \\ \hline
TS-F-029 & Verifica\G che il sistema permetta di selezionare la lunghezza desiderata del contenuto & RF-29 & NI \\ \hline
TS-F-030 & Verifica\G che il sistema permetta di avviare la generazione del contenuto & RF-30 & NI \\ \hline
TS-F-031 & Verifica\G che il sistema mostri un indicatore di caricamento durante la generazione & RF-31 & NI \\ \hline
TS-F-032 & Verifica\G che il sistema mostri il contenuto generato dall'AI\G & RF-32 & NI \\ \hline
TS-F-033 & Verifica\G che il sistema permetta di copiare il testo generato nella clipboard & RF-33 & NI \\ \hline
TS-F-034 & Verifica\G che il sistema permetta di valutare la qualità\G della risposta (pollice su/giù) & RF-34 & NI \\ \hline
TS-F-035 & Verifica\G che il sistema permetta di rigenerare il contenuto se non soddisfacente & RF-35 & NI \\ \hline
TS-F-036 & Verifica\G che il sistema salvi automaticamente la cronologia delle generazioni & RF-36 & NI \\ \hline
TS-F-037 & Verifica\G che il sistema permetta di visualizzare lo storico delle attività & RF-37 & NI \\ \hline
TS-F-038 & Verifica\G che il sistema permetta di filtrare lo storico per data & RF-38 & NI \\ \hline
TS-F-039 & Verifica\G che il sistema permetta di cancellare una voce dallo storico & RF-39 & NI \\ \hline

% --- BLOCCO CO-PILOT (DOCUMENTI) ---
TS-F-040 & Verifica\G che il sistema permetta all'Operatore di accedere al modulo\G Co-Pilot\G & RF-40 & NI \\ \hline
TS-F-041 & Verifica\G che il sistema permetta di caricare un file PDF da analizzare & RF-41 & NI \\ \hline
TS-F-042 & Verifica\G che il sistema validi il formato del file caricato (solo PDF) & RF-42 & NI \\ \hline
TS-F-043 & Verifica\G che il sistema validi la dimensione del file (max 20MB) & RF-43 & NI \\ \hline
TS-F-044 & Verifica\G che il sistema avvii l'analisi automatica del documento caricato & RF-44 & NI \\ \hline
TS-F-045 & Verifica\G che il sistema estragga i dati chiave dal documento (nominativi, date, importi) & RF-45 & NI \\ \hline
TS-F-046 & Verifica\G che il sistema mostri i dati estratti all'utente per la verifica\G & RF-46 & NI \\ \hline
TS-F-047 & Verifica\G che il sistema permetta all'utente di modificare manualmente i dati estratti errati & RF-47 & NI \\ \hline
TS-F-048 & Verifica\G che il sistema permetta di confermare i dati validati & RF-48 & NI \\ \hline
TS-F-049 & Verifica\G che il sistema salvi il documento e i metadati nel database & RF-49 & NI \\ \hline
TS-F-050 & Verifica\G che il sistema permetta di visualizzare la lista dei documenti processati & RF-50 & NI \\ \hline
TS-F-051 & Verifica\G che il sistema permetta di scaricare il documento originale & RF-51 & NI \\ \hline
TS-F-052 & Verifica\G che il sistema permetta di eliminare un documento processato & RF-52 & NI \\ \hline
TS-F-053 & Verifica\G che il sistema permetta di cercare un documento per nome o metadato & RF-53 & NI \\ \hline
TS-F-054 & Verifica\G che il sistema permetta di filtrare i documenti per stato di elaborazione & RF-54 & NI \\ \hline

% --- BLOCCO ANALYTICS & ALTRO ---
TS-F-055 & Verifica\G che il sistema permetta al Data Analyst\G di accedere alla Dashboard\G Analytics & RF-55 & NI \\ \hline
TS-F-056 & Verifica\G che il sistema mostri il numero totale di generazioni effettuate (Assistant) & RF-56 & NI \\ \hline
TS-F-057 & Verifica\G che il sistema mostri la distribuzione dei rating ricevuti & RF-57 & NI \\ \hline
TS-F-058 & Verifica\G che il sistema mostri il numero di documenti processati (Co-Pilot) & RF-58 & NI \\ \hline
TS-F-059 & Verifica\G che il sistema mostri la percentuale di confidenza media dell'AI\G & RF-59 & NI \\ \hline
TS-F-060 & Verifica\G che il sistema mostri il tasso di correzione manuale da parte degli operatori & RF-60 & NI \\ \hline
TS-F-061 & Verifica\G che il sistema permetta di esportare i dati di reportistica in formato CSV & RF-61 & NI \\ \hline
TS-F-062 & Verifica\G che il sistema garantisca la persistenza dei dati tra sessioni diverse & RF-62 & NI \\ \hline
TS-F-063 & Verifica\G che il sistema gestisca correttamente i timeout durante le chiamate API\G & RF-63 & NI \\ \hline
TS-F-064 & Verifica\G che il sistema permetta la navigazione intuitiva tra i moduli & RF-64 & NI \\ \hline
TS-F-065 & Verifica\G che il sistema supporti la visualizzazione su schermi di diverse dimensioni & RF-65 & NI \\ \hline
TS-F-066 & Verifica\G che il sistema gestisca errori di rete imprevisti & RF-66 & NI \\ \hline
TS-F-067 & Verifica\G che il sistema permetta l'ordinamento delle tabelle dati & RF-67 & NI \\ \hline
TS-F-068 & Verifica\G che il sistema supporti la paginazione per liste lunghe di elementi & RF-68 & NI \\ \hline
TS-F-069 & Verifica\G che il sistema permetta la selezione multipla di elementi ove applicabile & RF-69 & NI \\ \hline
TS-F-070 & Verifica\G che il sistema mostri notifiche di feedback per ogni azione importante & RF-70 & NI \\ \hline
TS-F-071 & Verifica\G che il sistema permetta di annullare l'ultima azione distruttiva (se previsto) & RF-71 & NI \\ \hline
TS-F-072 & Verifica\G che il sistema protegga le rotte amministrative da accessi non autorizzati & RF-72 & NI \\ \hline
TS-F-073 & Verifica\G che il sistema cripti le password nel database & RF-73 & NI \\ \hline
TS-F-074 & Verifica\G che il sistema gestisca correttamente i token di sessione scaduti & RF-74 & NI \\ \hline
TS-F-075 & Verifica\G che il sistema permetta il recupero della password (se previsto) & RF-75 & NI \\ \hline
TS-F-076 & Verifica\G che il sistema registri i log di sistema per le operazioni critiche & RF-76 & NI \\ \hline
TS-F-077 & Verifica\G che il sistema permetta la configurazione dei parametri globali (Admin) & RF-77 & NI \\ \hline
TS-F-078 & Verifica\G che il sistema supporti la modalità scura (Dark Mode) & RF-78 & NI \\ \hline
TS-F-079 & Verifica\G che il sistema permetta la personalizzazione del profilo utente & RF-79 & NI \\ \hline
TS-F-080 & Verifica\G che il sistema gestisca upload multipli di file (Batch Upload) & RF-80 & NI \\ \hline
TS-F-081 & Verifica\G che il sistema permetta di visualizzare l'anteprima del PDF caricato & RF-81 & NI \\ \hline
TS-F-082 & Verifica\G che il sistema evidenzi i dati estratti direttamente sul PDF (overlay) & RF-82 & NI \\ \hline
TS-F-083 & Verifica\G che il sistema permetta lo zoom e la navigazione nel visualizzatore PDF & RF-83 & NI \\ \hline
TS-F-084 & Verifica\G che il sistema permetta di ruotare le pagine del PDF se necessario & RF-84 & NI \\ \hline
TS-F-085 & Verifica\G che il sistema rilevi documenti corrotti o illeggibili & RF-85 & NI \\ \hline
TS-F-086 & Verifica\G che il sistema permetta di assegnare tag o categorie ai documenti & RF-86 & NI \\ \hline
TS-F-087 & Verifica\G che il sistema permetta di archiviare documenti vecchi & RF-87 & NI \\ \hline
TS-F-088 & Verifica\G che il sistema supporti la ricerca full-text nel contenuto dei documenti & RF-88 & NI \\ \hline
TS-F-089 & Verifica\G che il sistema permetta di condividere un documento con altri utenti & RF-89 & NI \\ \hline
TS-F-090 & Verifica\G che il sistema gestisca i permessi di visualizzazione sui documenti & RF-90 & NI \\ \hline
TS-F-091 & Verifica\G che il sistema permetta di creare template\G di messaggi predefiniti & RF-91 & NI \\ \hline
TS-F-092 & Verifica\G che il sistema permetta di modificare i template\G esistenti & RF-92 & NI \\ \hline
TS-F-093 & Verifica\G che il sistema permetta di eliminare template\G non più in uso & RF-93 & NI \\ \hline
TS-F-094 & Verifica\G che il sistema permetta di utilizzare variabili dinamiche nei template\G & RF-94 & NI \\ \hline
TS-F-095 & Verifica\G che il sistema permetta di inviare email ai destinatari estratti & RF-95 & NI \\ \hline
TS-F-096 & Verifica\G che il sistema permetta di pianificare l'invio delle email & RF-96 & NI \\ \hline
TS-F-097 & Verifica\G che il sistema mostri lo stato di invio delle email (inviata, fallita) & RF-97 & NI \\ \hline
TS-F-098 & Verifica\G che il sistema permetta di allegare il documento processato all'email & RF-98 & NI \\ \hline
TS-F-099 & Verifica\G che il sistema permetta di visualizzare l'anteprima dell'email prima dell'invio & RF-99 & NI \\ \hline
TS-F-100 & Verifica\G che il sistema gestisca correttamente gli indirizzi email non validi & RF-100 & NI \\ \hline
TS-F-101 & Verifica\G che il sistema permetta di configurare il server SMTP per l'invio & RF-101 & NI \\ \hline
TS-F-102 & Verifica\G che il sistema mostri notifiche in tempo reale & RF-102 & NI \\ \hline
TS-F-103 & Verifica\G che il sistema permetta di segnare le notifiche come lette & RF-103 & NI \\ \hline
TS-F-104 & Verifica\G che il sistema permetta di accedere al centro notifiche & RF-104 & NI \\ \hline
TS-F-105 & Verifica\G che il sistema permetta di configurare le preferenze di notifica & RF-105 & NI \\ \hline
TS-F-106 & Verifica\G che il sistema fornisca tooltips o aiuti contestuali & RF-106 & NI \\ \hline
TS-F-107 & Verifica\G che il sistema disponga di una sezione FAQ o guida utente & RF-107 & NI \\ \hline
TS-F-108 & Verifica\G che il sistema permetta di contattare il supporto tecnico & RF-108 & NI \\ \hline
TS-F-109 & Verifica\G che il sistema mostri la versione attuale del software & RF-109 & NI \\ \hline
TS-F-110 & Verifica\G che il sistema permetta il backup dei dati & RF-110 & NI \\ \hline
TS-F-111 & Verifica\G che il sistema permetta il ripristino dei dati da backup & RF-111 & NI \\ \hline
TS-F-112 & Verifica\G che il sistema gestisca correttamente la concorrenza di più utenti & RF-112 & NI \\ \hline
TS-F-113 & Verifica\G che il sistema prevenga la sovrascrittura accidentale di dati & RF-113 & NI \\ \hline
TS-F-114 & Verifica\G che il sistema registri l'ultimo accesso dell'utente & RF-114 & NI \\ \hline
TS-F-115 & Verifica\G che il sistema permetta di visualizzare i termini di servizio & RF-115 & NI \\ \hline
TS-F-116 & Verifica\G che il sistema permetta di visualizzare la privacy policy & RF-116 & NI \\ \hline
TS-F-117 & Verifica\G che il sistema sia conforme al GDPR\G (consenso cookie, ecc.) & RF-117 & NI \\ \hline
TS-F-118 & Verifica\G che il sistema permetta all'utente di richiedere la cancellazione dell'account & RF-118 & NI \\ \hline
TS-F-119 & Verifica\G che il sistema permetta all'utente di scaricare i propri dati personali & RF-119 & NI \\ \hline
TS-F-120 & Verifica\G che il sistema supporti l'autenticazione a due fattori (2FA) - se previsto & RF-120 & NI \\ \hline
TS-F-121 & Verifica\G che il sistema blocchi l'account dopo N tentativi di login falliti & RF-121 & NI \\ \hline
TS-F-122 & Verifica\G che il sistema richieda password complesse (lunghezza, caratteri speciali) & RF-122 & NI \\ \hline
TS-F-123 & Verifica\G che il sistema sanitizzi gli input per prevenire XSS & RF-123 & NI \\ \hline
TS-F-124 & Verifica\G che il sistema utilizzi query parametriche per prevenire SQL Injection & RF-124 & NI \\ \hline
TS-F-125 & Verifica\G che il sistema gestisca sessioni sicure (HTTPS, Secure Flag) & RF-125 & NI \\ \hline
TS-F-126 & Verifica\G che il sistema non esponga dati sensibili negli URL & RF-126 & NI \\ \hline
TS-F-127 & Verifica\G che il sistema gestisca correttamente i codici di stato HTTP (200, 404, 500) & RF-127 & NI \\ \hline
TS-F-128 & Verifica\G che il sistema mostri pagine di errore user-friendly & RF-128 & NI \\ \hline

\caption{Test\G di Sistema per Requisiti Funzionali\G}
\label{tab:test-sistema-funzionali}
\end{longtable}

\subsubsection{Test\G di Sistema - Requisiti Prestazionali\G}
\begin{longtable}{|p{2.5cm}|p{8cm}|p{2.5cm}|p{1.5cm}|}
\hline
\rowcolor{primaryblue!30} 
\textbf{Codice} & \textbf{Descrizione} & \textbf{Riferimento} & \textbf{Stato} \\
\hline
\endhead

TS-P-001 & Verifica\G che il sistema generi contenuti testuali tramite AI\G (Assistant) entro 5 secondi per testi fino a 500 parole & RP-01 & NI \\ \hline
TS-P-002 & Verifica\G che il sistema classifichi e partizioni documenti PDF (Co-Pilot) entro 3 secondi per pagina & RP-02 & NI \\ \hline
TS-P-003 & Verifica\G che il tempo di risposta dell'interfaccia\G utente per operazioni standard sia inferiore a 2 secondi & RP-03 & NI \\ \hline
TS-P-004 & Verifica\G che il sistema supporti l'upload di file PDF fino a 20 MB & RP-04 & NI \\ \hline
TS-P-005 & Verifica\G che la Dashboard\G di Analytics carichi le statistiche entro 3 secondi per dataset fino a 1000 documenti & RP-05 & NI \\ \hline
TS-P-006 & Verifica\G che il sistema garantisca una disponibilità del 99\% durante l'orario lavorativo (8:00-18:00) & RP-06 & NI \\ \hline
TS-P-007 & Verifica\G che il sistema sia in grado di processare almeno 50 documenti in parallelo senza degrado prestazionale & RP-07 & NI \\ \hline
TS-P-008 & Verifica\G utilizzo risorse CPU sotto carico massimo (Desiderabile) & RP-08 & NI \\ \hline

\caption{Test\G di Sistema per Requisiti Prestazionali\G}
\label{tab:test-sistema-prestazionali}
\end{longtable}

\subsubsection{Test\G di Sistema - Requisiti di Qualità\G}
\begin{longtable}{|p{2.5cm}|p{8cm}|p{2.5cm}|p{1.5cm}|}
\hline
\rowcolor{primaryblue!30} 
\textbf{Codice} & \textbf{Descrizione} & \textbf{Riferimento} & \textbf{Stato} \\
\hline
\endhead

TS-Q-001 & Verifica\G che sia presente la documentazione tecnica completa (diagrammi e descrizioni Use Case) & RQ-01 & NI \\ \hline
TS-Q-002 & Verifica\G che il codice sorgente sia commentato secondo gli standard definiti nelle Norme di Progetto\G & RQ-02 & NI \\ \hline
TS-Q-003 & Verifica\G che sia presente il manuale utente per l'installazione e l'utilizzo del sistema & RQ-03 & NI \\ \hline
TS-Q-004 & Verifica\G che il codice superi l'analisi statica senza errori critici (Code Smells) & RQ-04 & NI \\ \hline
TS-Q-005 & Verifica\G che l'interfaccia\G utente sia accessibile secondo le linee guida WCAG 2.1 (livello AA) & RQ-05 & NI \\ \hline
TS-Q-006 & Verifica\G che il codice sia coperto da test\G di unità per almeno l'80\% (Code Coverage) & RQ-06 & NI \\ \hline

\caption{Test\G di Sistema per Requisiti di Qualità\G}
\label{tab:test-sistema-qualita}
\end{longtable}

\subsubsection{Test\G di Sistema - Requisiti di Vincolo\G}
\begin{longtable}{|p{2.5cm}|p{8cm}|p{2.5cm}|p{1.5cm}|}
\hline
\rowcolor{primaryblue!30} 
\textbf{Codice} & \textbf{Descrizione} & \textbf{Riferimento} & \textbf{Stato} \\
\hline
\endhead

TS-V-001 & Verifica\G che il sistema utilizzi Git come sistema di controllo versione & RV-01 & NI \\ \hline
TS-V-002 & Verifica\G che API\G e Backend\G siano sviluppati in Ruby on Rails\G & RV-02 & NI \\ \hline
TS-V-003 & Verifica\G che il Frontend sia sviluppato utilizzando il framework React & RV-03 & NI \\ \hline
TS-V-004 & Verifica\G che il sistema sia compatibile con i browser Google Chrome e Mozilla Firefox (ultime versioni) & RV-04 & NI \\ \hline
TS-V-005 & Verifica\G che l'interfaccia\G sia responsive e utilizzabile da dispositivi mobili & RV-05 & NI \\ \hline
TS-V-006 & Verifica\G che la documentazione del codice sia redatta in lingua inglese (Opzionale) & RV-06 & NI \\ \hline

\caption{Test\G di Sistema per Requisiti di Vincolo\G}
\label{tab:test-sistema-vincolo}
\end{longtable}

\subsection{Test\G di Accettazione}
I test\G di accettazione validano il sistema rispetto agli scenari d'uso (Use Case) previsti, assicurando che l'utente possa completare i flussi di lavoro principali.

\begin{longtable}{|p{2.0cm}|p{8.5cm}|p{2.5cm}|p{2.5cm}|}
	\hline
	\rowcolor{primaryblue!30} 
	\textbf{Codice} & \textbf{Descrizione} & \textbf{Riferimento} & \textbf{Stato} \\
	\hline
	\endhead
	
	% --- UTENTE GENERICO ---
	\textbf{TA-001} & Verifica\G che un utente non registrato possa completare la procedura di registrazione (Happy Path). & UC-0A & Non Impl. \\
	\hline
	\textbf{TA-002} & Verifica\G che il sistema impedisca la registrazione con dati non validi o email già esistente. & UC-0A (Scenari alternativi) & Non Impl. \\
	\hline
	\textbf{TA-003} & Verifica\G che l'utente possa effettuare il Login e il Logout correttamente. & UC-0B, UC-0G & Non Impl. \\
	\hline
	\textbf{TA-004} & Verifica\G che l'utente possa visualizzare e modificare il proprio profilo e cambiare la password. & UC-0C, UC-0D & Non Impl. \\
	\hline
	
	% --- AMMINISTRATORE ---
	\textbf{TA-005} & Verifica\G che l'Amministratore\G possa consultare la lista utenti e visualizzare i dettagli di un singolo utente. & UC-0E & Non Impl. \\
	\hline
	\textbf{TA-006} & Verifica\G che l'Amministratore\G possa modificare il ruolo di un utente o eliminarlo. & UC-0F & Non Impl. \\
	\hline
	
	% --- HR MANAGER (ASSISTANT) ---
	\textbf{TA-007} & Verifica\G che l'HR Manager possa configurare una richiesta (Prompt\G, Tono, Lunghezza) e generare un contenuto. & UC-1A, UC-1B, UC-1C & Non Impl. \\
	\hline
	\textbf{TA-008} & Verifica\G che l'HR Manager possa visualizzare, copiare e modificare il testo generato dall'AI\G. & UC-1D, UC-1E & Non Impl. \\
	\hline
	\textbf{TA-009} & Verifica\G che l'HR Manager possa valutare (Feedback) o scartare un contenuto generato. & UC-1F, UC-1N & Non Impl. \\
	\hline
	\textbf{TA-010} & Verifica\G il salvataggio automatico nello storico e la possibilità di recuperare generazioni passate. & UC-1O & Non Impl. \\
	\hline
	
	% --- OPERATORE CDL (CO-PILOT) ---
	\textbf{TA-011} & Verifica\G che l'Operatore possa caricare un documento PDF e avviare l'analisi automatica. & UC-2A & Non Impl. \\
	\hline
	\textbf{TA-012} & Verifica\G che il sistema estragga correttamente i dati e li mostri all'operatore. & UC-2B & Non Impl. \\
	\hline
	\textbf{TA-013} & Verifica\G lo scenario\G "Human-in-the-loop": l'operatore corregge manualmente un dato estratto errato e conferma. & UC-2D, UC-2E & Non Impl. \\
	\hline
	\textbf{TA-014} & Verifica\G gestione template\G: creazione, modifica e utilizzo di un template\G di messaggio. & UC-2I & Non Impl. \\
	\hline
	\textbf{TA-015} & Verifica\G il flusso di invio: selezione destinatari, associazione documento e invio email (o pianificazione). & UC-2L, UC-2O & Non Impl. \\
	\hline
	
	% --- DATA ANALYST ---
	\textbf{TA-016} & Verifica\G che il Data Analyst\G possa consultare le Dashboard\G e filtrare le metriche per periodo temporale. & UC-3A, UC-3B & Non Impl. \\
	\hline

    \caption{Test\G di Accettazione}
    \label{tab:test-accettazione}
\end{longtable}

\section{Cruscotto\G di Valutazione}
Di seguito verranno mostrate le misurazioni effettuate durante il periodo che va 
dall'aggiudicazione del capitolatoG sino alla Requirements and Technology BaselineG (RTB)\G. 
Le misurazioni presenti saranno prese durante lo svolgimento delle attività per la Product BaselineG (PB)\G.

\subsection{MPC01 e MPC02 - Earned Value (EV)\G e Planned Value (PV)\G}
 \begin{figure}[H]
     \centering
     \includegraphics[width=0.9\textwidth]{grafici/MPC01_02.png}
     \caption{Grafico per periodo di MPC01 e MPC02}
 \end{figure}

 Dal grafico si osserva che l'andamento del Valore Guadagnato (\textit{Earned Value} - EV\G) segue fedelmente quello del Valore Pianificato (\textit{Planned Value} - PV\G), con un trend crescente che culmina nel sesto sprint\G, in corrispondenza del completamento delle attività per la \textit{Requirements and Technology Baseline} (RTB)\G.



\subsection{MPC03 e MPC07 - Actual cost (AC)\G e Estimate to complete (ETC)\G}
\begin{figure}[H]
	\centering
	\includegraphics[width=0.9\textwidth]{grafici/MPC03_07.png}
	\caption{Grafico per periodo di MPC03 e MPC07}
\end{figure}
L'andamento della metrica MPC03 (\textit{Actual Cost\G}) mostra una crescita costante dei costi sostenuti, in linea con l'intensificazione delle attività produttive durante la fase di \textit{Requirements and Technology Baseline} (RTB)\G. 
Tale incremento, culminato nel sesto sprint\G, rispecchia fedelmente la pianificazione temporale definita nel Piano di Progetto\G, dove il maggior carico di lavoro (e quindi di spesa) era previsto proprio nelle settimane antecedenti la consegna della candidatura\G.

Parallelamente, la metrica MPC07 (\textit{Estimate to Complete\G}) evidenzia una progressiva diminuzione del budget residuo necessario per il completamento del progetto\G. Questo trend inverso conferma che le risorse sono state consumate in modo coerente con l'avanzamento dei lavori, avvicinando il progetto\G al traguardo della \textit{Product Baseline} (PB)\G senza generare extra-costi imprevisti.

\subsection{MPC04 e MP05 - Cost Performance Index (CPI)\G e Schedule performance Index\G}
\begin{figure}[H]
	\centering
	\includegraphics[width=0.9\textwidth]{grafici/MPC04_05.png}
	\caption{Grafico per periodo di MPC04 e MPC05}
\end{figure}

L'analisi del \textit{Cost Performance Index} (CPI)\G mostra un percorso di netta crescita. Il progetto\G è iniziato con un indice inferiore alle aspettative, a causa delle fisiologiche difficoltà iniziali.

Dopo l'investimento iniziale, il processo produttivo è diventato altamente sostenibile, permettendo di recuperare il budget consumato.
Parallelamente, lo \textit{Schedule Performance Index} (SPI)\G si è mantenuto stabile e vicino al valore ideale per tutto il periodo, garantendo il rispetto delle scadenze per la candidatura\G.



\subsection{MPC06 - Estimated at completion (EAC)}
\begin{figure}[H]
	\centering
	\includegraphics[width=0.9\textwidth]{grafici/MPC06.png}
	\caption{Grafico per periodo di MPC06}
\end{figure}

L'andamento del costo stimato a finire (Estimated at Completion - EAC) racconta chiaramente il percorso di ottimizzazione intrapreso dal team. Il progetto\G ha attraversato una fase iniziale critica durante i primi sprint\G, in cui la stima dei costi\G finali superava sensibilmente il budget stanziato. Questa proiezione negativa era la diretta conseguenza delle difficoltà  iniziali  che avevano abbassato l'indice di efficienza\G CPI\G. Succesivamente si è innescato un trend di recupero costante. 


\subsection{MPC08 - Time Estimate At Completion\G}
\begin{figure}[H]
	\centering
	\includegraphics[width=0.9\textwidth]{grafici/MPC08.png}
	\caption{Grafico per periodo di MPC08}
\end{figure}
L'andamento della stima temporale a finire (\textit{Time Estimate At Completion\G}) conferma la solidità della pianificazione iniziale. La proiezione della data di completamento per la fase di \textit{Requirements and Technology Baseline} (RTB)\G è rimasta sostanzialmente invariata lungo tutto l'arco temporale osservato.



\subsection{MPC09 - Requirements Stability Index (RSI)\G}
\begin{figure}[H]
	\centering
	\includegraphics[width=0.9\textwidth]{grafici/MPC09.png}
	\caption{Grafico per periodo di MPC09}
\end{figure}
L'indice di stabilità dei requisiti (\textit{Requirements Stability Index\G}) mostra un andamento che riflette fedelmente il ciclo di vita dell'Analisi dei Requisiti\G.
Nello Sprint\G 2 si registra un picco negativo significativo. Tale valore, apparentemente critico, è in realtà indicatore di una intensa attività produttiva. Partendo da un set iniziale di requisiti, il team ha effettuato un'opera di espansione e dettaglio massiccia. Matematicamente, ciò ha portato il numero delle modifiche a superare il numero dei requisiti iniziali, generando l'indice negativo. Superata la fase critica di definizione, l'indice è risalito rapidamente. 

\subsection{MPC10 - Indice di Gulpease\G}
\begin{figure}[H]
	\centering
	\includegraphics[width=0.9\textwidth]{grafici/MPC10.png}
	\caption{Grafico per periodo di MPC10}
\end{figure}
In linea generale, il gruppo BugBusters ha posto grande attenzione alla redazione della documentazione: l'obiettivo primario è sempre stato quello di produrre elaborati che fossero non solo corretti tecnicamente, ma anche facilmente fruibili da tutti gli stakeholder\G.
Dall'analisi dei dati emerge una disparità nei valori di leggibilità tra le diverse tipologie di documenti, dovuta alla natura intrinseca del loro contenuto.
I dati sulla leggibilità mostrano una chiara differenza tra i documenti. L'Analisi dei Requisiti\G supera abbondantemente la soglia ottima grazie alla scelta di usare frasi brevi e semplici, ideali per farsi capire chiaramente dal cliente. Al contrario, il Glossario\G e le Norme di Progetto\G rimangono sotto la soglia minima per motivi strutturali: il primo è penalizzato dalla presenza di parole tecniche molto lunghe, mentre le seconde richiedono un linguaggio formale e rigoroso che non può essere semplificato oltre un certo limite senza perdere di precisione.

Il gruppo si impegna comunque, nelle prossime iterazioni, a raffinare ulteriormente la sintassi di tali documenti per migliorarne la leggibilità senza comprometterne il rigore formale.

\subsection{MPC13 - Quality metrics satisfied}
\begin{figure}[H]
	\centering
	\includegraphics[width=0.9\textwidth]{grafici/MPC13.png}
	\caption{Grafico per periodo di MPC13}
\end{figure}
L'andamento della percentuale di metriche soddisfatte offre una sintesi efficace della maturazione qualitativa del progetto\G. Il primo sprint\G ha risentito della bassa efficienza\G economica iniziale (CPI\G sotto soglia), mentre nei successivi due sprint\G è stato l'Indice di Stabilità dei Requisiti (RSI)\G a mancare l'obiettivo, a causa della necessaria fase di espansione dell'Analisi dei Requisiti\G.  
Superata la fase di assestamento, il trend ha mostrato un miglioramento netto. A partire dal quarto sprint\G, il team ha raggiunto una stabilità su tutti i fronti monitorati (Costi, Tempi, Documentazione e Processi), mantenendo l'indicatore vicino al valore minimo accettabile fino al termine della fase RTB\G. Questo risultato conferma che le misure correttive adottate sono state risolutive, portando il processo produttivo a un livello di affidabilità\G ottimale proprio nel momento decisivo della candidatura\G.

\subsection{MPC14 - Time Efficiency}
\begin{figure}[H]
	\centering
	\includegraphics[width=0.9\textwidth]{grafici/MPC14.png}
	\caption{Grafico per periodo di MPC14}
\end{figure}
L'analisi dell'efficienza\G temporale mostra un andamento notevolmente stabile. Questa costanza è un segnale positivo: indica che il team è riuscito a mantenere un rapporto equilibrato tra il lavoro produttivo (stesura documenti, sviluppo) e le ore di gestione (riunioni, auto-formazione), senza mai farsi sopraffare dall'overhead organizzativo.
Le lievi flessioni registrate nella fase centrale (Sprint\G 4 e 5) sono fisiologiche e riconducibili principalmente al rischio\G riguardante la sovrapposizione con sessione d'esami e alla necessità di maggiori confronti interni per la riorganizzazione dell'Analisi dei Requisiti\G.
Il picco positivo raggiunto nel sesto sprint\G testimonia la capacità  del gruppo di massimizzare la produttività nelle settimane decisive per la chiusura della candidatura\G.


\section{Iniziative di miglioramento}
L'ottimizzazione costante dei processi costituisce un pilastro fondamentale per la riuscita del progetto\G. Di seguito vengono esposte le problematiche operative riscontrate e le relative strategie di risoluzione adottate per superare tali ostacoli.

\subsection{Valutazioni sull'organizzazione}
\begin{longtable}{|p{3cm}|p{6cm}|p{5cm}|}
  \hline
  \textbf{Area} & \textbf{Problema Riscontrato} & \textbf{Contromisura Adottata} \\
  \hline
  \textbf{Tracciabilità\G} & L'assenza di un sistema di monitoraggio puntuale delle attività ostacola il flusso produttivo e compromette l'efficacia\G della programmazione operativa. & Adozione della funzionalità\G 'Issues' di GitHub\G per ottimizzare il controllo operativo e la supervisione dei flussi di lavoro. \\
  \hline
  \textbf{Controllo delle modifiche} & Operare senza un flusso di Pull Request obbligatorio riduce la stabilità del software e la tracciabilità\G delle integrazioni. & Attivazione della Branch Protection per inibire i push diretti e rendere mandatorio il processo di Code Review tramite Pull Request. \\
  \hline
  \textbf{Rendicontazione delle ore} & La mancanza di un sistema strutturato per la rendicontazione delle ore lavorate limita la capacità di analisi dell'efficienza\G e della produttività del team. & Implementazione\G di un foglio di calcolo condiviso per la registrazione puntuale delle ore dedicate alle attività progettuali, facilitando così il monitoraggio e l'analisi delle performance. \\
  \hline

  \endhead 
\hline
\end{longtable}


\subsection{Valutazioni sui ruoli}
\begin{longtable}{|p{3cm}|p{6cm}|p{5cm}|}
  \hline
  \textbf{Ruolo} & \textbf{Problema Riscontrato} & \textbf{Contromisura Adottata} \\
  \hline
  \endhead
  \textbf{Tutti i ruoli} & Per ottimizzare le ore produttive nelle fasi avanzate, è necessario superare il blocco bisettimanale dei ruoli, che attualmente lascia lacune nella copertura delle attività. & L'assegnazione dei ruoli diviene flessibile su base settimanale, previo allineamento tra le parti, mantenendo l'incompatibilità nel ricoprire funzioni simultanee. \\
\hline
\end{longtable}


\subsection{Valutazioni sugli strumenti}
\begin{longtable}{|p{3cm}|p{6cm}|p{5cm}|}
  \hline
  \textbf{Strumento} & \textbf{Problema Riscontrato} & \textbf{Contromisura Adottata} \\
  \hline
  \endhead
  \textbf{Titolo Problema} & Problema da descrivere & Contromisura spiegata \\
\hline
\end{longtable}


\subsection{Considerazioni finali}
L'iterazione e l'apprendimento continuo guidano la qualità\G del nostro lavoro. Le retrospettive ci hanno permesso di affinare i processi e aumentare l'efficienza\G. 
Il team resta focalizzato sul problem-solving proattivo per mantenere alti gli standard produttivi.



\end{document}