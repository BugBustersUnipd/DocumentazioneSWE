\documentclass[a4paper,11pt]{article}
   
\newcommand{\CurrentVersion}{0.0.4} % ultima versione, da cambiare ad ogni push significativo

\usepackage[utf8]{inputenc}
\usepackage[T1]{fontenc}
\usepackage[italian]{babel}
\usepackage[margin=2.5cm]{geometry}
\usepackage{graphicx}
\usepackage{grffile}
\usepackage{booktabs}
\usepackage{setspace}
\usepackage{titlesec}
\usepackage{float}
\usepackage{ifthen}
\usepackage[table]{xcolor}
\usepackage{tabularx}
\usepackage{needspace}
\usepackage{tcolorbox}
\usepackage{enumitem}
\usepackage[titles]{tocloft}
\usepackage[colorlinks=true,linkcolor=black,urlcolor=blue,citecolor=blue]{hyperref}

% Macro
% genera la stringa "\noindent (Riferimento alla tabella decisioni:
% \hyperref[RTB4]{RTB4})", si usa facendo \refDecisione{NomeLabel}{Testo}
\newcommand{\refDecisione}[2]{%
    \noindent (\textbf{Riferimento alla tabella decisioni: \hyperref[#1]{#2}})%
}


\definecolor{primaryblue}{RGB}{0,102,204}
\definecolor{secondaryblue}{RGB}{51,153,255}
\definecolor{lightgray}{RGB}{245,245,245}
\definecolor{darkgray}{RGB}{100,100,100}

\titleformat{\section}
 {\Large\bfseries\color{primaryblue}}
 {\thesection}{1em}{}

\titleformat{\subsection}
 {\large\bfseries\color{primaryblue}} % Sottosezione: colore secondaryblue
 {\thesubsection}{1em}{}

\titleformat{\subsubsection}
 {\normalsize\bfseries\color{secondaryblue}} % Sotto-sottosezione: colore secondaryblue
 {\thesubsubsection}{1em}{}

% per il footer con il numero di pagina
\usepackage{fancyhdr}
\usepackage{lastpage} % per ottenere il numero dell'ultima pagina da mettere nel footer


\usepackage{ltablex} %per far andare a capo le tabelle
\keepXColumns

\renewcommand{\sectionmark}[1]{\markright{#1}}
\newcommand{\G}{\textsubscript{\scalebox{0.6}{\textbf{G}}}}


% Macro tabella metrica (tabella uniforme, non spezzata e con stile coerente)
\newcommand{\MetricTable}[1]{
  \Needspace{10\baselineskip}% evita spezzamenti sgradevoli
  \begin{table}[H]
  \centering
  \arrayrulecolor{primaryblue}
  \begin{tabularx}{\linewidth}{|>{\raggedright\arraybackslash}p{2.5cm}|X|>{\raggedright\arraybackslash}p{3cm}|>{\raggedright\arraybackslash}p{3cm}|}
  \hline
  \rowcolor{primaryblue!40}
  		\textbf{\color{white} Metrica} & \textbf{\color{white} Descrizione} & \textbf{\color{white} Valore accettazione} & \textbf{\color{white} Valore ideale} \\
  \hline
  \ignorespaces
  #1
  \end{tabularx}
  \end{table}
}



\setlength{\parskip}{4pt}
\setlength{\parindent}{0pt}

\setlist[itemize]{leftmargin=*,itemsep=3pt}
\setlist[enumerate]{leftmargin=*,itemsep=3pt}

\graphicspath{{./}{../assets/images/}{./images/}}

\begin{document}

%configurazione per il footer
\pagestyle{fancy}
\fancyhf{} % pulisce tutti i campi del header e footer

% Header: sinistra e destra
\fancyhead[L]{Gruppo 4 - BugBusters} % sinistra
\fancyhead[R]{Piano di Qualifica\G}   % destra

\fancyfoot[L]{ \thepage\ di \pageref{LastPage}} %definisce il formato del footer
\fancyfoot[R]{ \nouppercase{\rightmark}} % nome della sezione


\renewcommand{\headrulewidth}{0pt} % rimuove la linea dell'header
\renewcommand{\footrulewidth}{0pt} % se vuoi anche togliere eventuale linea del footer

% abilitare numerazione e TOC fino al livello "paragraph" (subsubsubsection)
\setcounter{secnumdepth}{4}
\setcounter{tocdepth}{4}
% formattazione del nuovo livello per avere aspetto coerente
\titleformat{\paragraph}[block]{\normalsize\bfseries\color{secondaryblue}}{\theparagraph}{1em}{}
% alias comodo per usare "subsubsubsection"
\newcommand{\subsubsubsection}{\paragraph}

\begin{center}
  \IfFileExists{../../assets/Logo.jpg}{%
    \includegraphics[width=6cm,height=3cm,keepaspectratio]{../../assets/Logo.jpg} \\[0.8cm]
  }{%
    \fbox{\parbox[c][2.5cm][c]{6cm}{\centering Logo non trovato\\(../../assets/Logo.jpg)}}\\[0.5cm]
  }
  
  {\Large\bfseries\color{primaryblue} BugBusters}\\[0.5cm]

  {\Huge\bfseries\color{primaryblue} Piano di Qualifica\G}\\[0.3cm]
  {\Large\color{secondaryblue} Versione \CurrentVersion}\\[0.8cm]
\end{center}

\begin{center}
\begin{tcolorbox}[colback=lightgray,colframe=primaryblue,width=0.85\textwidth,arc=3mm,boxrule=0.5pt]
% Usa tabularx con una colonna fissa per l'etichetta e una colonna X per il contenuto
\begin{tabularx}{\linewidth}{@{}>{\raggedright\arraybackslash}p{3.5cm}>{\raggedright\arraybackslash}X@{}}
	{Stato} & In redazione \\
	{Responsabile} &  \\
	{Verificatore} &  \\
	{Redattori} & Luca Slongo \\
	{Distribuzione} & BugBusters, Eggon, Prof. Tullio Vardanega, Prof. Riccardo Cardin \\
\end{tabularx}
\end{tcolorbox}
\end{center}

\vspace{0.5cm}

\begin{center}
\begin{tcolorbox}[colback=secondaryblue!10,colframe=secondaryblue,width=0.9\textwidth,arc=3mm,boxrule=0.8pt,title={\bfseries Descrizione}]
Piano di Qualifica\G del Team BugBusters per il Capitolato C5 proposto da Eggon, che ha l'obiettivo di far rispettare uno standard di qualità per il codice e rispettare i requisiti funzionali\G prestabiliti.\end{tcolorbox}
\end{center}

\newpage

% Registro delle Modifiche (pagina 2)
\section*{Registro delle Modifiche}

\arrayrulecolor{primaryblue}
{\footnotesize
\begin{tabularx}{\textwidth}{|>{\raggedright\arraybackslash}p{1.5cm}|>{\raggedright\arraybackslash}p{2cm}|X|>{\raggedright\arraybackslash}p{2cm}|>{\raggedright\arraybackslash}p{2cm}|>{\raggedright\arraybackslash}p{2cm}|}
\hline
\rowcolor{primaryblue!40}
\textbf{\color{white} Versione} & \textbf{\color{white} Data} & \textbf{\color{white} Descrizione} & \textbf{\color{white} Redatto} & \textbf{\color{white} Verificato} & \textbf{\color{white} Approvato} \\
\hline
\CurrentVersion & 11/01/2026 & Aggiunto contenuto alla sezione 5 & Marco Piro & - & - \\
\hline
0.0.3 & 04/01/2026 & Aggiunte sezioni 4 e 5 & Marco Piro & - & - \\
\hline
0.0.2 & 29/12/2025 & Aggiunta Test di Sistema e di Accettazione & Marco Piro & - & - \\
\hline
0.0.1 & 03/12/2025 & Prima stesura del documento & Luca Slongo & - & - \\
\hline
\end{tabularx}
}

\newpage

% Indice cliccabile
\setcounter{tocdepth}{3} % Mostra fino al livello di sottosezione (2)
\tableofcontents
\listoftables
\listoffigures

\newpage
\section{Introduzione}

\subsection{Scopo del documento}
Lo scopo di redigere un Piano di Qualifica\G è garantire che il prodotto finale soddisfi in modo verificabile i requisiti stabiliti, assicurando qualità, affidabilità e conformità agli standard adottati. Questo documento definisce obiettivi, procedure e responsabilità legate alle attività di verifica e validazione per il controllo della qualità lungo tutto il ciclo di vita del progetto. Ciò permette di ridurre il rischio di errori, migliorare la trasparenza del processo e assicurare che il software sia robusto e coerente con le aspettative del committente.

\subsection {Glossario}
Il Glossario è un documento nel quale sono raccolte e spiegate in modo puntuale le definizioni dei termini tecnici e delle espressioni utilizzate nei documenti di progetto. Questo strumento è essenziale per garantire una comprensione uniforme tra tutti i membri del team e per facilitare la comunicazione con i soggetti esterni.

I termini che dispongono di una definizione nel Glossario saranno contrassegnati nel modo seguente: parola\G{}.

\subsection{Riferimenti}

\subsubsection{Riferimenti normativi}
\begin{itemize}
    \item Regolamento del progetto didattico\\
    \url{https://www.math.unipd.it/~tullio/IS-1/2025/Dispense/PD1.pdf} \\
    Ultimo Accesso XXX
    \item Capitolato\textsubscript{\scalebox{0.6}{\textbf{G}}} d'appalto C5: NEXUM - Eggon \\
    \url{https://www.math.unipd.it/~tullio/IS-1/2025/Progetto/C5.pdf} \\
    Ultimo Accesso XXX
    \item Norme di Progetto versione X.X.X \\
    \url{DA AGGIUNGERE APPENA CARICATO SUL SITO} \\
    Ultimo Accesso XXX
\end{itemize}

\subsubsection{Riferimenti informativi}
\begin{itemize}
    \item Glossario ver. X.X.X \\
    \url{DA AGGIUNGERE APPENA SI CARICA SUL SITO} \\
    Ultimo Accesso XXX
    \item Standard ISO/IEC 9126 \\
    \url{https://it.wikipedia.org/wiki/ISO/IEC_9126} \\
    Ultimo Accesso XXX
    \item Standard ISO/IEC 12207:1995 \\
    \url{https://www.math.unipd.it/~tullio/IS-1/2009/Approfondimenti/ISO_12207-1995.pdf} \\
    Ultimo Accesso XXX
\end{itemize}

\newpage

\section{Obiettivi stabiliti per la qualità}

È fondamentale stabilire degli obiettivi da raggiungere per assicurare la qualità prefissata del prodotto. 
Questo documento definisce i valori di accettazione e ottimalità delle metriche secondo gli standard definiti 
nelle Norme di Progetto.

\subsection{Qualità di processo}
Un indicatore della qualità di un prodotto è il metodo con cui è stato sviluppato. Se il processo di sviluppo segue 
delle linee guida ben definite, esso favorisce la buona riuscita del prodotto. Come stabilito nelle Norme di Progetto, 
nel nostro way of working abbiamo adottato lo Standard ISO/IEC 12207:1995 adattandolo alle nostre esigenze e a quelle 
del progetto. /////Oltre allo standard abbiamo deciso di effettuare delle revisioni periodiche per analizzare lo stato di avanzamento rispetto agli ovbiettivi stabilti./////LE FAREMO? INTANTO L'HO SCRITTO POI AL MASSIMO CANCELLIAMO

\subsubsection{Processi primari}

I processi primari sono quelle attività che iniziano o eseguono lo sviluppo, l'operazione o la manutenzione di prodotti software. Essi rappresentano le componenti fondamentali del ciclo di vita del progetto e sono suddivisi nelle seguenti categorie:
\subsubsubsection{Fornitura}

\MetricTable{
MPC01 & Earned value (EV) & $\geq 0$ & $\leq$ EAC \\
\hline
MPC02 & Planned value (PV) & $\geq 0$ & $\leq$ Budget at completion (BAC) \\
\hline
MPC03 & Actual cost (AC) & $\geq 0$ & $\leq$ EAC \\
\hline
MPC04 & Cost Performance Index (CPI) & $\geq 0.9$ & $\geq 1.0$ \\
\hline
MPC05 & Schedule Performance Index & $\geq 0$ & 1 \\
\hline
MPC06 & Estimated at completion (EAC) & $\pm 5\%$ rispetto al (BAC) & Budget at completion (BAC) \\
\hline
MPC07 & Estimate to complete (ETC) & $\geq 0$ & $\leq$ EAC \\
\hline
MPC08 & Time Estimate At Completion & $\geq 0$ & $\leq$ Durata pianificata \\
\hline}



\subsubsubsection{Sviluppo}
\MetricTable{
MPC09 & Requirements Stability Index (EAC) & $\pm 70\%$ & 100\% \\
\hline}


\subsubsection{Processi di supporto}
\subsubsubsection{Documentazione}
\MetricTable{
MPC10 & Indice di Gulpease del documento & $\geq 60\%$ & $\geq 80\%$ \\
\hline
MPC11 & Errori ortografici rilevati & $0$ & $0$ \\
\hline}
\subsubsubsection{Verifica}
\MetricTable{
MPC12 & Code Coverage & $\geq 80\%$ & $100\%$ \\
\hline
MPC13 & Test Success Rate & $100\%$ & $100\%$ \\
\hline}
\subsubsubsection{Gestione della qualità}
\MetricTable{
MPC14 & Quality metrics satisfied & $\geq 80\%$ & $100\%$ \\
\hline}

\subsubsection{Processi organizzativi}
\subsubsubsection{Gestione dei processi}
\MetricTable{
MPC15 & Time Efficiency & $\geq 50\%$ & $100\%$ \\
\hline}



\subsection{Qualità di prodotto}

Per qualità di prodotto si intende una valutazione complessiva del software sia dal punto di vista 
funzionale sia dal punto di vista strutturale. Il codice deve adempiere alle funzionalità prestabilite in 
modo efficiente e semplice, e al contempo essere manutenibile, affidabile e portabile. Il gruppo ha aderito allo 
standard ISO/IEC 9126 per garantire il rispetto di queste caratteristiche fondamentali, affinchè il prodotto 
sviluppato sia di alta qualità.

\subsubsection{Funzionalità}
\MetricTable{
MPD01 & Requisiti obbligatori soddisfatti & $100\%$ & $100\%$ \\
\hline
MPD02 & Requisiti desiderabili soddisfatti & $0\%$ & $100\%$ \\
\hline
MPD03 & Requisiti opzionali soddisfatti & $0\%$ & $100\%$ \\
\hline}

\subsubsection{Affidabilità}
\MetricTable{
MPD04 & Branch Coverage & $\geq 60\%$ & $\geq 80\%$ \\
\hline
MPD05 & Statement Coverage & $\geq 70\%$ & $\geq 90\%$ \\
\hline
MPD06 & Failure Density & $\leq 0.5$ & $\leq 0.1$ \\
\hline}

\subsubsection{Efficienza}
\MetricTable{
MPD07 & Time on Task & $\leq 60$ sec & $\leq 30$ sec \\
\hline
MPD08 & Error Rate & $\leq 5\%$ & $\leq 2\%$ \\
\hline}

\subsubsection{Usabilità}
\MetricTable{
MPD09 & Response Time & $\leq 2$ sec & $\leq 1$ sec \\
\hline}

\subsubsection{Mantenibilità}
\MetricTable{
MPD10 & Code Smells & $\leq 10$ & $\leq 5$ \\
\hline
MPD11 & Coefficient of Coupling & $\leq 0.4$ & $\leq 0.2$ \\
\hline
MPD12 & Cyclomatic complexity & $\leq 20$ & $\leq 10$ \\
\hline}

\subsubsection{Portabilità}
SERVE?? boh




\section{Metodi di testing}
Come stabilito nelle Norme di Progetto(METTERE SEZIONE), alla quale è disponibile
la nomenclatura utilizzata, i test da effettuare saranno:
\begin{itemize}
    \item Test di Unità
    \item Test di Integrazione
    \item Test di Sistema
    \item Test di Regressione
    \item Test di Accettazione
\end{itemize}

\subsection{Test di Sistema}
% Definisci i colori se non li hai già
\definecolor{statusred}{RGB}{200, 50, 50}

\begin{longtable}{|p{2cm}|p{8cm}|p{2.5cm}|p{3cm}|}
\hline
\textbf{Codice} & \textbf{Descrizione} & \textbf{Requisito} & \textbf{Stato} \\
\hline
\endhead

TS-001 & Verifica che un utente non registrato possa completare la procedura di registrazione inserendo dati validi. & RF-001 & \textcolor{statusred}{Non Implementato} \\
\hline
TS-002 & Verifica che il sistema impedisca la registrazione se l'email o la matricola sono già presenti o non valide. & RF-010, RF-012 & \textcolor{statusred}{Non Implementato} \\
\hline
TS-003 & Verifica che un utente registrato possa effettuare il login con credenziali corrette e accedere alla dashboard. & RF-014, RF-017 & \textcolor{statusred}{Non Implementato} \\
\hline
TS-004 & Verifica che un utente Admin possa modificare il ruolo di un altro utente (es. da Editor ad Admin). & RF-035 & \textcolor{statusred}{Non Implementato} \\
\hline
TS-005 & Verifica la generazione di un contenuto (titolo, testo, immagine) tramite AI Assistant inserendo un prompt valido. & RF-038, RF-039 & \textcolor{statusred}{Non Implementato} \\
\hline
TS-006 & Verifica che il sistema salvi correttamente nello storico ogni contenuto generato. & RF-042, RF-059 & \textcolor{statusred}{Non Implementato} \\
\hline
TS-007 & Verifica la possibilità di modificare manualmente un contenuto generato (titolo, testo o immagine). & RF-063 & \textcolor{statusred}{Non Implementato} \\
\hline
TS-008 & Verifica il caricamento di un file PDF nel modulo Co-Pilot e l'associazione dei metadati (categoria, azienda). & RF-071, RF-072 & \textcolor{statusred}{Non Implementato} \\
\hline
TS-009 & Verifica che il sistema estragga e visualizzi correttamente le informazioni dai documenti caricati (destinatario, tipologia, confidenza). & RF-078, RF-113 & \textcolor{statusred}{Non Implementato} \\
\hline
TS-010 & Verifica la possibilità di modificare il destinatario associato a un documento e salvare la modifica. & RF-094, RF-097 & \textcolor{statusred}{Non Implementato} \\
\hline
TS-011 & Verifica la creazione e il salvataggio di un template di messaggio. & RF-118 & \textcolor{statusred}{Non Implementato} \\
\hline
TS-012 & Verifica il corretto invio simulato di un documento con relativo messaggio al destinatario. & RF-131 & \textcolor{statusred}{Non Implementato} \\
\hline
TS-013 & Verifica che la dashboard Analytics mostri i KPI principali (es. numero prompt, rating medio). & RF-137, RF-138 & \textcolor{statusred}{Non Implementato} \\
\hline
TS-014 & Verifica il filtraggio delle statistiche Analytics per data di inizio e fine. & RF-147, RF-148 & \textcolor{statusred}{Non Implementato} \\
\hline
\caption{Tabella dei Test di Sistema}
\label{tab:test_sistema}
\end{longtable}

\subsection{Test di Accettazione}
\begin{longtable}{|p{2cm}|p{8cm}|p{3cm}|p{2.5cm}|}
  \hline
  \textbf{Codice} & \textbf{Descrizione Scenario} & \textbf{Caso d'Uso} & \textbf{Stato} \\
  \hline
  \endhead
  
  TA-001 & \textbf{Registrazione e Accesso:} L'utente si registra, conferma l'account ed effettua il primo accesso con successo. & UC-0A, UC-0B & \textcolor{statusred}{Non Implementato} \\
  \hline
  TA-002 & \textbf{Gestione Profilo:} L'utente accede al profilo, modifica la propria password o dati anagrafici e salva le modifiche. & UC-0D & \textcolor{statusred}{Non Implementato} \\
  \hline
  TA-003 & \textbf{Ciclo completo AI Assistant:} L'utente genera un testo, lo modifica, ne cambia l'immagine e lo pubblica/salva. & UC-1 & \textcolor{statusred}{Non Implementato} \\
  \hline
  TA-004 & \textbf{Gestione Storico Prompt:} L'utente cerca un vecchio prompt nello storico, lo duplica e rigenera un nuovo contenuto. & UC-1B, UC-1F & \textcolor{statusred}{Non Implementato} \\
  \hline
  TA-005 & \textbf{Analisi Documentale Co-Pilot:} L'operatore carica un cedolino, verifica che l'AI abbia riconosciuto l'azienda e il dipendente, e corregge eventuali errori. & UC-2A, UC-2C & \textcolor{statusred}{Non Implementato} \\
  \hline
  TA-006 & \textbf{Invio Comunicazioni:} L'operatore seleziona un documento validato, genera un messaggio da template e invia il tutto. & UC-2F, UC-2H & \textcolor{statusred}{Non Implementato} \\
  \hline
  TA-007 & \textbf{Monitoraggio Analytics:} L'auditor accede alla dashboard, filtra per l'ultimo mese e visualizza i grafici di utilizzo e performance. & UC-3A & \textcolor{statusred}{Non Implementato} \\
  \hline
  \caption{Tabella dei Test di Accettazione}
  \label{tab:test_accettazione}
  \end{longtable}


DA RIVEDERE!!!!!!!!!!!!!!!!!!!


\section{Cruscotto di Valutazione}
Di seguito verranno mostrate le misurazioni effettuate durante il periodo che va 
dall'aggiudicazione del capitolatoG sino alla Requirements and Technology BaselineG (RTB). 
Le misurazioni presenti saranno prese durante lo svolgimento delle attività per la Product BaselineG (PB).

\subsection{MPC01 e MPC02 - Earned Value (EV) e Planned Value (PV)}
Grafico

\subsection{MPC03 e MPC07 - Actual cost (AC) e Estimate to complete (ETC)}
Grafico

\subsection{MPC04 e MP05 - Cost Performance Index (CPI) e Schedule performance Index}
Grafico

\subsection{MPC06 - Estimated at completion (EAC)}
Grafico

\subsection{MPC08 - Time Estimate At Completion}
Grafico

\subsection{MPC09 - Requirements Stability Index (RSI)}
Grafico

\subsection{MPC10 - Indice di Gulpease}
Grafico

\subsection{MPC11 - Errori ortografici rilevati}
Grafico

\subsection{MPC14 - Quality metrics satisfied}
Grafico

\subsection{MPC15 - Time Efficiency}
Grafico



\section{Iniziative di miglioramento}
L'ottimizzazione costante dei processi costituisce un pilastro fondamentale per la riuscita del progetto. Di seguito vengono esposte le problematiche operative riscontrate e le relative strategie di risoluzione adottate per superare tali ostacoli.

\subsection{Valutazioni sull'organizzazione}
\begin{longtable}{|p{3cm}|p{6cm}|p{5cm}|}
  \hline
  \textbf{Area} & \textbf{Problema Riscontrato} & \textbf{Contromisura Adottata} \\
  \hline
  \textbf{Tracciabilità} & L'assenza di un sistema di monitoraggio puntuale delle attività ostacola il flusso produttivo e compromette l'efficacia della programmazione operativa. & Adozione della funzionalità 'Issues' di GitHub per ottimizzare il controllo operativo e la supervisione dei flussi di lavoro. \\
  \hline
  \textbf{Controllo delle modifiche} & Operare senza un flusso di Pull Request obbligatorio riduce la stabilità del software e la tracciabilità delle integrazioni. & Attivazione della Branch Protection per inibire i push diretti e rendere mandatorio il processo di Code Review tramite Pull Request. \\
  \hline
  \textbf{Rendicontazione delle ore} & La mancanza di un sistema strutturato per la rendicontazione delle ore lavorate limita la capacità di analisi dell'efficienza e della produttività del team. & Implementazione di un foglio di calcolo condiviso per la registrazione puntuale delle ore dedicate alle attività progettuali, facilitando così il monitoraggio e l'analisi delle performance. \\
  \hline

  \endhead 
\hline
\end{longtable}


\subsection{Valutazioni sui ruoli}
\begin{longtable}{|p{3cm}|p{6cm}|p{5cm}|}
  \hline
  \textbf{Ruolo} & \textbf{Problema Riscontrato} & \textbf{Contromisura Adottata} \\
  \hline
  \endhead
  \textbf{Tutti i ruoli} & Per ottimizzare le ore produttive nelle fasi avanzate, è necessario superare il blocco bisettimanale dei ruoli, che attualmente lascia lacune nella copertura delle attività. & L'assegnazione dei ruoli diviene flessibile su base settimanale, previo allineamento tra le parti, mantenendo l'incompatibilità nel ricoprire funzioni simultanee. \\
\hline
\end{longtable}


\subsection{Valutazioni sugli strumenti}
\begin{longtable}{|p{3cm}|p{6cm}|p{5cm}|}
  \hline
  \textbf{Strumento} & \textbf{Problema Riscontrato} & \textbf{Contromisura Adottata} \\
  \hline
  \endhead
  \textbf{Titolo Problema} & Problema da descrivere & Contromisura spiegata \\
\hline
\end{longtable}


\subsection{Considerazioni finali}
L'iterazione e l'apprendimento continuo guidano la qualità del nostro lavoro. Le retrospettive ci hanno permesso di affinare i processi e aumentare l'efficienza. 
Il team resta focalizzato sul problem-solving proattivo per mantenere alti gli standard produttivi.



\end{document}