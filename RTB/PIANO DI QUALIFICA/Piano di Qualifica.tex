\documentclass[a4paper,11pt]{article}
   
\newcommand{\CurrentVersion}{0.0.9} % ultima versione, da cambiare ad ogni push significativo

\usepackage[utf8]{inputenc}
\usepackage[T1]{fontenc}
\usepackage[italian]{babel}
\usepackage[margin=2.5cm]{geometry}
\usepackage{graphicx}
\usepackage{grffile}
\usepackage{booktabs}
\usepackage{setspace}
\usepackage{titlesec}
\usepackage{float}
\usepackage{ifthen}
\usepackage[table]{xcolor}
\usepackage{tabularx}
\usepackage{needspace}
\usepackage{tcolorbox}
\usepackage{enumitem}
\usepackage[titles]{tocloft}
\usepackage[colorlinks=true,linkcolor=black,urlcolor=blue,citecolor=blue]{hyperref}
\usepackage{xspace}
% Macro
% genera la stringa "\noindent (Riferimento alla tabella decisioni:
% \hyperref[RTB4]{RTB4})", si usa facendo \refDecisione{NomeLabel}{Testo}
\newcommand{\refDecisione}[2]{%
    \noindent (\textbf{Riferimento alla tabella decisioni: \hyperref[#1]{#2}})%
}


\definecolor{primaryblue}{RGB}{0,102,204}
\definecolor{secondaryblue}{RGB}{51,153,255}
\definecolor{lightgray}{RGB}{245,245,245}
\definecolor{darkgray}{RGB}{100,100,100}

\titleformat{\section}
 {\Large\bfseries\color{primaryblue}}
 {\thesection}{1em}{}

\titleformat{\subsection}
 {\large\bfseries\color{primaryblue}} % Sottosezione: colore secondaryblue
 {\thesubsection}{1em}{}

\titleformat{\subsubsection}
 {\normalsize\bfseries\color{secondaryblue}} % Sotto-sottosezione: colore secondaryblue
 {\thesubsubsection}{1em}{}

% per il footer con il numero di pagina
\usepackage{fancyhdr}
\usepackage{lastpage} % per ottenere il numero dell'ultima pagina da mettere nel footer

\usepackage{ltablex} %per far andare a capo le tabelle
\keepXColumns

\renewcommand{\sectionmark}[1]{\markright{#1}}
%\renewcommand{\G}{\textsubscript{\scalebox{0.6}{\textbf{G}}}\ }
\newcommand{\G}{\textsubscript{\scalebox{0.6}{\textbf{G}}}\xspace}

% Macro tabella metrica (tabella uniforme, non spezzata e con stile coerente)
\newcommand{\MetricTable}[1]{
  \Needspace{10\baselineskip}% evita spezzamenti sgradevoli
  \begin{table}[H]
  \centering
  \arrayrulecolor{primaryblue}
  \begin{tabularx}{\linewidth}{|>{\raggedright\arraybackslash}p{2.5cm}|X|>{\raggedright\arraybackslash}p{3cm}|>{\raggedright\arraybackslash}p{3cm}|}
  \hline
  \rowcolor{primaryblue!40}
      \textbf{\color{white} Metrica} & \textbf{\color{white} Descrizione} & \textbf{\color{white} Valore accettazione} & \textbf{\color{white} Valore ideale} \\
  \hline
  \ignorespaces
  #1
  \end{tabularx}
  \end{table}
}



\setlength{\parskip}{4pt}
\setlength{\parindent}{0pt}

\setlist[itemize]{leftmargin=*,itemsep=3pt}
\setlist[enumerate]{leftmargin=*,itemsep=3pt}

\graphicspath{{./}{../assets/images/}{./images/}}

\begin{document}

%configurazione per il footer
\pagestyle{fancy}
\fancyhf{} % pulisce tutti i campi del header e footer

% Header: sinistra e destra
\fancyhead[L]{Gruppo 4 - BugBusters} % sinistra
\fancyhead[R]{Piano di Qualifica\G}   % destra

\fancyfoot[L]{ \thepage\ di \pageref{LastPage}} %definisce il formato del footer
\fancyfoot[R]{ \nouppercase{\rightmark}} % nome della sezione


\renewcommand{\headrulewidth}{0pt} % rimuove la linea dell'header
\renewcommand{\footrulewidth}{0pt} % se vuoi anche togliere eventuale linea del footer

% abilitare numerazione e TOC fino al livello "paragraph" (subsubsubsection)
\setcounter{secnumdepth}{4}
\setcounter{tocdepth}{4}
% formattazione del nuovo livello per avere aspetto coerente
\titleformat{\paragraph}[block]{\normalsize\bfseries\color{secondaryblue}}{\theparagraph}{1em}{}
% alias comodo per usare "subsubsubsection"
\newcommand{\subsubsubsection}{\paragraph}

\begin{center}
  \IfFileExists{../../assets/Logo.jpg}{%
    \includegraphics[width=6cm,height=3cm,keepaspectratio]{../../assets/Logo.jpg} \\[0.8cm]
  }{%
    \fbox{\parbox[c][2.5cm][c]{6cm}{\centering Logo non trovato\\(../../assets/Logo.jpg)}}\\[0.5cm]
  }
  
  {\Large\bfseries\color{primaryblue} BugBusters}\\[0.5cm]

  {\Huge\bfseries\color{primaryblue} Piano di Qualifica\G}\\[0.3cm]
  {\Large\color{secondaryblue} Versione \CurrentVersion}\\[0.8cm]
\end{center}

\begin{center}
\begin{tcolorbox}[colback=lightgray,colframe=primaryblue,width=0.85\textwidth,arc=3mm,boxrule=0.5pt]
% Usa tabularx con una colonna fissa per l'etichetta e una colonna X per il contenuto
\begin{tabularx}{\linewidth}{@{}>{\raggedright\arraybackslash}p{3.5cm}>{\raggedright\arraybackslash}X@{}}
  {Stato} & In redazione \\
  {Verificatore\G} &  \\
  {Redattori} & Luca slongo, Marco Piro \\
  {Distribuzione} & BugBusters, Eggon, Prof. Tullio Vardanega, Prof. Riccardo Cardin \\
\end{tabularx}
\end{tcolorbox}
\end{center}

\vspace{0.5cm}

\begin{center}
\begin{tcolorbox}[colback=secondaryblue!10,colframe=secondaryblue,width=0.9\textwidth,arc=3mm,boxrule=0.8pt,title={\bfseries Descrizione}]
Piano di Qualifica\G del Team BugBusters per il Capitolato\G C5 proposto da Eggon, che ha l'obiettivo di far rispettare uno standard di qualità\G per il codice e rispettare i requisiti funzionali\G prestabiliti.\end{tcolorbox}
\end{center}

\newpage

% Registro delle Modifiche (pagina 2)
\section*{Registro delle Modifiche}

\arrayrulecolor{primaryblue}
{\footnotesize
\begin{tabularx}{\textwidth}{|>{\raggedright\arraybackslash}p{1.5cm}|>{\raggedright\arraybackslash}p{2cm}|X|>{\raggedright\arraybackslash}p{2cm}|>{\raggedright\arraybackslash}p{2cm}|>{\raggedright\arraybackslash}p{2cm}|}
\hline
\rowcolor{primaryblue!40}
\textbf{\color{white} Versione} & \textbf{\color{white} Data} & \textbf{\color{white} Descrizione} & \textbf{\color{white} Redatto} & \textbf{\color{white} Verificato} & \textbf{\color{white} Approvato} \\
\hline
0.0.9 & 01/02/2026 & Aggiunti ai termini presenti nel Glossario\textsubscript{\scalebox{0.6}{\textbf{G}}} la G & Alberto Autiero & - & - \\

\hline
0.0.8 & 01/02/2026 & Aggiunta descrizione grafici metriche & Marco Piro & - & - \\
\hline
0.0.7 & 04/02/2026 & Aggiunti grafici metriche, aggiornamento Test\G, rimossa matrice di Tracciamento & Marco Piro & - & - \\
\hline
0.0.6 & 19/01/2026 & Aggiornamento Test\G, aggiunti test\G di sistema Casi Limite e integrazione, cambiato alcune metriche di prodotto\G. Aggiunta matrice di Tracciamento & Marco Piro & - & - \\
\hline
0.0.5 & 15/01/2026 & Aggiornamento Test\G, aggiunti test\G di sistema prestazionali, eliminata metrica errori ortografici & Marco Piro & - & - \\
\hline
0.0.4 & 11/01/2026 & Aggiunto contenuto alla sezione 5 & Marco Piro & - & - \\
\hline
0.0.3 & 04/01/2026 & Aggiunte sezioni 4 e 5 & Marco Piro & - & - \\
\hline
0.0.2 & 29/12/2025 & Aggiunta Test\G di Sistema e di Accettazione & Marco Piro & - & - \\
\hline
0.0.1 & 03/12/2025 & Prima stesura del documento & Luca Slongo & - & - \\
\hline
\end{tabularx}
}

\newpage

% Indice cliccabile
\setcounter{tocdepth}{3} % Mostra fino al livello di sottosezione (2)
\tableofcontents
\listoftables
\listoffigures

\newpage
\section{Introduzione}

\subsection{Scopo del documento}
Il presente documento, denominato \textit{Piano di Qualifica\G}, ha lo scopo di definire le strategie, le procedure e le metriche adottate dal gruppo \textit{BugBusters} per garantire la qualità\G del prodotto\G software e dei processi produttivi relativi al progetto\G C5 (NEXUM), proposto dall'azienda \textit{Eggon}.

In particolare, questo documento si prefigge di:
\begin{itemize}
    \item \textbf{Definire gli obiettivi di qualità\G:} specificare i target qualitativi per il processo di sviluppo (efficienza\G, stabilità) e per il prodotto\G software (funzionalità\G, affidabilità\G, manutenibilità\G), in conformità con gli standard ISO/IEC 12207 e ISO/IEC 9126;
    \item \textbf{Identificare le metriche:} selezionare gli indicatori quantitativi più idonei per monitorare il raggiungimento degli obiettivi, fissando per ciascuno le soglie di accettazione e di ottimalità;
    \item \textbf{Pianificare le attività di verifica\G e validazione\G:} descrivere le metodologie di test\G (unità, integrazione, sistema, accettazione) e le procedure di analisi statica del codice e della documentazione;
    \item \textbf{Monitorare l'andamento del progetto\G:} fornire un resoconto puntuale (cruscotto\G di valutazione) delle misurazioni effettuate durante le varie fasi del ciclo di vita, permettendo al team di individuare tempestivamente criticità e attuare azioni correttive (miglioramento continuo).
\end{itemize}

\subsection{Glossario\G}
Al fine di evitare ambiguità e garantire una comprensione uniforme della terminologia utilizzata, è stato redatto un documento esterno denominato \textit{Glossario\G}.
I termini tecnici, gli acronimi e le parole con un significato specifico all'interno del progetto\G sono contrassegnati nel testo da una "G" in pedice (es. parola). La loro definizione completa è consultabile nel \textit{Glossario\G}.

\subsection{Riferimenti}

\subsubsection{Riferimenti normativi}
\begin{itemize}
    \item \textbf{Capitolato\G d'appalto C5 - NEXUM (Eggon):}\\
    \url{https://www.math.unipd.it/~tullio/IS-1/2025/Progetto/C5.pdf}
    
    \item \textbf{Norme di Progetto\G (vX.Y.Z):}\\
    Documento interno del gruppo \textit{BugBusters} che definisce le regole, i ruoli e le procedure operative.
    
    \item \textbf{Regolamento del progetto\G didattico:} \\
    \url{https://www.math.unipd.it/~tullio/IS-1/2025/Dispense/PD1.pdf}
\end{itemize}

\subsubsection{Riferimenti informativi}
\begin{itemize}
    \item \textbf{Glossario\G (vX.Y.Z):}\\
    Documento interno del gruppo \textit{BugBusters} contenente le definizioni dei termini tecnici.
    
    \item \textbf{Standard ISO/IEC 12207:1995:}\\
    \textit{Information technology - Software life cycle processes}.\\
    \url{https://www.math.unipd.it/~tullio/IS-1/2009/Approfondimenti/ISO_12207-1995.pdf}
    
    \item \textbf{Standard ISO/IEC 9126:}\\
    \textit{Software engineering - Product quality}.\\
    \url{https://it.wikipedia.org/wiki/ISO/IEC_9126}
    
    \item \textbf{Slide del corso di Ingegneria del Software:} \\
    Materiale didattico fornito dai docenti Prof. Tullio Vardanega e Prof. Riccardo Cardin.
\end{itemize}
\newpage

\section{Obiettivi stabiliti per la qualità\G}

È fondamentale stabilire degli obiettivi da raggiungere per assicurare la qualità\G prefissata del prodotto\G. 
Questo documento definisce i valori di accettazione e ottimalità delle metriche secondo gli standard definiti 
nelle Norme di Progetto\G.

\subsection{Qualità\G di processo}
Un indicatore della qualità\G di un prodotto\G è il metodo con cui è stato sviluppato. Se il processo di sviluppo segue 
delle linee guida ben definite, esso favorisce la buona riuscita del prodotto\G. Come stabilito nelle Norme di Progetto\G, 
nel nostro way of working\G abbiamo adottato lo Standard ISO/IEC 12207:1995 adattandolo alle nostre esigenze e a quelle 
del progetto\G. 

\subsubsection{Processi primari}

I processi primari sono quelle attività che iniziano o eseguono lo sviluppo, l'operazione o la manutenzione di prodotti software. Essi rappresentano le componenti fondamentali del ciclo di vita del progetto\G e sono suddivisi nelle seguenti categorie:
\subsubsubsection{Fornitura}

\MetricTable{
MPC01 & Earned value (EV)\G & $\geq 0$ & $\leq$ EAC \\
\hline
MPC02 & Planned value (PV)\G & $\geq 0$ & $\leq$ Budget at completion (BAC) \\
\hline
MPC03 & Actual cost (AC)\G & $\geq 0$ & $\leq$ EAC \\
\hline
MPC04 & Cost Performance Index (CPI)\G & $\geq 0.9$ & 1 \\
\hline
MPC05 & Schedule Performance Index (SPI)\G & $\geq 0.9$ & 1 \\
\hline
MPC06 & Estimated at completion (EAC) & $\pm 5\%$ rispetto al (BAC) & Budget at completion (BAC) \\
\hline
MPC07 & Estimate to complete (ETC)\G & $\geq 0$ & $\leq$ EAC \\
\hline
MPC08 & Time Estimate At Completion (TEAC)\G & $\geq 0$ & $\leq$ Durata pianificata \\
\hline}



\subsubsubsection{Sviluppo}
\MetricTable{
MPC09 & Requirements Stability Index\G & $\geq 80\%$ & 100\% \\
\hline}


\subsubsection{Processi di supporto}
\subsubsubsection{Documentazione}
\MetricTable{
MPC10 & Indice di Gulpease\G del documento & $\geq 60\%$ & $\geq 80\%$ \\
\hline
}
\subsubsubsection{Verifica\G}
\MetricTable{
MPC11 & Code Coverage\G & $\geq 80\%$ & $100\%$ \\
\hline
MPC12 & Test\G Success Rate & $100\%$ & $100\%$ \\
\hline}
\subsubsubsection{Gestione della qualità\G}
\MetricTable{
MPC13 & Quality metrics satisfied & $\geq 80\%$ & $100\%$ \\
\hline}

\subsubsection{Processi organizzativi}
\subsubsubsection{Gestione dei processi}
\MetricTable{
MPC14 & Time Efficiency & $\geq 50\%$ & $100\%$ \\
\hline}



\subsection{Qualità\G di prodotto\G}

Per qualità\G di prodotto\G si intende una valutazione complessiva del software sia dal punto di vista 
funzionale sia dal punto di vista strutturale. Il codice deve adempiere alle funzionalità\G prestabilite in 
modo efficiente e semplice, e al contempo essere manutenibile, affidabile e portabile. Il gruppo ha aderito allo 
standard ISO/IEC 9126 per garantire il rispetto di queste caratteristiche fondamentali, affinchè il prodotto\G 
sviluppato sia di alta qualità\G.

\subsubsection{Funzionalità\G}
\MetricTable{
MPD01 & Requisiti obbligatori soddisfatti & $100\%$ & $100\%$ \\
\hline
MPD02 & Requisiti desiderabili soddisfatti & $0\%$ & $100\%$ \\
\hline
MPD03 & Requisiti opzionali soddisfatti & $0\%$ & $100\%$ \\
\hline
MPD04 & AI\G Acceptance Rate (Rating $\geq$ 3/5) & $\geq 60\%$ & $\geq 80\%$ \\
\hline}

\subsubsection{Affidabilità\G}
\MetricTable{
MPD05 & Branch Coverage\G & $\geq 70\%$ & $\geq 85\%$ \\
\hline
MPD06 & Defect Density & $\leq 3$ / KLOC & $\leq 1$ / KLOC \\
\hline}

\subsubsection{Efficienza\G}
\MetricTable{
MPD07 & UI Response Time (Interfaccia) & $\leq 2$ sec & $\leq 0.5$ sec \\
\hline
MPD08 & Core Response Time - AI Generativo testo & $\leq 5$ sec & $\leq 3$ sec \\
\hline
MPD09 & Core Response Time - AI Gnerativo immagini & $\leq 10$ sec & $\leq 5$ sec \\
\hline
MPD010 & Core Response Time - AI Co-Pilot & $\leq 10$ sec & $\leq 5$ sec \\
\hline}

\subsubsection{Usabilità}
\MetricTable{
MPD11 & Click Count (Funzioni principali) & $\leq 5$ click & $\leq 3$ click \\
\hline
MPD12 & User Error Rate (Errori validazione\G) & $\leq 10\%$ & $\leq 5\%$ \\
\hline}

\subsubsection{Mantenibilità}
\MetricTable{
MPD13 & Blocker Code Smells & $0$ & $0$ \\
\hline
MPD14 & Cyclomatic complexity\G (per metodo) & $\leq 15$ & $\leq 10$ \\
\hline
MPD15 & Comment Intensity & $\geq 10\%$ & $\geq 20\%$ \\
\hline}

\subsubsection{Portabilità}
\MetricTable{
MPD16 & Supported Browsers (Test\G passati) & $100\%$ (Desktop) & $100\%$ (All devices) \\
\hline}



\section{Metodi di testing}
La strategia di verifica\G e validazione\G adottata dal gruppo \textit{BugBusters} mira a garantire che ogni rilascio software sia conforme ai requisiti specificati e privo di difetti critici.
I test\G dinamici pianificati seguono un approccio incrementale (piramide dei test\G), partendo dalle singole unità logiche fino alla validazione\G dell'intero sistema integrato.

\subsection{Riepilogo dei Requisiti}
La seguente tabella riassume la distribuzione dei requisiti definiti nell'Analisi dei Requisiti\G, che costituiscono la base per la pianificazione dei test\G.

\begin{table}[H]
\centering
\caption{Riepilogo dei requisiti}
\label{tab:riepilogo-requisiti}
\arrayrulecolor{primaryblue}
\begin{tabularx}{0.8\textwidth}{|X|c|c|c|c|}
\hline
\rowcolor{primaryblue!40}
\textbf{\color{white} Tipologia} & \textbf{\color{white} Obbligatorio} & \textbf{\color{white} Desiderabile} & \textbf{\color{white} Opzionale} & \textbf{\color{white} Totale} \\
\hline
Funzionali & 103 & 0 & 25 & 128 \\
\hline
Prestazionali & 5 & 3 & 0 & 8 \\
\hline
Qualità\G & 6 & 0 & 0 & 6 \\
\hline
Vincolo & 5 & 0 & 1 & 6 \\
\hline
\textbf{Totale} & \textbf{119} & \textbf{3} & \textbf{26} & \textbf{148} \\
\hline
\end{tabularx}
\end{table}

\subsection{Test di Integrazione}
I test di integrazione verificano la corretta comunicazione tra i sottosistemi e i moduli definiti nell'architettura, assicurando che le interfacce e lo scambio dati avvengano come previsto.

\begin{longtable}{|p{2.2cm}|p{7cm}|p{3.5cm}|}
  \hline
  \rowcolor{primaryblue!30} 
  \textbf{Codice} & \textbf{Descrizione Interfaccia\G} & \textbf{Moduli Coinvolti} \\
  \hline
  \endhead
  
  \textbf{TI-001} & Verifica scambio dati e gestione errori tramite chiamate API REST (format JSON). & Frontend (Angular) $\leftrightarrow$ Backend (Ruby on Rails) \\
  \hline
  \textbf{TI-002} & Verifica invio del contesto/prompt e ricezione dello stream di risposta dal servizio AI. & Backend (Assistant) $\leftrightarrow$ External LLM API \\
  \hline
  \textbf{TI-003} & Verifica dell'integrità dei dati salvati e recuperati (utenti, documenti, chat log). & Backend Logic $\leftrightarrow$ Database (PostgreSQL) \\
  \hline
  \textbf{TI-004} & Verifica del caricamento file, estrazione testo (OCR) e validazione formato. & Upload Service $\leftrightarrow$ PDF Parser Module \\
  \hline
  \textbf{TI-005} & Verifica dell'aggregazione dei dati per la generazione delle statistiche visualizzate nella dashboard. & Analytics Module $\leftrightarrow$ Database \\
  \hline
  \textbf{TI-006} & Verifica del sistema di autenticazione e gestione sessioni utente. & Auth Controller $\leftrightarrow$ Session Manager \\
  \hline
  
  \caption{Test di Integrazione}
  \label{tab:test-integrazione}
\end{longtable}


\subsection{Test\G di Sistema}

\subsubsection{Test di Sistema - Requisiti Funzionali}
\begin{longtable}{|p{2.2cm}|p{8.3cm}|p{2cm}|p{1.5cm}|}
\hline
\rowcolor{primaryblue!30} 
\textbf{Codice} & \textbf{Descrizione} & \textbf{Riferimento} & \textbf{Stato} \\
\hline
\endhead

% --- AUTENTICAZIONE E PROFILI (1-25) ---
TS-F-001 & Verifica che il sistema permetta la registrazione di un nuovo utente con email e password. & RF-1 & NI \\ \hline
TS-F-002 & Verifica che il sistema impedisca la registrazione con un formato email non valido. & RF-2 & NI \\ \hline
TS-F-003 & Verifica che il sistema richieda l'inserimento della password due volte per conferma. & RF-3 & NI \\ \hline
TS-F-004 & Verifica che il sistema impedisca la registrazione se le password non coincidono. & RF-4 & NI \\ \hline
TS-F-005 & Verifica che il sistema impedisca la registrazione se l'email è già presente nel sistema. & RF-5 & NI \\ \hline
TS-F-006 & Verifica che il sistema invii una email di conferma avvenuta registrazione. & RF-6 & NI \\ \hline
TS-F-007 & Verifica che l'utente possa effettuare il login con credenziali corrette. & RF-7 & NI \\ \hline
TS-F-008 & Verifica che il sistema mostri un errore in caso di credenziali errate. & RF-8 & NI \\ \hline
TS-F-009 & Verifica che il sistema permetta il recupero password tramite email. & RF-9 & NI \\ \hline
TS-F-010 & Verifica che il link di recupero password abbia una scadenza temporale. & RF-10 & NI \\ \hline
TS-F-011 & Verifica che l'utente possa impostare una nuova password tramite il link di recupero. & RF-11 & NI \\ \hline
TS-F-012 & Verifica che la nuova password rispetti i criteri di complessità (lunghezza, caratteri). & RF-12 & NI \\ \hline
TS-F-013 & Verifica che l'utente autenticato possa effettuare il logout. & RF-13 & NI \\ \hline
TS-F-014 & Verifica che il sistema reindirizzi alla home page dopo il logout. & RF-14 & NI \\ \hline
TS-F-015 & Verifica che l'utente autenticato possa visualizzare il proprio profilo. & RF-15 & NI \\ \hline
TS-F-016 & Verifica che l'utente possa modificare i propri dati anagrafici (Nome, Cognome). & RF-16 & NI \\ \hline
TS-F-017 & Verifica che l'utente possa modificare la propria immagine del profilo. & RF-17 & NI \\ \hline
TS-F-018 & Verifica che l'utente possa modificare la password dal profilo (richiedendo la vecchia). & RF-18 & NI \\ \hline
TS-F-019 & Verifica che l'utente possa richiedere la cancellazione del proprio account. & RF-19 & NI \\ \hline
TS-F-020 & Verifica che l'Admin possa visualizzare la lista completa degli utenti. & RF-20 & NI \\ \hline
TS-F-021 & Verifica che l'Admin possa filtrare gli utenti per ruolo (Admin, HR, Analyst, ecc.). & RF-21 & NI \\ \hline
TS-F-022 & Verifica che l'Admin possa disabilitare temporaneamente un utente. & RF-22 & NI \\ \hline
TS-F-023 & Verifica che l'Admin possa riabilitare un utente sospeso. & RF-23 & NI \\ \hline
TS-F-024 & Verifica che l'Admin possa modificare il ruolo di un utente. & RF-24 & NI \\ \hline
TS-F-025 & Verifica che l'Admin possa visualizzare i log di accesso degli utenti. & RF-25 & NI \\ \hline

% --- ASSISTANT AI (26-50) ---
TS-F-026 & Verifica che l'HR Manager possa accedere alla sezione Assistant. & RF-26 & NI \\ \hline
TS-F-027 & Verifica che il sistema permetta di inserire un prompt\G testuale multiriga. & RF-27 & NI \\ \hline
TS-F-028 & Verifica che il sistema permetta di selezionare un template\G di prompt predefinito. & RF-28 & NI \\ \hline
TS-F-029 & Verifica che il sistema permetta di scegliere il tono della risposta (es. Formale, Empatico). & RF-29 & NI \\ \hline
TS-F-030 & Verifica che il sistema permetta di definire la lunghezza approssimativa dell'output. & RF-30 & NI \\ \hline
TS-F-031 & Verifica che il sistema mostri uno stato di "Elaborazione in corso" dopo l'invio. & RF-31 & NI \\ \hline
TS-F-032 & Verifica che il sistema restituisca una risposta testuale generata dall'AI\G. & RF-32 & NI \\ \hline
TS-F-033 & Verifica che il sistema formatti correttamente la risposta (elenchi puntati, paragrafi). & RF-33 & NI \\ \hline
TS-F-034 & Verifica che l'utente possa interrompere la generazione in corso (Stop). & RF-34 & NI \\ \hline
TS-F-035 & Verifica che l'utente possa copiare l'intera risposta negli appunti con un click. & RF-35 & NI \\ \hline
TS-F-036 & Verifica che l'utente possa valutare la risposta positivamente (Pollice su). & RF-36 & NI \\ \hline
TS-F-037 & Verifica che l'utente possa valutare la risposta negativamente (Pollice giù). & RF-37 & NI \\ \hline
TS-F-038 & Verifica che il sistema richieda un motivo facoltativo in caso di feedback negativo. & RF-38 & NI \\ \hline
TS-F-039 & Verifica che l'utente possa rigenerare la risposta mantenendo lo stesso prompt. & RF-39 & NI \\ \hline
TS-F-040 & Verifica che il sistema salvi la conversazione nello storico. & RF-40 & NI \\ \hline
TS-F-041 & Verifica che l'utente possa visualizzare l'elenco delle chat passate. & RF-41 & NI \\ \hline
TS-F-042 & Verifica che l'utente possa rinominare una chat nello storico. & RF-42 & NI \\ \hline
TS-F-043 & Verifica che l'utente possa eliminare una singola chat dallo storico. & RF-43 & NI \\ \hline
TS-F-044 & Verifica che l'utente possa effettuare una ricerca testuale nello storico chat. & RF-44 & NI \\ \hline
TS-F-045 & Verifica che il sistema gestisca correttamente il limite di token per richiesta. & RF-45 & NI \\ \hline
TS-F-046 & Verifica che il sistema mostri un avviso se il servizio AI non è raggiungibile. & RF-46 & NI \\ \hline
TS-F-047 & Verifica che il sistema permetta di esportare la chat in formato PDF. & RF-47 & NI \\ \hline
TS-F-048 & Verifica che il sistema permetta di esportare la chat in formato TXT. & RF-48 & NI \\ \hline
TS-F-049 & Verifica che il sistema mantenga il contesto della conversazione (multi-turn). & RF-49 & NI \\ \hline
TS-F-050 & Verifica che l'utente possa pulire il contesto per iniziare un nuovo argomento. & RF-50 & NI \\ \hline

% --- CO-PILOT E DOCUMENTI (51-85) ---
TS-F-051 & Verifica che l'Operatore possa accedere alla sezione Co-Pilot\G. & RF-51 & NI \\ \hline
TS-F-052 & Verifica che il sistema permetta il caricamento di file PDF (Drag \& Drop). & RF-52 & NI \\ \hline
TS-F-053 & Verifica che il sistema permetta il caricamento tramite selezione file system. & RF-53 & NI \\ \hline
TS-F-054 & Verifica che il sistema blocchi file con estensione diversa da .pdf. & RF-54 & NI \\ \hline
TS-F-055 & Verifica che il sistema blocchi file superiori alla dimensione massima (es. 20MB). & RF-55 & NI \\ \hline
TS-F-056 & Verifica che il sistema visualizzi una barra di avanzamento durante l'upload. & RF-56 & NI \\ \hline
TS-F-057 & Verifica che il sistema avvii automaticamente l'analisi OCR/AI dopo l'upload. & RF-57 & NI \\ \hline
TS-F-058 & Verifica che il sistema visualizzi l'anteprima del PDF caricato nel visualizzatore. & RF-58 & NI \\ \hline
TS-F-059 & Verifica che il sistema permetta lo zoom e lo scroll del PDF. & RF-59 & NI \\ \hline
TS-F-060 & Verifica che il sistema estragga e mostri i campi chiave (Data, Importo, Nominativo). & RF-60 & NI \\ \hline
TS-F-061 & Verifica che il sistema evidenzi sul PDF la posizione del dato estratto (bounding box). & RF-61 & NI \\ \hline
TS-F-062 & Verifica che l'utente possa cliccare su un campo estratto per focalizzare il PDF. & RF-62 & NI \\ \hline
TS-F-063 & Verifica che il sistema indichi il livello di confidenza per ogni dato estratto. & RF-63 & NI \\ \hline
TS-F-064 & Verifica che l'utente possa modificare manualmente un valore estratto errato. & RF-64 & NI \\ \hline
TS-F-065 & Verifica che l'utente possa aggiungere un campo mancante non rilevato. & RF-65 & NI \\ \hline
TS-F-066 & Verifica che l'utente possa eliminare un campo estratto erroneamente. & RF-66 & NI \\ \hline
TS-F-067 & Verifica che l'utente possa salvare le modifiche ai dati estratti. & RF-67 & NI \\ \hline
TS-F-068 & Verifica che il sistema validi i tipi di dato (es. formato data corretto). & RF-68 & NI \\ \hline
TS-F-069 & Verifica che l'utente possa confermare l'analisi del documento (Stato "Validato"). & RF-69 & NI \\ \hline
TS-F-070 & Verifica che il sistema permetta di scartare il documento se illeggibile. & RF-70 & NI \\ \hline
TS-F-071 & Verifica che il sistema mostri la lista dei documenti caricati con relativo stato. & RF-71 & NI \\ \hline
TS-F-072 & Verifica che l'utente possa filtrare i documenti per stato (Da validare, Validato). & RF-72 & NI \\ \hline
TS-F-073 & Verifica che l'utente possa cercare un documento per nome file. & RF-73 & NI \\ \hline
TS-F-074 & Verifica che l'utente possa ordinare la lista per data di caricamento. & RF-74 & NI \\ \hline
TS-F-075 & Verifica che l'utente possa scaricare il file PDF originale. & RF-75 & NI \\ \hline
TS-F-076 & Verifica che l'utente possa scaricare i dati estratti in formato JSON/CSV. & RF-76 & NI \\ \hline
TS-F-077 & Verifica che il sistema gestisca il caricamento multiplo (Batch Upload). & RF-77 & NI \\ \hline
TS-F-078 & Verifica che l'utente possa eliminare un documento dal sistema. & RF-78 & NI \\ \hline
TS-F-079 & Verifica che il sistema permetta di associare tag o etichette al documento. & RF-79 & NI \\ \hline
TS-F-080 & Verifica che il sistema permetta di ruotare le pagine del PDF. & RF-80 & NI \\ \hline
TS-F-081 & Verifica che il sistema rilevi documenti protetti da password e chieda lo sblocco. & RF-81 & NI \\ \hline
TS-F-082 & Verifica che il sistema permetta di inviare i dati estratti via email (Template). & RF-82 & NI \\ \hline
TS-F-083 & Verifica che il sistema precompili il template email con i dati estratti. & RF-83 & NI \\ \hline
TS-F-084 & Verifica che l'utente possa modificare il testo dell'email prima dell'invio. & RF-84 & NI \\ \hline
TS-F-085 & Verifica che il sistema confermi l'avvenuto invio dell'email. & RF-85 & NI \\ \hline

% --- ANALYTICS E DASHBOARD (86-105) ---
TS-F-086 & Verifica che il Data Analyst possa accedere alla Dashboard Analytics. & RF-86 & NI \\ \hline
TS-F-087 & Verifica che il sistema mostri il numero totale di documenti processati. & RF-87 & NI \\ \hline
TS-F-088 & Verifica che il sistema mostri la percentuale di correzioni manuali (tasso errore AI). & RF-88 & NI \\ \hline
TS-F-089 & Verifica che il sistema mostri il numero di chat avviate nell'Assistant. & RF-89 & NI \\ \hline
TS-F-090 & Verifica che il sistema mostri la distribuzione dei feedback utente (Rating). & RF-90 & NI \\ \hline
TS-F-091 & Verifica che il sistema mostri il costo stimato delle API AI consumate. & RF-91 & NI \\ \hline
TS-F-092 & Verifica che il sistema permetta di filtrare le statistiche per periodo (Giorno, Mese, Anno). & RF-92 & NI \\ \hline
TS-F-093 & Verifica che il sistema mostri un grafico temporale dell'utilizzo. & RF-93 & NI \\ \hline
TS-F-094 & Verifica che il sistema mostri la classifica degli utenti più attivi. & RF-94 & NI \\ \hline
TS-F-095 & Verifica che il sistema permetta di esportare il report analytics in PDF. & RF-95 & NI \\ \hline
TS-F-096 & Verifica che il sistema permetta di esportare i dati grezzi in CSV. & RF-96 & NI \\ \hline
TS-F-097 & Verifica che il sistema mostri i tempi medi di risposta dell'AI. & RF-97 & NI \\ \hline
TS-F-098 & Verifica che il sistema mostri i tempi medi di elaborazione documenti. & RF-98 & NI \\ \hline
TS-F-099 & Verifica che l'Admin possa configurare le soglie di allarme per i costi. & RF-99 & NI \\ \hline
TS-F-100 & Verifica che il sistema invii notifica se il budget token è quasi esaurito. & RF-100 & NI \\ \hline
TS-F-101 & Verifica che la Dashboard si aggiorni in tempo reale (o near real-time). & RF-101 & NI \\ \hline
TS-F-102 & Verifica che il sistema permetta di confrontare periodi diversi (Mese corrente vs precedente). & RF-102 & NI \\ \hline
TS-F-103 & Verifica che il sistema mostri la ripartizione dei documenti per tipologia/tag. & RF-103 & NI \\ \hline
TS-F-104 & Verifica che il sistema mostri gli errori API più frequenti. & RF-104 & NI \\ \hline
TS-F-105 & Verifica che l'accesso ai dati sensibili analytics sia limitato ai ruoli autorizzati. & RF-105 & NI \\ \hline

% --- SISTEMA, SICUREZZA E VINCOLI (106-132) ---
TS-F-106 & Verifica che il sistema mostri una pagina 404 personalizzata per percorsi inesistenti. & RF-106 & NI \\ \hline
TS-F-107 & Verifica che il sistema mostri una pagina 500 generica in caso di crash server. & RF-107 & NI \\ \hline
TS-F-108 & Verifica che il sistema gestisca il timeout della sessione dopo inattività (es. 30 min). & RF-108 & NI \\ \hline
TS-F-109 & Verifica che il sistema permetta di estendere la sessione prima del timeout. & RF-109 & NI \\ \hline
TS-F-110 & Verifica che il sistema visualizzi un banner per il consenso dei cookie (GDPR). & RF-110 & NI \\ \hline
TS-F-111 & Verifica che il sistema permetta di accettare o rifiutare i cookie non essenziali. & RF-111 & NI \\ \hline
TS-F-112 & Verifica che il sistema disponga di una pagina Privacy Policy accessibile. & RF-112 & NI \\ \hline
TS-F-113 & Verifica che il sistema disponga di una pagina Termini di Servizio. & RF-113 & NI \\ \hline
TS-F-114 & Verifica che il sistema supporti la navigazione tramite tastiera (Tab index). & RF-114 & NI \\ \hline
TS-F-115 & Verifica che il sistema utilizzi testi alternativi (alt text) per le immagini. & RF-115 & NI \\ \hline
TS-F-116 & Verifica che il sistema sia responsive su dispositivi Tablet (Portrait/Landscape). & RF-116 & NI \\ \hline
TS-F-117 & Verifica che il sistema sia responsive su dispositivi Mobile. & RF-117 & NI \\ \hline
TS-F-118 & Verifica che il sistema supporti la modalità scura (Dark Mode). & RF-118 & NI \\ \hline
TS-F-119 & Verifica che il sistema mantenga la preferenza del tema (Dark/Light) tra le sessioni. & RF-119 & NI \\ \hline
TS-F-120 & Verifica che il sistema sanitizzi tutti gli input utente (Prevenzione XSS). & RF-120 & NI \\ \hline
TS-F-121 & Verifica che il sistema utilizzi token CSRF per le form. & RF-121 & NI \\ \hline
TS-F-122 & Verifica che le password siano salvate con hash sicuro (es. BCrypt). & RF-122 & NI \\ \hline
TS-F-123 & Verifica che le comunicazioni avvengano esclusivamente su HTTPS. & RF-123 & NI \\ \hline
TS-F-124 & Verifica che gli URL non contengano parametri sensibili in chiaro. & RF-124 & NI \\ \hline
TS-F-125 & Verifica che il sistema gestisca correttamente i caratteri speciali (UTF-8) nei form. & RF-125 & NI \\ \hline
TS-F-126 & Verifica che il sistema fornisca feedback visivo immediato (tooltip/errori) nei campi form. & RF-126 & NI \\ \hline
TS-F-127 & Verifica che il sistema permetta di contattare il supporto tecnico via form integrato. & RF-127 & NI \\ \hline
TS-F-128 & Verifica che il sistema mostri la versione attuale della build nel footer. & RF-128 & NI \\ \hline
TS-F-129 & Verifica che il sistema registri un Audit Log per le operazioni critiche (Delete/Edit). & RF-129 & NI \\ \hline
TS-F-130 & Verifica che l'utente possa scaricare i propri dati personali (Data Portability). & RF-130 & NI \\ \hline
TS-F-131 & Verifica che il sistema gestisca correttamente upload concorrenti di più utenti. & RF-131 & NI \\ \hline
TS-F-132 & Verifica che il sistema permetta la visualizzazione delle notifiche di sistema (es. Manutenzione). & RF-132 & NI \\ \hline

\caption{Test di Sistema - Requisiti Funzionali}
\label{tab:test-sistema-funzionali}
\end{longtable}




\newpage

\subsubsection{Test\G di Sistema - Requisiti Prestazionali\G}
\begin{longtable}{|p{2.5cm}|p{8cm}|p{2.5cm}|p{1.5cm}|}
\hline
\rowcolor{primaryblue!30} 
\textbf{Codice} & \textbf{Descrizione} & \textbf{Riferimento} & \textbf{Stato} \\
\hline
\endhead

TS-P-001 & Verifica\G che il sistema generi contenuti testuali tramite AI\G (Assistant) entro 5 secondi per testi fino a 500 parole & RP-01 & NI \\ \hline
TS-P-002 & Verifica\G che il sistema classifichi e partizioni documenti PDF (Co-Pilot) entro 3 secondi per pagina & RP-02 & NI \\ \hline
TS-P-003 & Verifica\G che il tempo di risposta dell'interfaccia\G utente per operazioni standard sia inferiore a 2 secondi & RP-03 & NI \\ \hline
TS-P-004 & Verifica\G che il sistema supporti l'upload di file PDF fino a 20 MB & RP-04 & NI \\ \hline
TS-P-005 & Verifica\G che la Dashboard\G di Analytics carichi le statistiche entro 3 secondi per dataset fino a 1000 documenti & RP-05 & NI \\ \hline
TS-P-006 & Verifica\G che il sistema garantisca una disponibilità del 99\% durante l'orario lavorativo (8:00-18:00) & RP-06 & NI \\ \hline
TS-P-007 & Verifica\G che il sistema sia in grado di processare almeno 50 documenti in parallelo senza degrado prestazionale & RP-07 & NI \\ \hline
TS-P-008 & Verifica\G utilizzo risorse CPU sotto carico massimo (Desiderabile) & RP-08 & NI \\ \hline

\caption{Test\G di Sistema per Requisiti Prestazionali\G}
\label{tab:test-sistema-prestazionali}
\end{longtable}

\subsubsection{Test\G di Sistema - Requisiti di Qualità\G}
\begin{longtable}{|p{2.5cm}|p{8cm}|p{2.5cm}|p{1.5cm}|}
\hline
\rowcolor{primaryblue!30} 
\textbf{Codice} & \textbf{Descrizione} & \textbf{Riferimento} & \textbf{Stato} \\
\hline
\endhead

TS-Q-001 & Verifica\G che sia presente la documentazione tecnica completa (diagrammi e descrizioni Use Case) & RQ-01 & NI \\ \hline
TS-Q-002 & Verifica\G che il codice sorgente sia commentato secondo gli standard definiti nelle Norme di Progetto\G & RQ-02 & NI \\ \hline
TS-Q-003 & Verifica\G che sia presente il manuale utente per l'installazione e l'utilizzo del sistema & RQ-03 & NI \\ \hline
TS-Q-004 & Verifica\G che il codice superi l'analisi statica senza errori critici (Code Smells) & RQ-04 & NI \\ \hline
TS-Q-005 & Verifica\G che l'interfaccia\G utente sia accessibile secondo le linee guida WCAG 2.1 (livello AA) & RQ-05 & NI \\ \hline
TS-Q-006 & Verifica\G che il codice sia coperto da test\G di unità per almeno l'80\% (Code Coverage\G) & RQ-06 & NI \\ \hline

\caption{Test\G di Sistema per Requisiti di Qualità\G}
\label{tab:test-sistema-qualita}
\end{longtable}

\newpage

\subsubsection{Test\G di Sistema - Requisiti di Vincolo\G}
\begin{longtable}{|p{2.5cm}|p{8cm}|p{2.5cm}|p{1.5cm}|}
\hline
\rowcolor{primaryblue!30} 
\textbf{Codice} & \textbf{Descrizione} & \textbf{Riferimento} & \textbf{Stato} \\
\hline
\endhead

TS-V-001 & Verifica\G che il sistema utilizzi Git come sistema di controllo versione & RV-01 & NI \\ \hline
TS-V-002 & Verifica\G che API\G e Backend\G siano sviluppati in Ruby on Rails\G & RV-02 & NI \\ \hline
TS-V-003 & Verifica\G che il Frontend sia sviluppato utilizzando il framework React & RV-03 & NI \\ \hline
TS-V-004 & Verifica\G che il sistema sia compatibile con i browser Google Chrome e Mozilla Firefox (ultime versioni) & RV-04 & NI \\ \hline
TS-V-005 & Verifica\G che l'interfaccia\G sia responsive e utilizzabile da dispositivi mobili & RV-05 & NI \\ \hline
TS-V-006 & Verifica\G che la documentazione del codice sia redatta in lingua inglese (Opzionale) & RV-06 & NI \\ \hline

\caption{Test\G di Sistema per Requisiti di Vincolo\G}
\label{tab:test-sistema-vincolo}
\end{longtable}

\subsection{Test\G di Accettazione}
I test\G di accettazione validano il sistema rispetto agli scenari d'uso (Use Case) previsti, assicurando che l'utente possa completare i flussi di lavoro principali.

\begin{longtable}{|p{2.0cm}|p{8.5cm}|p{2.5cm}|p{2.5cm}|}
  \hline
  \rowcolor{primaryblue!30} 
  \textbf{Codice} & \textbf{Descrizione} & \textbf{Riferimento} & \textbf{Stato} \\
  \hline
  \endhead
  
  % --- UTENTE GENERICO ---
  \textbf{TA-001} & Verifica\G che un utente non registrato possa completare la procedura di registrazione (Happy Path). & UC-0A & Non Impl. \\
  \hline
  \textbf{TA-002} & Verifica\G che il sistema impedisca la registrazione con dati non validi o email già esistente. & UC-0A (Scenari alternativi) & Non Impl. \\
  \hline
  \textbf{TA-003} & Verifica\G che l'utente possa effettuare il Login e il Logout correttamente. & UC-0B, UC-0G & Non Impl. \\
  \hline
  \textbf{TA-004} & Verifica\G che l'utente possa visualizzare e modificare il proprio profilo e cambiare la password. & UC-0C, UC-0D & Non Impl. \\
  \hline
  
  % --- AMMINISTRATORE ---
  \textbf{TA-005} & Verifica\G che l'Amministratore\G possa consultare la lista utenti e visualizzare i dettagli di un singolo utente. & UC-0E & Non Impl. \\
  \hline
  \textbf{TA-006} & Verifica\G che l'Amministratore\G possa modificare il ruolo di un utente o eliminarlo. & UC-0F & Non Impl. \\
  \hline
  
  % --- HR MANAGER (ASSISTANT) ---
  \textbf{TA-007} & Verifica\G che l'HR Manager possa configurare una richiesta (Prompt\G, Tono, Lunghezza) e generare un contenuto. & UC-1A, UC-1B, UC-1C & Non Impl. \\
  \hline
  \textbf{TA-008} & Verifica\G che l'HR Manager possa visualizzare, copiare e modificare il testo generato dall'AI\G. & UC-1D, UC-1E & Non Impl. \\
  \hline
  \textbf{TA-009} & Verifica\G che l'HR Manager possa valutare (Feedback) o scartare un contenuto generato. & UC-1F, UC-1N & Non Impl. \\
  \hline
  \textbf{TA-010} & Verifica\G il salvataggio automatico nello storico e la possibilità di recuperare generazioni passate. & UC-1O & Non Impl. \\
  \hline
  
  % --- OPERATORE CDL (CO-PILOT) ---
  \textbf{TA-011} & Verifica\G che l'Operatore possa caricare un documento PDF e avviare l'analisi automatica. & UC-2A & Non Impl. \\
  \hline
  \textbf{TA-012} & Verifica\G che il sistema estragga correttamente i dati e li mostri all'operatore. & UC-2B & Non Impl. \\
  \hline
  \textbf{TA-013} & Verifica\G lo scenario\G "Human-in-the-loop": l'operatore corregge manualmente un dato estratto errato e conferma. & UC-2D, UC-2E & Non Impl. \\
  \hline
  \textbf{TA-014} & Verifica\G gestione template\G: creazione, modifica e utilizzo di un template\G di messaggio. & UC-2I & Non Impl. \\
  \hline
  \textbf{TA-015} & Verifica\G il flusso di invio: selezione destinatari, associazione documento e invio email (o pianificazione). & UC-2L, UC-2O & Non Impl. \\
  \hline
  
  % --- DATA ANALYST ---
  \textbf{TA-016} & Verifica\G che il Data Analyst\G possa consultare le Dashboard\G e filtrare le metriche per periodo temporale. & UC-3A, UC-3B & Non Impl. \\
  \hline

    \caption{Test\G di Accettazione}
    \label{tab:test-accettazione}
\end{longtable}

\section{Cruscotto\G di Valutazione}
Di seguito verranno mostrate le misurazioni effettuate durante il periodo che va 
dall'aggiudicazione del capitolato\G
sino alla Requirements and Technology Baseline (RTB)\G. 

\subsection{MPC01 e MPC02 - Earned Value (EV)\G e Planned Value (PV)\G}
 \begin{figure}[H]
     \centering
     \includegraphics[width=0.9\textwidth]{grafici/MPC01_02.png}
     \caption{Grafico per periodo di MPC01 e MPC02}
 \end{figure}

 Dal grafico si osserva che l'andamento del Valore Guadagnato (\textit{Earned Value} - EV\G) segue fedelmente quello del Valore Pianificato (\textit{Planned Value} - PV\G), con un trend crescente che culmina nel sesto sprint\G, in corrispondenza del completamento delle attività per la \textit{Requirements and Technology Baseline} (RTB)\G.

\newpage

\subsection{MPC03 e MPC07 - Actual cost (AC)\G e Estimate to complete (ETC)\G}
\begin{figure}[H]
  \centering
  \includegraphics[width=0.9\textwidth]{grafici/MPC03_07.png}
  \caption{Grafico per periodo di MPC03 e MPC07}
\end{figure}
L'andamento della metrica MPC03 (\textit{Actual Cost\G}) mostra una crescita costante dei costi sostenuti, in linea con l'intensificazione delle attività produttive durante la fase di \textit{Requirements and Technology Baseline} (RTB)\G. 
Tale incremento, culminato nel sesto sprint\G, rispecchia fedelmente la pianificazione temporale definita nel Piano di Progetto\G, dove il maggior carico di lavoro (e quindi di spesa) era previsto proprio nelle settimane antecedenti la consegna della candidatura\G.

Parallelamente, la metrica MPC07 (\textit{Estimate to Complete\G}) evidenzia una progressiva diminuzione del budget residuo necessario per il completamento del progetto\G. Questo trend inverso conferma che le risorse sono state consumate in modo coerente con l'avanzamento dei lavori, avvicinando il progetto\G al traguardo della \textit{Product Baseline} (PB)\G senza generare extra-costi imprevisti.

\newpage

\subsection{MPC04 e MP05 - Cost Performance Index (CPI)\G e Schedule performance Index\G}
\begin{figure}[H]
  \centering
  \includegraphics[width=0.9\textwidth]{grafici/MPC04_05.png}
  \caption{Grafico per periodo di MPC04 e MPC05}
\end{figure}

L'analisi del \textit{Cost Performance Index} (CPI)\G mostra un percorso di netta crescita. Il progetto\G è iniziato con un indice inferiore alle aspettative, a causa delle difficoltà iniziali.

Dopo l'investimento iniziale, il processo produttivo è diventato altamente sostenibile, permettendo di recuperare il budget consumato.
Parallelamente, lo \textit{Schedule Performance Index} (SPI)\G si è mantenuto stabile e vicino al valore ideale per tutto il periodo, garantendo il rispetto delle scadenze per la candidatura\G.

\newpage

\subsection{MPC06 - Estimated at completion (EAC)}
\begin{figure}[H]
  \centering
  \includegraphics[width=0.9\textwidth]{grafici/MPC06.png}
  \caption{Grafico per periodo di MPC06}
\end{figure}

L'andamento del costo stimato a finire (Estimated at Completion - EAC) racconta chiaramente il percorso di ottimizzazione intrapreso dal team. Il progetto\G ha attraversato una fase iniziale critica durante i primi sprint\G, in cui la stima dei costi\G finali superava sensibilmente il budget stanziato. Questa proiezione negativa era la diretta conseguenza delle difficoltà  iniziali  che avevano abbassato l'indice di efficienza\G CPI\G. Succesivamente si è innescato un trend di recupero costante. 

\newpage

\subsection{MPC08 - Time Estimate At Completion\G}
\begin{figure}[H]
  \centering
  \includegraphics[width=0.9\textwidth]{grafici/MPC08.png}
  \caption{Grafico per periodo di MPC08}
\end{figure}
L'andamento della stima temporale a finire (\textit{Time Estimate At Completion\G}) conferma la solidità della pianificazione iniziale. La proiezione della data di completamento per la fase di \textit{Requirements and Technology Baseline} (RTB)\G è rimasta sostanzialmente invariata lungo tutto l'arco temporale osservato.

\newpage

\subsection{MPC09 - Requirements Stability Index (RSI)\G}
\begin{figure}[H]
  \centering
  \includegraphics[width=0.9\textwidth]{grafici/MPC09.png}
  \caption{Grafico per periodo di MPC09}
\end{figure}
L'indice di stabilità dei requisiti (\textit{Requirements Stability Index\G}) mostra un andamento che riflette fedelmente il ciclo di vita dell'Analisi dei Requisiti\G.
Nello Sprint\G 2 si registra un picco negativo significativo. Tale valore, apparentemente critico, è in realtà indicatore di una intensa attività produttiva. Partendo da un set iniziale di requisiti, il team ha effettuato un'opera di espansione e dettaglio massiccia. Matematicamente, ciò ha portato il numero delle modifiche a superare il numero dei requisiti iniziali, generando l'indice negativo. Superata la fase critica di definizione, l'indice è risalito rapidamente. 

\newpage

\subsection{MPC10 - Indice di Gulpease\G}
\begin{figure}[H]
  \centering
  \includegraphics[width=0.9\textwidth]{grafici/MPC10.png}
  \caption{Grafico per periodo di MPC10}
\end{figure}
In linea generale, il gruppo BugBusters ha posto grande attenzione alla redazione della documentazione: l'obiettivo primario è sempre stato quello di produrre elaborati che fossero non solo corretti tecnicamente, ma anche facilmente fruibili da tutti gli stakeholder\G.
Dall'analisi dei dati emerge una disparità nei valori di leggibilità tra le diverse tipologie di documenti, dovuta alla natura intrinseca del loro contenuto.
I dati sulla leggibilità mostrano una chiara differenza tra i documenti. L'Analisi dei Requisiti\G supera abbondantemente la soglia ottima grazie alla scelta di usare frasi brevi e semplici, ideali per farsi capire chiaramente dal cliente. Al contrario, il Glossario\G e le Norme di Progetto\G rimangono sotto la soglia minima per motivi strutturali: il primo è penalizzato dalla presenza di parole tecniche molto lunghe, mentre le seconde richiedono un linguaggio formale e rigoroso che non può essere semplificato oltre un certo limite senza perdere di precisione.

Il gruppo si impegna comunque, nelle prossime iterazioni, a raffinare ulteriormente la sintassi di tali documenti per migliorarne la leggibilità senza comprometterne il rigore formale.

\newpage


\subsection{MPC13 - Quality metrics satisfied}
\begin{figure}[H]
  \centering
  \includegraphics[width=0.9\textwidth]{grafici/MPC13.png}
  \caption{Grafico per periodo di MPC13}
\end{figure}
L'andamento della percentuale di metriche soddisfatte offre una sintesi efficace della maturazione qualitativa del progetto\G. Il primo sprint\G ha risentito della bassa efficienza\G economica iniziale (CPI\G sotto soglia), mentre nei successivi due sprint\G è stato l'Indice di Stabilità dei Requisiti (RSI)\G a mancare l'obiettivo, a causa della necessaria fase di espansione dell'Analisi dei Requisiti\G.  
Superata la fase di assestamento, il trend ha mostrato un miglioramento netto. A partire dal quarto sprint\G, il team ha raggiunto una stabilità su tutti i fronti monitorati (Costi, Tempi, Documentazione e Processi), mantenendo l'indicatore vicino al valore minimo accettabile fino al termine della fase RTB\G. Questo risultato conferma che le misure correttive adottate sono state risolutive, portando il processo produttivo a un livello di affidabilità\G ottimale proprio nel momento decisivo della candidatura\G.


\newpage

\subsection{MPC14 - Time Efficiency}
\begin{figure}[H]
  \centering
  \includegraphics[width=0.9\textwidth]{grafici/MPC14.png}
  \caption{Grafico per periodo di MPC14}
\end{figure}
L'analisi dell'efficienza\G temporale mostra un andamento notevolmente stabile. Questa costanza è un segnale positivo: indica che il team è riuscito a mantenere un rapporto equilibrato tra il lavoro produttivo (stesura documenti, sviluppo) e le ore di gestione (riunioni, auto-formazione), senza mai farsi sopraffare dall'overhead organizzativo.
Le lievi flessioni registrate nella fase centrale (Sprint\G 4 e 5) sono fisiologiche e riconducibili principalmente al rischio\G riguardante la sovrapposizione con sessione d'esami e alla necessità di maggiori confronti interni per la riorganizzazione dell'Analisi dei Requisiti\G.
Il picco positivo raggiunto nel sesto sprint\G testimonia la capacità  del gruppo di massimizzare la produttività nelle settimane decisive per la chiusura della candidatura\G.

\newpage

\section{Iniziative di miglioramento}
L'ottimizzazione costante dei processi costituisce un pilastro fondamentale per la riuscita del progetto\G. Di seguito vengono esposte le problematiche operative riscontrate e le relative strategie di risoluzione adottate per superare tali ostacoli.

\subsection{Valutazioni sull'organizzazione}
\begin{longtable}{|p{3cm}|p{6cm}|p{5cm}|}
  \hline
  \textbf{Area} & \textbf{Problema Riscontrato} & \textbf{Contromisura Adottata} \\
  \hline
  \textbf{Tracciabilità\G} & L'assenza di un sistema di monitoraggio puntuale delle attività ostacola il flusso produttivo e compromette l'efficacia\G della programmazione operativa. & Adozione della funzionalità\G 'Issues' di GitHub\G per ottimizzare il controllo operativo e la supervisione dei flussi di lavoro. \\
  \hline
  \textbf{Controllo delle modifiche} & Operare senza un flusso di Pull Request obbligatorio riduce la stabilità del software e la tracciabilità\G delle integrazioni. & Attivazione della Branch Protection per inibire i push diretti e rendere mandatorio il processo di Code Review tramite Pull Request. \\
  \hline
  \textbf{Rendicontazione delle ore} & La mancanza di un sistema strutturato per la rendicontazione delle ore lavorate limita la capacità di analisi dell'efficienza\G e della produttività del team. & Implementazione\G di un foglio di calcolo condiviso per la registrazione puntuale delle ore dedicate alle attività progettuali, facilitando così il monitoraggio e l'analisi delle performance. \\
  \hline

  \endhead 
\hline
\end{longtable}


\subsection{Valutazioni sui ruoli}
\begin{longtable}{|p{3cm}|p{6cm}|p{5cm}|}
  \hline
  \textbf{Ruolo} & \textbf{Problema Riscontrato} & \textbf{Contromisura Adottata} \\
  \hline
  \endhead
  \textbf{Tutti i ruoli} & Per ottimizzare le ore produttive nelle fasi avanzate, è necessario superare il blocco bisettimanale dei ruoli, che attualmente lascia lacune nella copertura delle attività. & L'assegnazione dei ruoli diviene flessibile su base settimanale, previo allineamento tra le parti, mantenendo l'incompatibilità nel ricoprire funzioni simultanee. \\
\hline
\end{longtable}


\subsection{Valutazioni sugli strumenti}
\begin{longtable}{|p{3cm}|p{6cm}|p{5cm}|}
  \hline
  \textbf{Strumento} & \textbf{Problema Riscontrato} & \textbf{Contromisura Adottata} \\
  \hline
  \endhead
  \textbf{Titolo Problema} & Problema da descrivere & Contromisura spiegata \\
\hline
\end{longtable}


\subsection{Considerazioni finali}
L'iterazione e l'apprendimento continuo guidano la qualità\G del nostro lavoro. Le retrospettive ci hanno permesso di affinare i processi e aumentare l'efficienza\G. 
Il team resta focalizzato sul problem-solving proattivo per mantenere alti gli standard produttivi.



\end{document}