\documentclass[a4paper,12pt]{article}

\usepackage[utf8]{inputenc}
\usepackage[T1]{fontenc}
\usepackage[italian]{babel}
\usepackage[margin=2.5cm]{geometry}
\usepackage{graphicx}
\usepackage{grffile}
\usepackage{booktabs}
\usepackage{setspace}
\usepackage{titlesec}
\usepackage{float}
\usepackage{ifthen}
\usepackage{xcolor}
\usepackage{tcolorbox}
\usepackage{enumitem}
\usepackage[colorlinks=true,linkcolor=black,urlcolor=primaryblue,citecolor=primaryblue]{hyperref}

\definecolor{primaryblue}{RGB}{0,102,204}
\definecolor{secondaryblue}{RGB}{51,153,255}
\definecolor{lightgray}{RGB}{245,245,245}
\definecolor{darkgray}{RGB}{100,100,100}

\titleformat{\section}
  {\Large\bfseries\color{primaryblue}}
  {\thesection}{1em}{}

\setlength{\parskip}{4pt}
\setlength{\parindent}{0pt}

\setlist[itemize]{leftmargin=*,itemsep=3pt}
\setlist[enumerate]{leftmargin=*,itemsep=3pt}

\graphicspath{{./}{../assets/images/}{./images/}}

\begin{document}

\begin{center}  \IfFileExists{../../../assets/Logo.jpg}{%
    \includegraphics[width=6cm,height=3cm,keepaspectratio]{../../../assets/Logo.jpg} \\[0.8cm]
  }{%
    \fbox{\parbox[c][2.5cm][c]{6cm}{\centering Logo non trovato\\(Logo.jpg)}}\\[0.5cm]
  }
  
  {\Large\bfseries\color{primaryblue} BugBusters}\\[0.3cm]
  {\small\color{darkgray} Email: \texttt{bugbusters.unipd@gmail.com}} \\[0.1cm]
  {\small\color{darkgray} Gruppo: 4} \\[0.5cm]

  {\large\bfseries Università degli Studi di Padova}\\[0.3cm]
  {\small Laurea in Informatica}\\[0.2cm]
  {\small Corso: Ingegneria del Software}\\[0.2cm]
  {\small Anno Accademico: 2025/2026}\\[0.8cm]

  {\Huge\bfseries\color{primaryblue} Verbale Esterno}\\[0.3cm]
  {\Large\color{secondaryblue} 22 ottobre 2025}\\[0.8cm]
\end{center}

\begin{center}
\begin{tcolorbox}[colback=lightgray,colframe=primaryblue,width=0.85\textwidth,arc=3mm,boxrule=0.5pt]
\begin{tabular}{@{}ll@{}}
\textbf{Redattori}    & Leonardo Salviato \\
\textbf{Verificatore}    & Alberto Pignat \\
\textbf{Uso}          & Interno \\
\textbf{Destinatari}  & BugBusters \\
\end{tabular}
\end{tcolorbox}
\end{center}

\vspace{0.5cm}

\begin{center}
\begin{tcolorbox}[colback=secondaryblue!10,colframe=secondaryblue,width=0.9\textwidth,arc=3mm,boxrule=0.8pt,title={\bfseries Abstract}]
Verbale della prima riunione avvenuta con l'azienda Miriade in via telematica per chiedere delucidazioni riguardanti il progetto "L'app che Protegge e Trasforma" e sui sistemi di collaborazione tra il gruppo e l'azienda.
\end{tcolorbox}
\end{center}

\newpage

\tableofcontents
\newpage


\section{Informazioni generali}

\begin{itemize}
    \item \textbf{Tipo riunione:} Esterna
    \item \textbf{Piattaforma:} Google Meet
    \item \textbf{Data:} 22/10/2025
    \item \textbf{Orario di inizio:} 16:00
    \item \textbf{Orario di fine:} 16:30
    \item \textbf{Presenti:}
    \begin{itemize}[leftmargin=1.5em, itemsep=3pt, label={\rule[0.5ex]{0.4em}{0.4em}}]
        \item Alberto Autiero
        \item Alberto Pignat
        \item Marco Piro
        \item Marco Favero
        \item Linor Sadè
        \item Luca Slongo
        \item Leonardo Salviato
    \end{itemize}
    \item \textbf{Assenti:}
    \item \textbf{Presenti Esterni:}
    \begin{itemize}[leftmargin=1.5em, itemsep=3pt, label={\rule[0.5ex]{0.4em}{0.4em}}]
        \item Emanuele Righetto
        \item Arianna Bellino
    \end{itemize}

\end{itemize}

\section{Ordine del giorno}

\begin{enumerate}
    \item Chiarimenti generali sul capitolato
    \item Chiarimenti sui requisiti minimi per una soddisfacente realizzazione del progetto
    \item Domande tecniche sulle tecnologie che sará necessario utilizzare
    \item Chiarimenti su come verrá disposta la formazione tecnica
    \item Chiarimenti sul supporto a livello contenutistico che avremo durante lo sviluppo
\end{enumerate}

\section{Svolgimento}
    La riunione è iniziata come previsto alle ore 16:00. I componenti del gruppo hanno espresso le loro perplessitá riguardanti il progetto. I punti trattati sono i seguenti
    \begin{itemize}
    \item \textbf{Chiarimenti generali sul capitolato e sull'organizzazione}\\
    \noindent
    \textit{Risposta:} \\
    \begin{itemize}
        \item L'azienda per progetti si organizza in sprint bisettimanali
        \item È disponibile supporto anche al di fuori dei metting sia richiesto tramite mail o tramite riunioni in sede a Padova
        \item L'applicazione per ogni utente sará anonima con tracciamento delle azioni interne all'app e localizzazione (per inviare supporto se necessario)
    \end{itemize}
    
    \vspace{1em}

    \item \textbf{Chiarimenti sui requisiti minimi per una soddisfacente realizzazione del progetto}\\
    \noindent
    Essendo un progetto alquanto completo quello presentato nel capitolato il team voleva sapere se era stata fatta una stima delle tempistiche per l'implementazione delle varie funzionalitá e se erano stati predisposti dei requisiti minimi ottenibili nel tempo che ci è stato assegnato.\\ \\
    \textit{Risposta:} 
    
    I requisiti minimi non sono stati ancora determinati, verranno indicati all'inizio del progetto in caso di conferma dell'assegnazione. Ci è stato detto comunque che l'attenzione sará rivolta alle funzionalitá principali per poi eventualmente allargarsi verso quelle secondarie. Data la mancanza di requisiti minimi non possono ancora essere determinate le tempistiche per il nostro team.
    \vspace{1em}

    \item \textbf{Domande tecniche sulle tecnologie che sarà necessario utilizzare}\\
    \noindent
    Approfondimento delle tecnologie sulla qual è importante specializzarsi per la riuscita del progetto.\\ \\
    \textit{Risposta:} 
    Le tecnologie necessarie sono molteplici, in particolare l'attenzione è stata rivolta a Flatter e ad AWS.
    \vspace{1em}

    \item \textbf{Chiarimenti su come verrà disposta la formazione tecnica}\\
    \noindent
    Dato l'utilizzo necessario di tecnologie nuove per i membri del team si voleva ottenere un chiarimento sul tipo di supporto che avremmo avuto a livello tecnico. \\ \\
    \textit{Risposta:} 
    La formazione a livello tecnico verrá svolta tramite corsi da loro indicati e durante il progetto avremo il loro supporto tecnico in caso di problemi che non superabili autonomamente.
    \vspace{1em}

    \item \textbf{Chiarimenti sul supporto a livello contenutistico che avremo durante lo sviluppo}\\
    \noindent
    Essendo che l'applicazione tratta tematiche di supporto per persone a rischio di violenza di genere (argomento alquanto sensibile) il team voleva sapere che tipo di sostegno avremmo avuto non a livello tecnico ma piú contenutistico per l'applicazione. \\ \\
    \textit{Risposta:} \\
    I rappresentanti ci hanno spiegato come non solo un membro del loro team potrá darci supporto in questo lato ma anche un'associazione che ha come fulcro questo tipo di tematiche.
    \vspace{1em}
\end{itemize}

La riunione si è conclusa alle 16:30.


\section{Esito riunione}
    L'esito della riunione è stato complessivamente positivo, le domande riguardanti il capitolato sono state risposte in maniera esaustiva. Si ringraziono l'azienda Miriade e i rappresentanti Emanuele Righetto e Arianna Bellino per la disponibilitá dimostrata.
    

\section{Approvazione esterna}
Il verbale è stato inviato via mail all'azienda Miriade per chiedere la loro approvazione sui contenuti sopra riportati. La prova della loro accettazione è la seguente firma:


\vfill
\begin{center}
    {\small\color{darkgray} Documento redatto e approvato dal gruppo BugBusters.}
\end{center}


\end{document}
