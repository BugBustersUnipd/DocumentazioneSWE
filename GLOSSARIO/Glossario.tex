\documentclass[a4paper,11pt]{article}

\usepackage[utf8]{inputenc}
\usepackage[T1]{fontenc}
\usepackage[italian]{babel}
\usepackage[margin=2.5cm]{geometry}
\usepackage{graphicx}
\usepackage{grffile}
\usepackage{booktabs}
\usepackage{setspace}
\usepackage{titlesec}
\usepackage{float}
\usepackage{ifthen}
\usepackage{tcolorbox}
\usepackage{enumitem}
\usepackage[titles]{tocloft}
\usepackage[colorlinks=true,linkcolor=black,urlcolor=primaryblue,citecolor=primaryblue]{hyperref}
\definecolor{primaryblue}{RGB}{0,102,204}
\definecolor{secondaryblue}{RGB}{51,153,255}
\definecolor{lightgray}{RGB}{245,245,245}
\definecolor{darkgray}{RGB}{100,100,100}

\usepackage[acronym,toc]{glossaries}
\makeglossaries

\usepackage{fancyhdr}
\usepackage{lastpage}

\setcounter{secnumdepth}{0}
\renewcommand{\thesection}{}
\renewcommand{\cftsecleader}{\cftdotfill{\cftsecdotsep}}
\setlength{\cftbeforesecskip}{2pt}

\setlength{\parskip}{4pt}
\setlength{\parindent}{0pt}

\setlist[itemize]{leftmargin=*,itemsep=3pt}
\setlist[enumerate]{leftmargin=*,itemsep=3pt}

\begin{document}

\pagestyle{fancy}
\fancyhf{}
\fancyhead[L]{Gruppo 4 - BugBusters}
\fancyhead[R]{Glossario}
\fancyfoot[L]{\thepage\ di \pageref{LastPage}}
\fancyfoot[R]{\nouppercase{\rightmark}}
\renewcommand{\headrulewidth}{0pt}
\renewcommand{\footrulewidth}{0pt}

\begin{center}
  \thispagestyle{empty}
  \IfFileExists{../assets/Logo.jpg}{%
    \includegraphics[width=6cm,height=3cm,keepaspectratio]{../assets/Logo.jpg} \\[0.8cm]
  }{%
    \fbox{\parbox[c][2.5cm][c]{6cm}{\centering Logo non trovato\\(Logo.jpg)}}\\[0.5cm]
  }
  {\Large\bfseries BugBusters}\\[0.3cm]
  {\small\color{darkgray} Email: \texttt{bugbusters.unipd@gmail.com}} \\[0.1cm]
  {\small\color{darkgray} Gruppo: 4} \\[0.5cm]

  {\large\bfseries Università degli Studi di Padova}\\[0.3cm]
  {\small Laurea in Informatica}\\[0.2cm]
  {\small Corso: Ingegneria del Software}\\[0.2cm]
  {\small Anno Accademico: 2025/2026}\\[0.8cm]

  {\Huge\bfseries Glossario}\\[0.8cm]
\end{center}

\begin{center}
\begin{tcolorbox}[colback=lightgray,width=0.85\textwidth,arc=3mm,boxrule=0.5pt]
\begin{tabular}{@{}ll@{}}
\textbf{Redattori}    & Alberto Autiero \\
\textbf{Verificatori} & [Nome Cognome] \\
\textbf{Uso}          & Interno \\
\textbf{Destinatari}  & Prof. Tullio Vardanega, Prof. Riccardo Cardin \\
\end{tabular}
\end{tcolorbox}
\end{center}

\newpage

\tableofcontents

\newpage

\section{Versioni del documento}

{\footnotesize
\begin{tabular}{|p{1.5cm}|p{1.8cm}|p{4cm}|p{2cm}|p{2cm}|p{2cm}|}
\hline
\textbf{Versione} & \textbf{Data} & \textbf{Descrizione} & \textbf{Redatto} & \textbf{Verificato} & \textbf{Approvato} \\
\hline
0.1.0 & 22/10/2025 & Prima stesura del documento & Alberto Autiero & - & - \\
\hline
\end{tabular}
}

\newpage

\section{Introduzione}
Questo documento nasce con lo scopo di evitare qualsiasi tipo di ambiguità o dubbi riguardanti la terminologia adoperata all'interno dei documenti del progetto. Per questo motivo, vengono di seguito esposte le definizioni dei termini specifici, degli acronimi e delle parole ambigue utilizzati nella documentazione, adottando una struttura alfabetica per facilitare la navigazione del documento. Il glossario ha lo scopo di garantire una comprensione univoca dei termini da parte di tutti i membri del team e dei destinatari della documentazione.

\newpage

\section{A}

\subsection{ACM (Access Control Mechanism)}
Meccanismo di controllo degli accessi che determina quali utenti o sistemi possono accedere a specifiche risorse e con quali privilegi.

\subsection{Agile}
Metodologia di sviluppo del software iterativa e incrementale che si basa su principi come la collaborazione continua con il cliente, la consegna frequente di software funzionante e la capacità di rispondere ai cambiamenti dei requisiti.

\subsection{AI Assistant Generativo}
Sistema di intelligenza artificiale in grado di creare autonomamente contenuti testuali, immagini o altri media a partire da prompt o dati di input, utilizzato per generare comunicati, testi HR e contenuti aziendali.

\subsection{AI Co-Pilot}
Modulo di supporto automatizzato per gli studi dei Consulenti del Lavoro (CdL), che assiste nei processi di gestione, riconoscimento e distribuzione documentale attraverso tecniche di intelligenza artificiale.

\subsection{Amministratore}
Figura responsabile della gestione dell'infrastruttura, degli strumenti di sviluppo e dei processi del progetto. Si occupa di configurare e mantenere gli ambienti di lavoro e garantire che il team abbia a disposizione gli strumenti necessari.

\subsection{Analisi dei Requisiti}
Processo sistematico volto a identificare, documentare e validare i bisogni, i vincoli e le esigenze del progetto, trasformandoli in requisiti formali che guideranno lo sviluppo del software.

\subsection{Analista}
Figura professionale specializzata nell'analisi dei requisiti e nella specifica delle funzionalità del sistema. Collabora con gli stakeholder per comprendere le esigenze e tradurle in specifiche tecniche.

\subsection{Angular}
Framework JavaScript open source per lo sviluppo di applicazioni web frontend, mantenuto da Google. Offre un'architettura basata su componenti e strumenti per lo sviluppo di single-page applications.

\subsection{Ansible}
Strumento di automazione IT per il provisioning, la gestione della configurazione e il deployment di applicazioni. Utilizza un linguaggio dichiarativo basato su YAML.

\subsection{API}
Application Programming Interface. Insieme di definizioni, protocolli e strumenti per la costruzione e l'integrazione di software applicativo. Consente a diversi sistemi di comunicare tra loro.

\subsection{Approvazione}
Processo formale attraverso il quale un documento o una funzionalità viene accettata e considerata completa dopo aver superato le verifiche e validazioni necessarie.

\subsection{Architettura}
Struttura fondamentale di un sistema software, comprendente i suoi componenti, le relazioni tra di essi e i principi che ne guidano la progettazione e l'evoluzione.

\subsection{Audit Trail}
Registro cronologico delle operazioni svolte nel sistema, utile per verifiche, tracciabilità e compliance, che documenta chi ha fatto cosa e quando.

\subsection{AUM (Authentication Mechanism)}
Meccanismo di autenticazione che verifica l'identità di un utente o di un sistema prima di concedere l'accesso a risorse protette.

\subsection{AWS}
Amazon Web Services. Piattaforma di cloud computing che offre servizi di calcolo, storage, database e altre funzionalità per supportare lo sviluppo e il deployment di applicazioni.

\subsection{AWS CDK V2}
AWS Cloud Development Kit versione 2. Framework di sviluppo che permette di definire l'infrastruttura cloud utilizzando linguaggi di programmazione familiari invece di template.

\newpage
\section{B}

\subsection{Backend}
Parte di un'applicazione software che gestisce la logica di business, l'elaborazione dei dati e la comunicazione con il database. Opera sul server ed è inaccessibile direttamente all'utente finale.

\subsection{Baseline}
Versione approvata e formalmente controllata di un documento o di un componente software che serve come riferimento per sviluppi successivi. Le modifiche successive richiedono procedure formali di controllo.

\subsection{BERT (Bidirectional Encoder Representations from Transformers)}
Modello di linguaggio bidirezionale pre-addestrato da Google, utilizzato per comprendere il contesto delle parole in una frase e per varie applicazioni di NLP.

\subsection{Best Practice}
Insieme di tecniche, metodi e procedure che sono state riconosciute come le più efficaci ed efficienti per raggiungere un obiettivo specifico in un determinato contesto.

\subsection{BLE (Bluetooth Low Energy)}
Protocollo di comunicazione wireless progettato per consumare poca energia, ideale per dispositivi IoT e wearable che richiedono lunga autonomia.

\subsection{Branch}
Ramo di sviluppo in un sistema di controllo versione come Git. Permette di lavorare su funzionalità isolate senza influenzare il codice principale fino al completamento e al merge.

\subsection{Broker MQTT}
Componente centrale nel protocollo MQTT che riceve i messaggi dai publisher e li inoltra ai subscriber interessati, gestendo le sottoscrizioni e la distribuzione dei messaggi.

\newpage
\section{C}

\subsection{Capitolato}
Documento contrattuale che specifica i requisiti, le caratteristiche tecniche e le condizioni di un progetto software proposto da un'azienda proponente per il corso di Ingegneria del Software.

\subsection{Caso d'uso}
Tecnica di specifica dei requisiti che descrive le interazioni tra gli attori (utenti o sistemi esterni) e il sistema software per raggiungere un obiettivo specifico.

\subsection{Cedolini Massivi}
File contenenti più documenti retributivi aggregati, da suddividere e assegnare ai singoli destinatari tramite riconoscimento automatico.

\subsection{CI/CD (Continuous Integration/Continuous Deployment)}
Insieme di pratiche e strumenti che automatizzano l'integrazione, il testing e il deployment del codice, permettendo rilasci frequenti e affidabili.

\subsection{CNN (Convolutional Neural Network)}
Tipo di rete neurale artificiale specializzata nell'elaborazione di dati con struttura a griglia, come le immagini, utilizzando operazioni di convoluzione.

\subsection{Confidence Score}
Misure numerica (0-1 o 0-100\%) che indica la probabilità che un riconoscimento automatico sia corretto, utilizzata per valutare l'affidabilità dei risultati di sistemi AI.

\subsection{Consulente del Lavoro (CdL)}
Professionista specializzato nella gestione delle risorse umane, della payroll, della compliance normativa e delle relazioni sindacali per aziende e dipendenti.

\subsection{Cross-Site Scripting (XSS)}
Vulnerabilità di sicurezza web che permette a un attaccante di iniettare script malevoli in pagine web visualizzate da altri utenti.

\subsection{Cruscotto/Dashboard}
Interfaccia utente che presenta in forma grafica e sintetica le metriche, gli indicatori di performance e lo stato corrente del progetto o dell'applicazione.

\newpage
\section{D}

\subsection{Decisione interna}
Scelta presa dal team di progetto riguardante aspetti tecnici, organizzativi o metodologici, documentata per garantire tracciabilità e coerenza nelle attività successive.

\subsection{Decision tree}
Algoritmo di machine learning che utilizza una struttura ad albero per modellare decisioni e i loro possibili esiti, spesso utilizzato per classificazione e regressione.

\subsection{DIP (Distribution Information Package)}
Formato standardizzato per la distribuzione di pacchetti di informazioni nella conservazione digitale, contenente metadati e dati oggetto di conservazione.

\subsection{Discord}
Piattaforma di comunicazione utilizzata dal team per la coordinazione quotidiana, le riunioni e la condivisione di informazioni in tempo reale.

\subsection{Dispatch}
Processo di distribuzione automatizzata dei documenti verso i destinatari finali attraverso diversi canali (es. app, portale, email, PEC).

\subsection{Distant Writing}
Tecnica di scrittura in cui l'utente specifica cosa vuole scrivere ma lascia la scrittura effettiva all'intelligenza artificiale, utilizzata nel "vibe coding" e nella generazione di contenuti.

\subsection{Docker}
Piattaforma per lo sviluppo, spedizione ed esecuzione di applicazioni all'interno di container software, che permettono di isolare l'applicazione e le sue dipendenze.

\newpage
\section{E}

\subsection{EasyOCR}
Modulo Python per estrarre testo da immagini, supporta oltre 80 lingue e vari script di scrittura, utilizzato per l'OCR in applicazioni di document processing.

\subsection{Economicità}
Principio secondo il quale le risorse del progetto (tempo, budget, personale) devono essere utilizzate in modo ottimale per massimizzare il valore prodotto.

\subsection{Efficacia}
Capacità di raggiungere gli obiettivi prefissati e produrre i risultati attesi, indipendentemente dalle risorse impiegate.

\subsection{Efficienza}
Rapporto tra i risultati ottenuti e le risorse impiegate per conseguirli. Un processo è efficiente quando raggiunge i suoi obiettivi utilizzando il minimo di risorse necessarie.

\subsection{EN18031}
Serie di standard europei che definiscono requisiti di sicurezza informatica per apparecchiature radio connesse (radio equipment) ai fini della conformità alla Radio Equipment Directive (RED). Copre temi come controllo accessi, autenticazione, aggiornamenti sicuri, storage protetto e protezione della privacy.

\subsection{Entity Resolution}
Processo di identificazione e associazione di entità (es. persone o aziende) a partire da dati parziali o duplicati, tramite algoritmi di matching e disambiguazione.

\subsection{ERP (Enterprise Resource Planning)}
Sistema software integrato che gestisce i processi aziendali fondamentali, come produzione, acquisti, vendite, contabilità e risorse umane.

\newpage
\section{F}

\subsection{FAISS (Facebook AI Similarity Search)}
Libreria sviluppata da Facebook per la ricerca efficiente di similarità e clustering di vettori di grandi dimensioni, comunemente usata in applicazioni di machine learning.

\subsection{Flutter}
Framework di Google per lo sviluppo di applicazioni multipiattaforma (iOS, Android, Web) utilizzando un unico codice base scritto in Dart.

\subsection{Funzionalità}
Caratteristica o capacità specifica che un sistema software deve possedere per soddisfare i bisogni degli utenti e gli obiettivi del progetto.

\newpage
\section{G}

\subsection{GATT (Generic Attribute Profile)}
Protocollo Bluetooth Low Energy che definisce come i dispositivi scambiano dati utilizzando servizi e caratteristiche, standard per la comunicazione tra dispositivi BLE.

\subsection{GDPR (General Data Protection Regulation)}
Regolamento generale sulla protezione dei dati dell'Unione Europea che stabilisce norme per la protezione e la libera circolazione dei dati personali.

\subsection{GitHub}
Piattaforma di hosting per repository Git che offre strumenti per il version control, la collaborazione e la gestione del ciclo di vita del software.

\subsection{GPT (Generative Pre-trained Transformer)}
Serie di modelli di linguaggio sviluppati da OpenAI, progettati per comprendere il contesto, generare contenuti e ragionare attraverso testo, immagini e altro.

\subsection{GraphQL}
Linguaggio di query e runtime per API che permette ai client di richiedere esattamente i dati di cui hanno bisogno, riducendo il over-fetching e under-fetching.

\newpage
\section{H}

\subsection{Human-in-the-Loop}
Approccio in cui l'intelligenza artificiale e gli esseri umani collaborano, con l'uomo che supervisiona, corregge o fornisce feedback al sistema AI, particolarmente utile quando la confidenza del sistema è bassa.

\newpage
\section{I}

\subsection{IaC (Infrastructure as Code)}
Approccio alla gestione dell'infrastruttura che utilizza file di configurazione e script per automatizzare il provisioning e la gestione delle risorse, trattando l'infrastruttura come software.

\subsection{IDE}
Integrated Development Environment. Ambiente di sviluppo integrato che fornisce strumenti completi per la scrittura, il testing e il debugging del codice in un'unica interfaccia.

\subsection{IoT (Internet of Things)}
Rete di dispositivi fisici connessi a Internet, in grado di raccogliere e scambiare dati, come sensori, elettrodomestici smart, veicoli e sistemi di automazione.

\subsection{Issue}
Segnalazione di un problema, un bug o una richiesta di miglioramento nel sistema di tracking del progetto. Ogni issue viene tracciata, assegnata e gestita fino alla risoluzione.

\newpage
\section{K}

\subsection{KNX}
Standard internazionale per l'automazione degli edifici e la domotica, che permette il controllo integrato di illuminazione, riscaldamento, sicurezza e altri sistemi.

\subsection{KNX IoT}
Protocollo di comunicazione standardizzato per l'automazione degli edifici e la domotica, estensione di KNX per dispositivi IoT con interfacce REST per il controllo degli impianti.

\subsection{KPI (Key Performance Indicator)}
Indicatore chiave di performance che misura l'efficacia di un processo, un'attività o un'organizzazione nel raggiungere i propri obiettivi.

\newpage
\section{L}

\subsection{Latex}
Sistema di composizione tipografica utilizzato per la produzione di documentazione tecnica e scientifica di alta qualità, particolarmente adatto per documenti complessi con formule matematicae.

\subsection{LLM (Large Language Model)}
Modello di linguaggio di grandi dimensioni addestrato su vasti corpus di testo, in grado di generare, comprendere e elaborare linguaggio naturale in modo sofisticato.

\newpage
\section{M}

\subsection{Machine Learning (ML)}
Campo dell'intelligenza artificiale che si occupa di sviluppare algoritmi che permettono ai computer di apprendere pattern dai dati e migliorare le proprie performance con l'esperienza.

\subsection{MarkDown}
Linguaggio di markup leggero per la formattazione di testo, utilizzato per documentazione, readme e messaggi, che può essere convertito in HTML e altri formati.

\subsection{Milestone}
Punto significativo nel ciclo di vita del progetto che segna il completamento di un insieme di attività o il raggiungimento di un obiettivo importante. Serve come punto di verifica del progresso.

\subsection{MongoDB}
Database NoSQL document-oriented che memorizza i dati in formato JSON-like, offrendo flessibilità nello schema e scalabilità orizzontale.

\subsection{MQTT (Message Queuing Telemetry Transport)}
Protocollo di messaggistica leggero e efficiente progettato per dispositivi con risorse limitate e reti a larghezza di banda ridotta, ampiamente usato in IoT.

\subsection{Multi-tenant}
Architettura software in cui una singola istanza dell'applicazione serve multiple organizzazioni (tenant), mantenendo l'isolamento dei dati e della configurazione.

\subsection{MVP (Minimum Viable Product)}
Versione minima di un prodotto che include solo le funzionalità essenziali necessarie per soddisfare gli utenti iniziali e raccogliere feedback per sviluppi futuri.

\newpage
\section{N}

\subsection{Nest.js}
Framework per la costruzione di applicazioni server-side efficienti e scalabili in Node.js, utilizzando TypeScript e ispirato ad Angular.

\subsection{NLP (Natural Language Processing)}
Campo dell'intelligenza artificiale che si occupa dell'interazione tra computer e linguaggio umano, includendo compiti come traduzione, analisi del sentiment e riconoscimento di entità.

\subsection{Norme di progetto}
Insieme di regole, procedure e convenzioni stabilite dal team per garantire coerenza, qualità e uniformità nelle attività di sviluppo e nella documentazione prodotta.

\newpage
\section{O}

\subsection{OAuth2}
Protocollo standard per l'autorizzazione che permette a terze parti di accedere a risorse protette senza condividere le credenziali dell'utente.

\subsection{Obsidian}
Software per la gestione di note personali basato su file MarkDown, con funzionalità di linking tra note e grafo della conoscenza, utilizzato per il personal knowledge management.

\subsection{OCR (Optical Character Recognition)}
Tecnologia che converte immagini di testo scritto o stampato in testo digitale machine-readable.

\subsection{OWASP (Open Web Application Security Project)}
Comunità globale che si occupa di sicurezza delle applicazioni web, nota per la pubblicazione della OWASP Top 10 sulle vulnerabilità web più critiche.

\newpage
\section{P}

\subsection{Pair Programming}
Pratica di sviluppo in cui due programmatori lavorano insieme su un'unica workstation, condividendo la scrittura del codice e la risoluzione dei problemi in tempo reale.

\subsection{PostgreSQL}
Sistema di gestione di database relazionale open source estensibile e conforme agli standard SQL, noto per la sua robustezza e funzionalità avanzate.

\subsection{Progettista}
Figura responsabile della progettazione dell'architettura software e delle soluzioni tecniche, garantendo che soddisfino i requisiti e siano realizzabili efficientemente.

\subsection{Programmatore}
Figura che implementa il codice sorgente secondo le specifiche tecniche, seguendo le best practice e gli standard di qualità definiti nel progetto.

\subsection{Prometheus}
Sistema di monitoraggio e alerting open source progettato per l'affidabilità e la scalabilità, ampiamente utilizzato per il monitoring di applicazioni e infrastrutture.

\subsection{Prompt}
Input (testo e/o altri dati) fornito a un modello di intelligenza artificiale per ottenere una risposta o un output specifico.

\subsection{Proponente}
Azienda o organizzazione che propone un capitolato d'appalto per il progetto del corso di Ingegneria del Software, definendone requisiti e obiettivi.

\subsection{Pull Request}
Richiesta di integrazione di modifiche da un branch di sviluppo al branch principale nel sistema di version control, soggetta a revisione e approvazione.

\newpage
\section{R}

\subsection{RBAC (Role-Based Access Control)}
Modello di controllo degli accessi in cui i permessi sono assegnati a ruoli specifici, e gli utenti ottengono i permessi attraverso l'assegnazione a questi ruoli.

\subsection{React}
Libreria JavaScript open source per la costruzione di interfacce utente, mantenuta da Facebook. Basata su componenti riutilizzabili e un virtual DOM per performance ottimizzate.

\subsection{RED (Radio Equipment Directive)}
Direttiva europea che stabilisce i requisiti per la messa sul mercato di apparecchiature radio, inclusi i dispositivi che utilizzano comunicazioni wireless.

\subsection{Redattore}
Membro del team responsabile della stesura e della produzione dei documenti di progetto, garantendo chiarezza, completezza e conformità alle norme stabilite.

\subsection{Repository}
Archivio centrale in cui vengono memorizzati e versionati i file sorgente, la documentazione e le risorse del progetto utilizzando un sistema di controllo versione.

\subsection{Responsabile}
Figura di riferimento del progetto con compiti di coordinamento, pianificazione, gestione delle risorse e comunicazione con docenti e proponenti.

\subsection{Requisiti funzionali}
Specificano cosa il sistema deve fare, descrivendo le funzionalità, i comportamenti e le interazioni che il software deve supportare.

\subsection{Requisiti non funzionali}
Definiscono come il sistema deve comportarsi in termini di prestazioni, sicurezza, affidabilità, usabilità e altri attributi di qualità, senza riguardo alle funzionalità specifiche.

\subsection{Retention Policy}
Politica che stabilisce per quanto tempo i dati o documenti devono essere conservati prima di essere eliminati, in base a requisiti legali o aziendali.

\subsection{RoBERTa (Robustly optimized BERT approach)}
Variante ottimizzata del modello BERT per la comprensione del linguaggio naturale, addestrata con strategie più efficaci per migliorare le performance.

\newpage
\section{S}

\subsection{Sandbox di Sviluppo}
Ambiente isolato per testare e validare funzionalità senza influenzare i sistemi di produzione, utilizzato per sviluppare e verificare nuove feature in sicurezza.

\subsection{SCRUM}
Framework agile per la gestione dello sviluppo software che enfatizza lo sviluppo iterativo, l'adattamento ai cambiamenti e la consegna incrementale di valore.

\subsection{Sei Cappelli per Pensare}
Tecnica di pensiero laterale sviluppata da Edward De Bono che utilizza sei cappelli metaforici per rappresentare diverse prospettive di pensiero (fatti, emozioni, critica, ottimismo, creatività, controllo).

\subsection{Serverless}
Architettura cloud computing in cui il provider cloud gestisce dinamicamente l'allocazione delle risorse di macchina, permettendo agli sviluppatori di focalizzarsi sul codice senza gestire l'infrastruttura.

\subsection{SQL Injection}
Vulnerabilità di sicurezza che permette a un attaccante di interferire con le query che un'applicazione effettua verso il database, potenzialmente accedendo a dati sensibili.

\subsection{SSL (Secure Sockets Layer)}
Vecchio protocollo per connessioni cifrate tra client e server; è stato sostituito da TLS e non è più ritenuto sicuro.

\subsection{Specifica Tecnica}
Documento che descrive in dettaglio l'architettura, il design e le scelte implementative del sistema software, guidando le attività di sviluppo.

\subsection{Sprint}
Periodo di tempo fisso (tipicamente 2-4 settimane) in Scrum durante il quale il team sviluppa e consegna un incremento di prodotto potenzialmente rilasciabile.

\newpage
\section{T}

\subsection{Terraform}
Strumento di Infrastructure as Code (IaC) che permette di definire e provisionare infrastruttura cloud usando un linguaggio dichiarativo (HCL), supportando multiple piattaforme cloud.

\subsection{Tesseract OCR}
Motore OCR open source sviluppato originariamente da HP e successivamente mantenuto da Google, noto per la sua accuratezza nel riconoscimento del testo.

\subsection{Test}
Processo sistematico di verifica che il software soddisfi i requisiti specificati e identifichi difetti, attraverso l'esecuzione controllata di casi di test.

\subsection{TLS (Transport Layer Security)}
Protocollo di sicurezza che fornisce comunicazioni private e integrità dei dati tra due applicazioni che comunicano attraverso una rete, successore di SSL.

\subsection{Transformer}
Architettura di rete neurale basata su meccanismi di attenzione, rivoluzionaria per le attività di elaborazione del linguaggio naturale e altre sequenze.

\subsection{TypeScript}
Superset di JavaScript che aggiunge tipizzazione statica opzionale e altre funzionalità; il codice TypeScript viene compilato in JavaScript eseguibile.

\newpage
\section{U}

\subsection{UWB (Ultra Wide Band)}
Tecnologia di comunicazione wireless a corto raggio che utilizza una larghezza di banda molto ampia per comunicazioni ad alta velocità e precisione nel posizionamento.

\newpage
\section{V}

\subsection{Validazione}
Processo che accerta che il prodotto software soddisfi le esigenze e le aspettative dell'utente finale e sia adatto allo scopo per cui è stato creato.

\subsection{Verifica}
Processo che determina se i prodotti di un'attività di sviluppo soddisfano i requisiti e le condizioni imposte all'inizio di tale attività.

\subsection{Verificatore}
Membro del team responsabile di controllare che documenti, codice e altri prodotti di lavoro rispettino gli standard di qualità e siano privi di errori.

\subsection{ViT (Vision Transformer)}
Architettura di rete neurale che applica il modello Transformer alle immagini, dividendo le immagini in patch e processandole come sequenze.

\newpage
\section{W}

\subsection{Way of Working}
Insieme di processi, metodologie, strumenti e pratiche adottati dal team per organizzare e svolgere le attività di progetto in modo coordinato ed efficiente.

\subsection{Whisper}
Sistema di riconoscimento vocale automatico sviluppato da OpenAI, in grado di trascrivere, tradurre e identificare la lingua parlata in audio.

\newpage
\section{Z}

\subsection{Zigbee}
Protocollo di comunicazione wireless a basso consumo energetico basato sullo standard IEEE 802.15.4, utilizzato in reti mesh per l'automazione domestica e industriale.

\end{document}