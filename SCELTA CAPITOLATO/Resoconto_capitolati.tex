\documentclass[a4paper,11pt]{article}
     
\newcommand{\CurrentVersion}{0.0.10} % ultima versione, da cambiare ad ogni push significativo

\usepackage[utf8]{inputenc}
\usepackage[T1]{fontenc}
\usepackage[italian]{babel}
\usepackage[margin=2.5cm]{geometry}
\usepackage{graphicx}
\usepackage{grffile}
\usepackage{booktabs}
\usepackage{setspace}
\usepackage{titlesec}
\usepackage{float}
\usepackage{ifthen}
\usepackage[table]{xcolor}
\usepackage{tabularx}
\usepackage{tcolorbox}
\usepackage{enumitem}
\usepackage[titles]{tocloft}
\usepackage[colorlinks=true,linkcolor=black,urlcolor=primaryblue,citecolor=primaryblue]{hyperref}

\definecolor{primaryblue}{RGB}{0,102,204}
\definecolor{secondaryblue}{RGB}{51,153,255}
\definecolor{lightgray}{RGB}{245,245,245}
\definecolor{darkgray}{RGB}{100,100,100}

\titleformat{\section}
  {\Large\bfseries\color{primaryblue}}
  {\thesection}{1em}{}

\titleformat{\subsection}
  {\large\bfseries\color{primaryblue}} % Sottosezione: colore secondaryblue
  {\thesubsection}{1em}{}

\titleformat{\subsubsection}
  {\normalsize\bfseries\color{secondaryblue}} % Sotto-sottosezione: colore secondaryblue
  {\thesubsubsection}{1em}{}

% per il footer con il numero di pagina
\usepackage{fancyhdr}
\usepackage{lastpage} % per ottenere il numero dell'ultima pagina da mettere nel footer


\usepackage{ltablex} %per far andare a capo le tabelle
\keepXColumns

\renewcommand{\sectionmark}[1]{\markright{#1}}




\setlength{\parskip}{4pt}
\setlength{\parindent}{0pt}

\setlist[itemize]{leftmargin=*,itemsep=3pt}
\setlist[enumerate]{leftmargin=*,itemsep=3pt}

\graphicspath{{./}{../assets/images/}{./images/}} 

\begin{document}

%configurazione per il footer
\pagestyle{fancy}
\fancyhf{} % pulisce tutti i campi del header e footer

% Header: sinistra e destra
\fancyhead[L]{Gruppo 4 - BugBusters} % sinistra
\fancyhead[R]{Resoconto capitolati}      % destra

\fancyfoot[L]{ \thepage\ di \pageref{LastPage}} %definisce il formato del footer
\fancyfoot[R]{ \nouppercase{\rightmark}} % nome della sezione


\renewcommand{\headrulewidth}{0pt}  % rimuove la linea dell'header
\renewcommand{\footrulewidth}{0pt}  % se vuoi anche togliere eventuale linea del footer


\begin{center}
  \thispagestyle{empty}
  \IfFileExists{../assets/Logo.jpg}{%
    \includegraphics[width=6cm,height=3cm,keepaspectratio]{../assets/Logo.jpg} \\[0.8cm]
  }{%
    \fbox{\parbox[c][2.5cm][c]{6cm}{\centering Logo non trovato\\(Logo.jpg)}}\\[0.5cm]
  }
  {\Large\bfseries BugBusters}\\[0.3cm]
  {\small\color{darkgray} Email: \texttt{bugbusters.unipd@gmail.com}} \\[0.1cm]
  {\small\color{darkgray} Gruppo: 4} \\[0.5cm]

  {\large\bfseries Università degli Studi di Padova}\\[0.3cm]
  {\small Laurea in Informatica}\\[0.2cm]
  {\small Corso: Ingegneria del Software}\\[0.2cm]
  {\small Anno Accademico: 2025/2026}\\[0.8cm]

  {\Huge\bfseries\color{primaryblue} Resoconto Capitolati}\\[0.8cm]
  {\Large\color{secondaryblue}Versione \CurrentVersion}\\[0.8cm]
\end{center}

\begin{center}
\begin{tcolorbox}[colback=lightgray,colframe=primaryblue,width=0.85\textwidth,arc=3mm,boxrule=0.5pt]
\begin{tabular}{@{}ll@{}}
\textbf{Redattori}    & Alberto Autiero, Linor Sadè\\
\textbf{Verificatori} & Alberto Pignat, Marco Piro \\
\textbf{Uso}          & Interno \\
\textbf{Destinatari}  & Prof. Tullio Vardanega, Prof. Riccardo Cardin \\
\textbf{Data}         & 20/10/2025 \\
\end{tabular}
\end{tcolorbox}
\end{center}

\vspace{0.5cm}

\begin{center}
\begin{tcolorbox}[colback=secondaryblue!10,colframe=secondaryblue,width=0.9\textwidth,arc=3mm,boxrule=0.8pt,title={\bfseries Abstract}]
Documento di analisi e valutazione dei capitolati proposti per l'anno accademico 2025/2026. Il documento include una valutazione dettagliata del capitolato scelto e un'analisi comparativa degli altri capitolati disponibili.
\end{tcolorbox}
\end{center}

\newpage
\section{Versioni del documento}

\setlength{\extrarowheight}{2pt} % padding extra verticale
\renewcommand{\arraystretch}{1.5} 

\arrayrulecolor{primaryblue}
{\footnotesize
\begin{tabularx}{\textwidth}{|>{\raggedright\arraybackslash}p{1.5cm}|>{\raggedright\arraybackslash}p{2cm}|X|>{\raggedright\arraybackslash}p{2cm}|>{\raggedright\arraybackslash}p{2cm}|>{\raggedright\arraybackslash}p{2cm}|}
\hline
\rowcolor{primaryblue!40}
\textbf{\color{white} Versione} & \textbf{\color{white} Data} & \textbf{\color{white} Descrizione} & \textbf{\color{white} Redatto} & \textbf{\color{white} Verificato} & \textbf{\color{white} Approvato} \\
\hline
\rowcolor{secondaryblue!10} \CurrentVersion & 26/10/2025 & Correzioni minime a C4, C5, C6, C7 & Linor Sadè & - & - \\
\hline
\rowcolor{secondaryblue!10} 0.0.9 & 24/10/2025 & Continuo del documento con modifiche di C5 e C7 relative ai meeting svolti successivamente. Minori modifiche di stile. & Linor Sadè & - & - \\
\hline
\rowcolor{secondaryblue!10} 0.0.8 & 24/10/2025 & Stesura di C4 e altre piccole modifiche & Linor Sadè & - & - \\
\hline
\rowcolor{secondaryblue!10} 0.0.7 & 23/10/2025 & Stesura di C6 & Linor Sadè & - & - \\
\hline
\rowcolor{secondaryblue!10} 0.0.6 & 22/10/2025 & Correzione indice & Linor Sadè & - & - \\
\hline
\rowcolor{secondaryblue!10} 0.0.5 & 22/10/2025 & Iniziato e terminato C7 & Linor Sadè & - & - \\
\hline
\rowcolor{secondaryblue!10} 0.0.4 & 22/10/2025 & Correzioni minori stile & Linor Sadè & - & - \\
\hline
\rowcolor{secondaryblue!10} 0.0.3 & 21/10/2025 & Terminato C5, iniziato C8, C9 & Linor Sadè & - & - \\
\hline
\rowcolor{secondaryblue!10} 0.0.2 & 21/10/2025 & Aggiunta introduzione, punti di forza/debolezza & Linor Sadè & - & - \\
\hline
\rowcolor{secondaryblue!10} 0.0.1 & 20/10/2025 & Prima stesura della struttura del documento & Alberto Autiero & Marco Favero & - \\
\hline
\end{tabularx}
}




% ho commentato perche non mi piaceva
% \renewcommand{\cftsecpagefont}{\normalfont}
% \renewcommand{\cftsecleader}{\cftdotfill{\cftsecdotsep}}
% \setlength{\cftbeforesecskip}{2pt}
% \begin{center}
% \begin{tcolorbox}[colback=lightgray,colframe=darkgray,width=0.9\textwidth,arc=2mm,boxrule=0.5pt]
% \tableofcontents
% \end{tcolorbox}
% \end{center}

\newpage
\tableofcontents

\newpage
\section{Metodologia di valutazione}
Per la valutazione dei capitolati proposti, il team ha adottato una metodologia strutturata che prevede i seguenti passaggi:
\begin{enumerate}
\item \textbf{Discussione:} Dopo le presentazioni delle aziende, il team si è trovato ``a caldo'' per discutere di ciascun capitolato esaminando l'interesse verso ciascun progetto (propensione, curiosità, motivazione) e le competenze richieste (tecnologie note, tecnologie da apprendere).
\item \textbf{Analisi individuale:} Ogni membro del team ha condotto un'analisi individuale di ogni capitolato, leggendone il relativo documento di proposta, valutandone più attentamente i punti di forza e di debolezza in base a criteri predefiniti.
\item \textbf{Richiesta di chiarimenti:} In caso di dubbi o incertezze riguardo a specifici aspetti dei capitolati, il team ha preparato una lista di domande da inviare ai proponenti per ottenere chiarimenti. In seguito sono stati fissati incontri con le aziende per discutere questi punti (se necessario).
  In particolare ci interessava capire:
  \begin{itemize}
    \item Il livello di supporto che l'azienda fornirà durante lo sviluppo del progetto.
    \item L'importanza nella conoscenza pregressa delle tecnologie richieste.
  \end{itemize}
\item \textbf{Compilazione tabelle:} Il team si è diviso il compito di compilare le tabelle dei punti di forza e di debolezza per ogni capitolato, assicurando che ogni membro contribuisse in modo equo.
\end{enumerate}
I punti 3 e 4 sono stati ripetuti più volte fino a quando il team non si è sentito sufficientemente preparato per prendere una decisione informata sul capitolato da scegliere.\\
Il team ha deciso di dare particolare importanza all'interesse verso l'argomento e le tecnologie rispetto alla complessità del progetto, senza tuttavia trascurare la valutazione della sua fattibilità.\\
Crediamo che lavorare su un progetto che suscita entusiasmo e curiosità porterà a un'esperienza di apprendimento più significativa e gratificante, anche se ciò comporta affrontare sfide tecniche più impegnative.

\paragraph{}
Qui di seguito i nomi degli analisti e i corrispondenti capitolati analizzati:

\begin{tabularx}{\textwidth}{|X|X|X|}
\hline
\rowcolor{lightgray!40}
\textbf{Analista} & \textbf{Capitolato} & \textbf{Verificatore} \\
\hline
Luca Slongo & C1, C2 & Alberto Autiero\\ 
\hline
Marco Piro & C3, C4 & Alberto Autiero \\ 
\hline
Linor Sadè & C5, C6, C7 & Alberto Pignat\\ 
\hline
Leonardo Salviato & C8, C9 & Alberto Pignat\\ 
\hline
\end{tabularx}

\newpage
\section{Elenco capitolati analizzati}
Di seguito sono elencati i capitolati analizzati dal team BugBusters. Per ogni capitolato, sono riportati i punti di forza e di debolezza identificati durante la fase di valutazione.

\newpage
\subsection{C1 - Automated EN18031 Compliance Verification}
\subsubsection{Breve descrizione}
Il progetto mira a automatizzare la verifica di conformità alla norma EN 18031, legata ai dispositivi radio (Wi-Fi, Bluetooth, IoT, ecc.). L’obiettivo è quello di garantire che le apparecchiature radio non danneggino o ne abusino le risorse, che salvaguardino la privacy e siano in grado di garantire sicurezza nelle transazioni.
\subsubsection{Caratteristiche funzionali}
Il progetto prevede lo sviluppo di un’interfaccia grafica che guida nella compilazione delle domande presenti nei decision tree relative ai requisiti.  
Il software deve essere in grado di:

\begin{itemize}[leftmargin=*]
    \item Importare documenti contenenti le informazioni preliminari necessarie, che descrivono le componenti di rete del caso da analizzare e i file che descrivono i decision tree.
    \item Eseguire i decision tree associati ai requisiti, rispettando la loro struttura gerarchica e le dipendenze, in modo interattivo, chiedendo all’utente di inserire le informazioni richieste in input.
    \item Restituire per ogni requisito un output che sia ``Non applicabile'', ``Pass'' o ``Fail''.
    \item Fornire una dashboard interattiva per visualizzare lo stato di ciascun requisito del decision tree, con la possibilità di:
    \begin{itemize}[leftmargin=*]
        \item modificare i documenti associati;  
        \item aggiornare i documenti e i decision tree tramite un editor;  
        \item salvare i decision tree modificati in formati come XML o JSON.
    \end{itemize}
\end{itemize}
\subsubsection{Tecnologie proposte}
Non sono stati presentati vincoli sulle tecnologie da utilizzare e sulla tipologia di applicazione da sviluppare, l'unica preferenza espressa è sull’uso di Python per la parte di Backend.
\subsubsection{Chiarimenti e colloqui con l'azienda}
Non abbiamo contattato l’azienda per ricevere ulteriori informazioni, tantomeno chiesto colloqui.
\subsubsection{Interesse del team}
Il team non ha mostrato particolare interesse verso questo capitolato, principalmente per il motivo che non è stato ritenuto un progetto particolarmente accattivante. È stato ritenuto anche complesso per la gestione dell’interfaccia e poco “libero” dal punto di vista della possibilità di mettere del proprio.
\subsubsection{Punti di forza e di debolezza}
{\footnotesize
\begin{tabularx}{\textwidth}{|X|X|}
\hline
\rowcolor{lightgray!40} % colore intestazione
\textbf{Punti di forza} & \textbf{Punti di debolezza} \\
\hline
\begin{itemize}
\item Requisiti ben definiti: I requisiti obbligatori e opzionali sono elencati in modo chiaro e strutturato. Questo riduce l'ambiguità e fornisce un'ottima checklist per la pianificazione e la verifica finale del progetto.
\item Caso studio concreto: La presenza di un caso studio specifico (la macchina da caffè connessa via Wi-Fi) fornisce un dominio applicativo tangibile per testare le funzionalità, evitando di lavorare in modo troppo astratto.

\item Bluewind si impegna esplicitamente per un supporto "ibrido" (online e in presenza), con incontri periodici. Questo è un enorme vantaggio, poiché fornisce un canale diretto per chiarire dubbi, ottenere feedback e allinearsi con le aspettative dell'azienda.

\end{itemize}
 & \begin{itemize}
\item Complessità del dominio: Il dominio normativo (EN 18031, direttiva RED) è intrinsecamente complesso. Comprendere appieno la logica dei "Decision Tree" e le interdipendenze tra i requisiti richiederà uno sforzo iniziale di analisi e studio.
\end{itemize} \\
\hline
\end{tabularx}
}

\subsection{C2 - Code Guardian}
\subsubsection{Breve descrizione}
Il progetto mira alla realizzazione di una piattaforma web modulare, basata su un sistema multi-agente, per:

\begin{itemize}[leftmargin=*]
    \item Analizzare repository GitHub in termini di qualità, sicurezza e manutenzione;
    \item Fornire report automatici e suggerimenti di remediation;
    \item Evidenziare vulnerabilità;
    \item Offrire una dashboard che mostra lo stato dei progetti analizzati.
\end{itemize}

\subsubsection{Caratteristiche funzionali}
Per il progetto è richiesto di rispettare i seguenti vincoli:

\begin{itemize}[leftmargin=*]
    \item Svolgere un design thinking iniziale con il committente per raccogliere i requisiti;
    \item Redigere il documento dei requisiti di business e lo user story mapping;
    \item Realizzare diagrammi UML relativi agli Use Cases di progetto;
    \item Definire lo schema di design relativo alla base dati;
    \item Presentare un MVP funzionante con demo live e documentazione tecnica.
\end{itemize}

Tramite questi passaggi si intende arrivare a un software in grado di svolgere le attività illustrate nella descrizione.

\subsubsection{Tecnologie proposte}

L’azienda proponente ha specificato chiaramente le tecnologie da utilizzare per lo sviluppo del progetto e ha fornito risorse informative per facilitare il ``Get started''.

Lista delle tecnologie e dei framework richiesti:

\begin{itemize}[leftmargin=*]
    \item \textbf{Backend / orchestratore:} Node.js, Python
    \item \textbf{Frontend:} React.js
    \item \textbf{Database:} MongoDB o PostgreSQL
    \item \textbf{CI/CD e integrazioni:} GitHub Actions
    \item \textbf{Architettura cloud:} AWS
\end{itemize}

\subsubsection{Chiarimenti e colloqui con l'azienda}
Non sono state richieste ulteriori informazioni o colloqui all’azienda proponente.
\subsubsection{Interesse del team}
Il team non ha preso in considerazione fin da subito il progetto perché ritenuto particolarmente complesso. L’utilizzo di uno strumento come quello che verrebbe realizzato da questo progetto sarebbe sicuramente molto utile, ma crearlo è tutt’altra cosa e non ha suscitato l’interesse del gruppo.
\subsubsection{Punti di forza e di debolezza}
{\footnotesize
\begin{tabularx}{\textwidth}{|X|X|}
\hline
\rowcolor{lightgray!40} % colore intestazione
\textbf{Punti di forza} & \textbf{Punti di debolezza} \\
\hline
\begin{itemize}
\item Tema Innovativo e ad Alto Potenziale: L'uso di un'architettura multi-agente per l'analisi automatizzata del codice è estremamente attuale e all'avanguardia.
\item Dominio Concreto e Utile: La piattaforma risolve problemi reali di qualità del codice, sicurezza e manutenzione dei repository. 
\item Sessione di mentoring sulle tecnologie che verranno utilizzate
\end{itemize}
 & \begin{itemize}
\item Alta Complessità Concettuale: L'architettura multi-agente è concettualmente avanzata. Progettare un sistema dove agenti specializzati comunicano efficacemente attraverso un orchestratore richiede una solida comprensione di pattern complessi.
\item  Requisiti di Testing Stringenti: La richiesta del 70\% di test coverage (obbligatorio) è apprezzabile professionalmente ma può essere impegnativa da raggiungere in un progetto universitario, specialmente per componenti AI.
\item Non sembra esserci flessibilità nella scelta delle tecnologie da utilizzare
\end{itemize} \\
\hline
\end{tabularx}
}

\newpage
\subsection{C3 - DIPReader}

\subsubsection{Breve descrizione}
DIPReader propone la realizzazione di un’applicazione per la consultazione locale di pacchetti di conservazione digitale (DIP): l’obiettivo è permettere a professionisti e operatori di accedere offline a documenti conservati, effettuare ricerche e visualizzare anteprime, con funzionalità opzionali avanzate quali ricerca semantica e verifica delle firme. Il progetto risponde a un’esigenza pratica e concreta.

\subsubsection{Caratteristiche funzionali}
Il software dovrà consentire:
\begin{itemize}[leftmargin=*]
    \item Accesso offline a pacchetti DIP locali;
    \item Ricerca tra i documenti e visualizzazione di anteprime;
    \item Funzionalità opzionali avanzate quali ricerca semantica e verifica delle firme digitali;
    \item Prestazioni efficienti anche con grandi volumi di documenti;
    \item Interfaccia intuitiva per utenti professionisti e operatori.
\end{itemize}
I requisiti non funzionali includono la gestione efficiente della memoria, tempi di risposta rapidi e la conformità a standard di conservazione digitale.

\subsubsection{Tecnologie proposte}
\begin{itemize}[leftmargin=*]
    \item Linguaggi di programmazione: Python o Node.js per il backend;
    \item Librerie per ricerca e indicizzazione dei documenti (ad esempio Elasticsearch o simili);
    \item Framework frontend: React.js o equivalente per interfaccia utente;
    \item Database: MongoDB o PostgreSQL per memorizzazione metadati;
    \item Gestione dei pacchetti DIP e formati standard (es. XML, PDF/A);
    \item Strumenti per verifica firme digitali.
\end{itemize}

\subsubsection{Chiarimenti e colloqui con l'azienda}
Abbiamo identificato aspetti chiave e criticità da discutere:
\begin{itemize}[leftmargin=*]
    \item Comprendere formati, metadati e requisiti di compliance per la conservazione digitale;
    \item Definire le funzionalità obbligatorie per l’MVP e distinguere quelle opzionali come la ricerca semantica o la verifica firme digitali;
    \item Validazione e testing con grandi volumi di documenti, considerando i limiti dell’ambiente universitario.
\end{itemize}

\subsubsection{Interesse del team}
Il team valuta positivamente il progetto per i seguenti motivi:
\begin{itemize}[leftmargin=*]
    \item Il dominio è specializzato e di alto valore, con competenze spendibili professionalmente;
    \item Il progetto affronta un problema concreto e ben definito, rendendo immediatamente applicabili i risultati sviluppati.
\end{itemize}
Tuttavia, il gruppo è consapevole delle criticità tecniche e normative che richiederanno un impegno iniziale significativo e un’attenta pianificazione delle risorse.


\subsubsection{Punti di forza e di debolezza}
{\footnotesize
\begin{tabularx}{\textwidth}{|X|X|}
\hline
\rowcolor{lightgray!40} % colore intestazione
\textbf{Punti di forza} & \textbf{Punti di debolezza} \\
\hline
\begin{itemize}
\item Dominio Specializzato e di Alto Valore: La conservazione digitale è un campo di nicchia ma cruciale, specialmente in ambito legale e amministrativo.
\item Problema Concreto e Ben Definito: L'esigenza di accedere a documenti conservati digitalmente in modalità offline è un requisito reale per molti professionisti. Il progetto risolve un problema tangibile.
\item Interazione con l’azienda: offre esempi di pacchetti estratti dal sistema di conservazione e la relativa documentazione
\end{itemize}
 & \begin{itemize}
\item Complessità del Dominio Normativo: La conservazione digitale è regolata da standard e normative complesse. Comprendere appieno il formato dei DIP e i requisiti di compliance richiederà uno sforzo iniziale significativo.
\item Sfide Tecniche per l'Offline: Implementare ricerche efficienti e visualizzazioni di anteprima completamente offline, specialmente per grandi volumi di dati, presenta sfide non banali di performance e gestione della memoria.

\item Ambiguità Architetturale: Non è completamente chiaro se l'applicazione debba essere una PWA, un'app desktop (Electron) o entrambe. Questa decisione avrà impatti significativi sull'architettura.
\item Scope potenzialmente ampio: Le funzionalità opzionali come la ricerca semantica (con FAISS) e la verifica delle firme digitali sono progetti ambiziosi che potrebbero distrarre dallo sviluppo del MVP.
\item Testing su Grandi Volumi: Garantire le performance con "grandi volumi" di dati potrebbe essere difficile da testare e validare in ambiente universitario.
\end{itemize} \\
\hline
\end{tabularx}
}

\subsection{C4 - L'app che Protegge e Trasforma}
\subsubsection{Breve descrizione}
Il progetto \textit{``L’app che Protegge e Trasforma''}, proposto da Miriade S.r.l., ha come obiettivo la realizzazione di un’applicazione mobile innovativa per i sistemi operativi iOS e Android, finalizzata alla prevenzione e al supporto delle vittime di violenza di genere.\\

L’app è pensata come uno strumento intelligente, sicuro e facilmente accessibile, capace di:
\begin{itemize}
    \item riconoscere segnali di pericolo e situazioni di rischio tramite analisi del linguaggio e dei comportamenti;
    \item offrire risorse immediate e personalizzate per il supporto psicologico, legale e d’emergenza;
    \item promuovere la consapevolezza e l’autonomia dell’utente attraverso moduli educativi e interattivi;
    \item garantire la sicurezza e la privacy dei dati sensibili mediante crittografia e modalità di utilizzo anonima.
\end{itemize}
L’obiettivo finale è creare un prodotto tecnologicamente avanzato ma anche eticamente responsabile, in grado di avere un impatto sociale positivo e concreto.

\subsubsection{Caratteristiche funzionali}
Le funzionalità principali richieste dall’azienda comprendono:
\begin{itemize}
    \item \textbf{Rilevamento e Alert}: algoritmi di intelligenza artificiale per individuare pattern di rischio e situazioni di pericolo, con invio di notifiche silenziose a contatti fidati o centri di aiuto.
    \item \textbf{Risorse e Supporto}: accesso geolocalizzato a servizi di assistenza (centri antiviolenza, supporto legale, psicologico) e invio rapido di richieste di aiuto.
    \item \textbf{Funzionalità di Sicurezza Personalizzate}: modalità stealth, diario criptato e pianificazione di percorsi sicuri con mappe interattive.
    \item \textbf{Formazione e Prevenzione}: moduli multimediali, quiz e contenuti educativi per aumentare la consapevolezza e l’empowerment.
    \item \textbf{Community di Supporto (opzionale)}: spazio anonimo e moderato per la condivisione di esperienze e il supporto reciproco.
\end{itemize}

Requisiti opzionali includono:
\begin{itemize}
    \item Integrazione con dispositivi indossabili per l’invio discreto di allarmi;
    \item Funzionalità di ``check-in'' di sicurezza e modalità ``fuga rapida'';
    \item Analisi sonora ambientale per rilevare rumori associati a situazioni di pericolo;
    \item Gamification dei percorsi educativi e supporto multilingue esteso;
    \item Accessibilità per utenti con disabilità e conformità alle leggi locali e internazionali.
\end{itemize}
Dal punto di vista tecnico, l’app dovrà rispettare rigorosi requisiti di sicurezza (\textit{security by design} e \textit{privacy by design}), con crittografia AES-256, autenticazione a due fattori, audit trail e conformità GDPR.

\subsubsection{Tecnologie proposte}
Le tecnologie suggerite da Miriade per la realizzazione del progetto comprendono:
\begin{itemize}
    \item \textbf{Mobile App}: sviluppo multipiattaforma con il framework \textit{Flutter} (fortemente consigliato).
    \item \textbf{Architettura}: basata su microservizi o su modello \textit{serverless}, con deployment su cloud.
    \item \textbf{Cloud Provider}: preferibilmente \textit{Amazon Web Services (AWS)}, con possibilità di alternative come Google Cloud.
    \item \textbf{Backend}: \textit{AWS Lambda} per la logica applicativa, \textit{Amazon API Gateway} per la gestione delle richieste e \textit{AWS Step Functions} per i flussi complessi.
    \item \textbf{Database}: \textit{Amazon DynamoDB} (NoSQL) per i dati ad alta scalabilità e \textit{Amazon RDS} (PostgreSQL/MySQL) per dati strutturati.
    \item \textbf{Storage}: \textit{Amazon S3} per contenuti multimediali e dati configurativi.
    \item \textbf{Autenticazione e Sicurezza}: \textit{Amazon Cognito} per la gestione delle identità e l’autenticazione sicura.
    \item \textbf{AI e Machine Learning}: utilizzo di \textit{Amazon SageMaker} e \textit{Amazon Bedrock} per le funzionalità di analisi del linguaggio e del comportamento (``Detective delle Relazioni'' e ``Specchio Intelligente'').
    \item \textbf{Comunicazione Asincrona}: \textit{Amazon Kinesis} o \textit{AWS SQS} per la gestione di code di messaggi.
    \item \textbf{Monitoraggio e Logging}: \textit{AWS CloudWatch} per l’osservabilità e la gestione degli eventi di sistema.
\end{itemize}
Questa infrastruttura, basata su tecnologie cloud e AI, garantisce sicurezza, scalabilità e manutenzione semplificata nel tempo.

\subsubsection{Chiarimenti e colloqui con l'azienda}
Qui di seguito sono riportate alcune delle domande poste a Miriade con le relative risposte fornite dall'azienda. 

\paragraph{}
\begin{center}
\small\textit{Nota: Le risposte dell’azienda non sono citazioni dirette, ma sono state riassunte e riformulate dal team in modo da essere più concise.}
\end{center}

{\footnotesize
\begin{tabularx}{\textwidth}{|>{\raggedright\arraybackslash}X|>{\raggedright\arraybackslash}X|}
\hline
\textbf{Domande} & \textbf{Risposte} \\
\hline
C'è la possibilità di avere incontri online anche settimanalmente?
&
\begin{itemize}
  \item L'azienda si organizza in sprint di 2 settimane
  \item è disponibile supporto anche al di fuori degli incontri programmati, tramite email o se richiesti anche tramite riunioni in sede a Padova.
\end{itemize}
\\
\hline
L'azienda ha pensato ad un eventuale formazione iniziale rispetto alle tecnologie da utilizzare e alla tematica della violenza di genere?
&
\begin{itemize}
  \item Sono disponibili corsi di formazione tecnica e il supporto di referenti specializzati per ogni area tecnica
  \item In azienda lavora una referente che, purtroppo, ha vissuto in prima persona la violenza di genere e che può fornire un supporto prezioso per comprendere le esigenze delle vittime.
  \item Miriade lavora con un'associazione che ha come fulcro sul tipo di tematiche che l'applicazione tratta. 
\end{itemize}\\
\hline
Essendo un progetto alquanto completo quello presentato nel capitolato il team voleva sapere se era stata fatta una stima delle tempistiche per l'implementazione delle varie funzionalitá e se erano stati predisposti dei requisiti minimi ottenibili nel tempo che ci è stato assegnato.
&
\begin{itemize}
  \item Il documento del capitolato fornisce una serie di aspettative per un'applicazione completa, ma i requisiti minimi non sono ancora stati definiti: in caso di conferma dell'assegnazione del capitolato, l'azienda lavorerà con il team per definire un MVP realistico.
  \item Data la mancanza di requisiti minimi non possono ancoar essere determinate le tempistiche per l'implementazione delle varie funzionalità.
\end{itemize}\\
\hline
\end{tabularx}
}  
\subsubsection{Interesse del team}
\parbox[t]{\linewidth}{%
Il team ha riconosciuto l'importanza sociale del progetto proposto da Miriade e si è dimostrato, anche per questo, interessato.
Inoltre, il team ha riconosciuto l'opportunità di implementare un'applicazione mobile completa, che copre l'intero ciclo di vita dello sviluppo software, dall'analisi dei requisiti allo sviluppo di un'applicazione non ancora esistente.
\\ Tuttavia, il team ha espresso alcune preoccupazioni riguardo alla complessità del dominio delle richieste: benchè le funzionalità richieste siano preferenziali e non requisiti minimi, il team teme che la mole di lavoro richiesta per implementare tutte le funzionalità possa essere eccessiva per il tempo a disposizione.
}
\subsubsection{Punti di forza e di debolezza}
{\footnotesize
\begin{tabularx}{\textwidth}{|X|X|}
\hline
\rowcolor{lightgray!40} % colore intestazione
\textbf{Punti di forza} & \textbf{Punti di debolezza} \\
\hline
\begin{itemize}
\item Impatto sociale elevatissimo: Il progetto ha uno scopo nobile e concretamente utile, prevenire e supportare vittime di violenza di genere.
\item Buon supporto aziendale: Miriade offre un supporto completo:
\begin{itemize}
  \item Referenti specializzati per ogni area (tecnica, design, dominio sociale)
  \item Formazione sulla tematica della violenza di genere, se richiesta
  \item Supporto multidisciplinare continuo
  \item Possibilità di incontri in sede
\end{itemize}
\item Ottima possibilità riguardo il ciclo di vita dell’applicazione: analisi, progettazione, sviluppo, test di sicurezza e controllo dei contenuti etici devono essere fatti in modo chiaro e preciso, apprendendo così competenze multidisciplinari.
\item Integrazione di AI/LM. Lavorare su queste tecnologie fornisce un'esperienza preziosa vista la loro crescente importanza nel settore.
\end{itemize}
 & \begin{itemize}
\item Alta complessità tecnica e progettuale:
\begin{itemize}
  \item L'architettura proposta è ambiziosa e forse eccessiva essendo per alcuni il primo affronto verso certe conoscenze
  \item La sicurezza dei dati è critica e richiede implementazioni robuste
\end{itemize}
\item 	Responsabilità e sensibilità del dominio: L'errore in un'app di questo tipo può avere conseguenze gravi, soprattutto dal punto di vista etico. La progettazione deve essere impeccabile sotto il profilo della sicurezza e dell'affidabilità.
\item Scope molto ampio: Le funzionalità previste sono numerose e ambiziose (rilevamento AI, allarmi silenziosi, diario criptato, moduli educativi, community). Il rischio di sovra-estendere il progetto è alto.
\end{itemize} \\
\hline
\end{tabularx}
}

\newpage
\subsection{C5 - Nexum}
\subsubsection{Breve descrizione}
\parbox[t]{\linewidth}{%
NEXUM nasce come piattaforma HR evoluta, in grado di connettere aziende, collaboratori e gli studi dei CdL. Il progetto prevede lo sviluppo di nuovi moduli per la piattaforma esistente, con l'obiettivo di migliorare l'efficienza dei processi HR e offrire funzionalità innovative ai suoi utenti.
Il team si occuperà di sviluppare due moduli principali:
}

\begin{itemize}
\item \textbf{AI Assistant Generativo per HR:} \\
\begin{minipage}[t]{\dimexpr\linewidth-2em}
Che dovrà permettere agli utenti di creare in autonomia contenuti accattivanti con titolo, descrizione e immagine di copertina attraverso l'uso di AI generativa, adeguando tono e stile della comunicazione a quello aziendale (formale, informale ...ecc).
\end{minipage}

\item \textbf{AI Co-Pilot per i CdL:} \\
\begin{minipage}[t]{\dimexpr\linewidth-2em}
Deve essere in grado di riconoscere la tipologia di documenti caricati (cedolini, comunicazioni, documenti da firmare, ecc.) e i destinatari, direttamente dal documento e consegnarli ai destinatari anche in modo massivo.
\end{minipage}
\end{itemize}

\subsubsection{Caratteristiche funzionali}
Tra i requisiti funzionali del progetto vi sono:
\begin{itemize}[noitemsep, topsep=0pt]
  \item Integrazione con la piattaforma esistente NEXUM
  \item Utilizzo di tecnologie cloud (AWS)
  \item Implementazione delle funzionalità di AI generativa e AI Co-Pilot
  \item Le funzionalità devono essere disponibili sulla dashboard o sulla PWA
\end{itemize}
Tra i requisiti non funzionali vi sono:
\begin{itemize}[noitemsep, topsep=0pt]
  \item fluidaità d'uso e interfaccia user-friendly
  \item le operazioni time-consuming dovranno essere delegate a sistemi batch
\end{itemize}
Altri requisiti di prestazione saranno concorati con il team in base alle tecnologie utilizzate e soluzioni proposte
\subsubsection{Tecnologie proposte}
Le tecnologie proposte per lo sviluppo del progetto includono:
\begin{itemize}[noitemsep, topsep=0pt]
  \item \textbf{Frontend:} Angular (dashboard amministrativa), Next.js (PWA utenti finali, hosting su AWS Amplify o S3+CloudFront)
  \item \textbf{Backend e API:} Ruby on Rails (stateless su ECS Fargate dietro ALB)
  \item \textbf{Background Jobs:} Sidekiq (su Fargate + SQS)
  \item \textbf{Database e cache:} PostgreSQL (RDS Multi-AZ con snapshot automatici), ElastiCache for Redis (cache e gestione sessioni)
  \item \textbf{Storage:} Amazon S3 (bucket separati per “uploads” e “processed”, Lifecycle e legal hold opzionale)
  \item \textbf{Sicurezza:} KMS (gestione chiavi per S3, RDS, Secrets), Secrets Manager (credenziali, API keys, JWT secrets), Amazon Cognito (gestione identità e accessi)
  \item \textbf{Comunicazioni:} SES (email), SNS (notifiche push/eventi)
  \item \textbf{Networking:} VPC (reti private e pubbliche per ECS/RDS/Redis, con ALB e NAT Gateway), Security Groups (accesso minimo privilegio)
  \item \textbf{Sicurezza web:} WAF + AWS Shield (protezione ALB/CloudFront), IAM policy granulari (permessi per ECS task roles, S3, SQS, Secrets, CloudWatch)
  \item \textbf{Osservabilità:} CloudWatch Logs/Metrics/Alarms/X-Ray (monitoring e tracing)
  \item \textbf{Compliance e threat detection:} AWS Config + GuardDuty
\end{itemize}

\subsubsection{Chiarimenti e colloqui con l'azienda}

\parbox[t]{\linewidth}{%
Qui di seguito sono riportate alcune delle domande poste ad Eggon con le relative risposte fornite dall'azienda. In particolare abbiamo posto domande che riguardavano il supporto di Eggon durante lo sviluppo del progetto, l'importanza della conoscenza pregressa delle tecnologie richieste e il coinvolgimento del team nel processo SCRUM dell'azienda, in quanto volevamo comprendere come l’azienda avrebbe affrontato la gestione di un progetto di questo tipo, trattandosi del primo anno in cui Eggon vi partecipava.
}

{\footnotesize
\begin{tabularx}{\textwidth}{|>{\raggedright\arraybackslash}X|>{\raggedright\arraybackslash}X|}
\hline
\textbf{Domande} & \textbf{Risposte} \\
\hline
Le tecnologie richieste per lo sviluppo del progetto sono diverse e molte sono completamente nuove per noi: quanto è rilevante per voi la conoscenza pregressa e quali sono le vostre aspettative rispetto al nostro apprendimento progressivo durante il progetto? Verrà fornito supporto o affiancamento nell'utilizzo di queste tecnologie? 
&
\begin{itemize}
  \item \textbf{Conoscenza pregressa:} non vincolante. Valutiamo impegno, qualità del codice e velocità di apprendimento.
  \item \textbf{Aspettative:} avanzamento sprint-by-sprint, PR piccole e frequenti, test minimi, documentazione essenziale.
  \item \textbf{Supporto Eggon:} kickoff e seed repository, canale e-mail/Telegram, code review e pairing su temi critici, sandbox (API mock, S3, chiavi temporanee).
\end{itemize} \\
\hline

Nella vostra esperienza avete già avuto modo di affidare una certa responsabilità operativa o decisionale a team che non avevano ancora esperienza nel mondo del lavoro? Se sì, quali risultati o insegnamenti ne avete tratto in termini di autonomia, qualità del lavoro e collaborazione con il vostro team interno?
&
\begin{itemize}
  \item Ci lavoriamo spesso: funziona quando suddividiamo il lavoro in milestone piccole con demo frequenti, manteniamo standard chiari (lint/test/review) e i blocchi emergono subito.
  \item \textbf{Obiettivo:} autonomia crescente — più guida all’inizio, più ownership col passare degli sprint.
\end{itemize} \\
\hline

\parbox[t]{\linewidth}{%
Avete parlato di includere il team di lavoro nel vostro processo SCRUM e nelle riunioni o stand-up periodiche: quale cadenza hanno questi incontri e come si svolgono concretamente? \\Considerando che abbiamo anche impegni universitari, ci potete chiarire se è previsto che partecipiamo a tutte le daily stand-up o solo ad alcune delle cerimonie principali (ad esempio sprint review o retrospettive)?
}
&
\begin{itemize}
  \item \textbf{Sprint:} 2 settimane.
  \item \textbf{Cerimonie:}
  \begin{itemize}
    \item Grooming/Planning ($\approx$ 1h; nei primi sprint può servire più tempo) — obbligatoria.
    \item Check-in asincroni su Telegram (daily in 3 righe: fatto / da fare / blocchi).
    \item Review + Retro ($\approx$ 1h) — obbligatorie con demo.
  \end{itemize}
  \item Calendario condiviso.
  \begin{itemize}
    \item Lo costruiamo insieme attorno ai vostri impegni di studio (lezioni, esami, sessioni).
    \item Una volta concordate milestone e scadenze, ci si impegna a rispettarle: fa parte del patto professionale azienda-fornitore e ci permette di coordinare bene tutto il team.
  \end{itemize}
\end{itemize} \\
\hline
\parbox[t]{\linewidth}{%
    Quali modelli LLM specifici prevedete di utilizzare? Oppure possiamo testare con diversi provider?
}
&
\begin{itemize}
  \item \textbf{Preferenza:} AWS Bedrock (integrazione e governance). Tramite Bedrock possiamo usare più modelli (Claude, Llama, Mistral).
  \item \textbf{Apertura ad alternative:} via libera a provider/idee creative, purché valutate su qualità output, aderenza al prompt, performance, costi e manutenibilità.
  \item \textbf{Requisito:} adapter per evitare lock-in
\end{itemize} \\
\hline
\end{tabularx}
}
\begin{center}
\small\textit{Nota: Le successive domande sono state esposte durante un incontro di approfondimento e conoscitivo con Eggon. 
Le risposte dell’azienda non sono citazioni dirette, ma sono state riassunte e riformulate dal team in modo da essere più concise.}
\end{center}

{\footnotesize
\begin{tabularx}{\textwidth}{|>{\raggedright\arraybackslash}X|>{\raggedright\arraybackslash}X|}
\hline
\textbf{Domande} & \textbf{Risposte} \\
\hline
\parbox[t]{\linewidth}{
Dovremo fare una parte di formazione iniziale per comprendere bene la piattaforma Nexum e l'integrazione tra i moduli?
} &
\parbox[t]{\linewidth}{
\begin{itemize}[leftmargin=*]
  \item Il dominio HR e CdL potrà essere appreso tramite interazioni con Eggon, in quanto conosce nel dettaglio come intende supportare i manager HR e i CdL nel loro lavoro;
  \item Ci sarà una fase di onboarding iniziale in cui vi verrà spiegata la code base delle componenti Nexum che userete;
  \item Ci aspettiamo una fase iniziale di studio e apprendimento delle tecnologie proposte.
\end{itemize}
} \\
\hline
\parbox[t]{\linewidth}{
Quali sono le vostre aspettative su questo progetto?
} &
\parbox[t]{\linewidth}{
  Il progetto non è orientato esclusivamente al risultato finale, ma punta sul processo e sul confronto tra idee. La presenza di più team universitari è considerata un valore aggiunto, poiché favorisce la diversità di approcci e soluzioni. Inoltre, i requisiti e gli use case riportati nel documento sono solo abbozzati e ci aspettiamo che vengano ridefiniti sulla base delle vostre analisi e proposte.
} \\
\hline
\parbox[t]{\linewidth}{
Potreste spiegarci come vi aspettate che vengano utilizzati gli LLM nel progetto? In particolare come debba l'AI rispecchiare i toni aziendali di ciascun cliente.
} &
\parbox[t]{\linewidth}{
Si utilizzano modelli pre-addestrati come base, che vengono successivamente adattati al dominio specifico tramite gli strumenti messi a disposizione da alcune piattaforme, come ad esempio Amazon Bedrock, attraverso funzionalità quali Knowledge Base e Guardrails.
A questo proposito, è fondamentale considerare gli aspetti di sicurezza, in particolare per prevenire possibili leak di informazioni tra diverse knowledge base aziendali, ad esempio causati da attacchi di prompt injection.
} \\
\hline
\parbox[t]{\linewidth}{
Quali tecnologie ritenete possano presentare le principali criticità o difficoltà di implementazione per il team? Avete dei consigli a riguardo?
} &
\parbox[t]{\linewidth}{
La documentazione di AWS risulta in alcuni casi poco chiara. Il codice di backend attualmente disponibile deriva da un progetto sviluppato alcuni anni fa e successivamente aggiornato alle versioni più recenti di Ruby on Rails. Tale codice presenta alcune parti ridondanti o non più necessarie, che potrebbero generare confusione durante lo sviluppo.
È prevedibile che l’integrazione con AWS rappresenti una delle principali difficoltà; si consiglia pertanto di esaminare le diverse librerie e SDK disponibili per Bedrock, preferibilmente in linguaggi già noti al team, per comprendere il funzionamento del servizio. Una volta acquisite le necessarie competenze, sarà possibile effettuare una migrazione del codice in Ruby on Rails.
} \\
\hline
\end{tabularx}
}

\subsubsection{Interesse del team}
\parbox[t]{\linewidth}{%
Il team ha mostrato un interesse significativo verso la proposta di Eggon. Inanzitutto, i rappresentanti dell'azienda hanno subito mostrato uno stile giovanile e informale, pur mantenendo professionalità e struttura. Ciò ha avuto un impatto positivo come prima impressione. \\Inoltre, il progetto proposto si allinea bene con le aspirazioni del team di acquisire esperienza pratica nello sviluppo di applicazioni reali, specialmente in un contesto aziendale. L'opportunità di lavorare su una piattaforma HR esistente come NEXUM, che ha una base di utenti reale, ha particolare attrattiva.\\ Infine, come già specificato all'inizio di questo documento, il team pone molta importanza al supporto che l'azienda può offrire durante lo sviluppo del progetto. Le risposte fornite da Eggon alle nostre domande hanno confermato che l'azienda è disposta a fornire un supporto strutturato e continuo, il che aumenta ulteriormente l'interesse del team verso questo capitolato.
}
\paragraph{}
Visto l'interesse del team, è stato organizzato anche un incontro di approfondimento con Eggon, durante il quale sono stati chiariti ulteriori dubbi e domande sul progetto.
\subsubsection{Punti di forza e di debolezza}
{\footnotesize
\begin{tabularx}{\textwidth}{|X|X|}
\hline
\rowcolor{lightgray!40} % colore intestazione
\textbf{Punti di forza} & \textbf{Punti di debolezza} \\
\hline
\begin{itemize}
\item Prodotto Rrale e integrazione con piattaforma esistente: NEXUM è una piattaforma HR già operativa. Sviluppare moduli che si integreranno in un prodotto commerciale fornisce un'esperienza di lavoro su codice legacy e integrazione con sistemi esistenti.
\item Processo SCRUM realistico (rispettare le scadenze dell’azienda)
\item Ottimo supporto dell’azienda: l’azienda ha un piano ben strutturato quali le cerimonie a cui il team deve necessariamente partecipare (con relativa durata) e il tipo di dialogo che vuole con il team (giornalmente). Inoltre siamo stati informati che il supporto al progetto è continuo (anche formazione e mentoring).
\item Tecnologie definite esaustivamente e  requisiti del progetto chiari, con due requisiti opzionali definiti.
\item L’azienda ha fornito un documento per il capitolato completo di casi d’uso e obiettivi misurabili in milestone, rendendo la pianificazione più semplice
\item Integrazione di AI. Lavorare su queste tecnologie fornisce un'esperienza preziosa vista la loro crescente importanza nel settore.
\item Abbiamo particolarmente apprezzato l’obiettivo finale di Eggon, volto a coinvolgere giovani talenti per permettere loro di conoscere l’azienda e farsi conoscere, offrendo anche un punto di vista fresco e diverso
\end{itemize}
& 
\begin{itemize}
\item NEXUM è una piattaforma già esistente e operativa, lo sviluppo di nuovi moduli (che verranno realmente utilizzati) comporta un livello di rischio elevato. L’integrazione in un prodotto reale, destinato a un mercato effettivo e a clienti concreti, implica vincoli progettuali e tecnologici particolarmente rigidi, dettati dalle esigenze e dagli standard aziendali. Questo può limitare la flessibilità del team e aumentare la complessità delle attività di sviluppo e validazione
\item Alta complessità riguardante le tecnologie da utilizzare con AWS, OCR (Optical Character Recognition), e integrazione dell’AI.
\end{itemize} \\
\hline
\end{tabularx}
}
\subsection{C6 - Second Brain}

\subsubsection{Breve descrizione}
Sviluppare una \textbf{web app} (in \textbf{HTML} e altre tecnologie web) che rappresenti un \textbf{editor basato su Markdown} con l’integrazione di strumenti di \textbf{Intelligenza Artificiale (AI)} per assistere l’utente nella scrittura e revisione dei testi.
In particolare l'applicazione dovrà essere in grado di interagire con l'AI per eseguire opreazioni come riassumere, riscrivere, tradurre e criticare il testo scritto dall'utente. Inoltre, l'applicazione dovrà supportare un prompt generativo associato all'editor per permettere la generazione automatica di testi.
\subsubsection{Caratteristiche funzionali}
Tra i requisiti funzionali principali del progetto, l'applicazione deve includere:
\begin{enumerate}[noitemsep, topsep=0pt]
  \item \textbf{Editing del testo:} area di scrittura che accetta testo e marcatori in formato \texttt{Markdown}.
  \item \textbf{Rendering grafico:} visualizzazione del testo formattato, applicando lo stile corrispondente ai marcatori Markdown.
  \item \textbf{Accesso a un LLM:} integrazione con un \textit{Large Language Model} in grado di operare sull’intero testo o su porzioni selezionate.
  \item \textbf{Comandi di base basati su AI:}
  \begin{itemize}
    \item Riassunto del testo.
    \item Riscrittura automatica del testo.
    \item Traduzione del testo in una lingua differente.
  \end{itemize}
  
  \item \textbf{Critica del testo:} implementazione di comandi di analisi secondo il modello dei “\textit{sei cappelli per pensare}” di \textit{Edward De Bono}.
  \item \textbf{Prompt generativo:} gestione di un prompt associato che consente la generazione automatica di un intero testo all’interno della finestra di editing.
  \item \textbf{Persistenza dei dati:} possibilità di salvare e leggere le note come file di testo.
\end{enumerate}
\paragraph{}
Zucchetti è più interessata all'idea di esplorare le caratteristiche dell'AI (e in particolare del concetto dei `Sei cappelli per pensare') piuttosto che ad un concreto uso dell'applicazione. Per questo motivo,
presentazione e salvataggio nel database, o un sistema di autenticazione utenti, sono opzionali e verrà come minimo richiesta la possibilità di salvare e leggere note dal file system.
\paragraph{}
Eventualmente l'Azienda potrebbe fornire un database volto all'integrazione dell'applicazione con un sistema di autenticazione utenti e salvataggio delle note. 
Estendendo il progetto in questa direzione, sarebbe possibile anche costruitire un sistema di \textit{linking} tra le note.
\subsubsection{Tecnologie proposte}
\begin{itemize}[noitemsep, topsep=0pt]
  \item \textbf{Frontend:} HTML5 per l'interfaccia web dell’editor; linguaggio di markup \textit{Markdown} per la scrittura;
  \item \textbf{Integrazione LLM:} API compatibili OpenAI per chiamate a modelli linguistici; supporto a modelli come \textit{Gemini} (Google), \textit{Mistral}, \textit{Gemma}; uso opzionale di \textit{llama.cpp} per esecuzione locale.
  \item \textbf{Backend e API (opzionale):} server-side in \textit{Python} o \textit{Java}; gestione delle richieste HTTP e delle interazioni con il database; middleware per superare la same-origin policy.
  \item \textbf{Database (opzionale):} archivio note e metadati su DB; supporto per collegamenti tra note (link nel markup).
  \item \textbf{Modelli ispiratori:} \textit{Joplin}, \textit{Obsidian}, \textit{Logseq} — esempi di editor Markdown; \textit{ChatGPT} come riferimento per l’interfaccia conversazionale.
\end{itemize}

\subsubsection{Chiarimenti e colloqui con l'azienda}
\footnotesize
\begin{tabularx}{\textwidth}{|>{\raggedright\arraybackslash}X|>{\raggedright\arraybackslash}X|}
\hline
\textbf{Domande} & \textbf{Risposte} \\
\hline
\parbox[t]{\linewidth}{
Quali modelli LLM saranno disponibili e con che modalità potremo accedervi? Possiamo utilizzare modelli diversi durante il progetto?
}
&
\begin{itemize}
  \item \textbf{Modelli forniti:} Claude e Gemini accessibili dall'esterno; Gemma, Deepseek e Qwen disponibili su infrastrutture interne BugBuster (tra 7B e 32B, versioni quantizzate).
  \item \textbf{Accesso:} verrà fornita una \texttt{API key} per connettersi a un endpoint dedicato gestito dall’azienda.
  \item \textbf{Utilizzo:} i modelli potranno essere usati anche per attività di traduzione e analisi linguistica.
\end{itemize} \\
\hline

\parbox[t]{\linewidth}{
È previsto un database specifico per l’applicazione? Ci sono preferenze o vincoli tecnologici?
}
&
\begin{itemize}
  \item \textbf{Database consigliato:} PostgreSQL.
  \item \textbf{Vincoli:} nessuno — il team può scegliere liberamente soluzioni alternative se più adatte.
\end{itemize} \\
\hline

\parbox[t]{\linewidth}{
È richiesta una gestione dell’autenticazione utenti? Se sì, quale approccio consigliate?
}
&
\begin{itemize}
  \item \textbf{Autenticazione:} non richiesta come requisito obbligatorio.
  \item \textbf{Suggerimento:} se si desidera includerla, utilizzare un sistema basato su token \textit{OAuth}.
\end{itemize} \\
\hline

\parbox[t]{\linewidth}{
L’applicazione dovrà includere funzioni di traduzione automatica o comparazione tra modelli linguistici?
}
&
\begin{itemize}
  \item \textbf{Traduzione:} gli LLM forniti possono essere impiegati anche per traduzioni automatiche.
  \item \textbf{Comparazione:} non richiesta, ma la possibilità di scegliere il modello di LLM da utilizzare è considerata una funzionalità opzionale interessante.
  \item \textbf{Sistema di confronto:} non richiesto — era parte del progetto dell’anno precedente.
\end{itemize} \\
\hline

\parbox[t]{\linewidth}{
Avete preferenze sui framework di sviluppo o sull’architettura (container, PWA, app nativa)?
}
&
\begin{itemize}
  \item \textbf{Framework:} completa libertà di scelta.
  \item \textbf{Applicazione:} non è richiesta un’app nativa. È sufficiente che l’interfaccia sia \textit{responsiva} e utilizzabile su PC e tablet.
  \item \textbf{PWA:} facoltativa; può essere sviluppata a discrezione del team.
  \item \textbf{Container:} scelta libera. L’app può essere eseguita in modo tradizionale oppure containerizzata. BugBuster utilizza \textit{Docker} o \textit{Podman} senza orchestratore, oppure \textit{Kubernetes} in ambienti completi.
\end{itemize} \\
\hline

\parbox[t]{\linewidth}{
Il progetto prevede aspetti di sicurezza applicativa o gestione avanzata degli accessi?
}
&
\begin{itemize}
  \item \textbf{Sicurezza:} non richiesta come parte obbligatoria del progetto.
  \item \textbf{Motivazione:} una trattazione completa del tema richiederebbe un impegno eccessivo rispetto ai tempi previsti.
\end{itemize} \\
\hline
\end{tabularx}


\subsubsection{Interesse del team}
Troviamo che il capitolato proposto da Zucchetti sia realistico per un team di studenti universitari che si interfacciano per la prima volta ad un progetto di questo calibro, sia dal punto di vista delle tempistiche e delle competenze richieste.
Inoltre, l'argomento dell'integrazione di modelli di linguaggio di grandi dimensioni (LLM) in applicazioni pratiche è estremamente attuale e rilevante nel contesto tecnologico odierno.
\subsubsection{Punti di forza e di debolezza}
{\footnotesize
\begin{tabularx}{\textwidth}{|X|X|}
\hline
\rowcolor{lightgray!40} % colore intestazione
\textbf{Punti di forza} & \textbf{Punti di debolezza} \\
\hline
\begin{itemize}
\item  L'integrazione degli LLM in un'applicazione fornisce un'esperienza preziosa vista la loro crescente importanza nel settore. Lavorare su questo progetto fornisce un'esperienza diretta con questo tipo di tecnologia.
\item  Ampio spazio per la progettazione e la ricerca: Il capitolato non specifica esattamente come implementare le funzionalità, ma si concentra sul cosa. Questo lascia al team una grande libertà nelle scelte architetturali, nella progettazione dell'interfaccia utente (UI/UX) e, soprattutto, nell'ingegnerizzazione dei prompt per l'LLM, che è il cuore del progetto.
\item Zucchetti si offre esplicitamente di supportare il team nelle parti più complesse (es. configurazione API LLM) e metterà a disposizione gli LLM stessi. 
\item Tempi di realizzazione adeguati: Zucchetti ha esplicitamente confermato che aggiungere altre funzionalità (es. tema della sicurezza) potrebbe impegnare al punta da sforare i tempi previsti per il progetto. Questo indica una consapevolezza realistica delle tempistiche e delle capacità del team.
\end{itemize}
 & \begin{itemize}
\item Le tecnologie non risultano chiaramente definite e i requisiti di progetto sono ampi. Inoltre, l’assenza di specifiche vincolanti e di requisiti opzionali chiaramente delineati comporta un’eccessiva libertà decisionale, che può generare incertezza nella pianificazione e aumentare il rischio di deviazioni dagli obiettivi progettuali.
\item  L’azienda propone lo sviluppo di un’applicazione web basata su HTML per la parte front-end. Tuttavia, il team di progetto, è interessato anche all'utilizzo di nuove tecnologie più moderne e avanzate per lo sviluppo web, come i framework \textit{React} e \textit{Angular}. 
\end{itemize} \\
\hline
\end{tabularx}
}
\subsection{C7 - Sistema di acquisizione dati da sensori}
\subsubsection{Breve descrizione}
\paragraph{} 
Il progetto mira a sviluppare un sistema distribuito di acquisizione e smistamento dati dai sensori BLE. 
La piattaforma deve essere in grado di ricevere, aggregare, normalizzare e smistare tali informazioni in modo affidabile, sicuro e scalabile.

\paragraph{} 
La piattaforma deve essere articolata in tre livelli principali:

\begin{itemize}
  \item \textbf{Sensori BLE:} dispositivi periferici che raccolgono dati dal campo (es. temperatura, umidità, movimento, segnali biometrici);
  \item \textbf{Gateway BLE--WiFi:} nodi intermedi che si connettono ai sensori, raccolgono i dati tramite profili standard o personalizzati, 
  li formattano secondo un modello interno e li inviano al cloud;
  \item \textbf{Cloud:} piattaforma centrale che gestisce la connessione sicura dei gateway, riceve e bufferizza i dati, 
  li rende disponibili tramite API e interfacce di visualizzazione, garantendo segregazione tra diversi tenant.
\end{itemize}

\paragraph{} 
I primi due livelli (sensori BLE e gateway) non sono oggetto di questo capitolato e devono essere considerati come strumenti esterni. 
Nel contesto del progetto, il team non deve realizzare un gateway fisico, ma sviluppare un \textbf{simulatore di gateway} 
in grado di riprodurre in modo verosimile il comportamento di un nodo BLE--WiFi. 
Tale simulatore sarà utilizzato per validare l'infrastruttura cloud e testare i flussi di comunicazione previsti.


\subsubsection{Caratteristiche funzionali}
Tra i requisiti funzionali del progetto vi sono:
\begin{enumerate}[noitemsep, topsep=0pt]
  \item \textbf{Acquisizione e generazione dati da sensori:} il sistema deve gestire dati provenienti da sensori Bluetooth Low Energy (BLE). Poiché i sensori reali non rientrano nell'ambito del progetto, i dati saranno generati dal simulatore di gateway, che produrrà valori realistici per 4--5 tipologie di sensori da concordare.

  \item \textbf{Simulatore di gateway:} in sostituzione di un gateway fisico, deve essere sviluppato un simulatore 
  in grado di riprodurne il comportamento principale.
  \begin{enumerate}
    \item generazione diretta dei dati simulati pronti alla trasmissione;
    \item invio al cloud tramite protocolli sicuri (SSL/TLS);
    \item gestione di più sensori simulati in parallelo;
    \item persistenza delle informazioni di commissioning (sensori/gateway) per garantire consistenza;
    \item capacità di rispondere ai messaggi dal cloud, anche in forma semplificata 
    (con supporto opzionale a flag di debug che contengano la risposta attesa).
  \end{enumerate}

  \item \textbf{Gestione multi-tenant:} il livello cloud deve garantire isolamento logico tra diversi tenant, 
  permettendo a ciascun cliente/utente di accedere solo ai propri dati.

  \item \textbf{Esposizione API centralizzate:} i dati raccolti dal simulatore e gestiti dal cloud devono essere accessibili 
  tramite API sicure, documentate e versionate (test).
  \begin{enumerate}
    \item ogni client deve essere identificato univocamente e certamente, per garantire il controllo 
    dell’accesso ai dati e la loro segregazione;
    \item le API devono fornire accesso ai dati nelle seguenti modalità:
    \begin{enumerate}
      \item \textit{on demand:} accesso su richiesta a uno o più dati di un gateway, 
      con possibilità di filtrare i dati per sensore e data/ora di ricezione;
      \item \textit{stream:} accesso ai dati in tempo reale attraverso uno stream continuo.
    \end{enumerate}
  \end{enumerate}

  \item \textbf{Interfaccia utente:} deve essere fornita una UI web che consenta la consultazione e l’esplorazione dei dati acquisiti, 
  con procedure semplificate per registrare e configurare nuovi sensori o gateway simulati.
\end{enumerate}
Tra i requisiti non funzionali vi sono: 
\begin{itemize}[noitemsep, topsep=0pt]
  \item \textbf{Scalabilità:}il sistema cloud deve progettato per essere scalabile orizzontalmente per gestire
un numero crescente di sensori, gateway e tenant senza degrado delle prestazioni;
  \item \textbf{Test automatizzati} e \textbf{code coverage:}lo sviluppo deve includere test unitari e di integrazione
con un livello minimo di copertura da concordare ad inizio progetto;
  \item \textbf{Monitoraggio:} devono essere predisposti strumenti di monitoraggio in tempo reale per le performance del sistema. 
  In aggiunta, devono essere presenti almeno funzionalità di alert di base che consentano di capire rapidamente 
  se un gateway sta funzionando correttamente o risulta non raggiungibile;
  \item \textbf{Versionamento} e \textbf{DevOps:} il codice deve essere gestito tramite Git con pipeline CI/CD 
  per il deploy su ambienti di test e produzione.
\end{itemize}

\subsubsection{Tecnologie proposte}
Le tecnologie proposte per lo sviluppo del progetto includono:
\begin{itemize}[noitemsep, topsep=0pt]
  \item \textbf{Sviluppo:} 
  interfaccia utente in \textbf{Angular} per la creazione di una \textbf{SPA} (Single Page Application); 
  microservizi sviluppati con \textbf{Node.js} e \textbf{Nest.js} in \textbf{TypeScript} per la parte di API.
  
  \item \textbf{Orchestrazione:} 
  il sistema sarà ospitato su \textit{Google Cloud Platform} e orchestrato tramite \textbf{Kubernetes}.
  
  \item \textbf{Archiviazione dati:} 
  \textit{MongoDB} per la memorizzazione di dati non strutturati (quelli grezzi raccolti dai sensori); 
  \textit{PostgreSQL} per la persistenza dei dati strutturati (quelli raccolti e manipolati per renderli \textit{valuable}).
  
  \item \textbf{Opzionale:} 
  \textit{Redis} come sistema di caching per ridurre la latenza e migliorare le prestazioni.
\end{itemize}
Tuttavia le tecnologie sopra elencate non sono vincolanti: il team può proporre alternative ma devono in ultimo essere approvate da M31.
\subsubsection{Chiarimenti e colloqui con l'azienda}
\parbox[t]{\linewidth}{%
Qui di seguito sono riportate alcune delle domande poste a M31 con le relative risposte fornite dall'azienda. Le domande vertono principalmente su aspetti tecnici del progetto, poichè il team voleva chiarire alcuni dubbi riguardanti l'architettura del sistema e le tecnologie da utilizzare. \\
Il team ha deciso di fissare una runione per discutere di questi aspetti, in modo da poter porre domande specifiche e ricevere risposte dettagliate direttamente dai referenti dell'azienda. Questo approccio ha permesso di ottenere chiarimenti più approfonditi e di comprendere meglio le aspettative dell'azienda riguardo al progetto.
\begin{center}
\small\textit{Nota: Le risposte dell’azienda non sono citazioni dirette, ma sono state riassunte e riformulate dal team in modo da essere più concise.}
\end{center}
}

{\footnotesize
\begin{tabularx}{\textwidth}{|>{\raggedright\arraybackslash}X|>{\raggedright\arraybackslash}X|}
\hline
\textbf{Domande} & \textbf{Risposte} \\
\hline

\parbox[t]{\linewidth}{%
\textbf{Modello di identità e provisioning dei gateway simulati} \\[4pt]
\begin{itemize}
  \item Quale modello di identità volete per i gateway simulati?
  \item Come è previsto che avvenga il provisioning dei gateway nel cloud (flusso tipico)?
  \item Immaginiamo debba simulare il provisioning reale di un gateway fisico; tuttavia, da quanto abbiamo visto, non esiste un’unica modalità. Quale approccio ritenete più in linea con i vostri gateway reali?
  \item Alcuni possibili tipi di provisioning che abbiamo considerato:
  \begin{itemize}
    \item Provisioning manuale (statico)
    \item Provisioning basato su certificati X.509 (autogenerati)
    \item Provisioning Just-In-Time (JIT)
    \item Provisioning Just-Enough-Time (JET)
  \end{itemize}
\end{itemize}
}
& Ci aspettiamo che lo decidiate voi, dopo un attento studio. L'idea è far si che per il PoC questo avventa manualmente attraverso l'interfaccia.\\
\hline

\parbox[t]{\linewidth}{%
\textbf{Simulazione sensori e profili BLE} \\[4pt]
\begin{itemize}
  \item Potete indicarci quali sensori dobbiamo simulare e, di conseguenza, quali profili BLE standard è opportuno utilizzare?
  \item Avete già definito eventuali profili custom?
\end{itemize}
}
& 
Alcuni esempi di sensori sono di: heartrate, temperatura, pressione sanguigna, saturazione ossigeno, ECG, glicemia\\ 
\hline

\parbox[t]{\linewidth}{%
Le tecnologie richieste per lo sviluppo del progetto sono diverse e molte sono completamente nuove per noi: quanto è rilevante per voi la conoscenza pregressa e quali sono le vostre aspettative rispetto al nostro apprendimento progressivo durante il progetto?
}
&
\begin{itemize}
  \item Le tecnologie indicate sono quelle da noi suggerite e utilizzate anche internamente. Questo ci permetterebbe di offrirvi un supporto più puntuale ed efficace. Le stesse tecnologie sono state consigliate anche ai gruppi che hanno lavorato al progetto proposto l’anno scorso, di complessità analoga.
  \item Siamo consapevoli del numero limitato di ore a vostra disposizione e della complessità — sicuramente stimolante — del progetto, ma grazie all’esperienza pregressa sappiamo cosa aspettarci in termini di output e possiamo venirvi incontro.
  \item Ad esempio, non è necessario approfondire GCP o Kubernetes: potete limitarvi a utilizzare Docker in locale o, se lo ritenete opportuno, Minikube.
  \item Un’altra possibilità è quella di utilizzare un’unica tecnologia per lo sviluppo dei vari componenti, sfruttando ad esempio TypeScript — con Node.js per la parte cloud e per il simulatore gateway, e Angular per la dashboard.
  \item Questi dettagli, tuttavia, li definiremo progressivamente, dopo che avrete avuto modo di condurre un adeguato studio preliminare del progetto, come è naturale che sia.
\end{itemize} \\
\hline
\parbox[t]{\linewidth}{%
  Verrà fornito supporto o affiancamento nell'utilizzo di queste tecnologie?
  Potreste indicarci una lista di conoscenze o competenze di base da acquisire per affrontare al meglio il capitolato?
}
&
\parbox[t]{\linewidth}{%
Non è previsto un supporto attivo, come corsi o lezioni dirette, e riteniamo che non sia necessario, poiché parte integrante del progetto consiste proprio nello studio e nella comprensione di queste tecnologie da parte vostra. Per questo motivo non riteniamo indispensabili competenze pregresse. Possiamo tuttavia suggerirvi di esplorare e analizzare soluzioni esistenti già presenti sul mercato, ad esempio nel mondo open source.
Naturalmente resteremo a disposizione per domande e supporto mirato e, se necessario, potremo organizzare incontri dedicati per aiutarvi a chiarire dubbi o risolvere eventuali blocchi.
} \\
\hline
\end{tabularx}
}
Oltre alle domande già discusse, la proponente ha ridefinito gli obiettivi del progetto in seguito a una richiesta esplicita del team. L’azienda ha inoltre sottolineato che lo studio architetturale e documentale riveste per loro un ruolo centrale, mentre l’applicazione rappresenta principalmente il risultato di tale studio; l’interfaccia utente è considerata tecnicamente di minor rilevanza.
\subsubsection{Interesse del team}
\parbox[t]{\linewidth}{%
L'interesse del team verso questo capitolato è stato inizialmente moderato, ma è cresciuto significativamente dopo aver approfondito i dettagli del progetto e aver interagito con l'azienda proponente, M31. \\
Inizialmente il team era attratto dall'idea di lavorare su un sistema distribuito e scalabile, ma aveva delle riserve riguardo alla complessità tecnica e alla mancanza di esperienza con molte delle tecnologie proposte. Tuttavia, dopo aver avuto l'opportunità di porre domande specifiche a M31 e ricevere chiarimenti, il team ha acquisito una maggiore comprensione delle sfide e delle opportunità offerte dal progetto. \\

}
\subsubsection{Punti di forza e di debolezza}
{\footnotesize
\begin{tabularx}{\textwidth}{|X|X|}
\hline
\rowcolor{lightgray!40} % colore intestazione
\textbf{Punti di forza} & \textbf{Punti di debolezza} \\
\hline
\begin{itemize}
\item  Il progetto affronta una sfida attuale e molto concreta nel campo IoT: acquisire, gestire e rendere fruibili dati da sensori in modo sicuro, scalabile e multi-tenant. Questo fornisce un'esperienza su problematiche industriali reali, non solo accademiche
\item Buon supporto da parte dell'azienda; eventualmente anche incontri in sede molto vicini alle nostre sedi di studio.
\item Forte opportunità di apprendimento: il progetto introduce tecnologie e pratiche non ancora esplorate dal team, accelerando la crescita tecnica. Nello specifico, il sistema di acquisizione dati da sensori si basa su un'architettura a microservizi e introduce tecnologie come Node.js e Nest.js con TypeScript (backend) , Go (componenti ad alte prestazioni) , Angular (frontend SPA) , MongoDB (database NoSQL) , PostgreSQL (database SQL) , Redis (caching) , Google Cloud Platform (infrastruttura cloud) , Kubernetes (orchestrazione container) , NATS o Kafka (comunicazione asincrona) , e Prometheus con Grafana (monitoring)
\end{itemize}
 & \begin{itemize}
\item Progettare e implementare un'architettura distribuita, sicura e multi-tenant è una sfida tecnica notevole. La gestione della concorrenza, della consistenza dei dati, della comunicazione asincrona e della tolleranza ai guasti richiede competenze solide e un'attenta progettazione.
\item 	I requisiti sono molti e articolati (dalle API on-demand e streaming, alla UI, al monitoring, alla sicurezza). C'è il concreto rischio di dover sacrificare la profondità di alcune funzionalità per coprirne la quantità entro i tempi del progetto.
\item Testare un'architettura distribuita è intrinsecamente difficile. Configurare ambienti di test end-to-end, simulare guasti e testare la scalabilità richiederà sforzi notevoli e una buona pianificazione.
\item  Il gateway e i dispositivi devono essere simulati. I dati grezzi che vengono generati dai sensori devono essere realistici. Inoltre deve essere posta particolare attenzione alla coerenza (p.es i dati delle temperature devono essere non eccessive e adatte al contesto). Eventualmente i dati devono essere normalizzati. Rispetto al maggiore competitor (Vimar), M31 non prevede utilizzo di veri sensori per aiutare il team nello sviluppo.
\end{itemize} \\
\hline
\end{tabularx}
}
\subsection{C8 - Smart Order}
\subsubsection{Breve descrizione}
\parbox[h]{\linewidth}{%
\textit{Analisi multimodale per la creazione automatica di
ordini}, proposto da Ergon Informatica Srl, prevede la realizzazione di una
piattaforma intelligente in grado di ricevere, interpretare e strutturare
automaticamente ordini provenienti da input testuali, vocali e visivi, trasformando
dati non strutturati in ordini cliente completi pronti per l'inserimento nei sistemi
gestionali aziendali. L'obiettivo del progetto è quello di migliorare l'efficienza dei
processi aziendali, riducendo al minimo l'intervento umano nelle fasi più ripetitive
e soggette a errore.
}

\subsubsection{Caratteristiche funzionali}
Tra i requisiti funzionali del progetto vi sono:
\begin{itemize}
  \item \textbf{Integrazione multimodale completa:} il sistema deve poter ricevere e gestire input di testo, immagini e audio, integrandoli in un unico flusso coerente.
  \item \textbf{Pipeline di elaborazione dati:} implementazione di tutti i layer funzionali descritti nel capitolato.
  \item \textbf{Output strutturato e integrazione gestionale:} gli ordini generati devono essere completi, coerenti e in formato strutturato (ad esempio JSON, XML o tabelle relazionali), pronti per l’inserimento in un sistema ERP.
  \item \textbf{Accuratezza e validazione automatica:} devono essere previsti controlli di coerenza e integrità dei dati, per garantire che gli ordini generati siano affidabili e corretti.
  \item \textbf{Interfaccia e API di comunicazione:} il sistema deve prevedere API REST per l’interazione tra il modello AI e i componenti esterni, in particolare con il database o l’applicativo gestionale.
  \item \textbf{Caso di studio e testing:} il sistema deve essere testato su un caso di studio reale fornito dall’azienda, includendo verifiche di performance e accuratezza.
\end{itemize}

Tra i requisiti non funzionali vi sono:
\begin{itemize}
  \item \textbf{Modularità e scalabilità del sistema:} ogni componente della pipeline deve essere progettato come modulo indipendente, aggiornabile senza impattare l’intero sistema.
  \item \textbf{Logging e feedback continuo:} implementazione di un sistema di monitoraggio e logging che consenta di raccogliere errori, aggiornare le regole aziendali e migliorare le prestazioni tramite retraining periodico.
  \item \textbf{Documentazione tecnica completa:} il sistema deve essere accompagnato da una documentazione chiara, aggiornata e completa di istruzioni per l’uso e la manutenzione.
  \item \textbf{Supporto e replicabilità:} il sistema deve essere facilmente replicabile in ambienti diversi e supportato da istruzioni di installazione e configurazione.
\end{itemize}
\subsubsection{Tecnologie proposte}
\begin{itemize}[noitemsep, topsep=0pt]
  \item \textbf{Database relazionale:} SQL Server Express, MySQL o MariaDB.
  \item \textbf{Modelli di linguaggio e NLP:} BERT, RoBERTa, GPT.
  \item \textbf{Visione computazionale e OCR:} Tesseract OCR, EasyOCR, Convolutional Neural Networks (CNN), Vision Transformer (ViT).
  \item \textbf{Riconoscimento vocale e trascrizione:} Whisper (OpenAI), Google Speech-to-Text.
  \item \textbf{API REST.}
  \item \textbf{Comunicazione da/per il database:} connettori da una fonte dati ODBC o indipendenza del modello LLM dal database.
  \item \textbf{Interfaccia utente:} .NET Blazor, React.js, Angular.
\end{itemize}

\subsubsection{Chiarimenti e colloqui con l'azienda}
Il gruppo non ha mostrato particolare interesse e ha preferito concentrarsi su altri capitolati, per questo motivo non sono stati richiesti chiarimenti alla proponente.
\subsubsection{Interesse del team}
\parbox[t]{\linewidth}{%
l gruppo ha valutato positivamente la modernità e la rilevanza del tema, che
combina più ambiti dell'intelligenza artificiale offrendo l'opportunità di
approfondire tecniche avanzate di Machine Learning e integrazione multimodale.
L'architettura proposta è risultata flessibile e ben strutturata, permettendo l'uso di
tecnologie moderne come LLM, OCR e sistemi speech-to-text, oltre a framework
web come React o Blazor. È stato inoltre ritenuto molto positivo il supporto
previsto da parte dell'azienda proponente e la possibilità di lavorare su un caso di
studio reale fornito da Ergon.
}
\subsubsection{Punti di forza e di debolezza}

{\footnotesize
\begin{tabularx}{\textwidth}{|X|X|}
\hline
\rowcolor{lightgray!40} % colore intestazione
\textbf{Punti di forza} & \textbf{Punti di debolezza} \\
\hline
\begin{itemize}
\item L'integrazione di NLP, Computer Vision e Speech-to-Text in un'unica pipeline è un campo di ricerca e sviluppo estremamente attuale e complesso. Offre un'esperienza di apprendimento su tecnologie AI d'avanguardia.
\item Viene fornito un ricco elenco di tecnologie suggerite per ogni componente (LLM, OCR, Speech-to-Text, UI), ma con la libertà di scegliere alternative. Questo permette al team di adattare lo stack tecnologico alle proprie competenze.
\end{itemize}
 & \begin{itemize}
\item Integrare e far collaborare modelli di AI diversi (LLM, Vision, Audio) in una pipeline coerente è una sfida tecnica notevolissima.
\item Il team dovrebbe padroneggiare o imparare rapidamente una vasta gamma di tecnologie AI e di framework, ciascuna con le sue complessità.
\item Il pre-processing di dati multimodali (pulizia del testo, elaborazione di immagini, trascrizione audio) è un compito laborioso e critico. La qualità dell'output finale dipende fortemente da questa fase.
\item I modelli di AI, specialmente gli LLM, possono essere imprevedibili. Validare l'accuratezza e l'affidabilità del sistema in scenari reali, dove un errore può significare un ordine sbagliato, è una sfida significativa.
\end{itemize} \\
\hline
\end{tabularx}
}


\newpage
\subsection{C9 - View4Life}
\subsubsection{Breve descrizione}
\parbox[t]{\linewidth}{%
Il capitolato proposto da Vimar S.p.A. ha come obiettivo la realizzazione di una piattaforma per la gestione intelligente di impianti domotici. all'interno di residenze protette per anziani, con l'intento di migliorare la sicurezza, il comfort e l'efficienza energetica degli ambienti. Il progetto prevede lo sviluppo di un'infrastruttura cloud e di un'applicazione web responsive dedicata al personale sanitario per il monitoraggio e il controllo dei dispositivi
}
\subsubsection{Caratteristiche funzionali}
Tra i requisiti funzionali vi sono:
\begin{itemize}
    \item L’applicativo web deve essere \textbf{responsive}, adattandosi a tutti i dispositivi.
    \item Deve possedere un \textbf{design semplice ed efficace}.
    \item L’applicativo deve \textbf{interfacciarsi con uno o più impianti Vimar View Wireless}.
    \item Deve \textbf{utilizzare l’interfaccia API KNX IoT} fornita da Vimar per comunicare con gli impianti.
    \item Deve includere un \textbf{sistema di accesso utente} dedicato al personale sanitario.
    \item Deve prevedere un \textbf{sistema di gestione degli allarmi} per gli operatori sanitari, ad esempio:
    \begin{itemize}
        \item Allarme per \textbf{rilevamento cadute}.
        \item Allarme per \textbf{accesso indesiderato a un’area in una certa fascia oraria}.
        \item Allarme \textbf{azionato manualmente dall’utente}.
    \end{itemize}
    \item Deve avere una \textbf{sezione dedicata ai dispositivi dell’impianto}, mostrando le informazioni relative a ciascuno.
    \item Deve includere una \textbf{sezione ``Analytics''} che presenti:
    \begin{itemize}
        \item \textbf{Statistiche della piattaforma e dell’impianto} sotto forma di grafici.
        \item \textbf{Suggerimenti per la riduzione dei consumi energetici}, basati sui dati raccolti dai sensori.
    \end{itemize}
    \item Deve fornire un \textbf{cruscotto informativo (dashboard)} per l’utente.
\end{itemize}

Tra i requisiti non funzionali vi sono:
\begin{itemize}
    \item L’applicativo deve essere \textbf{semplice, efficace e accessibile da qualsiasi dispositivo}.
    \item L’infrastruttura cloud deve essere \textbf{basata su container}.
    \item L’infrastruttura deve essere progettata secondo il principio di \textbf{Infrastructure as Code (IaC)}, evitando configurazioni manuali e favorendo l’uso di file di configurazione versionati.
    \item L’infrastruttura deve essere \textbf{replicabile, tracciabile e facilmente manutenibile}.
    \item L’intera infrastruttura deve essere \textbf{realizzata su Amazon Web Services (AWS)}.
\end{itemize}

\subsubsection{Tecnologie proposte}
\begin{itemize}
  \item \textbf{Infrastruttura Cloud:} deve utilizzare \textbf{Docker} con \textbf{docker-compose}, in modo da rispettare il principio di \textit{Infrastructure as Code}.
  \item \textbf{Applicativo web:} deve prevedere l’utilizzo della tecnologia \textbf{KNX IoT 3rd Party API}.
  \item \textbf{Autenticazione:} l’interfaccia KNX IoT richiede l’uso del protocollo \textbf{OAuth2} per la gestione sicura dell’autenticazione.
  \item \textbf{Notifiche di impianto:} l’applicativo deve utilizzare il meccanismo di ricezione delle notifiche \textit{push} previsto dallo standard KNX IoT.  
  Meccanismi di aggiornamento periodico tramite \textit{polling} non sono ammessi.
  \item \textbf{Versionamento:} il repository di lavoro deve essere gestito con \textbf{Git} e reso \textbf{pubblicamente accessibile}.  
  I sorgenti dovranno essere rilasciati con \textbf{licenza open source}.
  
  \item \textbf{Compatibilità delle licenze:} il gruppo di lavoro deve prestare attenzione alla compatibilità delle licenze open-source adottate rispetto alle librerie o ai framework utilizzati.
\end{itemize}

\subsubsection{Chiarimenti e colloqui con l'azienda}
{\footnotesize
\begin{tabularx}{\textwidth}{|>{\raggedright\arraybackslash}X|>{\raggedright\arraybackslash}X|}
\hline
\textbf{Domande} & \textbf{Risposte} \\
\hline
Abbiamo compreso che View4Life nasce come un dimostratore per mostrare quanto sia semplice e veloce realizzare un’integrazione con le tecnologie pubblicizzate da Vimar. Il contesto presentato, però, suggerisce uno scenario molto realistico e vicino a un possibile bisogno di mercato. Questo progetto deriva da richieste concrete o rimane principalmente un proof-of-concept dimostrativo? E, nel caso, pensate che possa avere in futuro uno spazio effettivo all’interno di Vimar? 
&
\begin{itemize}
\item Il contesto presentato ha in realtà una duplice valenza, arrivata specialmente negli ultimi mesi.
\item Dimostratore per integrazioni 3rd party via Cloud: mostrare che si possono controllare i dispositivi Vimar View Wireless via Cloud con API REST.
\item Dimostratore per esigenza di mercato: esplorare cosa potrebbe essere utile in base alle richieste commerciali.
\item Se l’output del progetto farà l’effetto WOW, ci sarà uno spazio effettivo; altrimenti sarà comunque pubblico con il vostro nome sopra.
\end{itemize} \\
\hline
Quanto è rilevante per voi la conoscenza pregressa delle tecnologie richieste, o è previsto che ci sia spazio per apprenderle progressivamente durante il progetto?
&
\begin{itemize}
\item La conoscenza pregressa non è quasi mai prevista: ci sarà tempo per imparare insieme.
\item La curva di apprendimento seguirà l’effetto Dunning-Kruger.
\item Sessioni di approfondimento saranno organizzate durante il progetto, ad esempio su Docker e KNX IoT.
\item Framework di sviluppo e altri strumenti saranno affrontati secondo le esigenze del team.
\end{itemize} \\
\hline
È preferibile sviluppare parte di front-end, back-end (ed eventuali microservizi) singolarmente e in maniera seriale, oppure sviluppare tutto in parallelo?
&
\begin{itemize}
\item Generalmente conviene il parallelismo per ridurre il tempo totale.
\item La strategia seriale può essere una buona alternativa, considerando i ruoli da ruotare.
\item Dal front-end si può lavorare sul mocking delle API prima che siano implementate.
\item L’approccio finale sarà discusso in base allo skillset del gruppo.
\end{itemize} \\
\hline
La documentazione per integratori di terze parti per interfaccia KNX IoT richiede più o meno quanta mole di lavoro per poterla capire e padroneggiare? O comunque il livello di difficoltà/ quantità di documentazione è molto consistente?
&
\begin{itemize}
\item La documentazione è approfondita, ma i concetti base devono essere padroneggiati bene.
\item Verrà organizzata una sessione di presentazione di KNX IoT e dei dispositivi via API.
\item Saranno forniti mini Jupiter Notebook o Postman per capire i primi passi.
\end{itemize} \\
\hline
Si sviluppa prima dall’idea o in maniera test-driven? Un requisito opzionale riguarda la possibilità di inserire una mappa degli appartamenti. L’azienda aveva pensato alla modalità di visualizzazione della mappa? Come viene aggiunta? È interattiva? (p.es cliccare i dispositivi, zoommare etc)
&
\begin{itemize}
\item Non esiste un giusto o sbagliato: dipende dallo stile del team.
\item Generalmente si parte dalla progettazione, poi implementazione test-driven o modulare.
\item La mappa degli appartamenti è opzionale: può essere un rendering basico con framework tipo Floorspace.js o una foto con box cliccabili.
\item La mappa non deve essere modificabile, ma mostrare la possibilità di utilizzo.
\end{itemize} \\
\hline
\end{tabularx}
}

\subsubsection{Interesse del team}
\parbox[t]{\linewidth}{%
Il gruppo ha valutato molto positivamente questo capitolato per la concretezza e
l'impatto sociale: lavorare su un sistema capace di migliorare la qualità della vita e
la sicurezza delle persone rappresenta una forte motivazione, dando al progetto un
valore tangibile e significativo. Sono stati inoltre apprezzati l'utilizzo di tecnologie
moderne e versatili, come Docker, AWS e framework web come Angular o React,
che offrono ampie opportunità di apprendimento e applicazione pratica. Un
ulteriore punto di forza è rappresentato dal supporto fornito da Vimar, che
garantisce incontri di avanzamento regolari e fornisce un kit hardware fisico per
testare l'integrazione con dispositivi reali, riducendo il divario tra teoria e pratica e
permettendo di affrontare concretamente le sfide dell'IoT.
\\
La principale complessità del progetto risiede nella sua ampiezza e completezza,
che richiede una pianificazione accurata e una buona capacità di priorizzazione.
Tuttavia, il gruppo ha interpretato questa caratteristica non come una difficoltà, ma
come un'opportunità di crescita: gestire un progetto così articolato permetterà
infatti di sviluppare competenze di organizzazione, coordinamento e gestione
tecnica avanzata
}

\subsubsection{Punti di forza e di debolezza}
{\footnotesize
\begin{tabularx}{\textwidth}{|X|X|}
\hline
\rowcolor{lightgray!40} % colore intestazione
\textbf{Punti di forza} & \textbf{Punti di debolezza} \\
\hline
\begin{itemize}
\item Dominio applicativo concreto e socialmente utile: Il contesto delle "residenze protette" fornisce uno scopo nobile e tangibile. Sviluppare un sistema che può migliorare la sicurezza e il benessere delle persone è un forte motivatore e rende il progetto molto più significativo rispetto a un dominio astratto.
\item Tecnologie moderne e ricercate
\item Supporto aziendale eccezionale e materiale fornito: Vimar fornisce un supporto strutturato con incontri bisettimanali/settimanali (SAL) e, aspetto cruciale, fornisce un kit hardware fisico per testare con dispositivi reali. Questo riduce enormemente il gap tra teoria e pratica e permette di affrontare problematiche reali di integrazione IoT.
\item Opportunità reale che l'MVP venga effettivamente adottato da Vimar, qualora raggiungesse la qualità necessaria
\end{itemize}
 & \begin{itemize}
\item Elevata complessità e ampio scope: La completezza del progetto è anche la sua principale sfida. Il team deve essere bravo a gestire la complessità e a priorizzare le funzionalità obbligatorie, per non perdersi negli optional.
\end{itemize} \\
\hline
\end{tabularx}
}


\end{document}