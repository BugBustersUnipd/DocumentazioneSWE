% Macro per i verbali del gruppo BugBusters
% Macro utili consigliate per la stesura dei verbali
% Queste macro sono pensate per essere piccole e riutilizzabili all'interno dei
% template dei verbali (es. per la tabella meta, per le righe della tabella decisioni,
% per la lista dei presenti e per il logo con fallback).


% genera la stringa "\noindent (Riferimento alla tabella decisioni:
% \hyperref[RTB4]{RTB4})", si usa facendo \refDecisione{NomeLabel}{Testo}
\newcommand{\refDecisione}[2]{%
    \noindent (\textbf{Riferimento alla tabella decisioni: \hyperref[#1]{#2}})%
}


% 1) Campi della tabella meta (usare dentro un tabular con due colonne)
% Esempio d'uso:
% \begin{tabular}{@{}ll@{}}
%   \VBField{Redattori}{Linor Sadè} \\
%   \VBField{Verificatore}{Marco Piro} \\
% \end{tabular}
\newcommand{\VBField}[2]{\textbf{#1} & #2 \\\relax}
\newcommand{\VBRedattori}[1]{\VBField{Redattori}{#1}}
\newcommand{\VBVerificatore}[1]{\VBField{Verificatore}{#1}}
\newcommand{\VBUso}[1]{\VBField{Uso}{#1}}
\newcommand{\VBDestinatari}[1]{\VBField{Destinatari}{#1}}
\newcommand{\VBVersione}[1]{\VBField{Versione}{#1}}

% 2) Ambiente per la lista dei presenti coerente con lo stile dei verbali
% Esempio d'uso: \begin{VBPresenti} \item Nome 1 \item Nome 2 \end{VBPresenti}
\newenvironment{VBPresenti}{%
    \begin{itemize}[leftmargin=1.5em, itemsep=3pt, label={\rule[0.5ex]{0.4em}{0.4em}}]
}{%
    \end{itemize}
}

% 3) Righe della tabella decisioni/azioni (3 colonne: ID, Descrizione, Incaricato)
% Lasciare il formato della prima cella libero per permettere l'uso di \href
% o label personalizzate. Esempio:
%   \DecisionRow{\href{...}{RTB1} \label{RTB1}}{Descrizione}{Linor Sadè}
% Nota: imposto anche il colore della regola orizzontale per garantire che
% le linee della tabella siano nel colore 'primaryblue' e che la riga colori
% correttamente anche quando usata all'interno di 'tabularx'.
\newcommand{\DecisionRow}[3]{%
    \rowcolor{secondaryblue!10} #1 & #2 & #3 \\
    \arrayrulecolor{primaryblue}\hline
}

% 4) Logo con fallback (accetta percorso e larghezza/altezza come stringhe, es.: 6cm,3cm)
% Esempio: \VBLogo{../../../../assets/Logo.jpg}{6cm}{3cm}
\newcommand{\VBLogo}[3]{%
    \IfFileExists{#1}{%
        \includegraphics[width=#2,height=#3,keepaspectratio]{#1}%
    }{%
        \fbox{\parbox[c][#3][c]{#2}{\centering Logo non trovato\\(#1)}}%
    }%
}