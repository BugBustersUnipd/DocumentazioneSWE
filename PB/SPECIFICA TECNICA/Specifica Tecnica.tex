\documentclass[a4paper,11pt]{article}
   
\newcommand{\CurrentVersion}{0.0.1} % ultima versione, da cambiare ad ogni push significativo

\usepackage[utf8]{inputenc}
\usepackage[T1]{fontenc}
\usepackage[italian]{babel}
\usepackage[margin=2.5cm]{geometry}
\usepackage{graphicx}
\usepackage{grffile}
\usepackage{booktabs}
\usepackage{setspace}
\usepackage{titlesec}
\usepackage{float}
\usepackage{ifthen}
\usepackage[table]{xcolor}
\usepackage{tabularx}
\usepackage{needspace}
\usepackage{tcolorbox}
\usepackage{enumitem}
\usepackage[titles]{tocloft}
\usepackage[colorlinks=true,linkcolor=black,urlcolor=blue,citecolor=blue]{hyperref}
\usepackage{xspace}
% Macro
% genera la stringa "\noindent (Riferimento alla tabella decisioni:
% \hyperref[RTB4]{RTB4})", si usa facendo \refDecisione{NomeLabel}{Testo}
\newcommand{\refDecisione}[2]{%
    \noindent (\textbf{Riferimento alla tabella decisioni: \hyperref[#1]{#2}})%
}


\definecolor{primaryblue}{RGB}{0,102,204}
\definecolor{secondaryblue}{RGB}{51,153,255}
\definecolor{lightgray}{RGB}{245,245,245}
\definecolor{darkgray}{RGB}{100,100,100}

\titleformat{\section}
 {\Large\bfseries\color{primaryblue}}
 {\thesection}{1em}{}

\titleformat{\subsection}
 {\large\bfseries\color{primaryblue}} % Sottosezione: colore secondaryblue
 {\thesubsection}{1em}{}

\titleformat{\subsubsection}
 {\normalsize\bfseries\color{secondaryblue}} % Sotto-sottosezione: colore secondaryblue
 {\thesubsubsection}{1em}{}

% per il footer con il numero di pagina
\usepackage{fancyhdr}
\usepackage{lastpage} % per ottenere il numero dell'ultima pagina da mettere nel footer

\usepackage{ltablex} %per far andare a capo le tabelle
\keepXColumns

\renewcommand{\sectionmark}[1]{\markright{#1}}
\renewcommand{\G}{\textsubscript{\scalebox{0.6}{\textbf{G}}}\ }

% Macro tabella metrica (tabella uniforme, non spezzata e con stile coerente)
\newcommand{\MetricTable}[1]{
  \Needspace{10\baselineskip}% evita spezzamenti sgradevoli
  \begin{table}[H]
  \centering
  \arrayrulecolor{primaryblue}
  \begin{tabularx}{\linewidth}{|>{\raggedright\arraybackslash}p{2.5cm}|X|>{\raggedright\arraybackslash}p{3cm}|>{\raggedright\arraybackslash}p{3cm}|}
  \hline
  \rowcolor{primaryblue!40}
      \textbf{\color{white} Metrica} & \textbf{\color{white} Descrizione} & \textbf{\color{white} Valore accettazione} & \textbf{\color{white} Valore ideale} \\
  \hline
  \ignorespaces
  #1
  \end{tabularx}
  \end{table}
}



\setlength{\parskip}{4pt}
\setlength{\parindent}{0pt}

\setlist[itemize]{leftmargin=*,itemsep=3pt}
\setlist[enumerate]{leftmargin=*,itemsep=3pt}

\graphicspath{{./}{../assets/images/}{./images/}}

\begin{document}

%configurazione per il footer
\pagestyle{fancy}
\fancyhf{} % pulisce tutti i campi del header e footer

% Header: sinistra e destra
\fancyhead[L]{Gruppo 4 - BugBusters} % sinistra
\fancyhead[R]{Specifica Tecnica\G}   % destra

\fancyfoot[L]{ \thepage\ di \pageref{LastPage}} %definisce il formato del footer
\fancyfoot[R]{ \nouppercase{\rightmark}} % nome della sezione


\renewcommand{\headrulewidth}{0pt} % rimuove la linea dell'header
\renewcommand{\footrulewidth}{0pt} % se vuoi anche togliere eventuale linea del footer

% abilitare numerazione e TOC fino al livello "paragraph" (subsubsubsection)
\setcounter{secnumdepth}{4}
\setcounter{tocdepth}{4}
% formattazione del nuovo livello per avere aspetto coerente
\titleformat{\paragraph}[block]{\normalsize\bfseries\color{secondaryblue}}{\theparagraph}{1em}{}
% alias comodo per usare "subsubsubsection"
\newcommand{\subsubsubsection}{\paragraph}

\begin{center}
  \IfFileExists{../../assets/Logo.jpg}{%
    \includegraphics[width=6cm,height=3cm,keepaspectratio]{../../assets/Logo.jpg} \\[0.8cm]
  }{%
    \fbox{\parbox[c][2.5cm][c]{6cm}{\centering Logo non trovato\\(../../assets/Logo.jpg)}}\\[0.5cm]
  }
  
  {\Large\bfseries\color{primaryblue} BugBusters}\\[0.5cm]

  {\Huge\bfseries\color{primaryblue} Specifia Tecnica\G}\\[0.3cm]
  {\Large\color{secondaryblue} Versione \CurrentVersion}\\[0.8cm]
\end{center}

\begin{center}
\begin{tcolorbox}[colback=lightgray,colframe=primaryblue,width=0.85\textwidth,arc=3mm,boxrule=0.5pt]
% Usa tabularx con una colonna fissa per l'etichetta e una colonna X per il contenuto
\begin{tabularx}{\linewidth}{@{}>{\raggedright\arraybackslash}p{3.5cm}>{\raggedright\arraybackslash}X@{}}
  {Stato} & In Redazione \\
  {Redattori\G} & Alberto Pignat \\
  {Verificatori\G} &  \\
  {Distribuzione} & BugBusters, Eggon, Prof. Tullio Vardanega, Prof. Riccardo Cardin\\
\end{tabularx}
\end{tcolorbox}
\end{center}

\vspace{0.5cm}

\begin{center}
\begin{tcolorbox}[colback=secondaryblue!10,colframe=secondaryblue,width=0.9\textwidth,arc=3mm,boxrule=0.8pt,title={\bfseries Descrizione}]
Documento dove viene spiegata in dettaglio la struttura architetturale, le tecnologie utilizzate e le API esposte del prodotto software.\end{tcolorbox}
\end{center}

\newpage

% Registro delle Modifiche (pagina 2)
\section*{Registro delle Modifiche}

\arrayrulecolor{primaryblue}
{\footnotesize
\begin{tabularx}{\textwidth}{|>{\raggedright\arraybackslash}p{1.5cm}|>{\raggedright\arraybackslash}p{2cm}|X|>{\raggedright\arraybackslash}p{2cm}|>{\raggedright\arraybackslash}p{2cm}|>{\raggedright\arraybackslash}p{2cm}|}
\hline
\rowcolor{primaryblue!40}
\textbf{\color{white} Versione} & \textbf{\color{white} Data} & \textbf{\color{white} Descrizione} & \textbf{\color{white} Redatto} & \textbf{\color{white} Verificato} & \textbf{\color{white} Approvato} \\
\rowcolor{secondaryblue!10}0.0.1 & 24/02/2026 & Prima stesura del documento & Alberto Pignat & - & - \\
\hline
\end{tabularx}
}

\newpage

% Indice cliccabile
\setcounter{tocdepth}{3} % Mostra fino al livello di sottosezione (2)
\tableofcontents
\listoftables
\listoffigures

\newpage
\section{Introduzione}
Il documento ha come scopo la descrizione dettagliata dell'architettura del sistema, delle tecnologie utilizzate e delle API esposte. Viene inoltre fornita una panoramica dei requisiti funzionali soddisfatti dal prodotto software.
\subsection{Glossario}
Il glossario\G raccoglie e definisce i termini, gli acronimi e le abbreviazioni 
impiegati nel documento e nel progetto\G Nexum. L'obiettivo è fornire definizioni univoche per ridurre 
ambiguità, garantire coerenza terminologica tra i membri del team e facilitare l'onboarding di nuovi partecipanti. \\
Per i termini tecnici e specifici utilizzati in questo documento, si fa riferimento al glossario\G disponibile al 
seguente \href{https://bugbustersunipd.github.io/DocumentazioneSWE/RTB/GLOSSARIO/Glossario.pdf}{link}.
Per maggiore usabilità e facilitá di consultazione, il glossario\G è accessibile anche aprendo dal nostro sito web i vari documenti con il viewer
pdf da noi sviluppato.
\subsubsection{Riferimenti normativi}
\begin{itemize}
\item \textbf{Capitolato\G
 d'appalto C5: Nexum - Piattaforma di consulenza e documentazione previdenziale}\\
\url{https://www.math.unipd.it/~tullio/IS-1/2025/Progetto/C5.pdf}
\end{itemize}

\section{Tecnologie}
Questa sezione descrive le tecnologie utilizzate nello sviluppo del prodotto software, inclusi i linguaggi, i framework, le librerie e gli strumenti di sviluppo adottati dal team.
\subsection{Linguaggi}
TODO METTERE TABELLA CON  NOME DESCRIZIONE  E VERSIONE
\begin{longtable}{|p{2cm}|p{7cm}|p{3cm}|p{2.5cm}|}
\caption{Tabella dei Linguaggi}\label{tab:linguaggi}\\
\hline
\rowcolor{primaryblue!40}
\textbf{Nome} & \textbf{Descrizione} & \textbf{Versione} \\
\hline
\endfirsthead

\multicolumn{4}{c}{{\bfseries \tablename\ \thetable{} -- continua dalla pagina precedente}}\\
\hline
\rowcolor{primaryblue!40}
\textbf{Nome} & \textbf{Descrizione} & \textbf{Versione} \\
\hline
\endhead

\hline
\multicolumn{4}{|r|}{{Continua nella prossima pagina...}}\\
\hline
\endfoot

\hline
\endlastfoot
% Inserire qui le righe della tabella dei linguaggi
Typescript & Linguaggio di programmazione basato su JavaScript che aggiunge la tipizzazione statica &  \\
\hline
HTML & Linguaggio di markup usato per creare pagine web &  \\
\hline
CSS & Linguaggio usato per definire lo stile delle pagine web  &  \\
\hline
Ruby & Linguaggio di programmazione orientato agli oggetti &  \\
\hline
SQL & Linguaggio per la gestione e manipolazione di database relazionali &  \\
\end{longtable}
\subsection{Framework}
TODO METTERE TABELLA CON  NOME DESCRIZIONE  E VERSIONE
\begin{longtable}{|p{2cm}|p{7cm}|p{3cm}|p{2.5cm}|}
\caption{Tabella dei Framework}\label{tab:framework}\\
\hline
\rowcolor{primaryblue!40}
\textbf{Nome} & \textbf{Descrizione} & \textbf{Versione} \\
\hline
\endfirsthead

\multicolumn{4}{c}{{\bfseries \tablename\ \thetable{} -- continua dalla pagina precedente}}\\
\hline
\rowcolor{primaryblue!40}
\textbf{Nome} & \textbf{Descrizione} & \textbf{Versione} \\
\hline
\endhead

\hline
\multicolumn{4}{|r|}{{Continua nella prossima pagina...}}\\
\hline
\endfoot

\hline
\endlastfoot
% Inserire qui le righe della tabella dei linguaggi
Angular & Framework open source per la creazione e sviluppo di single page applications. & v21 \\
\hline
Ruby on Rails & Framework open source server side per applicazioni web &  \\
\end{longtable}
\subsection{Librerie}
TODO METTERE TABELLA CON  NOME DESCRIZIONE  E VERSIONE 
\subsection{Strumenti di Sviluppo}
TODO METTERE TABELLA CON  NOME DESCRIZIONE  E VERSIONE
\begin{longtable}{|p{2cm}|p{7cm}|p{3cm}|p{2.5cm}|}
\caption{Tabella degli Strumenti di Sviluppo}\label{tab:strumenti_sviluppo}\\
\hline
\rowcolor{primaryblue!40}
\textbf{Nome} & \textbf{Descrizione} & \textbf{Versione} \\
\hline
\endfirsthead

\multicolumn{4}{c}{{\bfseries \tablename\ \thetable{} -- continua dalla pagina precedente}}\\
\hline
\rowcolor{primaryblue!40}
\textbf{Nome} & \textbf{Descrizione} & \textbf{Versione} \\
\hline
\endhead

\hline
\multicolumn{4}{|r|}{{Continua nella prossima pagina...}}\\
\hline
\endfoot

\hline
\endlastfoot
% Inserire qui le righe della tabella dei linguaggi
Visual Studio Code & IDE open source per lo sviluppo software sviluppato da Microsoft &  \\
\hline
Docker & Piattaforma che permette di creare, distribuire e eseguire applicazioni in container &   \\
\hline
PostGreSQL & Database relazionale open source &  \\
\hline
pgAdmin & Strumento grafico per amministrare e gestire database PostGreSQL &  \\
\end{longtable}

\section{API}
\section{Architettura del Sistema}
\subsection{Architettura Logica}
\subsection{Architettura di Deployement}
\subsection{Design Pattern}
\subsection{Backend}
\subsection{Frontend}
\subsection{Database}
\section{Requisiti Funzionali Soddisfatti}
\end{document}