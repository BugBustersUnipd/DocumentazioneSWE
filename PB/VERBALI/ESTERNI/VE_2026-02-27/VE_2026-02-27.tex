\documentclass[a4paper,12pt]{article}

\usepackage[utf8]{inputenc}
\usepackage[T1]{fontenc}
\usepackage[italian]{babel}
\usepackage[margin=2.5cm]{geometry}
\usepackage{graphicx}
\usepackage{grffile}
\usepackage{booktabs}
\usepackage{setspace}
\usepackage{titlesec}
\usepackage{float}
\usepackage{ifthen}
\usepackage{tcolorbox}
\usepackage{enumitem}
\usepackage[colorlinks=true,linkcolor=black,urlcolor=primaryblue,citecolor=primaryblue]{hyperref}
\usepackage{changepage} % per definire un rientro (un margine) solo in una parte del documento

\usepackage[table]{xcolor}   % per i colori (celle, testo e righe)
\usepackage{tabularx}               % per tabelle a larghezza adattiva (X)
\usepackage{array}                  % per comandi di formattazione avanzata nelle colonne (p, >{\raggedright})

% Macro
% genera la stringa "\noindent (Riferimento alla tabella decisioni:
% \hyperref[RTB4]{RTB4})", si usa facendo \refDecisione{NomeLabel}{Testo}
\newcommand{\refDecisione}[2]{%
    \noindent (\textbf{Riferimento alla tabella decisioni: \hyperref[#1]{#2}})%
}




\definecolor{primaryblue}{RGB}{0,102,204}
\definecolor{secondaryblue}{RGB}{51,153,255}
\definecolor{lightgray}{RGB}{245,245,245}
\definecolor{darkgray}{RGB}{100,100,100}

\titleformat{\section}
  {\Large\bfseries\color{primaryblue}}
  {\thesection}{1em}{}

\setlength{\parskip}{4pt}
\setlength{\parindent}{0pt}

\setlist[itemize]{leftmargin=*,itemsep=3pt}
\setlist[enumerate]{leftmargin=*,itemsep=3pt}
\graphicspath{{./}{../assets/images/}{./images/}}

\begin{document}

\begin{center}  \IfFileExists{../../../../assets/Logo.jpg}{%
    \includegraphics[width=6cm,height=3cm,keepaspectratio]{../../../../assets/Logo.jpg} \\[0.8cm]
  }{%
    \fbox{\parbox[c][2.5cm][c]{6cm}{\centering Logo non trovato\\(Logo.jpg)}}\\[0.5cm]
  }
  
  {\Large\bfseries\color{primaryblue} BugBusters}\\[0.3cm]
  {\small\color{darkgray} Email: \texttt{bugbusters.unipd@gmail.com}} \\[0.1cm]
  {\small\color{darkgray} Gruppo: 4} \\[0.5cm]

  {\large\bfseries Università degli Studi di Padova}\\[0.3cm]
  {\small Laurea in Informatica}\\[0.2cm]
  {\small Corso: Ingegneria del Software}\\[0.2cm]
  {\small Anno Accademico: 2025/2026}\\[0.8cm]

  {\Huge\bfseries\color{primaryblue} Verbale Esterno\G{}}\\[0.3cm]
  {\Large\color{secondaryblue} 27 febbraio 2026}\\[0.8cm]
\end{center}

\begin{center}
\begin{tcolorbox}[colback=lightgray,colframe=primaryblue,width=0.85\textwidth,arc=3mm,boxrule=0.5pt]
\begin{tabularx}{\linewidth}{@{}lX@{}}
\textbf{Redattore\G}    & Marco Favero \\
\textbf{Verificatore\G{}}    & Linor Sadè \\
\textbf{Uso}          & Esterno \\
\textbf{Destinatari}  & Prof. Tullio Vardanega, Prof. Riccardo Cardin, Eggon, BugBusters\\
\textbf{Versione} & 1.0.0\\

\end{tabularx}
\end{tcolorbox}
\end{center}

\vspace{0.5cm}

\begin{center}
\begin{tcolorbox}[colback=secondaryblue!10,colframe=secondaryblue,width=0.9\textwidth,arc=3mm,boxrule=0.8pt,title={\bfseries Abstract}]
Questo verbale documenta la riunione esterna del 27 febbraio 2026 con Nexum, durante la quale sono stati discussi aspetti tecnici, architetturali e strategici del progetto. 
Il team ha presentato i mock-up dell'applicazione con funzionalità di ricerca avanzata, generazione personalizzata tramite AICopilot e gestione documentale. 
Sono stati affrontati argomenti riguardanti le scelte tecnologiche, la strategia di sviluppo e fatte ulteriori domande sulla modellazione dell'applicazione.
\end{tcolorbox}
\end{center}

\newpage

\tableofcontents
\newpage

\section{Informazioni generali}

\begin{itemize}
    \item \textbf{Tipo riunione:} Esterna
    \item \textbf{Piattaforma:} Google Meet
    \item \textbf{Data:} 27/02/2026
    \item \textbf{Orario di inizio:} 15:00
    \item \textbf{Orario di fine:} 15:45
    \item \textbf{Presenti:}
    \begin{itemize}[leftmargin=1.5em, itemsep=3pt, label={\rule[0.5ex]{0.4em}{0.4em}}]
        \item Alberto Autiero
        \item Marco Favero
        \item Alberto Pignat
        \item Marco Piro
        \item Linor Sadè
        \item Leonardo Salviato
        \item Luca Slongo
    \end{itemize}
    \item \textbf{Assenti:} 
    \begin{itemize}[leftmargin=1.5em, itemsep=3pt, label={\rule[0.5ex]{0.4em}{0.4em}}]
    \item Nessuno
    \end{itemize}
    \item \textbf{Presenti Esterni:}
    \begin{itemize}[leftmargin=1.5em, itemsep=3pt, label={\rule[0.5ex]{0.4em}{0.4em}}]
      \item Gianpaolo Ferrarin
      \item Luca Iuzzolino
    \end{itemize}
\end{itemize}

\section{Ordine del giorno}

\begin{enumerate}
    \item \label{itm:mockup} Presentazione e revisione dei mock-up dell'applicazione.
    \item \label{itm:arch} Scelte tecnologiche e architettura del progetto.
    \item \label{itm:dev} Strategia di sviluppo e modellazione del database.
\end{enumerate}

\newpage
\section{Svolgimento}

\subsection*{\ref{itm:mockup}. Presentazione e revisione dei mock-up dell'applicazione}

Il team ha presentato i mock-up dell'applicazione standalone con le seguenti funzionalità:
\begin{itemize}
    \item \textbf{Interfaccia AIAssistant:} Ricerca avanzata con impostazione tono, stile e prompt; visualizzazione risultati con storico; 
    ricerca testuale su tutti i campi; editor modificabile per correzioni post-generazione.
    \item \textbf{Gestione documenti AICopilot:} Upload di PDF, CSV, JPEG con metadati; suddivisione documenti per destinatario con 
    stato validazione; preview, modifica estratti e gestione template invio.
\end{itemize}

\subsection*{\ref{itm:arch}. Scelte tecnologiche e architettura del progetto}

Si è parlato di architettura modulare con API Ruby on Rails in modalità API-only e 
front-end separato (Angular). Inoltre si è discusso dell'uso di docker, sconsigliato per la complessità,
e della possibilità di simulazione dei servizi Bucket S3 e gestione mail tramite Mailpit e MiniO.
Inoltre Luca Iuzzolino ha suggerito di seguire le best practice di ogni linguaggio, evitando di forzare pattern 
architetturali non nativi.


\subsection*{\ref{itm:dev}. Strategia di sviluppo e modellazione del database}
Ci è stato consigliato il seguente approccio:
\begin{itemize}
    \item \textbf{Priorità:} Partire dalla modellazione del database come base strutturale.
    \item \textbf{Approccio iterativo:} Prototipazione rapida per testare funzionalità chiave, seguita da raffinazione e documentazione continua.
    \item \textbf{Servizi specializzati:} Sperimentare soluzioni per container di iniezione dipendenze e separazione logiche (OCR, generazione immagini), bilanciando complessità e benefici.
\end{itemize}



\section{Tabella delle decisioni e azioni}
\renewcommand{\arraystretch}{1.5} 


\arrayrulecolor{primaryblue}
\sloppy
\begin{tabularx}{\textwidth}{
    |>{\raggedright\arraybackslash}p{3cm}|
    >{\raggedright\arraybackslash}X|
    >{\raggedright\arraybackslash}p{3cm}|
}
\hline
\rowcolor{primaryblue!40}
\textbf{\color{white} ID Decisione} & \textbf{\color{white} Descrizione} & \textbf{\color{white} Incaricato} \\
\hline
\DecisionRow{\href{https://github.com/BugBustersUnipd/DocumentazioneSWE/issues/187}{PB10}\label{PB10}}{Stesura del verbale}{Marco Favero} 
\DecisionRow{\href{https://github.com/BugBustersUnipd/DocumentazioneSWE/issues/188}{PB11}\label{PB11}}{Verifica del verbale}{Linor Sadè} 
\DecisionRow{\href{https://github.com/BugBustersUnipd/DocumentazioneSWE/issues/189}{PB12}\label{PB12}}{Creazione ER del DB e studio architettura}{Alberto Autiero, Marco Favero, Leonardo Salviato, Luca Slongo}
\DecisionRow{\href{https://github.com/BugBustersUnipd/DocumentazioneSWE/issues/190}{PB13}\label{PB13}}{Inizio prototipazione frontend}{Alberto Pignat, Marco Piro, Linor Sadè} 

\end{tabularx}
\fussy


\section{Esito Riunione}
Sono stati discussi tutti i punti all'ordine del giorno. La riunione ha confermato le scelte tecnologiche e fornito orientamenti chiari sulla strategia di sviluppo, con enfasi sulla modellazione dati come punto di partenza.\\
Si ringraziano Nexum e i rappresentanti Gianpaolo Ferrarin e Luca per la disponibilità e il supporto dimostrato.
\newpage 



\vfill
\begin{center}
    {\small\color{darkgray} Documento redatto e approvato dal gruppo BugBusters.}
\end{center}

\end{document}