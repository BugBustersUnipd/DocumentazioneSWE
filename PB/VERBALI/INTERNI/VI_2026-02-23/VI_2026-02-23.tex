\documentclass[a4paper,12pt]{article}

\usepackage[utf8]{inputenc}
\usepackage[T1]{fontenc}
\usepackage[italian]{babel}
\usepackage[margin=2.5cm]{geometry}
\usepackage{graphicx}
\usepackage{grffile}
\usepackage{booktabs}
\usepackage{setspace}
\usepackage{titlesec}
\usepackage{float}
\usepackage{ifthen}
\usepackage{tcolorbox}
\usepackage{enumitem}
\usepackage[colorlinks=true,linkcolor=black,urlcolor=primaryblue,citecolor=primaryblue]{hyperref}
\usepackage{changepage} % per definire un rientro (un margine) solo in una parte del documento

\usepackage[table]{xcolor}   % per i colori (celle, testo e righe)
\usepackage{tabularx}               % per tabelle a larghezza adattiva (X)
\usepackage{array}                  % per comandi di formattazione avanzata nelle colonne (p, >{\raggedright})

% Macro
% genera la stringa "\noindent (Riferimento alla tabella decisioni:
% \hyperref[RTB4]{RTB4})", si usa facendo \refDecisione{NomeLabel}{Testo}
\newcommand{\refDecisione}[2]{%
    \noindent (\textbf{Riferimento alla tabella decisioni: \hyperref[#1]{#2}})%
}



\definecolor{primaryblue}{RGB}{0,102,204}
\definecolor{secondaryblue}{RGB}{51,153,255}
\definecolor{lightgray}{RGB}{245,245,245}
\definecolor{darkgray}{RGB}{100,100,100}

\titleformat{\section}
  {\Large\bfseries\color{primaryblue}}
  {\thesection}{1em}{}

\setlength{\parskip}{4pt}
\setlength{\parindent}{0pt}

\setlist[itemize]{leftmargin=*,itemsep=3pt}
\setlist[enumerate]{leftmargin=*,itemsep=3pt}

\graphicspath{{./}{../assets/images/}{./images/}}

\begin{document}

\begin{center}
  \IfFileExists{../../../../assets/Logo.jpg}{%
    \includegraphics[width=6cm,height=3cm,keepaspectratio]{../../../../assets/Logo.jpg} \\[0.8cm]
  }{%
    \fbox{\parbox[c][2.5cm][c]{6cm}{\centering Logo non trovato\\(Logo.jpg)}}\\[0.5cm]
  }
  
  {\Large\bfseries\color{primaryblue} BugBusters}\\[0.3cm]
  {\small\color{darkgray} Email: \texttt{bugbusters.unipd@gmail.com}} \\[0.1cm]
  {\small\color{darkgray} Gruppo: 4} \\[0.5cm]

  {\large\bfseries Università degli Studi di Padova}\\[0.3cm]
  {\small Laurea in Informatica}\\[0.2cm]
  {\small Corso: Ingegneria del Software}\\[0.2cm]
  {\small Anno Accademico: 2025/2026}\\[0.8cm]

  {\Huge\bfseries\color{primaryblue} Verbale Interno\textsubscript{\scalebox{0.6}{\textbf{G}}}}\\[0.3cm]
  {\Large\color{secondaryblue} 23 febbraio 2026}\\[0.8cm]
\end{center}

\begin{center}
\begin{tcolorbox}[colback=lightgray,colframe=primaryblue,width=0.85\textwidth,arc=3mm,boxrule=0.5pt]
\begin{tabular}{@{}ll@{}}
\textbf{Redattore\textsubscript{\scalebox{0.6}{\textbf{G}}}}    & Marco Favero \\
\textbf{Verificatore\textsubscript{\scalebox{0.6}{\textbf{G}}}}    &  Linor Sadè  \\
\textbf{Uso}          & Interno \\
\textbf{Destinatari}  & BugBusters \\
\textbf{Versione} & 1.0.0\\
\end{tabular}
\end{tcolorbox}
\end{center}

\vspace{0.5cm}

\begin{center}
\begin{tcolorbox}[colback=secondaryblue!10,colframe=secondaryblue,width=0.9\textwidth,arc=3mm,boxrule=0.8pt,title={\bfseries Abstract}]
Verbale interno relativo alla riunione del 23 febbraio 2026, durante la quale sono state decise le 
prossime attività da svolgere per correggere i documenti che hanno presentato criticità durante le revisioni RTB e per organizzare il lavoro futuro.
\end{tcolorbox}
\end{center}

\newpage

\tableofcontents
\newpage

\section{Informazioni generali}

\begin{itemize}
    \item \textbf{Tipo riunione:} Interna
    \item \textbf{Piattaforma:} Discord\textsubscript{\scalebox{0.6}{\textbf{G}}}
    \item \textbf{Data:} 23/02/2026
    \item \textbf{Orario di inizio:} 15:00
    \item \textbf{Orario di fine:} 16:00
    \item \textbf{Presenti:}
    \begin{itemize}[leftmargin=1.5em, itemsep=3pt, label={\rule[0.5ex]{0.4em}{0.4em}}]
        \item Alberto Autiero
        \item Marco Favero
        \item Alberto Pignat
        \item Marco Piro
        \item Linor Sadè
        \item Leonardo Salviato
        \item Luca Slongo
    \end{itemize}
\end{itemize}

\section{Ordine del giorno}
\begin{enumerate}
    \item\label{itm:revisione_doc} Lettura e analisi del documento ``Esito RTB'' per analizzare le problematiche emerse e decidere le 
    azioni da intraprendere per risolverle;
    \item\label{itm:decisioni_future} Discussione su come procedere con il progetto e pianificazione delle attività future.
\end{enumerate}


\newpage
\section{Svolgimento}
Qui di seguito i punti discussi con più dettaglio:

\subsection*{\ref{itm:revisione_doc}. Lettura e analisi del documento ``Esito RTB''} 

Durante la riunione è stata effettuata una revisione dei punti citati nel documento ``Esito RTB'', 
analizzando le problematiche emerse e definendo le azioni correttive da intraprendere.

In particolare, sono stati discussi i seguenti aspetti:
\begin{itemize}
  \item collegamento tracciabile tra decisioni dei verbali e azioni nel TODO di progetto;
  \item revisione delle Norme di Progetto per rafforzare la gerarchia processo{\textendash}attività{\textendash}procedure{\textendash}strumenti e renderne più operativa l'applicazione;
  \item aggiornamento dell'Analisi dei Requisiti con particolare attenzione ai requisiti non funzionali;
  \item revisione del Piano di Qualifica per migliorare l'interpretazione delle metriche e inserire indicatori quantitativi sulla campagna di test;
  \item uniformità redazionale nei documenti (formato date e uso coerente delle maiuscole);
  \item miglioramento del ciclo di pianificazione nel Piano di Progetto: consuntivo, retrospettiva, misure correttive, aggiornamento dei rischi e nuova pianificazione;
  \item separazione più chiara tra repository di lavoro e vista di consultazione, con maggiore attenzione alla fruibilità dei contenuti;
  \item adozione sistematica della coppia \textit{azione-verifica} per ogni attività, registrando gli avanzamenti solo dopo esito positivo della verifica;
  \item gestione del ciclo di miglioramento continuo (PDCA) tramite tracciamento nei verbali e misurazione degli effetti nel Piano di Qualifica.
\end{itemize}

\subsection*{\ref{itm:decisioni_future}. Pianificazione delle attività future}
Durante la riunione è stato deciso di approfondire la struttura e le modalità di redazione dei documenti ``Specifica Tecnica'' e ``Manuale Utente''.\\
È stato inoltre pianificato di avviare le correzioni dei documenti che hanno ricevuto feedback durante le revisioni RTB, al fine di risolvere le criticità emerse e elevare la qualità complessiva della documentazione.\\
Il gruppo si impegna a dedicare il massimo sforzo al progetto durante questo periodo accademico, caratterizzato dall'assenza di altre attività curricolari 
concorrenti, con l'obiettivo di rispettare la pianificazione concordata e soddisfare pienamente le aspettative del corso.\\
Infine, è stato avviato un contatto con il team di sviluppo di Eggon per coordinare una riunione congiunta e allinearsi sulle strategie di avanzamento del progetto.\\


\section{Tabella delle decisioni e azioni}

\setlength{\extrarowheight}{2pt} % padding extra verticale
\renewcommand{\arraystretch}{1.5} 


\arrayrulecolor{primaryblue}
\sloppy
\begin{tabularx}{\textwidth}{
    |>{\raggedright\arraybackslash}p{3cm}|
    >{\raggedright\arraybackslash}X|
    >{\raggedright\arraybackslash}p{3cm}|
}
\hline
\rowcolor{primaryblue!40}
\textbf{\color{white} ID Decisione} & \textbf{\color{white} Descrizione} & \textbf{\color{white} Incaricato} \\

\hline
\rowcolor{secondaryblue!10} \href{https://github.com/BugBustersUnipd/DocumentazioneSWE/issues/175}{PB01}  & Uniformare i nomi dei documenti con formato data AAAAMMGG & Luca Slongo \\
\hline
\rowcolor{secondaryblue!10} \href{https://github.com/BugBustersUnipd/DocumentazioneSWE/issues/176}{PB02} & Cambiare ordine informazioni nel PdP & Alberto Autiero \\
\hline
\rowcolor{secondaryblue!10} \href{https://github.com/BugBustersUnipd/DocumentazioneSWE/issues/177}{PB03}  & Revisione del Piano di Qualifica: obiettivi di qualità, metriche e riallocazione dei contenuti non pertinenti & Marco Piro, Alberto Pignat \\
\hline
\rowcolor{secondaryblue!10} \href{https://github.com/BugBustersUnipd/DocumentazioneSWE/issues/178}{PB04} & Cambiare link in lettera candidatura RTB & Marco Piro \\
\hline
\rowcolor{secondaryblue!10} \href{https://github.com/BugBustersUnipd/DocumentazioneSWE/issues/179}{PB05} & Aggiornamento e correzione AdR & Leonardo Salviato, Marco Favero \\
\hline
\rowcolor{secondaryblue!10} \href{https://github.com/BugBustersUnipd/DocumentazioneSWE/issues/180}{PB06} & Mockup applicazione MVP & Linor Sadè \\
\hline
\rowcolor{secondaryblue!10} \href{https://github.com/BugBustersUnipd/DocumentazioneSWE/issues/181}{PB07} & Stesura Verbale & Marco Favero \\
\hline
\rowcolor{secondaryblue!10} \href{https://github.com/BugBustersUnipd/DocumentazioneSWE/issues/182}{PB08} & Prima stesura / informarsi su Manuale Utente & Luca Slongo \\
\hline
\rowcolor{secondaryblue!10} \href{https://github.com/BugBustersUnipd/DocumentazioneSWE/issues/183}{PB09} & Prima stesura / informarsi su Specifica Tecnica & Alberto Pignat \\
\hline


\end{tabularx}
\fussy

\vfill
\begin{center}
    {\small\color{darkgray} Documento redatto e approvato dal gruppo BugBusters.}
\end{center}

\end{document}