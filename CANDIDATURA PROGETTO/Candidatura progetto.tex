\documentclass[a4paper,11pt]{article}
\usepackage[utf8]{inputenc}
\usepackage[italian]{babel}
\usepackage[margin=2.5cm]{geometry}
\usepackage{hyperref}
\usepackage{graphicx}
\usepackage{xcolor}
\usepackage{colortbl}
\usepackage{tcolorbox}
\usepackage{fancyhdr}
\usepackage{titlesec}
\usepackage{enumitem}
\usepackage{amssymb}

% Definizione colori
\definecolor{primarycolor}{RGB}{0,102,204}
\definecolor{secondarycolor}{RGB}{51,153,255}
\definecolor{lightgray}{RGB}{240,240,240}

% Configurazione hyperref
\hypersetup{
    colorlinks=true,
    linkcolor=primarycolor,
    urlcolor=secondarycolor,
    citecolor=primarycolor
}

% Configurazione header e footer
\pagestyle{fancy}
\fancyhf{}
\fancyhead[L]{\small\textcolor{primarycolor}{BugBusters - Gruppo 4}}
\fancyhead[R]{\small\textcolor{primarycolor}{Candidatura Capitolato C5}}
\fancyfoot[C]{\thepage}
\renewcommand{\headrulewidth}{0.5pt}
\renewcommand{\footrulewidth}{0pt}

% Configurazione sezioni
\titleformat{\section}
{\color{primarycolor}\Large\bfseries}
{\thesection}{1em}{}[\titlerule]

\titleformat{\subsection}
{\color{secondarycolor}\large\bfseries}
{\thesubsection}{1em}{}

% Path per le immagini
\graphicspath{{./}{../assets/}}

% Data fissa al 23 Ottobre 2025
\newcommand{\fixeddate}{23 Ottobre 2025}

\begin{document}

% Header personalizzato per prima pagina
\thispagestyle{empty}

% Box intestazione (sfondo più chiaro e trasparente)
\begin{tcolorbox}[
    colback=primarycolor!40,
    colframe=primarycolor!80,
    coltext=white,
    arc=3mm,
    boxrule=0pt,
    left=12pt,
    right=12pt,
    top=16pt,
    bottom=16pt
]
    \color{white}
    \begin{minipage}[c]{0.7\textwidth}
        {\Huge\bfseries Candidatura}\\[0.2cm]
        {\huge Capitolato C5}\\[0.3cm]
        {\Large Nexum - Eggon}
    \end{minipage}
    \hfill
    \begin{minipage}[c]{0.25\textwidth}
        \raggedleft
        \IfFileExists{../assets/Logo.jpg}{%
            \includegraphics[width=3cm,height=3cm,keepaspectratio]{../assets/Logo.jpg}
        }{%
            \colorbox{white}{\parbox[c][3cm][c]{3cm}{\centering\tiny\color{black} Logo\\non trovato}}
        }
    \end{minipage}
\end{tcolorbox}

\vspace{1em}

% Informazioni gruppo
\begin{tcolorbox}[
    colback=lightgray,
    colframe=lightgray,
    arc=3mm,
    boxrule=0pt,
    left=10pt,
    right=10pt,
    top=10pt,
    bottom=10pt
]
    \textbf{Gruppo:} BugBusters (Gruppo 4)\\
    \textbf{Email:} \href{mailto:bugbusters.unipd@gmail.com}{bugbusters.unipd@gmail.com}\\
    \textbf{Data:} \fixeddate % Data fissa invece di \today
\end{tcolorbox}

\vspace{2em}

% Destinatari
\noindent
\textbf{Destinatari:}\\
Prof. Vardanega Tullio\\
Prof. Cardin Riccardo\\
Università degli Studi di Padova\\
Dipartimento di Matematica\\
Via Trieste, 63\\
35121, Padova

\vspace{2em}

\section*{Oggetto: Candidatura Capitolato C5 - Nexum}

Con la presente, il gruppo \textbf{BugBusters} comunica formalmente la propria candidatura per la realizzazione del capitolato:

\begin{center}
    {\Large\textbf{Nexum}}\\[0.3cm]
    Proposto da: \textbf{Eggon}\\
    Codice capitolato: \textbf{C5}
\end{center}

\vspace{1.5em}

\section*{Informazioni sul Progetto}

Come specificato nel documento di \href{https://bugbustersunipd.github.io/BugBusterSite/assets/docs/DICHIARAZIONE_IMPEGNI/Dichiarazione_impegni.pdf}{dichiarazione degli impegni}, BugBusters ha preventivato:

\begin{itemize}[leftmargin=2cm, itemsep=0.5em]
    \item[\textcolor{primarycolor}{$\blacktriangleright$}] \textbf{Costo totale:} 12790€
    \item[\textcolor{primarycolor}{$\blacktriangleright$}] \textbf{Data di consegna:} 21/03/2026
\end{itemize}

\vspace{1.5em}

\section*{Motivazioni della Scelta}

Il gruppo ha identificato i seguenti punti di forza che hanno orientato la scelta verso questo capitolato:

\begin{itemize}
    \item \textbf{Supporto aziendale strutturato e continuo:} Eggon ha dimostrato un approccio chiaro, disponibile e supportivo, con un piano di supporto ben definito che include:
    \begin{itemize}
        \item Partecipazione attiva alle cerimonie Scrum aziendali (Planning, Review, Retrospective). Questo permette al team di sperimentare un contesto lavorativo reale e di imparare metodologie agili.
        \item Supporto tecnico continuo tramite code review, pairing e un canale di comunicazione dedicato (email/Telegram).
        \item Disponibilità di un seed repository e di un ambiente sandbox per agevolare testing e sviluppo.
        \item Atteggiamento orientato alla crescita del team, non solo al risultato finale.
    \end{itemize}

    \item \textbf{Integrazione in un prodotto reale ed esistente:} NEXUM è una piattaforma HR già operativa, con utenti reali. Sviluppare moduli che saranno effettivamente utilizzati fornisce esperienza diretta su un prodotto commerciale, affrontando sfide come l'integrazione con codice legacy e standard aziendali.

    \item \textbf{Esperienza con tecnologie moderne e di rilievo:} utilizzo di tecnologie attuali e rilevanti nel mercato:
    \begin{itemize}
        \item AWS (Bedrock per LLM, S3, Cognito, ecc.)
        \item Ruby on Rails, Angular, Next.js
        \item AI Generativa e OCR
    \end{itemize}
    L'esperienza con questi tool rappresenta un valore aggiunto per il curriculum. Eggon incoraggia l'apprendimento progressivo e fornisce risorse per affrontare le tecnologie proposte.

    \item \textbf{Allineamento con le aspirazioni del team:} il team ha apprezzato lo stile giovane, informale e professionale di Eggon. L'opportunità di lavorare in un contesto aziendale reale, con un prodotto già sul mercato, è stata percepita come un valore aggiunto significativo.

    \item \textbf{Chiarezza di requisiti e flessibilità progettuale:} i requisiti sono ben definiti, ma l'azienda è aperta a ridefinirli insieme al team. Sono stati forniti casi d'uso dettagliati, obiettivi misurabili e criteri di accettazione chiari. Eggon ha espresso flessibilità nella scelta dei modelli LLM e nell'approccio architetturale.

    \item \textbf{Impegno nonostante la complessità:} pur riconoscendo la complessità tecnica del capitolato, il team si impegna concretamente nello sviluppo del progetto, forte del fatto che tutti i membri non hanno esami arretrati e possono quindi dedicare tempo ed energie complete al progetto.
    
    \item \textbf{Possibilità di impatto, visibilità e crescita professionale:} il progetto non è solo un esercizio accademico, ma un contributo a un prodotto commerciale. Eggon ha espresso l'intenzione di valorizzare il lavoro del team e di considerarlo come un'opportunità di recruiting. È stata menzionata la possibilità di stage e collaborazioni successive per i membri del team, offrendo un'opportunità concreta di crescita professionale e di ingresso in un'azienda del territorio.
    
    \item \textbf{Progetto bilanciato tra innovazione e fattibilità:} il progetto è stato scomposto in due moduli principali (AI Assistant e AI Co-Pilot), con una complessità tecnica gestibile e un buon equilibrio tra innovazione e realizzabilità.
\end{itemize}


\vspace{1em}

\begin{tcolorbox}[
    colback=lightgray,
    colframe=secondarycolor,
    arc=2mm,
    boxrule=0pt,
    left=10pt,
    right=10pt,
    top=8pt,
    bottom=8pt
]
    \small\textit{Per ulteriori dettagli si rimanda al \href{https://bugbustersunipd.github.io/BugBusterSite/assets/docs/SCELTA_CAPITOLATO/Resoconto_capitolati.pdf}{documento dedicato alla valutazione dei capitolati}.}
\end{tcolorbox}


\section*{Risorse e Contatti}

Tutta la documentazione prodotta da BugBusters è pubblicata nel \href{https://bugbustersunipd.github.io/BugBusterSite/}{sito ufficiale} del gruppo e mantenuta nel repository dedicato.
L'intero gruppo è a disposizione per qualsiasi chiarimento. Di seguito i membri di BugBusters: \\ \\ 
\begin{center}
\setlength{\arrayrulewidth}{0.8pt}
\arrayrulecolor{black} % bordi della tabella neri
\begin{tabular}{|l|l|r|}
\hline
\rowcolor{primarycolor!20}
\textcolor{black}{\textbf{Cognome}} & \textcolor{black}{\textbf{Nome}} & \textcolor{black}{\textbf{Matricola}} \\
\hline
Autiero & Alberto & 2082862 \\
\hline
Favero & Marco & 2101071 \\
\hline
Pignat & Alberto & 2111044 \\
\hline
Piro & Marco & 2068075 \\
\hline
Sadè & Linor & 2111942 \\
\hline
Salviato & Leonardo & 2101086 \\
\hline
Slongo & Luca & 2111009 \\
\hline
\end{tabular}
\arrayrulecolor{black}
\end{center}

\end{document}