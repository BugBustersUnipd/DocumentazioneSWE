\documentclass[a4paper,11pt]{article}
\usepackage[utf8]{inputenc}
\usepackage[italian]{babel}
\usepackage[margin=2.5cm]{geometry}
\usepackage{hyperref}
\usepackage{graphicx}
\usepackage{xcolor}
\usepackage{colortbl}
\usepackage{tcolorbox}
\usepackage{fancyhdr}
\usepackage{titlesec}
\usepackage{enumitem}
\usepackage{amssymb}

% Definizione colori
\definecolor{primarycolor}{RGB}{0,102,204}
\definecolor{secondarycolor}{RGB}{51,153,255}
\definecolor{lightgray}{RGB}{240,240,240}

% Configurazione hyperref
\hypersetup{
    colorlinks=true,
    linkcolor=primarycolor,
    urlcolor=secondarycolor,
    citecolor=primarycolor
}

% Configurazione header e footer
\pagestyle{fancy}
\fancyhf{}
\fancyhead[L]{\small\textcolor{primarycolor}{Bug Busters - Gruppo 4}}
\fancyhead[R]{\small\textcolor{primarycolor}{Candidatura Capitolato C5}}
\fancyfoot[C]{\thepage}
\renewcommand{\headrulewidth}{0.5pt}
\renewcommand{\footrulewidth}{0pt}

% Configurazione sezioni
\titleformat{\section}
{\color{primarycolor}\Large\bfseries}
{\thesection}{1em}{}[\titlerule]

\titleformat{\subsection}
{\color{secondarycolor}\large\bfseries}
{\thesubsection}{1em}{}

% Path per le immagini
\graphicspath{{./}{../assets/}}

% Data fissa al 23 Ottobre 2025
\newcommand{\fixeddate}{23 Ottobre 2025}

\begin{document}

% Header personalizzato per prima pagina
\thispagestyle{empty}

% Box intestazione (sfondo più chiaro e trasparente)
\begin{tcolorbox}[
    colback=primarycolor!40,
    colframe=primarycolor!80,
    coltext=white,
    arc=3mm,
    boxrule=0pt,
    left=12pt,
    right=12pt,
    top=16pt,
    bottom=16pt
]
    \color{white}
    \begin{minipage}[c]{0.7\textwidth}
        {\Huge\bfseries Candidatura}\\[0.2cm]
        {\huge Capitolato C5}\\[0.3cm]
        {\Large Nexum - Eggon}
    \end{minipage}
    \hfill
    \begin{minipage}[c]{0.25\textwidth}
        \raggedleft
        \IfFileExists{../assets/Logo.jpg}{%
            \includegraphics[width=3cm,height=3cm,keepaspectratio]{../assets/Logo.jpg}
        }{%
            \colorbox{white}{\parbox[c][3cm][c]{3cm}{\centering\tiny\color{black} Logo\\non trovato}}
        }
    \end{minipage}
\end{tcolorbox}

\vspace{1em}

% Informazioni gruppo
\begin{tcolorbox}[
    colback=lightgray,
    colframe=lightgray,
    arc=3mm,
    boxrule=0pt,
    left=10pt,
    right=10pt,
    top=10pt,
    bottom=10pt
]
    \textbf{Gruppo:} Bug Busters (Gruppo 4)\\
    \textbf{Email:} \href{mailto:bugbusters.unipd@gmail.com}{bugbusters.unipd@gmail.com}\\
    \textbf{Data:} \fixeddate % Data fissa invece di \today
\end{tcolorbox}

\vspace{2em}

% Destinatari
\noindent
\textbf{Destinatari:}\\
Prof. Vardanega Tullio\\
Prof. Cardin Riccardo\\
Università degli Studi di Padova\\
Dipartimento di Matematica\\
Via Trieste, 63\\
35121, Padova

\vspace{2em}

\section*{Oggetto: Candidatura Capitolato C5 - Nexum}

Con la presente, il gruppo \textbf{Bug Busters} comunica formalmente la propria candidatura per la realizzazione del capitolato:

\begin{center}
    {\Large\textbf{Nexum}}\\[0.3cm]
    Proposto da: \textbf{Eggon}\\
    Codice capitolato: \textbf{C5}
\end{center}

\vspace{1.5em}

\section*{Informazioni sul Progetto}

Come specificato nel documento di dichiarazione degli impegni, Bug Busters ha preventivato:

\begin{itemize}[leftmargin=2cm, itemsep=0.5em]
    \item[\textcolor{primarycolor}{$\blacktriangleright$}] \textbf{Costo totale:} 12790€
    \item[\textcolor{primarycolor}{$\blacktriangleright$}] \textbf{Data di consegna:} 21/03/2026
\end{itemize}

\vspace{1.5em}

\section*{Motivazioni della Scelta}

Il gruppo ha identificato i seguenti punti di forza che hanno orientato la scelta verso questo capitolato:

\begin{enumerate}[leftmargin=2cm, itemsep=0.8em]
    \item \textbf{Interesse tematico:} Il capitolato si distingue per la sua unicità rispetto agli altri progetti proposti nell'ambito del corso di Ingegneria del Software, suscitando immediato interesse nel gruppo.
    
    \item \textbf{Qualità del confronto:} Le risposte fornite dall'azienda Eggon durante l'incontro sono state esaustive e hanno chiarito efficacemente tutti i dubbi emersi.
    
    \item \textbf{Supporto aziendale:} Eggon ha dimostrato una forte disponibilità a collaborare attivamente con il gruppo, offrendo supporto continuo durante lo sviluppo del progetto.
\end{enumerate}

\vspace{1em}

\begin{tcolorbox}[
    colback=lightgray,
    colframe=secondarycolor,
    arc=2mm,
    boxrule=0pt,
    left=10pt,
    right=10pt,
    top=8pt,
    bottom=8pt
]
    \small\textit{Per ulteriori dettagli si rimanda al \href{../SCELTA CAPITOLATO/Resoconto_capitolati.pdf}{documento dedicato alla valutazione dei capitolati}.}
\end{tcolorbox}


\section*{Risorse e Contatti}

Tutta la documentazione prodotta da Bug Busters è pubblicata nel \href{https://bugbustersunipd.github.io/BugBusterSite/}{sito ufficiale} del gruppo e mantenuta nel repository dedicato.
L'intero gruppo è a disposizione per qualsiasi chiarimento. Di seguito i membri di Bug Busters: \\ \\ 
\begin{center}
\setlength{\arrayrulewidth}{0.8pt}
\arrayrulecolor{black} % bordi della tabella neri
\begin{tabular}{|l|l|r|}
\hline
\rowcolor{primarycolor!20}
\textcolor{black}{\textbf{Cognome}} & \textcolor{black}{\textbf{Nome}} & \textcolor{black}{\textbf{Matricola}} \\
\hline
Autiero & Alberto & 2082862 \\
\hline
Favero & Marco & 2101071 \\
\hline
Pignat & Alberto & 2111044 \\
\hline
Piro & Marco & 2068075 \\
\hline
Sadè & Linor & 2111942 \\
\hline
Salviato & Leonardo & 2101086 \\
\hline
Slongo & Luca & 2111009 \\
\hline
\end{tabular}
\arrayrulecolor{black}
\end{center}

\end{document}