\documentclass[a4paper,11pt]{letter}
\usepackage[utf8]{inputenc}
\usepackage[italian]{babel}
\usepackage[margin=2.5cm]{geometry}
\usepackage{graphicx}
\usepackage{xcolor}
\usepackage{colortbl}
\usepackage{tcolorbox}
\usepackage{enumitem}
\usepackage{amssymb}
\usepackage{tabularx}
\usepackage{hyperref}

% Definizione colori
\definecolor{primarycolor}{RGB}{0,102,204}
\definecolor{secondarycolor}{RGB}{51,153,255}
\definecolor{lightgray}{RGB}{240,240,240}
\definecolor{darkgray}{RGB}{100,100,100}

% Configurazione hyperref (usa i colori definiti)
\hypersetup{
    colorlinks=true,
    linkcolor=primarycolor,
    urlcolor=secondarycolor,
    citecolor=primarycolor
}

% Path per le immagini
\graphicspath{{./}{../assets/}}

% Data fissa al 23 Ottobre 2025
\newcommand{\fixeddate}{23 Ottobre 2025}

% Firma
\newcommand{\GruppoFirma}{BugBusters\\Gruppo 4\\bugbusters.unipd@gmail.com}

\begin{document}

\begin{letter}{Prof. Tullio Vardanega\\Prof. Riccardo Cardin\\Universit\`a degli Studi di Padova\\Dipartimento di Matematica\\Via Trieste, 63\\35121 Padova}

% Intestazione grafica con logo
\begin{tcolorbox}[
    colback=primarycolor!40,
    colframe=primarycolor!80,
    coltext=white,
    arc=3mm,
    boxrule=0pt,
    left=12pt,
    right=12pt,
    top=12pt,
    bottom=12pt
]
    \begin{minipage}[c]{0.7\textwidth}
        {\color{white}\Huge\bfseries Candidatura}\\[0.2cm]
        {\color{white}\huge Capitolato C5}\\[0.2cm]
        {\color{white}\Large Nexum - Eggon}
    \end{minipage}\hfill
    \begin{minipage}[c]{0.25\textwidth}
        \raggedleft
        \IfFileExists{../assets/Logo.jpg}{%
            \includegraphics[width=3cm,height=3cm,keepaspectratio]{../assets/Logo.jpg}
        }{%
            \colorbox{white}{\parbox[c][3cm][c]{3cm}{\centering\tiny\color{black} Logo\\non trovato}}
        }
    \end{minipage}
\end{tcolorbox}

\vspace{1em}

% Box informativo sintetico
\begin{tcolorbox}[
    colback=lightgray,
    colframe=lightgray,
    arc=3mm,
    boxrule=0pt,
    left=10pt,
    right=10pt,
    top=10pt,
    bottom=10pt
]
    	\textbf{Gruppo:} BugBusters (Gruppo 4)\\
    	\textbf{Email:} \href{mailto:bugbusters.unipd@gmail.com}{bugbusters.unipd@gmail.com}\\
    	\textbf{Data:} \fixeddate
\end{tcolorbox}

\vspace{1.5em}

\opening{Egregi Professori,}

Con la presente, il gruppo \textbf{BugBusters} comunica formalmente la propria candidatura per la realizzazione
 del capitolato \textbf{C5 -- Nexum}, proposto dall'azienda \textbf{Eggon}. 
 Per lo sviluppo del progetto è stato stimato un costo totale di \textbf{€12.790}, come preventivato nel file di 
 \href{https://bugbustersunipd.github.io/BugBusterSite/assets/docs/DICHIARAZIONE_IMPEGNI/Dichiarazione_impegni.pdf}{dichiarazione impegni},
  e il completamento è previsto entro il \textbf{21 marzo 2026}. Successivamente verranno illustrate le motivazioni che hanno 
 orientato la scelta, insieme ai riferimenti e alle informazioni utili per una valutazione 
 più approfondita.

\vspace{0.8em}

	\textbf{Motivazioni della scelta}


Il gruppo ha identificato i seguenti punti di forza che hanno orientato la scelta verso questo capitolato:

\begin{itemize}
    \item \textbf{Supporto aziendale strutturato e continuo:} Eggon ha dimostrato un approccio chiaro, disponibile e supportivo, con un piano di supporto ben definito che include:
    \begin{itemize}
        \item Partecipazione attiva alle cerimonie Scrum aziendali (Planning, Review, Retrospective). Questo permette al team di sperimentare un contesto lavorativo reale e di imparare metodologie agili.
        \item Supporto tecnico continuo tramite code review, pairing e un canale di comunicazione dedicato (email/Telegram).
        \item Disponibilità di un seed repository e di un ambiente sandbox per agevolare testing e sviluppo.
        \item Atteggiamento orientato alla crescita del team, non solo al risultato finale.
    \end{itemize}

    \item \textbf{Integrazione in un prodotto reale ed esistente:} Nexum è una piattaforma HR già operativa, con utenti reali. Sviluppare moduli che saranno effettivamente utilizzati fornisce esperienza diretta su un prodotto commerciale, affrontando sfide come l'integrazione con codice legacy e standard aziendali.

    \item \textbf{Esperienza con tecnologie moderne e di rilievo:} utilizzo di tecnologie attuali e rilevanti nel mercato:
    \begin{itemize}
        \item AWS (Bedrock per LLM, S3, Cognito, ecc.)
        \item Ruby on Rails, Angular, Next.js
        \item AI Generativa e OCR
    \end{itemize}
    L'esperienza con questi tool rappresenta un valore aggiunto per il curriculum. Eggon incoraggia l'apprendimento progressivo e fornisce risorse per affrontare le tecnologie proposte.

    \item \textbf{Allineamento con le aspirazioni del team:} il team ha apprezzato lo stile giovane, informale e professionale di Eggon. L'opportunità di lavorare in un contesto aziendale reale, con un prodotto già sul mercato, è stata percepita come un valore aggiunto significativo.

    \item \textbf{Chiarezza di requisiti e flessibilità progettuale:} i requisiti sono ben definiti, ma l'azienda è aperta a ridefinirli insieme al team. Sono stati forniti casi d'uso dettagliati, obiettivi misurabili e criteri di accettazione chiari. Eggon ha espresso flessibilità nella scelta dei modelli LLM e nell'approccio architetturale.

    \item \textbf{Impegno nonostante la complessità:} pur riconoscendo la complessità tecnica del capitolato, il team si impegna concretamente nello sviluppo del progetto, forte del fatto che tutti i membri non hanno esami arretrati e possono quindi dedicare tempo ed energie complete al progetto.
    
    \item \textbf{Possibilità di impatto, visibilità e crescita professionale:} il progetto non è solo un esercizio accademico, ma un contributo a un prodotto commerciale. Eggon ha espresso l'intenzione di valorizzare il lavoro del team e di considerarlo come un'opportunità di recruiting. È stata menzionata la possibilità di stage e collaborazioni successive per i membri del team, offrendo un'opportunità concreta di crescita professionale e di ingresso in un'azienda del territorio.
    
    \item \textbf{Progetto bilanciato tra innovazione e fattibilità:} il progetto è stato scomposto in due moduli principali (AI Assistant e AI Co-Pilot), con una complessità tecnica gestibile e un buon equilibrio tra innovazione e realizzabilità.
\end{itemize}



\vspace{0.8em}

Per ulteriori dettagli sulle motivazioni della scelta e sui nostri criteri di valutazione, 
vi invitiamo a prendere visione del documento riguardante la \href{https://bugbustersunipd.github.io/BugBusterSite/assets/docs/SCELTA_CAPITOLATO/Resoconto_capitolati.pdf}{valutazione dei capitolati}.


\vspace{1em}

	\textbf{Membri del gruppo}
\begin{center}
\setlength{\arrayrulewidth}{0.8pt}
\arrayrulecolor{black}
\begin{tabular}{|l|l|r|}
\hline
\rowcolor{primarycolor!20}
	\textcolor{black}{\textbf{Cognome}} & \textcolor{black}{\textbf{Nome}} & \textcolor{black}{\textbf{Matricola}} \\
\hline
Autiero & Alberto & 2082861 \\
\hline
Favero & Marco & 2101071 \\
\hline
Pignat & Alberto & 2111044 \\
\hline
Piro & Marco & 2068075 \\
\hline
Sad\`e & Linor & 2111942 \\
\hline
Salviato & Leonardo & 2101086 \\
\hline
Slongo & Luca & 2111009 \\
\hline
\end{tabular}
\arrayrulecolor{black}
\end{center}


\vspace{1.2em}

\closing{Cordiali saluti, \vspace{1.2em} \\ \GruppoFirma}

\end{letter}

\end{document}